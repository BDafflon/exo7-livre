
%%%%%%%%%%%%%%%%%% PREAMBULE %%%%%%%%%%%%%%%%%%

\documentclass[aspectratio=169,utf8]{beamer}
%\documentclass[aspectratio=169,handout]{beamer}

\usetheme{Boadilla}
%\usecolortheme{seahorse}
%\usecolortheme[RGB={245,66,24}]{structure}
\useoutertheme{infolines}

% packages
\usepackage{amsfonts,amsmath,amssymb,amsthm}
\usepackage[utf8]{inputenc}
\usepackage[T1]{fontenc}
\usepackage{lmodern}

\usepackage[francais]{babel}
\usepackage{fancybox}
\usepackage{graphicx}

\usepackage{float}
\usepackage{xfrac}

%\usepackage[usenames, x11names]{xcolor}
\usepackage{pgfplots}
\usepackage{datetime}


% ----------------------------------------------------------------------
% Pour les images
\usepackage{tikz}
\usetikzlibrary{calc,shadows,arrows.meta,patterns,matrix}

\newcommand{\tikzinput}[1]{\input{figures/#1.tikz}}
% --- les figures avec échelle éventuel
\newcommand{\myfigure}[2]{% entrée : échelle, fichier(s) figure à inclure
\begin{center}\small%
\tikzstyle{every picture}=[scale=1.0*#1]% mise en échelle + 0% (automatiquement annulé à la fin du groupe)
#2%
\end{center}}



%-----  Package unités -----
\usepackage{siunitx}
\sisetup{locale = FR,detect-all,per-mode = symbol}

%\usepackage{mathptmx}
%\usepackage{fouriernc}
%\usepackage{newcent}
%\usepackage[mathcal,mathbf]{euler}

%\usepackage{palatino}
%\usepackage{newcent}
% \usepackage[mathcal,mathbf]{euler}



% \usepackage{hyperref}
% \hypersetup{colorlinks=true, linkcolor=blue, urlcolor=blue,
% pdftitle={Exo7 - Exercices de mathématiques}, pdfauthor={Exo7}}


%section
% \usepackage{sectsty}
% \allsectionsfont{\bf}
%\sectionfont{\color{Tomato3}\upshape\selectfont}
%\subsectionfont{\color{Tomato4}\upshape\selectfont}

%----- Ensembles : entiers, reels, complexes -----
\newcommand{\Nn}{\mathbb{N}} \newcommand{\N}{\mathbb{N}}
\newcommand{\Zz}{\mathbb{Z}} \newcommand{\Z}{\mathbb{Z}}
\newcommand{\Qq}{\mathbb{Q}} \newcommand{\Q}{\mathbb{Q}}
\newcommand{\Rr}{\mathbb{R}} \newcommand{\R}{\mathbb{R}}
\newcommand{\Cc}{\mathbb{C}} 
\newcommand{\Kk}{\mathbb{K}} \newcommand{\K}{\mathbb{K}}

%----- Modifications de symboles -----
\renewcommand{\epsilon}{\varepsilon}
\renewcommand{\Re}{\mathop{\text{Re}}\nolimits}
\renewcommand{\Im}{\mathop{\text{Im}}\nolimits}
%\newcommand{\llbracket}{\left[\kern-0.15em\left[}
%\newcommand{\rrbracket}{\right]\kern-0.15em\right]}

\renewcommand{\ge}{\geqslant}
\renewcommand{\geq}{\geqslant}
\renewcommand{\le}{\leqslant}
\renewcommand{\leq}{\leqslant}
\renewcommand{\epsilon}{\varepsilon}

%----- Fonctions usuelles -----
\newcommand{\ch}{\mathop{\text{ch}}\nolimits}
\newcommand{\sh}{\mathop{\text{sh}}\nolimits}
\renewcommand{\tanh}{\mathop{\text{th}}\nolimits}
\newcommand{\cotan}{\mathop{\text{cotan}}\nolimits}
\newcommand{\Arcsin}{\mathop{\text{arcsin}}\nolimits}
\newcommand{\Arccos}{\mathop{\text{arccos}}\nolimits}
\newcommand{\Arctan}{\mathop{\text{arctan}}\nolimits}
\newcommand{\Argsh}{\mathop{\text{argsh}}\nolimits}
\newcommand{\Argch}{\mathop{\text{argch}}\nolimits}
\newcommand{\Argth}{\mathop{\text{argth}}\nolimits}
\newcommand{\pgcd}{\mathop{\text{pgcd}}\nolimits} 


%----- Commandes divers ------
\newcommand{\ii}{\mathrm{i}}
\newcommand{\dd}{\text{d}}
\newcommand{\id}{\mathop{\text{id}}\nolimits}
\newcommand{\Ker}{\mathop{\text{Ker}}\nolimits}
\newcommand{\Card}{\mathop{\text{Card}}\nolimits}
\newcommand{\Vect}{\mathop{\text{Vect}}\nolimits}
\newcommand{\Mat}{\mathop{\text{Mat}}\nolimits}
\newcommand{\rg}{\mathop{\text{rg}}\nolimits}
\newcommand{\tr}{\mathop{\text{tr}}\nolimits}


%----- Structure des exercices ------

\newtheoremstyle{styleexo}% name
{2ex}% Space above
{3ex}% Space below
{}% Body font
{}% Indent amount 1
{\bfseries} % Theorem head font
{}% Punctuation after theorem head
{\newline}% Space after theorem head 2
{}% Theorem head spec (can be left empty, meaning ‘normal’)

%\theoremstyle{styleexo}
\newtheorem{exo}{Exercice}
\newtheorem{ind}{Indications}
\newtheorem{cor}{Correction}


\newcommand{\exercice}[1]{} \newcommand{\finexercice}{}
%\newcommand{\exercice}[1]{{\tiny\texttt{#1}}\vspace{-2ex}} % pour afficher le numero absolu, l'auteur...
\newcommand{\enonce}{\begin{exo}} \newcommand{\finenonce}{\end{exo}}
\newcommand{\indication}{\begin{ind}} \newcommand{\finindication}{\end{ind}}
\newcommand{\correction}{\begin{cor}} \newcommand{\fincorrection}{\end{cor}}

\newcommand{\noindication}{\stepcounter{ind}}
\newcommand{\nocorrection}{\stepcounter{cor}}

\newcommand{\fiche}[1]{} \newcommand{\finfiche}{}
\newcommand{\titre}[1]{\centerline{\large \bf #1}}
\newcommand{\addcommand}[1]{}
\newcommand{\video}[1]{}

% Marge
\newcommand{\mymargin}[1]{\marginpar{{\small #1}}}

\def\noqed{\renewcommand{\qedsymbol}{}}


%----- Presentation ------
\setlength{\parindent}{0cm}

%\newcommand{\ExoSept}{\href{http://exo7.emath.fr}{\textbf{\textsf{Exo7}}}}

\definecolor{myred}{rgb}{0.93,0.26,0}
\definecolor{myorange}{rgb}{0.97,0.58,0}
\definecolor{myyellow}{rgb}{1,0.86,0}

\newcommand{\LogoExoSept}[1]{  % input : echelle
{\usefont{U}{cmss}{bx}{n}
\begin{tikzpicture}[scale=0.1*#1,transform shape]
  \fill[color=myorange] (0,0)--(4,0)--(4,-4)--(0,-4)--cycle;
  \fill[color=myred] (0,0)--(0,3)--(-3,3)--(-3,0)--cycle;
  \fill[color=myyellow] (4,0)--(7,4)--(3,7)--(0,3)--cycle;
  \node[scale=5] at (3.5,3.5) {Exo7};
\end{tikzpicture}}
}


\newcommand{\debutmontitre}{
  \author{} \date{} 
  \thispagestyle{empty}
  \hspace*{-10ex}
  \begin{minipage}{\textwidth}
    \titlepage  
  \vspace*{-2.5cm}
  \begin{center}
    \LogoExoSept{2.5}
  \end{center}
  \end{minipage}

  \vspace*{-0cm}
  
  % Astuce pour que le background ne soit pas discrétisé lors de la conversion pdf -> png
\begin{tikzpicture}
        \fill[opacity=0,green!60!black] (0,0)--++(0,0)--++(0,0)--++(0,0)--cycle; 
\end{tikzpicture}

% toc S'affiche trop tot :
% \tableofcontents[hideallsubsections, pausesections]
}

\newcommand{\finmontitre}{
  \end{frame}
  \setcounter{framenumber}{0}
} % ne marche pas pour une raison obscure

%----- Commandes supplementaires ------

% \usepackage[landscape]{geometry}
% \geometry{top=1cm, bottom=3cm, left=2cm, right=10cm, marginparsep=1cm
% }
% \usepackage[a4paper]{geometry}
% \geometry{top=2cm, bottom=2cm, left=2cm, right=2cm, marginparsep=1cm
% }

%\usepackage{standalone}


% New command Arnaud -- november 2011
\setbeamersize{text margin left=24ex}
% si vous modifier cette valeur il faut aussi
% modifier le decalage du titre pour compenser
% (ex : ici =+10ex, titre =-5ex

\theoremstyle{definition}
%\newtheorem{proposition}{Proposition}
%\newtheorem{exemple}{Exemple}
%\newtheorem{theoreme}{Théorème}
%\newtheorem{lemme}{Lemme}
%\newtheorem{corollaire}{Corollaire}
%\newtheorem*{remarque*}{Remarque}
%\newtheorem*{miniexercice}{Mini-exercices}
%\newtheorem{definition}{Définition}

% Commande tikz
\usetikzlibrary{calc}
\usetikzlibrary{patterns,arrows}
\usetikzlibrary{matrix}
\usetikzlibrary{fadings} 

%definition d'un terme
\newcommand{\defi}[1]{{\color{myorange}\textbf{\emph{#1}}}}
\newcommand{\evidence}[1]{{\color{blue}\textbf{\emph{#1}}}}
\newcommand{\assertion}[1]{\emph{\og#1\fg}}  % pour chapitre logique
%\renewcommand{\contentsname}{Sommaire}
\renewcommand{\contentsname}{}
\setcounter{tocdepth}{2}



%------ Encadrement ------

\usepackage{fancybox}


\newcommand{\mybox}[1]{
\setlength{\fboxsep}{7pt}
\begin{center}
\shadowbox{#1}
\end{center}}

\newcommand{\myboxinline}[1]{
\setlength{\fboxsep}{5pt}
\raisebox{-10pt}{
\shadowbox{#1}
}
}

%--------------- Commande beamer---------------
\newcommand{\beameronly}[1]{#1} % permet de mettre des pause dans beamer pas dans poly


\setbeamertemplate{navigation symbols}{}
\setbeamertemplate{footline}  % tiré du fichier beamerouterinfolines.sty
{
  \leavevmode%
  \hbox{%
  \begin{beamercolorbox}[wd=.333333\paperwidth,ht=2.25ex,dp=1ex,center]{author in head/foot}%
    % \usebeamerfont{author in head/foot}\insertshortauthor%~~(\insertshortinstitute)
    \usebeamerfont{section in head/foot}{\bf\insertshorttitle}
  \end{beamercolorbox}%
  \begin{beamercolorbox}[wd=.333333\paperwidth,ht=2.25ex,dp=1ex,center]{title in head/foot}%
    \usebeamerfont{section in head/foot}{\bf\insertsectionhead}
  \end{beamercolorbox}%
  \begin{beamercolorbox}[wd=.333333\paperwidth,ht=2.25ex,dp=1ex,right]{date in head/foot}%
    % \usebeamerfont{date in head/foot}\insertshortdate{}\hspace*{2em}
    \insertframenumber{} / \inserttotalframenumber\hspace*{2ex} 
  \end{beamercolorbox}}%
  \vskip0pt%
}


\definecolor{mygrey}{rgb}{0.5,0.5,0.5}
\setlength{\parindent}{0cm}
%\DeclareTextFontCommand{\helvetica}{\fontfamily{phv}\selectfont}

% background beamer
\definecolor{couleurhaut}{rgb}{0.85,0.9,1}  % creme
\definecolor{couleurmilieu}{rgb}{1,1,1}  % vert pale
\definecolor{couleurbas}{rgb}{0.85,0.9,1}  % blanc
\setbeamertemplate{background canvas}[vertical shading]%
[top=couleurhaut,middle=couleurmilieu,midpoint=0.4,bottom=couleurbas] 
%[top=fondtitre!05,bottom=fondtitre!60]



\makeatletter
\setbeamertemplate{theorem begin}
{%
  \begin{\inserttheoremblockenv}
  {%
    \inserttheoremheadfont
    \inserttheoremname
    \inserttheoremnumber
    \ifx\inserttheoremaddition\@empty\else\ (\inserttheoremaddition)\fi%
    \inserttheorempunctuation
  }%
}
\setbeamertemplate{theorem end}{\end{\inserttheoremblockenv}}

\newenvironment{theoreme}[1][]{%
   \setbeamercolor{block title}{fg=structure,bg=structure!40}
   \setbeamercolor{block body}{fg=black,bg=structure!10}
   \begin{block}{{\bf Th\'eor\`eme }#1}
}{%
   \end{block}%
}


\newenvironment{proposition}[1][]{%
   \setbeamercolor{block title}{fg=structure,bg=structure!40}
   \setbeamercolor{block body}{fg=black,bg=structure!10}
   \begin{block}{{\bf Proposition }#1}
}{%
   \end{block}%
}

\newenvironment{corollaire}[1][]{%
   \setbeamercolor{block title}{fg=structure,bg=structure!40}
   \setbeamercolor{block body}{fg=black,bg=structure!10}
   \begin{block}{{\bf Corollaire }#1}
}{%
   \end{block}%
}

\newenvironment{mydefinition}[1][]{%
   \setbeamercolor{block title}{fg=structure,bg=structure!40}
   \setbeamercolor{block body}{fg=black,bg=structure!10}
   \begin{block}{{\bf Définition} #1}
}{%
   \end{block}%
}

\newenvironment{lemme}[0]{%
   \setbeamercolor{block title}{fg=structure,bg=structure!40}
   \setbeamercolor{block body}{fg=black,bg=structure!10}
   \begin{block}{\bf Lemme}
}{%
   \end{block}%
}

\newenvironment{remarque}[1][]{%
   \setbeamercolor{block title}{fg=black,bg=structure!20}
   \setbeamercolor{block body}{fg=black,bg=structure!5}
   \begin{block}{Remarque #1}
}{%
   \end{block}%
}


\newenvironment{exemple}[1][]{%
   \setbeamercolor{block title}{fg=black,bg=structure!20}
   \setbeamercolor{block body}{fg=black,bg=structure!5}
   \begin{block}{{\bf Exemple }#1}
}{%
   \end{block}%
}


\newenvironment{miniexercice}[0]{%
   \setbeamercolor{block title}{fg=structure,bg=structure!20}
   \setbeamercolor{block body}{fg=black,bg=structure!5}
   \begin{block}{Mini-exercices}
}{%
   \end{block}%
}


\newenvironment{tp}[0]{%
   \setbeamercolor{block title}{fg=structure,bg=structure!40}
   \setbeamercolor{block body}{fg=black,bg=structure!10}
   \begin{block}{\bf Travaux pratiques}
}{%
   \end{block}%
}
\newenvironment{exercicecours}[1][]{%
   \setbeamercolor{block title}{fg=structure,bg=structure!40}
   \setbeamercolor{block body}{fg=black,bg=structure!10}
   \begin{block}{{\bf Exercice }#1}
}{%
   \end{block}%
}
\newenvironment{algo}[1][]{%
   \setbeamercolor{block title}{fg=structure,bg=structure!40}
   \setbeamercolor{block body}{fg=black,bg=structure!10}
   \begin{block}{{\bf Algorithme}\hfill{\color{gray}\texttt{#1}}}
}{%
   \end{block}%
}


\setbeamertemplate{proof begin}{
   \setbeamercolor{block title}{fg=black,bg=structure!20}
   \setbeamercolor{block body}{fg=black,bg=structure!5}
   \begin{block}{{\footnotesize Démonstration}}
   \footnotesize
   \smallskip}
\setbeamertemplate{proof end}{%
   \end{block}}
\setbeamertemplate{qed symbol}{\openbox}


\makeatother
% Couleur à définir

   
%%%%%%%%%%%%%%%%%%%%%%%%%%%%%%%%%%%%%%%%%%%%%%%%%%%%%%%%%%%%%
%%%%%%%%%%%%%%%%%%%%%%%%%%%%%%%%%%%%%%%%%%%%%%%%%%%%%%%%%%%%%


\begin{document}


\title{{\bf Séries}}
\subtitle{Séries à termes positifs}

\begin{frame}
  
  \debutmontitre

  \pause

{\footnotesize
\hfill
\setbeamercovered{transparent=50}
\begin{minipage}{0.6\textwidth}
  \begin{itemize}
    \item<3-> Convergence par les sommes partielles
    \item<4-> Théorème de comparaison
    \item<5-> Théorème des équivalents
  \end{itemize}
\end{minipage}
}

\end{frame}

\setcounter{framenumber}{0}



%%%%%%%%%%%%%%%%%%%%%%%%%%%%%%%%%%%%%%%%%%%%%%%%%%%%%%%%%%%%%%%%
\section{Convergence par les sommes partielles}

\begin{frame}

\textbf{Rappels.}
Soit $(s_n)_{n\ge0}$ une suite croissante de nombres réels
\begin{itemize}
  \item\pause Si la suite $(s_n)$ est majorée, alors elle converge
  \item\pause Sinon la suite $(s_n)$ tend vers $+\infty$
\end{itemize}

\vspace{.3cm}
\pause

Soit une série $\sum u_k$ à \defi{termes positifs}, c'est-à-dire $u_k\ge 0$ pour tout $k$

\pause
Alors $S_n = \displaystyle\sum_{k=0}^n u_k$ est une suite croissante, puisque :
\vspace{-.3cm}
$$S_{n}-S_{n-1} = u_n \ge 0$$

\pause
\begin{proposition}
Une série à termes positifs est convergente si et seulement si 
la suite des sommes partielles est majorée
\end{proposition}

\begin{itemize}
  \item\pause $\sum u_k$ converge si et seulement si $\exists M>0 \quad \forall n\ge 0 \quad S_n \le M$
  \item\pause si $\sum u_k$ converge alors sa somme $S=\lim S_n$ vérifie $S_n \le S$ pour tout $n$
  \item\pause $\sum_{k\ge0} q^k$ converge si $0<q<1$ et diverge si $q \ge 1$
\end{itemize}

\end{frame}


%%%%%%%%%%%%%%%%%%%%%%%%%%%%%%%%%%%%%%%%%%%%%%%%%%%%%%%%%%%%%%%%
\section{Théorème de comparaison}

\begin{frame}

\begin{theoreme}[de comparaison]
\pause
Soient $\sum u_k$ et $\sum v_k$ deux séries à termes positifs ou nuls

\pause
On suppose qu'il existe $k_0\ge 0$ tel que $u_k \le v_k$ pour tout $k\ge k_0$
\begin{itemize}
\item\pause Si $\sum v_k$ converge alors $\sum u_k$ converge
\item\pause Si $\sum u_k$ diverge alors $\sum v_k$ diverge
\end{itemize}
\end{theoreme}

\pause
\begin{proof}
$k_0=0$
\begin{itemize}
\item\pause Notons \ $S_n=u_0+\cdots+u_n$ \ et \ $S'_n = v_0+\cdots+v_n$

\begin{itemize}
\item\pause Les suites $(S_n)$ et $(S'_n)$ sont croissantes
\item\pause Pour tout $n \ge 0$, $S_n\le S'_n$
\end{itemize}
\item\pause Si la série $\sum v_k$ converge, alors la suite
$(S'_n)$ converge. Soit $S'$ sa limite. 

\pause
La suite $(S_n)$ est croissante et majorée par $S'$, \pause donc elle converge, et ainsi la série $\sum u_k$ converge aussi
\item\pause Si la série $\sum u_k$ diverge, alors $(S_n)$ tend vers $+\infty$, \pause et de même pour $(S'_n)$ : la série $\sum v_k$ diverge\qedhere
\end{itemize}
\end{proof}
\end{frame}


%%%%%%%%%%%%%%%%%%%%%%%%%%%%%%%%%%%%%%%%%%%%%%%%%%%%%%%%%%%%%%%%
\section{Exemples}

\begin{frame}

\begin{exemple}
\pause
La série
$
\displaystyle\sum_{k=0}^{+\infty} \frac{1}{(k+1)(k+2)} $ converge. Nous allons en déduire que
\vspace{-.2cm}
\mybox{$\displaystyle
\sum_{k=1}^{+\infty} \frac{1}{k^2}
\quad\text{ converge}
$}
\vspace{-.2cm}
\pause
\begin{itemize}
\item on a 
$
\displaystyle\lim_{k\to+\infty}
\frac{\frac{1}{2k^2}}{\frac{1}{(k+1)(k+2)}}=\frac{1}{2}
$
\item\pause En particulier, il existe $k_0$ tel que pour $k \ge k_0$
\vspace{-1ex}
$$
\frac{1}{2k^2} \le \frac{1}{(k+1)(k+2)}
$$
\item\pause On en déduit que la série de terme général $\frac{1}{2k^2}$ converge

\pause
D'où le résultat par linéarité
\end{itemize}
\end{exemple}

\end{frame}

%%%%%%%%%%%%%%
\begin{frame}

\begin{exemple}
\mybox{$\displaystyle\text{La  \defi{série exponentielle}} \quad \sum_{k\ge 0} \frac{1}{k!} \quad \text{converge}$}
\vspace{-.1cm}
où $0!=1$ et $k!=1\cdot 2\cdot 3\dots \cdot k$ pour $k\ge 1$

\pause
\vspace{.2cm}
En effet :
\begin{itemize}
\item $\frac{1}{k!}\le \frac{1}{k(k-1)}$ pour $k \ge 2$
\item\pause Mais $\displaystyle\sum_{k \ge 2}\frac{1}{k(k-1)} = \displaystyle\sum_{k\ge0} \frac{1}{(k+1)(k+2)}$
est une série convergente
\item\pause Donc la série exponentielle $\displaystyle\sum_{k\ge 0} \frac{1}{k!}$ converge
\end{itemize}
\vspace{-.1cm}
\pause
Par définition : la somme $\displaystyle\sum_{k=0}^{+\infty} \frac{1}{k!} =e = \exp(1)$ est le nombre d'Euler
\end{exemple}

\end{frame}

%%%%%%%%%%%%%%
\begin{frame}
\begin{exemple}
Inversement nous avons vu que la série $\displaystyle\sum_{k\ge1} \frac{1}{k}$ diverge 

\pause
On en déduit que les séries 
$\displaystyle\sum_{k\ge1} \frac{\ln(k)}{k}$ et $\displaystyle\sum_{k\ge1} \frac{1}{\sqrt{k}}$ divergent également
\end{exemple}

\end{frame}

%%%%%%%%%%%%%%
\begin{frame}

\begin{exemple}
Soit $(a_k)_{k\ge 1}$ une suite, $a_k\in\{0,1,\ldots,9\}$. La série $\displaystyle \sum_{k\geq1} \frac{a_k}{10^k}$ converge

\pause
\vspace{-.1cm}
En effet :
\begin{itemize}
\item son terme général $u_k=\frac{a_k}{10^k}$ est majoré par $\frac{9}{10^k}$
\item\pause mais la série géométrique $\sum \frac{1}{10^k}$ converge, car $\frac{1}{10}<1$
\item\pause la série $\sum \frac{9}{10^k}$ converge aussi par linéarité, d'où le résultat
\end{itemize}
\vspace{.1cm}

\pause
Une somme $\displaystyle\sum_{k=1}^{+\infty} \frac{a_k}{10^k}$ est une écriture décimale d'un réel
$x$, $0 \le x \le 1$

\begin{itemize}
\item\pause par exemple, si $a_k = 3$ pour tout $k$ : 

\hspace{-.9cm}
$\displaystyle\sum_{k=1}^{+\infty} \tfrac{3}{10^k} 
\pause= \tfrac{3}{10}+\tfrac{3}{100}+\tfrac{3}{1000}\cdots
\pause= 0,3+0,03+0,003\cdots \pause= 0,333\ldots \pause= \tfrac13$
\item\pause ou
$\displaystyle\sum_{k=1}^{+\infty} \tfrac{3}{10^k} 
\pause= \tfrac{3}{10} \displaystyle\sum_{k=0}^{+\infty} \tfrac{1}{10^k}
\pause=  \tfrac{3}{10} \cdot \tfrac{1}{1-\tfrac{1}{10}}
\pause= \tfrac13$
\end{itemize}
\end{exemple}

\end{frame}

%%%%%%%%%%%%%%%%%%%%%%%%%%%%%%%%%%%%%%%%%%%%%%%%%%%%%%%%%%%%%%%%
\section{Théorème des équivalents}

\begin{frame}
Soient $(u_k)$ et $(v_k)$ deux suites \evidence{strictement positives} 

\pause
Alors les suites $(u_k)$ et $(v_k)$ sont \defi{équivalentes} si 
$$\lim_{k\to+\infty} \frac{u_k}{v_k}=1$$
\pause
On note alors $$u_k \sim v_k$$

\pause
\begin{theoreme}[des équivalents]
Soient $(u_k)$ et $(v_k)$ deux suites strictement positives

\pause
Si $u_k \sim v_k$ alors les séries $\sum u_k$ et $\sum v_k$ sont de même nature
\end{theoreme}

\begin{itemize}
\item\pause $\sum u_k$ et $\sum v_k$ sont toutes les deux convergentes ou toutes les deux divergentes
\item\pause Si $(u_k)$ et $(v_k)$ sont strictement négatives, la conclusion reste vraie
\end{itemize}

\end{frame}

%%%%%%%%%%%%%%
\begin{frame}
Par exemple
\begin{itemize}
\item\pause les suites $\frac{1}{k^2}$ et $\frac{1}{(k+1)(k+2)}=\frac{1}{k^2+3k+2}$
sont équivalentes
\item\pause $\sum \frac{1}{(k+1)(k+2)}$ converge, \pause donc $\sum \frac{1}{k^2}$ converge
\end{itemize}

\pause
\begin{proof}
\begin{itemize}
\item\pause Par hypothèse, pour tout $\epsilon>0$, il existe $k_0$ tel que pour
tout $k \ge k_0$
$$\left|\frac{u_k}{v_k} -1\right| < \epsilon$$
\pause
autrement dit 
$$(1-\epsilon)v_k < u_k <(1+\epsilon) v_k$$

\item\pause Fixons $\epsilon <1$
\begin{itemize}
\item\pause Si $\sum u_k$ converge, \pause alors $\sum(1-\epsilon) v_k$ converge, \pause donc $\sum v_k$ aussi
\smallskip
\item\pause Si $\sum u_k$ diverge, alors $\sum (1+\epsilon)v_k$ diverge, et $\sum v_k$ aussi 
\qedhere
\end{itemize}
\end{itemize}
\end{proof}
\end{frame}



%%%%%%%%%%%%%%%%%%%%%%%%%%%%%%%%%%%%%%%%%%%%%%%%%%%%%%%%%%%%%%%%
\section{Exemples}

\begin{frame}
\begin{exemple}
Les deux séries 
$$
\sum \frac{k^2+3k+1}{k^4+2k^3+4}
\qquad \text{ et } \qquad 
\sum \frac{k +\ln(k)}{k^3}
\quad \text{ convergent}
$$
\vspace{-.3cm}
\begin{itemize}
\item\pause Dans les deux cas le terme général est équivalent à $\frac{1}{k^2}$
\item\pause Or la série $\sum \frac{1}{k^2}$ converge
\end{itemize}
\end{exemple}

\pause
\begin{exemple}
Par contre
$$
\sum \frac{k^2+3k+1}{k^3+2k^2+4}
\qquad \text{ et } \qquad 
\sum \frac{k +\ln(k)}{k^2}
\;\mbox{ divergent}
$$
\vspace{-.3cm}
\begin{itemize}
\item\pause Dans les deux cas le terme général est équivalent à $\frac{1}{k}$
\item\pause Or la série $\sum \frac{1}{k}$ diverge
\end{itemize}\end{exemple}

\end{frame}

%%%%%%%%%%%%%%
\begin{frame}

\begin{exemple}
Est-ce que la série $\displaystyle\sum_{k \ge 1}  \ln \big(\tanh k\big)$ converge ?

\begin{itemize}
  \item\pause Remarquons que, pour $k >0$, $0<\tanh k<1$ 
  
  \item\pause $\tanh k = \frac{\sh k}{\ch k} = \frac{e^k-e^{-k}}{e^k+e^{-k}}
  \pause= 1+  \frac{-2e^{-k}}{e^k+e^{-k}} = 1+\frac{-2e^{-2k}}{1+e^{-2k}}$

  
  \item\pause Comme $\lim_{x\to 0} \frac{\ln(1+x)}{x}=1$ alors, si
  $u_k \to 0$, $\ln(1+u_k) \sim u_k$
  
  \pause Ainsi 
  $$\ln(\tanh k) = \ln\left( 1+\frac{-2e^{-2k}}{1+e^{-2k}} \right)
  \pause\sim  \frac{-2e^{-2k}}{1+e^{-2k}} \pause \sim -2e^{-2k}$$
      
  \item\pause Les suites $\ln(\tanh k)$ et $-2e^{-2k}$ sont strictement négatives  
  \item\pause La série $\sum e^{-2k}= \sum (e^{-2})^k$ converge
  \item\pause Alors la série $\sum\ln(\tanh k)$ converge également 
\end{itemize}

\end{exemple}
\end{frame}

%%%%%%%%%%%%%%%%%%%%%%%%%%%%%%%%%%%%%%%%%%%%%%%%%%%%%%%%%%%%%%%%
\section{Mini-exercices}

\begin{frame}
\begin{miniexercice}
\begin{enumerate}

  \item Montrer que la série 
  $\sum_{k\geq1} \frac{(\ln k)^\alpha}{k^3}$ converge pour tout $\alpha \in \Rr$.
  
  \item Montrer que si $\sum u_k$ converge et $u_k>0$, alors $\sum u_k^2$ converge.

  \item Soient  $\sum u_k$ et $\sum v_k$ vérifiant $u_k>0$, $v_k>0$ et $0<m \le \frac{u_k}{v_k} \le M$.
  Montrer que les deux séries sont de même nature.

  \item Par comparaison ou recherche d'équivalent, déterminer la nature de la série
  $\sum_{k \ge 1} \frac{\ln k}{k}$.
  Même question avec les séries de terme général 
  $\sin \left(\frac{1}{(k-1)(k+1)}\right)$ ;
  $\sqrt{1+\frac{1}{k^2}}-\sqrt{1-\frac{1}{k^2}}$ ; 
  $\ln\left(\sqrt{1-\frac{1}{\sqrt[3]{k^2}}}\right)$.
  
  
  \item \'Ecrire la série associée au développement décimal $0,999\ldots$ Calculer sa somme $S$ grâce à la série associée à $10 \cdot S$, en simplifiant $10 \cdot S - S$. Retrouver $S$ par une série géométrique.

  \item Justifier que la série $\sum_{k=2}^{+\infty} \frac{1}{k^2-1}$ est convergente.
  Décomposer $\frac{1}{k^2-1}$ en éléments simples. 
  Déterminer une expression des sommes partielles $S_n$.
  En déduire que $\sum_{k=2}^{+\infty} \frac{1}{k^2-1} = \frac34$.

%  \item Nous admettons ici que $\sum_{k=0}^{+\infty}\frac{1}{k!} = e$.
%  Sans calculs, déterminer les sommes :
%  $$\sum_{k=0}^{+\infty}\frac{k}{k!} \qquad 
%  \sum_{k=0}^{+\infty}\frac{k(k-1)}{k!} \qquad
%  \sum_{k=0}^{+\infty}\frac{k^2}{k!}$$
\end{enumerate}
\end{miniexercice}
\end{frame}

\end{document}
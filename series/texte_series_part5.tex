
%%%%%%%%%%%%%%%%%% PREAMBULE %%%%%%%%%%%%%%%%%%


\documentclass[12pt]{article}

\usepackage{amsfonts,amsmath,amssymb,amsthm}
\usepackage[utf8]{inputenc}
\usepackage[T1]{fontenc}
\usepackage[francais]{babel}


% packages
\usepackage{amsfonts,amsmath,amssymb,amsthm}
\usepackage[utf8]{inputenc}
\usepackage[T1]{fontenc}
%\usepackage{lmodern}

\usepackage[francais]{babel}
\usepackage{fancybox}
\usepackage{graphicx}

\usepackage{float}

%\usepackage[usenames, x11names]{xcolor}
\usepackage{tikz}
\usepackage{datetime}

\usepackage{mathptmx}
%\usepackage{fouriernc}
%\usepackage{newcent}
\usepackage[mathcal,mathbf]{euler}

%\usepackage{palatino}
%\usepackage{newcent}


% Commande spéciale prompteur

%\usepackage{mathptmx}
%\usepackage[mathcal,mathbf]{euler}
%\usepackage{mathpple,multido}

\usepackage[a4paper]{geometry}
\geometry{top=2cm, bottom=2cm, left=1cm, right=1cm, marginparsep=1cm}

\newcommand{\change}{{\color{red}\rule{\textwidth}{1mm}\\}}

\newcounter{mydiapo}

\newcommand{\diapo}{\newpage
\hfill {\normalsize  Diapo \themydiapo \quad \texttt{[\jobname]}} \\
\stepcounter{mydiapo}}


%%%%%%% COULEURS %%%%%%%%%%

% Pour blanc sur noir :
%\pagecolor[rgb]{0.5,0.5,0.5}
% \pagecolor[rgb]{0,0,0}
% \color[rgb]{1,1,1}



%\DeclareFixedFont{\myfont}{U}{cmss}{bx}{n}{18pt}
\newcommand{\debuttexte}{
%%%%%%%%%%%%% FONTES %%%%%%%%%%%%%
\renewcommand{\baselinestretch}{1.5}
\usefont{U}{cmss}{bx}{n}
\bfseries

% Taille normale : commenter le reste !
%Taille Arnaud
%\fontsize{19}{19}\selectfont

% Taille Barbara
%\fontsize{21}{22}\selectfont

%Taille François
%\fontsize{25}{30}\selectfont

%Taille Pascal
%\fontsize{25}{30}\selectfont

%Taille Laura
%\fontsize{30}{35}\selectfont


%\myfont
%\usefont{U}{cmss}{bx}{n}

%\Huge
%\addtolength{\parskip}{\baselineskip}
}


% \usepackage{hyperref}
% \hypersetup{colorlinks=true, linkcolor=blue, urlcolor=blue,
% pdftitle={Exo7 - Exercices de mathématiques}, pdfauthor={Exo7}}


%section
% \usepackage{sectsty}
% \allsectionsfont{\bf}
%\sectionfont{\color{Tomato3}\upshape\selectfont}
%\subsectionfont{\color{Tomato4}\upshape\selectfont}

%----- Ensembles : entiers, reels, complexes -----
\newcommand{\Nn}{\mathbb{N}} \newcommand{\N}{\mathbb{N}}
\newcommand{\Zz}{\mathbb{Z}} \newcommand{\Z}{\mathbb{Z}}
\newcommand{\Qq}{\mathbb{Q}} \newcommand{\Q}{\mathbb{Q}}
\newcommand{\Rr}{\mathbb{R}} \newcommand{\R}{\mathbb{R}}
\newcommand{\Cc}{\mathbb{C}} 
\newcommand{\Kk}{\mathbb{K}} \newcommand{\K}{\mathbb{K}}

%----- Modifications de symboles -----
\renewcommand{\epsilon}{\varepsilon}
\renewcommand{\Re}{\mathop{\text{Re}}\nolimits}
\renewcommand{\Im}{\mathop{\text{Im}}\nolimits}
%\newcommand{\llbracket}{\left[\kern-0.15em\left[}
%\newcommand{\rrbracket}{\right]\kern-0.15em\right]}

\renewcommand{\ge}{\geqslant}
\renewcommand{\geq}{\geqslant}
\renewcommand{\le}{\leqslant}
\renewcommand{\leq}{\leqslant}

%----- Fonctions usuelles -----
\newcommand{\ch}{\mathop{\mathrm{ch}}\nolimits}
\newcommand{\sh}{\mathop{\mathrm{sh}}\nolimits}
\renewcommand{\tanh}{\mathop{\mathrm{th}}\nolimits}
\newcommand{\cotan}{\mathop{\mathrm{cotan}}\nolimits}
\newcommand{\Arcsin}{\mathop{\mathrm{Arcsin}}\nolimits}
\newcommand{\Arccos}{\mathop{\mathrm{Arccos}}\nolimits}
\newcommand{\Arctan}{\mathop{\mathrm{Arctan}}\nolimits}
\newcommand{\Argsh}{\mathop{\mathrm{Argsh}}\nolimits}
\newcommand{\Argch}{\mathop{\mathrm{Argch}}\nolimits}
\newcommand{\Argth}{\mathop{\mathrm{Argth}}\nolimits}
\newcommand{\pgcd}{\mathop{\mathrm{pgcd}}\nolimits} 

\newcommand{\Card}{\mathop{\text{Card}}\nolimits}
\newcommand{\Ker}{\mathop{\text{Ker}}\nolimits}
\newcommand{\id}{\mathop{\text{id}}\nolimits}
\newcommand{\ii}{\mathrm{i}}
\newcommand{\dd}{\mathrm{d}}
\newcommand{\Vect}{\mathop{\text{Vect}}\nolimits}
\newcommand{\Mat}{\mathop{\mathrm{Mat}}\nolimits}
\newcommand{\rg}{\mathop{\text{rg}}\nolimits}
\newcommand{\tr}{\mathop{\text{tr}}\nolimits}
\newcommand{\ppcm}{\mathop{\text{ppcm}}\nolimits}

%----- Structure des exercices ------

\newtheoremstyle{styleexo}% name
{2ex}% Space above
{3ex}% Space below
{}% Body font
{}% Indent amount 1
{\bfseries} % Theorem head font
{}% Punctuation after theorem head
{\newline}% Space after theorem head 2
{}% Theorem head spec (can be left empty, meaning ‘normal’)

%\theoremstyle{styleexo}
\newtheorem{exo}{Exercice}
\newtheorem{ind}{Indications}
\newtheorem{cor}{Correction}


\newcommand{\exercice}[1]{} \newcommand{\finexercice}{}
%\newcommand{\exercice}[1]{{\tiny\texttt{#1}}\vspace{-2ex}} % pour afficher le numero absolu, l'auteur...
\newcommand{\enonce}{\begin{exo}} \newcommand{\finenonce}{\end{exo}}
\newcommand{\indication}{\begin{ind}} \newcommand{\finindication}{\end{ind}}
\newcommand{\correction}{\begin{cor}} \newcommand{\fincorrection}{\end{cor}}

\newcommand{\noindication}{\stepcounter{ind}}
\newcommand{\nocorrection}{\stepcounter{cor}}

\newcommand{\fiche}[1]{} \newcommand{\finfiche}{}
\newcommand{\titre}[1]{\centerline{\large \bf #1}}
\newcommand{\addcommand}[1]{}
\newcommand{\video}[1]{}

% Marge
\newcommand{\mymargin}[1]{\marginpar{{\small #1}}}



%----- Presentation ------
\setlength{\parindent}{0cm}

%\newcommand{\ExoSept}{\href{http://exo7.emath.fr}{\textbf{\textsf{Exo7}}}}

\definecolor{myred}{rgb}{0.93,0.26,0}
\definecolor{myorange}{rgb}{0.97,0.58,0}
\definecolor{myyellow}{rgb}{1,0.86,0}

\newcommand{\LogoExoSept}[1]{  % input : echelle
{\usefont{U}{cmss}{bx}{n}
\begin{tikzpicture}[scale=0.1*#1,transform shape]
  \fill[color=myorange] (0,0)--(4,0)--(4,-4)--(0,-4)--cycle;
  \fill[color=myred] (0,0)--(0,3)--(-3,3)--(-3,0)--cycle;
  \fill[color=myyellow] (4,0)--(7,4)--(3,7)--(0,3)--cycle;
  \node[scale=5] at (3.5,3.5) {Exo7};
\end{tikzpicture}}
}



\theoremstyle{definition}
%\newtheorem{proposition}{Proposition}
%\newtheorem{exemple}{Exemple}
%\newtheorem{theoreme}{Théorème}
\newtheorem{lemme}{Lemme}
\newtheorem{corollaire}{Corollaire}
%\newtheorem*{remarque*}{Remarque}
%\newtheorem*{miniexercice}{Mini-exercices}
%\newtheorem{definition}{Définition}




%definition d'un terme
\newcommand{\defi}[1]{{\color{myorange}\textbf{\emph{#1}}}}
\newcommand{\evidence}[1]{{\color{blue}\textbf{\emph{#1}}}}



 %----- Commandes divers ------

\newcommand{\codeinline}[1]{\texttt{#1}}

%%%%%%%%%%%%%%%%%%%%%%%%%%%%%%%%%%%%%%%%%%%%%%%%%%%%%%%%%%%%%
%%%%%%%%%%%%%%%%%%%%%%%%%%%%%%%%%%%%%%%%%%%%%%%%%%%%%%%%%%%%%


\begin{document}

\debuttexte

%%%%%%%%%%%%%%%%%%%%%%%%%%%%%%%%%%%%%%%%%%%%%%%%%%%%%%%%%%%
\diapo

Voici une leçon consacrée à la comparaison série-intégrale, qui fait la jonction entre les séries et les intégrales impropres. C'est un lien essentiel entre deux objets mathématiques qui sont au final assez proches.

\change

\change
nous commencerons par énoncer le théorème de comparaison série/intégrale,

\change
puis nous en donnerons la démonstration.

\change
Nous l'appliquerons ensuite aux séries de Riemann qui sont des séries de références,

\change
et aux séries de Bertrand.

\change
Enfin, nous donnerons quelques applications des résultats précédents.

%%%%%%%%%%%%%%%%%%%%%%%%%%%%%%%%%%%%%%%%%%%%%%%%%%%%%%%%%%%
\diapo

Pour l'ensemble de cette leçon, 
il faut connaître les intégrales impropres 
$\int_0^{+\infty} f(t) \;\dd t$.

Si ce n'est pas encore votre cas, il faut quand 
même étudier cette séquence et retenir de 
le résultat sur les séries de Riemann.



\change
Théorème.

Soit $f : [0,+\infty[ \to [0,+\infty[$ une fonction décroissante.

\change
Alors la série $\sum_{k \ge 0} f(k)$ (dont le terme général est $u_k = f(k)$) 
et l'intégrale impropre $\int_0^{+\infty} f(t) \; \dd t$ sont de même nature.

\change
"De même nature" signifie que la série et l'intégrale du théorème sont soit toutes deux
convergentes, soit toutes deux divergentes.

\change
\textbf{Attention !} Il est important que $f$ soit positive et décroissante.

%%%%%%%%%%%%%%%%%%%%%%%%%%%%%%%%%%%%%%%%%%%%%%%%%%%%%%%%%%%
\diapo

Le plus simple est de bien comprendre le dessin et de refaire la démonstration
chaque fois que l'on en a besoin.

Soit $k\in\Nn$. 

\change
Comme $f$ est décroissante, pour $k\le t \le k+1$, on a $f(k+1)\le f(t)\le f(k)$ (attention à l'ordre). 

\change
En intégrant sur l'intervalle $[k,k+1]$ de longueur $1$, on obtient:

\change
$$f(k+1)\le \int_k^{k+1} f(t) \; \dd t \le f(k)$$  

\change
Sur le dessin cette inégalité signifie que l'aire sous la courbe, 
entre les abscisses $k$ et $k+1$, est comprise entre l'aire du 
rectangle vert de hauteur $f(k+1)$ et de base $1$ et 
l'aire du rectangle bleu de hauteur $f(k)$ et de même base $1$.

Poursuivons la démonstration à la diapo suivante.

%%%%%%%%%%%%%%%%%%%%%%%%%%%%%%%%%%%%%%%%%%%%%%%%%%%%%%%%%%%
\diapo

On vient d'obtenir cet encadrement pour tout $k\in\Nn$.

\change
On somme à présent ces inégalités pour $k$ variant de $0$ à $n-1$ :

\change
$$\sum_{k=0}^{n-1} f(k+1) \le \sum_{k=0}^{n-1} \int_{k}^{k+1} f(t) \; \dd t
\le \sum_{k=0}^{n-1} f(k).$$

\change
Soit :
$$u_1+\cdots+u_{n} \le \int_0^{n} f(t)\; \dd t \le u_0+\cdots+u_{n-1}.$$

\change
La série $\sum u_k$ converge et a pour somme $S$ si et seulement si la suite des sommes
partielles converge vers $S$. 

\change 
Si c'est le cas $\int_0^{n} f(t)\; \dd t$ est majorée par $S$

\change
et comme $\int_0^x f(t)\; \dd t$ est une fonction croissante de $x$ (par positivité de $f$), l'intégrale converge. 

\change
Réciproquement, si l'intégrale converge, alors $\int_0^{n} f(t)\;\dd t$ est majorée, 

\change
la suite des sommes partielles aussi, et la série converge.

%%%%%%%%%%%%%%%%%%%%%%%%%%%%%%%%%%%%%%%%%%%%%%%%%%%%%%%%%%%
\diapo

Le théorème de comparaison et le théorème des équivalents permettent de ramener l'étude des séries à termes positifs à un catalogue de séries dont la convergence est connue. Dans ce catalogue, on trouve les séries de Riemann et les séries de Bertrand.

Commençons par les \defi{séries de Riemann} $\sum_{k\ge 1} \frac{1}{k^\alpha}$, pour $\alpha>0$
un réel.

\change
Proposition.

\change

Si $\alpha >1$ alors $\displaystyle  \sum_{k = 1}^{+\infty} \frac{1}{k^\alpha}$ converge

\change
Si $0 <\alpha \le1$ alors $\displaystyle  \sum_{k \ge 1} \frac{1}{k^\alpha}$ diverge


%%%%%%%%%%%%%%%%%%%%%%%%%%%%%%%%%%%%%%%%%%%%%%%%%%%%%%%%%%%
\diapo

\change
Dans le théorème précédent sur la comparaison série/intégrale, rien n'oblige à démarrer de $0$ : pour $m\in\Nn$,
la série $\sum_{k \ge m} f(k)$ et l'intégrale impropre $\int_m^{+\infty} f(t) \;\dd t$ sont de même nature.

\change
Nous l'appliquons à $f : [1,+\infty[ \to [0,+\infty[$ définie par $f(t)=\frac{1}{t^\alpha}$. 

\change
Pour $\alpha >0$, c'est une fonction décroissante et positive.

\change
On peut appliquer le théorème de comparaison série/intégrale.

\change
Or on sait que :
$$
\int_1^{x} \frac{1}{t^\alpha}\;\dd t$$

est égal à $ \displaystyle{\frac{1}{1-\alpha}(x^{1-\alpha}-1)}$ si $\alpha\neq 1$,

\change
et à $\ln(x)$ si $\alpha=1$.

\change
Pour $\alpha > 1$, $\int_1^{+\infty} \frac{1}{t^\alpha} \; \dd t $ 
est convergente, donc la série $\sum_{k = 1}^{+\infty} \frac{1}{k^\alpha}$ converge.

\change
Pour $0<\alpha\le 1$, $\int_1^{+\infty} \frac{1}{t^\alpha} \; \dd t $ 
est divergente, donc la série $\sum_{k \ge 1} \frac{1}{k^\alpha}$ diverge.


%%%%%%%%%%%%%%%%%%%%%%%%%%%%%%%%%%%%%%%%%%%%%%%%%%%%%%%%%%%
\diapo

Une famille de séries plus sophistiquées sont les \defi{séries de Bertrand} : 
$\displaystyle\sum_{k\ge2} \frac{1}{k^\alpha(\ln k)^\beta}$
où $\alpha > 0$ et $\beta \in \Rr$.

\change
Proposition.

Si \quad $\alpha>1$ \quad  alors la série de Bertrand converge. 

Si \quad $0<\alpha<1$ \quad  alors elle diverge.

Pour l'instant c'est comme les séries de Riemann.

\change
Si \quad $\alpha=1$ \quad et \quad $\begin{cases} \beta>1 & \text{alors elle converge.}\\   
\beta\le 1 & \text{alors elle diverge.}
\end{cases}$

\change
La démonstration est la même que pour les séries de Riemann.

\change
Par exemple pour le cas $\alpha=1$ : 

\change
$$
\int_2^{x} \frac{1}{t(\ln t)^{\beta}}\;\dd t =
\left\{\begin{array}{ll}
\displaystyle{\frac{1}{1-\beta}\left((\ln x)^{1-\beta}-(\ln 2)^{1-\beta}\right)}&\text{si }\beta\neq
    1\\[1.5ex]
\ln(\ln x)-\ln(\ln 2) &\text{si }\beta=1
\end{array}\right. 
$$

%%%%%%%%%%%%%%%%%%%%%%%%%%%%%%%%%%%%%%%%%%%%%%%%%%%%%%%%%%%
\diapo

Nous retrouvons en particulier le fait que :

 $\sum \frac{1}{k^2}$ converge (prendre $\alpha=2$),

\change
alors que $\sum \frac{1}{k}$ diverge (prendre $\alpha=1$).

\change
Terminons avec deux exemples d'utilisation des équivalents 
avec les séries de Riemann et de Bertrand.

Tout d'abord, la série 
  $$\sum_{k \ge 1} \ln\left(1+\frac{1}{\sqrt{k^3}}\right)$$ 
  est-elle convergente ?
 
\change 
Comme
 $$\ln\left(1+\frac{1}{\sqrt{k^3}}\right) \quad \sim \quad \frac{1}{\sqrt{k^3}}$$
  
\change  
et comme la série de Riemann $\sum \frac{1}{\sqrt{k^3}}=\sum \frac{1}{k^{\frac32}}$ converge (car $\frac32>1$) 

\change
alors par le théorème des équivalents la série $\sum_{k=1}^{+\infty} \ln\left(1+\frac{1}{\sqrt{k^3}}\right)$ converge également.

%%%%%%%%%%%%%%%%%%%%%%%%%%%%%%%%%%%%%%%%%%%%%%%%%%%%%%%%%%%
\diapo
 La série
$$\sum_{k \ge 1} 
\frac{1-\cos\left(\frac{1}{k\sqrt{\ln k}}\right)}{\sin\left(\frac{1}{k}\right)}$$
est-elle convergente ?

\change 
On cherche un équivalent du terme général (qui est positif) :  
$$\frac{1-\cos\left(\frac{1}{k\sqrt{\ln k}}\right)}
{\sin\left(\frac{1}{k}\right)} \quad \sim \quad \frac{1}{2k\ln k}$$

\change
Or la série de Bertrand $\sum \frac{1}{k\ln k}$ diverge, 

\change
donc
notre série diverge aussi.

%%%%%%%%%%%%%%%%%%%%%%%%%%%%%%%%%%%%%%%%%%%%%%%%%%%%%%%%%%%
\diapo


\end{document}


%%%%%%%%%%%%%%%%%% PREAMBULE %%%%%%%%%%%%%%%%%%

\documentclass[aspectratio=169,utf8]{beamer}
%\documentclass[aspectratio=169,handout]{beamer}

\usetheme{Boadilla}
%\usecolortheme{seahorse}
%\usecolortheme[RGB={245,66,24}]{structure}
\useoutertheme{infolines}

% packages
\usepackage{amsfonts,amsmath,amssymb,amsthm}
\usepackage[utf8]{inputenc}
\usepackage[T1]{fontenc}
\usepackage{lmodern}

\usepackage[francais]{babel}
\usepackage{fancybox}
\usepackage{graphicx}

\usepackage{float}
\usepackage{xfrac}

%\usepackage[usenames, x11names]{xcolor}
\usepackage{pgfplots}
\usepackage{datetime}


% ----------------------------------------------------------------------
% Pour les images
\usepackage{tikz}
\usetikzlibrary{calc,shadows,arrows.meta,patterns,matrix}

\newcommand{\tikzinput}[1]{\input{figures/#1.tikz}}
% --- les figures avec échelle éventuel
\newcommand{\myfigure}[2]{% entrée : échelle, fichier(s) figure à inclure
\begin{center}\small%
\tikzstyle{every picture}=[scale=1.0*#1]% mise en échelle + 0% (automatiquement annulé à la fin du groupe)
#2%
\end{center}}



%-----  Package unités -----
\usepackage{siunitx}
\sisetup{locale = FR,detect-all,per-mode = symbol}

%\usepackage{mathptmx}
%\usepackage{fouriernc}
%\usepackage{newcent}
%\usepackage[mathcal,mathbf]{euler}

%\usepackage{palatino}
%\usepackage{newcent}
% \usepackage[mathcal,mathbf]{euler}



% \usepackage{hyperref}
% \hypersetup{colorlinks=true, linkcolor=blue, urlcolor=blue,
% pdftitle={Exo7 - Exercices de mathématiques}, pdfauthor={Exo7}}


%section
% \usepackage{sectsty}
% \allsectionsfont{\bf}
%\sectionfont{\color{Tomato3}\upshape\selectfont}
%\subsectionfont{\color{Tomato4}\upshape\selectfont}

%----- Ensembles : entiers, reels, complexes -----
\newcommand{\Nn}{\mathbb{N}} \newcommand{\N}{\mathbb{N}}
\newcommand{\Zz}{\mathbb{Z}} \newcommand{\Z}{\mathbb{Z}}
\newcommand{\Qq}{\mathbb{Q}} \newcommand{\Q}{\mathbb{Q}}
\newcommand{\Rr}{\mathbb{R}} \newcommand{\R}{\mathbb{R}}
\newcommand{\Cc}{\mathbb{C}} 
\newcommand{\Kk}{\mathbb{K}} \newcommand{\K}{\mathbb{K}}

%----- Modifications de symboles -----
\renewcommand{\epsilon}{\varepsilon}
\renewcommand{\Re}{\mathop{\text{Re}}\nolimits}
\renewcommand{\Im}{\mathop{\text{Im}}\nolimits}
%\newcommand{\llbracket}{\left[\kern-0.15em\left[}
%\newcommand{\rrbracket}{\right]\kern-0.15em\right]}

\renewcommand{\ge}{\geqslant}
\renewcommand{\geq}{\geqslant}
\renewcommand{\le}{\leqslant}
\renewcommand{\leq}{\leqslant}
\renewcommand{\epsilon}{\varepsilon}

%----- Fonctions usuelles -----
\newcommand{\ch}{\mathop{\text{ch}}\nolimits}
\newcommand{\sh}{\mathop{\text{sh}}\nolimits}
\renewcommand{\tanh}{\mathop{\text{th}}\nolimits}
\newcommand{\cotan}{\mathop{\text{cotan}}\nolimits}
\newcommand{\Arcsin}{\mathop{\text{arcsin}}\nolimits}
\newcommand{\Arccos}{\mathop{\text{arccos}}\nolimits}
\newcommand{\Arctan}{\mathop{\text{arctan}}\nolimits}
\newcommand{\Argsh}{\mathop{\text{argsh}}\nolimits}
\newcommand{\Argch}{\mathop{\text{argch}}\nolimits}
\newcommand{\Argth}{\mathop{\text{argth}}\nolimits}
\newcommand{\pgcd}{\mathop{\text{pgcd}}\nolimits} 


%----- Commandes divers ------
\newcommand{\ii}{\mathrm{i}}
\newcommand{\dd}{\text{d}}
\newcommand{\id}{\mathop{\text{id}}\nolimits}
\newcommand{\Ker}{\mathop{\text{Ker}}\nolimits}
\newcommand{\Card}{\mathop{\text{Card}}\nolimits}
\newcommand{\Vect}{\mathop{\text{Vect}}\nolimits}
\newcommand{\Mat}{\mathop{\text{Mat}}\nolimits}
\newcommand{\rg}{\mathop{\text{rg}}\nolimits}
\newcommand{\tr}{\mathop{\text{tr}}\nolimits}


%----- Structure des exercices ------

\newtheoremstyle{styleexo}% name
{2ex}% Space above
{3ex}% Space below
{}% Body font
{}% Indent amount 1
{\bfseries} % Theorem head font
{}% Punctuation after theorem head
{\newline}% Space after theorem head 2
{}% Theorem head spec (can be left empty, meaning ‘normal’)

%\theoremstyle{styleexo}
\newtheorem{exo}{Exercice}
\newtheorem{ind}{Indications}
\newtheorem{cor}{Correction}


\newcommand{\exercice}[1]{} \newcommand{\finexercice}{}
%\newcommand{\exercice}[1]{{\tiny\texttt{#1}}\vspace{-2ex}} % pour afficher le numero absolu, l'auteur...
\newcommand{\enonce}{\begin{exo}} \newcommand{\finenonce}{\end{exo}}
\newcommand{\indication}{\begin{ind}} \newcommand{\finindication}{\end{ind}}
\newcommand{\correction}{\begin{cor}} \newcommand{\fincorrection}{\end{cor}}

\newcommand{\noindication}{\stepcounter{ind}}
\newcommand{\nocorrection}{\stepcounter{cor}}

\newcommand{\fiche}[1]{} \newcommand{\finfiche}{}
\newcommand{\titre}[1]{\centerline{\large \bf #1}}
\newcommand{\addcommand}[1]{}
\newcommand{\video}[1]{}

% Marge
\newcommand{\mymargin}[1]{\marginpar{{\small #1}}}

\def\noqed{\renewcommand{\qedsymbol}{}}


%----- Presentation ------
\setlength{\parindent}{0cm}

%\newcommand{\ExoSept}{\href{http://exo7.emath.fr}{\textbf{\textsf{Exo7}}}}

\definecolor{myred}{rgb}{0.93,0.26,0}
\definecolor{myorange}{rgb}{0.97,0.58,0}
\definecolor{myyellow}{rgb}{1,0.86,0}

\newcommand{\LogoExoSept}[1]{  % input : echelle
{\usefont{U}{cmss}{bx}{n}
\begin{tikzpicture}[scale=0.1*#1,transform shape]
  \fill[color=myorange] (0,0)--(4,0)--(4,-4)--(0,-4)--cycle;
  \fill[color=myred] (0,0)--(0,3)--(-3,3)--(-3,0)--cycle;
  \fill[color=myyellow] (4,0)--(7,4)--(3,7)--(0,3)--cycle;
  \node[scale=5] at (3.5,3.5) {Exo7};
\end{tikzpicture}}
}


\newcommand{\debutmontitre}{
  \author{} \date{} 
  \thispagestyle{empty}
  \hspace*{-10ex}
  \begin{minipage}{\textwidth}
    \titlepage  
  \vspace*{-2.5cm}
  \begin{center}
    \LogoExoSept{2.5}
  \end{center}
  \end{minipage}

  \vspace*{-0cm}
  
  % Astuce pour que le background ne soit pas discrétisé lors de la conversion pdf -> png
\begin{tikzpicture}
        \fill[opacity=0,green!60!black] (0,0)--++(0,0)--++(0,0)--++(0,0)--cycle; 
\end{tikzpicture}

% toc S'affiche trop tot :
% \tableofcontents[hideallsubsections, pausesections]
}

\newcommand{\finmontitre}{
  \end{frame}
  \setcounter{framenumber}{0}
} % ne marche pas pour une raison obscure

%----- Commandes supplementaires ------

% \usepackage[landscape]{geometry}
% \geometry{top=1cm, bottom=3cm, left=2cm, right=10cm, marginparsep=1cm
% }
% \usepackage[a4paper]{geometry}
% \geometry{top=2cm, bottom=2cm, left=2cm, right=2cm, marginparsep=1cm
% }

%\usepackage{standalone}


% New command Arnaud -- november 2011
\setbeamersize{text margin left=24ex}
% si vous modifier cette valeur il faut aussi
% modifier le decalage du titre pour compenser
% (ex : ici =+10ex, titre =-5ex

\theoremstyle{definition}
%\newtheorem{proposition}{Proposition}
%\newtheorem{exemple}{Exemple}
%\newtheorem{theoreme}{Théorème}
%\newtheorem{lemme}{Lemme}
%\newtheorem{corollaire}{Corollaire}
%\newtheorem*{remarque*}{Remarque}
%\newtheorem*{miniexercice}{Mini-exercices}
%\newtheorem{definition}{Définition}

% Commande tikz
\usetikzlibrary{calc}
\usetikzlibrary{patterns,arrows}
\usetikzlibrary{matrix}
\usetikzlibrary{fadings} 

%definition d'un terme
\newcommand{\defi}[1]{{\color{myorange}\textbf{\emph{#1}}}}
\newcommand{\evidence}[1]{{\color{blue}\textbf{\emph{#1}}}}
\newcommand{\assertion}[1]{\emph{\og#1\fg}}  % pour chapitre logique
%\renewcommand{\contentsname}{Sommaire}
\renewcommand{\contentsname}{}
\setcounter{tocdepth}{2}



%------ Encadrement ------

\usepackage{fancybox}


\newcommand{\mybox}[1]{
\setlength{\fboxsep}{7pt}
\begin{center}
\shadowbox{#1}
\end{center}}

\newcommand{\myboxinline}[1]{
\setlength{\fboxsep}{5pt}
\raisebox{-10pt}{
\shadowbox{#1}
}
}

%--------------- Commande beamer---------------
\newcommand{\beameronly}[1]{#1} % permet de mettre des pause dans beamer pas dans poly


\setbeamertemplate{navigation symbols}{}
\setbeamertemplate{footline}  % tiré du fichier beamerouterinfolines.sty
{
  \leavevmode%
  \hbox{%
  \begin{beamercolorbox}[wd=.333333\paperwidth,ht=2.25ex,dp=1ex,center]{author in head/foot}%
    % \usebeamerfont{author in head/foot}\insertshortauthor%~~(\insertshortinstitute)
    \usebeamerfont{section in head/foot}{\bf\insertshorttitle}
  \end{beamercolorbox}%
  \begin{beamercolorbox}[wd=.333333\paperwidth,ht=2.25ex,dp=1ex,center]{title in head/foot}%
    \usebeamerfont{section in head/foot}{\bf\insertsectionhead}
  \end{beamercolorbox}%
  \begin{beamercolorbox}[wd=.333333\paperwidth,ht=2.25ex,dp=1ex,right]{date in head/foot}%
    % \usebeamerfont{date in head/foot}\insertshortdate{}\hspace*{2em}
    \insertframenumber{} / \inserttotalframenumber\hspace*{2ex} 
  \end{beamercolorbox}}%
  \vskip0pt%
}


\definecolor{mygrey}{rgb}{0.5,0.5,0.5}
\setlength{\parindent}{0cm}
%\DeclareTextFontCommand{\helvetica}{\fontfamily{phv}\selectfont}

% background beamer
\definecolor{couleurhaut}{rgb}{0.85,0.9,1}  % creme
\definecolor{couleurmilieu}{rgb}{1,1,1}  % vert pale
\definecolor{couleurbas}{rgb}{0.85,0.9,1}  % blanc
\setbeamertemplate{background canvas}[vertical shading]%
[top=couleurhaut,middle=couleurmilieu,midpoint=0.4,bottom=couleurbas] 
%[top=fondtitre!05,bottom=fondtitre!60]



\makeatletter
\setbeamertemplate{theorem begin}
{%
  \begin{\inserttheoremblockenv}
  {%
    \inserttheoremheadfont
    \inserttheoremname
    \inserttheoremnumber
    \ifx\inserttheoremaddition\@empty\else\ (\inserttheoremaddition)\fi%
    \inserttheorempunctuation
  }%
}
\setbeamertemplate{theorem end}{\end{\inserttheoremblockenv}}

\newenvironment{theoreme}[1][]{%
   \setbeamercolor{block title}{fg=structure,bg=structure!40}
   \setbeamercolor{block body}{fg=black,bg=structure!10}
   \begin{block}{{\bf Th\'eor\`eme }#1}
}{%
   \end{block}%
}


\newenvironment{proposition}[1][]{%
   \setbeamercolor{block title}{fg=structure,bg=structure!40}
   \setbeamercolor{block body}{fg=black,bg=structure!10}
   \begin{block}{{\bf Proposition }#1}
}{%
   \end{block}%
}

\newenvironment{corollaire}[1][]{%
   \setbeamercolor{block title}{fg=structure,bg=structure!40}
   \setbeamercolor{block body}{fg=black,bg=structure!10}
   \begin{block}{{\bf Corollaire }#1}
}{%
   \end{block}%
}

\newenvironment{mydefinition}[1][]{%
   \setbeamercolor{block title}{fg=structure,bg=structure!40}
   \setbeamercolor{block body}{fg=black,bg=structure!10}
   \begin{block}{{\bf Définition} #1}
}{%
   \end{block}%
}

\newenvironment{lemme}[0]{%
   \setbeamercolor{block title}{fg=structure,bg=structure!40}
   \setbeamercolor{block body}{fg=black,bg=structure!10}
   \begin{block}{\bf Lemme}
}{%
   \end{block}%
}

\newenvironment{remarque}[1][]{%
   \setbeamercolor{block title}{fg=black,bg=structure!20}
   \setbeamercolor{block body}{fg=black,bg=structure!5}
   \begin{block}{Remarque #1}
}{%
   \end{block}%
}


\newenvironment{exemple}[1][]{%
   \setbeamercolor{block title}{fg=black,bg=structure!20}
   \setbeamercolor{block body}{fg=black,bg=structure!5}
   \begin{block}{{\bf Exemple }#1}
}{%
   \end{block}%
}


\newenvironment{miniexercice}[0]{%
   \setbeamercolor{block title}{fg=structure,bg=structure!20}
   \setbeamercolor{block body}{fg=black,bg=structure!5}
   \begin{block}{Mini-exercices}
}{%
   \end{block}%
}


\newenvironment{tp}[0]{%
   \setbeamercolor{block title}{fg=structure,bg=structure!40}
   \setbeamercolor{block body}{fg=black,bg=structure!10}
   \begin{block}{\bf Travaux pratiques}
}{%
   \end{block}%
}
\newenvironment{exercicecours}[1][]{%
   \setbeamercolor{block title}{fg=structure,bg=structure!40}
   \setbeamercolor{block body}{fg=black,bg=structure!10}
   \begin{block}{{\bf Exercice }#1}
}{%
   \end{block}%
}
\newenvironment{algo}[1][]{%
   \setbeamercolor{block title}{fg=structure,bg=structure!40}
   \setbeamercolor{block body}{fg=black,bg=structure!10}
   \begin{block}{{\bf Algorithme}\hfill{\color{gray}\texttt{#1}}}
}{%
   \end{block}%
}


\setbeamertemplate{proof begin}{
   \setbeamercolor{block title}{fg=black,bg=structure!20}
   \setbeamercolor{block body}{fg=black,bg=structure!5}
   \begin{block}{{\footnotesize Démonstration}}
   \footnotesize
   \smallskip}
\setbeamertemplate{proof end}{%
   \end{block}}
\setbeamertemplate{qed symbol}{\openbox}


\makeatother
% Couleur à définir

% Commande spécifique à ce chapitre
\usepackage{mathtools}
   
%%%%%%%%%%%%%%%%%%%%%%%%%%%%%%%%%%%%%%%%%%%%%%%%%%%%%%%%%%%%%
%%%%%%%%%%%%%%%%%%%%%%%%%%%%%%%%%%%%%%%%%%%%%%%%%%%%%%%%%%%%%


\begin{document}


\title{{\bf Séries}}
\subtitle{Produits de deux séries}

\begin{frame}
  
  \debutmontitre

  \pause

{\footnotesize
\hfill
\setbeamercovered{transparent=50}
\begin{minipage}{0.6\textwidth}
  \begin{itemize}
    \item<3-> Motivation
    \item<4-> Le produit de Cauchy
    \item<5-> Exemple
    \item<6-> Contre-exemple
  \end{itemize}
\end{minipage}
}

\end{frame}

\setcounter{framenumber}{0}



%%%%%%%%%%%%%%%%%%%%%%%%%%%%%%%%%%%%%%%%%%%%%%%%%%%%%%%%%%%%%%%%
\section{Motivation}

\begin{frame}
\begin{itemize}
  \item Il y a plusieurs façons d'ordonner les termes lorsqu'on développe un produit de sommes
  \uncover<3->{\item Regroupons les termes en fonction de la somme des indices}
\end{itemize}

\uncover<2->{
$$
\big(a_0+a_1\big)\big(b_0+b_1\big)
=
\only<2-3>{
a_0b_0 +  a_0b_1 + a_1b_0+ a_1b_1
}
\only<4->{
\underbrace{a_0b_0}_{\mathllap{\color{blue}\text{somme des indices}= \ \color{blue}{0}}}
+  \underbrace{a_0b_1 + a_1b_0}_{\color{blue}{1}}
 +  \underbrace{a_1b_1}_{\color{blue}{2}}
}
$$
}
\uncover<2->{\hspace{-.4cm}
\begin{align*}
\big(a_0+a_1+a_2\big)\big(b_0+b_1+b_2\big)  
=
\only<2-4>{
&
a_0b_0+a_0b_1 + a_1b_0 +a_0b_2+a_1b_1+a_2b_0\\
&\\
&+a_1b_2+a_2b_1+a_2b_2
}
\only<5->{
&
\underbrace{a_0b_0}_{\mathllap{\color{blue}\text{somme des indices}=\ \color{blue}{0}}}
+\underbrace{a_0b_1 + a_1b_0}_{\color{blue}{1}}  +\underbrace{a_0b_2+a_1b_1+a_2b_0}_{\color{blue}{2}}\\
&+\underbrace{a_1b_2+a_2b_1}_{\color{blue}{3}}
+\underbrace{a_2b_2}_{\color{blue}{4}} 
}
\end{align*}
}


\pause\pause\pause\pause\pause
Voici différentes façons d'écrire un produit de deux sommes
$$\hspace{-.4cm}
\left(\sum_{i=0}^n a_i\right)\; \left(\sum_{j=0}^n b_j\right)
= 
\pause
\sum_{i=0}^n \sum_{j=0}^n a_ib_j
\pause
= \sum_{0 \le k \le 2n} \sum_{i+j=k} a_ib_j
\pause
= \sum_{0 \le k \le 2n} \sum_{0 \le i \le k} a_ib_{k-i}$$

\end{frame}


%%%%%%%%%%%%%%%%%%%%%%%%%%%%%%%%%%%%%%%%%%%%%%%%%%%%%%%%%%%%%%%%
\section{Le produit de Cauchy}

\begin{frame}
\begin{mydefinition}
Soient $\displaystyle\sum_{i \ge 0} a_i$ et $\displaystyle\sum_{j \ge 0} b_j$ deux séries

\pause
On appelle \defi{série produit} ou \defi{produit de Cauchy}
la série $\displaystyle\sum_{k \ge 0} c_k$
o\`u \mybox{$\displaystyle c_k=\sum_{i=0}^k a_i b_{k-i}$} 
\end{mydefinition}

\pause
\medskip

\mybox{$\displaystyle c_k=\sum_{i+j=k} a_i b_j$}

\end{frame}

\begin{frame}
\begin{theoreme}
Si $\displaystyle\sum_{i=0}^{+\infty} a_i$ et $\displaystyle\sum_{j=0}^{+\infty} b_j$ sont absolument convergentes, \pause alors la série produit 
\vspace{-.5cm}
$$\sum_{k=0}^{+\infty} c_k = \sum_{k=0}^{+\infty} \left(\sum_{i=0}^k a_ib_{k-i}\right)$$
\vspace{-.4cm}

est absolument convergente \pause et l'on a:

\vspace{-.3cm}
\mybox{$\displaystyle
\sum_{k=0}^{+\infty} c_k = \left(\sum_{i=0}^{+\infty} a_i\right)\ \times\ \left(\sum_{j=0}^{+\infty} b_j\right)$
}
\end{theoreme}

\pause
\begin{proof}
\begin{tabular}{ll}
\textbf{Notations :}&
\pause $S_n=a_0+\dots+a_n$, $S_n\to S$ \\
\pause& $T_n=b_0+\dots +b_n$, $T_n\to T$\\
\pause& $P_n= c_0+\dots+c_n$
\end{tabular}

\pause
On doit montrer que $P_n \to S\cdot T$
\noqed
\end{proof}
\end{frame}


\begin{frame}
\begin{proof}
\textbf{Premier cas :} $a_k\ge 0, b_k \ge 0$ ($\forall k$)

\begin{itemize}
\item\pause Alors $c_k\ge 0$ et on a 
$$P_n \le S_n \cdot T_n\le S \cdot T$$ 

\pause
La suite $(P_n)$ est croissante et majorée, donc convergente : $P_n\to P$

\item\pause Or on a aussi :

\vspace{-.5cm}
\begin{minipage}{0.39\textwidth}
$$P_n \le S_n \cdot T_n \le P_{2n}$$  
\end{minipage}
\pause
\begin{minipage}{0.49\textwidth}
\myfigure{.6}{
\tikzinput{fig_series03} 
}  
\end{minipage}

\item\pause Donc en faisant $n\to+\infty$, on a: $P\le S \cdot T\le P$. \pause Donc $P_n\to S \cdot T$\noqed\qedhere
\end{itemize}
\end{proof}
\end{frame}


\begin{frame}
\begin{proof}
\textbf{Second cas :} $a_k\in\Cc, b_k \in \Cc$ ($\forall k$)
\begin{itemize}
  \item\pause 
On pose :
\begin{itemize}
  \item $S_n'=|a_0|+\dots+|a_n|$, $S_n'\to S'$
  
  \item\pause $T_n'=|b_0|+\dots+|b_n|$, $T_n'\to T'$
  
  \item\pause $P_n'= c_0'+\dots+ c_n'$ o\`u $c_k'=\sum_{i=0}^k|a_ib_{k-i}|$
\end{itemize}
\pause
D'après le premier cas, $P_n'\to P'$ avec $P'=S' \cdot T'$
\item\pause Alors
$$|S_n \cdot T_n - P_n|= \bigl|\sum_{\stackrel{0\le i,j\le n}{i+j>n}} a_ib_j\bigr| 
\pause \le \sum_{\stackrel{0\le i,j\le n}{i+j>n}} |a_ib_j| 
\pause = S_n' \cdot T_n'-P_n' 
\pause \to S' \cdot T'-P'=0
$$

\pause
Ainsi $P_n=S_n \cdot T_n-(S_n \cdot T_n-P_n)\to S \cdot T-0=S \cdot T$

\pause
Donc la série $\sum c_k$ est convergente et sa somme est $S \cdot T$

\item\pause  De plus, $|c_k|\le c_k'$. \pause La convergence de $\sum c_k'$ implique 
donc la convergence absolue de $\sum c_k$ \qedhere
\end{itemize}
\end{proof}
\end{frame}


%%%%%%%%%%%%%%%%%%%%%%%%%%%%%%%%%%%%%%%%%%%%%%%%%%%%%%%%%%%%%%%%
\section{Exemple}

\begin{frame}
\begin{exemple}
\begin{tabular}{ll}
Soient :
& $\displaystyle\sum_{i=0}^{+\infty} a_i$ une série absolument convergente \\
\pause
& $\displaystyle\sum_{j=0}^{+\infty} b_j$ la série définie par $b_j = \dfrac{1}{2^j}$ (absolument convergente)
\end{tabular}
 

\begin{itemize}
  \item\pause Notons
$$c_k = \sum_{i=0}^k a_ib_{k-i} = \sum_{i=0}^k a_i \times \frac{1}{2^{k-i}}$$

\item\pause Alors la série $\sum c_k$ converge absolument et
$$\sum_{k=0}^{+\infty} c_k 
= \left(\sum_{i=0}^{+\infty} a_i\right)\ \times\ \left(\sum_{j=0}^{+\infty} b_j\right)
\pause = 2 \sum_{i=0}^{+\infty} a_i$$
\end{itemize}
\end{exemple}
\end{frame}


%%%%%%%%%%%%%%%%%%%%%%%%%%%%%%%%%%%%%%%%%%%%%%%%%%%%%%%%%%%%%%%%
\section{Contre-exemple}

\begin{frame}
$\sum a_i$ et $\sum b_j$ doivent être \evidence{absolument}
convergentes

\pause
\begin{exemple}
Soient $a_i=b_i=\frac{(-1)^{i}}{\sqrt {i+1}}, i\ge 0$. \pause Alors $\sum a_i$ et $\sum b_j$ sont convergentes mais pas absolument convergentes. \pause On a :
\begin{itemize}
\item $c_k = \displaystyle\sum_{i=0}^k a_i b_{k-i}  
\pause = \sum_{i=0}^k \tfrac{(-1)^{i}}{\sqrt {i+1}} \tfrac{(-1)^{k-i}}{\sqrt{k-i+1}} 
\pause = (-1)^{k} \sum_{i=0}^k \tfrac{1}{\sqrt{(i+1)(k-i+1)}}$

\item\pause Or $(x+1)(k-x+1)=-x^2+kx+(k+1) \le \frac{(k+2)^2}{4}$ (pour $x\in\Rr$)

\pause
D'o\`u $\sqrt{(i+1)(k-i+1)} \le \frac{(k+2)}{2}$

\item\pause Ainsi
$$|c_k|=\sum_{i=0}^k\frac{1}{\sqrt{(i+1)(k-i+1)}} 
\pause\ge \sum_{i=0}^k \frac{2}{k+2} 
\pause= \frac{2(k+1)}{k+2} \to 2$$
\item\pause Donc le terme général $c_k$ ne tend pas vers $0$ \pause : la série $\sum c_k$ diverge
\end{itemize}
\end{exemple}
\end{frame}


%%%%%%%%%%%%%%%%%%%%%%%%%%%%%%%%%%%%%%%%%%%%%%%%%%%%%%%%%%%%%%%%
\section{Mini-exercices}

\begin{frame}
\begin{miniexercice}
\begin{enumerate}
  \item Trouver une expression simple du terme général de la série produit
  $$\sum_{i=0}^{+\infty} \frac{1}{3^i} \times \sum_{j=0}^{+\infty} \frac{1}{3^j}.$$
  Calculer la somme de cette série produit.
  \item On admet ici que pour $x\in \Rr$ la série $\sum_{k=0}^{+\infty} \frac{x^k}{k!}$ converge et vaut
  $\exp(x)$.
  Que vaut la série produit associée à $\exp(a) \times \exp(b)$ ?
  (Vous utiliserez la formule du binôme de Newton.)
\end{enumerate}
\end{miniexercice}
\end{frame}

\end{document}
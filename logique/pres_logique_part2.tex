
%%%%%%%%%%%%%%%%%% PREAMBULE %%%%%%%%%%%%%%%%%%

\documentclass[aspectratio=169,utf8]{beamer}
%\documentclass[aspectratio=169,handout]{beamer}

\usetheme{Boadilla}
%\usecolortheme{seahorse}
%\usecolortheme[RGB={245,66,24}]{structure}
\useoutertheme{infolines}

% packages
\usepackage{amsfonts,amsmath,amssymb,amsthm}
\usepackage[utf8]{inputenc}
\usepackage[T1]{fontenc}
\usepackage{lmodern}

\usepackage[francais]{babel}
\usepackage{fancybox}
\usepackage{graphicx}

\usepackage{float}
\usepackage{xfrac}

%\usepackage[usenames, x11names]{xcolor}
\usepackage{pgfplots}
\usepackage{datetime}


% ----------------------------------------------------------------------
% Pour les images
\usepackage{tikz}
\usetikzlibrary{calc,shadows,arrows.meta,patterns,matrix}

\newcommand{\tikzinput}[1]{\input{figures/#1.tikz}}
% --- les figures avec échelle éventuel
\newcommand{\myfigure}[2]{% entrée : échelle, fichier(s) figure à inclure
\begin{center}\small%
\tikzstyle{every picture}=[scale=1.0*#1]% mise en échelle + 0% (automatiquement annulé à la fin du groupe)
#2%
\end{center}}



%-----  Package unités -----
\usepackage{siunitx}
\sisetup{locale = FR,detect-all,per-mode = symbol}

%\usepackage{mathptmx}
%\usepackage{fouriernc}
%\usepackage{newcent}
%\usepackage[mathcal,mathbf]{euler}

%\usepackage{palatino}
%\usepackage{newcent}
% \usepackage[mathcal,mathbf]{euler}



% \usepackage{hyperref}
% \hypersetup{colorlinks=true, linkcolor=blue, urlcolor=blue,
% pdftitle={Exo7 - Exercices de mathématiques}, pdfauthor={Exo7}}


%section
% \usepackage{sectsty}
% \allsectionsfont{\bf}
%\sectionfont{\color{Tomato3}\upshape\selectfont}
%\subsectionfont{\color{Tomato4}\upshape\selectfont}

%----- Ensembles : entiers, reels, complexes -----
\newcommand{\Nn}{\mathbb{N}} \newcommand{\N}{\mathbb{N}}
\newcommand{\Zz}{\mathbb{Z}} \newcommand{\Z}{\mathbb{Z}}
\newcommand{\Qq}{\mathbb{Q}} \newcommand{\Q}{\mathbb{Q}}
\newcommand{\Rr}{\mathbb{R}} \newcommand{\R}{\mathbb{R}}
\newcommand{\Cc}{\mathbb{C}} 
\newcommand{\Kk}{\mathbb{K}} \newcommand{\K}{\mathbb{K}}

%----- Modifications de symboles -----
\renewcommand{\epsilon}{\varepsilon}
\renewcommand{\Re}{\mathop{\text{Re}}\nolimits}
\renewcommand{\Im}{\mathop{\text{Im}}\nolimits}
%\newcommand{\llbracket}{\left[\kern-0.15em\left[}
%\newcommand{\rrbracket}{\right]\kern-0.15em\right]}

\renewcommand{\ge}{\geqslant}
\renewcommand{\geq}{\geqslant}
\renewcommand{\le}{\leqslant}
\renewcommand{\leq}{\leqslant}
\renewcommand{\epsilon}{\varepsilon}

%----- Fonctions usuelles -----
\newcommand{\ch}{\mathop{\text{ch}}\nolimits}
\newcommand{\sh}{\mathop{\text{sh}}\nolimits}
\renewcommand{\tanh}{\mathop{\text{th}}\nolimits}
\newcommand{\cotan}{\mathop{\text{cotan}}\nolimits}
\newcommand{\Arcsin}{\mathop{\text{arcsin}}\nolimits}
\newcommand{\Arccos}{\mathop{\text{arccos}}\nolimits}
\newcommand{\Arctan}{\mathop{\text{arctan}}\nolimits}
\newcommand{\Argsh}{\mathop{\text{argsh}}\nolimits}
\newcommand{\Argch}{\mathop{\text{argch}}\nolimits}
\newcommand{\Argth}{\mathop{\text{argth}}\nolimits}
\newcommand{\pgcd}{\mathop{\text{pgcd}}\nolimits} 


%----- Commandes divers ------
\newcommand{\ii}{\mathrm{i}}
\newcommand{\dd}{\text{d}}
\newcommand{\id}{\mathop{\text{id}}\nolimits}
\newcommand{\Ker}{\mathop{\text{Ker}}\nolimits}
\newcommand{\Card}{\mathop{\text{Card}}\nolimits}
\newcommand{\Vect}{\mathop{\text{Vect}}\nolimits}
\newcommand{\Mat}{\mathop{\text{Mat}}\nolimits}
\newcommand{\rg}{\mathop{\text{rg}}\nolimits}
\newcommand{\tr}{\mathop{\text{tr}}\nolimits}


%----- Structure des exercices ------

\newtheoremstyle{styleexo}% name
{2ex}% Space above
{3ex}% Space below
{}% Body font
{}% Indent amount 1
{\bfseries} % Theorem head font
{}% Punctuation after theorem head
{\newline}% Space after theorem head 2
{}% Theorem head spec (can be left empty, meaning ‘normal’)

%\theoremstyle{styleexo}
\newtheorem{exo}{Exercice}
\newtheorem{ind}{Indications}
\newtheorem{cor}{Correction}


\newcommand{\exercice}[1]{} \newcommand{\finexercice}{}
%\newcommand{\exercice}[1]{{\tiny\texttt{#1}}\vspace{-2ex}} % pour afficher le numero absolu, l'auteur...
\newcommand{\enonce}{\begin{exo}} \newcommand{\finenonce}{\end{exo}}
\newcommand{\indication}{\begin{ind}} \newcommand{\finindication}{\end{ind}}
\newcommand{\correction}{\begin{cor}} \newcommand{\fincorrection}{\end{cor}}

\newcommand{\noindication}{\stepcounter{ind}}
\newcommand{\nocorrection}{\stepcounter{cor}}

\newcommand{\fiche}[1]{} \newcommand{\finfiche}{}
\newcommand{\titre}[1]{\centerline{\large \bf #1}}
\newcommand{\addcommand}[1]{}
\newcommand{\video}[1]{}

% Marge
\newcommand{\mymargin}[1]{\marginpar{{\small #1}}}

\def\noqed{\renewcommand{\qedsymbol}{}}


%----- Presentation ------
\setlength{\parindent}{0cm}

%\newcommand{\ExoSept}{\href{http://exo7.emath.fr}{\textbf{\textsf{Exo7}}}}

\definecolor{myred}{rgb}{0.93,0.26,0}
\definecolor{myorange}{rgb}{0.97,0.58,0}
\definecolor{myyellow}{rgb}{1,0.86,0}

\newcommand{\LogoExoSept}[1]{  % input : echelle
{\usefont{U}{cmss}{bx}{n}
\begin{tikzpicture}[scale=0.1*#1,transform shape]
  \fill[color=myorange] (0,0)--(4,0)--(4,-4)--(0,-4)--cycle;
  \fill[color=myred] (0,0)--(0,3)--(-3,3)--(-3,0)--cycle;
  \fill[color=myyellow] (4,0)--(7,4)--(3,7)--(0,3)--cycle;
  \node[scale=5] at (3.5,3.5) {Exo7};
\end{tikzpicture}}
}


\newcommand{\debutmontitre}{
  \author{} \date{} 
  \thispagestyle{empty}
  \hspace*{-10ex}
  \begin{minipage}{\textwidth}
    \titlepage  
  \vspace*{-2.5cm}
  \begin{center}
    \LogoExoSept{2.5}
  \end{center}
  \end{minipage}

  \vspace*{-0cm}
  
  % Astuce pour que le background ne soit pas discrétisé lors de la conversion pdf -> png
\begin{tikzpicture}
        \fill[opacity=0,green!60!black] (0,0)--++(0,0)--++(0,0)--++(0,0)--cycle; 
\end{tikzpicture}

% toc S'affiche trop tot :
% \tableofcontents[hideallsubsections, pausesections]
}

\newcommand{\finmontitre}{
  \end{frame}
  \setcounter{framenumber}{0}
} % ne marche pas pour une raison obscure

%----- Commandes supplementaires ------

% \usepackage[landscape]{geometry}
% \geometry{top=1cm, bottom=3cm, left=2cm, right=10cm, marginparsep=1cm
% }
% \usepackage[a4paper]{geometry}
% \geometry{top=2cm, bottom=2cm, left=2cm, right=2cm, marginparsep=1cm
% }

%\usepackage{standalone}


% New command Arnaud -- november 2011
\setbeamersize{text margin left=24ex}
% si vous modifier cette valeur il faut aussi
% modifier le decalage du titre pour compenser
% (ex : ici =+10ex, titre =-5ex

\theoremstyle{definition}
%\newtheorem{proposition}{Proposition}
%\newtheorem{exemple}{Exemple}
%\newtheorem{theoreme}{Théorème}
%\newtheorem{lemme}{Lemme}
%\newtheorem{corollaire}{Corollaire}
%\newtheorem*{remarque*}{Remarque}
%\newtheorem*{miniexercice}{Mini-exercices}
%\newtheorem{definition}{Définition}

% Commande tikz
\usetikzlibrary{calc}
\usetikzlibrary{patterns,arrows}
\usetikzlibrary{matrix}
\usetikzlibrary{fadings} 

%definition d'un terme
\newcommand{\defi}[1]{{\color{myorange}\textbf{\emph{#1}}}}
\newcommand{\evidence}[1]{{\color{blue}\textbf{\emph{#1}}}}
\newcommand{\assertion}[1]{\emph{\og#1\fg}}  % pour chapitre logique
%\renewcommand{\contentsname}{Sommaire}
\renewcommand{\contentsname}{}
\setcounter{tocdepth}{2}



%------ Encadrement ------

\usepackage{fancybox}


\newcommand{\mybox}[1]{
\setlength{\fboxsep}{7pt}
\begin{center}
\shadowbox{#1}
\end{center}}

\newcommand{\myboxinline}[1]{
\setlength{\fboxsep}{5pt}
\raisebox{-10pt}{
\shadowbox{#1}
}
}

%--------------- Commande beamer---------------
\newcommand{\beameronly}[1]{#1} % permet de mettre des pause dans beamer pas dans poly


\setbeamertemplate{navigation symbols}{}
\setbeamertemplate{footline}  % tiré du fichier beamerouterinfolines.sty
{
  \leavevmode%
  \hbox{%
  \begin{beamercolorbox}[wd=.333333\paperwidth,ht=2.25ex,dp=1ex,center]{author in head/foot}%
    % \usebeamerfont{author in head/foot}\insertshortauthor%~~(\insertshortinstitute)
    \usebeamerfont{section in head/foot}{\bf\insertshorttitle}
  \end{beamercolorbox}%
  \begin{beamercolorbox}[wd=.333333\paperwidth,ht=2.25ex,dp=1ex,center]{title in head/foot}%
    \usebeamerfont{section in head/foot}{\bf\insertsectionhead}
  \end{beamercolorbox}%
  \begin{beamercolorbox}[wd=.333333\paperwidth,ht=2.25ex,dp=1ex,right]{date in head/foot}%
    % \usebeamerfont{date in head/foot}\insertshortdate{}\hspace*{2em}
    \insertframenumber{} / \inserttotalframenumber\hspace*{2ex} 
  \end{beamercolorbox}}%
  \vskip0pt%
}


\definecolor{mygrey}{rgb}{0.5,0.5,0.5}
\setlength{\parindent}{0cm}
%\DeclareTextFontCommand{\helvetica}{\fontfamily{phv}\selectfont}

% background beamer
\definecolor{couleurhaut}{rgb}{0.85,0.9,1}  % creme
\definecolor{couleurmilieu}{rgb}{1,1,1}  % vert pale
\definecolor{couleurbas}{rgb}{0.85,0.9,1}  % blanc
\setbeamertemplate{background canvas}[vertical shading]%
[top=couleurhaut,middle=couleurmilieu,midpoint=0.4,bottom=couleurbas] 
%[top=fondtitre!05,bottom=fondtitre!60]



\makeatletter
\setbeamertemplate{theorem begin}
{%
  \begin{\inserttheoremblockenv}
  {%
    \inserttheoremheadfont
    \inserttheoremname
    \inserttheoremnumber
    \ifx\inserttheoremaddition\@empty\else\ (\inserttheoremaddition)\fi%
    \inserttheorempunctuation
  }%
}
\setbeamertemplate{theorem end}{\end{\inserttheoremblockenv}}

\newenvironment{theoreme}[1][]{%
   \setbeamercolor{block title}{fg=structure,bg=structure!40}
   \setbeamercolor{block body}{fg=black,bg=structure!10}
   \begin{block}{{\bf Th\'eor\`eme }#1}
}{%
   \end{block}%
}


\newenvironment{proposition}[1][]{%
   \setbeamercolor{block title}{fg=structure,bg=structure!40}
   \setbeamercolor{block body}{fg=black,bg=structure!10}
   \begin{block}{{\bf Proposition }#1}
}{%
   \end{block}%
}

\newenvironment{corollaire}[1][]{%
   \setbeamercolor{block title}{fg=structure,bg=structure!40}
   \setbeamercolor{block body}{fg=black,bg=structure!10}
   \begin{block}{{\bf Corollaire }#1}
}{%
   \end{block}%
}

\newenvironment{mydefinition}[1][]{%
   \setbeamercolor{block title}{fg=structure,bg=structure!40}
   \setbeamercolor{block body}{fg=black,bg=structure!10}
   \begin{block}{{\bf Définition} #1}
}{%
   \end{block}%
}

\newenvironment{lemme}[0]{%
   \setbeamercolor{block title}{fg=structure,bg=structure!40}
   \setbeamercolor{block body}{fg=black,bg=structure!10}
   \begin{block}{\bf Lemme}
}{%
   \end{block}%
}

\newenvironment{remarque}[1][]{%
   \setbeamercolor{block title}{fg=black,bg=structure!20}
   \setbeamercolor{block body}{fg=black,bg=structure!5}
   \begin{block}{Remarque #1}
}{%
   \end{block}%
}


\newenvironment{exemple}[1][]{%
   \setbeamercolor{block title}{fg=black,bg=structure!20}
   \setbeamercolor{block body}{fg=black,bg=structure!5}
   \begin{block}{{\bf Exemple }#1}
}{%
   \end{block}%
}


\newenvironment{miniexercice}[0]{%
   \setbeamercolor{block title}{fg=structure,bg=structure!20}
   \setbeamercolor{block body}{fg=black,bg=structure!5}
   \begin{block}{Mini-exercices}
}{%
   \end{block}%
}


\newenvironment{tp}[0]{%
   \setbeamercolor{block title}{fg=structure,bg=structure!40}
   \setbeamercolor{block body}{fg=black,bg=structure!10}
   \begin{block}{\bf Travaux pratiques}
}{%
   \end{block}%
}
\newenvironment{exercicecours}[1][]{%
   \setbeamercolor{block title}{fg=structure,bg=structure!40}
   \setbeamercolor{block body}{fg=black,bg=structure!10}
   \begin{block}{{\bf Exercice }#1}
}{%
   \end{block}%
}
\newenvironment{algo}[1][]{%
   \setbeamercolor{block title}{fg=structure,bg=structure!40}
   \setbeamercolor{block body}{fg=black,bg=structure!10}
   \begin{block}{{\bf Algorithme}\hfill{\color{gray}\texttt{#1}}}
}{%
   \end{block}%
}


\setbeamertemplate{proof begin}{
   \setbeamercolor{block title}{fg=black,bg=structure!20}
   \setbeamercolor{block body}{fg=black,bg=structure!5}
   \begin{block}{{\footnotesize Démonstration}}
   \footnotesize
   \smallskip}
\setbeamertemplate{proof end}{%
   \end{block}}
\setbeamertemplate{qed symbol}{\openbox}


\makeatother
\usecolortheme[RGB={151,53,151}]{structure}

%%%%%%%%%%%%%%%%%%%%%%%%%%%%%%%%%%%%%%%%%%%%%%%%%%%%%%%%%%%%%
%%%%%%%%%%%%%%%%%%%%%%%%%%%%%%%%%%%%%%%%%%%%%%%%%%%%%%%%%%%%%

\title{{\bf Logique et raisonnements}}
\subtitle{Raisonnements}
\author{}

\date{}

\begin{document}

\begin{frame}
  
  \debutmontitre

  \pause

{\footnotesize
\hfill
\setbeamercovered{transparent=50}
\begin{minipage}{0.6\textwidth}
  \begin{itemize}
    \item<3-> Raisonnement direct
    \item<4-> Cas par cas
    \item<5-> Contraposition
    \item<6-> Absurde
    \item<7-> Contre-exemple
    \item<8-> Récurrence
  \end{itemize}
\end{minipage}
}

\end{frame}

\setcounter{framenumber}{0}





%---------------------------------------------------------------
\section{Raisonnement direct}

\begin{frame}


\defi{Raisonnement direct}

Pour montrer que l'assertion \assertion{$P \implies Q$} est vraie :

on suppose que $P$ est vraie et on montre qu'alors $Q$ est vraie


\pause
\medskip


\begin{exemple}
Montrer que si $a,b \in \Qq$ alors $a+b\in \Qq$
\end{exemple}

\pause

\begin{proof}
Soient $a\in \Qq$, $b\in \Qq$

\pause

\uncover<5->{
\qquad $a = \frac{p}{q}$ \ \ (avec $p \in \Zz$, $q\in \Nn^*$) \\
\qquad $b = \frac{p'}{q'}$ \ \ (avec $p' \in \Zz$ et $q'\in \Nn^*$)\\
$$a+b= \frac{p}{q} + \frac{p'}{q'} = \frac{pq'+qp'}{qq'}$$
\qquad or $pq'+qp' \in \Zz$, $qq' \in \Nn^*$
}

\uncover<4->{donc $a+b \in \Qq$}
\end{proof}

\end{frame}


%---------------------------------------------------------------
\section{Cas par cas}

\begin{frame}

\defi{Cas par cas}

Pour vérifier une assertion $P(x)$ pour tous les $x$ dans $E$ :

\qquad on montre l'assertion pour les $x$ dans une partie $A$ de $E$ 

\qquad puis pour les $x$ n'appartenant pas à $A$

\pause

\begin{exemple}
Montrer que pour tout $x\in \Rr$, \ \ $|x-1| \le x^2-x+1$
\end{exemple}

\pause
\medskip

\begin{proof}
Soit $x \in \Rr$. Distinguons deux cas

\pause

\uncover<4->{\textbf{Premier cas : $x \ge 1$.}}
\uncover<6->{ \quad Alors $|x-1|=x-1$
$$\begin{array}{rcl}
 x^2-x+1 - |x-1| 
   &=& x^2-x+1 - (x-1) \\
   &=& x^2 -2x + 2 \\
   &=& (x-1)^2 + 1 \ge 0 \\
\end{array}$$
}



\uncover<5->{\textbf{Deuxième cas : $x < 1$.}}
\uncover<7->{ \quad Alors $|x-1| = -(x-1)$

Nous obtenons \  $x^2-x+1 - |x-1| = x^2-x+1 + (x-1) = x^2 \ge 0$
}

\pause
\pause
\pause
\pause

\textbf{Conclusion.} \quad Dans tous les cas $x^2-x+1 - |x-1| \ge 0$
\end{proof}

  
\end{frame}


%---------------------------------------------------------------
\section{Contraposition}

\begin{frame}

\defi{Contraposition} 

Basée sur l'équivalence entre les assertions
\mybox{
\assertion{$P \implies Q$} \quad  équivaut à \quad  \assertion{$\text{non}(Q) \implies \text{non}(P)$}
}

\pause

Pour montrer l'assertion \assertion{$P \implies Q$}, 
on suppose que $\text{non}(Q)$ est vraie et on montre qu'alors $\text{non}(P)$ est vraie

\pause

\begin{exemple}
Soit $n \in \Nn$. Montrer que si $n^2$ est pair alors $n$ est pair
\end{exemple}

\pause

\begin{proof}
Supposons que $n$ n'est pas pair

\medskip

\pause

\uncover<6->{
 \qquad Alors $n$ est impair : il existe $k\in \Nn$ tel que $n=2k+1$.

 \qquad Ainsi $n^2 = (2k+1)^2 = 4k^2 + 4k + 1$ est impair
}

\uncover<5->{
\medskip 
Donc $n^2$ n'est pas pair
}

\pause\pause
Par contraposition ceci est équivalent à : si $n^2$ est pair alors $n$ est pair
\end{proof}

\end{frame}


%---------------------------------------------------------------
\section{Absurde}

\begin{frame}

\defi{Raisonnement par l'absurde} 

Pour montrer \assertion{$P \implies Q$} on suppose à la fois que $P$ est vraie et que $Q$ est fausse 
et on cherche une contradiction

\pause

\begin{exemple}
Soient $a,b \ge 0$. Montrer que si $\frac{a}{1+b}=\frac{b}{1+a}$ alors $a=b$
\end{exemple}

\pause

\begin{proof}
Raisonnons par l'absurde en supposant $\frac{a}{1+b}=\frac{b}{1+a}$ \textbf{et} $a \neq b$
\bigskip 

\uncover<5->{
\qquad Alors $a(1+a)=b(1+b)$, donc $a+a^2=b+b^2$ 

\qquad d'où $a^2-b^2 = b-a$ et donc $(a-b)(a+b)= -(a-b)$

\qquad Comme $a\neq b$ en divisant par $a-b$ on obtient $a+b = -1$
}

\uncover<4->{
\bigskip 
Nous obtenons une contradiction
} 
\uncover<5->{car $a,b \ge 0$}

\pause\pause\pause

{
Conclusion : si $\frac{a}{1+b}=\frac{b}{1+a}$ alors $a=b$
}
\end{proof}
  
\end{frame}


%---------------------------------------------------------------
\section{Contre-exemple}

\begin{frame}

\defi{Contre-exemple}

Montrer que \assertion{$\forall x \in E \quad P(x)$} est fausse

équivaut à montrer \assertion{$\exists x \in E \quad \text{non } P(x)$} 
est vraie

\pause
\bigskip

\begin{exemple}
Montrer que cette assertion est fausse :

\qquad \assertion{Tout entier positif est somme de trois carrés}
\end{exemple}

\pause

Les carrés sont les $0^2$, $1^2$, $2^2$, $3^2$,... 

Par exemple $6 = 2^2+1^2+1^2$

\pause
\bigskip

\begin{proof}
Un contre-exemple est $7$ \pause : les carrés inférieurs à $7$ sont $0$, $1$, $4$
mais avec trois de ces nombres on ne peut faire $7$
\end{proof}  

\end{frame}


%---------------------------------------------------------------
\section{Récurrence}

\begin{frame}
  
\defi{Récurrence} 

Pour montrer qu'une assertion $P(n)$ est vraie pour tout $n \in \Nn$

\pause
\bigskip


\begin{itemize}
  \item \evidence{initialisation}: on prouve $P(0)$

\pause

  \item \evidence{hérédité}
  \begin{itemize}
    \item on fixe $n\ge 0$
    \item on suppose que $P(n)$ est vraie
    \item on démontre que $P(n+1)$ est vraie
   \end{itemize}

\pause

  \item \evidence{conclusion} : par le principe de récurrence,

$P(n)$ est vraie pour tout $n\in\Nn$

\end{itemize}

\end{frame}


%---------------------------------------------------------------

\begin{frame}
  

\begin{exemple}
Montrer que pour tout $n\in \Nn$, $2^n > n$
\end{exemple}

\pause

\begin{proof}
Pour $n\ge 0$, notons $P(n)$ l'assertion \ \assertion{$2^n > n$}

\bigskip

\textbf{Initialisation.} Pour $n=0$, $2^0=1>0$, donc $P(0)$ est vraie

\bigskip

\uncover<3->{
\textbf{Hérédité.}
Fixons $n\ge 0$. Supposons $P(n)$ vraie
}


\uncover<4->{
\quad 
$$\begin{array}{rcl}
 2^{n+1} &=& 2^n + 2^n \\
         &>& n + 2^n \qquad \text{ car par } P(n) : 2^n > n \\
         &>& n + 1 \qquad \text{ car } 2^n \ge 1 \\
\end{array}$$
}

\uncover<3->{donc $P(n+1)$ est vraie}


\bigskip

\uncover<5->{
\textbf{Conclusion.} Par le principe de récurrence 
$P(n)$ est vraie pour tout $n\ge 0$
}

\end{proof}

\end{frame}



%---------------------------------------------------------------
\section{Mini-exercices}

\begin{frame}
\begin{miniexercice}
\begin{enumerate}
  \item (Raisonnement direct) Soient $a,b \in \Rr_+$. Montrer que si $a \le b$ alors $a \le \frac{a+b}{2} \le b$ et
$a \le \sqrt{ab} \le b$.
  \item (Cas par cas) Montrer que pour tout $n\in \Nn$, $n(n+1)$ est divisible par $2$ (distinguer
les $n$ pairs des $n$ impairs).
  \item (Contraposée ou absurde) Soient $a,b \in \Zz$. Montrer que si $b\neq 0$ alors
$a+b\sqrt{2} \notin \Qq$. (On utilisera que $\sqrt 2 \notin \Qq$.)
  \item (Absurde) Soit $n \in\Nn^*$. Montrer que $\sqrt{n^2+1}$ n'est pas un entier.
  \item (Contre-exemple) Est-ce que pour tout $x\in \Rr$ on a $x<2 \implies x^2<4$ ?
  \item (Récurrence) Montrer que pour tout $n \ge 1$, $1+2+ \cdots +n = \frac{n(n+1)}{2}$.
  \item (Récurrence) Fixons un réel $x\ge 0$. Montrer que pour tout entier $n \ge 1$, \ \  $(1+x)^n \ge 1+nx$.
\end{enumerate}  
\end{miniexercice}



\end{frame}

\end{document}
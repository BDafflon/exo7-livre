
%%%%%%%%%%%%%%%%%% PREAMBULE %%%%%%%%%%%%%%%%%%


\documentclass[12pt]{article}

\usepackage{amsfonts,amsmath,amssymb,amsthm}
\usepackage[utf8]{inputenc}
\usepackage[T1]{fontenc}
\usepackage[francais]{babel}


% packages
\usepackage{amsfonts,amsmath,amssymb,amsthm}
\usepackage[utf8]{inputenc}
\usepackage[T1]{fontenc}
%\usepackage{lmodern}

\usepackage[francais]{babel}
\usepackage{fancybox}
\usepackage{graphicx}

\usepackage{float}

%\usepackage[usenames, x11names]{xcolor}
\usepackage{tikz}
\usepackage{datetime}

\usepackage{mathptmx}
%\usepackage{fouriernc}
%\usepackage{newcent}
\usepackage[mathcal,mathbf]{euler}

%\usepackage{palatino}
%\usepackage{newcent}


% Commande spéciale prompteur

%\usepackage{mathptmx}
%\usepackage[mathcal,mathbf]{euler}
%\usepackage{mathpple,multido}

\usepackage[a4paper]{geometry}
\geometry{top=2cm, bottom=2cm, left=1cm, right=1cm, marginparsep=1cm}

\newcommand{\change}{{\color{red}\rule{\textwidth}{1mm}\\}}

\newcounter{mydiapo}

\newcommand{\diapo}{\newpage
\hfill {\normalsize  Diapo \themydiapo \quad \texttt{[\jobname]}} \\
\stepcounter{mydiapo}}


%%%%%%% COULEURS %%%%%%%%%%

% Pour blanc sur noir :
%\pagecolor[rgb]{0.5,0.5,0.5}
% \pagecolor[rgb]{0,0,0}
% \color[rgb]{1,1,1}



%\DeclareFixedFont{\myfont}{U}{cmss}{bx}{n}{18pt}
\newcommand{\debuttexte}{
%%%%%%%%%%%%% FONTES %%%%%%%%%%%%%
\renewcommand{\baselinestretch}{1.5}
\usefont{U}{cmss}{bx}{n}
\bfseries

% Taille normale : commenter le reste !
%Taille Arnaud
%\fontsize{19}{19}\selectfont

% Taille Barbara
%\fontsize{21}{22}\selectfont

%Taille François
%\fontsize{25}{30}\selectfont

%Taille Pascal
%\fontsize{25}{30}\selectfont

%Taille Laura
%\fontsize{30}{35}\selectfont


%\myfont
%\usefont{U}{cmss}{bx}{n}

%\Huge
%\addtolength{\parskip}{\baselineskip}
}


% \usepackage{hyperref}
% \hypersetup{colorlinks=true, linkcolor=blue, urlcolor=blue,
% pdftitle={Exo7 - Exercices de mathématiques}, pdfauthor={Exo7}}


%section
% \usepackage{sectsty}
% \allsectionsfont{\bf}
%\sectionfont{\color{Tomato3}\upshape\selectfont}
%\subsectionfont{\color{Tomato4}\upshape\selectfont}

%----- Ensembles : entiers, reels, complexes -----
\newcommand{\Nn}{\mathbb{N}} \newcommand{\N}{\mathbb{N}}
\newcommand{\Zz}{\mathbb{Z}} \newcommand{\Z}{\mathbb{Z}}
\newcommand{\Qq}{\mathbb{Q}} \newcommand{\Q}{\mathbb{Q}}
\newcommand{\Rr}{\mathbb{R}} \newcommand{\R}{\mathbb{R}}
\newcommand{\Cc}{\mathbb{C}} 
\newcommand{\Kk}{\mathbb{K}} \newcommand{\K}{\mathbb{K}}

%----- Modifications de symboles -----
\renewcommand{\epsilon}{\varepsilon}
\renewcommand{\Re}{\mathop{\text{Re}}\nolimits}
\renewcommand{\Im}{\mathop{\text{Im}}\nolimits}
%\newcommand{\llbracket}{\left[\kern-0.15em\left[}
%\newcommand{\rrbracket}{\right]\kern-0.15em\right]}

\renewcommand{\ge}{\geqslant}
\renewcommand{\geq}{\geqslant}
\renewcommand{\le}{\leqslant}
\renewcommand{\leq}{\leqslant}

%----- Fonctions usuelles -----
\newcommand{\ch}{\mathop{\mathrm{ch}}\nolimits}
\newcommand{\sh}{\mathop{\mathrm{sh}}\nolimits}
\renewcommand{\tanh}{\mathop{\mathrm{th}}\nolimits}
\newcommand{\cotan}{\mathop{\mathrm{cotan}}\nolimits}
\newcommand{\Arcsin}{\mathop{\mathrm{Arcsin}}\nolimits}
\newcommand{\Arccos}{\mathop{\mathrm{Arccos}}\nolimits}
\newcommand{\Arctan}{\mathop{\mathrm{Arctan}}\nolimits}
\newcommand{\Argsh}{\mathop{\mathrm{Argsh}}\nolimits}
\newcommand{\Argch}{\mathop{\mathrm{Argch}}\nolimits}
\newcommand{\Argth}{\mathop{\mathrm{Argth}}\nolimits}
\newcommand{\pgcd}{\mathop{\mathrm{pgcd}}\nolimits} 

\newcommand{\Card}{\mathop{\text{Card}}\nolimits}
\newcommand{\Ker}{\mathop{\text{Ker}}\nolimits}
\newcommand{\id}{\mathop{\text{id}}\nolimits}
\newcommand{\ii}{\mathrm{i}}
\newcommand{\dd}{\mathrm{d}}
\newcommand{\Vect}{\mathop{\text{Vect}}\nolimits}
\newcommand{\Mat}{\mathop{\mathrm{Mat}}\nolimits}
\newcommand{\rg}{\mathop{\text{rg}}\nolimits}
\newcommand{\tr}{\mathop{\text{tr}}\nolimits}
\newcommand{\ppcm}{\mathop{\text{ppcm}}\nolimits}

%----- Structure des exercices ------

\newtheoremstyle{styleexo}% name
{2ex}% Space above
{3ex}% Space below
{}% Body font
{}% Indent amount 1
{\bfseries} % Theorem head font
{}% Punctuation after theorem head
{\newline}% Space after theorem head 2
{}% Theorem head spec (can be left empty, meaning ‘normal’)

%\theoremstyle{styleexo}
\newtheorem{exo}{Exercice}
\newtheorem{ind}{Indications}
\newtheorem{cor}{Correction}


\newcommand{\exercice}[1]{} \newcommand{\finexercice}{}
%\newcommand{\exercice}[1]{{\tiny\texttt{#1}}\vspace{-2ex}} % pour afficher le numero absolu, l'auteur...
\newcommand{\enonce}{\begin{exo}} \newcommand{\finenonce}{\end{exo}}
\newcommand{\indication}{\begin{ind}} \newcommand{\finindication}{\end{ind}}
\newcommand{\correction}{\begin{cor}} \newcommand{\fincorrection}{\end{cor}}

\newcommand{\noindication}{\stepcounter{ind}}
\newcommand{\nocorrection}{\stepcounter{cor}}

\newcommand{\fiche}[1]{} \newcommand{\finfiche}{}
\newcommand{\titre}[1]{\centerline{\large \bf #1}}
\newcommand{\addcommand}[1]{}
\newcommand{\video}[1]{}

% Marge
\newcommand{\mymargin}[1]{\marginpar{{\small #1}}}



%----- Presentation ------
\setlength{\parindent}{0cm}

%\newcommand{\ExoSept}{\href{http://exo7.emath.fr}{\textbf{\textsf{Exo7}}}}

\definecolor{myred}{rgb}{0.93,0.26,0}
\definecolor{myorange}{rgb}{0.97,0.58,0}
\definecolor{myyellow}{rgb}{1,0.86,0}

\newcommand{\LogoExoSept}[1]{  % input : echelle
{\usefont{U}{cmss}{bx}{n}
\begin{tikzpicture}[scale=0.1*#1,transform shape]
  \fill[color=myorange] (0,0)--(4,0)--(4,-4)--(0,-4)--cycle;
  \fill[color=myred] (0,0)--(0,3)--(-3,3)--(-3,0)--cycle;
  \fill[color=myyellow] (4,0)--(7,4)--(3,7)--(0,3)--cycle;
  \node[scale=5] at (3.5,3.5) {Exo7};
\end{tikzpicture}}
}



\theoremstyle{definition}
%\newtheorem{proposition}{Proposition}
%\newtheorem{exemple}{Exemple}
%\newtheorem{theoreme}{Théorème}
\newtheorem{lemme}{Lemme}
\newtheorem{corollaire}{Corollaire}
%\newtheorem*{remarque*}{Remarque}
%\newtheorem*{miniexercice}{Mini-exercices}
%\newtheorem{definition}{Définition}




%definition d'un terme
\newcommand{\defi}[1]{{\color{myorange}\textbf{\emph{#1}}}}
\newcommand{\evidence}[1]{{\color{blue}\textbf{\emph{#1}}}}



 %----- Commandes divers ------

\newcommand{\codeinline}[1]{\texttt{#1}}

%%%%%%%%%%%%%%%%%%%%%%%%%%%%%%%%%%%%%%%%%%%%%%%%%%%%%%%%%%%%%
%%%%%%%%%%%%%%%%%%%%%%%%%%%%%%%%%%%%%%%%%%%%%%%%%%%%%%%%%%%%%


\begin{document}

\debuttexte


%%%%%%%%%%%%%%%%%%%%%%%%%%%%%%%%%%%%%%%%%%%%%%%%%%%%%%%%%%%
\diapo

Je vais maintenant vous présenter le plan d'étude 
d'une courbe paramétrée.


\change

Voici les points que nous allons aborder.

\change

Nous allons d'abord discuter des différentes étapes du plan d'étude, 

\change

puis nous l'illustrerons sur l'exemple de la courbe de Lissajous.

\change

Avant de faire une étude complète d'une nouvelle courbe.


%%%%%%%%%%%%%%%%%%%%%%%%%%%%%%%%%%%%%%%%%%%%%%%%%%%%%%%%%%%
\diapo

Voici un plan d'étude d'une courbe paramétrée.

\begin{enumerate}
\item On commence naturellement par déterminer 
le \evidence{domaine de définition de la courbe.}

Le point $M(t)$ est défini si et seulement 
si ses deux coordonnées $x(t)$ et $y(t)$ sont définies.

%\change

Il faut ensuite déterminer un \evidence{domaine d'étude} 
(plus petit que le domaine de définition) 
grâce aux symétries, périodicités\ldots 

\change

\item On calcule ensuite le \evidence{vecteur dérivé.} 

Ceci revient à calculer les dérivées 
$x'(t)$ et $y'(t)$ des coordonnées du point $M(t)$.

Les valeurs de $t$ pour lesquelles $x'(t)=0$ 
fournissent les points à tangente verticale et les valeurs de 
$t$ pour lesquelles $y'(t)=0$ fournissent les 
points à tangente horizontale. Enfin, les valeurs de $t$ pour 
lesquelles $x'(t)=y'(t)=0$ fournissent les points singuliers, 
en lesquels on n'a encore aucun renseignement sur la tangente.

\change

\item On dresse ensuite le 
\evidence{tableau de variations conjointes}, 

\change
qui a l'allure suivante. 

L'étude des signes de $x'$ et $y'$ permet de 
connaître les variations de $x$ et $y$.

\change

\item  On étudie ensuite les \evidence{points singuliers} 

\change

\item puis on en fait de même avec les \evidence{branches infinies.}

\change

\item On passe enfin à la \evidence{construction de la courbe.}

Après avoir placé les éléments remarquables, 
comme les tangentes horizontales ou verticales, 
les points singuliers, les asymptotes, 
on s'aide du tableau des variations conjointes 
pour construire la courbe.

\change
Par exemple si $x$ croît et $y$ croît,
on va vers la droite et vers le haut.


\change
Si $x$ croît et $y$ décroît, on va vers la droite et vers le bas.
Etc.

\change

\item On termine par la détermination éventuelle 
des \evidence{points multiples}.

On attend souvent la construction de la courbe pour 
voir s'il y a des points multiples.

\end{enumerate}




%%%%%%%%%%%%%%%%%%%%%%%%%%%%%%%%%%%%%%%%%%%%%%%%%%%%%%%%%%%
\diapo

Voyons à présent comment appliquer le plan d'étude 
à cette courbe de Lissajous, que l'on a déjà partiellement étudiée,
et dont l'équation 
paramétrique est donnée par 

$$\left\{
\begin{array}{l}
x=\sin(2t)\\
y=\sin(3t)
\end{array}
\right.$$

\change
  Pour tout réel $t$, $M(t)$ existe 
  
\change
  et $M(t+2\pi)=M(t)$. 
On obtient la courbe complète quand $t$ décrit $[-\pi,\pi]$.

\change
 $M(-t)=s_O(M(t))$, donc on limite l'étude à $[0,\pi]$, 
 
\change
puis $M(\pi-t)=s_{(Oy)}(M(t))$. On étudie et on construit l'arc
quand $t$ parcourt $[0,\frac{\pi}{2}]$
, puis on obtient la courbe 
complète par réflexion d'axe $(Oy)$ 
puis par symétrie centrale de centre
$O$. 

\change
On détermine facilement une région qui contient la courbe :

\change
pour tout réel $t$, $|x(t)|\leq1$ et $|y(t)|\leq1$. 
Le support de la courbe est donc contenu dans le carré 

$$|x|\leq1 \text{ et }|y|\leq1$$

En particulier il n'y a pas de branches infinies.


\change
On calcule tout aussi facilement le vecteur dérivé :

\change
$$\overrightarrow{\frac{\dd M}{\dd t}}(t)
=\big(2\cos(2t),3\cos(3t)\big)$$
ce qui déterminera la direction des tangentes.



\change
La tangente est verticale lorsque
$x'(t)=0$ 

c'est-à-dire 
$\cos(2t)=0$

ce qui sur notre intervalle d'étude $[0,\frac{\pi}{2}]$

a pour solution $t=\tfrac{\pi}{4}$.

\change
La tangente est horizontale lorsque
$y'(t)=0$ 

c'est-à-dire 
$\cos(3t)=0$

qui a pour solutions $t = \tfrac{\pi}{6}$ et $\tfrac{\pi}{2}$

On note que les deux composantes du vecteur dérivé 
ne s'annulent pas simultanément : tous les points sont des points 
réguliers.


\change

Grâce au vecteur dérivé, on calcule immédiatement 
le vecteur directeur de la tangente en $M(0)$ : c'est $(2,3)$

\change
et par exemple en $M(\tfrac{\pi}{3})$  c'est $(-1,-3)$.



%%%%%%%%%%%%%%%%%%%%%%%%%%%%%%%%%%%%%%%%%%%%%%%%%%%%%%%%%%%
\diapo

La connaissance des variations de la fonction $\sin$ 
permet de connaitre les variations des coordonnées de $M(t)$ 
sur le domaine d'étude $[0,\frac{\pi}{2}]$.

\change
la fonction $x(t)$ est croissante sur $[0,\frac{\pi}{4}]$ 
et décroissante 
sur $[\frac{\pi}{4},\frac{\pi}{2}]$ ; 


\change
et de même, 
la fonction $y(t)$ croît sur $[0,\frac{\pi}{6}]$ et décroît 
sur $[\frac{\pi}{6},\frac{\pi}{2}]$.


\change
On peut maintenant tracer le graphes de la courbe de Lissajou.

On commence par placer quelques points remarquables :
pour les valeurs $t=0$, $t=\pi/6$, $t=\pi/4$,
$t=\pi/3$, $t = \pi/2$.

On sait que la courbe va rester à l'intérieur de ce carré.

\change
En ces points on connait les tangentes.

\change
Maintenant on trace la courbe, on part en $t=0$ de l'origine,
on sait que jusqu'à $t=\pi/6$, $x$ et $y$ sont croissantes
et on arrive à une tangente horizontale.

Puis $x$ continue de croître et $y$ décroît.

Puis $x$ et $y$ décroissent à partir de $t=\pi/4$.

\change
Maintenant on effectue la symétrie par rapport à l'axe des ordonnées pour avoir la courbe 
sur $[0,\pi]$. 

\change
Puis la symétrie par rapport à l'origine pour avoir 
toute la courbe.

%%%%%%%%%%%%%%%%%%%%%%%%%%%%%%%%%%%%%%%%%%%%%%%%%%%%%%%%%%%
\diapo

Nous allons recommencer avec l'étude de la courbe 
donnée par :

$$\left\{
\begin{array}{l}
x(t)=\dfrac{t^3}{t^2-1}\\[3mm]
y(t)=\dfrac{t(3t-2)}{3(t-1)}
\end{array}
\right..$$

\change

 Commençons par le \textbf{domaine d'étude}.  
  
\change

Le point $M(t)$ n'est pas défini en $t = \pm1$. 
Aucune réduction intéressante
du domaine n'apparaît clairement et on étudie la courbe
sur $D=\Rr\setminus\{-1,1\}$.



\change

Pour les \textbf{variations conjointes des coordonnées}. 
on calcule les dérivées :

\change
$$x'(t)=\frac{t^2(t^2-3)}{(t^2-1)^2}$$ 

et 

$$y'(t)=\frac{3t^2-6t+2}{3(t-1)^2}\; .$$

Le signe de ces expressions nous permettra de déterminer
les variations de $x(t)$ et $y(t)$.

\change

Les fonctions $x'$ et $y'$ ne s'annulent jamais en même temps,
la courbe est donc régulière. Ainsi la tangente est dirigée par le vecteur dérivé.

\change

La tangente est horizontale en $1-\frac{1}{\sqrt{3}}$ et $1+\frac{1}{\sqrt{3}}$.

\change

La tangente est verticale en $0$, $\sqrt{3}$ et $-\sqrt{3}$.

\change  

On peut aussi placer quelques points spécifiques.

\change 
Par exemple la courbe coupe l'axe des ordonnées lorsque 
$x(t)=0$, dont la seule solution est à $t=0$. 

\change
Ainsi seul le point $M(0)=(0,0)$ est sur l'axe des ordonnées et sur la courbe.

\change

La courbe coupe l'axe des abscisses en $t=0$ et $t=\frac{2}{3}$.


\change
C'est à dire à l'origine et au point $M(\frac{2}{3})=(-\frac{8}{15},0)$.





%%%%%%%%%%%%%%%%%%%%%%%%%%%%%%%%%%%%%%%%%%%%%%%%%%%%%%%%%%%
\diapo

Il y a pas mal de travail pour dresser le 
\textbf{tableau de variations conjointes}, 
qui a l'allure suivante.



Avec les signes des dérivées de $x$, on a les variations de $x$.

Avec les signes des dérivées de $y$, on a les variations de $y$.

On en déduit les variations de la courbes.


Par exemple de $-\infty$ à $-\sqrt3$, $x$ et $y$ sont croissantes donc la courbes
va vers le haut à droite.
  

On note qu'il y va y avoir des branches qui partent à l'infini
et $-1$ et $+1$, car l'une des coordonnées tend vers $\pm \infty$
et aussi en quant $t$ tend vers $+$ ou $-\infty$.

Prenez le temps d'étudier ce tableau en détails.


%%%%%%%%%%%%%%%%%%%%%%%%%%%%%%%%%%%%%%%%%%%%%%%%%%%%%%%%%%%
\diapo


On passe à l'étude des asymptotes.

\change

Quand $t$ tend vers $+\infty$, $x(t)$ et $y(t)$ tendent 
toutes deux vers $+\infty$ et il y a donc une branche infinie.
Même chose quand $t$ tend vers $-\infty$.

\change

\'Etudions la limite du rapport $\frac{y(t)}{x(t)}$.

Un petit calcul donne :
$$\frac{y(t)}{x(t)} =\frac{(3t-2)(t+1)}{3t^2}.$$

Cette expression tend vers $a=1$ quand $t$ tend vers $+\infty$ ou
vers $-\infty$.

\change  
Pour savoir s'il y a une asymptote il faut calculer
  $$y(t)-a\times x(t) = y(t)-x(t)=\frac{t^2-2t}{3(t-1)(t+1)}$$
  
  qui tend $b=\tfrac{1}{3}$
  
\change
Ainsi la droite d'équation $y=ax+b$, cad $y=x+\frac{1}{3}$ est 
 asymptote à la courbe en $+\infty$ et $-\infty$.


\change 

Pour déterminer la position relative, on calculerait encore %que 

$$y(t)-\big(x(t)+\frac{1}{3}\big)$$

%$$y(t)-\big(x(t)+\frac{1}{3}\big)=\frac{-2t+1}{3(t-1)(t+1)}$$

Le signe de cette expression nous permet de dire si la courbe
est au-dessus ou en-dessous de l'asymptote.

\change  

Il faut faire le même travail en $t=-1$.

Plus précisément en $-1$ à droite et en $-1$ à gauche.

\change

On trouverait que la droite horizontale d'équation $y=-\frac{5}{6}$ 
est asymptote.
  
  
\change  

Et pareil en $t=+1$

\change
où la droite $y=\frac{2}{3}x+\frac{1}{2}$ est asymptote à la courbe.



%%%%%%%%%%%%%%%%%%%%%%%%%%%%%%%%%%%%%%%%%%%%%%%%%%%%%%%%%%%
\diapo

On peut enfin passer au \textbf{tracé de la courbe.}

On commencerait naturellement par placer 
quelques points particuliers et leur tangente. 
Puis on dessine les asymptotes,
On s'aide du tableau de variations conjointes
pour tracer la courbe :

une première branche de $-\infty$ à $-1$

une seconde de $-1$ à $+1$

et la dernière de $+1$ à $+\infty$.


% Le tracé fait apparaître un \textbf{point double}.





%%%%%%%%%%%%%%%%%%%%%%%%%%%%%%%%%%%%%%%%%%%%%%%%%%%%%%%%%%%
\diapo

Avec ces exercices, vous pourrez mettre en 
œuvre le plan d'étude des courbes paramétrées. 




\end{document}

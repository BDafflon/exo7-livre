
%%%%%%%%%%%%%%%%%% PREAMBULE %%%%%%%%%%%%%%%%%%


\documentclass[12pt]{article}

\usepackage{amsfonts,amsmath,amssymb,amsthm}
\usepackage[utf8]{inputenc}
\usepackage[T1]{fontenc}
\usepackage[francais]{babel}


% packages
\usepackage{amsfonts,amsmath,amssymb,amsthm}
\usepackage[utf8]{inputenc}
\usepackage[T1]{fontenc}
%\usepackage{lmodern}

\usepackage[francais]{babel}
\usepackage{fancybox}
\usepackage{graphicx}

\usepackage{float}

%\usepackage[usenames, x11names]{xcolor}
\usepackage{tikz}
\usepackage{datetime}

\usepackage{mathptmx}
%\usepackage{fouriernc}
%\usepackage{newcent}
\usepackage[mathcal,mathbf]{euler}

%\usepackage{palatino}
%\usepackage{newcent}


% Commande spéciale prompteur

%\usepackage{mathptmx}
%\usepackage[mathcal,mathbf]{euler}
%\usepackage{mathpple,multido}

\usepackage[a4paper]{geometry}
\geometry{top=2cm, bottom=2cm, left=1cm, right=1cm, marginparsep=1cm}

\newcommand{\change}{{\color{red}\rule{\textwidth}{1mm}\\}}

\newcounter{mydiapo}

\newcommand{\diapo}{\newpage
\hfill {\normalsize  Diapo \themydiapo \quad \texttt{[\jobname]}} \\
\stepcounter{mydiapo}}


%%%%%%% COULEURS %%%%%%%%%%

% Pour blanc sur noir :
%\pagecolor[rgb]{0.5,0.5,0.5}
% \pagecolor[rgb]{0,0,0}
% \color[rgb]{1,1,1}



%\DeclareFixedFont{\myfont}{U}{cmss}{bx}{n}{18pt}
\newcommand{\debuttexte}{
%%%%%%%%%%%%% FONTES %%%%%%%%%%%%%
\renewcommand{\baselinestretch}{1.5}
\usefont{U}{cmss}{bx}{n}
\bfseries

% Taille normale : commenter le reste !
%Taille Arnaud
%\fontsize{19}{19}\selectfont

% Taille Barbara
%\fontsize{21}{22}\selectfont

%Taille François
%\fontsize{25}{30}\selectfont

%Taille Pascal
%\fontsize{25}{30}\selectfont

%Taille Laura
%\fontsize{30}{35}\selectfont


%\myfont
%\usefont{U}{cmss}{bx}{n}

%\Huge
%\addtolength{\parskip}{\baselineskip}
}


% \usepackage{hyperref}
% \hypersetup{colorlinks=true, linkcolor=blue, urlcolor=blue,
% pdftitle={Exo7 - Exercices de mathématiques}, pdfauthor={Exo7}}


%section
% \usepackage{sectsty}
% \allsectionsfont{\bf}
%\sectionfont{\color{Tomato3}\upshape\selectfont}
%\subsectionfont{\color{Tomato4}\upshape\selectfont}

%----- Ensembles : entiers, reels, complexes -----
\newcommand{\Nn}{\mathbb{N}} \newcommand{\N}{\mathbb{N}}
\newcommand{\Zz}{\mathbb{Z}} \newcommand{\Z}{\mathbb{Z}}
\newcommand{\Qq}{\mathbb{Q}} \newcommand{\Q}{\mathbb{Q}}
\newcommand{\Rr}{\mathbb{R}} \newcommand{\R}{\mathbb{R}}
\newcommand{\Cc}{\mathbb{C}} 
\newcommand{\Kk}{\mathbb{K}} \newcommand{\K}{\mathbb{K}}

%----- Modifications de symboles -----
\renewcommand{\epsilon}{\varepsilon}
\renewcommand{\Re}{\mathop{\text{Re}}\nolimits}
\renewcommand{\Im}{\mathop{\text{Im}}\nolimits}
%\newcommand{\llbracket}{\left[\kern-0.15em\left[}
%\newcommand{\rrbracket}{\right]\kern-0.15em\right]}

\renewcommand{\ge}{\geqslant}
\renewcommand{\geq}{\geqslant}
\renewcommand{\le}{\leqslant}
\renewcommand{\leq}{\leqslant}

%----- Fonctions usuelles -----
\newcommand{\ch}{\mathop{\mathrm{ch}}\nolimits}
\newcommand{\sh}{\mathop{\mathrm{sh}}\nolimits}
\renewcommand{\tanh}{\mathop{\mathrm{th}}\nolimits}
\newcommand{\cotan}{\mathop{\mathrm{cotan}}\nolimits}
\newcommand{\Arcsin}{\mathop{\mathrm{Arcsin}}\nolimits}
\newcommand{\Arccos}{\mathop{\mathrm{Arccos}}\nolimits}
\newcommand{\Arctan}{\mathop{\mathrm{Arctan}}\nolimits}
\newcommand{\Argsh}{\mathop{\mathrm{Argsh}}\nolimits}
\newcommand{\Argch}{\mathop{\mathrm{Argch}}\nolimits}
\newcommand{\Argth}{\mathop{\mathrm{Argth}}\nolimits}
\newcommand{\pgcd}{\mathop{\mathrm{pgcd}}\nolimits} 

\newcommand{\Card}{\mathop{\text{Card}}\nolimits}
\newcommand{\Ker}{\mathop{\text{Ker}}\nolimits}
\newcommand{\id}{\mathop{\text{id}}\nolimits}
\newcommand{\ii}{\mathrm{i}}
\newcommand{\dd}{\mathrm{d}}
\newcommand{\Vect}{\mathop{\text{Vect}}\nolimits}
\newcommand{\Mat}{\mathop{\mathrm{Mat}}\nolimits}
\newcommand{\rg}{\mathop{\text{rg}}\nolimits}
\newcommand{\tr}{\mathop{\text{tr}}\nolimits}
\newcommand{\ppcm}{\mathop{\text{ppcm}}\nolimits}

%----- Structure des exercices ------

\newtheoremstyle{styleexo}% name
{2ex}% Space above
{3ex}% Space below
{}% Body font
{}% Indent amount 1
{\bfseries} % Theorem head font
{}% Punctuation after theorem head
{\newline}% Space after theorem head 2
{}% Theorem head spec (can be left empty, meaning ‘normal’)

%\theoremstyle{styleexo}
\newtheorem{exo}{Exercice}
\newtheorem{ind}{Indications}
\newtheorem{cor}{Correction}


\newcommand{\exercice}[1]{} \newcommand{\finexercice}{}
%\newcommand{\exercice}[1]{{\tiny\texttt{#1}}\vspace{-2ex}} % pour afficher le numero absolu, l'auteur...
\newcommand{\enonce}{\begin{exo}} \newcommand{\finenonce}{\end{exo}}
\newcommand{\indication}{\begin{ind}} \newcommand{\finindication}{\end{ind}}
\newcommand{\correction}{\begin{cor}} \newcommand{\fincorrection}{\end{cor}}

\newcommand{\noindication}{\stepcounter{ind}}
\newcommand{\nocorrection}{\stepcounter{cor}}

\newcommand{\fiche}[1]{} \newcommand{\finfiche}{}
\newcommand{\titre}[1]{\centerline{\large \bf #1}}
\newcommand{\addcommand}[1]{}
\newcommand{\video}[1]{}

% Marge
\newcommand{\mymargin}[1]{\marginpar{{\small #1}}}



%----- Presentation ------
\setlength{\parindent}{0cm}

%\newcommand{\ExoSept}{\href{http://exo7.emath.fr}{\textbf{\textsf{Exo7}}}}

\definecolor{myred}{rgb}{0.93,0.26,0}
\definecolor{myorange}{rgb}{0.97,0.58,0}
\definecolor{myyellow}{rgb}{1,0.86,0}

\newcommand{\LogoExoSept}[1]{  % input : echelle
{\usefont{U}{cmss}{bx}{n}
\begin{tikzpicture}[scale=0.1*#1,transform shape]
  \fill[color=myorange] (0,0)--(4,0)--(4,-4)--(0,-4)--cycle;
  \fill[color=myred] (0,0)--(0,3)--(-3,3)--(-3,0)--cycle;
  \fill[color=myyellow] (4,0)--(7,4)--(3,7)--(0,3)--cycle;
  \node[scale=5] at (3.5,3.5) {Exo7};
\end{tikzpicture}}
}



\theoremstyle{definition}
%\newtheorem{proposition}{Proposition}
%\newtheorem{exemple}{Exemple}
%\newtheorem{theoreme}{Théorème}
\newtheorem{lemme}{Lemme}
\newtheorem{corollaire}{Corollaire}
%\newtheorem*{remarque*}{Remarque}
%\newtheorem*{miniexercice}{Mini-exercices}
%\newtheorem{definition}{Définition}




%definition d'un terme
\newcommand{\defi}[1]{{\color{myorange}\textbf{\emph{#1}}}}
\newcommand{\evidence}[1]{{\color{blue}\textbf{\emph{#1}}}}



 %----- Commandes divers ------

\newcommand{\codeinline}[1]{\texttt{#1}}

%%%%%%%%%%%%%%%%%%%%%%%%%%%%%%%%%%%%%%%%%%%%%%%%%%%%%%%%%%%%%
%%%%%%%%%%%%%%%%%%%%%%%%%%%%%%%%%%%%%%%%%%%%%%%%%%%%%%%%%%%%%



\begin{document}

\debuttexte


%%%%%%%%%%%%%%%%%%%%%%%%%%%%%%%%%%%%%%%%%%%%%%%%%%%%%%%%%%%
\diapo

Dans cette dernière partie, nous allons voir comment 
les techniques usuelles d'intégration se transposent 
au cas des intégrales impropres.

\change

\change

Nous commencerons par l'intégration par parties,

\change

puis nous passerons au changement de variable,

\change

avant de conclure en proposant un plan d'étude 
type pour une intégrale impropre.



%%%%%%%%%%%%%%%%%%%%%%%%%%%%%%%%%%%%%%%%%%%%%%%%%%%%%%%%%%%%%%%%
\diapo

Le théorème d'intégration par parties pour les intégrales 
impropres est la formule suivante 
$\int u v'= \big[uv\big] - \int u' v$.

\change

L'énoncé complet est le suivant :

 Soient $u$ et $v$ sont deux fonctions de classe 
 $\mathcal{C}^1$ sur l'intervalle $[a,+\infty[$.
 
Il y a une hypothèse supplémentaire :
$\lim_{t\to+\infty} u(t)v(t)$ existe et est finie.

Alors d'une part les intégrales 
$\displaystyle\int_a^{+\infty} u(t) \, v'(t)\;\dd t$
et $\displaystyle\int_a^{+\infty} u'(t) \, v(t)\;\dd t$ 
sont de même nature.

Et le plus important : en cas de convergence la formule usuelle 
est valide, c'est-à-dire  :
$$\int_a^{+\infty} u(t) \, v'(t)\;\dd t 
= \big[uv\big]_a^{+\infty} - \int_a^{+\infty} u'(t) \, v(t)\;\dd t$$

\change

On rappelle que la notation $\big[uv\big]_a^{+\infty}$ signifie 
$\lim_{t\to+\infty} (uv)(t) - (uv)(a)$.


\change

Le théorème se prouve grâce à la formule usuelle d'intégration 
par parties sur l'intervalle fermé $[a,x]$ [montrer] :
% $$\int_a^{x} u(t) \, v'(t)\;\dd t 
% = \big[uv\big]_a^{x} - \int_a^{x} u'(t) \, v(t)\;\dd t$$
faire tendre $x\to +\infty$,
en remarquant que par hypothèse le crochet a une limite finie 
lorsque $x\to +\infty$, 

% donc l'intégrale dans le membre de 
% gauche admet une limite finie si et seulement si 
% celle dans le membre de droite admet une limite finie. 
% Si ces limites existent, le fait que limites et somme 
% commutent donne bien l'égalité du théorème.

Le mieux n'est pas d'appliquer ce théorème, 
mais d'effectuer à chaque fois la preuve,
c'est-à-dire faire une intégration par parties sur l'intervalle $[a,x]$ et
en vérifiant bien que les objets ont une limite lorsque $x\to+ \infty$.


%%%%%%%%%%%%%%%%%%%%%%%%%%%%%%%%%%%%%%%%%%%%%%%%%%%%%%%%%%%%%%%% 
\diapo

Voici un exemple. calculons, pour $\lambda >0$ fixé, 
l'espérance  de la loi exponentielle, qui est 
$\int_0^{+\infty} \lambda t e^{-\lambda t}\;\dd t$.

\change
Pour cela, on effectue l'intégration par 
parties avec $u = \lambda t$, $v' = e^{-\lambda t}$ : 

\change
$\int_0^{x} \lambda t e^{-\lambda t}\;\dd t$
est donc $= \int_0^{x} u(t) \, v'(t)\;\dd t  $


\change
en effectuant l'intégration par parties on trouve 
$\displaystyle \big[uv\big]_0^{x} - \int_0^{x} u'(t) \, v(t)\;\dd t $

[montrer $u$, $v'$]

ici $u(t) = \lambda t$ se dérive en $u'(t) = \lambda$
et $v'(t) = e^{-\lambda t}$ s'intègre en 
$v(t)= \frac{-1}{\lambda}e^{-\lambda t}$

\change
d'où en remplaçant :

$$\Big[ \lambda t \cdot \frac{-1}{\lambda}e^{-\lambda t}\Big]_0^{x} 
- \int_0^{x} \lambda \cdot  \frac{-1}{\lambda}e^{-\lambda t}\;\dd t $$

\change
En évaluant le crochet qui est nul en $0$ et en 
simplifiant la deuxième intégrale, on obtient :

 $$\int_0^{x} \lambda t e^{-\lambda t}\;\dd t  = -xe^{-\lambda x} + \int_0^{x} e^{-\lambda t} \;\dd t$$

\change
Cette dernière intégration est facile car 
comme on l'a vu $e^{-\lambda t}$ s'intègre en 
$\frac{-1}{\lambda}e^{-\lambda t}$, ce qui donne

$$-xe^{-\lambda x} + \Big[\frac{-1}{\lambda}e^{-\lambda t}\Big]_0^{x} $$

\change
En évaluant à nouveau le crochet, on obtient 

$$-xe^{-\lambda x}-\frac{1}{\lambda} \left(e^{-\lambda x} -1\right) $$
 
\change
Comme $\lambda >0$, 
 $xe^{-\lambda x}$ et $e^{-\lambda x}$ 
tendent vers $0$ quand $x \to +\infty $, donc 

$$\int_0^{x}  \longrightarrow \ \  \frac{1}{\lambda} \qquad \text{ lorsque } x \to +\infty $$

\change
Conclusion $\int_0^{+\infty} \lambda t e^{-\lambda t}\;\dd t$
converge et vaut $\frac{1}{\lambda}$.



%%%%%%%%%%%%%%%%%%%%%%%%%%%%%%%%%%%%%%%%%%%%%%%%%%%%%%%%%%%%%%%% 
\diapo

Le théorème de changement de variable pour les intégrales impropres 
est la formule 
$$\int_{a}^{+\infty} f(x) \;\dd x 
= \int_\alpha^\beta f\big(\varphi(t)\big)\cdot\varphi'(t) \;\dd t$$

\change
Voici l'énoncé complet :

On considère une fonction continue $f$ définie sur un intervalle 
$I = [a,+\infty[$. 

Soit $J= [\alpha,\beta[$ un autre intervalle avec 
$\alpha\in \Rr$ et $\beta\in \Rr$ ou $\beta=+\infty$. 

On se donne  un difféomorphisme $\varphi : J \to I$ de classe $\mathcal{C}^1$ entre ces deux intervalles.

Alors l'intégrale de départ 
c'est-à-dire $\int_{a}^{+\infty} f(x) \;\dd x$ 
et celle obtenue après changement de variable 
c'est-à-dire 
$\int_\alpha^\beta f\big(\varphi(t)\big)\cdot\varphi'(t) \;\dd t$ 
sont de même nature.

Et surtout si elles convergent, elles ont la même valeur.
%$$\int_{a}^{+\infty} f(x) \;\dd x = \int_\alpha^\beta f\big(\varphi(t)\big)\cdot\varphi'(t) \;\dd t$$

\change
On rappelle que $\varphi$ un \defi{difféomorphisme de classe $\mathcal{C}^1$}
si $\varphi$ est une application $\mathcal{C}^1$, bijective, dont la bijection réciproque est
aussi $\mathcal{C}^1$.

La démonstration est la même que celle du théorème 
pour le changement de variable usuel.

Encore une fois, au lieu d'appliquer directement ce théorème, 
on peut effectuer un changement de variable classique sur l'intervalle $[a,x]$,
puis étudier la limite lorsque $x\to +\infty$.



%%%%%%%%%%%%%%%%%%%%%%%%%%%%%%%%%%%%%%%%%%%%%%%%%%%%%%%%%%%%%%%% 
\diapo

L'exemple suivant est particulièrement intéressant : 
comme on va le voir tout de suite, l'intégrale de Fresnel  
$\int_1^{+\infty} \sin (t^2) \; \dd t$ converge.

En d'autres termes la fonction $f(t) = \sin(t^2)$ 
a une intégrale convergente, bien qu'elle ne tende pas vers 
$0$ lorsque $t\to+\infty$, comme on le voit sur le graphe.
L'aire sous la courbe est finie, bien que les valeurs $+1$ et $-1$ 
soient atteintes une infinité de fois.

C'est un comportement opposé à celui des séries : 
pour une série convergente, le terme général tend toujours vers $0$. C'est faux pour les intégrales.

\change 
Pour montrer la convergence de l'intégrale de Fresnel, 

\change
on effectue le changement de variable $u=t^2$, ce qui donne
$t=\sqrt{u}$ et  $\dd t=\frac{\dd u}{2\sqrt u}$. 

\change
D'autre part la fonction racine carrée un difféomorphisme entre 
les intervalles $[1,x^2]$ et $[1,x]$.
La fonction racine carrée est bien $C^1$, sa bijection réciproque 
est la fonction carrée qui est aussi $C^1$.


\change 
On applique le théorème de changement de variable :
$\int f(t) \;\dd t 
= \int f\big(\varphi(u)\big)\cdot\varphi'(u) \;\dd u$

\change
Plus prosaiquement on remplace $t^2$ par $u$
et $dt$ par $du/(2\sqrt u)$.

On fait bien attention aux bornes :
$t$ varie de $1$ à $x$, ce qui fait que $u$ varie de $1$ à $x^2$ :

$$\int_1^x \sin (t^2) \; \dd t = \int_1^{x^2} \sin (u) \; \frac{\dd u}{2\sqrt u}.$$

\change 
Or on vérifie immédiatement grâce au théorème d'Abel 
que l'intégrale $\int_1^{+\infty} \frac{\sin u}{\sqrt u} \; \dd u$
converge, 

\change 
donc  
$\int_1^{x^2} \sin (u) \; \frac{\dd u}{2\sqrt u}$ admet 
une limite finie lorsque $x\to+\infty$, 

\change
ce qui prouve que $\int_1^x \sin (t^2) \; \dd t$ 
admet aussi une limite finie. 

\change

On a bien prouvé que $\int_1^{+\infty} \sin (t^2) \; \dd t$ converge.

%%%%%%%%%%%%%%%%%%%%%%%%%%%%%%%%%%%%%%%%%%%%%%%%%%%%%%%%%%%%%%%% 
\diapo


Je vous résume maintenant l'ensemble des techniques vues dans ce chapitre
pour l'étude des intégrales impropres, avec éventuellement plusieurs points incertains. Pour
illustrer le plan d'étude, nous détaillerons l'exemple introductif du chapitre :

Soit
$$I = \int_{-\infty}^{+\infty} \frac{\sin |t|}{|t|^{3/2}}\;\dd t$$  


\change

La première étape 1 : est d'identifier le ou les points incertains.

\change
Ici la fonction $|t|^{-3/2}\sin |t|$ est définie sur $\Rr^*$.

\change
Elle est paire, et tend vers $0$, mais en oscillant, quand $t$ tend vers $\pm \infty$, 

\change
Par contre elle tend vers $+\infty$ en $0^-$ et en $0^+$, 
[figure]. 

\change
Il y a donc $4$ points incertains à étudier 
$-\infty$, $0^-$, $0^+$ et $+\infty$.  [figure]



%%%%%%%%%%%%%%%%%%%%%%%%%%%%%%%%%%%%%%%%%%%%%%%%%%%%%%%%%%%%%%%%  
\diapo

La deuxième étape c'est d'isoler les points incertains :
on découpe le domaine de définition en autant de
sous-intervalles qu'il y a de points incertains, de manière à ce
que les problèmes soient tous situés sur une seule borne de chaque
intervalle.

\change
Dans notre exemple, on divisera $\R^*$ en 4 intervalles : 
$]-\infty,-1]$, $[-1,0[$, $]0,1]$ et $[1,+\infty[$.  

Je vous rappelle que le point de découpe, ici $-1$ et ici $+1$, n'a pas d'importance
pour la convergence.

\change
On étudie chacune des intégrales obtenues  $I_1$, $I_2$, $I_3$ et $I_4$  
séparément. 

\change
L'intégrale $I$ ne converge que si *chacun* des quatre morceaux converge, 

\change
et dans ce cas $I=I_1+I_2+I_3+I_4$, d'après la relation de Chasles.


\change
Ensuite on peut se ramener à une intégrale sur un intervalle du type 
$[a,+\infty[$ ou du type $]a,b]$.
Car c'est sur ce type d'intervalle que l'on a énoncé nos théorèmes.

\change
On utiliserait pour cela  le changement de variable 
$t\mapsto -t$. 

\change
Dans l'exemple considéré, la fonction est paire, et donc
$I_1=I_4$ et $I_2=I_3$, et on n'a plus que deux intégrales à étudier.

  


%%%%%%%%%%%%%%%%%%%%%%%%%%%%%%%%%%%%%%%%%%%%%%%%%%%%%%%%%%%%%%%% 
\diapo

La dernière étape c'est de déterminer si l'intégrale converge,
ou mieux de calculer l'intégrale.


Si c'est possible, on calculer une primitive de la fonction à intégrer.

Si on a une primitive explicite, le problème est ramené à un calcul 
de limite. 

Sinon,  on passe alors à l'étape suivante pour décider de la convergence.

\change
Tout d'abord si la fonction est de signe constant :

\change
On peut utiliser le théorème des équivalents 
ou le théorème de comparaison, en choisissant le plus adapté.

\change
Revenons à notre exemple, 

\change
$$
t^{-3/2} \sin |t|  \;\mathop{\sim}_{0^+}\; t^{-1/2}\;.
$$
car $\sin t \sim t$

et les fonctions positives autour de $t=0$.

\change
Comme on sait que $\int_0^1 \frac{1}{t^{1/2}}\;\dd t$ converge
car $1/2 < 1$
(intégrale de Riemann)

\change
alors par le théorème des équivalents,
l'intégrale $I_3 = \int_{0}^{1} |t|^{-3/2} \sin |t| \;\dd t$ 
converge aussi.

\change 
Enfin, voici le cas où la fonction n'est pas de signe constant.

\change
On peut commencer par étudier l'intégrale de $|f|$, 
avec le théorème des équivalents ou le théorème de comparaison. 
Si l'intégrale de $|f|$ converge,  l'intégrale de $f$ converge aussi.


\change
Si l'intégrale de $|f|$ ne converge pas,
on peut essayer en dernier recours d'appliquer le théorème d'Abel. 

\change
Revenons à notre exemple : 

(.../...) 

\change
on majore $t^{-3/2} \big|\sin |t|$ par $t^{-3/2}$.

\change
On sait que l'intégrale de Riemann 
$\int_1^{+\infty} \frac{1}{t^{3/2}}\;\dd t$ converge.

\change
Donc grâce au théorème de comparaison :
l'intégrale $I_4=\int_{1}^{+\infty} |t|^{-3/2} \sin |t| \;\dd t.$ 
est absolument convergente,  donc convergente.

\change
On conclut que
$I = \int_{-\infty}^{+\infty} \frac{\sin |t|}{|t|^{3/2}}\;\dd t
=  I_1+I_2+I_3+I_4$

$= 2 I_3 + 2 I_4$ par parité 

est une intégrale convergente.


 

%%%%%%%%%%%%%%%%%%%%%%%%%%%%%%%%%%%%%%%%%%%%%%%%%%%%%%%%%%%%%%%% 
 \diapo

 A vous de vous exercer au changement de variable 
 et à l'intégration par parties avec ces énoncés !

\end{document}

\documentclass[class=report,crop=false]{standalone}
\usepackage[screen]{../exo7book}

\begin{document}

%====================================================================
\chapitre{Intégrales impropres}
%====================================================================

%\insertvideo{}{partie 1. Définitions et premières propriétés}

%\insertvideo{}{partie 2. Fonctions positives}

%\insertvideo{}{partie 3. Fonctions oscillantes}

%\insertvideo{}{partie 4. Intégrales impropres sur un intervalle borné}

%\insertvideo{}{partie 5. Intégration par parties -- Changement de variable}



%%%%%%%%%%%%%%%%%%%%%%%%%%%%%%%%%%%%%%%%%%%%%%%%%%%%%%%%%%%%%%%%
\section{Définitions et premières propriétés}


La plupart des intégrales que vous rencontrerez ne sont pas des
aires de domaines bornés du plan. Nous allons apprendre ici à
calculer les intégrales de domaines non bornés, 
soit parce que
l'intervalle d'intégration est infini (allant jusqu'à $+\infty$ ou $-\infty$), 
soit parce que la fonction à intégrer tend vers l'infini aux bornes de
l'intervalle. Pour assimiler ce chapitre, vous avez juste besoin d'une
petite révision des techniques de calcul des primitives, et d'une
bonne compréhension de la notion de limite.

%---------------------------------------------------------------
\subsection{Points incertains}



Considérons par exemple la fonction $f$ qui à
$t\in\;]-\infty,0[\cup]0,+\infty[$ associe $f(t) = \frac{\sin |t|}{|t|^{\frac32}}$.
Comment donner un sens à l'intégrale de $f$ sur $\Rr$ ?

\myfigure{0.8}{
\tikzinput{fig_intimp01} 
}

\begin{itemize}
  \item On commence d'abord par identifier les \defi{points incertains}, soit
$+\infty$, soit $-\infty$ d'une part, et d'autre part le ou les points au voisinage
desquels la fonction n'est pas bornée ($t=0$ dans notre exemple).
  
  \item On découpe ensuite chaque intervalle d'intégration en autant d'intervalles
qu'il faut pour que chacun d'eux ne contienne qu'\evidence{un seul point
incertain}, placé à l'une des deux bornes. 

  \item Nous souhaitons une définition qui respecte la relation de Chasles.
  Ainsi l'intégrale sur l'intervalle complet est la somme des
intégrales sur les intervalles du découpage.

  \item Dans l'exemple de la
fonction $f(t)=\frac{\sin |t|}{|t|^{\frac32}}$ ci-dessus, 
il faut découper les deux intervalles de définition $]-\infty,0[$ et $]0,+\infty[$
en $4$ sous-intervalles : $2$ pour isoler
$-\infty$ et $+\infty$, et $2$ autres pour le point incertain $0$.

  \item On pourra écrire pour cet exemple :
$$
\int_{-\infty}^{+\infty} f(t)\;\dd t
=
\int_{-\infty}^{-1} f(t)\;\dd t+
\int_{-1}^{0} f(t)\;\dd t+
\int_{0}^{1} f(t)\;\dd t+
\int_{1}^{+\infty} f(t)\;\dd t\;.
$$
  \item Le seul but est d'isoler les difficultés : les choix de $-1$ et $1$
comme points de découpage sont arbitraires (par exemple $-3$ et $10$ auraient
convenu tout aussi bien). 
\end{itemize}


%---------------------------------------------------------------
\subsection{Convergence/divergence}


Par ce découpage, et par changement de variable $t\mapsto-t$,
on se ramène à des intégrales de deux types.
\begin{enumerate}
\item Intégrale sur $[a,+\infty[$.
\item Intégrale sur $]a,b]$, avec la fonction non bornée en $a$.
\end{enumerate}

Nous devons donc définir une intégrale, appelée \defi{intégrale impropre}, dans ces deux cas.
\begin{definition}
\label{def:intcv}
\begin{enumerate}
\item Soit $f$ une fonction continue sur $[a,+\infty[$.
On dit que l'intégrale $\int_a^{+\infty} f(t)\;\dd t$ \defi{converge} si
la limite, lorsque $x$ tend vers $+\infty$, de la primitive 
$\int_a^{x} f(t)\;\dd t$ existe et est finie. Si c'est le cas, on pose :
\begin{equation}
\label{intcv1}
\int_a^{+\infty} f(t)\;\dd t = \lim_{x\rightarrow+\infty} 
\int_a^x f(t)\;\dd t\;. 
\end{equation}
Dans le cas contraire, on dit que l'intégrale \defi{diverge}.

\item Soit $f$ une fonction continue sur $]a,b]$.
On dit que l'intégrale $\int_a^b f(t)\;\dd t$ \defi{converge} si
la limite à droite, lorsque $x$ tend vers $a$, de 
$\int_x^{b} f(t)\;\dd t$ existe et est finie. Si c'est le cas, on pose :
\begin{equation}
\label{intcv2}
\int_a^{b} f(t)\;\dd t = \lim_{x\rightarrow a^+} 
\int_x^b f(t)\;\dd t\;. 
\end{equation}
Dans le cas contraire, on dit que l'intégrale \defi{diverge}.
\end{enumerate}
\end{definition}


\begin{remarque*}
\begin{itemize}
  \item Convergence équivaut donc à limite finie. Divergence signifie soit qu'il 
  n'y a pas de limite, soit que la limite est infinie.
  
  \item Observons que la deuxième définition est cohérente avec l'intégrale d'une fonction 
qui serait continue sur $[a,b]$ tout entier (au lieu de $]a,b]$).
On sait que la primitive $\int_x^b f(t)\;\dd t$ est une fonction continue.
Par conséquent, l'intégrale usuelle $\int_a^b f(t)\;\dd t$ est aussi
la limite de $\int_x^b f(t)\;\dd t$ (lorsque $x \to a^+$). Dans ce cas, les 
deux intégrales coïncident. 
\end{itemize}
\end{remarque*}



%---------------------------------------------------------------
\subsection{Exemples}

Quand on peut calculer une primitive $F(x)$ de la fonction à intégrer (par 
exemple $F(x)=\int_a^{x} f(t)\;\dd t$),
l'étude de la convergence se ramène à un calcul de limite de $F(x)$. 
Voici plusieurs exemples.


\begin{exemple}
L'intégrale
$$
\int_0^{+\infty} \frac{1}{1+t^2}\;\dd t\qquad \text{ converge.}
$$
En effet,
$$
\int_0^{x} \frac{1}{1+t^2}\;\dd t =\Big[\arctan t\Big]_0^x
=\arctan x
\qquad \text{ et } \qquad 
\lim_{x\rightarrow+\infty}\arctan x = \frac{\pi}{2}\;.
$$
On pourra écrire :
$$
\int_0^{+\infty} \frac{1}{1+t^2}\;\dd t = \Big[\arctan t\Big]_0^{+\infty}
=\frac{\pi}{2}\;,
$$
à condition de se souvenir que $\Big[\arctan t\Big]_0^{+\infty}$
désigne une limite en $+\infty$.  

\myfigure{1}{
\tikzinput{fig_intimp07} 
}

Cela prouve que le domaine sous la courbe n'est pas borné, mais 
cependant son aire est finie !
\end{exemple}

\begin{exemple}
Par contre, l'intégrale
$$
\int_0^{+\infty} \frac{1}{1+t}\;\dd t\qquad \text{ diverge.}
$$
En effet,
$$
\int_0^{x} \frac{1}{1+t}\;\dd t = \Big[\ln(1+t)\Big]_0^x
=\ln(1+x)
\qquad \text{ et } \qquad 
\lim_{x\rightarrow+\infty}\ln(1+x) = +\infty\;.
$$  
\end{exemple}

\begin{exemple}
L'intégrale
$$
\int_0^1 \ln t\;\dd t\qquad \text{ converge.}
$$
En effet,
$$
\int_x^1 \ln t\;\dd t = \Big[t\ln t-t\Big]_x^1 = x-x\ln x-1
\qquad \text{ et } \qquad 
\lim_{x\rightarrow 0^+} (x-x\ln x-1) = -1\;.
$$
On pourra écrire :
$$
\int_0^1 \ln t\;\dd t = \Big[t\ln t-t\Big]_0^1 = -1\;.
$$  
\end{exemple}

\begin{exemple}
Par contre, l'intégrale
$$
\int_0^1 \frac{1}{t}\;\dd t\qquad \text{ diverge.}
$$
En effet,
$$
\int_x^1 \frac{1}{t}\;\dd t = \Big[\ln t\Big]_x^1 = -\ln x
\qquad \text{ et } \qquad 
\lim_{x\rightarrow 0^+} -\ln x = +\infty\;.
$$  
\end{exemple}






%---------------------------------------------------------------
\subsection{Relation de Chasles}

Lorsqu'elle converge, cette nouvelle intégrale vérifie les mêmes 
propriétés que l'intégrale de Riemann usuelle, à commencer par la relation de Chasles :

\begin{proposition}[Relation de Chasles]
Soit $f : [a,+\infty[ \to \Rr$ une fonction continue et soit
  $a' \in [a,+\infty[$. 
  Alors les intégrales impropres $\int_a^{+\infty} f(t) \;\dd t$ 
  et $\int_{a'}^{+\infty} f(t) \;\dd t $ sont de même nature. 
  Si elles convergent, alors 
  \mybox{$\displaystyle \int_a^{+\infty} f(t) \;\dd t = \int_{a}^{a'} f(t) \;\dd t+ \int_{a'}^{+\infty} f(t) \;\dd t.$}
\end{proposition}

<<~\^Etre de même nature~>> signifie que les deux intégrales sont convergentes en même temps
ou bien divergentes en même temps.

Le relation de Chasles implique donc que la convergence ne dépend
pas du comportement de la fonction sur des intervalles bornés,
mais seulement de son comportement au voisinage de $+\infty$.

\begin{proof}
La preuve découle de la relation de Chasles pour les intégrales usuelles, avec $a \le a' \le x$ : 
$$\int_a^x f(t) \;\dd t = \int_a^{a'}f(t) \;\dd t+\int_{a'}^x f(t) \;\dd t\;.$$
Puis on passe à la limite (lorsque $x\to+\infty$).
\end{proof}

\bigskip

Bien sûr, si on est dans le cas d'une fonction continue $f : \ ]a,b] \to \Rr$ avec
$b' \in ]a,b]$, alors on a un résultat similaire, et en cas de convergence :
  $$\int_a^{b} f(t) \;\dd t = \int_{a}^{b'} f(t) \;\dd t+ \int_{b'}^{b} f(t) \;\dd t.$$

Dans ce cas la convergence de l'intégrale ne dépend pas de $b$, mais seulement 
du comportement de $f$ au voisinage de $a$.


%---------------------------------------------------------------
\subsection{Linéarité}

Le résultat suivant est une conséquence immédiate de la
linéarité des intégrales usuelles et des limites.

\begin{proposition}[Linéarité de l'intégrale]
\label{prop:lineariteintegralescv}
Soient $f$ et $g$ deux fonctions continues sur
$[a,+\infty[$, et $\lambda,\mu$ deux réels. Si
les intégrales $\int_a^{+\infty} f(t)\;\dd t$ et 
$\int_a^{+\infty} g(t)\;\dd t$ convergent, alors 
$\int_a^{+\infty} \big(\lambda f(t)+\mu g(t)\big)\;\dd t$ converge et
$$\int_a^{+\infty} \big(\lambda f(t) + \mu g(t)\big)\;\dd t = \lambda \int_a^{+\infty}
f(t)\;\dd t +\mu \int_a^{+\infty} g(t)\;\dd t\;.$$
\end{proposition}

\medskip

Les mêmes relations sont valables pour les 
fonctions d'un intervalle $]a,b]$, non bornées en $a$.

Remarque : la réciprocité dans la linéarité est fausse,
il est possible de trouver deux fonctions $f,g$ telles que
$\int_a^{+\infty} f+g$ converge, sans que $\int_a^{+\infty} f$, ni
$\int_a^{+\infty} g$ convergent. Trouvez un tel exemple !

%---------------------------------------------------------------
\subsection{Positivité}

\begin{proposition}[Positivité de l'intégrale]
Soient $f,g : [a,+\infty[ \to \Rr$ des fonctions continues, ayant une intégrale convergente.
\mybox{$\displaystyle \text{ Si }\quad f \le g \quad \text{ alors }\quad
\int_a^{+\infty} f(t)\;\dd t  \le \int_a^{+\infty} g(t)\;\dd t.$}
\end{proposition}

En particulier, l'intégrale (convergente) d'une fonction positive est positive :
\mybox{Si \quad $f\ge 0$ \quad alors \quad $\displaystyle \int_a^{+\infty} f(t)\;\dd t \ge 0$}

\medskip

Une nouvelle fois, les mêmes relations sont valables pour les 
fonctions définies sur un intervalle $]a,b]$, non bornées en $a$, 
en prenant bien soin d'avoir $a<b$.

\begin{remarque*}
Si l'on ne souhaite pas distinguer les deux types d'intégrales impropres
sur un intervalle $[a,+\infty[$ (ou $]-\infty,b]$) d'une part
et $]a,b]$ (ou $[a,b[$) d'autre part, alors il est pratique de rajouter les deux extrémités 
à la droite numérique :
\[ \overline{\Rr}=\Rr\cup\{-\infty,+\infty\}  \]
Ainsi l'intervalle $I=[a,b[$ avec $a\in \Rr$ et $b \in \overline{\Rr}$
désigne l'intervalle infini $[a,+\infty[$ (si $b=+\infty$) 
ou l'intervalle fini $[a,b[$ (si $b < +\infty$).
De même pour un intervalle $I'=]a,b]$ avec $a=-\infty$ ou $a \in \Rr$.
\end{remarque*}


%---------------------------------------------------------------
\subsection{Critère de Cauchy}

On termine par une caractérisation de la convergence un peu plus délicate
(qui peut être passée lors d'une première lecture).


Rappelons d'abord le critère de Cauchy pour les limites.

\textbf{Rappel} : 
Soit $f : [a,+\infty[ \to \Rr$. 
Alors $\lim_{x\to +\infty} f(x)$ existe et est finie  si et seulement si 
$$\forall \epsilon>0\quad\exists M \ge a \qquad 
\left( u,v \ge M \implies \big|f(u)-f(v)\big|<\epsilon\right).$$

 
\begin{theoreme}[Critère de Cauchy]
\label{th:intimpcauchy}
Soit $f : [a,+\infty[ \to \Rr$ une fonction continue.
L'intégrale impropre $\int_a^{+\infty} f(t) \; \dd t$ converge
si et seulement si
$$\forall \epsilon>0  \quad \exists  M \ge a \qquad
\left( u,v \ge M \implies \Bigl|\int_u^v f(t) \;\dd t\Bigr|<\epsilon\right).$$
\end{theoreme}

\begin{proof}
Il suffit d'appliquer le rappel ci-dessus à la fonction $F(x)=\int_a^x f(t) \; \dd t$ 
et en notant que $\big|F(u)-F(v)\big|=\big|\int_u^vf(t)\; \dd t\big|$.
\end{proof}

%---------------------------------------------------------------
\subsection{Cas de deux points incertains}

On peut considérer les intégrales doublement impropres, c'est-à-dire lorsque les 
deux extrémités de l'intervalle de définition sont des points incertains. Il s'agit juste de se ramener à deux 
intégrales ayant chacune un seul point incertain.

\begin{definition}
Soient $a,b\in \overline{\Rr}$ avec $a<b$. 
Soit $f : ]a,b[ \to \Rr$ une fonction continue.
On dit que l'intégrale $\int_a^b f(t)\;\dd t$ \defi{converge} s'il existe
$c \in ]a,b[$ tel que les \evidence{deux} intégrales
impropres $\int_a^c f(t)\;\dd t$ et $\int_c^b f(t)\;\dd t$ convergent.
La valeur de cette intégrale doublement impropre est alors
$$\int_a^c f(t)\;\dd t+\int_c^b f(t)\;\dd t.$$
\end{definition}

Les relations de Chasles impliquent que la nature et la valeur de cette intégrale
doublement impropre ne dépendent pas du choix de $c$, avec $a<c<b$.

\medskip

\textbf{Attention !}
Si une des deux intégrales $\int_a^c f(t)\;\dd t$ ou bien
$\int_c^b f(t)\;\dd t$ diverge, alors $\int_a^b f(t)\;\dd t$ diverge.
Prenons l'exemple de $\int_{-x}^{+x} t\;\dd t$ qui vaut toujours $0$,
pourtant $\int_{-\infty}^{+\infty}t\;\dd t$ diverge ! En effet, quel que soit 
$c\in \Rr$, $\int_{c}^{+x} t\;\dd t = \frac{x^2}{2}-\frac{c^2}{2}$ tend vers $+\infty$
(lorsque $x\to+\infty$).


\begin{exemple}
Est-ce que l'intégrale suivante converge ?
$$\int_{-\infty}^{+\infty} \frac{t \;\dd t}{(1+t^2)^2}$$

On choisit (au hasard) $c=2$. Il s'agit de savoir si les deux intégrales
$$\int_{-\infty}^{2} \frac{t \;\dd t}{(1+t^2)^2} \quad \text{ et } \quad 
\int_{2}^{+\infty} \frac{t \;\dd t}{(1+t^2)^2}$$
convergent.

En notant qu'une primitive de $\frac{t}{(1+t^2)^2}$ est $-\frac12\frac{1}{1+t^2}$, on obtient :
$$\int_{x}^{2} \frac{t \;\dd t}{(1+t^2)^2}
= -\frac12 \Big[ \frac{1}{1+t^2}\Big]_{x}^{2}
= -\frac12 \left(\frac15 - \frac{1}{1+x^2}\right)
\to -\frac{1}{10} \quad \text{ lorsque } x \to -\infty.$$
Donc $\int_{-\infty}^{2} \frac{t \;\dd t}{(1+t^2)^2}$ converge et vaut $-\frac{1}{10}$.

De même $$\int_{2}^{x} \frac{t \;\dd t}{(1+t^2)^2}
= -\frac12 \Big[ \frac{1}{1+t^2}\Big]_{2}^{x}
= -\frac12 \left(\frac{1}{1+x^2}-\frac15\right)
\to +\frac{1}{10} \quad \text{ lorsque } x \to +\infty.$$
Donc $\int_{2}^{+\infty} \frac{t \;\dd t}{(1+t^2)^2}$ converge et vaut $+\frac{1}{10}$.


Ainsi $\int_{-\infty}^{+\infty} \frac{t \;\dd t}{(1+t^2)^2}$ converge et vaut
$-\frac{1}{10}+\frac{1}{10} = 0$. Ce n'est pas surprenant car la fonction est une fonction impaire.
Refaites les calculs pour une autre valeur de $c$ 
et vérifiez que l'on obtient le même résultat.

\end{exemple}

\bigskip
\bigskip
\bigskip

On termine en expliquant le plan du reste du chapitre.
Lorsque l'on ne sait pas calculer une primitive, on a
recours à deux types de méthode : soit la fonction est de signe constant
au voisinage du point incertain, soit elle change de signe une infinité de fois
dans ce voisinage (on dit alors qu'elle <<~oscille~>>).
Nous distinguerons aussi le cas où le point incertain est $\pm \infty$
ou bien une valeur finie.
Il y a donc quatre cas distincts, selon le type du point incertain, et le signe, constant ou
non, de la fonction à intégrer. Ces quatre types sont schématisés
dans la figure suivante et leur étude fait l'objet des sections suivantes. 

\myfigure{0.8}{
\tikzinput{fig_intimp02}
\tikzinput{fig_intimp03}\\
\tikzinput{fig_intimp04}
\tikzinput{fig_intimp05}
}

Différents types d'intégrales :
 intervalle non borné, fonction de signe constant ;
 intervalle non borné, fonction oscillante ;
 intervalle borné, fonction de signe constant ;
 intervalle borné, fonction oscillante.


%---------------------------------------------------------------
%\subsection{Mini-exercices}


\begin{miniexercices}
\begin{enumerate}
  \item Pour chacune des intégrales suivantes, déterminer le point incertain, dire
  si l'intégrale converge, et si c'est le cas, calculer la valeur de l'intégrale :
  $$
  \int_0^1 \frac{1}{\sqrt {1-t}} \;\dd t \qquad
  \int_0^{+\infty}\cos t \;\dd t \qquad
  \int_0^1 \frac{1}{1-t} \;\dd t \qquad
  \int_{-\infty}^{\ln 2} e^t \;\dd t
  $$
  
  \item Même exercice pour ces intégrales ayant deux points incertains :
  $$
  \int_{-\infty}^{+\infty} \frac{\dd t}{1+t^2} \qquad
  \int_{-\infty}^{1} \frac{\dd t}{(t-1)^2} \qquad
  \int_{-\infty}^{+\infty} e^{-|t|}\;\dd t \qquad
  \int_0^{+\infty} \frac1t \;\dd t$$
  
   \item \'Ecrire la preuve de la linéarité des intégrales impropres. Même chose 
   pour la positivité.
   
\end{enumerate}
\end{miniexercices}



%%%%%%%%%%%%%%%%%%%%%%%%%%%%%%%%%%%%%%%%%%%%%%%%%%%%%%%%%%%%%%%%
\section{Fonctions positives}


Nous considérons ici $\int_a^{+\infty} f(t)\;\dd t$, où $f$ est de
signe constant au voisinage de $+\infty$. Quitte à réduire
l'intervalle d'intégration, et à changer éventuellement le signe
de $f$ s'il est négatif, 
nous supposerons que la fonction est positive ou
nulle sur l'intervalle d'intégration $[a,+\infty[$.

\myfigure{1}{
\tikzinput{fig_intimp02}
}


Rappelons que, par définition,
$$
\int_a^{+\infty} f(t)\;\dd t = \lim_{x\rightarrow +\infty} \int_a^x f(t)\;\dd t\;.
$$
Observons que si la fonction $f$ est positive, alors la primitive
$\int_a^x f(t)\;\dd t$ est une fonction croissante de $x$ (car sa
dérivée est $f(x)$). Quand $x$ tend vers l'infini, ou bien
$\int_a^x f(t)\;\dd t$ est bornée, et l'intégrale $\int_a^{+\infty}
f(t)\;\dd t$ converge, ou bien $\int_a^x f(t)\;\dd t$ tend vers
$+\infty$. 


%---------------------------------------------------------------
\subsection{Théorème de comparaison}

Si on ne peut pas (ou si on ne veut pas) calculer une primitive de $f$, 
on étudie la convergence en
comparant avec des intégrales dont la convergence est connue,
gr\^ace au théorème suivant.

\begin{theoreme}
\label{th:comparaisonintegrales1}
Soient $f$ et $g$ deux fonctions positives et continues sur
$[a,+\infty[$. Supposons que $f$ soit majorée par $g$ au voisinage de
$+\infty$ :
$$
\exists A\ge a \quad \forall t>A \qquad f(t)\le g(t)\;.
$$
\begin{enumerate}
  \item Si \  $\int_a^{+\infty} g(t)\;\dd t$ \ converge alors \  $\int_a^{+\infty} f(t)\;\dd t$ \  converge.
  \item Si \  $\int_a^{+\infty} f(t)\;\dd t$ \  diverge alors \  $\int_a^{+\infty} g(t)\;\dd t$ \  diverge.
\end{enumerate}
\end{theoreme}


\begin{proof}
Comme nous l'avons observé, la convergence des intégrales ne
dépend pas de la borne de gauche de l'intervalle, et nous pouvons
nous contenter d'étudier $\int_A^x f(t)\;\dd t$ et $\int_A^x g(t)\;\dd t$.
Or en utilisant la positivité de l'intégrale, on obtient que, pour
tout $x \ge A$,
$$\int_A^x f(t)\;\dd t \le \int_A^x g(t)\;\dd t\;.$$ 
Si $\int_A^{+\infty} g(t)\;\dd t$ converge, alors $\int_A^x f(t)\;\dd t$ est
une fonction croissante et majorée par $\int_A^{+\infty} g(t)\;\dd t$,
donc convergente. Inversement, si $\int_A^{x} f(t)\;\dd t$ tend vers
$+\infty$, alors $\int_A^{x} g(t)\;\dd t$ tend vers $+\infty$ également.
\end{proof}

Voici une application typique du théorème \ref{th:comparaisonintegrales1} de comparaison.
\begin{exemple}
Montrons que l'intégrale 
$$\int_1^{+\infty} t^\alpha e^{-t}\;\dd t\quad\text{ converge,}$$
quel que soit le réel $\alpha$. 

\begin{itemize}
  \item Pour cela nous écrivons d'abord : $t^\alpha e^{-t} = t^\alpha e^{-t/2}\,e^{-t/2}$.
  
  \item On sait que $\lim_{t\rightarrow+\infty} t^\alpha e^{-t/2} =0$, pour tout $\alpha$, 
  car l'exponentielle l'emporte sur les puissances de $t$.
  
  \item En particulier, il existe un réel $A>0$ tel que :
$$\forall t>A\qquad t^\alpha e^{-t/2}\le 1\;.$$ 
  
  \item En multipliant les deux membres de l'inégalité 
par $e^{-t/2}$ on obtient :
$$\forall t>A\qquad t^\alpha e^{-t}\le e^{-t/2}\;.$$

  \item Or l'intégrale $\int_1^{+\infty} e^{-t/2}\;\dd t$ converge. En effet :
$$
\int_1^x e^{-t/2}\;\dd t = \left[-2 e^{-t/2}\right]_1^x = 
2e^{-1/2} -2e^{-x/2}
\quad\text{ et }\quad
\lim_{x\rightarrow+\infty} 2e^{-1/2} -2e^{-x/2} =
2e^{-1/2}\;. 
$$
  
  \item On peut donc appliquer le théorème \ref{th:comparaisonintegrales1} de
comparaison : 
puisque 
$\int_1^{+\infty} e^{-t/2}\;\dd t$ converge, on en déduit que 
$\int_1^{+\infty} t^\alpha e^{-t}\;\dd t$ converge aussi.
\end{itemize}
\end{exemple}


%---------------------------------------------------------------
\subsection{Théorème des équivalents}


Gr\^ace au théorème \ref{th:comparaisonintegrales1} de comparaison, 
on peut remplacer la fonction à intégrer par un équivalent au voisinage de $+\infty$ pour
étudier la convergence d'une intégrale.

\begin{theoreme}[Théorème des équivalents]
\label{th:equivalentintegrales1}
Soient $f$ et $g$ deux fonctions continues et strictement positives sur
$[a,+\infty[$. Supposons qu'elles soient équivalentes au voisinage de $+\infty$, c'est-à-dire :
$$\lim_{t\rightarrow+\infty}\frac{f(t)}{g(t)} = 1\;.$$
Alors l'intégrale $\int_a^{+\infty} f(t)\;\dd t$ converge si et seulement si 
$\int_a^{+\infty} g(t)\;\dd t$ converge.
\end{theoreme}

Attention : il est important que $f$ et $g$ soient positives !

On notera le fait que $f$ et $g$ sont équivalentes au voisinage de $+\infty$ par : 
$\displaystyle f(t) \ \underset{+\infty}{\sim}\ g(t)$.
 
\begin{proof}
Dire que deux fonctions sont équivalentes au voisinage de $+\infty$,
c'est dire que leur rapport tend vers $1$, ou encore :
$$\forall \epsilon>0\quad \exists A>a\quad \forall t>A \qquad
\left|\frac{f(t)}{g(t)}-1\right|<\epsilon\;,$$
soit encore :
$$\forall \epsilon>0 \quad \exists A>a \quad \forall t>A \qquad (1-\epsilon)g(t)<f(t)<(1+\epsilon)g(t)\;.$$
Fixons $\epsilon<1$, et appliquons le théorème \ref{th:comparaisonintegrales1} de comparaison
sur l'intervalle $[A,+\infty[$. Si l'intégrale $\int_A^{+\infty}
  f(t)\;\dd t$ converge, alors l'intégrale 
$\int_A^{+\infty} (1-\epsilon)g(t)\;\dd t$ converge, donc l'intégrale
  $\int_A^{+\infty} g(t)\;\dd t$ aussi par linéarité. 

Inversement, si $\int_A^{+\infty} f(t)\;\dd t$ diverge, alors $\int_A^{+\infty}
  (1+\epsilon) g(t)\;\dd t$ diverge, donc 
$\int_A^{+\infty} g(t)\;\dd t$ diverge aussi. 
\end{proof}

\begin{exemple}
Est-ce que l'intégrale 
$$\int_1^{+\infty} \frac{t^5+3t+1}{t^3+4}e^{-t}\;\dd t\quad\text{ converge ?}$$
Comme 
$$\frac{t^5+3t+1}{t^3+4}e^{-t} \quad\underset{+\infty}{\sim}\quad t^2e^{-t}\;,
$$
et que nous avons déjà montré que l'intégrale 
$\int_1^{+\infty} t^2e^{-t}\;\dd t$ converge, alors
notre intégrale converge.
\end{exemple}




%---------------------------------------------------------------
\subsection{Intégrales de Riemann}

Pour l'étude de la convergence d'une intégrale pour laquelle on n'a pas de primitive, 
l'utilisation des équivalents permet de se ramener 
à un catalogue d'intégrales dont la nature est connue. Les
plus classiques sont les intégrales de Riemann et de Bertrand.


Une \defi{intégrale de Riemann} est :
$$\int_1^{+\infty} \frac{1}{t^{\alpha}}\;\dd t$$\\
où $\alpha>0$.


Dans ce cas, la primitive est explicite :
$$
\int_1^{+\infty} \frac{1}{t^{\alpha}}\;\dd t = 
\left\{
\begin{array}{lcl}
\displaystyle{\lim_{x\rightarrow+\infty} 
\Big[\tfrac{1}{-\alpha+1}\frac{1}{t^{\alpha-1}}\Big]_1^x}
&\quad\text{si}&\alpha\neq 1\\[2ex]
\displaystyle{\lim_{x\rightarrow+\infty} 
\Big[\ln t\Big]_1^x}
&\quad\text{si}&\alpha= 1
\end{array}
\right.
$$
On en déduit immédiatement la nature (convergente ou divergente)
des intégrales de Riemann.

\medskip

\mybox{

\begin{minipage}{0.5\textwidth}\vspace*{-2ex}
$$\text{Si } \quad \alpha > 1\quad \text{ alors }\quad 
\int_1^{+\infty} \frac{1}{t^{\alpha}}\;\dd t \quad\text{ converge.}$$
$$\text{Si } \quad \alpha\le 1\quad \text{ alors }\quad 
\int_1^{+\infty} \frac{1}{t^{\alpha}}\;\dd t \quad\text{ diverge.}$$
\end{minipage}
}


%---------------------------------------------------------------
\subsection{Intégrales de Bertrand}

Une \defi{intégrale de Bertrand} est 
$$\int_2^{+\infty} \frac{1}{t\;(\ln t)^{\beta}}\;\dd t$$
où $\beta \in \Rr$.


La primitive est explicite :
$$
\int_2^{+\infty} \frac{1}{t\;(\ln t)^{\beta}}\;\dd t = 
\left\{
\begin{array}{lcl}
\displaystyle{\lim_{x\rightarrow+\infty} 
\Big[\tfrac{1}{-\beta+1}(\ln t)^{-\beta+1}\Big]_2^x}
&\quad\text{si}&\beta\neq 1\\[2ex]
\displaystyle{\lim_{x\rightarrow+\infty} 
\Big[\ln(\ln t)\Big]_2^x}
&\quad\text{si}&\beta= 1
\end{array}
\right.
$$


On en déduit la nature des intégrales de Bertrand.
$$\text{Si } \beta > 1\quad \text{ alors }\quad
\int_2^{+\infty}\frac{1}{t\;(\ln t)^{\beta}}\;\dd t
\quad\text{ converge.}
$$

$$
\text{Si } \beta\le 1\quad \text{ alors }\quad
\int_2^{+\infty}\frac{1}{t\;(\ln t)^{\beta}}\;\dd t \quad\text{ diverge.}
$$



%---------------------------------------------------------------
\subsection{Application}

Voici un exemple d'application :
\begin{exemple}
Est-ce que l'intégrale 
$$
\int_2^{+\infty} 
\sqrt{t^2+3t}\,\ln\left(\cos\frac{1}{t}\right)
\,\sin^2\left(\frac{1}{\ln t}\right)\;\dd t\qquad \text{ converge ?}
$$

Le point incertain est $+\infty$. Pour répondre à la question, calculons un équivalent de la fonction au voisinage de $+\infty$.
On a :
$$\sqrt{t^2+3t} = t\sqrt{1+\frac{3}{t}} \quad  \underset{+\infty}{\sim}\quad   t$$
$$\ln\left(\cos\frac{1}{t}\right) = 
\ln\left(1-\frac{1}{2t^2}+o\left(\frac{1}{t^2}\right)\right)
\quad \underset{+\infty}{\sim}\quad  -\frac{1}{2t^2}$$
$$\sin^2\left(\frac{1}{\ln t}\right) 
\quad\underset{+\infty}{\sim}\quad \left(\frac{1}{\ln t}\right)^2$$

D'où un équivalent de la fonction au voisinage de
$+\infty$ : 
$$\sqrt{t^2+3t}\,\ln\left(\cos\frac{1}{t}\right)\,
\,\sin^2\left(\frac{1}{\ln t}\right)
\quad \underset{+\infty}{\sim}\quad  -\frac{1}{2t\;(\ln t)^2}$$

Remarquons que dans cette équivalence les deux fonctions sont négatives 
au voisinage de $+\infty$. 
D'après le théorème \ref{th:equivalentintegrales1},
les deux intégrales associées sont de même nature.
Mais comme l'intégrale de Bertrand 
$\int_2^{+\infty} \frac{1}{t\;(\ln t)^2}\;\dd t$ converge, 
alors notre intégrale initiale est aussi convergente.
\end{exemple}



%---------------------------------------------------------------
%\subsection{Mini-exercices}


\begin{miniexercices}
\begin{enumerate}
  \item \'Etudier la convergence des intégrales suivantes :  
  $$\int_{1}^{+\infty} \frac{1}{t}\sin\left(\frac{1}{t}\right) \;\dd t \qquad
  \int_{\frac{3}{\pi}}^{+\infty}\ln\left(\cos\frac{1}{t}\right) \;\dd t \qquad
  \int_{-\infty}^{+\infty} te^{-|t|}\;\dd t \qquad
  \int_{-\infty}^{\ln 2} \frac{e^{-t}}{t^2+1}\;\dd t$$
  
  \item Montrer que $\int_1^{+\infty} (\ln t)^\alpha e^{-t}\;\dd t$ converge,
quel que soit le réel $\alpha$. 

  \item \'Etudier la convergence des intégrales suivantes, en fonction du paramètre $\alpha>0$ :
  $$\int_{1}^{+\infty} \left(1-\sqrt[3]{1+\frac{1}{t^\alpha}}\right) \;\dd t \qquad
  \int_{\frac{1}{\pi}}^{+\infty} \left(\frac{1}{t^\alpha}-\sin\frac{1}{t^\alpha}\right)  \;\dd t \qquad
  \int_{-\infty}^{\pi} \alpha^t \;\dd t$$
  
  \item Discuter selon $\alpha>0$ et $\beta \in \Rr$ la nature de l'intégrale 
  de Bertrand généralisée
  $$\int_2^{+\infty} \frac{1}{t^\alpha(\ln t)^{\beta}}\;\dd t.$$
  
  \item Si $f : [a,+\infty[ \to \Rr$ est une fonction continue et positive telle que
  $\int_a^{+\infty} f(t)\;\dd t = 0$, montrer alors que $f$ est identiquement nulle.
  above
\end{enumerate}
\end{miniexercices}




%%%%%%%%%%%%%%%%%%%%%%%%%%%%%%%%%%%%%%%%%%%%%%%%%%%%%%%%%%%%%%%%
\section{Fonctions oscillantes}

Nous considérons ici $\int_a^{+\infty} f(t)\;\dd t$, où $f(t)$
oscille jusqu'à l'infini entre des valeurs positives et négatives.


\myfigure{1}{
\tikzinput{fig_intimp03}
}

La définition de l'intégrale impropre reste la même :
$$\int_a^{+\infty} f(t)\;\dd t = \lim_{x\rightarrow +\infty} \int_a^x f(t)\;\dd t\;.$$

Contrairement au cas des fonctions positives, où la limite était
soit finie, soit égale à $+\infty$, tous les comportements sont
possibles ici : les valeurs de $\int_a^x f(t)\;\dd t$ peuvent tendre vers
une limite finie, vers $+\infty$ ou $-\infty$, ou bien encore osciller
entre deux valeurs finies (comme $\int_a^x \sin t \;\dd t$), ou
s'approcher alternativement de $+\infty$ et $-\infty$ (comme $\int_a^x
t\sin t \;\dd t$). 


%---------------------------------------------------------------
\subsection{Intégrale absolument convergente}

Le cas le plus favorable est celui où la \emph{valeur absolue} de $f$
converge.

\begin{definition}
Soit $f$ une fonction continue sur $[a,+\infty[$. On dit que
  $\int_a^{+\infty} f(t)\;\dd t$ est \defi{absolument convergente} si 
$\int_a^{+\infty} \big|f(t)\big|\;\dd t$ converge.
\end{definition}

Le théorème suivant est souvent utilisé pour démontrer la
convergence d'une intégrale. Malheureusement, il ne permet pas de
calculer la valeur de cette intégrale. 

\begin{theoreme}
\label{th:acvimpliquecv}
Si l'intégrale $\int_a^{+\infty} f(t)\;\dd t$ est absolument
convergente, alors elle est convergente. 
\end{theoreme}

Autrement dit, être absolument convergent est plus fort qu'être convergent.


\begin{proof} 
C'est une conséquence du critère de Cauchy (théorème \ref{th:intimpcauchy})
appliqué à $|f|$, puis à $f$.

Comme $\int_a^{+\infty} \big|f(t)\big| \; \dd t$ converge
alors par le critère de Cauchy (sens direct) :
$$\forall \epsilon>0  \quad \exists  M \ge a \qquad
\left( u,v \ge M \implies \int_u^v \big|f(t)\big| \;\dd t<\epsilon\right).$$
Mais comme 
$$\left|\int_u^v f(t) \;\dd t\right| \le \int_u^v \big|f(t)\big| \;\dd t < \epsilon$$
alors par le critère de Cauchy (sens réciproque),
$\int_a^{+\infty} f(t) \; \dd t$ converge.
\end{proof}


\begin{exemple}
Par exemple,
$$\int_1^{+\infty} \frac{\sin t}{t^2}\;\dd t\quad\text{ est absolument convergente,}$$
donc convergente. 
En effet, pour tout $t$,
$$\frac{|\sin t |}{t^2}\le \frac{1}{t^2}\;.$$
Or l'intégrale de Riemann $\int_1^{+\infty} \frac{1}{t^2}\;\dd t$ est
convergente. D'où le résultat par le théorème \ref{th:comparaisonintegrales1} de comparaison.  
\end{exemple}



%---------------------------------------------------------------
\subsection{Intégrale semi-convergente}

\begin{definition}
Une intégrale $\int_a^{+\infty} f(t)\;\dd t$ est \defi{semi-convergente} si 
elle est convergente mais pas absolument convergente.
\end{definition}

\begin{exemple}
\label{ex:abel}
$$\int_1^{+\infty} \frac{\sin t }{t}\;\dd t \quad\text{ est semi-convergente.}$$

Nous allons prouver qu'elle est convergente, mais pas absolument convergente.
\begin{enumerate}

  
  \item \textbf{L'intégrale est convergente.}

  Pour le montrer, effectuons une intégration par parties (avec $u'=\sin t$, $v=\frac1t$) :
$$\int_1^x \frac{\sin t}{t}\;\dd t  = \left[\frac{-\cos t}{t}\right]_1^x
-  \int_1^x \frac{\cos t}{t^2}\;\dd t\;.$$

Examinons les deux termes :
  \begin{itemize}
    \item $\left[\frac{-\cos t}{t}\right]_1^x = -\frac{\cos x}{x}+\cos 1$.
    Or la fonction $\frac{\cos x}{x}$ tend vers $0$ (lorsque $x \to +\infty$), 
    car $\cos x$ est bornée et $\frac{1}{x}$ tend vers $0$.
    Donc $\left[\frac{-\cos t}{t}\right]_1^x$ admet une limite finie (qui est $\cos 1$). 
    
    \item Pour l'autre terme, notons d'abord que 
    $\int_1^{+\infty} \frac{\cos t}{t^2}\;\dd t$ est une intégrale absolument convergente.
    En effet $\frac{| \cos t |}{t^2} \le \frac{1}{t^2}$
    et l'intégrale de Riemann $\int_1^{+\infty} \frac{1}{t^2}\;\dd t$ converge.
    
    Par conséquent, $\int_1^{+\infty} \frac{\cos t}{t^2}\;\dd t$ converge, ce qui signifie exactement
    que  $\int_1^{x} \frac{\cos t}{t^2}\;\dd t$ admet une limite finie.    
  \end{itemize}

Conclusion : $\int_1^x \frac{\sin t}{t}\;\dd t$ admet une limite finie (lorsque $x\to+\infty$), et 
donc par définition $\int_1^{+\infty} \frac{\sin t}{t}\;\dd t$ converge.

  
  \item \textbf{L'intégrale n'est pas absolument convergente.}
  
  Voici un moyen de le vérifier. Comme $\big| \sin t \big|\le 1$ pour tout
$t$, on a :
$$\frac{|\sin t|}{t} \ge \frac{\sin^2 t}{t} = \frac{1-\cos(2t)}{2t}\;.$$
En appliquant une intégration par parties à 
$\frac{\cos(2t)}{t}$ (avec $u'=\cos(2t)$ et $v=\frac1t$), on obtient :
$$\int_1^x \frac{1-\cos(2t)}{2t}\;\dd t
 = \frac{1}{2}\Big[\ln t\Big]_1^x - \frac{1}{4}
 \left[\frac{\sin(2t)}{t}\right]_1^x
-\frac{1}{4}\int_1^x \frac{\sin(2t)}{t^2}\;\dd t.
$$ 
Or $\int_1^{+\infty}\frac{\sin(2t)}{t^2}\;\dd t$ converge absolument. 
Des trois termes de la somme ci-dessus, les deux
derniers convergent, et le premier tend vers $+\infty$. Donc
l'intégrale diverge, et par le théorème \ref{th:comparaisonintegrales1} de comparaison, 
l'intégrale $\int_1^{+\infty}
\frac{|\sin t|}{t}\;\dd t$ diverge également.
  
\end{enumerate}  
\end{exemple}


%---------------------------------------------------------------
\subsection{Théorème d'Abel}


Pour montrer qu'une intégrale converge, quand elle n'est pas
absolument convergente, on dispose du théorème suivant.

\begin{theoreme}[Théorème d'Abel]
\label{th:abelintcv1}
Soit $f$ une fonction $\mathcal{C}^1$ sur $[a,+\infty[$, positive,
décroissante, ayant une limite nulle en $+\infty$.
Soit $g$ une fonction continue sur $[a,+\infty[$, telle que la
primitive $\int_a^x g(t)\;\dd t$ soit bornée.
Alors l'intégrale
$$
\int_a^{+\infty} f(t)\,g(t)\;\dd t\quad\text{ converge.}
$$
\end{theoreme}

Avec $f(t)=\frac1t$ et $g(t)=\sin t$, on retrouve que 
l'intégrale $\displaystyle \int_1^{+\infty} \frac{\sin t }{t}\;\dd t$ converge.

\begin{proof} 
C'est une généralisation de l'exemple \ref{ex:abel} précédent.
Pour tout $x\ge a$, posons $G(x) = \int_a^x g(t)\;\dd t$. Par
hypothèse, $G$ est bornée, donc il existe $M$ tel que, pour tout $x$,  
$|G(x)|\le M$. Effectuons maintenant une intégration par parties :
$$\int_a^x f(t)\,g(t)\;\dd t = \Big[ f(t)\,G(t)\Big ]_a^x 
-\int_a^x f'(t)\,G(t)\;\dd t\;.$$
Comme $G$ est bornée et $f$ tend vers $0$, le terme 
entre crochets converge. Montrons maintenant que le second terme
converge aussi, en vérifiant que 
$$\int_a^{+\infty} f'(t)\,G(t)\;\dd t\quad
\text{ est absolument convergente.} $$
On a :
$$\big|f'(t)\,G(t)\big| =\big|f'(t)\big|\,\big|G(t)\big| \le \big(-f'(t)\big)\,M\;,$$ 
car $f$ est décroissante (donc $f'(t)\le 0$) et $|G|$ est bornée
par $M$. Par le théorème \ref{th:comparaisonintegrales1} de comparaison, 
il suffit donc de montrer que
$\int_a^{+\infty} -f'(t) \;\dd t$ est convergente. Or :
$$
\int_a^x -f'(t)\;\dd t = f(a)-f(x)\quad \text{ et } \quad
\lim_{x\rightarrow+\infty}(f(a)-f(x)) = f(a)\;. 
$$ 
\end{proof} 

\begin{exemple}
Comme exemple d'application, si $\alpha$ est un réel strictement
positif, et $k$ un entier positif \emph{impair}, alors l'intégrale 
$$
\int_1^{+\infty} \frac{\sin^k(t)}{t^\alpha}\;\dd t\quad\text{ converge.}
$$
Remarquons que cette intégrale n'est absolument convergente que 
pour $\alpha>1$.
On vérifie que les hypothèses du théorème \ref{th:abelintcv1} sont
satisfaites pour $f(t) = \frac{1}{t^\alpha}$ et $g(t)=\sin^k(t)$. Pour
s'assurer que la primitive de $\sin^k$ est bornée, il suffit de
penser à une linéarisation, qui transformera $\sin^k(t)$ en une
combinaison linéaire des $\sin(\ell t)$, $\ell=1,\ldots, k$, dont la
primitive sera toujours bornée.  
\end{exemple}


%---------------------------------------------------------------
%\subsection{Mini-exercices}


\begin{miniexercices}
\begin{enumerate}

 
  \item Soient $f :[a,+\infty[ \to \Rr$ une fonction continue et $\alpha > \frac12$
  tel que $\lim_{t\to+\infty} t^\alpha f(t) = 0$.
  Montrer que l'intégrale $\int_a^{+\infty} \frac{f(t)}{t^\alpha}\;\dd t$ est  absolument convergente.
  En déduire que si $P(t)$ est un polynôme alors
  $\int_0^{+\infty} P(t)e^{-t}\;\dd t$ converge.


  \item En admettant que le théorème d'Abel est vrai pour une fonction $g$ à valeurs complexes,
  montrer que $\int_1^{+\infty} \frac{e^{\ii t}}{t^\alpha}\; \dd t$ est convergente pour tout $\alpha > 0$.
  Est-elle toujours absolument convergente ?
  
  \item Montrer $\int_{n\pi}^{(n+1)\pi} \left|\frac{\sin t}{t}\right| \;\dd t \ge \frac{2}{(n+1)\pi}$.
  (Indication : garder le sinus dans une première minoration.)
  En utilisant un résultat sur les séries, en déduire une nouvelle démonstration du fait 
  que l'intégrale $\int_\pi^{+\infty} \frac{\sin t}{t} \;\dd t$ n'est pas absolument convergente. 
\end{enumerate}
\end{miniexercices}





%%%%%%%%%%%%%%%%%%%%%%%%%%%%%%%%%%%%%%%%%%%%%%%%%%%%%%%%%%%%%%%%
\section{Intégrales impropres sur un intervalle borné}


%---------------------------------------------------------------
\subsection{Fonctions positives}

Nous traitons ici le cas où la fonction à intégrer tend vers
l'infini en l'une des bornes de l'intervalle d'intégration. 
Le traitement est tout à fait analogue au cas d'une fonction positive 
sur un intervalle non borné et l'on omettra les démonstrations.

Quitte à réduire l'intervalle d'intégration, et à changer
éventuellement le signe de $f$, nous pouvons supposer que la
fonction est positive ou nulle sur l'intervalle d'intégration
$]a,b]$, et tend vers $+\infty$ en $a$.  

\myfigure{1}{
\tikzinput{fig_intimp04}
}

Rappelons que, par définition,
$$\int_a^b f(t)\;\dd t = \lim_{x\rightarrow a^+} \int_x^b f(t)\;\dd t\;.$$
Observons que si la fonction $f$ est positive, alors 
$\int_x^b f(t)\;\dd t$ croît quand $x$ décroît vers $a$ : 
soit $\int_x^b f(t)\;\dd t$ est bornée, et l'intégrale $\int_a^b
f(t)\;\dd t$ est convergente, soit $\int_x^b f(t)\;\dd t$ tend vers
$+\infty$. 

%---------------------------------------------------------------
\subsection{Théorème de comparaison}


\begin{theoreme}
\label{th:comparaisonintegrales2}
Soient $f$ et $g$ deux fonctions positives et continues sur $]a,b]$. 
Supposons que $f$ soit majorée par $g$ au voisinage de $a$, c'est-à-dire :
$$\exists \epsilon>0 \quad \forall t\in ]a,a+\epsilon] \qquad f(t)\le g(t)\;.$$
\begin{enumerate}
  \item Si \  $\int_a^b g(t)\;\dd t$ \  converge alors \  $\int_a^b f(t)\;\dd t$ \  converge.
  \item Si \  $\int_a^b f(t)\;\dd t$ \  diverge alors \  $\int_a^b g(t)\;\dd t$ \  diverge.
\end{enumerate}
\end{theoreme}


\begin{exemple}
Fixons un réel $\alpha$. 
Est-ce que l'intégrale 
$$\int_0^1 \frac{(-\ln t)^\alpha}{\sqrt{t}}\;\dd t\quad\text{ converge ?}$$

\begin{itemize}
  \item Pour le savoir nous écrivons :
$$\frac{(-\ln t)^\alpha}{\sqrt{t}} = \big((-\ln t)^\alpha t^{1/4}\big)\,t^{-3/4}\;. $$
  
  \item On sait que $\lim_{t\rightarrow 0^+}  (-\ln t)^\alpha t^{1/4}=0$, 
  pour tout $\alpha$ (les puissances de $t$ l'emportent sur le logarithme). 
  
  \item En particulier, il existe un réel $\epsilon>0$ tel que :
$$\forall t\in]0,\epsilon]\quad (-\ln t)^\alpha t^{1/4}\le 1\;.$$ 
  
  \item En multipliant les deux membres de l'inégalité 
par $t^{-3/4}$ on obtient :
$$\forall t\in]0,\epsilon]\quad \frac{(-\ln t)^\alpha}{\sqrt{t}} \le t^{-3/4}\;.$$
  
  \item Or l'intégrale $\int_0^1 t^{-3/4}\;\dd t$ converge. En effet :
$$
\int_x^1 t^{-3/4}\;\dd t = \Big[4 t^{1/4}\Big]_x^1 = 
4-4x^{1/4}
\quad\text{ et }\quad
\lim_{x\rightarrow 0^+} (4-4x^{1/4}) =
4\;. 
$$

  \item On peut donc appliquer le théorème \ref{th:comparaisonintegrales2} de
comparaison : puisque $\int_0^1 t^{-3/4}\;\dd t$ converge, alors 
$\int_0^1 \frac{(-\ln t)^\alpha}{\sqrt{t}}\;\dd t$ converge aussi, quel que soit $\alpha$.  
\end{itemize}
\end{exemple}


%---------------------------------------------------------------
\subsection{Théorème des équivalents}

Gr\^ace au théorème \ref{th:comparaisonintegrales2} de comparaison, 
on peut remplacer la fonction à intégrer par un équivalent au voisinage de $a$ pour
étudier la convergence d'une intégrale.

\begin{theoreme}
\label{th:equivalentintegrales2}
Soient $f$ et $g$ deux fonctions continues et strictement positives sur
$]a,b]$. 
Supposons qu'elles soient équivalentes au voisinage de $a$, c'est-à-dire :
$$\lim_{t\rightarrow a^+}\frac{f(t)}{g(t)} = 1\;.$$
Alors l'intégrale $\int_a^b f(t)\;\dd t$ converge si et seulement si 
$\int_a^b g(t)\;\dd t$ converge.
\end{theoreme}

Attention : il est important que $f$ et $g$ soient positives.

L'équivalence de $f$ et $g$ au voisinage de $a$ sera notée par : 
$\displaystyle f(t) \ \underset{a}{\sim}\ g(t)$ (ou bien 
$\displaystyle f(t) \ \underset{a^+}{\sim}\ g(t)$ pour préciser 
que la limite en $a$ est la limite à droite).


\begin{exemple}
L'intégrale 
$$\int_0^1 \sqrt{\frac{-\ln t+1}{\sin t}}\;\dd t\qquad\text{ converge.}$$
En effet,
$$\sqrt{\frac{-\ln t+1}{\sin t}} \quad\underset{0^+}{\sim}\quad
\frac{(-\ln t)^{1/2}}{\sqrt{t}}\;,$$
et nous avons déjà montré que l'intégrale 
$\int_0^1 \frac{(-\ln t)^{1/2}}{\sqrt{t}}\;\dd t$ converge.   
\end{exemple}

\bigskip

L'utilisation des équivalents permet ainsi de ramener l'étude de la
convergence d'une intégrale pour laquelle on n'a pas de primitive
à un catalogue d'intégrales dont la convergence est connue. Les
plus classiques sont du type $\int_0^1 \frac{1}{t^\alpha}\;\dd t$, mais
attention, la convergence en fonction du paramètre $\alpha$ est
inversée par rapport aux intégrales de Riemann.

\begin{exemple}
$$\text{Si } \alpha < 1\quad \text{ alors } \quad \int_0^1 \frac{1}{t^\alpha}\;\dd t
\quad\text{ converge.}$$
$$\text{Si } \alpha \ge 1\quad \text{ alors } \quad \int_0^1 \frac{1}{t^\alpha}\;\dd t
\quad\text{ diverge.}$$  
\end{exemple}

%---------------------------------------------------------------
\subsection{Fonctions oscillantes}


Le dernier cas à traiter est celui où la fonction à intégrer
oscille au voisinage d'une des bornes, prenant des valeurs
arbitrairement proches de $+\infty$ ou $-\infty$.  

\myfigure{1}{
\tikzinput{fig_intimp05}
}

Le changement de variable $u=\frac{1}{t-a}$ permet de se ramener au cas
précédent d'une fonction oscillante sur un intervalle non borné, 
ce qui nous dispensera de donner autant de détails.

Rappelons que, par définition,
$$\int_a^b f(t)\;\dd t = \lim_{x\rightarrow a^+} \int_x^b f(t)\;\dd t\;.$$
La notion importante est toujours la convergence absolue.

\begin{definition}
Soit $f$ une fonction continue sur $]a,b]$. On dit que
  $\int_a^b f(t)\;\dd t$ est \defi{absolument convergente} si 
$\int_a^b \big|f(t)\big|\;\dd t$ est une intégrale convergente.
\end{definition}

Le théorème suivant se démontre de la même fa\c{c}on que le théorème \ref{th:acvimpliquecv}.

\begin{theoreme}
Si l'intégrale $\int_a^b f(t)\;\dd t$ est absolument convergente,
alors elle est convergente. 
\end{theoreme}


\begin{exemple}
\begin{enumerate}
  \item L'intégrale
$$\int_0^1 \frac{\sin\frac1t}{\sqrt{t}}\;\dd t\quad\text{ est absolument convergente,}$$
donc convergente. En effet, pour tout $t$,
$$\frac{\big|\sin\frac1t\big|}{\sqrt{t}}\le \frac{1}{\sqrt{t}}\;.$$
Or l'intégrale $\int_0^1 \frac{1}{\sqrt{t}}\;\dd t$
converge, d'où le résultat par le théorème \ref{th:comparaisonintegrales2} de comparaison.
  
  \item Par contre, 
$$
\int_0^1 \frac{\sin\frac1t}{t}\;\dd t\quad\text{ n'est pas absolument convergente,}
$$
mais elle est convergente.
Pour le voir, effectuons le changement de variable $t\mapsto \frac1u$ :
$$
\int_x^1 \frac{\sin\frac1t}{t}\;\dd t = 
\int_{1/x}^1 u\sin u\frac{-1}{u^2}\;\dd u = 
\int_1^{1/x} \frac{\sin u}{u}\;\dd u\;.
$$
Lorsque $x \to 0^+$ alors $\frac1x \to +\infty$.
Nous avons déjà montré que l'intégrale 
$\int_1^{+\infty} \frac{\sin u}{u}\;\dd u$ est convergente, sans être
absolument convergente. 
\end{enumerate}
  
\end{exemple}

\bigskip

On pourrait énoncer un théorème d'Abel analogue au théorème
\ref{th:abelintcv1}, mais cela n'est pas vraiment utile. D'une part
les fonctions auxquelles il s'appliquerait se rencontrent rarement,
et d'autre part, il est en général facile de se ramener à un
problème sur $[c,+\infty[$, par le changement de variable 
$t\mapsto u=\frac{1}{t-a}$ : nous l'avons déjà fait
pour $\int_0^1 \frac{\sin \frac 1t}{t}\;\dd t$.   


%---------------------------------------------------------------
%\subsection{Mini-exercices}


\begin{miniexercices}
\begin{enumerate}
  \item \'Etudier la convergence des intégrales :
  $$\int_0^1 \frac{\dd t}{1-t} \qquad 
  \int_0^1 \frac{\dd t}{\sqrt{1-t}} \qquad 
  \int_0^1 \ln \frac1t \;\dd t \qquad
  \int_0^1 t\sin \frac1t \;\dd t$$
  
  \item Pour quelles valeurs de $\beta \in \Rr$ l'intégrale
  $\int_0^1 \frac{\dd t}{t(-\ln t)^\beta}$ converge ?
  
  \item Prouver le théorème de comparaison et le théorème des équivalents de cette section, 
  en vous inspirant des démonstrations des sections précédentes.
 
  \item Même travail pour montrer qu'une intégrale absolument convergente est convergente.

\end{enumerate}
\end{miniexercices}




%%%%%%%%%%%%%%%%%%%%%%%%%%%%%%%%%%%%%%%%%%%%%%%%%%%%%%%%%%%%%%%%
\section{Intégration par parties -- Changement de variable}

%---------------------------------------------------------------
\subsection{Intégration par parties}



\begin{theoreme}
Soient $u$ et $v$ deux fonctions de classe $\mathcal{C}^1$ sur l'intervalle $[a,+\infty[$.
Supposons que $\lim_{t\to+\infty} u(t)v(t)$ existe et soit finie.
Alors les intégrales $\displaystyle\int_a^{+\infty} u(t) \, v'(t)\;\dd t$
et $\displaystyle\int_a^{+\infty} u'(t) \, v(t)\;\dd t$ sont de même nature.
En cas de convergence on a :
$$\int_a^{+\infty} u(t) \, v'(t)\;\dd t 
= \big[uv\big]_a^{+\infty} - \int_a^{+\infty} u'(t) \, v(t)\;\dd t$$
\end{theoreme}

On rappelle que $\big[uv\big]_a^{+\infty} = \lim_{t\to+\infty} (uv)(t) - (uv)(a)$.

Le mieux n'est pas d'appliquer le théorème, mais d'effectuer la preuve à chaque fois,
c'est-à-dire en faisant une intégration par parties sur l'intervalle $[a,x]$ et
en vérifiant bien que les objets ont une limite lorsque $x\to+ \infty$.

\begin{proof}
C'est la formule usuelle d'intégration par parties 
$$\int_a^{x} u(t) \, v'(t)\;\dd t 
= \big[uv\big]_a^{x} - \int_a^{x} u'(t) \, v(t)\;\dd t$$
en notant que par hypothèse le crochet a une limite finie lorsque $x\to +\infty$.
\end{proof}


\begin{exemple}
Soit $\lambda > 0$. Que vaut l'espérance de la loi exponentielle :
$$\int_0^{+\infty} \lambda t e^{-\lambda t}\;\dd t \quad \text{?}$$

On effectue l'intégration par parties avec $u = \lambda t$, $v' = e^{-\lambda t}$.
On a donc $u' = \lambda$ et $v = \frac{-1}{\lambda}e^{-\lambda t}$.
Ainsi
\begin{align*}
\int_0^{x} \lambda t e^{-\lambda t}\;\dd t 
 & = \int_0^{x} u(t) \, v'(t)\;\dd t  \\
 & = \big[uv\big]_0^{x} - \int_0^{x} u'(t) \, v(t)\;\dd t \\
 & = \Big[ \lambda t \cdot \frac{-1}{\lambda}e^{-\lambda t}\Big]_0^{x} 
 - \int_0^{x} \lambda \cdot  \frac{-1}{\lambda}e^{-\lambda t}\;\dd t \\
 & = -xe^{-\lambda x} + \int_0^{x} e^{-\lambda t} \;\dd t\\
 & = -xe^{-\lambda x} + \Big[\frac{-1}{\lambda}e^{-\lambda t}\Big]_0^{x} \\
 & = -xe^{-\lambda x}-\frac{1}{\lambda} \left(e^{-\lambda x} -1\right) \\
 & \longrightarrow \ \  \frac{1}{\lambda} \qquad \text{ lorsque } x \to +\infty \\
\end{align*}
Ainsi l'intégrale converge et 
$$\int_0^{+\infty} \lambda t e^{-\lambda t}\;\dd t = \frac{1}{\lambda}.$$
\end{exemple}



%---------------------------------------------------------------
\subsection{Changement de variable}

\begin{theoreme}
Soit $f$ une fonction définie sur un intervalle $I = [a,+\infty[$. Soit
$J= [\alpha,\beta[$ un intervalle avec $\alpha\in \Rr$ et $\beta\in \Rr$ ou $\beta=+\infty$. 
Soit $\varphi : J \to I$ un difféomorphisme de classe $\mathcal{C}^1$. 
Les intégrales $\int_{a}^{+\infty} f(x) \;\dd x$ et
$\int_\alpha^\beta f\big(\varphi(t)\big)\cdot\varphi'(t) \;\dd t$ sont de même nature.
En cas de convergence, on a :
$$\int_{a}^{+\infty} f(x) \;\dd x = \int_\alpha^\beta f\big(\varphi(t)\big)\cdot\varphi'(t) \;\dd t$$
\end{theoreme} 

La démonstration est la même que le changement de variable usuel.
Encore une fois, le mieux n'est pas d'appliquer le théorème mais 
d'effectuer un changement de variable classique sur l'intervalle $[a,x]$,
puis d'étudier les limites lorsque $x\to +\infty$.


On rappelle que $\varphi : J \to I$ un \defi{difféomorphisme de classe $\mathcal{C}^1$}
si $\varphi$ est une application $\mathcal{C}^1$, bijective, dont la bijection réciproque est
aussi $\mathcal{C}^1$.


\medskip 

L'exemple suivant est particulièrement intéressant :
la fonction $f(t) = \sin(t^2)$ a une intégrale convergente,
mais ne tend pas vers $0$ (lorsque $t\to+\infty$). 
C'est à mettre en opposition avec le cas des séries : 
pour une série convergente le terme général tend toujours vers $0$.

\begin{exemple}
L'intégrale de Fresnel 
$$\int_1^{+\infty} \sin (t^2) \; \dd t \quad \text{converge.}$$

\myfigure{1.2}{
\tikzinput{fig_intimp06} 
} 

On effectue le changement de variable $u=t^2$, qui induit 
$t=\sqrt{u}$,  $\dd t=\frac{\dd u}{2\sqrt u}$.
$\varphi : u \mapsto t = \sqrt{u}$ est un difféomorphisme 
  entre $u\in[1,x^2]$ et $t\in[1,x]$.
D'où
$$\int_1^x \sin (t^2) \; \dd t = \int_1^{x^2} \sin (u) \; \frac{\dd u}{2\sqrt u}.$$




Or par le théorème d'Abel $\int_1^{+\infty} \frac{\sin u}{\sqrt u} \; \dd u$
converge, donc $\int_1^{x^2} \sin (u) \; \frac{\dd u}{2\sqrt u}$ admet une limite finie (lorsque 
$x\to+\infty$), ce qui prouve que $\int_1^x \sin (t^2) \; \dd t$
admet aussi une limite finie. Conclusion : $\int_1^{+\infty} \sin (t^2) \; \dd t$ converge.



\end{exemple}


\begin{exemple}
Calculons la valeur des deux intégrales 
$$I = \int_0^{\frac\pi2} \ln(\sin t) \;\dd t \qquad J = \int_0^{\frac\pi2} \ln(\cos t) \;\dd t.$$

\begin{enumerate}
  \item \textbf{L'intégrale $I$ converge.}
  
  Le point incertain est en $t=0$. Comme $\sin t \underset{0^+}{\sim} t$,
  $\ln t \le \frac{1}{\sqrt t}$ (pour $t$ assez petit), et l'intégrale 
  $\int_0^{\frac\pi2} \frac{1}{\sqrt t} \;\dd t$ converge,
  alors l'intégrale $\int_0^{\frac\pi2} \ln t \;\dd t$ converge, 
  ce qui implique que $I$ converge.
  
  \item \textbf{Vérifions $I=J$.}
  
  Effectuons le changement de variable $t=\frac\pi2 - u$. On a $\dd t = - \dd u$
  et un difféomorphisme entre $t\in[x,\frac\pi2]$ et $u\in[\frac\pi2-x,0]$.
  Ainsi
  $$\int_x^{\frac\pi2} \ln(\sin t) \;\dd t
  = \int_{\frac\pi2-x}^0 \ln\left(\sin \left(\frac\pi2 - u\right)\right) \big(\;-\dd u\big)
  = \int_0^{\frac\pi2-x} \ln(\cos u) \;\dd u.$$
  Ainsi, lorsque $x\to 0$, cela prouve $I=J$ (et en particulier $J$ converge).
  
  \item \textbf{Calcul de $I+J$.}
  
  \begin{align*}
  I+J  
  & = \int_0^{\frac\pi2} \ln(\sin t) \;\dd t + \int_0^{\frac\pi2} \ln(\cos t) \;\dd t \\
  & = \int_0^{\frac\pi2} \big(\ln(\sin t)  + \ln(\cos t)\big) \;\dd t \\
  & = \int_0^{\frac\pi2} \ln(\sin t \cdot \cos t) \;\dd t \\
  & = \int_0^{\frac\pi2} \ln\big(\tfrac12 \sin(2t)\big) \;\dd t \\
  & =  -\frac\pi2\ln 2 + \int_0^{\frac\pi2}\ln\big(\sin(2t)\big) \;\dd t \\
  \end{align*}
  Et comme $I=J$, on a 
  $$2I = -\frac\pi2\ln 2 + K.$$
    Il nous reste à évaluer $K = \int_0^{\frac\pi2} \ln\big(\sin(2t)\big) \;\dd t$ :
  \begin{align*}
  K 
  & = \int_0^{\frac\pi2} \ln\big(\sin(2t)\big) \;\dd t \\
  & = \frac12 \int_0^{\pi} \ln(\sin u) \;\dd u \qquad \text{(changement de variable $u = 2t$)} \\
  & = \frac12 I  + \frac12 \int_{\frac\pi2}^\pi \ln(\sin u) \;\dd u \\
  & = \frac12 I  + \frac12 \int_{\frac\pi2}^0 \ln\big(\sin (\pi-v)\big) \; (- \dd v)
      \qquad \text{(changement de variable $v = \pi - u$)} \\
  & = \frac12 I  + \frac12 \int_0^{\frac\pi2} \ln(\sin v) \; \dd v \\
  & = \frac12 I +\frac 12 I \\
  & = I \\
  \end{align*}
  
 \item \textbf{Conclusion.}
 
 Ainsi comme $2I = -\frac\pi2\ln 2 + K$ et $K=I$ on trouve :
 $$I = \int_0^{\frac\pi2} \ln(\sin t) \;\dd t= -\frac\pi2\ln 2$$
 et $J=I$.
  
  
\end{enumerate}


\end{exemple}


%---------------------------------------------------------------
\subsection{Plan d'étude}
%
Nous résumons ici l'ensemble des techniques vues dans ce chapitre
pour l'étude de l'intégrale d'une fonction $f$ sur un intervalle
non borné de $\Rr$, la fonction étant éventuellement non
bornée au voisinage d'un ou plusieurs points de l'intervalle. Pour
illustrer le plan d'étude, nous détaillerons l'exemple introductif :

\begin{minipage}{0.39\textwidth}
$$I = \int_{-\infty}^{+\infty} \frac{\sin |t|}{|t|^{3/2}}\;\dd t$$  
\end{minipage}
\begin{minipage}{0.59\textwidth}
\myfigure{0.6}{
\tikzinput{fig_intimp01} 
}  
\end{minipage}




\begin{enumerate}
\item \textbf{Identifier le ou les points incertains.}\\
La fonction $|t|^{-3/2}\sin |t|$ est définie sur deux intervalles $]-\infty,0[$ et $]0,+\infty[$.
La fonction est paire, elle tend vers $0$ en
oscillant quand $t$ tend vers $-\infty$ et $+\infty$, elle n'est pas définie en $0$ et
tend vers $+\infty$ en $0^-$ et en $0^+$ (voir la figure). 
Il y a donc $4$ points incertains à étudier.
  
\item \textbf{Isoler les points incertains.}\\
Pour cela, il faut découper chaque intervalle d'étude en autant de
sous-intervalles qu'il y a de points incertains, de manière à ce
que les problèmes soient tous situés sur une borne de chaque
intervalle. Dans notre exemple, on divisera en 4 intervalles :
$$
I_1 = \int_{-\infty}^{-1} |t|^{-3/2} \sin |t| \;\dd t\;,\quad
I_2 = \int_{-1}^{0} |t|^{-3/2} \sin |t| \;\dd t\;,
$$
$$
I_3 = \int_{0}^{1} |t|^{-3/2} \sin |t| \;\dd t\;,\quad
I_4 = \int_{1}^{+\infty} |t|^{-3/2} \sin |t| \;\dd t.
$$
Rappelons que le choix des points de découpe, ici $-1$ ici $+1$, n'a pas d'importance
pour la convergence. Chacune des intégrales obtenues doit être étudiée
séparément. L'intégrale $I$ n'est définie que si chacun des
morceaux converge.

\item \textbf{Se ramener à une intégrale sur $[a,+\infty[$ ou sur $]a,b]$.}\\
Pour cela, il suffit d'effectuer le changement de variable 
$t\mapsto -t$. Dans notre cas, puisque la fonction est paire,
$I_1=I_4$ et $I_2=I_3$.

\item \textbf{Calculer une primitive si c'est possible.}\\
Ayant une primitive, le problème est ramené à un calcul de
limite. Si on n'a pas de primitive explicite, alors :

\item \textbf{Si la fonction est de signe constant.}\\
Changer éventuellement le signe pour se ramener à une fonction positive.
Calculer un équivalent au voisinage du point incertain et utiliser
le théorème \ref{th:equivalentintegrales1} ou 
\ref{th:equivalentintegrales2} des équivalents. Si l'équivalent ne donne pas la
réponse directement, utiliser le théorème \ref{th:comparaisonintegrales1}
ou \ref{th:comparaisonintegrales2} de comparaison.
Dans notre exemple, l'intégrale $I_3$ est celle d'une fonction
positive, tendant vers $+\infty$ en $0^+$ :
$$
t^{-3/2} \sin |t|  \;\underset{0^+}{\sim}\; t^{-1/2}\;.
$$
Or l'intégrale $\int_0^1 \frac{1}{t^{1/2}}\;\dd t$ converge, donc $I_3$
converge.

\item \textbf{Si la fonction n'est pas de signe constant.}\\
Commencer par étudier l'intégrale de $|f|$, comme dans le cas
précédent (équivalent ou comparaison). Si elle converge,
l'intégrale étudiée est absolument convergente, donc
convergente. Si l'intégrale n'est pas absolument convergente, il
faut essayer de mettre la fonction sous forme d'un produit pour
appliquer le théorème d'Abel (théorème \ref{th:abelintcv1}). Dans notre
exemple, l'intégrale $I_4$ est absolument convergente : on le
déduit du théorème de comparaison, car
$$
t^{-3/2} \big|\sin |t| \big|\le t^{-3/2}\;, 
$$
et l'intégrale de Riemann $\int_1^{+\infty} \frac{1}{t^{3/2}}\;\dd t$ est
convergente. On pourrait aussi appliquer le théorème d'Abel
avec $f(t) = t^{-3/2}$ et $g(t) = \sin t$.
\end{enumerate}


\begin{miniexercices} 
\begin{enumerate}
  \item Soit $I_n = \int_0^{+\infty} t^n e^{-t}\;\dd t$. 
  Calculer $I_0$. Par intégration par parties, trouver une relation de 
  récurrence entre $I_{n+1}$ et $I_n$. En déduire que $I_n = n!$.
  
  \item Par intégration par parties, montrer que l'intégrale
  $\int_0^1 \frac{\ln t}{(1+t)^2}\; \dd t$ converge et vaut $-\ln 2$.
  
  \item Soit $I= \int_0^{+\infty} \ln(1+\frac{1}{t^2}) \;\dd t$. Identifier les points incertains.
  Faire une intégration par parties en écrivant $1\cdot \ln(1+\frac{1}{t^2})$. Montrer que $I$ converge
  et que $I = \pi$. 
  
  \item Soit $I=\int_0^1 \left(-\ln t\right)^\beta \;\dd t$.
  Identifier les points incertains. 
  \`A l'aide du changement de variable $t=\frac 1u$, montrer que cette intégrale
  converge quel que soit $\beta \in \Rr$.
  
  \item Soit $I=\int_0^{+\infty}\frac{\dd t}{t^2+\sqrt t}$.
  Identifier les points incertains. 
  \`A l'aide du changement de variable $t=u^2$, montrer que l'intégrale
  converge et que $I = \frac{4\sqrt3\pi}{9}$.


\end{enumerate}
\end{miniexercices}





\auteurs{
\begin{itemize}
  \item[$\bullet$] D'après un cours de Luc Rozoy et Bernard Ycart de l'université de Grenoble
  pour le site \texttt{\href{http://ljk.imag.fr/membres/Bernard.Ycart/mel/}{M\at ths en Ligne}}.
 
  \item[$\bullet$] et un cours de Raymond Mortini, de l'université de Lorraine,

  \item[$\bullet$] mixé, révisé par Arnaud Bodin. Relu par Stéphanie Bodin et Vianney Combet.
\end{itemize}
}

\finchapitre

\end{document}


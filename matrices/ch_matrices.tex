\documentclass[class=report,crop=false]{standalone}
\usepackage[screen]{../exo7book}

\begin{document}

%====================================================================
\chapitre{Matrices}
%====================================================================

\insertvideo{Sdg9O-HqSN0}{partie 1. Définition}

\insertvideo{rxz6nYDwE1Q}{partie 2. Multiplication de matrices}

\insertvideo{a5XQKEQnf_Y}{partie 3. Inverse d'une matrice : définition}

\insertvideo{DalK5o-DHB4}{partie 4. Inverse d'une matrice : calcul}

\insertvideo{G6R50olUptI}{partie 5. Inverse d'une matrice : systèmes linéaires et matrices élémentaires}

\insertvideo{IWqJaBkENoQ}{partie 6. Matrices triangulaires, transposition, trace, matrices symétriques}

\insertfiche{fic00157.pdf}{Calculs sur les matrices}

\bigskip
\bigskip
\bigskip


Les matrices sont des tableaux de nombres. La résolution d'un
certain nombre de problèmes d'algèbre linéaire se ramène
à des manipulations sur les matrices. Ceci est vrai en particulier
pour la résolution des systèmes linéaires.

\medskip

\noindent{\bf Dans ce chapitre, $\Kk$ désigne un corps. On peut penser à $\Qq$, $\Rr$ ou $\Cc$.}


%%%%%%%%%%%%%%%%%%%%%%%%%%%%%%%%%%%%%%%%%%%%%%%%%%%%%%%%%%%%%%%%
\section{Définition}


%---------------------------------------------------------------
\subsection{Définition}

\begin{definition}
\sauteligne
\begin{itemize}
  \item Une \defi{matrice}\index{matrice} $A$ est un tableau rectangulaire
d'éléments de $\Kk$.
  \item Elle est dite de \defi{taille} $n \times p$ si le tableau
possède $n$ lignes et $p$ colonnes.
  \item Les nombres du tableau sont appelés les \defi{coefficients} de $A$.
  \item Le coefficient situé à  la $i$-ème ligne et à la $j$-ème colonne
  est noté $a_{i,j}$.
\end{itemize}
\end{definition}


 Un tel tableau est représenté de la manière suivante :
$$A=\begin{pmatrix}
a_{1,1}& a_{1,2}& \dots & a_{1,j}& \dots & a_{1,p}\cr
a_{2,1}& a_{2,2}& \dots & a_{2,j}& \dots &a_{2,p}\cr
\dots & \dots & \dots & \dots & \dots & \dots \cr
a_{i,1}& a_{i,2} & \dots & a_{i,j}& \dots &a_{i,p} \cr
\dots & \dots & \dots & \dots & \dots & \dots \cr
a_{n,1}& a_{n,2}& \dots & a_{n,j}& \dots & a_{n,p}\cr
\end{pmatrix}
\quad \text{ ou } \quad
A= \big(a_{i,j}\big)_{\substack{1\leq i \leq n \\ 1\leq j \leq p}}
\quad \text{ ou } \quad
\big(a_{i,j}\big).
$$

\begin{exemple}
$$ A  =  \left(
 \begin{array}{ccc}
1 & -2 & 5\\
0 & 3  & 7
\end{array}
\right)$$
est une matrice $2\times 3$ avec, par exemple, $a_{1,1}=1$ et $a_{2,3}=7$.
 \end{exemple}

Encore quelques définitions :
\begin{definition}
\sauteligne
\begin{itemize}

  \item Deux matrices sont \defi{égales} lorsqu'elles ont la même taille
  et que les coefficients correspondants sont égaux.

  \item L'ensemble des matrices à $n$ lignes et $p$ colonnes à
coefficients dans $\Kk$ est noté $M_{n,p}(\Kk)$.
Les éléments de $M_{n,p}(\mathbb{R})$ sont appelés
\defi{matrices réelles}.
\end{itemize}
\end{definition}




%---------------------------------------------------------------
\subsection{Matrices particulières}

Voici quelques types de matrices intéressantes :

\begin{itemize}
  \item Si $n=p$ (même nombre de lignes que de colonnes), la matrice est dite 
  \defi{matrice carrée}\index{matrice!carrée}.
  On note $M_{n}(\Kk)$ au lieu de $M_{n,n}(\Kk)$.

  \[
\begin{pmatrix}
 {\color{myred}a_{1,1}} & a_{1,2} & \dots & a_{1,n}\\
 a_{2,1} & {\color{myred}a_{2,2}} & \dots & a_{2,n}\\
 \vdots& \vdots & {\color{myred}\ddots}  & \vdots\\
 a_{n,1} & a_{n,2} & \dots & {\color{myred}a_{n,n}}
\end{pmatrix}
 \]

 Les éléments $a_{1,1}, a_{2,2}, \ldots, a_{n,n}$ forment la \defi{diagonale principale}\index{diagonale}
 de la matrice.

  \item  Une matrice qui n'a qu'une seule ligne ($n=1$) est appelée \defi{matrice ligne}
  ou \defi{vecteur ligne}. On la note
$$A=\begin{pmatrix}
a_{1,1}& a_{1,2}&  \ldots & a_{1,p}\cr
\end{pmatrix}.$$

  \item De même, une matrice qui n'a qu'une seule colonne ($p=1$) est appelée \defi{matrice
colonne} ou \defi{vecteur colonne}. On la note
$$A=\begin{pmatrix}
a_{1,1}\cr
a_{2,1}\cr
\enspace \vdots \hfill \cr
a_{n,1}\cr
\end{pmatrix}.$$


  \item La matrice (de taille $n\times p$) dont tous les coefficients sont des zéros
est appelée la \defi{matrice nulle} et est notée $0_{n,p}$ ou plus simplement $0$.
Dans le calcul matriciel, la matrice nulle joue le rôle du nombre $0$ pour les réels.

\end{itemize}


%---------------------------------------------------------------
\subsection{Addition de matrices}


\begin{definition}[Somme de deux matrices]
Soient $A$ et $B$ deux matrices ayant la même taille $n\times p$.
Leur \defi{somme} $C=A+B$ est la matrice de taille $n\times p$ définie par
\[c_{ij}=a_{ij}+b_{ij}.\]
\end{definition}

En d'autres termes, on somme coefficients par coefficients.
Remarque : on note indifféremment $a_{ij}$ où $a_{i,j}$
pour les coefficients de la matrice $A$.


\begin{exemple}
$$\text{Si} \qquad
A  =  \begin{pmatrix}
3 & -2\\
1 & 7
\end{pmatrix}
\qquad \text{et} \qquad
B = \begin{pmatrix}
0 & 5 \\
2 & -1
    \end{pmatrix}
\qquad \text{alors} \qquad A + B = \begin{pmatrix}
3 & 3\\
3 & 6
\end{pmatrix}.
$$
$$
\text{Par contre si } \qquad
B' =  \begin{pmatrix}
-2\\8  \end{pmatrix}
\qquad \text{ alors } \quad A+B' \quad \text{ n'est pas définie.}
$$
\end{exemple}


\begin{definition}[Produit d'une matrice par un scalaire]
Le produit d'une matrice $A=\big(a_{ij}\big)$ de $M_{n,p}(\Kk)$
par un scalaire $\alpha \in \Kk$ est la matrice
$\big(\alpha a_{ij}\big)$ formée en
multipliant chaque coefficient de $A$ par $\alpha$. Elle est notée $\alpha \cdot A$ (ou simplement $\alpha A$).
\end{definition}

\begin{exemple}
$$
\text{Si} \qquad
A  = \begin{pmatrix}
1& 2 & 3\cr
0& 1& 0\cr
\end{pmatrix}
\qquad \text{et} \qquad \alpha = 2
\qquad \text{alors} \qquad
\alpha A =
\begin{pmatrix}
2& 4 & 6\cr
0& 2& 0\cr
\end{pmatrix}.$$
\end{exemple}



La matrice $(-1)A$ est l'\defi{opposée} de $A$ et est notée $-A$.
La \defi{différence} $A-B$ est définie par $A + (-B)$.

\begin{exemple}
$$\text{Si} \quad A =  \begin{pmatrix}
2 & -1 & 0\\
4 & -5 & 2
\end{pmatrix}
\quad \text{et} \quad
B  =  \begin{pmatrix}
-1 & 4 & 2\\
7 & -5 & 3
\end{pmatrix}
\quad \text{alors} \quad
A-B  = \begin{pmatrix}
3 & -5 & -2\\
-3 & 0 & -1
\end{pmatrix}.$$
\end{exemple}

L'addition et la multiplication par un scalaire se comportent sans surprises :
\begin{proposition}
Soient $A$, $B$ et $C$ trois matrices appartenant à $M_{n,p}(\Kk)$.
Soient $\alpha \in \Kk$ et $\beta \in \Kk$ deux scalaires.
\begin{enumerate}
  \item $A + B = B + A$ : la somme est commutative,
  \item $A + (B+C) = (A + B) + C$ : la somme est associative,
  \item $A + 0 = A$ : la matrice nulle est l'élément neutre de l'addition,
  \item $(\alpha + \beta )A =\alpha A + \beta A$,
  \item $\alpha (A+B)=\alpha A + \alpha B$.
\end{enumerate}
\end{proposition}

\begin{proof}
Prouvons par exemple le quatrième point.
Le terme général de $(\alpha + \beta ) A $ est égal à
$(\alpha + \beta)a_{ij}$. D'après les règles de calcul dans $\Kk$,
$(\alpha + \beta)a_{ij}$ est égal à  $\alpha a_{ij}+ \beta a_{ij}$
qui est le terme général de la matrice $\alpha A + \beta A$.
\end{proof}

%---------------------------------------------------------------
%\subsection{Mini-exercices}

\begin{miniexercices}
\sauteligne
\begin{enumerate}
  \item Soient
  $A=\left(\begin{smallmatrix}-7&2\\0&-1\\1&-4\end{smallmatrix}\right)$,
  $B=\left(\begin{smallmatrix}1&2&3\\2&3&1\\3&2&1\end{smallmatrix}\right)$,
  $C=\left(\begin{smallmatrix}21&-6\\0&3\\-3&12\end{smallmatrix}\right)$,
  $D=\frac12\left(\begin{smallmatrix}1&0&1\\0&1&0\\1&1&1\end{smallmatrix}\right)$,
  $E=\left(\begin{smallmatrix}1&2\\-3&0\\-8&6\end{smallmatrix}\right)$.
  Calculer toutes les sommes possibles de deux de ces matrices.
  Calculer $3A+2C$ et $5B-4D$. Trouver $\alpha$ tel que $A-\alpha C$ soit la matrice nulle.

  \item Montrer que si $A+B=A$, alors $B$ est la matrice nulle.

  \item Que vaut $0\cdot A$ ? et $1\cdot A$ ? Justifier l'affirmation : $\alpha(\beta A) = (\alpha\beta)A$.
  Idem avec $nA = A+A+\cdots+A$ ($n$ occurrences de $A$).

\end{enumerate}
\end{miniexercices}


%%%%%%%%%%%%%%%%%%%%%%%%%%%%%%%%%%%%%%%%%%%%%%%%%%%%%%%%%%%%%%%%
\section{Multiplication de matrices}

%---------------------------------------------------------------
\subsection{Définition du produit}


Le produit $AB$ de deux matrices $A$ et $B$ est défini si et seulement si le nombre de colonnes de
$A$ est égal au nombre de lignes de $B$.

\begin{definition}[Produit de deux matrices]
\index{matrice!produit}
Soient $A=(a_{ij})$ une matrice $n\times p$ et $B=(b_{ij})$ une matrice $p\times q$.
Alors le produit $C=AB$ est une matrice $n\times q$ dont les coefficients $c_{ij}$
sont définis par :

\mybox{$\displaystyle
c_{ij} = \sum_{k=1}^p a_{ik}b_{kj}
$}
\end{definition}

On peut écrire le coefficient de façon plus développée, à savoir :
$$c_{ij}=a_{i1}b_{1j}+a_{i2}b_{2j}+ \dots +
a_{ik}b_{kj}+ \dots + a_{ip}b_{pj}.$$


Il est commode de disposer les calculs de la
façon suivante.

$$\begin{array}{ccl}
&\begin{pmatrix}
&&&{\color{myred}\times}&&\\
&&&{\color{myred}\times}&&\\
\hphantom{-}&\hphantom{-}&\hphantom{-}&{\color{myred}\times}&\hphantom{-}&\hphantom{-}\\
&&&{\color{myred}\times}&&
\end{pmatrix}&\leftarrow B\\
A\to\begin{pmatrix}
&&&\\
&&&\\
{\color{myred}\times}&{\color{myred}\times}&{\color{myred}\times}&{\color{myred}\times}\\
&&&
\end{pmatrix}
&\begin{pmatrix}
&&&|&&\\
&&&|&&\\
-&-&-&{\color{blue}c_{ij}}&\hphantom{-}&\hphantom{-}\\
&&&&&
\end{pmatrix}&\leftarrow AB\\
\end{array}
$$


Avec cette disposition, on considère d'abord la ligne
de la matrice $A$ située à gauche du coefficient que l'on veut
calculer (ligne représentée par des $\times $ dans $A$)
et aussi la colonne de la matrice $B$ située au-dessus du coefficient
que l'on veut calculer (colonne représentée par des ${\color{myred}\times}$ dans $B$).
On calcule le produit du premier coefficient de la ligne par le premier coefficient
de la colonne ($a_{i1} \times b_{1j}$), que l'on ajoute au produit du deuxième coefficient de la ligne par le deuxième coefficient
de la colonne ($a_{i2} \times b_{2j}$), que l'on ajoute au produit du troisième\ldots



%---------------------------------------------------------------
\subsection{Exemples}

\begin{exemple}
$$A =\begin{pmatrix}
1 & 2 & 3\cr
2 & 3 & 4\cr
\end{pmatrix}\qquad B =
\begin{pmatrix}
1&2\cr
-1&1 \cr
1&1\cr
\end{pmatrix}
$$


On dispose d'abord le produit correctement (à gauche) : la matrice obtenue
est de taille $2\times2$.
Puis on calcule chacun des coefficients,
en commençant par le premier coefficient $c_{11} = 1\times 1\ +\ 2\times(-1)\ +\ 3\times1=2$ (au milieu),
puis les autres (à droite).

$$\begin{array}{cc}
  &  \begin{pmatrix}
1&2\cr
-1&1 \cr
1&1\cr
\end{pmatrix} \\
\begin{pmatrix}
1 & 2 & 3\cr
2 & 3& 4\cr
\end{pmatrix}
   &\!\!
 \begin{pmatrix} c_{11} & c_{12} \\ c_{21} & c_{22} \end{pmatrix}
  \end{array}
   \quad
 \begin{array}{cc}
  & \begin{pmatrix}
{\color{myred}1}&2\cr
{\color{myred}-1}&1 \cr
{\color{myred}1}&1\cr
\end{pmatrix}  \\
\begin{pmatrix}
{\color{myred}1} & {\color{myred}2} & {\color{myred}3}\cr
2 & 3& 4\cr
\end{pmatrix}
   &\!\! \begin{pmatrix} {\color{blue}2} & c_{12} \\ c_{21} & c_{22} \end{pmatrix}
  \end{array}
  \quad
 \begin{array}{cc}
  &
\begin{pmatrix}
1&2\cr
-1&1 \cr
1&1\cr
\end{pmatrix}\\
\begin{pmatrix}
1 & 2 & 3\cr
2 & 3& 4\cr
\end{pmatrix}
   &\!\! \begin{pmatrix} 2 & 7 \cr 3 & 11 \end{pmatrix}
  \end{array}
 $$


\end{exemple}

\bigskip

Un exemple intéressant est le produit d'un vecteur ligne par un vecteur colonne :
$$u = \begin{pmatrix} a_1 & a_2 & \cdots & a_n \end{pmatrix} \quad
v = \begin{pmatrix} b_1 \\ b_2 \\ \vdots \\ b_n \end{pmatrix}$$
Alors $u \times v$ est une matrice de taille $1\times 1$ dont l'unique coefficient est
$a_1 b_1 + a_2 b_2 + \cdots + a_n b_n$.
Ce nombre s'appelle le \defi{produit scalaire}\index{produit scalaire} des vecteurs $u$ et $v$.

\medskip

Calculer le coefficient $c_{ij}$ dans le produit $A\times B$
revient donc à calculer le produit scalaire des vecteurs formés
par la $i$-ème ligne de $A$ et la $j$-ème colonne de $B$.


%---------------------------------------------------------------
\subsection{Pièges à éviter}

\textbf{Premier piège. Le produit de matrices n'est pas commutatif en général.}

En effet, il se peut que $AB$ soit défini mais pas $BA$, ou que $AB$ et $BA$
soient tous deux définis mais pas de la même taille.
Mais même dans le cas où $AB$ et $BA$ sont définis et de la même taille, on a en général $AB\neq BA$.

\begin{exemple}
$$\begin{pmatrix}
5&1\\3&-2
\end{pmatrix}
\begin{pmatrix}
2&0\\4&3
\end{pmatrix}=
\begin{pmatrix}
14&3\\-2&-6
\end{pmatrix}
\qquad \text{ mais } \qquad
\begin{pmatrix}
2&0\\4&3
\end{pmatrix}
\begin{pmatrix}
5&1\\3&-2
\end{pmatrix}=
\begin{pmatrix}
10&2\\29&-2
\end{pmatrix}.
$$
\end{exemple}

\bigskip

\textbf{Deuxième piège. $AB=0$ n'implique pas $A=0$ ou $B=0$.}

Il peut arriver que le produit de deux matrices non nulles soit nul.
En d'autres termes, on peut avoir $A \neq 0$ et $B \neq 0$ mais $AB  =  0$.
\begin{exemple}
$$
A  = \begin{pmatrix}
 0 & -1\\
 0 & 5\end{pmatrix}\qquad B  =
\begin{pmatrix}
2 & -3\\
0 & 0 \end{pmatrix}
\qquad \text{et} \qquad
 AB  =  \begin{pmatrix}
 0 & 0\\
 0 & 0\end{pmatrix}.
$$
\end{exemple}

\bigskip


\textbf{Troisième piège. $AB=AC$ n'implique pas $B=C$.}
On peut avoir $AB  = AC$ et $B \neq C$.

\begin{exemple}
$$
 A  =  \begin{pmatrix}
 0 & -1\\
 0 & 3\end{pmatrix}
 \qquad
 B  = \begin{pmatrix}
4 & -1\\
5 & 4 \end{pmatrix}
 \qquad
 C  = \begin{pmatrix}
 2 & 5\\
 5 & 4\end{pmatrix}
 \quad  \text{et}  \quad
 AB  = AC = \begin{pmatrix}
  -5 & -4\\
  15 & 12\end{pmatrix}.
$$
\end{exemple}

%---------------------------------------------------------------
\subsection{Propriétés du produit de matrices}

Malgré les difficultés soulevées au-dessus, le produit vérifie les propriétés suivantes :
\begin{proposition}
\sauteligne
\begin{enumerate}
  \item $A (BC) = (AB) C$ : associativité du produit,

  \item $A(B+C) = AB + AC$ \quad et \quad  $(B+C) A = BA + CA$ : distributivité du produit par rapport à la somme,

  \item $A\cdot 0 = 0$ \quad et \quad $0\cdot A= 0$.
\end{enumerate}
\end{proposition}


\begin{proof}
Posons $A=(a_{ij}) \in M_{n,p}(\Kk)$, $B=(b_{ij})\in M_{p,q}(\Kk)$
et $C=(c_{ij})\in M_{q,r}(\Kk)$.
Prouvons que $A(BC) = (AB) C$ en montrant que les matrices $A(BC)$ et $(AB) C$
ont les mêmes coefficients.

Le terme d'indice $(i,k)$ de la matrice $AB$ est
$x_{ik}={\displaystyle \sum_{\ell=1}^{p}}a_{i \ell}b_{\ell k}$.
Le terme d'indice $(i,j)$ de la matrice $(AB)C$ est donc
$$ \sum_{k=1}^{q}x_{ik}c_{kj}=\sum_{k=1}^{q}
\left ( \sum_{\ell=1}^{p}a_{i\ell}b_{\ell k} \right )c_{kj}.$$
Le terme d'indice $(\ell,j)$ de la matrice $BC$ est
$y_{\ell j}={\displaystyle \sum_{k=1}^{q}}b_{\ell k}c_{kj}$.
Le terme d'indice $(i,j)$ de la matrice $A(BC)$ est donc
$$\sum_{\ell=1}^{p}a_{i\ell}\left (  \sum_{k=1}^{q}b_{\ell k}c_{kj}\right ).$$
Comme dans $\Kk$ la multiplication est distributive et associative,
les coefficients de $(AB)C$ et $A(BC)$ coïncident.
Les autres démonstrations se font comme celle de l'associativité.
\end{proof}

%---------------------------------------------------------------
\subsection{La matrice identité}


La matrice carrée suivante s'appelle la \defi{matrice identité}\index{matrice!identité} :
 \[
 I_n = \left(
 \begin{array}{cccc}
1 & 0 & \dots & 0\\
0& 1& \dots & 0\\
 \vdots& \vdots & \ddots  & \vdots\\
0 & 0 & \dots &1
\end{array}
\right)
 \]

Ses éléments diagonaux sont égaux à $1$ et tous ses autres éléments sont égaux à $0$.
Elle se note $I_n$ ou simplement $I$.
Dans le calcul matriciel, la matrice identité joue un rôle
analogue à celui du nombre $1$ pour les réels.
C'est l'élément neutre pour la multiplication. En
d'autres termes :

\begin{proposition}
Si $A$ est une matrice $n \times p$, alors
$$ I_n \cdot A = A \qquad \text{et} \qquad A \cdot I_p = A.$$
\end{proposition}

\begin{proof}
Nous allons détailler la preuve.
Soit  $A \in M_{n,p}(\Kk)$ de terme général $a_{ij}$.
La matrice unité d'ordre $p$ est telle que tous les éléments de la
diagonale principale sont égaux à $1$, les autres étant tous nuls.

On peut formaliser cela en introduisant le symbole de Kronecker.
Si $i$ et $j$ sont deux entiers, on appelle \defi{symbole de Kronecker}\index{symbole de Kronecker}, et on
note $\delta_{i,j}$, le réel qui vaut $0$ si $i$ est différent de $j$,
et $1$ si $i$ est
égal à $j$. Donc
$$\delta_{i,j}=\begin{cases}
0 & \text{ si } \;\; i\neq j \\
1 & \text{ si } \;\; i=j.
\end{cases}$$

Alors le terme général de la matrice identité $I_{p}$ est $\delta_{i,j}$
avec $i$ et $j$ entiers, compris entre $1$ et $p$.

La matrice produit $AI_{p}$
est une matrice appartenant à $M_{n,p}(\Kk)$
dont le terme général $c_{ij}$ est donné par la formule
$c_{ij}={\displaystyle \sum_{k=1}^{p}a_{ik}\delta_{kj}}$.
Dans cette somme, $i$ et $j$ sont fixés et $k$ prend toutes les valeurs
comprises entre $1$ et $p$. Si $k\neq j$ alors $\delta_{kj}=0$,
et si $k=j$ alors $\delta_{kj}=1$.
Donc dans la somme qui définit $c_{ij}$,
tous les termes correspondant à des valeurs de $k$ différentes de $j$
sont nuls et il reste donc
$c_{ij}=a_{ij}\delta_{jj}=a_{ij}1=a_{ij}$.
Donc les matrices $AI_{p}$ et $A$ ont le même terme général et
sont donc égales.
L'égalité $I_{n}A=A$ se démontre de la même façon.
\end{proof}

%---------------------------------------------------------------
\subsection{Puissance d'une matrice}

Dans l'ensemble $M_{n}(\Kk)$ des matrices carrées de taille $n \times n$ à
coefficients dans $\Kk$,
la multiplication des matrices est une
opération interne : si $A,B \in M_n(\Kk)$ alors $AB \in M_n(\Kk)$.

En particulier, on peut multiplier une matrice carrée par elle-même :
on note $A^2 = A \times A$, $A^3 = A \times A \times A$.


On peut ainsi définir les puissances successives d'une matrice :
\index{matrice!puissance}
\begin{definition}
Pour tout $A\in M_n(\Kk)$,
on définit les puissances successives de $A$ par $A^0=I_n$ et
$A^{p+1}=A^p \times A$ pour tout $p\in\N$.
Autrement dit, $A^p = \underbrace{A \times A \times \cdots \times A}_{p \text{ facteurs}}$.
\end{definition}


\begin{exemple}
On cherche à calculer $A^{p}$ avec
$A=\begin{pmatrix}
1 & 0  & 1 \cr
0 & -1 & 0\cr
0 & 0  & 2 \cr
\end{pmatrix}$.
On calcule $A^{2}$, $A^3$ et $A^{4}$ et on obtient :
$$A^{2}= \begin{pmatrix}
1 & 0 & 3 \cr
0 & 1 & 0\cr
0 & 0 & 4 \cr
\end{pmatrix}
\qquad
A^{3}=A^2 \times A =
\begin{pmatrix}
1 & 0 & 7 \cr
0 & -1& 0\cr
0 & 0 & 8 \cr
\end{pmatrix}
\qquad
A^{4}=A^3 \times A =
\begin{pmatrix}
1 & 0 & 15 \cr
0 & 1& 0\cr
0 & 0 & 16 \cr
\end{pmatrix}.
$$
L'observation de ces premières puissances permet de penser que la formule
est : $A^{p}= \begin{pmatrix}
1 & 0       & 2^p-1 \cr
0 & (-1)^{p}& 0\cr
0 & 0       & 2^p \cr
\end{pmatrix}$.
Démontrons ce résultat par récurrence.

Il est vrai pour $p=0$ (on trouve l'identité).
On le suppose vrai pour un entier $p$ et on va
le démontrer pour $p+1$. On a, d'après la
définition,
$$A ^{p+1}=A ^{p} \times A =
\begin{pmatrix}
1 & 0       & 2^p-1 \cr
0 & (-1)^{p}& 0\cr
0 & 0       & 2^p \cr
\end{pmatrix} \times
\begin{pmatrix}
1 & 0 & 1 \cr
0 & -1& 0\cr
0 & 0 & 2 \cr
\end{pmatrix}
=\begin{pmatrix}
1 & 0       & 2^{p+1}-1 \cr
0 & (-1)^{p+1}& 0\cr
0 & 0       & 2^{p+1} \cr
\end{pmatrix}
.$$
Donc la propriété est démontrée.
\end{exemple}


%---------------------------------------------------------------
\subsection{Formule du binôme}


Comme la multiplication n'est pas commutative, les identités binomiales usuelles sont fausses.
En particulier, $(A+B)^2$ ne vaut en général pas $A^2+2AB+B^2$, mais on sait seulement que
$$(A+B)^2= A^2+{\color{myred}AB+BA}+B^2.$$


\begin{proposition}[Calcul de $(A+B)^{p}$ lorsque $AB=BA$]
\index{formule!du binôme de Newton}
Soient $A$ et $B$ deux éléments de $M_{n}(\Kk)$ qui \defi{commutent}, c'est-à-dire tels que $AB=BA$.
Alors, pour tout entier $p \ge 0$, on a la formule
$$(A+B)^{p}= \sum_{k=0}^{p} \binom{p}{k} A^{p-k}B^{k}$$
où $\binom{p}{k}$ désigne le coefficient du binôme.
\end{proposition}

La démonstration est similaire à celle de la formule du binôme pour $(a+b)^p$, avec $a,b \in \Rr$.


\begin{exemple}
Soit $A=\begin{pmatrix}
1&1&1&1\cr
0 & 1&2&1 \cr
0&0&1&3 \cr
0&0&0&1
\end{pmatrix}$.
On pose
$N=A-I= \begin{pmatrix}
0&1&1&1\cr
0 &0&2&1 \cr
0&0&0&3 \cr
0&0&0&0
\end{pmatrix}$.
La matrice $N$ est nilpotente
(c'est-à-dire il existe $k \in \Nn$ tel que $N^{k}=0$) comme le montrent les
calculs suivants :
$$N^{2}= \begin{pmatrix}
0&0&2&4\cr
0 &0&0&6 \cr
0&0&0&0 \cr
0&0&0&0
\end{pmatrix} \qquad
N^{3}= \begin{pmatrix}
0&0&0&6\cr
0 &0&0&0 \cr
0&0&0&0 \cr
0&0&0&0
\end{pmatrix}\qquad \text{ et } \qquad N^{4}=0.$$
Comme on a $A=I+N$ et les matrices $N$ et $I$ commutent (la matrice identité
commute avec toutes les matrices), on peut
appliquer la formule du binôme de Newton.
On utilise que $I^k=I$ pour tout $k$ et surtout que
$N^{k}=0$ si $k \geq 4$.
On obtient
$$
A^{p} = \sum_{k=0}^{p} \binom{p}{k} N^{k} I^{p-k}
= \displaystyle \sum_{k=0}^{{\color{myred}3}} \binom{p}{k} N^{k}
 = I+pN+\tfrac{p(p-1)}{2!}N^{2}+ \tfrac{p(p-1)(p-2)}{3!}N^{3}.$$
D'où
$$A^{p}= \begin{pmatrix}
1&p&p^{2}&p(p^{2}-p+1)\cr
0 & 1&2p&p(3p-2) \cr
0&0&1&3p \cr
0&0&0&1
\end{pmatrix}.$$
\end{exemple}



%---------------------------------------------------------------
%\subsection{Mini-exercices}

\begin{miniexercices}
\sauteligne
\begin{enumerate}
  \item Soient
  $A=\left(\begin{smallmatrix}0&2&-2\\6&-4&0\end{smallmatrix} \right)$,
  $B=\left(\begin{smallmatrix}2&1&0\\0&1&0\\2&-2&-3\end{smallmatrix} \right)$,
  $C=\left(\begin{smallmatrix}8&2\\-3&2\\-5&5\end{smallmatrix} \right)$,
  $D=\left(\begin{smallmatrix}5\\2\\-1\end{smallmatrix} \right)$,
  $E=\left(\begin{matrix}x&y&z\end{matrix} \right)$.
  Quels produits sont possibles ? Les calculer !

  \item Soient $A= \left(\begin{smallmatrix}0&0&1\\0&1&0\\1&1&2 \end{smallmatrix} \right)$
  et $B=\left(\begin{smallmatrix}1&0&0\\0&0&2\\1&-1&0\end{smallmatrix} \right)$.
  Calculer $A^2$, $B^2$, $AB$ et $BA$.

  \item Soient $A= \left(\begin{smallmatrix} 2&0&0\\0&2&0\\0&0&2\end{smallmatrix} \right)$
  et $B=\left(\begin{smallmatrix}0&0&0\\2&0&0\\3&1&0\end{smallmatrix} \right)$.
  Calculer $A^p$ et $B^p$ pour tout $p\ge0$. Montrer que $AB=BA$. Calculer $(A+B)^p$.

\end{enumerate}
\end{miniexercices}


%%%%%%%%%%%%%%%%%%%%%%%%%%%%%%%%%%%%%%%%%%%%%%%%%%%%%%%%%%%%%%%%
\section{Inverse d'une matrice : définition}

%---------------------------------------------------------------
\subsection{Définition}


\begin{definition}[Matrice inverse]
\index{matrice!inverse}
Soit $A$ une matrice  carrée de taille $n \times n$. S'il existe une matrice carrée
$B$ de taille $n \times n$ telle que
 $$ AB = I\qquad \text{et} \qquad BA = I, $$
 on dit que $A$ est \defi{inversible}. On appelle $B$ l'\defi{inverse de $A$}
 et on la note $A^{-1}$.
\end{definition}

On verra plus tard qu'il suffit en fait de vérifier une seule des
conditions $AB=I$ ou bien $BA=I$.


\bigskip

\begin{itemize}
  \item Plus généralement, quand $A$ est inversible, pour tout $p\in \Nn$, on note :
$$A^{-p}=(A^{-1})^p = \underbrace{A^{-1} A^{-1} \cdots A^{-1}}_{{p \text{ facteurs}}}.$$


  \item L'ensemble des matrices inversibles de $M_{n}(\Kk)$ est noté
$GL_{n}(\Kk)$.
\end{itemize}


%---------------------------------------------------------------
\subsection{Exemples}

\begin{exemple}
Soit $A=\left(\begin{smallmatrix}
1 & 2 \cr
0 & 3 \cr
\end{smallmatrix}\right)$.
\'Etudier si $A$ est inversible, c'est étudier l'existence d'une matrice
$B= \left(\begin{smallmatrix}
a & b \cr
c & d\cr
\end{smallmatrix}\right)$ à coefficients dans $\Kk$, telle que $AB=I$ et $BA=I$.
Or $AB=I$ équivaut à :
$$
AB=I \iff
\begin{pmatrix}
1 & 2 \cr
0 & 3\cr
\end{pmatrix}
\begin{pmatrix}
a & b \cr
c & d\cr
\end{pmatrix}
=\begin{pmatrix}
1 & 0 \cr
0 & 1\cr
\end{pmatrix}
\iff
\begin{pmatrix}
a+2c & b+2d \cr
3c & 3d\cr
\end{pmatrix}
=\begin{pmatrix}
1 & 0 \cr
0 & 1\cr
\end{pmatrix}
$$
Cette égalité équivaut au système  :
$$\left \{ \begin{array}{l}
a+2c=1\\
b+2d=0\\
3c=0\\
3d=1\\
\end{array} \right .$$

Sa résolution est immédiate : $a=1$, $b=-\frac23$, $c=0$, $d=\frac13$.
Il n'y a donc qu'une seule matrice possible, à savoir
$B=\left(\begin{smallmatrix}
1 & -\frac23 \cr
0 & \frac13\cr
\end{smallmatrix}\right)$.
Pour prouver qu'elle convient, il faut aussi montrer l'égalité $BA=I$,
dont la vérification est laissée au lecteur.
La matrice $A$ est donc inversible et $A^{-1}= \begin{pmatrix}
1 & -\frac23 \cr
0 & \frac13\cr
\end{pmatrix}$.
\end{exemple}

\begin{exemple}
La matrice
$ A = \left(\begin{smallmatrix}
3 & 0\\
5 & 0\end{smallmatrix}\right)$
n'est pas inversible. En effet, soit
$B= \begin{pmatrix}
a & b \cr
c & d\cr
\end{pmatrix}$ une matrice quelconque.
Alors le produit
$$
BA =
\begin{pmatrix}
a & b \cr
c & d\cr
\end{pmatrix}
\begin{pmatrix}
3 & 0\\
5 & 0
\end{pmatrix}
=
\begin{pmatrix}
3a+5b & 0\\
3c+5d      & 0
\end{pmatrix}$$
ne peut jamais être égal à la matrice identité.
\end{exemple}


\begin{exemple}
\sauteligne
\begin{itemize}
  \item Soit $I_{n}$ la matrice carrée identité de taille $n\times n$.
C'est une matrice inversible, et son inverse est elle-même par l'égalité
 $I_{n}I_{n}=I_{n}$.

  \item La matrice nulle $0_n$ de taille $n \times n$ n'est pas
inversible. En effet on sait que, pour toute matrice $B$ de $M_{n}(\Kk)$,
on a $B0_{n}=0_n$, qui ne peut jamais être la matrice identité.
\end{itemize}

\end{exemple}

%---------------------------------------------------------------
\subsection{Propriétés}


%---------------------------------------------------------------
\subsubsection{Unicité}


\begin{proposition}
Si $A$ est inversible, alors son inverse est unique.
\end{proposition}

\begin{proof}
La méthode classique pour mener à bien une telle démonstration est de
supposer l'existence de deux matrices $B_{1}$ et $B_{2}$
satisfaisant aux conditions imposées et de démontrer que $B_{1}=B_{2}$.

Soient donc $B_{1}$ telle que $AB_{1}=B_{1}A=I_{n}$ et $B_{2}$ telle que
$AB_{2}=B_{2}A=I_{n}$.
Calculons $B_{2}(AB_{1})$.
D'une part, comme $AB_{1}=I_{n}$, on a  $B_{2}(AB_{1})=B_{2}.$
D'autre part, comme le produit des matrices est associatif, on a
$B_{2}(AB_{1})=(B_{2}A)B_{1}=I_{n}B_{1}=B_{1}$.
Donc $B_{1}=B_{2}$.
\end{proof}

%---------------------------------------------------------------
\subsubsection{Inverse de l'inverse}

\begin{proposition}
Soit $A$ une matrice inversible. Alors
$A^{-1}$ est aussi inversible et on a :
\mybox{$(A^{-1})^{-1}=A$}
\end{proposition}


%---------------------------------------------------------------
\subsubsection{Inverse d'un produit}


\begin{proposition}
 Soient $A$ et $B$ deux matrices inversibles de même taille. Alors
 $AB$ est inversible et
 \mybox{$\displaystyle (AB)^{-1} = B^{-1} A^{-1}$}
\end{proposition}

Il faut bien faire attention à l'inversion de l'ordre !

\begin{proof}
  Il suffit de montrer $(B^{-1}A^{-1}) (AB) = I$ et $(AB) (B^{-1} A^{-1}) = I$.
 Cela suit de
 \begin{align*}
   (B^{-1}A^{-1}) (AB) &= B^{-1}(AA^{-1})B = B^{-1}IB=B^{-1}B=I,\\
   \text{et } \quad (AB)(B^{-1} A^{-1}) &= A(BB^{-1})A^{-1} = A I A^{-1} = A A^{-1} = I.
 \end{align*}
\end{proof}


De façon analogue, on montre que si $A_1, \dots , A_m$ sont inversibles, alors
\[(A_1 A_2 \cdots A_m)^{-1} = A_m^{-1} A_{m-1}^{-1} \cdots A^{-1}_1.\]



%---------------------------------------------------------------
\subsubsection{Simplification par une matrice inversible}

Si $C$ est une matrice quelconque de $M_{n}(\Kk)$, nous avons vu
que  la relation  $AC=BC$
où $A$ et $B$ sont des éléments de  $M_{n}(\Kk)$
n'entraîne pas forcément l'égalité $A=B$.
En revanche, si $C$ est une matrice inversible, on a la proposition suivante :

\begin{proposition}
Soient $A$ et $B$ deux matrices de $M_{n}(\Kk)$ et $C$ une matrice
inversible de $M_{n}(\Kk)$.
Alors l'égalité $AC=BC$ implique l'égalité $A=B$.
\end{proposition}

\begin{proof}
Ce résultat est immédiat : si on multiplie à droite l'égalité
$AC=BC$ par $C^{-1}$, on obtient l'égalité :
$(AC)C^{-1}=(BC)C^{-1}$. En utilisant l'associativité
du produit des matrices on a $A(CC^{-1})=B(CC^{-1})$,
ce qui donne d'après la définition de l'inverse $AI=BI$, d'où
$A=B$.
\end{proof}

%---------------------------------------------------------------
%\subsection{Mini-exercices}

\begin{miniexercices}
\sauteligne
\begin{enumerate}
  \item Soient $A=\left(\begin{smallmatrix}-1 & -2 \\ 3  & 4 \end{smallmatrix}\right)$
  et $B = \left(\begin{smallmatrix}2 & 1 \\ 5  & 3 \end{smallmatrix}\right)$. Calculer
  $A^{-1}$, $B^{-1}$ , $(AB)^{-1}$, $(BA)^{-1}$, $A^{-2}$.

  \item Calculer l'inverse de $\left(\begin{smallmatrix}1&0&0\\0&2&0\\1&0&3\end{smallmatrix}\right)$.

  \item Soit $A=\left(\begin{smallmatrix}-1&-2&0\\2&3&0\\0&0&1\end{smallmatrix}\right)$.
  Calculer $2A-A^2$. Sans calculs, en déduire $A^{-1}$.
\end{enumerate}
\end{miniexercices}

%%%%%%%%%%%%%%%%%%%%%%%%%%%%%%%%%%%%%%%%%%%%%%%%%%%%%%%%%%%%%%%%
\section{Inverse d'une matrice : calcul}
\index{matrice!inverse}

Nous allons voir une méthode pour calculer l'inverse d'une matrice quelconque de manière efficace.
Cette méthode est une reformulation de la méthode du pivot de Gauss pour les systèmes linéaires.
Auparavant, nous commençons par une formule directe dans le cas simple des matrices $2\times 2$.

%---------------------------------------------------------------
\subsection{Matrices $2 \times 2$}
 Considérons la matrice $2\times 2$ :
$A = \begin{pmatrix}
 a & b\\
 c & d
     \end{pmatrix}.
$
\begin{proposition}

Si $ad - bc \not= 0$,  alors $A$ est inversible et
\mybox{$A^{-1} = \frac{1}{ad-bc} \begin{pmatrix}
d & -b\\
 -c & a
\end{pmatrix}$}
\end{proposition}


\begin{proof}
On vérifie que si $B=\frac{1}{ad-bc}  \left(\begin{smallmatrix}
d & -b\\
 -c & a
\end{smallmatrix}\right)$ alors $AB = \left(\begin{smallmatrix}
 1 & 0\\
 0 & 1
 \end{smallmatrix}\right)$. Idem pour $BA$.
\end{proof}


%---------------------------------------------------------------
\subsection{Méthode de Gauss pour inverser les matrices}

La méthode pour inverser une matrice $A$
consiste à faire des opérations élémentaires sur les lignes de la matrice $A$
jusqu'à la transformer en la matrice identité $I$.
On fait simultanément les mêmes opérations élémentaires en partant de la matrice $I$.
On aboutit alors à une matrice qui est $A^{-1}$. La preuve sera vue dans la section suivante.

\bigskip

En pratique, on fait les deux opérations en même temps en adoptant la disposition suivante :
à côté de la matrice $A$ que l'on veut inverser, on rajoute la matrice identité
pour former un tableau $(A\ |\ I)$.
Sur les lignes de cette matrice augmentée, on effectue des opérations
élémentaires jusqu'à obtenir
le tableau $(I\ |\ B)$. Et alors $B=A^{-1}$.

\bigskip

Ces opérations élémentaires sur les lignes sont :
\begin{enumerate}
  \item $L_i \leftarrow \lambda L_i$ avec $\lambda \neq 0$ :
  on peut multiplier une ligne par un réel non nul (ou un élément de $\Kk\setminus\{0\}$).

  \item $L_i \leftarrow L_i+\lambda L_j$ avec $\lambda \in \Kk$ (et $j\neq i$) :
  on peut ajouter à la ligne $L_i$ un multiple d'une autre ligne $L_j$.

  \item $L_i \leftrightarrow L_j$ : on peut échanger deux lignes.
\end{enumerate}

\bigskip

N'oubliez pas : tout ce que vous faites sur la partie gauche de la matrice augmentée,
vous devez aussi le faire sur la partie droite.



%---------------------------------------------------------------
\subsection{Un exemple}

Calculons l'inverse de
$
 A = \begin{pmatrix}
 1 & 2 & 1\\
 4 & 0 & -1\\
 -1 & 2 & 2
\end{pmatrix}\, .
$

Voici la matrice augmentée, avec les lignes numérotées :
$$(A\ |\ I) = \left(\begin{array}{ccc|ccc}
1 & 2 & 1 & 1 & 0 & 0\\
4 & 0 & -1 & 0 & 1 & 0\\
-1 & 2 & 2 & 0 & 0 & 1
\end{array}\right)
\begin{array}{l} {\scriptstyle L_1}  \\ {\scriptstyle L_2} \\  {\scriptstyle L_3}\end{array}
$$

On applique la méthode de Gauss pour faire apparaître des $0$ sur la première colonne,
d'abord sur la deuxième ligne par l'opération élémentaire
$L_2 \leftarrow L_2 - 4 L_1$ qui conduit à la matrice augmentée :
$$
\left(\begin{array}{ccc|ccc}
1 & 2 & 1 & 1 & 0 & 0\\
0 & -8 & -5 & -4 & 1 & 0\\
-1 & 2 & 2 & 0 & 0 & 1
\end{array}\right)
\begin{array}{l} ~ \\  {\scriptstyle L_2 \leftarrow L_2 - 4 L_1} \\ ~ \end{array}
$$
Puis un $0$ sur la première colonne, à la troisième ligne, avec $L_3 \leftarrow L_3 + L_1$ :

$$
\left(\begin{array}{ccc|ccc}
1 & 2 & 1 & 1 & 0 & 0\\
0 & -8 & -5 & -4 & 1 & 0\\
0 & 4 & 3 & 1 & 0 & 1
\end{array}\right)
\begin{array}{l} ~ \\ ~ \\ {\scriptstyle L_3 \leftarrow L_3 + L_1}   \end{array}
$$
On multiplie la ligne $L_2$ afin qu'elle commence par $1$ :
$$
\left(\begin{array}{ccc|ccc}
1 & 2 & 1 & 1 & 0 & 0\\
0 & 1 & \frac58 & \frac12 & -\frac18 & 0\\
0 & 4 & 3 & 1 & 0 & 1
\end{array}\right)
\begin{array}{l} ~ \\  {\scriptstyle L_2 \leftarrow -\frac{1}{8} L_2} \\ ~ \end{array}
$$
On continue afin de faire apparaître des $0$ partout sous la diagonale,
et on multiplie la ligne $L_3$.
Ce qui termine la première partie de la méthode de Gauss :
$$
\left(\begin{array}{ccc|ccc}
1 & 2 & 1 & 1 & 0 & 0\\
0 & 1 & \frac58 & \frac12 & -\frac18 & 0\\
0 & 0 & \frac12 & -1 & \frac12 & 1
\end{array}\right)
\begin{array}{l} \\  \\ {\scriptstyle L_3 \leftarrow L_3 -4L_2}   \end{array}
$$
puis
$$
\left(\begin{array}{ccc|ccc}
1 & 2 & 1 & 1 & 0 & 0\\
0 & 1 & \frac58 & \frac12 & -\frac18 & 0\\
0 & 0 & 1 & -2 & 1 & 2
\end{array}\right)
\begin{array}{l}  \\  \\ {\scriptstyle L_3 \leftarrow 2L_3}   \end{array}
$$

Il ne reste plus qu'à \og remonter \fg{} pour faire apparaître des zéros au-dessus de la diagonale:



$$
\left(\begin{array}{ccc|ccc}
1 & 2 & 1 & 1 & 0 & 0\\
0 & 1 & 0 & \frac{7}{4} & -\frac{3}{4} & -\frac{5}{4}\\
0 & 0 & 1 & -2 & 1 & 2
\end{array}\right)
\begin{array}{l} ~ \\  {\scriptstyle L_2 \leftarrow L_2-\frac{5}{8} L_3} \\ ~ \end{array}
$$
puis
$$
\left(\begin{array}{ccc|ccc}
1 & 0 & 0 & -\frac{1}{2} & \frac{1}{2} & \frac{1}{2}\\
0 & 1 & 0 &\frac{7}{4} & -\frac{3}{4} & -\frac{5}{4}\\
0 & 0 & 1 & -2 & 1 & 2
\end{array}\right)
\begin{array}{l}  {\scriptstyle L_1 \leftarrow L_1-2L_2-L_3} \\ ~ \\ ~ \end{array}
$$

Ainsi l'inverse de $A$ est la matrice obtenue à droite et après avoir factorisé
tous les coefficients par $\frac14$, on a obtenu :
$$
A^{-1} = \frac{1}{4}
\begin{pmatrix}
-2 & 2 & 2\\
7 & -3 & -5\\
-8 & 4 & 8
\end{pmatrix}
$$

Pour se rassurer sur ses calculs,
on n'oublie pas de vérifier rapidement que $A \times A^{-1} = I$.

%---------------------------------------------------------------
%\subsection{Mini-exercices}

\begin{miniexercices}
\sauteligne
\begin{enumerate}
  \item Si possible calculer l'inverse des matrices :
  $\left(\begin{smallmatrix}3&1\\7&2\end{smallmatrix}\right)$,
  $\left(\begin{smallmatrix}2&-3\\-5&4\end{smallmatrix}\right)$,
  $\left(\begin{smallmatrix}0&2\\3&0\end{smallmatrix}\right)$,
  $\left(\begin{smallmatrix}\alpha+1&1\\2&\alpha\end{smallmatrix}\right)$.

  \item Soit
  $A(\theta)=\left(\begin{smallmatrix} \cos \theta & -\sin\theta \\ \sin\theta & \cos\theta\end{smallmatrix}\right)$.
  Calculer $A(\theta)^{-1}$.

  \item Calculer l'inverse des matrices :
  $\left(\begin{smallmatrix}1&3&0\\2&1&-1\\-2&1&1\end{smallmatrix}\right)$,
  $\left(\begin{smallmatrix}2&-2&1\\3&0&5\\1&1&2\end{smallmatrix}\right)$,
  $\left(\begin{smallmatrix}1&0&1&0\\0&2&-2&0\\-1&2&0&1\\0&2&1&3\end{smallmatrix}\right)$,
  $\left(\begin{smallmatrix}2&1&1&1\\1&0&0&1\\0&1&-1&2\\0&1&1&0\end{smallmatrix}\right)$,
  $\left(\begin{smallmatrix}1&1&1&0&0\\0&1&2&0&0\\-1&1&2&0&0\\0&0&0&2&1\\0&0&0&5&3\end{smallmatrix}\right)$.

\end{enumerate}
\end{miniexercices}

%%%%%%%%%%%%%%%%%%%%%%%%%%%%%%%%%%%%%%%%%%%%%%%%%%%%%%%%%%%%%%%%
\section{Inverse d'une matrice : systèmes linéaires et matrices élémentaires}


%---------------------------------------------------------------
\subsection{Matrices et systèmes linéaires}

Le système linéaire
\[ \left\{
\begin{array}{ccccccccc}
a_{11} \  x_1 &+& a_{12}\  x_2 &+& \cdots &+& a_{1p}\  x_p & = & b_1\\
a_{21}\  x_1 &+& a_{22}\  x_2 &+& \cdots &+& a_{2p}\  x_p & = & b_2\\
&&\dots  && &&\\
a_{n1}\  x_1 &+& a_{n2}\  x_2 &+& \cdots &+& a_{np}\  x_p & = & b_n
\end{array} \right.
\]
peut s'écrire sous forme matricielle :
\begin{equation*}\begin{array}{cccc}
\underbrace{
\left(
\begin{array}{ccc}
a_{11} & \dots & a_{1p}\\
a_{21} & \dots & a_{2p}\\
\vdots &&\vdots\\
a_{n1} &\dots & a_{np}
\end{array}
\right)
}
&
\underbrace{
\left(
\begin{array}{c}
x_1\\
x_2\\
\vdots\\
x_p
\end{array}
\right)
}
& = &
\underbrace{
\left(
\begin{array}{c}
b_1\\
b_2\\
\vdots\\
b_n
\end{array}
\right).
}
\\
A & X & &B
\end{array}\end{equation*}

On appelle $A \in M_{n,p}(\Kk)$ la matrice des coefficients du système.
$B\in M_{n,1}(\Kk)$ est le vecteur du second membre.
Le vecteur $X \in M_{p,1}(\Kk)$ est une solution du système si et seulement si
$$AX = B.$$

Nous savons que :
\begin{theoreme}
Un système d'équations linéaires n'a soit aucune solution,
soit une seule solution, soit une infinité de solutions.
\end{theoreme}


%---------------------------------------------------------------
\subsection{Matrices inversibles et systèmes linéaires}


Considérons le cas où le nombre d'équations égale le nombre d'inconnues :

\begin{equation*}\begin{array}{cccc}
\underbrace{
\left(
\begin{array}{ccc}
a_{11} & \dots & a_{1n}\\
a_{21} & \dots & a_{2n}\\
\vdots &&\vdots\\
a_{n1} &\dots & a_{nn}
\end{array}
\right)
}
&
\underbrace{
\left(
\begin{array}{c}
x_1\\
x_2\\
\vdots\\
x_n
\end{array}
\right)
}
& = &
\underbrace{
\left(
\begin{array}{c}
b_1\\
b_2\\
\vdots\\
b_n
\end{array}
\right).
}
\\
A & X & & B
\end{array}\end{equation*}


Alors $A \in M_n(\Kk)$ est une matrice carrée et
$B$ un vecteur de $M_{n,1}(\Kk)$.
Pour tout second membre, nous pouvons utiliser les matrices
pour trouver la solution du système linéaire.
\begin{proposition}
Si la matrice $A$ est inversible, alors
la solution du système $AX=B$ est unique et est :
\mybox{$X = A^{-1}B$.}
\end{proposition}

La preuve est juste de vérifier que si $X = A^{-1}B$,
alors $AX = A\big(A^{-1}B\big) = \big(AA^{-1}\big)B = I \cdot B=B$.
Réciproquement si $AX=B$, alors nécessairement $X= A^{-1}B$.
Nous verrons bientôt que si la matrice n'est pas inversible,
alors soit il n'y a pas de solution, soit une infinité.

%---------------------------------------------------------------
\subsection{Les matrices élémentaires}
\index{matrice!elementaire@élémentaire}

Pour calculer l'inverse d'une matrice $A$, et aussi pour résoudre des systèmes linéaires, nous avons utilisé trois opérations
 élémentaires sur les lignes qui sont :
\begin{enumerate}
  \item $L_i \leftarrow \lambda L_i$ avec $\lambda \neq 0$ :
  on peut multiplier une ligne par un réel non nul (ou un élément de $\Kk\setminus\{0\}$).

  \item $L_i \leftarrow L_i+\lambda L_j$ avec $\lambda \in \Kk$ (et $j\neq i$) :
  on peut ajouter à la ligne $L_i$ un multiple d'une autre ligne $L_j$.

  \item $L_i \leftrightarrow L_j$ : on peut échanger deux lignes.
\end{enumerate}

Nous allons définir trois matrices élémentaires
$E_{L_i \leftarrow \lambda L_i}$, $E_{L_i \leftarrow L_i+\lambda L_j}$,
$E_{L_i \leftrightarrow L_j}$
correspondant à ces opérations. Plus précisément, le produit $E\times A$ correspondra
à l'opération élémentaire sur $A$. Voici les définitions accompagnées d'exemples.


\begin{enumerate}
  \item La matrice $E_{L_i \leftarrow \lambda L_i}$ est la matrice obtenue en
    multipliant par $\lambda$ la $i$-ème ligne de la matrice identité $I_n$,
    où $\lambda$ est un nombre réel non nul.


    $$ E_{L_2 \leftarrow 5 L_2}=
    \begin{pmatrix}
    1 & 0 & 0 & 0\\
    0 & 5 & 0 & 0\\
    0 & 0 & 1 & 0\\
    0 & 0 & 0 & 1\end{pmatrix}$$


\item  La matrice $E_{L_i \leftarrow L_i+\lambda L_j}$ est la  matrice
obtenue en ajoutant $\lambda$ fois la $j$-ème ligne de $I_n$ à la $i$-ème ligne de $I_n$.


 $$  E_{L_2 \leftarrow L_2 -3 L_1}=
    \begin{pmatrix}
    1 & 0 & 0 & 0\\
    -3 & 1 & 0 & 0\\
    0 & 0 & 1 & 0\\
    0 & 0 & 0 & 1
    \end{pmatrix}$$



   \item La matrice  $E_{L_i \leftrightarrow L_j}$ est la matrice obtenue
   en permutant les $i$-ème et $j$-ème lignes de $I_n$.

   $$ E_{L_2 \leftrightarrow L_4} = E_{L_4 \leftrightarrow L_2} =
   \begin{pmatrix}
    1 & 0 & 0 & 0\\
    0 & 0 & 0 & 1\\
    0 & 0 & 1 & 0\\
    0 & 1 & 0 & 0
    \end{pmatrix}$$

\end{enumerate}


Les opérations élémentaires sur les lignes sont réversibles,
ce qui entraîne l'inversibilité des matrices élémentaires.

\bigskip

Le résultat de la multiplication d'un matrice élémentaire $E$ par $A$
est la matrice obtenue en effectuant
l'opération élémentaire correspondante sur $A$.
Ainsi :
\begin{enumerate}
  \item La matrice $E_{L_i \leftarrow \lambda L_i} \times A$ est la matrice obtenue en
    multipliant par $\lambda$ la $i$-ème ligne de $A$.

  \item La matrice $E_{L_i \leftarrow L_i+\lambda L_j} \times A$ est la  matrice obtenue
  en ajoutant $\lambda$ fois la $j$-ème ligne de $A$ à la $i$-ème ligne de $A$.

  \item La matrice  $E_{L_i \leftrightarrow L_j} \times A$ est la matrice obtenue
   en permutant les $i$-ème et $j$-ème lignes de $A$.
\end{enumerate}

\begin{exemple}
\sauteligne
\begin{enumerate}
    \item
$$E_{L_2 \leftarrow \frac13 L_2}  \times A
= \begin{pmatrix}
  1&0&0\\0&\frac13&0\\0&0&1
  \end{pmatrix}
  \times
  \begin{pmatrix}
  x_1&x_2&x_3\\y_1&y_2&y_3\\z_1&z_2&z_3
  \end{pmatrix}
  =   \begin{pmatrix}
  x_1&x_2&x_3\\ \frac13y_1&\frac13y_2&\frac13y_3\\z_1&z_2&z_3
  \end{pmatrix}
$$

  \item
$$\hspace*{-1.1em}E_{L_1 \leftarrow L_1-7 L_3}  \times A
= \begin{pmatrix}
  1&0&-7\\0&1&0\\0&0&1
  \end{pmatrix}
  \times
  \begin{pmatrix}
  x_1&x_2&x_3\\y_1&y_2&y_3\\z_1&z_2&z_3
  \end{pmatrix}
  \!=\!   \begin{pmatrix}
  x_1-7z_1&x_2-7z_2&x_3-7z_3\\ y_1&y_2&y_3\\z_1&z_2&z_3
  \end{pmatrix}
$$

   \item
$$E_{L_2 \leftrightarrow L_3}  \times A
= \begin{pmatrix}
  1&0&0\\0&0&1\\0&1&0
  \end{pmatrix}
  \times
  \begin{pmatrix}
  x_1&x_2&x_3\\y_1&y_2&y_3\\z_1&z_2&z_3
  \end{pmatrix}
  =   \begin{pmatrix}
  x_1&x_2&x_3\\z_1&z_2&z_3\\ y_1&y_2&y_3
  \end{pmatrix}
$$
\end{enumerate}
\end{exemple}



%---------------------------------------------------------------
\subsection{\'Equivalence à une matrice échelonnée}


\begin{definition}
Deux matrices $A$ et $B$ sont dites \defi{équivalentes par lignes} si l'une
peut être obtenue à partir de l'autre par une suite d'opérations
élémentaires sur les lignes. On note $A \sim B$.
\end{definition}

\begin{definition}
Une matrice est \defi{échelonnée}\index{matrice!echelonnee@échelonnée} si :
\begin{itemize}
  \item le nombre de zéros commençant une ligne croît strictement ligne par ligne
  jusqu'à ce qu'il ne reste plus que des zéros.
\end{itemize}

Elle est \defi{échelonnée réduite}\index{matrice!reduite@réduite} si en plus :
\begin{itemize}
\setcounter{enumi}{1}
  \item le premier coefficient non nul d'une ligne (non nulle) vaut $1$ ;

  \item et c'est le seul élément non nul de sa colonne.
\end{itemize}
\end{definition}

Exemple d'une matrice échelonnée (à gauche) et échelonnée réduite (à droite) ;
les $*$ désignent des coefficients quelconques, les $+$ des coefficients non nuls :
$$
\begin{pmatrix}
+ & * & * & * & * & * & * \\
0 & 0 & + & * & * & * & * \\
0 & 0 & 0 & + & * & * & * \\
0 & 0 & 0 & 0 & 0 & 0 & + \\
0 & 0 & 0 & 0 & 0 & 0 & 0 \\
0 & 0 & 0 & 0 & 0 & 0 & 0 \\
\end{pmatrix}
\qquad \qquad
\begin{pmatrix}
1 & * & 0 & 0 & * & * & 0 \\
0 & 0 & 1 & 0 & * & * & 0 \\
0 & 0 & 0 & 1 & * & * & 0 \\
0 & 0 & 0 & 0 & 0 & 0 & 1 \\
0 & 0 & 0 & 0 & 0 & 0 & 0 \\
0 & 0 & 0 & 0 & 0 & 0 & 0 \\
\end{pmatrix}
$$

\begin{theoreme}\label{Gauss}
\'Etant donnée une matrice $A\in M_{n,p}(\Kk)$, il existe une unique
matrice échelonnée réduite $U$ obtenue à partir de $A$
par des opérations élémentaires sur les lignes.
\end{theoreme}

Ce théorème permet donc de se ramener par des opérations élémentaires
à des matrices dont la structure est beaucoup plus simple :
les matrices échelonnées réduites.

\begin{proof}
Nous admettons l'unicité.

L'existence se démontre grâce à
l'algorithme de Gauss. L'idée générale consiste à utiliser des substitutions de
lignes pour placer des zéros là où il faut de façon à
créer d'abord une forme échelonnée, puis une forme
échelonnée réduite.


\bigskip

Soit $A$ une matrice $n\times p$ quelconque.

\bigskip

\textbf{Partie A. Passage à une forme échelonnée.}

\textbf{\'Etape A.1.}  \emph{Choix du pivot.}

On commence par inspecter la
première colonne. Soit elle ne contient que des zéros,
auquel cas on passe directement à l'étape A.3,
soit elle contient au moins un terme non nul.
On choisit alors un tel terme, que l'on appelle le \defi{pivot}\index{pivot}.
Si c'est le terme $a_{11}$, on passe directement à l'étape A.2 ;
si c'est un terme $a_{i1}$ avec $i\neq1$, on échange les lignes $1$
et $i$ ($L_1 \leftrightarrow L_i$) et on passe à l'étape A.2.

Au terme de l'étape A.1, soit la matrice $A$ a sa première colonne nulle (à gauche)
ou bien on obtient une matrice équivalente dont le premier coefficient $a'_{11}$ est non nul (à droite) :
$$\begin{pmatrix}
0&a_{12}&\cdots&a_{1j}&\cdots&a_{1p}\\
0&a_{22}&\cdots&a_{2j}&\cdots&a_{2p}\\
\vdots&\vdots&&\vdots&&\vdots\\
0&a_{i2}&\cdots&a_{ij}&\cdots&a_{ip}\\
\vdots&\vdots&&\vdots&&\vdots\\
0&a_{n2}&\cdots&a_{nj}&\cdots&a_{np}\\
\end{pmatrix}=A
\quad \text{ou} \quad
\begin{pmatrix}
a'_{11}&a'_{12}&\cdots&a'_{1j}&\cdots&a'_{1p}\\
a'_{21}&a'_{22}&\cdots&a'_{2j}&\cdots&a'_{2p}\\
\vdots&\vdots&&\vdots&&\vdots\\
a'_{i1}&a'_{i2}&\cdots&a'_{ij}&\cdots&a'_{ip}\\
\vdots&\vdots&&\vdots&&\vdots\\
a'_{n1}&a'_{n2}&\cdots&a'_{nj}&\cdots&a'_{np}\\
\end{pmatrix}\sim A.
$$

\medskip
\textbf{\'Etape A.2.}  \emph{\'Elimination.}


On ne touche plus à la ligne $1$, et on se sert du pivot $a'_{11}$
pour éliminer tous les termes $a'_{i1}$ (avec $i\ge 2$) situés sous le pivot.
Pour cela, il suffit de remplacer la ligne $i$ par
elle-même moins $\frac{a'_{i1}}{a'_{11}}\times$ la ligne $1$, ceci pour $i=2,\ldots,n$ :
$L_2 \leftarrow L_2 - \frac{a'_{21}}{a'_{11}}L_1$,
$L_3 \leftarrow L_3 - \frac{a'_{31}}{a'_{11}}L_1$,\ldots

Au terme de l'étape A.2, on a obtenu une matrice de la forme
$$\begin{pmatrix}
a'_{11}&a'_{12}&\cdots&a'_{1j}&\cdots&a'_{1p}\\
0&a''_{22}&\cdots&a''_{2j}&\cdots&a''_{2p}\\
\vdots&\vdots&&\vdots&&\vdots\\
0&a''_{i2}&\cdots&a''_{ij}&\cdots&a''_{ip}\\
\vdots&\vdots&&\vdots&&\vdots\\
0&a''_{n2}&\cdots&a''_{nj}&\cdots&a''_{np}\\
\end{pmatrix}\sim A.
$$

\medskip
\textbf{\'Etape A.3.}  \emph{Boucle.}

Au début de l'étape A.3, on a obtenu dans tous les cas de figure une matrice de la forme
$$\begin{pmatrix}
a^1_{11}&a^1_{12}&\cdots&a^1_{1j}&\cdots&a^1_{1p}\\
0&a^1_{22}&\cdots&a^1_{2j}&\cdots&a^1_{2p}\\
\vdots&\vdots&&\vdots&&\vdots\\
0&a^1_{i2}&\cdots&a^1_{ij}&\cdots&a^1_{ip}\\
\vdots&\vdots&&\vdots&&\vdots\\
0&a^1_{n2}&\cdots&a^1_{nj}&\cdots&a^1_{np}\\
\end{pmatrix}\sim A
$$
dont la première colonne est bien celle d'une matrice échelonnée.
On va donc conserver cette première colonne. Si $a^1_{11}\neq0$,
on conserve aussi la première ligne, et l'on repart avec l'étape A.1
en l'appliquant cette fois à la sous-matrice $(n-1)\times(p-1)$
(ci-dessous à gauche : on \og oublie \fg{} la première ligne et la première colonne de $A$) ;
si $a^1_{11}=0$, on repart avec l'étape A.1 en l'appliquant à la sous-matrice
$n\times(p-1)$ (à droite, on \og oublie \fg{} la première colonne) :
$$\begin{pmatrix}
a^1_{22}&\cdots&a^1_{2j}&\cdots&a^1_{2p}\\
\vdots&&\vdots&&\vdots\\
a^1_{i2}&\cdots&a^1_{ij}&\cdots&a^1_{ip}\\
\vdots&&\vdots&&\vdots\\
a^1_{n2}&\cdots&a^1_{nj}&\cdots&a^1_{np}\\
\end{pmatrix}
\qquad \qquad
\begin{pmatrix}
a^1_{12}&\cdots&a^1_{1j}&\cdots&a^1_{1p}\\
a^1_{22}&\cdots&a^1_{2j}&\cdots&a^1_{2p}\\
\vdots&&\vdots&&\vdots\\
a^1_{i2}&\cdots&a^1_{ij}&\cdots&a^1_{ip}\\
\vdots&&\vdots&&\vdots\\
a^1_{n2}&\cdots&a^1_{nj}&\cdots&a^1_{np}\\
\end{pmatrix}
$$

Au terme de cette deuxième itération de la boucle, on aura obtenu une matrice de la forme
$$\begin{pmatrix}
a^1_{11}&a^1_{12}&\cdots&a^1_{1j}&\cdots&a^1_{1p}\\
0&a^2_{22}&\cdots&a^2_{2j}&\cdots&a^2_{2p}\\
\vdots&\vdots&&\vdots&&\vdots\\
0&0&\cdots&a^2_{ij}&\cdots&a^2_{ip}\\
\vdots&\vdots&&\vdots&&\vdots\\
0&0&\cdots&a^2_{nj}&\cdots&a^2_{np}\\
\end{pmatrix}\sim A,
$$
et ainsi de suite.

Comme chaque itération de la boucle travaille sur une matrice qui a
une colonne de moins que la précédente, alors au bout d'au
plus $p-1$ itérations de la boucle, on aura obtenu une
matrice échelonnée.

\bigskip

\textbf{Partie B. Passage à une forme échelonnée réduite.}

\textbf{\'Etape B.1.}  \emph{Homothéties.}

On repère le premier élément non nul de chaque ligne non nulle,
et on multiplie cette ligne par l'inverse de cet élément.
Exemple : si le premier élément non nul de la ligne $i$ est $\alpha\neq0$, alors
on effectue $L_i \leftarrow \frac1\alpha L_i$.
Ceci crée une matrice échelonnée avec des $1$ en position de pivots.


\medskip
\textbf{\'Etape B.2.}  \emph{\'Elimination.}

On élimine les termes situés au-dessus des positions de pivot
comme précédemment, en procédant à partir du bas à droite de la matrice.
Ceci ne modifie pas la structure échelonnée de la matrice en raison de
la disposition des zéros dont on part.

\end{proof}


\begin{exemple}
Soit
$$A=\begin{pmatrix}1&2&3&4\\
0&2&4&6\\
-1&0&1&0
\end{pmatrix}.$$

\textbf{A. Passage à une forme échelonnée.}

Première itération de la boucle, étape A.1. Le choix du pivot est tout fait,
on garde $a_{11}^1=1$.

Première itération de la boucle, étape A.2. On ne fait rien sur la ligne $2$ qui contient
déjà un zéro en bonne position et on remplace la ligne 3 par $L_3 \leftarrow L_3 + L_1$.
On obtient
$$A\sim\begin{pmatrix}1&2&3&4\\
0&2&4&6\\
0&2&4&4
\end{pmatrix}.$$

Deuxième itération de la boucle, étape A.1. Le choix du pivot est tout fait,
on garde $a^2_{22}=2$.

Deuxième itération de la boucle, étape A.2. On remplace  la ligne $3$ avec
l'opération $L_3 \leftarrow L_3 - L_2$. On obtient
$$A\sim\begin{pmatrix}1&2&3&4\\
0&2&4&6\\
0&0&0&-2
\end{pmatrix}.$$
Cette matrice est échelonnée.

\medskip

\textbf{B. Passage à une forme échelonnée réduite.}

\'Etape B.1, homothéties. On multiplie la ligne $2$ par $\frac12$
et la ligne $3$ par $-\frac12$ et l'on obtient
$$A\sim\begin{pmatrix}1&2&3&4\\
0&1&2&3\\
0&0&0&1
\end{pmatrix}.$$

{\'E}tape B.2, première itération.
On ne touche plus à la ligne $3$ et on remplace la ligne $2$ par
$L_2 \leftarrow L_2-3L_3$ et $L_1 \leftarrow L_1 - 4L_3$.
On obtient
$$A\sim\begin{pmatrix}1&2&3&0\\
0&1&2&0\\
0&0&0&1
\end{pmatrix}.$$

{\'E}tape B.2, deuxième itération. On ne touche plus à la ligne $2$
et on remplace la ligne $1$ par $L_1 \leftarrow L_1-2L_2$. On obtient
$$A\sim\begin{pmatrix}1&0&-1&0\\
0&1&2&0\\
0&0&0&1
\end{pmatrix}$$
qui est bien échelonnée et réduite.
\end{exemple}


%---------------------------------------------------------------
\subsection{Matrices élémentaires et inverse d'une matrice}


\begin{theoreme}
\label{th:invequi}
Soit $A\in M_{n}(\Kk)$.
La matrice $A$ est inversible si et seulement si sa forme échelonnée réduite
est la matrice identité $I_n$.
\end{theoreme}

\begin{proof}
Notons $U$ la forme échelonnée réduite de $A$.
Et notons $E$ le produit de matrices élémentaires tel que $EA=U$.
\begin{itemize}
  \item[$\Longleftarrow$] Si $U=I_n$ alors $EA=I_n$. Ainsi par définition, $A$ est inversible et
  $A^{-1}= E$.

  \item[$\Longrightarrow$] Nous allons montrer que si $U\neq I_n$, alors $A$ n'est pas inversible.
  \begin{itemize}
    \item Supposons $U\neq I_n$. Alors la dernière ligne de $U$ est nulle (sinon il y aurait un pivot
    sur chaque ligne donc ce serait $I_n$).
    \item Cela entraîne que $U$ n'est pas inversible : en effet, pour tout matrice
     carrée $V$, la dernière ligne de $UV$ est nulle ; on n'aura donc jamais $UV=I_{n}$.
    \item Alors, $A$ n'est pas inversible non plus : en effet, si $A$ était inversible, on
aurait $U=EA$ et $U$ serait inversible comme produit de matrices inversibles
($E$ est inversible car c'est un produit de matrices élémentaires qui sont inversibles).
  \end{itemize}
\end{itemize}
\end{proof}

\begin{remarque*}
Justifions maintenant notre méthode pour calculer $A^{-1}$.

Nous partons de $(A|I)$ pour arriver par des opérations élémentaires sur les lignes
à $(I|B)$. Montrons que $B=A^{-1}$.
Faire une opération élémentaire signifie multiplier à gauche par une des matrices
élémentaires. Notons $E$ le produit de ces matrices élémentaires.
Dire que l'on arrive à la fin du processus à $I$ signifie $EA=I$. Donc $A^{-1}=E$.
Comme on fait les mêmes opérations sur la partie droite du tableau, alors on obtient
$EI = B$. Donc $B=E$. Conséquence : $B=A^{-1}$.
\end{remarque*}


\begin{corollaire}
Les assertions suivantes sont équivalentes :
\begin{itemize}
  \item[(i)] La matrice $A$ est inversible.

  \item[(ii)] Le système linéaire $AX=\left(\begin{smallmatrix} 0 \\ \vdots \\ 0\end{smallmatrix}\right)$ a une unique solution
  $X=\left(\begin{smallmatrix} 0 \\ \vdots \\ 0\end{smallmatrix}\right)$.

  \item[(iii)] Pour tout second membre $B$, le système linéaire $AX=B$
  a une unique solution $X$.
\end{itemize}
\end{corollaire}

\begin{proof}
Nous avons déjà vu $(i)  \implies (ii)$ et
$(i) \implies (iii)$.

Nous allons seulement montrer $(ii) \implies (i)$.
Nous raisonnons par contraposée : nous allons montrer
la proposition équivalente $\text{non} (i) \implies \text{non} (ii)$.
Si $A$ n'est pas inversible, alors sa forme échelonnée réduite $U$ contient
un premier zéro sur sa diagonale, disons à la place $\ell$.
Alors $U$ à la forme suivante
\[\left( \begin{array}{ccccccc}
1 & 0 & \cdots & c_1 & * & \cdots& * \\
0& \ddots  & 0 &\vdots &&\cdots & * \\
0 & 0 & 1 & c_{\ell-1} && \cdots &* \\
0 & \cdots & 0 & {\color{myred}0} &*&\cdots &*\\
0 & \cdots & 0& 0& * & \cdots &*  \\
\vdots & \vdots & \vdots & \cdots &0&\ddots & \vdots \\
0& \cdots &&&\cdots&0&*
\end{array}\right).
\qquad \text{On note} \qquad
X = \begin{pmatrix} -c_1 \\ \vdots \\ -c_{\ell-1} \\ 1 \\ 0 \\ \vdots \\ 0 \end{pmatrix}.
\]

Alors $X$ n'est pas le vecteur nul, mais $UX$ est le vecteur nul.
Comme $A = E^{-1}U$, alors $AX$ est le vecteur nul. Nous avons donc trouvé
un vecteur non nul $X$ tel que $AX=0$.
\end{proof}


%---------------------------------------------------------------
%\subsection{Mini-exercices}

\begin{miniexercices}
\sauteligne
\begin{enumerate}
  \item Exprimer les systèmes linéaires suivants sous forme matricielle
  et les résoudre en inversant la matrice :
  $\left\{\begin{array}{l}2x+4y=7\\-2x+3y=-14\end{array} \right.$,\quad
  $\left\{\begin{array}{l}x+z=1\\-2y+3z=1\\x+z=1\end{array} \right.$,\quad
  $\left\{\begin{array}{l}x+t=\alpha\\x-2y=\beta\\x+y+t=2\\y+t=4\end{array} \right.$.


  \item \'Ecrire les matrices $4\times 4$ correspondant aux opérations élémentaires :
  $L_2 \leftarrow \frac13 L_2$, $L_3 \leftarrow L_3-\frac14 L_2$, $L_1 \leftrightarrow L_4$.
  Sans calculs, écrire leurs inverses. \'Ecrire la matrice $4\times 4$
  de l'opération $L_1 \leftarrow L_1-2L_3+3L_4$.

  \item \'Ecrire les matrices suivantes sous forme échelonnée, puis échelonnée réduite :
  $\left(\begin{smallmatrix}1&2&3\\1&4&0\\-2&-2&-3\end{smallmatrix}\right)$,
  $\left(\begin{smallmatrix}1&0&2\\1&-1&1\\2&-2&3\end{smallmatrix}\right)$,
  $\left(\begin{smallmatrix}2&0&-2&0\\0&-1&1&0\\1&-2&1&4\\-1&2&-1&-2\end{smallmatrix}\right)$.

\end{enumerate}
\end{miniexercices}

%%%%%%%%%%%%%%%%%%%%%%%%%%%%%%%%%%%%%%%%%%%%%%%%%%%%%%%%%%%%%%%%
\section{Matrices triangulaires, transposition, trace, matrices symétriques}

%---------------------------------------------------------------
\subsection{Matrices triangulaires, matrices diagonales}

\index{matrice!triangulaire}

Soit $A$ une matrice de taille $n \times n$. On dit que $A$ est \defi{triangulaire inférieure}
si ses éléments au-dessus de la diagonale sont nuls, autrement dit :
$$
 i < j \  \Longrightarrow \ a_{ij} = 0.$$

Une matrice triangulaire inférieure a la forme suivante:
$$\begin{pmatrix}
a_{11} & 0 &\cdots&\cdots& 0\\
a_{21}&a_{22}&\ddots&&\vdots\\
\vdots&\vdots&\ddots&\ddots&\vdots\\
\vdots & \vdots &&\ddots&0\\
a_{n1}&a_{n2}&\cdots&\cdots&a_{nn}
\end{pmatrix}
$$

\bigskip

On dit que $A$ est \defi{triangulaire supérieure} si ses éléments en-dessous
de la diagonale sont nuls,  autrement dit :
$$
i > j \ \Longrightarrow \ a_{ij} = 0. $$

Une matrice triangulaire supérieure a la forme suivante:
$$\begin{pmatrix}
a_{11} & a_{12} &\dots&\dots&\dots & a_{1n}\\
0&a_{22}&\dots&\dots&\dots&a_{2n}\\
\vdots&\ddots&\ddots&&&\vdots\\
\vdots&&\ddots&\ddots&&\vdots\\
\vdots & &&\ddots&\ddots&\vdots\\
0&\dots&\dots&\dots&0&a_{nn}
  \end{pmatrix}
$$


 \begin{exemple}
Deux matrices triangulaires inférieures (à gauche), une matrice triangulaire supérieure (à droite) :
$$
\begin{pmatrix}
 4 & 0 & 0\\
 0 & -1 & 0\\
 3 & -2 & 3
\end{pmatrix}\qquad\qquad
\begin{pmatrix}
 5 & 0\\
 1 & -2
\end{pmatrix}\qquad\qquad
\begin{pmatrix}
 1 & 1 & -1\\
 0 & -1 & -1\\
 0 & 0 & -1
\end{pmatrix}
$$
  \end{exemple}


Une matrice qui est triangulaire inférieure \evidence{et} triangulaire supérieure
est dite \defi{diagonale}\index{matrice!diagonale}.
Autrement dit : $i\neq j \ \Longrightarrow \ a_{ij} = 0$.


\begin{exemple} Exemples de matrices diagonales:
$$\begin{pmatrix}
 -1 &  0 &0\\
 0 & 6 & 0\\
 0 & 0 &0
 \end{pmatrix} \qquad \text{ et } \qquad
 \begin{pmatrix}
 2 & 0\\
 0 & 3
\end{pmatrix}$$
\end{exemple}


\begin{exemple}[Puissances d'une matrice diagonale]
Si $D$ est une matrice diagonale, il est très facile de calculer ses puissances $D^p$
(par récurrence sur $p$) :
$$D=\begin{pmatrix}
\alpha_{1}& 0& \dots &  \dots & 0\cr
0& \alpha_{2}& 0&  \dots &0\cr
\vdots & \ddots & \ddots & \ddots & \vdots \cr
0& \dots & 0 & \alpha_{n-1} & 0 \cr
0& \dots & \dots & 0 & \alpha_{n}\cr
\end{pmatrix}
\qquad \implies \qquad
D^{p}=\begin{pmatrix}
\alpha_{1}^{p}& 0& \dots &  \dots & 0\cr
0& \alpha_{2}^{p}& 0&  \dots &0\cr
\vdots & \ddots & \ddots & \ddots & \vdots \cr
0& \dots & 0 & \alpha_{n-1}^{p} & 0 \cr
0& \dots & \dots & 0 & \alpha_{n}^{p}\cr
\end{pmatrix}$$
\end{exemple}



\begin{theoreme}
Une matrice $A$ de taille $n\times n$, triangulaire, est inversible
si et seulement si ses éléments diagonaux sont tous non nuls.
\end{theoreme}

\begin{proof}
Supposons que $A$ soit triangulaire supérieure.

\begin{itemize}
  \item Si les éléments de la diagonale sont tous non nuls, alors
  la matrice $A$ est déjà sous la forme échelonnée.
En multipliant chaque ligne $i$ par l'inverse de l'élément diagonal $a_{ii}$,
on obtient des $1$ sur la diagonale.
De ce fait, la forme échelonnée réduite de $A$ sera la matrice identité.
Le théorème \ref{th:invequi} permet de conclure que $A$ est inversible.

  \item Inversement, supposons qu'au moins l'un des éléments diagonaux soit nul et notons $a_{\ell\ell}$
le premier élément nul de la diagonale. En multipliant les lignes $1$ à
$\ell-1$ par l'inverse de leur élément diagonal, on obtient une matrice de la forme
\[\left( \begin{array}{ccccccc}
1 & * & \cdots &&& \cdots& * \\
0& \ddots  & * &\cdots &&\cdots & * \\
0 & 0 & 1 & * && \cdots &* \\
0 & \cdots & 0 & {\color{myred}0} &*&\cdots &*\\
0 & \cdots & 0& 0& * & \cdots &*  \\
\vdots & \vdots & \vdots & \cdots &0&\ddots & \vdots \\
0& \cdots &&&\cdots&0&*
\end{array}\right).\]


Il est alors clair que la colonne numéro $\ell$ de la forme échelonnée réduite ne
contiendra pas de $1$ comme pivot. La forme échelonnée réduite de $A$
ne peut donc pas être $I_n$ et par le théorème \ref{th:invequi},
$A$ n'est pas inversible.
\end{itemize}

Dans le cas d'une matrice triangulaire inférieure, on utilise
la transposition (qui fait l'objet de la section suivante) et on
obtient une matrice triangulaire supérieure. On applique alors
la démonstration ci-dessus.
\end{proof}



%---------------------------------------------------------------
\subsection{La transposition}

Soit $A$ la matrice de taille $n\times p$
$$
A = \left(
\begin{array}{cccc}
a_{11} & a_{12} & \dots & a_{1p}\\
a_{21} & a_{22} & \dots & a_{2p}\\
\vdots & \vdots &&\vdots\\
a_{n1} & a_{n2} & \dots & a_{np}
\end{array}\right).
$$

\begin{definition}
On appelle \defi{matrice transposée}\index{matrice!transposee@transposée}\index{transposee@transposée} de $A$ la matrice $A^T$ de taille $p \times n$
définie par :
$$
A^T = \left(
\begin{array}{cccc}
a_{11} & a_{21} & \dots & a_{n1}\\
a_{12} & a_{22} & \dots & a_{n2}\\
\vdots & \vdots &&\vdots\\
a_{1p} & a_{2p} &\dots & a_{np}
\end{array}\right)\, .
$$
\end{definition}

Autrement dit : le coefficient à la place $(i,j)$ de $A^T$  est $a_{ji}$.
Ou encore la $i$-ème ligne de $A$ devient la $i$-ème colonne de $A^{T}$
(et réciproquement la $j$-ème colonne de $A^T$ est la $j$-ème ligne de $A$).

\bigskip

{\bf Notation :} La transposée de la matrice $A$ se note aussi souvent $^{t\!}A$.

\begin{exemple}

$$
\begin{pmatrix}
1&2&3\\
4&5&-6\\
-7&8&9\\
\end{pmatrix}^T
=
\begin{pmatrix}
1&4&-7\\
2&5&8\\
3&-6&9\\
\end{pmatrix}$$
$$\begin{pmatrix}
 0 & 3\\
 1 & -5\\
 -1 & 2
\end{pmatrix}^T=
\begin{pmatrix}
 0 & 1 & -1\\
 3 &-5 & 2
\end{pmatrix}
\qquad\qquad
(1\quad -2\quad 5)^T  =
\begin{pmatrix}
 1 \\
 -2\\
 5
\end{pmatrix}
$$
\end{exemple}

L'opération de transposition obéit aux règles suivantes:
\begin{theoreme}
\sauteligne
 \begin{enumerate}
    \item $(A + B)^T = A^T + B^T$
    \item $(\alpha A)^T = \alpha A^T $
    \item $(A^T)^T = A$
    \item \myboxinline{$(AB)^T = B^T A^T$}
    \item Si $A$ est inversible, alors $A^T$ l'est aussi et on a $(A^T)^{-1}=(A^{-1})^T$.
  \end{enumerate}
\label{theotranspo}
\end{theoreme}

Notez bien l'inversion : $(AB)^T = B^T A^T$,
comme pour $(AB)^{-1} = B^{-1} A^{-1}$.

%---------------------------------------------------------------

\subsection{La trace}

Dans le cas d'une matrice carrée de taille $n\times n$,
les éléments $a_{11}$, $a_{22}, \ldots, a_{nn}$ sont appelés les \defi{éléments diagonaux}.

Sa \defi{diagonale principale}\index{diagonale} est la
diagonale $(a_{11},a_{22}, \ldots, a_{nn})$.
  \[
\begin{pmatrix}
 {\color{myred}a_{11}} & a_{12} & \dots & a_{1n}\\
 a_{21} & {\color{myred}a_{22}} & \dots & a_{2n}\\
 \vdots& \vdots & {\color{myred}\ddots}  & \vdots\\
 a_{n1} & a_{n2} & \dots & {\color{myred}a_{nn}}
\end{pmatrix}
 \]


\begin{definition}
La \defi{trace}\index{trace}\index{matrice!trace} de la matrice $A$ est
le nombre obtenu en additionnant les éléments diagonaux de $A$.
Autrement dit,
\mybox{$\tr A = a_{11} + a_{22} + \cdots +a_{nn}.$}
\end{definition}

\begin{exemple}
\sauteligne
\begin{itemize}
  \item Si
$ A = \left( \begin{smallmatrix}
2 & 1\\
0& 5
\end{smallmatrix}\right)$,
alors $\tr A = 2 + 5 = 7$.

  \item Pour $B = \left(\begin{smallmatrix}
1 & 1 &2\\
5&2&8\\
11 & 0 & -10
\end{smallmatrix}\right)$,  $\tr B = 1 + 2 -10 =-7$.
\end{itemize}
\end{exemple}

\begin{theoreme}
Soient $A$ et $B$ deux matrices $n \times n$. Alors :
\begin{enumerate}
\item $\tr(A + B)$ = $\tr A$ + $\tr B$,
\item $\tr(\alpha A)$ = $\alpha$ $\tr A$ pour tout $\alpha \in \Kk$,
\item $\tr(A^T)$ = $\tr A $,
\item $\tr(AB)$ = $\tr(BA)$.
\end{enumerate}
\end{theoreme}


\begin{proof}
~
\begin{enumerate}
\item Pour tout $1 \leq i \leq n$, le coefficient $(i,i)$ de $A+B$ est $a_{ii} + b_{ii}$.
Ainsi, on a bien $\tr(A + B)$ = $\tr(A)$ + $\tr(B)$.

\item On a $\tr(\alpha A) = \alpha a_{11} +\dots+ \alpha a_{nn}
= \alpha (a_{11} +\dots+ a_{nn})= \alpha \tr A$.

\item \'Etant donné que la transposition ne change pas les éléments diagonaux,
la trace de $A$ est égale à la trace de $A^T$.

\item Notons $c_{ij}$ les coefficients de $AB$.
Alors par définition
$$c_{ii} = a_{i1}b_{1i} + a_{i2}b_{2i} +\dots+ a_{in}b_{ni}.$$
Ainsi,
$$\begin{array}{cccccc}
\tr(AB) & = & a_{11}b_{11} & + a_{12}b_{21} & +\cdots &+ a_{1n}b_{n1}\\
 & &+ a_{21}b_{12} & + a_{22}b_{22}& +\cdots & + a_{2n}b_{n2}\\
& & \vdots &&&\\
& & +a_{n1}b_{1n} & + a_{n2}b_{2n}& +\cdots & + a_{nn}b_{nn}.
\end{array}$$

On peut réarranger les termes pour obtenir

$$\begin{array}{cccccc}
\tr(AB) &=& a_{11}b_{11} & + a_{21}b_{12} & +\cdots &+ a_{n1}b_{1n}\\
&&+  a_{12}b_{21} & + a_{22}b_{22}& +\cdots & + a_{n2}b_{2n}\\
 && \vdots &&\\
 &&+ a_{1n}b_{n1} & + a_{2n}b_{n2}& +\cdots & + a_{nn}b_{nn}.
\end{array}$$

En utilisant la commutativité de la multiplication dans $\Kk$, la première ligne devient
$$b_{11}a_{11} + b_{12}a_{21} +\dots+ b_{1n}a_{n1}$$
qui vaut le coefficient $(1,1)$ de $BA$. On note $d_{ij}$ les coefficients de $BA$.
En faisant de même avec les
autres lignes, on voit finalement que
$$\tr(AB) = d_{11} +\dots+ d_{nn} = \tr(BA).$$
\end{enumerate}
\end{proof}


%---------------------------------------------------------------
\subsection{Matrices symétriques}


\begin{definition}
Une matrice $A$ de taille $n \times n$ est \defi{symétrique}\index{matrice!symetrique@symétrique} si elle est égale
à sa transposée, c'est-à-dire si
$$A = A^T,$$
ou encore si $a_{ij}=a_{ji}$ pour tout $i,j=1, \ldots, n$.
Les coefficients sont donc symétriques par rapport à la diagonale.
\end{definition}

\begin{exemple} Les matrices suivantes sont symétriques :
 $$ \begin{pmatrix}
 0 & 2\\
 2 & 4\end{pmatrix} \qquad
\begin{pmatrix}
 -1 & 0 & 5\\
 0 & 2 & -1\\
 5 & -1 & 0\end{pmatrix}
$$
\end{exemple}

\begin{exemple}
Pour une matrice $B$ quelconque, les matrices $B \cdot B^T$ et $B^T \cdot B$ sont symétriques.

Preuve : $(BB^T)^T = (B^T)^T B^T = BB^T$. Idem pour $B^TB$.
\end{exemple}


%---------------------------------------------------------------
\subsection{Matrices antisymétriques}

\begin{definition}
Une matrice $A$ de taille $n \times n$ est \defi{antisymétrique}\index{matrice!antisymetrique@antisymétrique} si
$$A^T = -A,$$
c'est-à-dire si $a_{ij} = -a_{ji}$ pour tout $i,j=1, \ldots, n.$
\end{definition}

\begin{exemple}
$$\begin{pmatrix}
0 & -1\\
1 & 0
\end{pmatrix} \qquad
\begin{pmatrix}
0 & 4 & 2\\
-4 & 0 & -5\\
-2 & 5 & 0
\end{pmatrix}$$
\end{exemple}

Remarquons que les éléments diagonaux d'une matrice antisymétrique sont toujours tous nuls.

\begin{exemple}
Toute matrice est la somme d'une matrice symétrique et d'une matrice antisymétrique.

\medskip

Preuve :
Soit $A$ une matrice. Définissons $B=\frac12(A+A^T)$ et $C=\frac12(A-A^T)$.
Alors d'une part $A=B+C$ ; d'autre part $B$ est
symétrique, car $B^T=\frac12(A^T+(A^T)^T)=\frac12(A^T+A)=B$ ; et enfin $C$ est
antisymétrique, car $C^T=\frac12(A^T-(A^T)^T)=-C$.

Exemple :
$$\text{ Pour } \quad A=\begin{pmatrix} 2 & 10 \\ 8 & -3 \end{pmatrix}
\qquad\text{ alors }\qquad A \ \ = \ \ \underbrace{\begin{pmatrix}
2 & 9\\
9 & -3\end{pmatrix}}_{\text{symétrique}}
\quad\ \  + \quad \underbrace{\begin{pmatrix}
0 & 1\\
-1 & 0\end{pmatrix}}_{\text{antisymétrique}}.$$
\end{exemple}



%---------------------------------------------------------------
%\subsection{Mini-exercices}

\begin{miniexercices}
\sauteligne
\begin{enumerate}
  \item Montrer que la somme de deux matrices triangulaires supérieures reste triangulaire supérieure.
  Montrer que c'est aussi valable pour le produit.

  \item Montrer que si $A$ est triangulaire supérieure, alors $A^T$ est triangulaire inférieure.
  Et si $A$ est diagonale ?

  \item Soit $A = \left(\begin{smallmatrix}x_1 \\ x_2 \\ \vdots \\ x_n \end{smallmatrix} \right)$.
  Calculer $A^T \cdot A$, puis $A\cdot A^T$.

  \item Soit $A=\left(\begin{smallmatrix}a&b\\c&d\end{smallmatrix}\right)$.
  Calculer $\tr(A\cdot A^T)$.

  \item Soit $A$ une matrice de taille $2\times 2$ inversible. Montrer
  que si $A$ est symétrique, alors $A^{-1}$ aussi. Et si $A$ est antisymétrique ?

  \item Montrer que la décomposition d'une matrice sous la forme
  \og symétrique + antisymétrique \fg{} est unique.

\end{enumerate}
\end{miniexercices}

%%%%%%%%%%%%%%%%%%%%%%%%%%%%%%%%%%%%%%%%%%%%%%%%%%%%%%%%%%%%%%%%
%%%%%%%%%%%%%%%%%%%%%%%%%%%%%%%%%%%%%%%%%%%%%%%%%%%%%%%%%%%%%%%%

\bigskip
\bigskip

\auteurs{
\begin{itemize}
  \item D'après un cours de Eva Bayer-Fluckiger, Philippe Chabloz, Lara Thomas
  de l'\'Ecole Polytechnique Fédérale de Lausanne,

  \item et un cours de Sophie Chemla de l'université Pierre et Marie Curie,
  reprenant des parties de cours de H. Ledret et d'une équipe de l'université de Bordeaux animée par J. Queyrut,

  \item mixés et révisés par Arnaud Bodin, relu par Vianney Combet.
\end{itemize}
}

\finchapitre
\end{document}

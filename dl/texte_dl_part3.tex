
%%%%%%%%%%%%%%%%%% PREAMBULE %%%%%%%%%%%%%%%%%%


\documentclass[12pt]{article}

\usepackage{amsfonts,amsmath,amssymb,amsthm}
\usepackage[utf8]{inputenc}
\usepackage[T1]{fontenc}
\usepackage[francais]{babel}


% packages
\usepackage{amsfonts,amsmath,amssymb,amsthm}
\usepackage[utf8]{inputenc}
\usepackage[T1]{fontenc}
%\usepackage{lmodern}

\usepackage[francais]{babel}
\usepackage{fancybox}
\usepackage{graphicx}

\usepackage{float}

%\usepackage[usenames, x11names]{xcolor}
\usepackage{tikz}
\usepackage{datetime}

\usepackage{mathptmx}
%\usepackage{fouriernc}
%\usepackage{newcent}
\usepackage[mathcal,mathbf]{euler}

%\usepackage{palatino}
%\usepackage{newcent}


% Commande spéciale prompteur

%\usepackage{mathptmx}
%\usepackage[mathcal,mathbf]{euler}
%\usepackage{mathpple,multido}

\usepackage[a4paper]{geometry}
\geometry{top=2cm, bottom=2cm, left=1cm, right=1cm, marginparsep=1cm}

\newcommand{\change}{{\color{red}\rule{\textwidth}{1mm}\\}}

\newcounter{mydiapo}

\newcommand{\diapo}{\newpage
\hfill {\normalsize  Diapo \themydiapo \quad \texttt{[\jobname]}} \\
\stepcounter{mydiapo}}


%%%%%%% COULEURS %%%%%%%%%%

% Pour blanc sur noir :
%\pagecolor[rgb]{0.5,0.5,0.5}
% \pagecolor[rgb]{0,0,0}
% \color[rgb]{1,1,1}



%\DeclareFixedFont{\myfont}{U}{cmss}{bx}{n}{18pt}
\newcommand{\debuttexte}{
%%%%%%%%%%%%% FONTES %%%%%%%%%%%%%
\renewcommand{\baselinestretch}{1.5}
\usefont{U}{cmss}{bx}{n}
\bfseries

% Taille normale : commenter le reste !
%Taille Arnaud
%\fontsize{19}{19}\selectfont

% Taille Barbara
%\fontsize{21}{22}\selectfont

%Taille François
%\fontsize{25}{30}\selectfont

%Taille Pascal
%\fontsize{25}{30}\selectfont

%Taille Laura
%\fontsize{30}{35}\selectfont


%\myfont
%\usefont{U}{cmss}{bx}{n}

%\Huge
%\addtolength{\parskip}{\baselineskip}
}


% \usepackage{hyperref}
% \hypersetup{colorlinks=true, linkcolor=blue, urlcolor=blue,
% pdftitle={Exo7 - Exercices de mathématiques}, pdfauthor={Exo7}}


%section
% \usepackage{sectsty}
% \allsectionsfont{\bf}
%\sectionfont{\color{Tomato3}\upshape\selectfont}
%\subsectionfont{\color{Tomato4}\upshape\selectfont}

%----- Ensembles : entiers, reels, complexes -----
\newcommand{\Nn}{\mathbb{N}} \newcommand{\N}{\mathbb{N}}
\newcommand{\Zz}{\mathbb{Z}} \newcommand{\Z}{\mathbb{Z}}
\newcommand{\Qq}{\mathbb{Q}} \newcommand{\Q}{\mathbb{Q}}
\newcommand{\Rr}{\mathbb{R}} \newcommand{\R}{\mathbb{R}}
\newcommand{\Cc}{\mathbb{C}} 
\newcommand{\Kk}{\mathbb{K}} \newcommand{\K}{\mathbb{K}}

%----- Modifications de symboles -----
\renewcommand{\epsilon}{\varepsilon}
\renewcommand{\Re}{\mathop{\text{Re}}\nolimits}
\renewcommand{\Im}{\mathop{\text{Im}}\nolimits}
%\newcommand{\llbracket}{\left[\kern-0.15em\left[}
%\newcommand{\rrbracket}{\right]\kern-0.15em\right]}

\renewcommand{\ge}{\geqslant}
\renewcommand{\geq}{\geqslant}
\renewcommand{\le}{\leqslant}
\renewcommand{\leq}{\leqslant}

%----- Fonctions usuelles -----
\newcommand{\ch}{\mathop{\mathrm{ch}}\nolimits}
\newcommand{\sh}{\mathop{\mathrm{sh}}\nolimits}
\renewcommand{\tanh}{\mathop{\mathrm{th}}\nolimits}
\newcommand{\cotan}{\mathop{\mathrm{cotan}}\nolimits}
\newcommand{\Arcsin}{\mathop{\mathrm{Arcsin}}\nolimits}
\newcommand{\Arccos}{\mathop{\mathrm{Arccos}}\nolimits}
\newcommand{\Arctan}{\mathop{\mathrm{Arctan}}\nolimits}
\newcommand{\Argsh}{\mathop{\mathrm{Argsh}}\nolimits}
\newcommand{\Argch}{\mathop{\mathrm{Argch}}\nolimits}
\newcommand{\Argth}{\mathop{\mathrm{Argth}}\nolimits}
\newcommand{\pgcd}{\mathop{\mathrm{pgcd}}\nolimits} 

\newcommand{\Card}{\mathop{\text{Card}}\nolimits}
\newcommand{\Ker}{\mathop{\text{Ker}}\nolimits}
\newcommand{\id}{\mathop{\text{id}}\nolimits}
\newcommand{\ii}{\mathrm{i}}
\newcommand{\dd}{\mathrm{d}}
\newcommand{\Vect}{\mathop{\text{Vect}}\nolimits}
\newcommand{\Mat}{\mathop{\mathrm{Mat}}\nolimits}
\newcommand{\rg}{\mathop{\text{rg}}\nolimits}
\newcommand{\tr}{\mathop{\text{tr}}\nolimits}
\newcommand{\ppcm}{\mathop{\text{ppcm}}\nolimits}

%----- Structure des exercices ------

\newtheoremstyle{styleexo}% name
{2ex}% Space above
{3ex}% Space below
{}% Body font
{}% Indent amount 1
{\bfseries} % Theorem head font
{}% Punctuation after theorem head
{\newline}% Space after theorem head 2
{}% Theorem head spec (can be left empty, meaning ‘normal’)

%\theoremstyle{styleexo}
\newtheorem{exo}{Exercice}
\newtheorem{ind}{Indications}
\newtheorem{cor}{Correction}


\newcommand{\exercice}[1]{} \newcommand{\finexercice}{}
%\newcommand{\exercice}[1]{{\tiny\texttt{#1}}\vspace{-2ex}} % pour afficher le numero absolu, l'auteur...
\newcommand{\enonce}{\begin{exo}} \newcommand{\finenonce}{\end{exo}}
\newcommand{\indication}{\begin{ind}} \newcommand{\finindication}{\end{ind}}
\newcommand{\correction}{\begin{cor}} \newcommand{\fincorrection}{\end{cor}}

\newcommand{\noindication}{\stepcounter{ind}}
\newcommand{\nocorrection}{\stepcounter{cor}}

\newcommand{\fiche}[1]{} \newcommand{\finfiche}{}
\newcommand{\titre}[1]{\centerline{\large \bf #1}}
\newcommand{\addcommand}[1]{}
\newcommand{\video}[1]{}

% Marge
\newcommand{\mymargin}[1]{\marginpar{{\small #1}}}



%----- Presentation ------
\setlength{\parindent}{0cm}

%\newcommand{\ExoSept}{\href{http://exo7.emath.fr}{\textbf{\textsf{Exo7}}}}

\definecolor{myred}{rgb}{0.93,0.26,0}
\definecolor{myorange}{rgb}{0.97,0.58,0}
\definecolor{myyellow}{rgb}{1,0.86,0}

\newcommand{\LogoExoSept}[1]{  % input : echelle
{\usefont{U}{cmss}{bx}{n}
\begin{tikzpicture}[scale=0.1*#1,transform shape]
  \fill[color=myorange] (0,0)--(4,0)--(4,-4)--(0,-4)--cycle;
  \fill[color=myred] (0,0)--(0,3)--(-3,3)--(-3,0)--cycle;
  \fill[color=myyellow] (4,0)--(7,4)--(3,7)--(0,3)--cycle;
  \node[scale=5] at (3.5,3.5) {Exo7};
\end{tikzpicture}}
}



\theoremstyle{definition}
%\newtheorem{proposition}{Proposition}
%\newtheorem{exemple}{Exemple}
%\newtheorem{theoreme}{Théorème}
\newtheorem{lemme}{Lemme}
\newtheorem{corollaire}{Corollaire}
%\newtheorem*{remarque*}{Remarque}
%\newtheorem*{miniexercice}{Mini-exercices}
%\newtheorem{definition}{Définition}




%definition d'un terme
\newcommand{\defi}[1]{{\color{myorange}\textbf{\emph{#1}}}}
\newcommand{\evidence}[1]{{\color{blue}\textbf{\emph{#1}}}}



 %----- Commandes divers ------

\newcommand{\codeinline}[1]{\texttt{#1}}

%%%%%%%%%%%%%%%%%%%%%%%%%%%%%%%%%%%%%%%%%%%%%%%%%%%%%%%%%%%%%
%%%%%%%%%%%%%%%%%%%%%%%%%%%%%%%%%%%%%%%%%%%%%%%%%%%%%%%%%%%%%



\begin{document}

\debuttexte

%%%%%%%%%%%%%%%%%%%%%%%%%%%%%%%%%%%%%%%%%%%%%%%%%%%%%%%%%%%
\diapo

Passons au coeur de ce chapitre : comment calculer des développements limités ?

\change

Maintenant que l'on connaît les DL des fonctions usuelles, nous allons voir
comment les manipuler.


\change

On comme par la somme et le produit de deux DL

\change

Puis la composition
 
\change 

la division

\change

et enfin l'intégration.

%%%%%%%%%%%%%%%%%%%%%%%%%%%%%%%%%%%%%%%%%%%%%%%%%%%%%%%%%%%
\diapo

Partons de deux fonctions  $f$ et $g$ qui admettent des DL en $0$ à l'ordre $n$ 


\change

 Alors la somme $f+g$ admet aussi un DL en $0$ l'ordre $n$ 

\change

et ce DL est :
$(c_0+d_0)+(c_1+d_1)x+\cdots+(c_n+d_n)x^n$ + un reste.

Chaque coefficient est tout simplement la somme du coefficient de $f$ +
le coefficient de $g$, et le reste est la somme des restes.

\change

De même le produit $f\times g$ admet un DL en $0$ l'ordre $n$

\change

Et voici comme on calcule le  DL du produit


 
on effectue le produit de la partie polynomiale de $f$
par la partie polynomiale  de $g$

Mais il va y avoir des monômes dont le degré  dépasse $n$.


On ne retient que les termes de degré $\le n$.



En résumé 
$T_n$ est le produit des parties polynomiales 
tronqué à l'ordre $n$

\change

Et \defi{Tronquer} un polynôme à l'ordre $n$ signifie que l'on conserve seulement 
les monômes de degré $\le n$. 


%%%%%%%%%%%%%%%%%%%%%%%%%%%%%%%%%%%%%%%%%%%%%%%%%%%%%%%%%%%
\diapo

Voyons un exemple. Nous souhaitons calculer le DL de 
$\cos x \times \sqrt{1+x}$ en $0$  à l'ordre $2$.  

\change

On connaît le DL de $\cos x$ en $0$ c'est 
$1-\frac{1}{2}x^2$ + le reste


\change

et celui 
$\sqrt{1+x}$ est $1+\frac{1}{2}x-\frac{1}{8}x^2$ + le reste

\change

On développe cette expression,

on multiplie d'abord tout le second membre par $1$

\change

puis par $-\frac{1}{2}x^2$

\change

puis par le premier reste.


\change

On reprend


\change 

en développant chacune des lignes

\change


\change

On va maintenant regrouper les termes par degré,

on rassemble d'abord les termes constants : il n'y a que $1$

puis les termes de degré $1$ : il n'y a que $\frac12 x$

et ensuite les termes de degré $2$ : il y a $-\frac{1}{8}x^2$ et $-\frac{1}{2}x^2$.

\change

Que va t-on faire de tous les autres termes ?

Remarquez que les autres termes sont

$x^2\epsilon_2(x)$ $-\frac{1}{4}x^3$+$\frac{1}{16}x^4$ et des $x^4 \epsilon(x)$

ils sont plus grand que l'ordre $2$.

Plus précisément ils sont bien de la forme $x^2$ fois une fonction qui tend vers $0$.
On va donc tous les rassembler dans un seul reste.

\change

C'est presque fini :

la partie polynomiale du DL est donc $1+\frac{1}{2}x-\frac{5}{8}x^2$

\change

et on note $x^2\epsilon(x)$ tous les termes d'ordre strictement plus grand que $2$,
où $\epsilon(x)$ tend vers $0$ lorsque $x$ tend vers $0$.

Voici donc le DL $\cos x \times \sqrt{1+x}$ en $0$  à l'ordre $2$,
c'est $1+\frac{1}{2}x-\frac{5}{8}x^2$ + un reste.

%%%%%%%%%%%%%%%%%%%%%%%%%%%%%%%%%%%%%%%%%%%%%%%%%%%%%%%%%%%
\diapo

Les calculs de l'exemple précédent semblent à première vue plutôt pénibles et longuets.
Il n'en n'est rien. On a en fait écrit beaucoup de choses superflues, 
qui à la fin sont dans le reste et n'avaient pas besoin d'être explicitées !


Avec l'habitude les calculs se font très vite car on n'écrit plus les termes inutiles. 
Voici le même calcul avec cette fois la notation <<petit o>>.


\change

On commence comme avant en écrivant chacun des DL à l'ordre $2$.

\change

on développe ligne par ligne

\change

et dès qu'apparaît un terme $x^3$, $x^4$ ou $x^2\epsilon_1(x)$ on écrit juste $o(x^2)$

il reste juste $-\frac{1}{2}x^2 + o(x^2)$

\change
 Et pour le dernier produit développé tous les termes sont d'ordre plus grand que $x^2$
donc on écrit seulement $o(x^2)$.

\change

Il ne reste plus qu'à regrouper les termes et c'est terminé !



%%%%%%%%%%%%%%%%%%%%%%%%%%%%%%%%%%%%%%%%%%%%%%%%%%%%%%%%%%%
\diapo

Passons à une opération plus délicate : la composition.

On reprend nos deux fonctions $f$ et $g$ ayant des DL à l'ordre $n$.

\change

On note $C(x)$ la partie polynomiale de $f$ 

\change 

et $D(x)$ celle de $g$ 

\change 

Nous supposons que $g(0)=0$, autrement dit le terme constant $d_0$ du $DL$ de $g$ est nul.

Alors la composition $f\circ g$ admet un DL en $0$ à l'ordre $n$

\change

Comment se calcule ce DL ?

la partie polynomiale de $f\circ g$ est
le polynôme tronqué à l'ordre $n$ de la composition $C(D(x))$

Donc on prend la partie polynomiale de $f$, on prend celle de $g$, on compose ces deux polynômes
et on oublie tous les termes de degré plus grand que $n$.

%%%%%%%%%%%%%%%%%%%%%%%%%%%%%%%%%%%%%%%%%%%%%%%%%%%%%%%%%%%
\diapo

Nous allons décortiquer un premier exemple en détaillant pas à pas les étapes.

On souhaite calculer le DL de $\sin\big(\ln(1+x)\big)$ en $0$ à l'ordre $3$

\change

On pose ici $f(u)=\sin u$ et $g(x)=\ln(1+x)$ 

(pour plus de clarté il est préférable de donner des noms différents aux variables de deux fonctions,
 ici $x$ et $u$).

\change 

On a bien $f\circ g(x) = \sin\big(\ln(1+x)\big)$ 

\change

et $g(0)=0$.

\change 

On écrit le DL à l'ordre 3 de $f(u)=\sin u = u-\frac{u^3}{3!}+u^3\epsilon_1(u)$ pour $u$ proche de $0$.

\change 

Puis le DL de $g$ : $g(x)=\ln(1+x)=x-\frac{x^2}{2}+\frac{x^3}{3}+x^3\epsilon_2(x)$ pour $x$ proche de $0$. 

Et nous allons noter $u=g(x)$.

\change

On aura besoin de calculer un DL à l'ordre 3 de $u^2$ 
(que l'on obtient bien sûr comme le DL du produit
 $u\times u$):

$u^2 = \big(x-\frac{x^2}{2}+\frac{x^3}{3}+x^3\epsilon_2(x)\big)^2$

\change

$ = x^2-x^3+x^3\epsilon_3(x)$

\change 

et aussi $u^3$ qui est $u \times u^2$, 

\change 

on trouve $u^3=x^3+x^3\epsilon_4(x)$. 

\change

Rassemblons tout cela pour conclure 

$=f\circ g(x)$

\change

c'est donc

$f(u)$

\change

qui vaut $u-\frac{u^3}{3!}+u^3\epsilon_1(u)$

\change

et on remplace $u$ et $u^3$
par les expressions obtenues 

$\big(x-\frac{1}{2}x^{2}+\frac{1}{3}x^3\big) -\frac16 x^3  +x^3\epsilon(x)$

\change

On regroupe les termes et obtient le DL de $\sin\big(\ln(1+x)\big)$

$x-\frac{1}{2}x^{2}+\frac16 x^3  +x^3\epsilon(x)$. 

Ne vous inquiétez pas avec un peu entraînement vous écrirez directement les calculs 
à partir de l'étape $6$.


%%%%%%%%%%%%%%%%%%%%%%%%%%%%%%%%%%%%%%%%%%%%%%%%%%%%%%%%%%%
\diapo


On passe à un autre exemple de composition pour calculer le
DL $\sqrt{\cos x}$ en $0$ à l'ordre $4$. 

\change

Pour la racine carrée, le DL que l'on connaît qui s'en rapproche le plus 
est celui $\sqrt{1+u}$.

Ecrivons le DL de $f(u)=\sqrt{1+u}$ en $u=0$ à l'ordre $2$ :

\change

On utilise cette fois la notation <<petit o>>.
$\sqrt{1+u}=1+\frac{1}{2}u-\frac{1}{8}u^2 + o(u^2)$.

On verra plus tard pourquoi l'ordre $2$ suffit alors que l'on souhaite un DL de notre fonction à l'ordre $4$.


\change

On pose $u(x)=\cos x-1$ 

\change

alors on a $h(x)= f\big(u(x)\big)$ et aussi $u(0)=0$. 

\change

On connaît le DL de $\cos x$ donc celui de $u=\cos x -1$ en $x=0$ à l'ordre $4$ est :

$-\frac{1}{2}x^2+\frac{1}{24}x^4+o(x^4)$.

\change

On en déduit celui de $u^2$ : $\frac{1}{4}x^4 + o(x^4)$.

\change

Passons au calcul du DL de $h(x)=\sqrt{\cos x}$ 

\change

nous avons écrit la fonction $h(x)$
sous la forme $f\big(u\big)$ 

\change

dont le DL $1+\frac{1}{2}u-\frac{1}{8}u^2 + o(u^2)$

\change 

on remplace $u$ et $u^2$ par leurs expressions.

\change

et on regroupe les termes par degré.


\change

En conclusion le DL $\sqrt{\cos x}$
est $1-\frac{1}{4}x^2-\frac{1}{96}x^4+o(x^4)$


Remarquez que le DL de $u=cos x -1$ commence par $x^2$.
Donc un DL de $f$ à l'ordre $2$ en $u$ donne en fait 
un DL de $f$ à l'ordre $4$.




%%%%%%%%%%%%%%%%%%%%%%%%%%%%%%%%%%%%%%%%%%%%%%%%%%%%%%%%%%%
\diapo

Après la multiplication et la composition voici la division.
On souhaite calculer le DL de $f/g$.
En fait calculer le quotient c'est trouver le DL de l'inverse $1/g$ et multiplier par $f$.

\change

La méthode proposée ici repose sur le développement limité de 
de $\frac{1}{1+u}$ qui est $1-u+u^2-u^3+\cdots$

\change

Dans un premier cas si le terme constant du DL de $g$ est $1$ alors on note $u$
les termes suivants et $g$ s'écrit alors $1+u$.

\change

Le quotient $f/g$ n'est autre $f \times \frac{1}{1+u}$.

On calcule le DL de $\frac{1}{1+u}$ par cette formule et on multiplie par le DL de $f$.

\change

Si le DL de $g$ commence par une constante $d_0$ qui n'est pas $1$ et qui n'est non plus pas zéro.

Alors on divise par $d_0$ pour obtenir à une expression $1$ + qq chose.

Ce qui nous ramène au premier cas.

\change 

Si $d_0=0$ alors on peut factoriser le DL de $g$ par $x$ ou $x^2$ ou $x^3$,...

On factorise par la plus grande puissance possible
et cela nous ramène au premier ou second cas.



%%%%%%%%%%%%%%%%%%%%%%%%%%%%%%%%%%%%%%%%%%%%%%%%%%%%%%%%%%%
\diapo

Nous allons voir un exemple pour chacune des $3$ situations précédentes.

Commençons par le DL de $\tan x$ comme quotient de $\sin x$ par $\cos x$.

\change

Tout d'abord le DL de $\sin x$ en $0$ à l'ordre $5$ est $x-\frac{x^3}{6}+\frac{x^5}{120}+x^5\epsilon(x)$.

\change

Et celui de $\cos x$ est  $\cos  x=1-\frac{x^2}{2}+\frac{x^4}{24} +x^5\epsilon(x)$
(le terme d'ordre $5$ est nul).

\change

Nous posons $u= -\frac{x^2}{2}+\frac{x^4}{24} +x^5\epsilon(x)$

de sorte que le dénominateur $\cos x$ s'écrive $1+u$.

\change

Nous aurons besoin de connaître le DL de $u^2$ :

\change

après calculs on trouve :
$ \frac{x^4}{4}+x^5\epsilon(x)$

\change

et pour $u^3$, il est en fait déjà d'ordre supérieur à $5$.

(J'ai abusivement noté $\epsilon(x)$ pour les différents restes.)


\change

Passons au calcul de $1/\cos x$.

Nous avons écrit $\frac{1}{\cos x}$ sous la forme $1/(1+u)$

\change

qui se développe sous la forme 
 $1-u+u^2-u^3+u^3\epsilon(u)$

\change

On remplace $u$, $u^2$ et $u^3$ par ce que l'on a déjà calculé.

\change

et on regroupe.

\change


Ce n'est pas fini :
$\tan x= \sin x \times \frac1{\cos x} $

\change

On connaît maintenant le DL de chacun des facteurs.

\change

Que l'on multiplie pour obtenir après calculs
que le DL de $\tan x$
à l'ordre $5$ est 
$x +\frac{x^3}{3}+\frac{2}{15}x^5+x^5\epsilon(x)$.

Il est indispensable de savoir refaire cet exemple
et il est bon d'apprendre que jusqu'à l'ordre $3$
le DL de $\tan x$ est $x +\frac{x^3}{3}$.





%%%%%%%%%%%%%%%%%%%%%%%%%%%%%%%%%%%%%%%%%%%%%%%%%%%%%%%%%%%
\diapo

Continuons nos exemples.
avec le DL de $\frac{1+x}{2+x}$.

Ici le dénominateur commence par $2$.

\change

On le factorise par $2$ afin de faire apparaître l'expression
$\frac{1}{1+u}$

\change

Que l'on développe en $1-u+u^2-u^3+u^4$

\change

On calcule ensuite le produit et on regroupe les termes pour trouver :

$\frac12+\frac{x}{4}-\frac{x^2}{8}+\frac{x^3}{16}-\frac{x^4}{32} + o(x^4)$.


(pause)
 
\change 

On termine avec le quotient de $\frac{\sin x}{\sh x}$.


\change

On écrit les deux DL à l'ordre $5$ en faisant bien attention aux signes.

Notez que c'est une forme indéterminée en $x=0$ car les deux fonctions s'annulent.

\change

On factorise chacune d'elle par une puissance de $x$ ici seulement par $x$.

\change

La fraction se simplifie et le dénominateur à maintenant une constante égale à $1$.

On retombe en terrain connu : on développe le quotient puis on multiplie par le numérateur

\change

\change

pour trouver que 
$\frac{\sin x}{\sh x}= 1-\frac{x^2}{2}+\frac{x^4}{18} + o(x^4)$.


%%%%%%%%%%%%%%%%%%%%%%%%%%%%%%%%%%%%%%%%%%%%%%%%%%%%%%%%%%%
\diapo

[petit f, grand F]

Terminons par l'intégration des développements limités.

Soit $f : I\to \Rr$ une fonction de classe $\mathcal{C}^n$ dont le DL 
en $0$ à l'ordre $n$ est le suivant :

\change

On note $F$ est une primitive de $f$.

Alors premièrement $F$ admet un DL en $0$ à l'ordre $n+1$ 

\change

et deuxièmement ce DL s'écrit 
$$c_0x+c_1\frac{x^2}{2}+ c_2\frac{x^3}{3}+\cdots+c_n\frac{x^{n+1}}{n+1} 
+x^{n+1}\eta(x)$$  

\change

Comme une primitive est déterminée à une constante près,
alors le terme constant du DL est ici  $F(0)$

\change

Cela signifie que l'on intègre la partie polynomiale terme à terme pour obtenir le DL de $F(x)$ :
le monôme $c_k x^k$ du DL de $f$ devient
$c_k\frac{x^{k+1}}{k+1}$.

Et l'on n'oublie pas d'ajouter le terme constant $F(0)$




%%%%%%%%%%%%%%%%%%%%%%%%%%%%%%%%%%%%%%%%%%%%%%%%%%%%%%%%%%%
\diapo

Passons aux exemples.

Tout d'abord comment calculer le DL de $\arctan x$ ?

La fonction $\arctan x$ a le bon goût d'avoir une dérivée d'expression simple.

\change

en effet $\arctan' x=\frac{1}{1+x^2}$. 

\change

C'est donc une expression de la forme $\frac{1}{1+u}$ -où $u=x^2$-
dont on connaît le DL.

On trouve donc le DL de $\arctan' x=1-x^2+x^4-x^6...$.

\change

On intègre terme à terme pour obtenir 
le DL de $\arctan x$

Le terme $(-1)^kx^{2k}$ une fois intégré devient $\frac{(-1)^k}{2k+1}x^{2k+1}$

\change

Et comme $\arctan(0)=0$ alors le terme constant est nul

\change

Voici donc le DL de $\arctan x$ :
$x-\frac{x^3}{3}+\frac{x^5}{5}-\frac{x^7}{7} +\cdots$

(pause)

\change

La méthode est la même pour obtenir un DL de $\arcsin x$ en $0$ à l'ordre $5$.

\change

On sait que la dérivée d'arcsin est $\frac{1}{\sqrt{1-x^2}}$
que l'on écrit ici $(1-x^2)^{-\frac{1}{2}}$

\change

Nous sommes devant une fonction de la forme $(1+u)^{-\frac12}$ avec $u=-x^2$

On trouve ceci

$1-\frac{1}{2}(-x^2)
+\frac{-\frac{1}{2}(-\frac{1}{2}-1)}{2}(-x^2)^2+x^4\epsilon(x) $

\change

et après regroupement cela donne

$1+\frac{1}{2}x^2 + \frac{3}{8}x^4$ + le reste.

\change

Il ne reste plus qu'à intégrer
pour obtenir le DL de $\arcsin x$

$x+\frac{1}{6}x^3+\frac{3}{40}x^5$ + le reste. 

Comme $\arcsin(0)=0$ alors le terme constant est nul.

%%%%%%%%%%%%%%%%%%%%%%%%%%%%%%%%%%%%%%%%%%%%%%%%%%%%%%%%%%%
\diapo

Il est vraiment important de savoir calculer des développements limités.
Il n'y a qu'une seule méthode pour cela : s’exercer encore et encore !

Avec l’entraînement vous saurez poser proprement les calculs,
vous calculerez plus vite et vous saurez d'avance jusque 
qu'à quel ordre on doit écrire les DL intermédiaires.




\end{document}
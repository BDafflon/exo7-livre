
%%%%%%%%%%%%%%%%%% PREAMBULE %%%%%%%%%%%%%%%%%%


\documentclass[12pt]{article}

\usepackage{amsfonts,amsmath,amssymb,amsthm}
\usepackage[utf8]{inputenc}
\usepackage[T1]{fontenc}
\usepackage[francais]{babel}


% packages
\usepackage{amsfonts,amsmath,amssymb,amsthm}
\usepackage[utf8]{inputenc}
\usepackage[T1]{fontenc}
%\usepackage{lmodern}

\usepackage[francais]{babel}
\usepackage{fancybox}
\usepackage{graphicx}

\usepackage{float}

%\usepackage[usenames, x11names]{xcolor}
\usepackage{tikz}
\usepackage{datetime}

\usepackage{mathptmx}
%\usepackage{fouriernc}
%\usepackage{newcent}
\usepackage[mathcal,mathbf]{euler}

%\usepackage{palatino}
%\usepackage{newcent}


% Commande spéciale prompteur

%\usepackage{mathptmx}
%\usepackage[mathcal,mathbf]{euler}
%\usepackage{mathpple,multido}

\usepackage[a4paper]{geometry}
\geometry{top=2cm, bottom=2cm, left=1cm, right=1cm, marginparsep=1cm}

\newcommand{\change}{{\color{red}\rule{\textwidth}{1mm}\\}}

\newcounter{mydiapo}

\newcommand{\diapo}{\newpage
\hfill {\normalsize  Diapo \themydiapo \quad \texttt{[\jobname]}} \\
\stepcounter{mydiapo}}


%%%%%%% COULEURS %%%%%%%%%%

% Pour blanc sur noir :
%\pagecolor[rgb]{0.5,0.5,0.5}
% \pagecolor[rgb]{0,0,0}
% \color[rgb]{1,1,1}



%\DeclareFixedFont{\myfont}{U}{cmss}{bx}{n}{18pt}
\newcommand{\debuttexte}{
%%%%%%%%%%%%% FONTES %%%%%%%%%%%%%
\renewcommand{\baselinestretch}{1.5}
\usefont{U}{cmss}{bx}{n}
\bfseries

% Taille normale : commenter le reste !
%Taille Arnaud
%\fontsize{19}{19}\selectfont

% Taille Barbara
%\fontsize{21}{22}\selectfont

%Taille François
%\fontsize{25}{30}\selectfont

%Taille Pascal
%\fontsize{25}{30}\selectfont

%Taille Laura
%\fontsize{30}{35}\selectfont


%\myfont
%\usefont{U}{cmss}{bx}{n}

%\Huge
%\addtolength{\parskip}{\baselineskip}
}


% \usepackage{hyperref}
% \hypersetup{colorlinks=true, linkcolor=blue, urlcolor=blue,
% pdftitle={Exo7 - Exercices de mathématiques}, pdfauthor={Exo7}}


%section
% \usepackage{sectsty}
% \allsectionsfont{\bf}
%\sectionfont{\color{Tomato3}\upshape\selectfont}
%\subsectionfont{\color{Tomato4}\upshape\selectfont}

%----- Ensembles : entiers, reels, complexes -----
\newcommand{\Nn}{\mathbb{N}} \newcommand{\N}{\mathbb{N}}
\newcommand{\Zz}{\mathbb{Z}} \newcommand{\Z}{\mathbb{Z}}
\newcommand{\Qq}{\mathbb{Q}} \newcommand{\Q}{\mathbb{Q}}
\newcommand{\Rr}{\mathbb{R}} \newcommand{\R}{\mathbb{R}}
\newcommand{\Cc}{\mathbb{C}} 
\newcommand{\Kk}{\mathbb{K}} \newcommand{\K}{\mathbb{K}}

%----- Modifications de symboles -----
\renewcommand{\epsilon}{\varepsilon}
\renewcommand{\Re}{\mathop{\text{Re}}\nolimits}
\renewcommand{\Im}{\mathop{\text{Im}}\nolimits}
%\newcommand{\llbracket}{\left[\kern-0.15em\left[}
%\newcommand{\rrbracket}{\right]\kern-0.15em\right]}

\renewcommand{\ge}{\geqslant}
\renewcommand{\geq}{\geqslant}
\renewcommand{\le}{\leqslant}
\renewcommand{\leq}{\leqslant}

%----- Fonctions usuelles -----
\newcommand{\ch}{\mathop{\mathrm{ch}}\nolimits}
\newcommand{\sh}{\mathop{\mathrm{sh}}\nolimits}
\renewcommand{\tanh}{\mathop{\mathrm{th}}\nolimits}
\newcommand{\cotan}{\mathop{\mathrm{cotan}}\nolimits}
\newcommand{\Arcsin}{\mathop{\mathrm{Arcsin}}\nolimits}
\newcommand{\Arccos}{\mathop{\mathrm{Arccos}}\nolimits}
\newcommand{\Arctan}{\mathop{\mathrm{Arctan}}\nolimits}
\newcommand{\Argsh}{\mathop{\mathrm{Argsh}}\nolimits}
\newcommand{\Argch}{\mathop{\mathrm{Argch}}\nolimits}
\newcommand{\Argth}{\mathop{\mathrm{Argth}}\nolimits}
\newcommand{\pgcd}{\mathop{\mathrm{pgcd}}\nolimits} 

\newcommand{\Card}{\mathop{\text{Card}}\nolimits}
\newcommand{\Ker}{\mathop{\text{Ker}}\nolimits}
\newcommand{\id}{\mathop{\text{id}}\nolimits}
\newcommand{\ii}{\mathrm{i}}
\newcommand{\dd}{\mathrm{d}}
\newcommand{\Vect}{\mathop{\text{Vect}}\nolimits}
\newcommand{\Mat}{\mathop{\mathrm{Mat}}\nolimits}
\newcommand{\rg}{\mathop{\text{rg}}\nolimits}
\newcommand{\tr}{\mathop{\text{tr}}\nolimits}
\newcommand{\ppcm}{\mathop{\text{ppcm}}\nolimits}

%----- Structure des exercices ------

\newtheoremstyle{styleexo}% name
{2ex}% Space above
{3ex}% Space below
{}% Body font
{}% Indent amount 1
{\bfseries} % Theorem head font
{}% Punctuation after theorem head
{\newline}% Space after theorem head 2
{}% Theorem head spec (can be left empty, meaning ‘normal’)

%\theoremstyle{styleexo}
\newtheorem{exo}{Exercice}
\newtheorem{ind}{Indications}
\newtheorem{cor}{Correction}


\newcommand{\exercice}[1]{} \newcommand{\finexercice}{}
%\newcommand{\exercice}[1]{{\tiny\texttt{#1}}\vspace{-2ex}} % pour afficher le numero absolu, l'auteur...
\newcommand{\enonce}{\begin{exo}} \newcommand{\finenonce}{\end{exo}}
\newcommand{\indication}{\begin{ind}} \newcommand{\finindication}{\end{ind}}
\newcommand{\correction}{\begin{cor}} \newcommand{\fincorrection}{\end{cor}}

\newcommand{\noindication}{\stepcounter{ind}}
\newcommand{\nocorrection}{\stepcounter{cor}}

\newcommand{\fiche}[1]{} \newcommand{\finfiche}{}
\newcommand{\titre}[1]{\centerline{\large \bf #1}}
\newcommand{\addcommand}[1]{}
\newcommand{\video}[1]{}

% Marge
\newcommand{\mymargin}[1]{\marginpar{{\small #1}}}



%----- Presentation ------
\setlength{\parindent}{0cm}

%\newcommand{\ExoSept}{\href{http://exo7.emath.fr}{\textbf{\textsf{Exo7}}}}

\definecolor{myred}{rgb}{0.93,0.26,0}
\definecolor{myorange}{rgb}{0.97,0.58,0}
\definecolor{myyellow}{rgb}{1,0.86,0}

\newcommand{\LogoExoSept}[1]{  % input : echelle
{\usefont{U}{cmss}{bx}{n}
\begin{tikzpicture}[scale=0.1*#1,transform shape]
  \fill[color=myorange] (0,0)--(4,0)--(4,-4)--(0,-4)--cycle;
  \fill[color=myred] (0,0)--(0,3)--(-3,3)--(-3,0)--cycle;
  \fill[color=myyellow] (4,0)--(7,4)--(3,7)--(0,3)--cycle;
  \node[scale=5] at (3.5,3.5) {Exo7};
\end{tikzpicture}}
}



\theoremstyle{definition}
%\newtheorem{proposition}{Proposition}
%\newtheorem{exemple}{Exemple}
%\newtheorem{theoreme}{Théorème}
\newtheorem{lemme}{Lemme}
\newtheorem{corollaire}{Corollaire}
%\newtheorem*{remarque*}{Remarque}
%\newtheorem*{miniexercice}{Mini-exercices}
%\newtheorem{definition}{Définition}




%definition d'un terme
\newcommand{\defi}[1]{{\color{myorange}\textbf{\emph{#1}}}}
\newcommand{\evidence}[1]{{\color{blue}\textbf{\emph{#1}}}}



 %----- Commandes divers ------

\newcommand{\codeinline}[1]{\texttt{#1}}

%%%%%%%%%%%%%%%%%%%%%%%%%%%%%%%%%%%%%%%%%%%%%%%%%%%%%%%%%%%%%
%%%%%%%%%%%%%%%%%%%%%%%%%%%%%%%%%%%%%%%%%%%%%%%%%%%%%%%%%%%%%

\begin{document}

\debuttexte

%%%%%%%%%%%%%%%%%%%%%%%%%%%%%%%%%%%%%%%%%%%%%%%%%%%%%%%%%%%
\diapo

\change

Les nombres premiers sont --en quelque sorte-- les briques élémentaires des entiers :
tout entier s'écrit comme produit de nombre premiers.

\change

Après avoir défini les nombres premiers

nous montrerons qu'il en existe un infinité.

\change

Nous verrons comment trouver des nombres premiers

\change

et nous énoncerons ensuite le théorème de décomposition
en produit de facteurs premiers.



%%%%%%%%%%%%%%%%%%%%%%%%%%%%%%%%%%%%%%%%%%%%%%%%%%%%%%%%%%%
\diapo

Un {nombre premier} $p$ est un entier $\ge 2$ dont les seuls diviseurs  
positifs sont $1$ et $p$.


\change
Les premiers nombres premiers sont $2, 3, 5, 7, 11$,

par exemple $7$ est divisible par $1$ et par $7$ et par aucun autre entier.

\change


Par contre $4$ est divisible par $1, 4$ et aussi $2$ donc n'est donc pas un nombre premier.

$6$ est divisible par $1,2,3,6$ n'est pas un nombre premier. etc.





%%%%%%%%%%%%%%%%%%%%%%%%%%%%%%%%%%%%%%%%%%%%%%%%%%%%%%%%%%%
\diapo

Commençons par un résultat élémentaire :

Tout entier $n \ge 2$ admet un diviseur qui est un nombre premier


\change


La preuve est la suivante : nous allons montrer que le plus petit diviseur $\ge 2$
de $n$ est un nombre premier.

Considérons donc l'ensemble $\mathcal{D}$ des entiers $k$ plus grands que $2$ tel que $k$ divise $n$.

\change

Tout d'abord $\mathcal{D}$ est non vide car $n \in \mathcal{D}$

\change

donc on peut définir $p$ le minimum de l'ensemble $\mathcal{D}$

et comme $p$ appartient à $\mathcal{D}$, $p$ divise $n$.

\change 

Il nous reste à montrer que $p$ est un nombre premier,


\change

nous allons le prouver par l'absurde.

Si $p$ n'est pas un nombre premier  
alors il admet un diviseur $q$ autre que $1$ et $p$ lui-même.

\change

Donc $q$ est plus grand que $2$ et comme $q$ divise $p$ il divise aussi $n$.

Donc $q$ est un élément de l'ensemble $\mathcal{D}$.

Mais $q$ est strictement plus petit que $p$

\change

Nous avons trouvé une contradiction car $p$ est le plus petit élément de $\mathcal{D}$
et l'on vient d'exhiber un autre élément $q$ de $\mathcal{D}$ strictement plus petit.

Conclusion : $p$ est un bien un diviseur premier de $n$.



%%%%%%%%%%%%%%%%%%%%%%%%%%%%%%%%%%%%%%%%%%%%%%%%%%%%%%%%%%%
\diapo

Proposition :

Il existe une infinité de nombres premiers

\change

La démonstration se fait par l'absurde .

Nous supposons qu'il n'y a qu'un nombre fini de nombre premiers.

\change

Alors on peut tous les noter $p_1$, $p_2$, jusqu'à $p_n$ : on les a tous pris.

\change

Notons  $N=p_1\times p_2\times \cdots \times p_n+ 1$

Par le lemme précédent $N$ admet un diviseur $p$ qui est un nombre premier

\change

Alors d'une part $p$ est l'un des entiers $p_i$
donc $p | p_1\times \cdots \times p_n$

\change 

D'autre part $p|N$ 


donc $p$ divise la différence $N-p_1\times \cdots \times p_n$

mais cette différence vaut $1$ par définition de $N$, donc $p$ divise $1$.

\change

Cela implique que $p=1$

\change

Mais comme $p$ est un nombre premier alors $p$ doit être plus grand que $2$.

Cette contradiction nous permet de conclure qu'il existe une infinité de nombres premiers.



%%%%%%%%%%%%%%%%%%%%%%%%%%%%%%%%%%%%%%%%%%%%%%%%%%%%%%%%%%%
\diapo

Comment trouver les nombres premiers ? 

Le \evidence{crible d'Eratosthène} permet de trouver les premiers nombres premiers.

Pour cela on écrit les premiers entiers : pour notre exemple de $2$ à $25$

\change

$2$ ne peut avoir comme diviseurs que $1$ et $2$ et est donc premier.

On entoure $2$. Ensuite on raye ou grise tous les multiples suivants de $2$ qui ne seront donc pas premiers


\change

Le premier nombre restant de la liste est $3$ et est nécessairement premier : 

il n'est pas divisible par un diviseur
plus petit (sinon il serait grisé). 

On entoure $3$ et on grise tous les multiples de $3$

\change

Le premier nombre restant est $5$ et est donc premier. 

On grise les multiples de $5$ (ici le seul nouveau grisé est $25$).

\change 

$7$ est donc premier.

Ainsi de suite : $11, 13, 17, 19, 23$ sont premiers.

Nous avons entourer tous les nombres premiers inférieur à $25$.

\change

Comment tester si un nombre $n$ est un nombre premier, sans le crible d'Eratosthène.

Si un nombre $n$ n'est pas premier alors un de ses facteurs est $\le \sqrt{n}$.

(Prouvez le par l'absurde.)

\change

Par exemple pour tester si un nombre $\le 100$ est premier
il suffit de tester les diviseurs $\le \sqrt{100}=10$.

\change

Et comme s'il admet un diviseur, il admet un diviseur premier (toujours 
grâce à notre lemme) 

il suffit donc de tester la divisibilité par $2, 3, 5$ et $7$


\change
  Par exemple $89$ n'est divisible ni par $2,3,5,7$ c'est donc un nombre premier


%%%%%%%%%%%%%%%%%%%%%%%%%%%%%%%%%%%%%%%%%%%%%%%%%%%%%%%%%%%
\diapo


Le lemme d'Euclide est le résultat suivant : 

Soit $p$ un nombre premier.
Si $p$ divise le produit $a$ fois $b$ alors $p|a$ ou $p | b$


\change

La preuve est courte :

Si $p$ ne divise pas $a$ alors $p$ et $a$ sont premiers entre eux 

\change

Ainsi par le lemme de Gauss $p | b$

\change

Appliquons ceci pour prouver le résultat suivant : 

si $p$ est un nombre premier alors $\sqrt{p}$ n'est pas un nombre rationnel.

\change

La preuve est la même que pour montrer que $\sqrt 2$ est irrationnel.

Revoyons-la rapidement. 

Par l'absurde : écrivons $\sqrt p =\frac ab$

avec $\pgcd(a,b)=1$

\change

En élevant au carré on obtient
$p b^2 = a^2$ donc $p | a^2$ donc par le lemme d'Euclide $p | a$

\change

 On écrit $a = p a'$ avec $a'$ un entier donc $b^2 = p a'^2$

\change

 Ainsi $p | b^2$ et donc encore par le lemme d'Euclide $p|b$

\change

 Maintenant $p|a$ et $p|b$ donc $a$ et $b$ ne sont pas premiers entre eux. Contradiction

\change 

Conclusion $\sqrt p$ n'est pas rationnel


%%%%%%%%%%%%%%%%%%%%%%%%%%%%%%%%%%%%%%%%%%%%%%%%%%%%%%%%%%%
\diapo

Voici le théorème de décomposition des entiers en produit de facteurs premiers.

Fixons $n\ge 2$ un entier.

Il existe des nombres premiers $p_1$ jusqu'à $p_r$
et des exposants entiers $\alpha_1$ jusqu'à $\alpha_r$ tels que :

$n = p_1^{\alpha_1} \times p_2^{\alpha_2} \times \cdots \times p_r^{\alpha_r}$

\change

De plus les facteurs premiers $p_i$ et les exposants $\alpha_i$ sont uniques.


\change

Il y a donc deux résultats : l'existence et l'unicité.

\change

Par exemple la décomposition de $24$ en produits de facteurs premiers est $24 = 2^3 \times 3$

\change 

Mais $2^2 \times 9$ n'est pas la décomposition en facteurs premiers de $36$ car 
$9$ n'est pas premier. C'est $2^2\times 3^2$.

\change

Répondons à une question classique.

Pourquoi $1$ n'est pas un nombre premier ?

\change

Une des principales raison pour laquelle nous convenons de dire
que $1$ n'est pas un nombre premier, c'est que sinon 
il n'y aurait plus unicité de la décomposition 

\change

Par exemple $24$ s'écrirait $2^3 \times 3$ mais aussi $1 \times 2^3 \times 3$ 
ou encore $1^2 \times 2^3 \times 3$


%%%%%%%%%%%%%%%%%%%%%%%%%%%%%%%%%%%%%%%%%%%%%%%%%%%%%%%%%%%
\diapo

Voici quelques exemples :

pour trouver la décomposition de $504$ en produit de nombres premiers, on divise $504$
en $2$, cela fait $252$, que l'on redivise en $2$ etc, quand on peut plus diviser par $2$ on 
teste la division par $3$, puis par $5$, etc.

On trouve 
$504 = 2^{\color{red}{3}} \times 3^{\color{red}{2}} \times 7$

\change

Même chose avec $300$ qui vaut $2^{\color{blue}{2}} \times 3 \times 5^{\color{blue}{2}}$


Nous allons pouvoir en déduire le pgcd et le ppcm de ces deux nombres.

\change

On commence par réécrire $504$ et $300$ avec tous les nombres premiers, quitte
à mettre des puissance zéros.


\change

Le pgcd s'obtient en prenant le plus petit exposant de chaque facteur premier

\change

L'exposant pour le facteur premier $2$ est donc $2$, l'exposant pour le facteur $3$ est $1$,

l'exposant pour le facteur $5$  est zéro, celui pour le facteur $7$ est zéro également.

Le pgcd est donc $2^{\color{blue}{2}} \times 3^{\color{blue}{1}} \times 
5^{\color{red}{0}} \times 7^{\color{blue}{0}}$ 

\change

qui vaut $12$.

\change 

Pour le ppcm on prend le plus grand exposant de chaque facteur premier

\change

Le ppcm de $504$ et $300$ vaut donc 

$2^{\color{red}{3}} \times 3^{\color{red}{2}} 
\times 5^{\color{blue}{2}} \times 7^{\color{red}{1}} = 12\,600$


%%%%%%%%%%%%%%%%%%%%%%%%%%%%%%%%%%%%%%%%%%%%%%%%%%%%%%%%%%%
\diapo


Comme d'habitude, travaillez votre cours en répondant aux questions suivantes.


\end{document}
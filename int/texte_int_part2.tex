
%%%%%%%%%%%%%%%%%% PREAMBULE %%%%%%%%%%%%%%%%%%


\documentclass[12pt]{article}

\usepackage{amsfonts,amsmath,amssymb,amsthm}
\usepackage[utf8]{inputenc}
\usepackage[T1]{fontenc}
\usepackage[francais]{babel}


% packages
\usepackage{amsfonts,amsmath,amssymb,amsthm}
\usepackage[utf8]{inputenc}
\usepackage[T1]{fontenc}
%\usepackage{lmodern}

\usepackage[francais]{babel}
\usepackage{fancybox}
\usepackage{graphicx}

\usepackage{float}

%\usepackage[usenames, x11names]{xcolor}
\usepackage{tikz}
\usepackage{datetime}

\usepackage{mathptmx}
%\usepackage{fouriernc}
%\usepackage{newcent}
\usepackage[mathcal,mathbf]{euler}

%\usepackage{palatino}
%\usepackage{newcent}


% Commande spéciale prompteur

%\usepackage{mathptmx}
%\usepackage[mathcal,mathbf]{euler}
%\usepackage{mathpple,multido}

\usepackage[a4paper]{geometry}
\geometry{top=2cm, bottom=2cm, left=1cm, right=1cm, marginparsep=1cm}

\newcommand{\change}{{\color{red}\rule{\textwidth}{1mm}\\}}

\newcounter{mydiapo}

\newcommand{\diapo}{\newpage
\hfill {\normalsize  Diapo \themydiapo \quad \texttt{[\jobname]}} \\
\stepcounter{mydiapo}}


%%%%%%% COULEURS %%%%%%%%%%

% Pour blanc sur noir :
%\pagecolor[rgb]{0.5,0.5,0.5}
% \pagecolor[rgb]{0,0,0}
% \color[rgb]{1,1,1}



%\DeclareFixedFont{\myfont}{U}{cmss}{bx}{n}{18pt}
\newcommand{\debuttexte}{
%%%%%%%%%%%%% FONTES %%%%%%%%%%%%%
\renewcommand{\baselinestretch}{1.5}
\usefont{U}{cmss}{bx}{n}
\bfseries

% Taille normale : commenter le reste !
%Taille Arnaud
%\fontsize{19}{19}\selectfont

% Taille Barbara
%\fontsize{21}{22}\selectfont

%Taille François
%\fontsize{25}{30}\selectfont

%Taille Pascal
%\fontsize{25}{30}\selectfont

%Taille Laura
%\fontsize{30}{35}\selectfont


%\myfont
%\usefont{U}{cmss}{bx}{n}

%\Huge
%\addtolength{\parskip}{\baselineskip}
}


% \usepackage{hyperref}
% \hypersetup{colorlinks=true, linkcolor=blue, urlcolor=blue,
% pdftitle={Exo7 - Exercices de mathématiques}, pdfauthor={Exo7}}


%section
% \usepackage{sectsty}
% \allsectionsfont{\bf}
%\sectionfont{\color{Tomato3}\upshape\selectfont}
%\subsectionfont{\color{Tomato4}\upshape\selectfont}

%----- Ensembles : entiers, reels, complexes -----
\newcommand{\Nn}{\mathbb{N}} \newcommand{\N}{\mathbb{N}}
\newcommand{\Zz}{\mathbb{Z}} \newcommand{\Z}{\mathbb{Z}}
\newcommand{\Qq}{\mathbb{Q}} \newcommand{\Q}{\mathbb{Q}}
\newcommand{\Rr}{\mathbb{R}} \newcommand{\R}{\mathbb{R}}
\newcommand{\Cc}{\mathbb{C}} 
\newcommand{\Kk}{\mathbb{K}} \newcommand{\K}{\mathbb{K}}

%----- Modifications de symboles -----
\renewcommand{\epsilon}{\varepsilon}
\renewcommand{\Re}{\mathop{\text{Re}}\nolimits}
\renewcommand{\Im}{\mathop{\text{Im}}\nolimits}
%\newcommand{\llbracket}{\left[\kern-0.15em\left[}
%\newcommand{\rrbracket}{\right]\kern-0.15em\right]}

\renewcommand{\ge}{\geqslant}
\renewcommand{\geq}{\geqslant}
\renewcommand{\le}{\leqslant}
\renewcommand{\leq}{\leqslant}

%----- Fonctions usuelles -----
\newcommand{\ch}{\mathop{\mathrm{ch}}\nolimits}
\newcommand{\sh}{\mathop{\mathrm{sh}}\nolimits}
\renewcommand{\tanh}{\mathop{\mathrm{th}}\nolimits}
\newcommand{\cotan}{\mathop{\mathrm{cotan}}\nolimits}
\newcommand{\Arcsin}{\mathop{\mathrm{Arcsin}}\nolimits}
\newcommand{\Arccos}{\mathop{\mathrm{Arccos}}\nolimits}
\newcommand{\Arctan}{\mathop{\mathrm{Arctan}}\nolimits}
\newcommand{\Argsh}{\mathop{\mathrm{Argsh}}\nolimits}
\newcommand{\Argch}{\mathop{\mathrm{Argch}}\nolimits}
\newcommand{\Argth}{\mathop{\mathrm{Argth}}\nolimits}
\newcommand{\pgcd}{\mathop{\mathrm{pgcd}}\nolimits} 

\newcommand{\Card}{\mathop{\text{Card}}\nolimits}
\newcommand{\Ker}{\mathop{\text{Ker}}\nolimits}
\newcommand{\id}{\mathop{\text{id}}\nolimits}
\newcommand{\ii}{\mathrm{i}}
\newcommand{\dd}{\mathrm{d}}
\newcommand{\Vect}{\mathop{\text{Vect}}\nolimits}
\newcommand{\Mat}{\mathop{\mathrm{Mat}}\nolimits}
\newcommand{\rg}{\mathop{\text{rg}}\nolimits}
\newcommand{\tr}{\mathop{\text{tr}}\nolimits}
\newcommand{\ppcm}{\mathop{\text{ppcm}}\nolimits}

%----- Structure des exercices ------

\newtheoremstyle{styleexo}% name
{2ex}% Space above
{3ex}% Space below
{}% Body font
{}% Indent amount 1
{\bfseries} % Theorem head font
{}% Punctuation after theorem head
{\newline}% Space after theorem head 2
{}% Theorem head spec (can be left empty, meaning ‘normal’)

%\theoremstyle{styleexo}
\newtheorem{exo}{Exercice}
\newtheorem{ind}{Indications}
\newtheorem{cor}{Correction}


\newcommand{\exercice}[1]{} \newcommand{\finexercice}{}
%\newcommand{\exercice}[1]{{\tiny\texttt{#1}}\vspace{-2ex}} % pour afficher le numero absolu, l'auteur...
\newcommand{\enonce}{\begin{exo}} \newcommand{\finenonce}{\end{exo}}
\newcommand{\indication}{\begin{ind}} \newcommand{\finindication}{\end{ind}}
\newcommand{\correction}{\begin{cor}} \newcommand{\fincorrection}{\end{cor}}

\newcommand{\noindication}{\stepcounter{ind}}
\newcommand{\nocorrection}{\stepcounter{cor}}

\newcommand{\fiche}[1]{} \newcommand{\finfiche}{}
\newcommand{\titre}[1]{\centerline{\large \bf #1}}
\newcommand{\addcommand}[1]{}
\newcommand{\video}[1]{}

% Marge
\newcommand{\mymargin}[1]{\marginpar{{\small #1}}}



%----- Presentation ------
\setlength{\parindent}{0cm}

%\newcommand{\ExoSept}{\href{http://exo7.emath.fr}{\textbf{\textsf{Exo7}}}}

\definecolor{myred}{rgb}{0.93,0.26,0}
\definecolor{myorange}{rgb}{0.97,0.58,0}
\definecolor{myyellow}{rgb}{1,0.86,0}

\newcommand{\LogoExoSept}[1]{  % input : echelle
{\usefont{U}{cmss}{bx}{n}
\begin{tikzpicture}[scale=0.1*#1,transform shape]
  \fill[color=myorange] (0,0)--(4,0)--(4,-4)--(0,-4)--cycle;
  \fill[color=myred] (0,0)--(0,3)--(-3,3)--(-3,0)--cycle;
  \fill[color=myyellow] (4,0)--(7,4)--(3,7)--(0,3)--cycle;
  \node[scale=5] at (3.5,3.5) {Exo7};
\end{tikzpicture}}
}



\theoremstyle{definition}
%\newtheorem{proposition}{Proposition}
%\newtheorem{exemple}{Exemple}
%\newtheorem{theoreme}{Théorème}
\newtheorem{lemme}{Lemme}
\newtheorem{corollaire}{Corollaire}
%\newtheorem*{remarque*}{Remarque}
%\newtheorem*{miniexercice}{Mini-exercices}
%\newtheorem{definition}{Définition}




%definition d'un terme
\newcommand{\defi}[1]{{\color{myorange}\textbf{\emph{#1}}}}
\newcommand{\evidence}[1]{{\color{blue}\textbf{\emph{#1}}}}



 %----- Commandes divers ------

\newcommand{\codeinline}[1]{\texttt{#1}}

%%%%%%%%%%%%%%%%%%%%%%%%%%%%%%%%%%%%%%%%%%%%%%%%%%%%%%%%%%%%%
%%%%%%%%%%%%%%%%%%%%%%%%%%%%%%%%%%%%%%%%%%%%%%%%%%%%%%%%%%%%%



\begin{document}

\debuttexte

%%%%%%%%%%%%%%%%%%%%%%%%%%%%%%%%%%%%%%%%%%%%%%%%%%%%%%%%%%%
\diapo

\change

Nous avons vu la définition de l'intégrale passons à ses propriétés.

\change

Les trois propriétés fondamentales de l'intégrale sont

la relation de Chasles, 

\change

la positivité 

\change

et la linéarité.

\change

On terminera par une preuve.

%%%%%%%%%%%%%%%%%%%%%%%%%%%%%%%%%%%%%%%%%%%%%%%%%%%%%%%%%%%
\diapo

Fixons trois réels $a,b,c$ quelconques la relation de Chasles
affirme que $\int_a^b f(x)\;dx 
= \int_a^c f(x)\;dx + \int_c^b f(x)\;dx$

\change
Cette relation se justifie par les remarques suivantes :

Supposons que les trois réels soient ordonnés ainsi : $a<c<b$.
Si $f$ est intégrable sur $[a,c]$ et $[c,b]$, alors $f$ est intégrable sur $[a,b]$

\change

Voici comment on définit l'intégrale si les bornes sont renversées :
si $a<b$ on dit que $\int_b^a f(x) \;dx$ est $-\int_a^b f(x) \; dx$.

\change

Enfin par définition 
$$\int_a^a f(x) \;dx=0$$




%%%%%%%%%%%%%%%%%%%%%%%%%%%%%%%%%%%%%%%%%%%%%%%%%%%%%%%%%%%
\diapo

La deuxième propriété fondamentale est la positivité de l'intégrale :

Soient $f$ et $g$ deux fonctions intégrables sur $[a,b]$, avec $a\le b$.

Supposons que $f \le g$ (c'est-à-dire $ \forall x f(x)\le g(x)$).

Alors $\int_a^b f(x)\;dx \le\int_a^b g(x)\;dx$

\change


En particulier si $f\ge 0$ alors $\int_a^bf(x)\;dx \ge 0$.

Autrement dit l'intégrale d'une fonction positive est positive.


%%%%%%%%%%%%%%%%%%%%%%%%%%%%%%%%%%%%%%%%%%%%%%%%%%%%%%%%%%%
\diapo

Partant de deux fonctions $f,g$ intgérables sur $[a,b]$
alors $f+g$ est une fonction intégrable 



et l'intégrale de la somme est la somme des intégrales.

\change

Pour tout réel $\lambda$,  $\lambda f$ est
intégrable et 
l'intégrale de $\lambda f$

est $\lambda$ fois l'intégrale de $f$.

\change


Ces deux premiers points se rassemblent en un seul qui s'appelle la linéarité de l'intégrale :


pour tous réels $\lambda,\mu$ et pour deux fonctions intégrables $f$ et $g$ alors

$\int \lambda f+\mu g = \lambda\int f +\mu\int g$

\change

Voyons un petit exemple :

$\int_0^1 \big(7x^2-e^x\big) \; dx  $

\change

$= 7 \int_0^1 x^2\; dx \ \ - \  \int_0^1 e^x \; dx$

on a calculé lors de la leçon précédente chacune de ces deux intégrales, 

\change

\change

donc le résultat est $\frac{10}{3}-e$


%%%%%%%%%%%%%%%%%%%%%%%%%%%%%%%%%%%%%%%%%%%%%%%%%%%%%%%%%%%
\diapo

Si $f$ est une fonction intégrable alors $|f|$ est une fonction intégrable sur $[a,b]$.


et on a la relation importante :
$\left\vert\int_a^b f(x) \;dx\right\vert\le\int_a^b\big\vert f(x)\big\vert \;dx$

La valeur absolue de l'intégrale est inférieur à l'intégrale de la valeur absolue.

\change


On continue avec un avertissement 
si $f$ et $g$ sont deux fonctions intégrables alors 
$f \times g$ est aussi une fonction intégrable sur $[a,b]$

\change

mais en général $\int_a^b (f g)(x)\;dx \neq
\big(\int_a^b f(x)\;dx\big)\big(\int_a^b g(x)\;dx\big)$.

\change

Appliquons ceci à un petit exercice :
soit $I_n = \int_1^n \frac{\sin(n x)}{1+x^n} \; dx$. 

\change

Nous n'allons pas calculer $I_n$ mais nous allons montrer que $I_n \to 0$ lorsque $n\to +\infty$.

\change

$|I_n| = \left| \int_1^n \frac{\sin(n x)}{1+x^n} \; dx \right|$

\change

et donc notre inégalité s'écrit $|I_n| \le  \int_1^n  \frac{|\sin(n x)|}{1+x^n} \; dx$

\change

on majore $|\sin(n x)|$ par $1$.

\change

Et maintenant on utilise que $\frac{1}{1+x^n} \le  \frac{1}{x^n}$

\change

On a donc montré que 
$|I_n| \le \int_1^n  \frac{1}{x^n} \; dx$

\change

Il ne reste plus qu'à calculer cette dernière intégrale 
(en anticipant un peu sur la suite du chapitre) :
$ = \int_1^n x^{-n} \; dx$

\change

$ = \left[\frac{x^{-n+1}}{-n+1}\right]_1^n $

\change

$= \frac{n^{-n+1}}{-n+1} - \frac{1}{-n+1} $

\change

$n^{-n+1}$ tend vers $0$ donc chacun de ces termes tend vers $0$.

\change

Donc $|I_n|$ est majorée par une suite qui tend vers $0$ 

ainsi $I_n$ tend vers $0$ lorsque $n \to +\infty$.



%%%%%%%%%%%%%%%%%%%%%%%%%%%%%%%%%%%%%%%%%%%%%%%%%%%%%%%%%%%
\diapo


Nous terminons avec une partie de la preuve de la linéarité de l'intégrale:
$\int \lambda f= \lambda \int f$.

L'idée est la suivante : il est facile de voir que pour des fonctions en escalier
l'intégrale est alors une somme finie et cette relation est vraie. 

Comme les fonctions en escalier 
approchent autant qu'on le souhaite les fonctions intégrables 
alors cela implique que la relation est vraie pour toutes les fonctions.




\change

Fixons une fonction $f : [a,b] \to \Rr$ intégrable et soit $\lambda$ un réel positif.

Soit aussi un réel $\epsilon >0 $. 

\change

Il existe des fonctions en escalier $\phi^-$ et $\phi^+$ qui approche suffisamment $f$

c'est-à-dire d'une part $\phi^- \le f \le \phi^+$ 

\change

et d'autre part on
$\int_a^b f(x)\; dx  \ - \epsilon\  \le \int_a^b \phi^-(x)\;dx$

\change

et $\int_a^b \phi^+(x)\;dx \  \le \int_a^b f(x)\; dx  \ + \epsilon$

\change

Quitte à rajouter des points, on peut supposer que la subdivision $(x_0,x_1,\ldots,x_n)$ de $[a,b]$ 
est suffisamment fine pour que $\phi^-$ et $\phi^+$ soient toutes les deux constantes 
sur les intervalles $]x_{i-1},x_i[$ ; on note $c_i^-$ et $c_i^+$ leurs valeurs respectives.


\change

Tout d'abord $\lambda \phi^-$ et $\lambda \phi^+$ 
sont encore des fonctions en escalier

et elles vérifient l'inégalité 
$\lambda \phi^- \le \lambda f \le \lambda\phi^+$. 

\change

Comme $\lambda \phi^-$ et $\lambda \phi^+$ sont des fonctions en escalier
alors leur intégrale est par définition une somme finie 
donc 
$\int_a^b \lambda \phi^-(x)\;dx =  \sum_{i=1}^n \lambda c_i^-(x_{i}-x_{i-1})
= \lambda \sum_{i=1}^n  c_i^-(x_{i}-x_{i-1}) = \lambda\int_a^b  \phi^-(x)\;dx$

On a une relation semblable pour $\phi^+$.

\change

Jusqu'ici nous avons prouvé que 

$\lambda\int_a^b \phi^-(x)\;dx \le I^-(\lambda f) \le I^+(\lambda f) 
\le \lambda\int_a^b \phi^+(x)\;dx$

\change

Utilisons maintenant les deux inégalités du début 
 on obtient

$\lambda \int_a^b f(x)\; dx \  -\lambda \epsilon \ \le I^-(\lambda f) \le I^+(\lambda f) \le  \lambda \int_a^b f(x)\; dx \ +\lambda \epsilon$

\change

Il ne reste plus qu'à faire tendre $\epsilon \to 0$ pour obtenir que 
$I^-(\lambda f) = I^+(\lambda f)$,


\change

Nous avons d'une part prouver que $\lambda f$ est intégrable

et en plus que $\int_a^b \lambda f(x)\; dx = \lambda \int_a^b f(x)\; dx$.

Si $\lambda \le 0$ le raisonnement est similaire.


%%%%%%%%%%%%%%%%%%%%%%%%%%%%%%%%%%%%%%%%%%%%%%%%%%%%%%%%%%%
\diapo

Comme d'habitude commencez par ces exercices simples pour 
vérifier si vous avez bien compris le cours.


\end{document}
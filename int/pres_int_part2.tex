
%%%%%%%%%%%%%%%%%% PREAMBULE %%%%%%%%%%%%%%%%%%

\documentclass[aspectratio=169,utf8]{beamer}
%\documentclass[aspectratio=169,handout]{beamer}

\usetheme{Boadilla}
%\usecolortheme{seahorse}
%\usecolortheme[RGB={245,66,24}]{structure}
\useoutertheme{infolines}

% packages
\usepackage{amsfonts,amsmath,amssymb,amsthm}
\usepackage[utf8]{inputenc}
\usepackage[T1]{fontenc}
\usepackage{lmodern}

\usepackage[francais]{babel}
\usepackage{fancybox}
\usepackage{graphicx}

\usepackage{float}
\usepackage{xfrac}

%\usepackage[usenames, x11names]{xcolor}
\usepackage{pgfplots}
\usepackage{datetime}


% ----------------------------------------------------------------------
% Pour les images
\usepackage{tikz}
\usetikzlibrary{calc,shadows,arrows.meta,patterns,matrix}

\newcommand{\tikzinput}[1]{\input{figures/#1.tikz}}
% --- les figures avec échelle éventuel
\newcommand{\myfigure}[2]{% entrée : échelle, fichier(s) figure à inclure
\begin{center}\small%
\tikzstyle{every picture}=[scale=1.0*#1]% mise en échelle + 0% (automatiquement annulé à la fin du groupe)
#2%
\end{center}}



%-----  Package unités -----
\usepackage{siunitx}
\sisetup{locale = FR,detect-all,per-mode = symbol}

%\usepackage{mathptmx}
%\usepackage{fouriernc}
%\usepackage{newcent}
%\usepackage[mathcal,mathbf]{euler}

%\usepackage{palatino}
%\usepackage{newcent}
% \usepackage[mathcal,mathbf]{euler}



% \usepackage{hyperref}
% \hypersetup{colorlinks=true, linkcolor=blue, urlcolor=blue,
% pdftitle={Exo7 - Exercices de mathématiques}, pdfauthor={Exo7}}


%section
% \usepackage{sectsty}
% \allsectionsfont{\bf}
%\sectionfont{\color{Tomato3}\upshape\selectfont}
%\subsectionfont{\color{Tomato4}\upshape\selectfont}

%----- Ensembles : entiers, reels, complexes -----
\newcommand{\Nn}{\mathbb{N}} \newcommand{\N}{\mathbb{N}}
\newcommand{\Zz}{\mathbb{Z}} \newcommand{\Z}{\mathbb{Z}}
\newcommand{\Qq}{\mathbb{Q}} \newcommand{\Q}{\mathbb{Q}}
\newcommand{\Rr}{\mathbb{R}} \newcommand{\R}{\mathbb{R}}
\newcommand{\Cc}{\mathbb{C}} 
\newcommand{\Kk}{\mathbb{K}} \newcommand{\K}{\mathbb{K}}

%----- Modifications de symboles -----
\renewcommand{\epsilon}{\varepsilon}
\renewcommand{\Re}{\mathop{\text{Re}}\nolimits}
\renewcommand{\Im}{\mathop{\text{Im}}\nolimits}
%\newcommand{\llbracket}{\left[\kern-0.15em\left[}
%\newcommand{\rrbracket}{\right]\kern-0.15em\right]}

\renewcommand{\ge}{\geqslant}
\renewcommand{\geq}{\geqslant}
\renewcommand{\le}{\leqslant}
\renewcommand{\leq}{\leqslant}
\renewcommand{\epsilon}{\varepsilon}

%----- Fonctions usuelles -----
\newcommand{\ch}{\mathop{\text{ch}}\nolimits}
\newcommand{\sh}{\mathop{\text{sh}}\nolimits}
\renewcommand{\tanh}{\mathop{\text{th}}\nolimits}
\newcommand{\cotan}{\mathop{\text{cotan}}\nolimits}
\newcommand{\Arcsin}{\mathop{\text{arcsin}}\nolimits}
\newcommand{\Arccos}{\mathop{\text{arccos}}\nolimits}
\newcommand{\Arctan}{\mathop{\text{arctan}}\nolimits}
\newcommand{\Argsh}{\mathop{\text{argsh}}\nolimits}
\newcommand{\Argch}{\mathop{\text{argch}}\nolimits}
\newcommand{\Argth}{\mathop{\text{argth}}\nolimits}
\newcommand{\pgcd}{\mathop{\text{pgcd}}\nolimits} 


%----- Commandes divers ------
\newcommand{\ii}{\mathrm{i}}
\newcommand{\dd}{\text{d}}
\newcommand{\id}{\mathop{\text{id}}\nolimits}
\newcommand{\Ker}{\mathop{\text{Ker}}\nolimits}
\newcommand{\Card}{\mathop{\text{Card}}\nolimits}
\newcommand{\Vect}{\mathop{\text{Vect}}\nolimits}
\newcommand{\Mat}{\mathop{\text{Mat}}\nolimits}
\newcommand{\rg}{\mathop{\text{rg}}\nolimits}
\newcommand{\tr}{\mathop{\text{tr}}\nolimits}


%----- Structure des exercices ------

\newtheoremstyle{styleexo}% name
{2ex}% Space above
{3ex}% Space below
{}% Body font
{}% Indent amount 1
{\bfseries} % Theorem head font
{}% Punctuation after theorem head
{\newline}% Space after theorem head 2
{}% Theorem head spec (can be left empty, meaning ‘normal’)

%\theoremstyle{styleexo}
\newtheorem{exo}{Exercice}
\newtheorem{ind}{Indications}
\newtheorem{cor}{Correction}


\newcommand{\exercice}[1]{} \newcommand{\finexercice}{}
%\newcommand{\exercice}[1]{{\tiny\texttt{#1}}\vspace{-2ex}} % pour afficher le numero absolu, l'auteur...
\newcommand{\enonce}{\begin{exo}} \newcommand{\finenonce}{\end{exo}}
\newcommand{\indication}{\begin{ind}} \newcommand{\finindication}{\end{ind}}
\newcommand{\correction}{\begin{cor}} \newcommand{\fincorrection}{\end{cor}}

\newcommand{\noindication}{\stepcounter{ind}}
\newcommand{\nocorrection}{\stepcounter{cor}}

\newcommand{\fiche}[1]{} \newcommand{\finfiche}{}
\newcommand{\titre}[1]{\centerline{\large \bf #1}}
\newcommand{\addcommand}[1]{}
\newcommand{\video}[1]{}

% Marge
\newcommand{\mymargin}[1]{\marginpar{{\small #1}}}

\def\noqed{\renewcommand{\qedsymbol}{}}


%----- Presentation ------
\setlength{\parindent}{0cm}

%\newcommand{\ExoSept}{\href{http://exo7.emath.fr}{\textbf{\textsf{Exo7}}}}

\definecolor{myred}{rgb}{0.93,0.26,0}
\definecolor{myorange}{rgb}{0.97,0.58,0}
\definecolor{myyellow}{rgb}{1,0.86,0}

\newcommand{\LogoExoSept}[1]{  % input : echelle
{\usefont{U}{cmss}{bx}{n}
\begin{tikzpicture}[scale=0.1*#1,transform shape]
  \fill[color=myorange] (0,0)--(4,0)--(4,-4)--(0,-4)--cycle;
  \fill[color=myred] (0,0)--(0,3)--(-3,3)--(-3,0)--cycle;
  \fill[color=myyellow] (4,0)--(7,4)--(3,7)--(0,3)--cycle;
  \node[scale=5] at (3.5,3.5) {Exo7};
\end{tikzpicture}}
}


\newcommand{\debutmontitre}{
  \author{} \date{} 
  \thispagestyle{empty}
  \hspace*{-10ex}
  \begin{minipage}{\textwidth}
    \titlepage  
  \vspace*{-2.5cm}
  \begin{center}
    \LogoExoSept{2.5}
  \end{center}
  \end{minipage}

  \vspace*{-0cm}
  
  % Astuce pour que le background ne soit pas discrétisé lors de la conversion pdf -> png
\begin{tikzpicture}
        \fill[opacity=0,green!60!black] (0,0)--++(0,0)--++(0,0)--++(0,0)--cycle; 
\end{tikzpicture}

% toc S'affiche trop tot :
% \tableofcontents[hideallsubsections, pausesections]
}

\newcommand{\finmontitre}{
  \end{frame}
  \setcounter{framenumber}{0}
} % ne marche pas pour une raison obscure

%----- Commandes supplementaires ------

% \usepackage[landscape]{geometry}
% \geometry{top=1cm, bottom=3cm, left=2cm, right=10cm, marginparsep=1cm
% }
% \usepackage[a4paper]{geometry}
% \geometry{top=2cm, bottom=2cm, left=2cm, right=2cm, marginparsep=1cm
% }

%\usepackage{standalone}


% New command Arnaud -- november 2011
\setbeamersize{text margin left=24ex}
% si vous modifier cette valeur il faut aussi
% modifier le decalage du titre pour compenser
% (ex : ici =+10ex, titre =-5ex

\theoremstyle{definition}
%\newtheorem{proposition}{Proposition}
%\newtheorem{exemple}{Exemple}
%\newtheorem{theoreme}{Théorème}
%\newtheorem{lemme}{Lemme}
%\newtheorem{corollaire}{Corollaire}
%\newtheorem*{remarque*}{Remarque}
%\newtheorem*{miniexercice}{Mini-exercices}
%\newtheorem{definition}{Définition}

% Commande tikz
\usetikzlibrary{calc}
\usetikzlibrary{patterns,arrows}
\usetikzlibrary{matrix}
\usetikzlibrary{fadings} 

%definition d'un terme
\newcommand{\defi}[1]{{\color{myorange}\textbf{\emph{#1}}}}
\newcommand{\evidence}[1]{{\color{blue}\textbf{\emph{#1}}}}
\newcommand{\assertion}[1]{\emph{\og#1\fg}}  % pour chapitre logique
%\renewcommand{\contentsname}{Sommaire}
\renewcommand{\contentsname}{}
\setcounter{tocdepth}{2}



%------ Encadrement ------

\usepackage{fancybox}


\newcommand{\mybox}[1]{
\setlength{\fboxsep}{7pt}
\begin{center}
\shadowbox{#1}
\end{center}}

\newcommand{\myboxinline}[1]{
\setlength{\fboxsep}{5pt}
\raisebox{-10pt}{
\shadowbox{#1}
}
}

%--------------- Commande beamer---------------
\newcommand{\beameronly}[1]{#1} % permet de mettre des pause dans beamer pas dans poly


\setbeamertemplate{navigation symbols}{}
\setbeamertemplate{footline}  % tiré du fichier beamerouterinfolines.sty
{
  \leavevmode%
  \hbox{%
  \begin{beamercolorbox}[wd=.333333\paperwidth,ht=2.25ex,dp=1ex,center]{author in head/foot}%
    % \usebeamerfont{author in head/foot}\insertshortauthor%~~(\insertshortinstitute)
    \usebeamerfont{section in head/foot}{\bf\insertshorttitle}
  \end{beamercolorbox}%
  \begin{beamercolorbox}[wd=.333333\paperwidth,ht=2.25ex,dp=1ex,center]{title in head/foot}%
    \usebeamerfont{section in head/foot}{\bf\insertsectionhead}
  \end{beamercolorbox}%
  \begin{beamercolorbox}[wd=.333333\paperwidth,ht=2.25ex,dp=1ex,right]{date in head/foot}%
    % \usebeamerfont{date in head/foot}\insertshortdate{}\hspace*{2em}
    \insertframenumber{} / \inserttotalframenumber\hspace*{2ex} 
  \end{beamercolorbox}}%
  \vskip0pt%
}


\definecolor{mygrey}{rgb}{0.5,0.5,0.5}
\setlength{\parindent}{0cm}
%\DeclareTextFontCommand{\helvetica}{\fontfamily{phv}\selectfont}

% background beamer
\definecolor{couleurhaut}{rgb}{0.85,0.9,1}  % creme
\definecolor{couleurmilieu}{rgb}{1,1,1}  % vert pale
\definecolor{couleurbas}{rgb}{0.85,0.9,1}  % blanc
\setbeamertemplate{background canvas}[vertical shading]%
[top=couleurhaut,middle=couleurmilieu,midpoint=0.4,bottom=couleurbas] 
%[top=fondtitre!05,bottom=fondtitre!60]



\makeatletter
\setbeamertemplate{theorem begin}
{%
  \begin{\inserttheoremblockenv}
  {%
    \inserttheoremheadfont
    \inserttheoremname
    \inserttheoremnumber
    \ifx\inserttheoremaddition\@empty\else\ (\inserttheoremaddition)\fi%
    \inserttheorempunctuation
  }%
}
\setbeamertemplate{theorem end}{\end{\inserttheoremblockenv}}

\newenvironment{theoreme}[1][]{%
   \setbeamercolor{block title}{fg=structure,bg=structure!40}
   \setbeamercolor{block body}{fg=black,bg=structure!10}
   \begin{block}{{\bf Th\'eor\`eme }#1}
}{%
   \end{block}%
}


\newenvironment{proposition}[1][]{%
   \setbeamercolor{block title}{fg=structure,bg=structure!40}
   \setbeamercolor{block body}{fg=black,bg=structure!10}
   \begin{block}{{\bf Proposition }#1}
}{%
   \end{block}%
}

\newenvironment{corollaire}[1][]{%
   \setbeamercolor{block title}{fg=structure,bg=structure!40}
   \setbeamercolor{block body}{fg=black,bg=structure!10}
   \begin{block}{{\bf Corollaire }#1}
}{%
   \end{block}%
}

\newenvironment{mydefinition}[1][]{%
   \setbeamercolor{block title}{fg=structure,bg=structure!40}
   \setbeamercolor{block body}{fg=black,bg=structure!10}
   \begin{block}{{\bf Définition} #1}
}{%
   \end{block}%
}

\newenvironment{lemme}[0]{%
   \setbeamercolor{block title}{fg=structure,bg=structure!40}
   \setbeamercolor{block body}{fg=black,bg=structure!10}
   \begin{block}{\bf Lemme}
}{%
   \end{block}%
}

\newenvironment{remarque}[1][]{%
   \setbeamercolor{block title}{fg=black,bg=structure!20}
   \setbeamercolor{block body}{fg=black,bg=structure!5}
   \begin{block}{Remarque #1}
}{%
   \end{block}%
}


\newenvironment{exemple}[1][]{%
   \setbeamercolor{block title}{fg=black,bg=structure!20}
   \setbeamercolor{block body}{fg=black,bg=structure!5}
   \begin{block}{{\bf Exemple }#1}
}{%
   \end{block}%
}


\newenvironment{miniexercice}[0]{%
   \setbeamercolor{block title}{fg=structure,bg=structure!20}
   \setbeamercolor{block body}{fg=black,bg=structure!5}
   \begin{block}{Mini-exercices}
}{%
   \end{block}%
}


\newenvironment{tp}[0]{%
   \setbeamercolor{block title}{fg=structure,bg=structure!40}
   \setbeamercolor{block body}{fg=black,bg=structure!10}
   \begin{block}{\bf Travaux pratiques}
}{%
   \end{block}%
}
\newenvironment{exercicecours}[1][]{%
   \setbeamercolor{block title}{fg=structure,bg=structure!40}
   \setbeamercolor{block body}{fg=black,bg=structure!10}
   \begin{block}{{\bf Exercice }#1}
}{%
   \end{block}%
}
\newenvironment{algo}[1][]{%
   \setbeamercolor{block title}{fg=structure,bg=structure!40}
   \setbeamercolor{block body}{fg=black,bg=structure!10}
   \begin{block}{{\bf Algorithme}\hfill{\color{gray}\texttt{#1}}}
}{%
   \end{block}%
}


\setbeamertemplate{proof begin}{
   \setbeamercolor{block title}{fg=black,bg=structure!20}
   \setbeamercolor{block body}{fg=black,bg=structure!5}
   \begin{block}{{\footnotesize Démonstration}}
   \footnotesize
   \smallskip}
\setbeamertemplate{proof end}{%
   \end{block}}
\setbeamertemplate{qed symbol}{\openbox}


\makeatother
\usecolortheme[RGB={0,45,179}]{structure}

%%%%%%%%%%%%%%%%%%%%%%%%%%%%%%%%%%%%%%%%%%%%%%%%%%%%%%%%%%%%%
%%%%%%%%%%%%%%%%%%%%%%%%%%%%%%%%%%%%%%%%%%%%%%%%%%%%%%%%%%%%%



\begin{document}



\title{{\bf Intégrales}}
\subtitle{Propriétés de l'intégrale}

\begin{frame}
  
  \debutmontitre

  \pause

{\footnotesize
\hfill
\setbeamercovered{transparent=50}
\begin{minipage}{0.6\textwidth}
  \begin{itemize}
    \item<3-> Relation de Chasles
    \item<4-> Positivité de l'intégrale
    \item<5-> Linéarité de l'intégrale
    \item<6-> Une preuve
  \end{itemize}
\end{minipage}
}

\end{frame}

\setcounter{framenumber}{0}


%%%%%%%%%%%%%%%%%%%%%%%%%%%%%%%%%%%%%%%%%%%%%%%%%%%%%%%%%%%%%%%%


%---------------------------------------------------------------
\section*{Relation de Chasles}


\begin{frame}

\begin{proposition}[Relation de Chasles]
Pour $a,b,c$ quelconques
\mybox{$\displaystyle\int_a^b f(x)\;dx = \int_a^c f(x)\;dx + \int_c^b f(x)\;dx$}
\end{proposition}


\pause
\bigskip

\begin{itemize}
\setlength{\itemsep}{10pt}
  \item Soient $a<c<b$. Si $f$ est intégrable sur $[a,c]$ et $[c,b]$

 alors $f$ est intégrable sur $[a,b]$

\pause

  \item $a<b  \quad \int_b^a f(x) \;dx= -\int_a^b f(x) \; dx$

\pause

  \item $\int_a^a f(x) \;dx=0$
\end{itemize}

\end{frame}





%---------------------------------------------------------------
\section*{Positivité de l'intégrale}


\begin{frame}

\begin{proposition}[Positivité de l'intégrale]
Soit $a \le b$ deux réels et $f$ et $g$ deux fonctions intégrables sur $[a,b]$
\mybox{Si $f\le g$ \quad alors \quad  $\displaystyle \int_a^b f(x)\;dx \le\int_a^b g(x)\;dx$}
\end{proposition}

\pause
\bigskip
\bigskip

En particulier l'intégrale d'une fonction positive est positive :
\mybox{Si \quad $f\ge 0$ \quad alors \quad $\displaystyle \int_a^bf(x)\;dx \ge 0$}

\end{frame}


%---------------------------------------------------------------
\section*{Linéarité de l'intégrale}


\begin{frame}

\begin{enumerate}
  \item $f+g$ est intégrable et $\int_a^b (f+g)(x) \; dx= \int_a^b f(x) \; dx+ \int_a^b g(x) \; dx$

\pause

  \item Pour tout réel $\lambda$,  $\lambda f$ est
intégrable et $\int_a^b \lambda f(x) \; dx= \lambda \int_a^b f(x) \; dx$
\end{enumerate}

\pause

\begin{proposition}[Linéarité de l'intégrale]
\mybox{$\displaystyle 
\int_a^b \big(\lambda f(x)+\mu g(x)\big)\;dx= \lambda\int_a^b f(x)\;dx+\mu\int_a^b g(x)\;dx$}
\end{proposition}

\pause
\bigskip

\begin{exemple}
$\int_0^1 \big(7x^2-e^x\big) \; dx  \pause= 7 \int_0^1 x^2\; dx \ \ - \  \int_0^1 e^x \; dx \pause= 7 \frac 13 \ - \ (e-1) \pause= \frac{10}{3}-e$
\end{exemple}

\end{frame}




\begin{frame}

\begin{proposition}
\raisebox{13pt}{\ $|f|$ est intégrable sur $[a,b]$ \ \ et} \pause
\myboxinline{$\displaystyle\left\vert\int_a^b f(x) \;dx\right\vert\le\int_a^b\big\vert f(x)\big\vert \;dx$}
\end{proposition}

\pause
\medskip

$f \times g$ est intégrable \pause
mais $\int_a^b (f g)(x)\;dx \alert{\neq}
\big(\int_a^b f(x)\;dx\big)\big(\int_a^b g(x)\;dx\big)$

\pause
\medskip

\begin{exemple}

\begin{itemize}
  \item $I_n = \int_1^n \frac{\sin(n x)}{1+x^n} \; dx$ \pause \quad  $I_n \to 0$ ?

\pause
\smallskip

  \item $|I_n| = \left| \int_1^n \frac{\sin(n x)}{1+x^n} \; dx \right| \pause\le  \int_1^n  \frac{|\sin(n x)|}{1+x^n} \; dx
\pause\le \int_1^n  \frac{1}{1+x^n} \; dx \pause\le  \int_1^n  \frac{1}{x^n} \; dx$

\pause
\smallskip

  \item  $\int_1^n  \frac{1}{x^n} \; dx \pause= \int_1^n x^{-n} \; dx \pause= \left[\frac{x^{-n+1}}{-n+1}\right]_1^n 
\pause= \frac{n^{-n+1}}{-n+1} - \frac{1}{-n+1}  \pause\xrightarrow[n\to+\infty]{} 0$
\end{itemize}

\end{exemple}
\end{frame}



%---------------------------------------------------------------
\section*{Une preuve}


\begin{frame}

\begin{proof}[]
Preuve de $\int \lambda f= \lambda \int f$
\medskip

\pause

Soit $f$ intégrable et $\lambda \ge 0$, soit $\epsilon >0 $

\pause

Il existe  $\phi^-$ et $\phi^+$ en escalier telles que $\phi^- \le f \le \phi^+$ et
\pause
\begin{equation*} 
\int_a^b f(x)\; dx  \ - \epsilon\  \le \int_a^b \phi^-(x)\;dx \pause \qquad \text{ et } \qquad \int_a^b \phi^+(x)\;dx \  \le \int_a^b f(x)\; dx  \ + \epsilon\
\end{equation*}

\pause

Subdivision $(x_0,x_1,\ldots,x_n)$  telle que $\phi^-=c_i^-$ et $\phi^+=c_i^+$ sur $]x_{i-1},x_i[$ 

\pause

\begin{itemize}
  \item $\lambda \phi^-$ et $\lambda \phi^+$ sont en escalier et $\lambda \phi^- \le \lambda f \le \lambda\phi^+$

\pause
  \item $\int_a^b \lambda \phi^-(x)\;dx =  \sum_{i=1}^n  \lambda c_i^-(x_{i}-x_{i-1})= \lambda \sum_{i=1}^n  c_i^-(x_{i}-x_{i-1}) = \lambda\int_a^b  \phi^-(x)\;dx$

\pause
  \item $\lambda\int_a^b \phi^-(x)\;dx \le I^-(\lambda f) \le I^+(\lambda f) 
\le \lambda\int_a^b \phi^+(x)\;dx$

\pause
  \item $\lambda \int_a^b f(x)\; dx \  -\lambda \epsilon \ \le I^-(\lambda f) \le I^+(\lambda f) \le  \lambda \int_a^b f(x)\; dx \ +\lambda \epsilon$

\pause
  \item Lorsque $\epsilon \to 0$ cela prouve $I^-(\lambda f) = I^+(\lambda f)$

\pause
  \item $\lambda f$ est intégrable
et $\int_a^b \lambda f(x)\; dx = \lambda \int_a^b f(x)\; dx$

\vspace*{-2ex}
\end{itemize}


\end{proof}

\end{frame}






%%%%%%%%%%%%%%%%%%%%%%%%%%%%%%%%%%%%%%%%%%%%%%%%%%%%%%%%%%%%%%%%
\section*{Mini-exercices}


\begin{frame}
\begin{miniexercice}
\begin{enumerate}
  \item En admettant que $\int_0^1 x^n \; dx = \frac1{n+1}$. Calculer l'intégrale
$\int_0^1 P(x)\; dx$ où $P(x)=a_n x^n+\cdots+a_1x+a_0$. Trouver un polynôme $P(x)$ 
non nul de degré $2$ dont l'intégrale est nulle : $\int_0^1 P(x) \; dx=0$.

  \item A-t-on $\int_a^b f(x)^2 \; dx = \left( \int_a^b f(x) \; dx \right)^2$ ;
$\int_a^b \sqrt{f(x)} \; dx = \sqrt{\int_a^b f(x) \; dx}$ ;
$\int_a^b |f(x)| \; dx = \left| \int_a^b f(x) \; dx \right|$ ; 
$\int |f(x)+g(x)| \; dx = \left| \int_a^b f(x) \; dx \right| + \left| \int_a^b g(x) \; dx \right|$ ?

  \item Peut-on trouver $a<b$ tels que $\int_a^b x\; dx = -1$ ;
$\int_a^b x\; dx = 0$ ; $\int_a^b x\; dx = +1$ ?
Mêmes questions avec $\int_a^b x^2 \; dx$.

  \item Montrer que $0 \le \int_1^2  \sin^2 x \; dx \le 1$ et $\left|\int_a^b \cos^3 x \; dx \right| \le |b-a|$.
\end{enumerate}
\end{miniexercice}
\end{frame}


\end{document}
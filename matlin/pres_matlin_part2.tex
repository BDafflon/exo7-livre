
%%%%%%%%%%%%%%%%%% PREAMBULE %%%%%%%%%%%%%%%%%%

\documentclass[aspectratio=169,utf8]{beamer}
%\documentclass[aspectratio=169,handout]{beamer}

\usetheme{Boadilla}
%\usecolortheme{seahorse}
%\usecolortheme[RGB={245,66,24}]{structure}
\useoutertheme{infolines}

% packages
\usepackage{amsfonts,amsmath,amssymb,amsthm}
\usepackage[utf8]{inputenc}
\usepackage[T1]{fontenc}
\usepackage{lmodern}

\usepackage[francais]{babel}
\usepackage{fancybox}
\usepackage{graphicx}

\usepackage{float}
\usepackage{xfrac}

%\usepackage[usenames, x11names]{xcolor}
\usepackage{pgfplots}
\usepackage{datetime}


% ----------------------------------------------------------------------
% Pour les images
\usepackage{tikz}
\usetikzlibrary{calc,shadows,arrows.meta,patterns,matrix}

\newcommand{\tikzinput}[1]{\input{figures/#1.tikz}}
% --- les figures avec échelle éventuel
\newcommand{\myfigure}[2]{% entrée : échelle, fichier(s) figure à inclure
\begin{center}\small%
\tikzstyle{every picture}=[scale=1.0*#1]% mise en échelle + 0% (automatiquement annulé à la fin du groupe)
#2%
\end{center}}



%-----  Package unités -----
\usepackage{siunitx}
\sisetup{locale = FR,detect-all,per-mode = symbol}

%\usepackage{mathptmx}
%\usepackage{fouriernc}
%\usepackage{newcent}
%\usepackage[mathcal,mathbf]{euler}

%\usepackage{palatino}
%\usepackage{newcent}
% \usepackage[mathcal,mathbf]{euler}



% \usepackage{hyperref}
% \hypersetup{colorlinks=true, linkcolor=blue, urlcolor=blue,
% pdftitle={Exo7 - Exercices de mathématiques}, pdfauthor={Exo7}}


%section
% \usepackage{sectsty}
% \allsectionsfont{\bf}
%\sectionfont{\color{Tomato3}\upshape\selectfont}
%\subsectionfont{\color{Tomato4}\upshape\selectfont}

%----- Ensembles : entiers, reels, complexes -----
\newcommand{\Nn}{\mathbb{N}} \newcommand{\N}{\mathbb{N}}
\newcommand{\Zz}{\mathbb{Z}} \newcommand{\Z}{\mathbb{Z}}
\newcommand{\Qq}{\mathbb{Q}} \newcommand{\Q}{\mathbb{Q}}
\newcommand{\Rr}{\mathbb{R}} \newcommand{\R}{\mathbb{R}}
\newcommand{\Cc}{\mathbb{C}} 
\newcommand{\Kk}{\mathbb{K}} \newcommand{\K}{\mathbb{K}}

%----- Modifications de symboles -----
\renewcommand{\epsilon}{\varepsilon}
\renewcommand{\Re}{\mathop{\text{Re}}\nolimits}
\renewcommand{\Im}{\mathop{\text{Im}}\nolimits}
%\newcommand{\llbracket}{\left[\kern-0.15em\left[}
%\newcommand{\rrbracket}{\right]\kern-0.15em\right]}

\renewcommand{\ge}{\geqslant}
\renewcommand{\geq}{\geqslant}
\renewcommand{\le}{\leqslant}
\renewcommand{\leq}{\leqslant}
\renewcommand{\epsilon}{\varepsilon}

%----- Fonctions usuelles -----
\newcommand{\ch}{\mathop{\text{ch}}\nolimits}
\newcommand{\sh}{\mathop{\text{sh}}\nolimits}
\renewcommand{\tanh}{\mathop{\text{th}}\nolimits}
\newcommand{\cotan}{\mathop{\text{cotan}}\nolimits}
\newcommand{\Arcsin}{\mathop{\text{arcsin}}\nolimits}
\newcommand{\Arccos}{\mathop{\text{arccos}}\nolimits}
\newcommand{\Arctan}{\mathop{\text{arctan}}\nolimits}
\newcommand{\Argsh}{\mathop{\text{argsh}}\nolimits}
\newcommand{\Argch}{\mathop{\text{argch}}\nolimits}
\newcommand{\Argth}{\mathop{\text{argth}}\nolimits}
\newcommand{\pgcd}{\mathop{\text{pgcd}}\nolimits} 


%----- Commandes divers ------
\newcommand{\ii}{\mathrm{i}}
\newcommand{\dd}{\text{d}}
\newcommand{\id}{\mathop{\text{id}}\nolimits}
\newcommand{\Ker}{\mathop{\text{Ker}}\nolimits}
\newcommand{\Card}{\mathop{\text{Card}}\nolimits}
\newcommand{\Vect}{\mathop{\text{Vect}}\nolimits}
\newcommand{\Mat}{\mathop{\text{Mat}}\nolimits}
\newcommand{\rg}{\mathop{\text{rg}}\nolimits}
\newcommand{\tr}{\mathop{\text{tr}}\nolimits}


%----- Structure des exercices ------

\newtheoremstyle{styleexo}% name
{2ex}% Space above
{3ex}% Space below
{}% Body font
{}% Indent amount 1
{\bfseries} % Theorem head font
{}% Punctuation after theorem head
{\newline}% Space after theorem head 2
{}% Theorem head spec (can be left empty, meaning ‘normal’)

%\theoremstyle{styleexo}
\newtheorem{exo}{Exercice}
\newtheorem{ind}{Indications}
\newtheorem{cor}{Correction}


\newcommand{\exercice}[1]{} \newcommand{\finexercice}{}
%\newcommand{\exercice}[1]{{\tiny\texttt{#1}}\vspace{-2ex}} % pour afficher le numero absolu, l'auteur...
\newcommand{\enonce}{\begin{exo}} \newcommand{\finenonce}{\end{exo}}
\newcommand{\indication}{\begin{ind}} \newcommand{\finindication}{\end{ind}}
\newcommand{\correction}{\begin{cor}} \newcommand{\fincorrection}{\end{cor}}

\newcommand{\noindication}{\stepcounter{ind}}
\newcommand{\nocorrection}{\stepcounter{cor}}

\newcommand{\fiche}[1]{} \newcommand{\finfiche}{}
\newcommand{\titre}[1]{\centerline{\large \bf #1}}
\newcommand{\addcommand}[1]{}
\newcommand{\video}[1]{}

% Marge
\newcommand{\mymargin}[1]{\marginpar{{\small #1}}}

\def\noqed{\renewcommand{\qedsymbol}{}}


%----- Presentation ------
\setlength{\parindent}{0cm}

%\newcommand{\ExoSept}{\href{http://exo7.emath.fr}{\textbf{\textsf{Exo7}}}}

\definecolor{myred}{rgb}{0.93,0.26,0}
\definecolor{myorange}{rgb}{0.97,0.58,0}
\definecolor{myyellow}{rgb}{1,0.86,0}

\newcommand{\LogoExoSept}[1]{  % input : echelle
{\usefont{U}{cmss}{bx}{n}
\begin{tikzpicture}[scale=0.1*#1,transform shape]
  \fill[color=myorange] (0,0)--(4,0)--(4,-4)--(0,-4)--cycle;
  \fill[color=myred] (0,0)--(0,3)--(-3,3)--(-3,0)--cycle;
  \fill[color=myyellow] (4,0)--(7,4)--(3,7)--(0,3)--cycle;
  \node[scale=5] at (3.5,3.5) {Exo7};
\end{tikzpicture}}
}


\newcommand{\debutmontitre}{
  \author{} \date{} 
  \thispagestyle{empty}
  \hspace*{-10ex}
  \begin{minipage}{\textwidth}
    \titlepage  
  \vspace*{-2.5cm}
  \begin{center}
    \LogoExoSept{2.5}
  \end{center}
  \end{minipage}

  \vspace*{-0cm}
  
  % Astuce pour que le background ne soit pas discrétisé lors de la conversion pdf -> png
\begin{tikzpicture}
        \fill[opacity=0,green!60!black] (0,0)--++(0,0)--++(0,0)--++(0,0)--cycle; 
\end{tikzpicture}

% toc S'affiche trop tot :
% \tableofcontents[hideallsubsections, pausesections]
}

\newcommand{\finmontitre}{
  \end{frame}
  \setcounter{framenumber}{0}
} % ne marche pas pour une raison obscure

%----- Commandes supplementaires ------

% \usepackage[landscape]{geometry}
% \geometry{top=1cm, bottom=3cm, left=2cm, right=10cm, marginparsep=1cm
% }
% \usepackage[a4paper]{geometry}
% \geometry{top=2cm, bottom=2cm, left=2cm, right=2cm, marginparsep=1cm
% }

%\usepackage{standalone}


% New command Arnaud -- november 2011
\setbeamersize{text margin left=24ex}
% si vous modifier cette valeur il faut aussi
% modifier le decalage du titre pour compenser
% (ex : ici =+10ex, titre =-5ex

\theoremstyle{definition}
%\newtheorem{proposition}{Proposition}
%\newtheorem{exemple}{Exemple}
%\newtheorem{theoreme}{Théorème}
%\newtheorem{lemme}{Lemme}
%\newtheorem{corollaire}{Corollaire}
%\newtheorem*{remarque*}{Remarque}
%\newtheorem*{miniexercice}{Mini-exercices}
%\newtheorem{definition}{Définition}

% Commande tikz
\usetikzlibrary{calc}
\usetikzlibrary{patterns,arrows}
\usetikzlibrary{matrix}
\usetikzlibrary{fadings} 

%definition d'un terme
\newcommand{\defi}[1]{{\color{myorange}\textbf{\emph{#1}}}}
\newcommand{\evidence}[1]{{\color{blue}\textbf{\emph{#1}}}}
\newcommand{\assertion}[1]{\emph{\og#1\fg}}  % pour chapitre logique
%\renewcommand{\contentsname}{Sommaire}
\renewcommand{\contentsname}{}
\setcounter{tocdepth}{2}



%------ Encadrement ------

\usepackage{fancybox}


\newcommand{\mybox}[1]{
\setlength{\fboxsep}{7pt}
\begin{center}
\shadowbox{#1}
\end{center}}

\newcommand{\myboxinline}[1]{
\setlength{\fboxsep}{5pt}
\raisebox{-10pt}{
\shadowbox{#1}
}
}

%--------------- Commande beamer---------------
\newcommand{\beameronly}[1]{#1} % permet de mettre des pause dans beamer pas dans poly


\setbeamertemplate{navigation symbols}{}
\setbeamertemplate{footline}  % tiré du fichier beamerouterinfolines.sty
{
  \leavevmode%
  \hbox{%
  \begin{beamercolorbox}[wd=.333333\paperwidth,ht=2.25ex,dp=1ex,center]{author in head/foot}%
    % \usebeamerfont{author in head/foot}\insertshortauthor%~~(\insertshortinstitute)
    \usebeamerfont{section in head/foot}{\bf\insertshorttitle}
  \end{beamercolorbox}%
  \begin{beamercolorbox}[wd=.333333\paperwidth,ht=2.25ex,dp=1ex,center]{title in head/foot}%
    \usebeamerfont{section in head/foot}{\bf\insertsectionhead}
  \end{beamercolorbox}%
  \begin{beamercolorbox}[wd=.333333\paperwidth,ht=2.25ex,dp=1ex,right]{date in head/foot}%
    % \usebeamerfont{date in head/foot}\insertshortdate{}\hspace*{2em}
    \insertframenumber{} / \inserttotalframenumber\hspace*{2ex} 
  \end{beamercolorbox}}%
  \vskip0pt%
}


\definecolor{mygrey}{rgb}{0.5,0.5,0.5}
\setlength{\parindent}{0cm}
%\DeclareTextFontCommand{\helvetica}{\fontfamily{phv}\selectfont}

% background beamer
\definecolor{couleurhaut}{rgb}{0.85,0.9,1}  % creme
\definecolor{couleurmilieu}{rgb}{1,1,1}  % vert pale
\definecolor{couleurbas}{rgb}{0.85,0.9,1}  % blanc
\setbeamertemplate{background canvas}[vertical shading]%
[top=couleurhaut,middle=couleurmilieu,midpoint=0.4,bottom=couleurbas] 
%[top=fondtitre!05,bottom=fondtitre!60]



\makeatletter
\setbeamertemplate{theorem begin}
{%
  \begin{\inserttheoremblockenv}
  {%
    \inserttheoremheadfont
    \inserttheoremname
    \inserttheoremnumber
    \ifx\inserttheoremaddition\@empty\else\ (\inserttheoremaddition)\fi%
    \inserttheorempunctuation
  }%
}
\setbeamertemplate{theorem end}{\end{\inserttheoremblockenv}}

\newenvironment{theoreme}[1][]{%
   \setbeamercolor{block title}{fg=structure,bg=structure!40}
   \setbeamercolor{block body}{fg=black,bg=structure!10}
   \begin{block}{{\bf Th\'eor\`eme }#1}
}{%
   \end{block}%
}


\newenvironment{proposition}[1][]{%
   \setbeamercolor{block title}{fg=structure,bg=structure!40}
   \setbeamercolor{block body}{fg=black,bg=structure!10}
   \begin{block}{{\bf Proposition }#1}
}{%
   \end{block}%
}

\newenvironment{corollaire}[1][]{%
   \setbeamercolor{block title}{fg=structure,bg=structure!40}
   \setbeamercolor{block body}{fg=black,bg=structure!10}
   \begin{block}{{\bf Corollaire }#1}
}{%
   \end{block}%
}

\newenvironment{mydefinition}[1][]{%
   \setbeamercolor{block title}{fg=structure,bg=structure!40}
   \setbeamercolor{block body}{fg=black,bg=structure!10}
   \begin{block}{{\bf Définition} #1}
}{%
   \end{block}%
}

\newenvironment{lemme}[0]{%
   \setbeamercolor{block title}{fg=structure,bg=structure!40}
   \setbeamercolor{block body}{fg=black,bg=structure!10}
   \begin{block}{\bf Lemme}
}{%
   \end{block}%
}

\newenvironment{remarque}[1][]{%
   \setbeamercolor{block title}{fg=black,bg=structure!20}
   \setbeamercolor{block body}{fg=black,bg=structure!5}
   \begin{block}{Remarque #1}
}{%
   \end{block}%
}


\newenvironment{exemple}[1][]{%
   \setbeamercolor{block title}{fg=black,bg=structure!20}
   \setbeamercolor{block body}{fg=black,bg=structure!5}
   \begin{block}{{\bf Exemple }#1}
}{%
   \end{block}%
}


\newenvironment{miniexercice}[0]{%
   \setbeamercolor{block title}{fg=structure,bg=structure!20}
   \setbeamercolor{block body}{fg=black,bg=structure!5}
   \begin{block}{Mini-exercices}
}{%
   \end{block}%
}


\newenvironment{tp}[0]{%
   \setbeamercolor{block title}{fg=structure,bg=structure!40}
   \setbeamercolor{block body}{fg=black,bg=structure!10}
   \begin{block}{\bf Travaux pratiques}
}{%
   \end{block}%
}
\newenvironment{exercicecours}[1][]{%
   \setbeamercolor{block title}{fg=structure,bg=structure!40}
   \setbeamercolor{block body}{fg=black,bg=structure!10}
   \begin{block}{{\bf Exercice }#1}
}{%
   \end{block}%
}
\newenvironment{algo}[1][]{%
   \setbeamercolor{block title}{fg=structure,bg=structure!40}
   \setbeamercolor{block body}{fg=black,bg=structure!10}
   \begin{block}{{\bf Algorithme}\hfill{\color{gray}\texttt{#1}}}
}{%
   \end{block}%
}


\setbeamertemplate{proof begin}{
   \setbeamercolor{block title}{fg=black,bg=structure!20}
   \setbeamercolor{block body}{fg=black,bg=structure!5}
   \begin{block}{{\footnotesize Démonstration}}
   \footnotesize
   \smallskip}
\setbeamertemplate{proof end}{%
   \end{block}}
\setbeamertemplate{qed symbol}{\openbox}


\makeatother
\usecolortheme[RGB={56,98,238}]{structure}
    
%%%%%%%%%%%%%%%%%%%%%%%%%%%%%%%%%%%%%%%%%%%%%%%%%%%%%%%%%%%%%
%%%%%%%%%%%%%%%%%%%%%%%%%%%%%%%%%%%%%%%%%%%%%%%%%%%%%%%%%%%%%


\begin{document}


\title{{\bf Matrices et applications linéaires}}
\subtitle{Applications linéaires en dimension finie}

\begin{frame}
  
  \debutmontitre

  \pause

{\footnotesize
\hfill
\setbeamercovered{transparent=50}
\begin{minipage}{0.6\textwidth}
  \begin{itemize}
    \item<3-> Construction et caractérisation
    \item<4-> Rang d'une application linéaire
    \item<5-> Théorème du rang
    \item<6-> Application linéaire entre deux espaces de même dimension
  \end{itemize}
\end{minipage}
}

\end{frame}

\setcounter{framenumber}{0}


%%%%%%%%%%%%%%%%%%%%%%%%%%%%%%%%%%%%%%%%%%%%%%%%%%%%%%%%%%%%%%%%
\section{Construction et caractérisation}

\begin{frame}

Soient $E$ et $F$ deux espaces vectoriels sur un même corps $\Kk$

\pause

\begin{theoreme}[Construction d'une application linéaire]
Supposons $E$ de dimension finie $n$
et soit $(e_1,\dots,e_n)$ une base de $E$

Alors pour tout choix $(v_1, \ldots ,v_n)$ 
de $n$ vecteurs de $F$, il existe une et une seule 
application linéaire $f : E \to F$
 telle que pour tout $i=1,\ldots,n$ 
$$f(e_i)=v_i$$
\end{theoreme}  

\bigskip
\pause

\begin{exemple}
Il existe une unique application linéaire $f : \Rr^n \to \Rr[X]$  
telle que 

\centerline{$f(e_i) = (X+1)^i$ pour $i=1,\ldots,n$}
\vspace*{-4ex}
\small
\pause
$$f(x_1,\ldots,x_n) 
\pause = f(x_1e_1+\cdots +x_n e_n) 
\pause = x_1f(e_1)+\cdots +x_nf(e_n) 
\pause = \sum_{i=1}^n x_i(X+1)^i$$%

\vspace*{-2ex}
\end{exemple}


\end{frame}


%%%%%%%%%%%%%%%%%%%%%%%%%%%%%%%%%%%%%%%%%%%%%%%%%%%%%%%%%%%%%%%%
\section{Rang d'une application linéaire}

\begin{frame}


$f : E \to F$ une application linéaire

$\Im f = f(E) =  \big\{ f(x) \ | \ x \in E \big\}$

\pause

\begin{proposition}
Si $E$ est de dimension finie, alors 
\begin{itemize}
  \item $\Im f = f(E)$ est un espace vectoriel de dimension finie
  \pause
  \item Si $(e_1,\ldots,e_n)$ est une base de $E$, alors
  
  \centerline{$\Im f = \Vect \big( f(e_1),\ldots,f(e_n) \big)$}
  
  \pause
\end{itemize}

\medskip

La dimension de cet espace vectoriel $\Im f$ est appelée \defi{rang de $f$}
\vspace*{-1ex}
\mybox{$\rg (f) = \dim \Im f = \dim \Vect \big( f(e_1),\ldots,f(e_n) \big)$}
\end{proposition}

\pause
\begin{proposition}
Si $E$ et $F$ sont de dimension finie 
alors 
\vspace*{-2ex}
$$\rg(f) \le \min \left ( \dim E, \dim F \right)$$
\end{proposition}
\end{frame}


\begin{frame}
\begin{exemple}[Calcul du rang de $f$]
$f : \Rr^3 \to \Rr^2$ \qquad $f(x,y,z) = (3x-4y+2z,2x-3y-z)$
\pause
\begin{itemize}
  \item $e_1 = \left(\begin{smallmatrix} 1 \\ 0 \\ 0 \end{smallmatrix}\right)$ \quad 
$e_2=\left(\begin{smallmatrix} 0 \\ 1 \\ 0 \end{smallmatrix}\right)$ \quad
$e_3=\left(\begin{smallmatrix} 0 \\ 0 \\ 1 \end{smallmatrix}\right)$
\pause
  \item $(e_1,e_2,e_3)$ est la base canonique de $\Rr^3$
\pause  
  \item $v_1 = f(e_1) = f\left(\begin{smallmatrix} 1 \\ 0 \\ 0\end{smallmatrix}\right) 
= \left(\begin{smallmatrix} 3 \\ 2 \end{smallmatrix}\right)$

\pause
$\qquad
v_2 = f(e_2) = f\left(\begin{smallmatrix} 0 \\ 1 \\ 0\end{smallmatrix}\right) 
= \left(\begin{smallmatrix} -4 \\ -3 \end{smallmatrix}\right)$

\pause
$\qquad\qquad v_3 = f(e_3) = f\left(\begin{smallmatrix} 0 \\ 0 \\ 1 \end{smallmatrix}\right) 
= \left(\begin{smallmatrix} 2 \\ -1 \end{smallmatrix}\right)
$

\pause
  \item $\rg f = \rg A$ où $A = \begin{pmatrix}
3 & -4 & 2 \\
2 & -3 & -1 \\
\end{pmatrix}$

\pause
  \item 
  \begin{itemize}
    \item $3$ vecteurs donc $\rg f \le 3$
\pause  
     \item  $\rg f \le \dim \Rr^2 = 2$
\pause  
     \item $f \neq 0$ donc $\rg f \ge 1$
\pause  
     \item $\rg f = 1$ ou $2$
   \end{itemize}
   
\pause 
  \item $v_1$ et $v_2$ sont linéairement indépendants, donc $\rg f = 2$
\end{itemize}  

\end{exemple}

\end{frame}


%%%%%%%%%%%%%%%%%%%%%%%%%%%%%%%%%%%%%%%%%%%%%%%%%%%%%%%%%%%%%%%%
\section{Théorème du rang}

\begin{frame}

\begin{itemize}
  \item $f : E \to F$ est une application linéaire
\pause  
  \item $E$ est de dimension finie
\pause  
  \item le \defi{noyau} de $f$ est $\Ker f=\big\{x \in E \mid f(x)=0_{F}\big\}$
 \pause 
  \item l'\defi{image} de $f$ est $\Im f = f(E) = \big\{ f(x) \mid x \in E \big\}$
\end{itemize}

\pause

\begin{theoreme}[du rang]
\mybox{$\dim E =\dim \Ker f + \dim \Im f$}
\end{theoreme}

\pause

\mybox{$\dim E = \dim \Ker f + \rg f$}


\end{frame}

\begin{frame}
\begin{exemple}
$f : \Rr^4 \rightarrow  \Rr^3$

$f(x_1,x_2,x_3,x_4)=(x_1-x_2+x_3,2x_1+2x_2+6x_3+4x_4,-x_1-2x_3-x_4)$

\smallskip
\pause
\evidence{Première méthode} On calcule d'abord le noyau
\pause
{\small
\begin{itemize}
  \item   $(x_1,x_2,x_3,x_4) \in \Ker f \iff f(x_1,x_2,x_3,x_4) = (0,0,0)$
\pause

  $\iff \left\{ 
  \begin{array}{cccccccccc}
   x_1  &-& x_2  &+& x_3  & &      &=& 0 \\
   2x_1 &+& 2x_2 &+& 6x_3 &+& 4x_4 &=& 0 \\
   -x_1 & &      &-& 2x_3 &-& x_4  &=& 0 
  \end{array}\right.$

\pause  
  $\iff \left\{ 
  \begin{array}{cccccccccc}
   x_1  &-& x_2  &+& x_3  & &      &=& 0 \\
        & & x_2  &+& x_3  &+& x_4  &=& 0 
  \end{array}  \right.$

  \vspace*{-1ex}
\pause
 
  \item $\Ker f = \bigg\{ (-2x_3-x_4,-x_3-x_4,x_3,x_4) \mid x_3,x_4 \in \Rr\bigg\}$
  
  \vspace*{-1ex}
  
\pause  
  \item $\Ker f  = \left\{ x_3 \left(\begin{smallmatrix}-2 \\-1 \\1 \\0 \end{smallmatrix}\right)
  + x_4 \left(\begin{smallmatrix}-1\\-1\\0\\1 \end{smallmatrix}\right)\mid x_3,x_4 \in \Rr \right\}
\pause  = \Vect \left( \left(\begin{smallmatrix}-2 \\-1 \\1 \\0 \end{smallmatrix}\right),
  \left(\begin{smallmatrix}-1\\-1\\0\\1 \end{smallmatrix}\right)
  \right) $

  \pause
  \item $\dim \Ker f = 2$

\end{itemize}
}

\pause
Théorème du rang : $\dim \Im f \pause= \dim \Rr^4 - \dim \Ker f \pause= 4-2 = 2$
\end{exemple}

\end{frame}


\begin{frame}
\begin{exemple}
$f : \Rr^4 \rightarrow  \Rr^3$

$f(x_1,x_2,x_3,x_4)=(x_1-x_2+x_3,2x_1+2x_2+6x_3+4x_4,-x_1-2x_3-x_4)$

\smallskip
\pause
\evidence{Deuxième méthode} On calcule d'abord l'image
\pause 
{\small
\begin{itemize}
  \item $(e_1,e_2,e_3,e_4)$ la base canonique de $\Rr^4$
  
  \pause
  \item   
  $v_1 = f(e_1) = f\left(\begin{smallmatrix} 1 \\ 0 \\ 0 \\0 \end{smallmatrix}\right) 
= \left(\begin{smallmatrix} 1 \\ 2 \\ -1 \end{smallmatrix}\right) \quad
\pause
v_2 = f(e_2) = f\left(\begin{smallmatrix} 0 \\ 1 \\ 0 \\ 0\end{smallmatrix}\right) 
= \left(\begin{smallmatrix} -1 \\ 2 \\ 0 \end{smallmatrix}\right)$

\pause
$
v_3 = f(e_3) = f\left(\begin{smallmatrix} 0 \\ 0 \\ 1 \\  0\end{smallmatrix}\right)
= \left(\begin{smallmatrix} 1 \\ 6 \\ -2 \end{smallmatrix}\right) \quad
\pause
v_4 = f(e_4) = f\left(\begin{smallmatrix} 0 \\ 0 \\ 0 \\1 \end{smallmatrix}\right) 
= \left(\begin{smallmatrix} 0 \\ 4 \\ -1 \end{smallmatrix}\right)
$

\pause  
  \item $A = \begin{pmatrix}
    1&-1&1&0\\
    2&2&6&4\\
    -1&0&-2&-1\\
  \end{pmatrix}
\pause  \sim 
  \begin{pmatrix}
    1&0&0&0\\
    2&4&0&0\\
    -1&-1&0&0\\
  \end{pmatrix}$

 \pause 
  \item $\rg A = 2$
 
 \pause 
  \item Ainsi $\rg f = \dim \Im f = \dim \Vect\big( f(e_1),f(e_2),f(e_3),f(e_4) \big) = 2$
  

\end{itemize}
}

\pause 
Théorème du rang : $\dim \Ker f = \dim \Rr^4 - \rg f = 4-2 =2$

\end{exemple}

\end{frame}



\begin{frame}
\begin{exemple}
$$\begin{array}{rcl}
f : \Rr_n[X] &\longrightarrow&  \Rr_n[X] \\
        P(X) &\longmapsto& P''(X) 
  \end{array}$$
  
\pause

Quel est le rang et la dimension du noyau de $f$ ?

\pause
\begin{itemize}
  \item \evidence{Première méthode} on calcule d'abord le noyau
  \pause
  \begin{itemize}
    \item $P(X) \in \Ker f  \iff P''(X) = 0 \iff P'(X)=a \Leftrightarrow P(X) = aX+b$
    \pause
    \item $\Ker f = \Vect(1,X)$ donc $\dim \Ker f = 2$
    \pause
    \item Par le théorème du rang, $\rg f = \dim \Im f = \dim \Rr_n[X] - \dim \Ker f = (n+1) - 2 = n-1$
  \end{itemize}
  
  \pause
  \item \evidence{Deuxième méthode} on calcule d'abord l'image
  \begin{itemize}
  \pause
    \item $(1,X,X^2,\ldots,X^n)$ est une base de l'espace de départ $\Rr_n[X]$
    \pause
    \item donc $\rg f = \dim \Im f = \dim \Vect \big(f(1), f(X), \ldots, f(X^n) \big)$
    \pause
    \item $f(1)=0$ et $f(X)=0$\pause, pour $k \ge 2$, $f(X^k) = k(k-1)X^{k-2}$
    \pause
    \item $\big\{f(X^2), f(X^3), \ldots, f(X^n) \big\} 
    = \big\{2, 6X, 12X^2, \ldots, n(n-1)X^{n-2}\big\}$
    \pause
    \item Degrés échelonnés implique $\rg f = n-1$
    \pause
    \item Par le théorème du rang, $\dim \Ker f = \dim \Rr_n[X] - \rg f = 2$
  \end{itemize}
\end{itemize}
\end{exemple}
\end{frame}

%%%%%%%%%%%%%%%%%%%%%%%%%%%%%%%%%%%%%%%%%%%%%%%%%%%%%%%%%%%%%%%%
\section{Entre espaces de même dimension}

\begin{frame}

Un \defi{isomorphisme} est une application linéaire bijective
\begin{proposition}
Si $f : E \to F$ un isomorphisme entre espaces vectoriels de dimension finie
alors 
\vspace*{-1ex}
$$\dim E =\dim F$$
\end{proposition}

\pause

\begin{theoreme}
\label{th:eqapplinbij}
Soit $f : E \to F$ une application linéaire où $E$ et $F$ 
de dimension finie

Si $\dim E = \dim F$, alors les assertions suivantes sont équivalentes
\pause
\begin{itemize}
  \item[(i)] $f$ est bijective
  \item[(ii)] $f$ est injective
  \item[(iii)] $f$ est surjective
\end{itemize}
\end{theoreme}

\pause

\begin{proof} 
$f$ injective 
\pause
$\iff$ $\Ker f = \{0\}$
\pause
$\iff$ $\dim \Im f =\dim E$
\pause

\hfill $\iff$ $\dim \Im f=\dim F$
\pause
$\iff$ $\Im f =F$
\pause
$\iff$ $f$ surjective
\end{proof}


\end{frame}

\begin{frame}
\begin{exemple}
\begin{itemize}
  \item $f : \Rr^2 \to \Rr^2$ définie par $f(x,y) = (x-y,x+y)$
  \pause 
  \item $f$ est-elle bijective ?
  \pause 
  \item Espace de départ et espace d'arrivée de même dimension
  \pause 
  \item 
  $(x,y) \in \Ker f \pause \iff f(x,y)=0 \pause \iff (x-y,x+y)=(0,0)$

  $\pause \iff 
\left\{
\begin{array}{rcl}
x+y & = & 0 \\
x-y & = & 0 \\
\end{array}
\right. 
% \iff 
% \left\{
% \begin{array}{rcl}
% x & = & 0 \\
% y & = & 0 \\
% \end{array}
% \right.
\pause \iff (x,y) = (0,0)$
  \pause 
  \item $\Ker f = \big\{(0,0)\big\}$ donc $f$ est injective
  \pause 
  \item Par le théorème précédent $f$ est un isomorphisme
\end{itemize}

\end{exemple}
\end{frame}


%%%%%%%%%%%%%%%%%%%%%%%%%%%%%%%%%%%%%%%%%%%%%%%%%%%%%%%%%%%%%%%%
\section{Mini-exercices}

\begin{frame}
\begin{miniexercice}
\begin{enumerate}
  \item Soit $(e_1,e_2,e_3)$ la base canonique de $\Rr^3$. Donner l'expression
  de $f(x,y,z)$ où $f : \Rr^3 \to \Rr^3$
  est l'application linéaire 
  qui envoie $e_1$ sur son opposé,
  qui envoie $e_2$ sur le vecteur nul et
  qui envoie $e_3$ sur la somme des trois vecteurs $e_1, e_2, e_3$.
  
  \item Soit $f : \Rr^3 \to \Rr^2$ définie par $f(x,y,z)=(x-2y-3z,2y+3z)$.
  Calculer une base du noyau de $f$, une base de l'image de $f$ et vérifier le théorème du rang.
  
  \item Idem avec $f : \Rr^3 \to \Rr^3$, $f(x,y,z)=(-y+z,x+z,x+y)$.
  
  \item Même question avec l'application linéaire $f : \Rr_n[X] \to \Rr_n[X]$ qui à 
  $X^k$ associe $X^{k-1}$ pour $1 \le k \le n$ et qui à $1$ associe $0$.
  
  \item Lorsque c'est possible, calculer la dimension du noyau, le rang et dire 
  si $f$ peut être injective, surjective, bijective :
  \vspace*{-0.5ex}
    \begin{itemize}
      \item Une application linéaire surjective $f : \Rr^7 \to \Rr^4$.
      \item Une application linéaire injective $f : \Rr^5 \to \Rr^8$.
      \item Une application linéaire surjective $f : \Rr^4 \to \Rr^4$.
      \item Une application linéaire injective $f : \Rr^6 \to \Rr^6$.      
    \end{itemize}

\vspace*{-1ex}  
\end{enumerate}

\end{miniexercice}

\end{frame}

\end{document}

%%%%%%%%%%%%%%%%%% PREAMBULE %%%%%%%%%%%%%%%%%%

\documentclass[aspectratio=169,utf8]{beamer}
%\documentclass[aspectratio=169,handout]{beamer}

\usetheme{Boadilla}
%\usecolortheme{seahorse}
%\usecolortheme[RGB={245,66,24}]{structure}
\useoutertheme{infolines}

% packages
\usepackage{amsfonts,amsmath,amssymb,amsthm}
\usepackage[utf8]{inputenc}
\usepackage[T1]{fontenc}
\usepackage{lmodern}

\usepackage[francais]{babel}
\usepackage{fancybox}
\usepackage{graphicx}

\usepackage{float}
\usepackage{xfrac}

%\usepackage[usenames, x11names]{xcolor}
\usepackage{pgfplots}
\usepackage{datetime}


% ----------------------------------------------------------------------
% Pour les images
\usepackage{tikz}
\usetikzlibrary{calc,shadows,arrows.meta,patterns,matrix}

\newcommand{\tikzinput}[1]{\input{figures/#1.tikz}}
% --- les figures avec échelle éventuel
\newcommand{\myfigure}[2]{% entrée : échelle, fichier(s) figure à inclure
\begin{center}\small%
\tikzstyle{every picture}=[scale=1.0*#1]% mise en échelle + 0% (automatiquement annulé à la fin du groupe)
#2%
\end{center}}



%-----  Package unités -----
\usepackage{siunitx}
\sisetup{locale = FR,detect-all,per-mode = symbol}

%\usepackage{mathptmx}
%\usepackage{fouriernc}
%\usepackage{newcent}
%\usepackage[mathcal,mathbf]{euler}

%\usepackage{palatino}
%\usepackage{newcent}
% \usepackage[mathcal,mathbf]{euler}



% \usepackage{hyperref}
% \hypersetup{colorlinks=true, linkcolor=blue, urlcolor=blue,
% pdftitle={Exo7 - Exercices de mathématiques}, pdfauthor={Exo7}}


%section
% \usepackage{sectsty}
% \allsectionsfont{\bf}
%\sectionfont{\color{Tomato3}\upshape\selectfont}
%\subsectionfont{\color{Tomato4}\upshape\selectfont}

%----- Ensembles : entiers, reels, complexes -----
\newcommand{\Nn}{\mathbb{N}} \newcommand{\N}{\mathbb{N}}
\newcommand{\Zz}{\mathbb{Z}} \newcommand{\Z}{\mathbb{Z}}
\newcommand{\Qq}{\mathbb{Q}} \newcommand{\Q}{\mathbb{Q}}
\newcommand{\Rr}{\mathbb{R}} \newcommand{\R}{\mathbb{R}}
\newcommand{\Cc}{\mathbb{C}} 
\newcommand{\Kk}{\mathbb{K}} \newcommand{\K}{\mathbb{K}}

%----- Modifications de symboles -----
\renewcommand{\epsilon}{\varepsilon}
\renewcommand{\Re}{\mathop{\text{Re}}\nolimits}
\renewcommand{\Im}{\mathop{\text{Im}}\nolimits}
%\newcommand{\llbracket}{\left[\kern-0.15em\left[}
%\newcommand{\rrbracket}{\right]\kern-0.15em\right]}

\renewcommand{\ge}{\geqslant}
\renewcommand{\geq}{\geqslant}
\renewcommand{\le}{\leqslant}
\renewcommand{\leq}{\leqslant}
\renewcommand{\epsilon}{\varepsilon}

%----- Fonctions usuelles -----
\newcommand{\ch}{\mathop{\text{ch}}\nolimits}
\newcommand{\sh}{\mathop{\text{sh}}\nolimits}
\renewcommand{\tanh}{\mathop{\text{th}}\nolimits}
\newcommand{\cotan}{\mathop{\text{cotan}}\nolimits}
\newcommand{\Arcsin}{\mathop{\text{arcsin}}\nolimits}
\newcommand{\Arccos}{\mathop{\text{arccos}}\nolimits}
\newcommand{\Arctan}{\mathop{\text{arctan}}\nolimits}
\newcommand{\Argsh}{\mathop{\text{argsh}}\nolimits}
\newcommand{\Argch}{\mathop{\text{argch}}\nolimits}
\newcommand{\Argth}{\mathop{\text{argth}}\nolimits}
\newcommand{\pgcd}{\mathop{\text{pgcd}}\nolimits} 


%----- Commandes divers ------
\newcommand{\ii}{\mathrm{i}}
\newcommand{\dd}{\text{d}}
\newcommand{\id}{\mathop{\text{id}}\nolimits}
\newcommand{\Ker}{\mathop{\text{Ker}}\nolimits}
\newcommand{\Card}{\mathop{\text{Card}}\nolimits}
\newcommand{\Vect}{\mathop{\text{Vect}}\nolimits}
\newcommand{\Mat}{\mathop{\text{Mat}}\nolimits}
\newcommand{\rg}{\mathop{\text{rg}}\nolimits}
\newcommand{\tr}{\mathop{\text{tr}}\nolimits}


%----- Structure des exercices ------

\newtheoremstyle{styleexo}% name
{2ex}% Space above
{3ex}% Space below
{}% Body font
{}% Indent amount 1
{\bfseries} % Theorem head font
{}% Punctuation after theorem head
{\newline}% Space after theorem head 2
{}% Theorem head spec (can be left empty, meaning ‘normal’)

%\theoremstyle{styleexo}
\newtheorem{exo}{Exercice}
\newtheorem{ind}{Indications}
\newtheorem{cor}{Correction}


\newcommand{\exercice}[1]{} \newcommand{\finexercice}{}
%\newcommand{\exercice}[1]{{\tiny\texttt{#1}}\vspace{-2ex}} % pour afficher le numero absolu, l'auteur...
\newcommand{\enonce}{\begin{exo}} \newcommand{\finenonce}{\end{exo}}
\newcommand{\indication}{\begin{ind}} \newcommand{\finindication}{\end{ind}}
\newcommand{\correction}{\begin{cor}} \newcommand{\fincorrection}{\end{cor}}

\newcommand{\noindication}{\stepcounter{ind}}
\newcommand{\nocorrection}{\stepcounter{cor}}

\newcommand{\fiche}[1]{} \newcommand{\finfiche}{}
\newcommand{\titre}[1]{\centerline{\large \bf #1}}
\newcommand{\addcommand}[1]{}
\newcommand{\video}[1]{}

% Marge
\newcommand{\mymargin}[1]{\marginpar{{\small #1}}}

\def\noqed{\renewcommand{\qedsymbol}{}}


%----- Presentation ------
\setlength{\parindent}{0cm}

%\newcommand{\ExoSept}{\href{http://exo7.emath.fr}{\textbf{\textsf{Exo7}}}}

\definecolor{myred}{rgb}{0.93,0.26,0}
\definecolor{myorange}{rgb}{0.97,0.58,0}
\definecolor{myyellow}{rgb}{1,0.86,0}

\newcommand{\LogoExoSept}[1]{  % input : echelle
{\usefont{U}{cmss}{bx}{n}
\begin{tikzpicture}[scale=0.1*#1,transform shape]
  \fill[color=myorange] (0,0)--(4,0)--(4,-4)--(0,-4)--cycle;
  \fill[color=myred] (0,0)--(0,3)--(-3,3)--(-3,0)--cycle;
  \fill[color=myyellow] (4,0)--(7,4)--(3,7)--(0,3)--cycle;
  \node[scale=5] at (3.5,3.5) {Exo7};
\end{tikzpicture}}
}


\newcommand{\debutmontitre}{
  \author{} \date{} 
  \thispagestyle{empty}
  \hspace*{-10ex}
  \begin{minipage}{\textwidth}
    \titlepage  
  \vspace*{-2.5cm}
  \begin{center}
    \LogoExoSept{2.5}
  \end{center}
  \end{minipage}

  \vspace*{-0cm}
  
  % Astuce pour que le background ne soit pas discrétisé lors de la conversion pdf -> png
\begin{tikzpicture}
        \fill[opacity=0,green!60!black] (0,0)--++(0,0)--++(0,0)--++(0,0)--cycle; 
\end{tikzpicture}

% toc S'affiche trop tot :
% \tableofcontents[hideallsubsections, pausesections]
}

\newcommand{\finmontitre}{
  \end{frame}
  \setcounter{framenumber}{0}
} % ne marche pas pour une raison obscure

%----- Commandes supplementaires ------

% \usepackage[landscape]{geometry}
% \geometry{top=1cm, bottom=3cm, left=2cm, right=10cm, marginparsep=1cm
% }
% \usepackage[a4paper]{geometry}
% \geometry{top=2cm, bottom=2cm, left=2cm, right=2cm, marginparsep=1cm
% }

%\usepackage{standalone}


% New command Arnaud -- november 2011
\setbeamersize{text margin left=24ex}
% si vous modifier cette valeur il faut aussi
% modifier le decalage du titre pour compenser
% (ex : ici =+10ex, titre =-5ex

\theoremstyle{definition}
%\newtheorem{proposition}{Proposition}
%\newtheorem{exemple}{Exemple}
%\newtheorem{theoreme}{Théorème}
%\newtheorem{lemme}{Lemme}
%\newtheorem{corollaire}{Corollaire}
%\newtheorem*{remarque*}{Remarque}
%\newtheorem*{miniexercice}{Mini-exercices}
%\newtheorem{definition}{Définition}

% Commande tikz
\usetikzlibrary{calc}
\usetikzlibrary{patterns,arrows}
\usetikzlibrary{matrix}
\usetikzlibrary{fadings} 

%definition d'un terme
\newcommand{\defi}[1]{{\color{myorange}\textbf{\emph{#1}}}}
\newcommand{\evidence}[1]{{\color{blue}\textbf{\emph{#1}}}}
\newcommand{\assertion}[1]{\emph{\og#1\fg}}  % pour chapitre logique
%\renewcommand{\contentsname}{Sommaire}
\renewcommand{\contentsname}{}
\setcounter{tocdepth}{2}



%------ Encadrement ------

\usepackage{fancybox}


\newcommand{\mybox}[1]{
\setlength{\fboxsep}{7pt}
\begin{center}
\shadowbox{#1}
\end{center}}

\newcommand{\myboxinline}[1]{
\setlength{\fboxsep}{5pt}
\raisebox{-10pt}{
\shadowbox{#1}
}
}

%--------------- Commande beamer---------------
\newcommand{\beameronly}[1]{#1} % permet de mettre des pause dans beamer pas dans poly


\setbeamertemplate{navigation symbols}{}
\setbeamertemplate{footline}  % tiré du fichier beamerouterinfolines.sty
{
  \leavevmode%
  \hbox{%
  \begin{beamercolorbox}[wd=.333333\paperwidth,ht=2.25ex,dp=1ex,center]{author in head/foot}%
    % \usebeamerfont{author in head/foot}\insertshortauthor%~~(\insertshortinstitute)
    \usebeamerfont{section in head/foot}{\bf\insertshorttitle}
  \end{beamercolorbox}%
  \begin{beamercolorbox}[wd=.333333\paperwidth,ht=2.25ex,dp=1ex,center]{title in head/foot}%
    \usebeamerfont{section in head/foot}{\bf\insertsectionhead}
  \end{beamercolorbox}%
  \begin{beamercolorbox}[wd=.333333\paperwidth,ht=2.25ex,dp=1ex,right]{date in head/foot}%
    % \usebeamerfont{date in head/foot}\insertshortdate{}\hspace*{2em}
    \insertframenumber{} / \inserttotalframenumber\hspace*{2ex} 
  \end{beamercolorbox}}%
  \vskip0pt%
}


\definecolor{mygrey}{rgb}{0.5,0.5,0.5}
\setlength{\parindent}{0cm}
%\DeclareTextFontCommand{\helvetica}{\fontfamily{phv}\selectfont}

% background beamer
\definecolor{couleurhaut}{rgb}{0.85,0.9,1}  % creme
\definecolor{couleurmilieu}{rgb}{1,1,1}  % vert pale
\definecolor{couleurbas}{rgb}{0.85,0.9,1}  % blanc
\setbeamertemplate{background canvas}[vertical shading]%
[top=couleurhaut,middle=couleurmilieu,midpoint=0.4,bottom=couleurbas] 
%[top=fondtitre!05,bottom=fondtitre!60]



\makeatletter
\setbeamertemplate{theorem begin}
{%
  \begin{\inserttheoremblockenv}
  {%
    \inserttheoremheadfont
    \inserttheoremname
    \inserttheoremnumber
    \ifx\inserttheoremaddition\@empty\else\ (\inserttheoremaddition)\fi%
    \inserttheorempunctuation
  }%
}
\setbeamertemplate{theorem end}{\end{\inserttheoremblockenv}}

\newenvironment{theoreme}[1][]{%
   \setbeamercolor{block title}{fg=structure,bg=structure!40}
   \setbeamercolor{block body}{fg=black,bg=structure!10}
   \begin{block}{{\bf Th\'eor\`eme }#1}
}{%
   \end{block}%
}


\newenvironment{proposition}[1][]{%
   \setbeamercolor{block title}{fg=structure,bg=structure!40}
   \setbeamercolor{block body}{fg=black,bg=structure!10}
   \begin{block}{{\bf Proposition }#1}
}{%
   \end{block}%
}

\newenvironment{corollaire}[1][]{%
   \setbeamercolor{block title}{fg=structure,bg=structure!40}
   \setbeamercolor{block body}{fg=black,bg=structure!10}
   \begin{block}{{\bf Corollaire }#1}
}{%
   \end{block}%
}

\newenvironment{mydefinition}[1][]{%
   \setbeamercolor{block title}{fg=structure,bg=structure!40}
   \setbeamercolor{block body}{fg=black,bg=structure!10}
   \begin{block}{{\bf Définition} #1}
}{%
   \end{block}%
}

\newenvironment{lemme}[0]{%
   \setbeamercolor{block title}{fg=structure,bg=structure!40}
   \setbeamercolor{block body}{fg=black,bg=structure!10}
   \begin{block}{\bf Lemme}
}{%
   \end{block}%
}

\newenvironment{remarque}[1][]{%
   \setbeamercolor{block title}{fg=black,bg=structure!20}
   \setbeamercolor{block body}{fg=black,bg=structure!5}
   \begin{block}{Remarque #1}
}{%
   \end{block}%
}


\newenvironment{exemple}[1][]{%
   \setbeamercolor{block title}{fg=black,bg=structure!20}
   \setbeamercolor{block body}{fg=black,bg=structure!5}
   \begin{block}{{\bf Exemple }#1}
}{%
   \end{block}%
}


\newenvironment{miniexercice}[0]{%
   \setbeamercolor{block title}{fg=structure,bg=structure!20}
   \setbeamercolor{block body}{fg=black,bg=structure!5}
   \begin{block}{Mini-exercices}
}{%
   \end{block}%
}


\newenvironment{tp}[0]{%
   \setbeamercolor{block title}{fg=structure,bg=structure!40}
   \setbeamercolor{block body}{fg=black,bg=structure!10}
   \begin{block}{\bf Travaux pratiques}
}{%
   \end{block}%
}
\newenvironment{exercicecours}[1][]{%
   \setbeamercolor{block title}{fg=structure,bg=structure!40}
   \setbeamercolor{block body}{fg=black,bg=structure!10}
   \begin{block}{{\bf Exercice }#1}
}{%
   \end{block}%
}
\newenvironment{algo}[1][]{%
   \setbeamercolor{block title}{fg=structure,bg=structure!40}
   \setbeamercolor{block body}{fg=black,bg=structure!10}
   \begin{block}{{\bf Algorithme}\hfill{\color{gray}\texttt{#1}}}
}{%
   \end{block}%
}


\setbeamertemplate{proof begin}{
   \setbeamercolor{block title}{fg=black,bg=structure!20}
   \setbeamercolor{block body}{fg=black,bg=structure!5}
   \begin{block}{{\footnotesize Démonstration}}
   \footnotesize
   \smallskip}
\setbeamertemplate{proof end}{%
   \end{block}}
\setbeamertemplate{qed symbol}{\openbox}


\makeatother
\usecolortheme[RGB={56,98,238}]{structure}
   
%%%%%%%%%%%%%%%%%%%%%%%%%%%%%%%%%%%%%%%%%%%%%%%%%%%%%%%%%%%%%
%%%%%%%%%%%%%%%%%%%%%%%%%%%%%%%%%%%%%%%%%%%%%%%%%%%%%%%%%%%%%


\begin{document}


\title{{\bf Matrices et applications linéaires}}
\subtitle{Rang d'une famille de vecteurs}

\begin{frame}
  
  \debutmontitre

  \pause

{\footnotesize
\hfill
\setbeamercovered{transparent=50}
\begin{minipage}{0.6\textwidth}
  \begin{itemize}
    \item<3-> Définition
    \item<4-> Rang d'une matrice
    \item<5-> Opérations conservant le rang
    \item<6-> Rang et matrice inversible
    \item<7-> Rang engendré par les vecteurs lignes
  \end{itemize}
\end{minipage}
}

\end{frame}

\setcounter{framenumber}{0}


%%%%%%%%%%%%%%%%%%%%%%%%%%%%%%%%%%%%%%%%%%%%%%%%%%%%%%%%%%%%%%%%
\section{Définition}

\begin{frame}

\begin{itemize}
  \item $E$ un $\Kk$-espace vectoriel
  \pause
  \item $\{v_1, \ldots ,v_p\}$ des vecteurs de $E$
  \pause
  \item $\Vect(v_1, \ldots ,v_p)$ est de dimension finie
\end{itemize}

\pause

\begin{mydefinition}
Le \defi{rang} de la famille $\{v_1, \ldots ,v_p\}$ 
est la dimension du sous-espace vectoriel engendré par les vecteurs $v_1, \dots ,v_p$

\pause

\mybox{$\rg(v_1, \dots ,v_p) = \dim \Vect(v_1, \ldots ,v_p)$}
\end{mydefinition}

\pause

\begin{proposition}
\begin{enumerate}
  \item $0 \le \rg (v_1, \ldots ,v_p) \le p$
\pause  
  \item $\rg (v_1, \ldots ,v_p) \le \dim E$
\end{enumerate}
\end{proposition}

\end{frame}


\begin{frame}
\begin{exemple}
Quel est le rang de $\{v_1,v_2,v_3\}$ dans l'espace vectoriel $\Rr^4$ ?
$$
v_1 = \left(\begin{smallmatrix} 1\\0\\1\\0 \end{smallmatrix}\right) \qquad
v_2 = \left(\begin{smallmatrix} 0\\1\\1\\1 \end{smallmatrix}\right) \qquad
v_3 = \left(\begin{smallmatrix} -1\\1\\0\\1 \end{smallmatrix}\right)$$
\vspace*{-2ex}
\pause
\begin{itemize}[<+->]
  \item Dans $\Rr^4$ donc $\rg(v_1,v_2,v_3) \le 4$
  \item $3$ vecteurs donc $\rg(v_1,v_2,v_3) \le 3$
  \item $v_1 \neq 0$ donc $\rg(v_1,v_2,v_3) \ge 1$
  \item  $\rg(v_1,v_2,v_3) \ge \rg(v_1,v_2)=2$ car $v_1$ et $v_2$ pas colinéaires
%  \item Le rang vaut $2$ ou $3$
  \item La famille $\{v_1,v_2,v_3\}$ est libre ou liée ?
  \item Résoudre $\lambda_1 v_1 + \lambda_2 v_2 + \lambda_3 v_3 = 0$
  \item On trouve $v_1-v_2+v_3=0$, la famille est donc liée
  \item Ainsi $\Vect(v_1,v_2,v_3)= \Vect(v_1,v_2)$
  \item $\rg (v_1,v_2,v_3) = \dim \Vect(v_1,v_2,v_3) = 2$
\end{itemize}

\end{exemple}
\end{frame}


%%%%%%%%%%%%%%%%%%%%%%%%%%%%%%%%%%%%%%%%%%%%%%%%%%%%%%%%%%%%%%%%
\section{Rang d'une matrice}

\begin{frame}

\begin{mydefinition}
Le rang d'une matrice est le rang de ses vecteurs colonnes
\end{mydefinition}
\pause
\begin{exemple}
Calcul du rang de la matrice 
$$A = \begin{pmatrix}
  1 & 2 & -\frac12 & 0 \\
  2 & 4 & -1       & 0
  \end{pmatrix} \in M_{2,4}(\Rr)$$
\pause  
\begin{itemize}
  \item C'est le rang des vecteurs de $\Rr^2$
\centerline{  
$\bigg\{ 
v_1 = \left(\begin{smallmatrix}1 \\ 2  \end{smallmatrix}\right),
v_2 = \left(\begin{smallmatrix}2 \\ 4  \end{smallmatrix}\right),
v_3 = \left(\begin{smallmatrix}-\frac 12 \\ -1  \end{smallmatrix}\right),
v_4 = \left(\begin{smallmatrix}0 \\ 0  \end{smallmatrix}\right)
\bigg\}$
}
\pause
  \item Tous ces vecteurs sont colinéaires à $v_1$
\pause
  \item Donc $\rg \{ v_1, v_2, v_3, v_4 \} = 1$
\pause 
  \item Ainsi $\rg A = 1$
\end{itemize}

\end{exemple}

\end{frame}


\begin{frame}
\begin{mydefinition}
Une matrice est échelonnée par rapport aux colonnes si
le nombre de zéros commençant une colonne croît strictement colonne après colonne,
jusqu'à ce qu'il ne reste plus que des zéros
\end{mydefinition}

$$
\begin{pmatrix}
+ & 0 & 0 & 0 & 0 & 0 \\
* & 0 & 0 & 0 & 0 & 0 \\
* & + & 0 & 0 & 0 & 0 \\
* & * & + & 0 & 0 & 0 \\
* & * & * & 0 & 0 & 0 \\
* & * & * & + & 0 & 0 \\
\end{pmatrix}
$$ 

\centerline{$*$ coefficients quelconques \qquad $+$ coefficients non nuls}

\begin{proposition}\label{prop:rangmatech}
\vspace*{-1.5ex}
Le rang d'une matrice échelonnée par colonnes est égal au nombre
de colonnes non nulles
\end{proposition} 

\end{frame}

%%%%%%%%%%%%%%%%%%%%%%%%%%%%%%%%%%%%%%%%%%%%%%%%%%%%%%%%%%%%%%%%
\section{Opérations conservant le rang}

\begin{frame}
\begin{proposition}
\label{prop:opcolonnes}
Le rang d'une matrice est conservé par les opérations élémentaires suivantes sur les 
colonnes $C_1, C_2, \ldots, C_p$ :
\pause
\begin{enumerate}
  \item $C_i \leftarrow \lambda C_i$ avec $\lambda \neq 0$ : 
  on multiplie une colonne par un scalaire non nul
\pause    
  \item $C_i \leftarrow C_i+\lambda C_j$ avec $\lambda \in \Kk$ (et $j\neq i$) :
  on ajoute à la colonne $C_i$ un multiple d'une autre colonne $C_j$
\pause  
  \item $C_i \leftrightarrow C_j$ : on échange deux colonnes
\end{enumerate}
\end{proposition}

\pause
Plus généralement, l'opération 
$C_i \leftarrow C_i + \lambda_1 C_1 + \lambda_2 C_2 + \cdots + \lambda_p C_p$ 
conserve le rang de la matrice




\end{frame}

\begin{frame}
\textbf{Méthodologie.} Comment calculer le rang d'une matrice ?

\pause
\begin{itemize}
  \item Méthode de Gauss sur les colonnes de la matrice $A$

\pause  
  \item Opérations élémentaires
  \begin{itemize}
    \item $C_i \leftarrow \lambda C_i$
    \item $C_i \leftarrow C_i+\lambda C_j$
    \item $C_i \leftrightarrow C_j$
  \end{itemize}

\pause  
  \item Transformer la matrice $A$ en une matrice échelonnée

\pause  
  \item Le rang de la matrice est le nombre de colonnes non nulles
\end{itemize}

\end{frame}


%%%%%%%%%%%%%%%%%%%%%%%%%%%%%%%%%%%%%%%%%%%%%%%%%%%%%%%%%%%%%%%%
\section{Exemples}

\begin{frame}
\begin{exemple}
Quel est le rang de la famille des 5 vecteurs suivants de $\Rr^4$ ?
$$
v_1=\begin{pmatrix}1 \\ 1 \\ 1 \\ 1\end{pmatrix} \quad
v_2=\begin{pmatrix}-1 \\ 2 \\ 0 \\ 1\end{pmatrix} \quad
v_3=\begin{pmatrix}3 \\ 2 \\ -1 \\ -3\end{pmatrix} \quad
v_4=\begin{pmatrix}3 \\ 5 \\ 0 \\ -1\end{pmatrix} \quad
v_5=\begin{pmatrix}3 \\ 8 \\ 1 \\ 1\end{pmatrix}$$

\pause

$$\begin{pmatrix}
1&-1&3&3&3\cr
1&2&2&5&8\cr
1&0&-1&0&1\cr
1&1&-3&-1&1 \cr
\end{pmatrix}
\uncover<4->{
\ \sim\ 
\begin{pmatrix}
1&0&0&0&0\cr
1&3&-1&2&5\cr
1&1&-4&-3&-2\cr
1&2&-6&-4&-2 \cr
\end{pmatrix}
}$$
\pause
$C_2 \leftarrow C_2+C_1$, 
$C_3\leftarrow C_3-3C_1$ , $C_4\leftarrow
C_4-3C_1$, 
$C_5\leftarrow C_5-3C_1$
\end{exemple}
\end{frame}

\begin{frame}
\begin{exemple}
\uncover<2->{$C_2 \leftrightarrow C_3$}
\vspace*{-1ex}
$$\begin{pmatrix}
1&0&0&0&0\cr
1&3&-1&2&5\cr
1&1&-4&-3&-2\cr
1&2&-6&-4&-2 \cr
\end{pmatrix}
\pause\pause
\ \sim\ 
\begin{pmatrix}
1&0&0&0&0\cr
1&-1&3&2&5\cr
1&-4&1&-3&-2\cr
1&-6&2&-4&-2 \cr
\end{pmatrix}$$
\pause
\vspace*{-1ex}
$C_3 \leftarrow C_3+3C_2$, 
$C_4\leftarrow C_4+2C_2$ et 
$C_5\leftarrow C_5+5C_2$
\vspace*{-1ex}
$$ \sim\ 
\begin{pmatrix}
1&0&0&0&0\cr
1&-1&0&0&0\cr
1&-4&-11&-11&-22\cr
1&-6&-16&-16&-32 \cr
\end{pmatrix}
\pause
\sim\ 
\begin{pmatrix}
1&0&0&0&0\cr
1&-1&0&0&0\cr
1&-4&-11&0&0\cr
1&-6&-16&0&0 \cr
\end{pmatrix}
$$
\vspace*{-1ex}
$C_4\leftarrow C_4-C_3$
et $C_5\leftarrow C_5-2C_3$

\pause

\begin{itemize}
  \item $\rg \{v_1,v_2,v_3,v_4,v_5\}=3$
  \vspace*{-1ex}
  \pause
  \item $\Vect (v_1,v_2,v_3,v_4,v_5)=
\Vect \left( \left(\begin{smallmatrix}1\\1\\1\\1\end{smallmatrix}\right) ,\left(\begin{smallmatrix}0\\-1\\-4\\-6\end{smallmatrix}\right), 
\left(\begin{smallmatrix}0\\0\\-11\\-16\end{smallmatrix}\right)\right )$
\end{itemize}

\end{exemple}
\end{frame}



%%%%%%%%%%%%%%%%%%%%%%%%%%%%%%%%%%%%%%%%%%%%%%%%%%%%%%%%%%%%%%%%
\section{Rang et matrice inversible}

\begin{frame}

\begin{theoreme}
Une matrice carrée de taille $n$ est inversible si et seulement si elle
est de rang $n$
\end{theoreme}

\end{frame}

%%%%%%%%%%%%%%%%%%%%%%%%%%%%%%%%%%%%%%%%%%%%%%%%%%%%%%%%%%%%%%%%
\section{Rang engendré par les vecteurs lignes}

\begin{frame}

Une matrice  $A$ est une superposition de vecteurs lignes $(w_1,\ldots,w_n)$

\pause

\begin{proposition}
$\rg A = \dim \Vect (w_1,\ldots,w_n)$
\end{proposition}

\pause

\begin{itemize}
  \item l'espace vectoriel engendré par les vecteurs colonnes
et l'espace vectoriel engendré par les vecteurs lignes sont de même dimension
 
\pause

  \item \myboxinline{$\rg A = \rg A^T$}
  
\pause
  
  \item le rang d'une matrice est égal au rang de sa transposée
\end{itemize}

\end{frame}


%%%%%%%%%%%%%%%%%%%%%%%%%%%%%%%%%%%%%%%%%%%%%%%%%%%%%%%%%%%%%%%%
\section{Mini-exercices}

\begin{frame}

\begin{miniexercice}
\begin{enumerate}
 
  \item Quel est le rang de la famille de vecteurs
  $\left( 
  \left(\begin{smallmatrix} 1\\2\\1 \end{smallmatrix}\right),
  \left(\begin{smallmatrix} 3\\4\\2 \end{smallmatrix}\right),
  \left(\begin{smallmatrix} 0\\-2\\-1 \end{smallmatrix}\right),
  \left(\begin{smallmatrix} 2\\2\\1 \end{smallmatrix}\right)
  \right)$ ?
  
  Même question pour
  $\left( 
  \left(\begin{smallmatrix} 1\\t\\1 \end{smallmatrix}\right),
  \left(\begin{smallmatrix} t\\1\\t \end{smallmatrix}\right),
  \left(\begin{smallmatrix} 1\\1\\t \end{smallmatrix}\right)
  \right)$ 
  en fonction du paramètre $t\in \Rr$.
  
  \item Mettre sous forme échelonnée par rapport aux colonnes la matrice 
  $\left(\begin{smallmatrix} 1 &2  &-4 &-2 &-1 \cr
                0 &-2 &4  &2  &0  \cr
                1 &1  &-2 &-1 &1  \cr \end{smallmatrix}\right)$. Calculer son rang.
                Idem avec             
                $\left(\begin{smallmatrix} 1  &7  &2  &5  \cr
                -2 &1  &1  &5  \cr
                -1 &2  &1  &4  \cr
                1  &4  &1  &2  \cr \end{smallmatrix}\right)$. 
                
  \item Calculer le rang de $\left(\begin{smallmatrix} 2&\phantom-4&-5&-7 \cr -1&3&1&2 \cr 1&a&-2&b \end{smallmatrix}\right)$ en fonction de
$a$ et $b$.

  \item Calculer les rangs précédents en utilisant les vecteurs lignes.
  
  \item Soit $f:E \to F$ une application linéaire.
Quelle inégalité relie $\rg (f(v_1), \ldots ,f(v_p))$
et $\rg (v_1, \ldots , v_p)$ ? Que se passe-t-il si $f$ est injective ?
   
\end{enumerate}
\end{miniexercice}

\end{frame}

\end{document}
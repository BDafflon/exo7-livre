
%%%%%%%%%%%%%%%%%% PREAMBULE %%%%%%%%%%%%%%%%%%

\documentclass[aspectratio=169,utf8]{beamer}
%\documentclass[aspectratio=169,handout]{beamer}

\usetheme{Boadilla}
%\usecolortheme{seahorse}
%\usecolortheme[RGB={245,66,24}]{structure}
\useoutertheme{infolines}

% packages
\usepackage{amsfonts,amsmath,amssymb,amsthm}
\usepackage[utf8]{inputenc}
\usepackage[T1]{fontenc}
\usepackage{lmodern}

\usepackage[francais]{babel}
\usepackage{fancybox}
\usepackage{graphicx}

\usepackage{float}
\usepackage{xfrac}

%\usepackage[usenames, x11names]{xcolor}
\usepackage{pgfplots}
\usepackage{datetime}


% ----------------------------------------------------------------------
% Pour les images
\usepackage{tikz}
\usetikzlibrary{calc,shadows,arrows.meta,patterns,matrix}

\newcommand{\tikzinput}[1]{\input{figures/#1.tikz}}
% --- les figures avec échelle éventuel
\newcommand{\myfigure}[2]{% entrée : échelle, fichier(s) figure à inclure
\begin{center}\small%
\tikzstyle{every picture}=[scale=1.0*#1]% mise en échelle + 0% (automatiquement annulé à la fin du groupe)
#2%
\end{center}}



%-----  Package unités -----
\usepackage{siunitx}
\sisetup{locale = FR,detect-all,per-mode = symbol}

%\usepackage{mathptmx}
%\usepackage{fouriernc}
%\usepackage{newcent}
%\usepackage[mathcal,mathbf]{euler}

%\usepackage{palatino}
%\usepackage{newcent}
% \usepackage[mathcal,mathbf]{euler}



% \usepackage{hyperref}
% \hypersetup{colorlinks=true, linkcolor=blue, urlcolor=blue,
% pdftitle={Exo7 - Exercices de mathématiques}, pdfauthor={Exo7}}


%section
% \usepackage{sectsty}
% \allsectionsfont{\bf}
%\sectionfont{\color{Tomato3}\upshape\selectfont}
%\subsectionfont{\color{Tomato4}\upshape\selectfont}

%----- Ensembles : entiers, reels, complexes -----
\newcommand{\Nn}{\mathbb{N}} \newcommand{\N}{\mathbb{N}}
\newcommand{\Zz}{\mathbb{Z}} \newcommand{\Z}{\mathbb{Z}}
\newcommand{\Qq}{\mathbb{Q}} \newcommand{\Q}{\mathbb{Q}}
\newcommand{\Rr}{\mathbb{R}} \newcommand{\R}{\mathbb{R}}
\newcommand{\Cc}{\mathbb{C}} 
\newcommand{\Kk}{\mathbb{K}} \newcommand{\K}{\mathbb{K}}

%----- Modifications de symboles -----
\renewcommand{\epsilon}{\varepsilon}
\renewcommand{\Re}{\mathop{\text{Re}}\nolimits}
\renewcommand{\Im}{\mathop{\text{Im}}\nolimits}
%\newcommand{\llbracket}{\left[\kern-0.15em\left[}
%\newcommand{\rrbracket}{\right]\kern-0.15em\right]}

\renewcommand{\ge}{\geqslant}
\renewcommand{\geq}{\geqslant}
\renewcommand{\le}{\leqslant}
\renewcommand{\leq}{\leqslant}
\renewcommand{\epsilon}{\varepsilon}

%----- Fonctions usuelles -----
\newcommand{\ch}{\mathop{\text{ch}}\nolimits}
\newcommand{\sh}{\mathop{\text{sh}}\nolimits}
\renewcommand{\tanh}{\mathop{\text{th}}\nolimits}
\newcommand{\cotan}{\mathop{\text{cotan}}\nolimits}
\newcommand{\Arcsin}{\mathop{\text{arcsin}}\nolimits}
\newcommand{\Arccos}{\mathop{\text{arccos}}\nolimits}
\newcommand{\Arctan}{\mathop{\text{arctan}}\nolimits}
\newcommand{\Argsh}{\mathop{\text{argsh}}\nolimits}
\newcommand{\Argch}{\mathop{\text{argch}}\nolimits}
\newcommand{\Argth}{\mathop{\text{argth}}\nolimits}
\newcommand{\pgcd}{\mathop{\text{pgcd}}\nolimits} 


%----- Commandes divers ------
\newcommand{\ii}{\mathrm{i}}
\newcommand{\dd}{\text{d}}
\newcommand{\id}{\mathop{\text{id}}\nolimits}
\newcommand{\Ker}{\mathop{\text{Ker}}\nolimits}
\newcommand{\Card}{\mathop{\text{Card}}\nolimits}
\newcommand{\Vect}{\mathop{\text{Vect}}\nolimits}
\newcommand{\Mat}{\mathop{\text{Mat}}\nolimits}
\newcommand{\rg}{\mathop{\text{rg}}\nolimits}
\newcommand{\tr}{\mathop{\text{tr}}\nolimits}


%----- Structure des exercices ------

\newtheoremstyle{styleexo}% name
{2ex}% Space above
{3ex}% Space below
{}% Body font
{}% Indent amount 1
{\bfseries} % Theorem head font
{}% Punctuation after theorem head
{\newline}% Space after theorem head 2
{}% Theorem head spec (can be left empty, meaning ‘normal’)

%\theoremstyle{styleexo}
\newtheorem{exo}{Exercice}
\newtheorem{ind}{Indications}
\newtheorem{cor}{Correction}


\newcommand{\exercice}[1]{} \newcommand{\finexercice}{}
%\newcommand{\exercice}[1]{{\tiny\texttt{#1}}\vspace{-2ex}} % pour afficher le numero absolu, l'auteur...
\newcommand{\enonce}{\begin{exo}} \newcommand{\finenonce}{\end{exo}}
\newcommand{\indication}{\begin{ind}} \newcommand{\finindication}{\end{ind}}
\newcommand{\correction}{\begin{cor}} \newcommand{\fincorrection}{\end{cor}}

\newcommand{\noindication}{\stepcounter{ind}}
\newcommand{\nocorrection}{\stepcounter{cor}}

\newcommand{\fiche}[1]{} \newcommand{\finfiche}{}
\newcommand{\titre}[1]{\centerline{\large \bf #1}}
\newcommand{\addcommand}[1]{}
\newcommand{\video}[1]{}

% Marge
\newcommand{\mymargin}[1]{\marginpar{{\small #1}}}

\def\noqed{\renewcommand{\qedsymbol}{}}


%----- Presentation ------
\setlength{\parindent}{0cm}

%\newcommand{\ExoSept}{\href{http://exo7.emath.fr}{\textbf{\textsf{Exo7}}}}

\definecolor{myred}{rgb}{0.93,0.26,0}
\definecolor{myorange}{rgb}{0.97,0.58,0}
\definecolor{myyellow}{rgb}{1,0.86,0}

\newcommand{\LogoExoSept}[1]{  % input : echelle
{\usefont{U}{cmss}{bx}{n}
\begin{tikzpicture}[scale=0.1*#1,transform shape]
  \fill[color=myorange] (0,0)--(4,0)--(4,-4)--(0,-4)--cycle;
  \fill[color=myred] (0,0)--(0,3)--(-3,3)--(-3,0)--cycle;
  \fill[color=myyellow] (4,0)--(7,4)--(3,7)--(0,3)--cycle;
  \node[scale=5] at (3.5,3.5) {Exo7};
\end{tikzpicture}}
}


\newcommand{\debutmontitre}{
  \author{} \date{} 
  \thispagestyle{empty}
  \hspace*{-10ex}
  \begin{minipage}{\textwidth}
    \titlepage  
  \vspace*{-2.5cm}
  \begin{center}
    \LogoExoSept{2.5}
  \end{center}
  \end{minipage}

  \vspace*{-0cm}
  
  % Astuce pour que le background ne soit pas discrétisé lors de la conversion pdf -> png
\begin{tikzpicture}
        \fill[opacity=0,green!60!black] (0,0)--++(0,0)--++(0,0)--++(0,0)--cycle; 
\end{tikzpicture}

% toc S'affiche trop tot :
% \tableofcontents[hideallsubsections, pausesections]
}

\newcommand{\finmontitre}{
  \end{frame}
  \setcounter{framenumber}{0}
} % ne marche pas pour une raison obscure

%----- Commandes supplementaires ------

% \usepackage[landscape]{geometry}
% \geometry{top=1cm, bottom=3cm, left=2cm, right=10cm, marginparsep=1cm
% }
% \usepackage[a4paper]{geometry}
% \geometry{top=2cm, bottom=2cm, left=2cm, right=2cm, marginparsep=1cm
% }

%\usepackage{standalone}


% New command Arnaud -- november 2011
\setbeamersize{text margin left=24ex}
% si vous modifier cette valeur il faut aussi
% modifier le decalage du titre pour compenser
% (ex : ici =+10ex, titre =-5ex

\theoremstyle{definition}
%\newtheorem{proposition}{Proposition}
%\newtheorem{exemple}{Exemple}
%\newtheorem{theoreme}{Théorème}
%\newtheorem{lemme}{Lemme}
%\newtheorem{corollaire}{Corollaire}
%\newtheorem*{remarque*}{Remarque}
%\newtheorem*{miniexercice}{Mini-exercices}
%\newtheorem{definition}{Définition}

% Commande tikz
\usetikzlibrary{calc}
\usetikzlibrary{patterns,arrows}
\usetikzlibrary{matrix}
\usetikzlibrary{fadings} 

%definition d'un terme
\newcommand{\defi}[1]{{\color{myorange}\textbf{\emph{#1}}}}
\newcommand{\evidence}[1]{{\color{blue}\textbf{\emph{#1}}}}
\newcommand{\assertion}[1]{\emph{\og#1\fg}}  % pour chapitre logique
%\renewcommand{\contentsname}{Sommaire}
\renewcommand{\contentsname}{}
\setcounter{tocdepth}{2}



%------ Encadrement ------

\usepackage{fancybox}


\newcommand{\mybox}[1]{
\setlength{\fboxsep}{7pt}
\begin{center}
\shadowbox{#1}
\end{center}}

\newcommand{\myboxinline}[1]{
\setlength{\fboxsep}{5pt}
\raisebox{-10pt}{
\shadowbox{#1}
}
}

%--------------- Commande beamer---------------
\newcommand{\beameronly}[1]{#1} % permet de mettre des pause dans beamer pas dans poly


\setbeamertemplate{navigation symbols}{}
\setbeamertemplate{footline}  % tiré du fichier beamerouterinfolines.sty
{
  \leavevmode%
  \hbox{%
  \begin{beamercolorbox}[wd=.333333\paperwidth,ht=2.25ex,dp=1ex,center]{author in head/foot}%
    % \usebeamerfont{author in head/foot}\insertshortauthor%~~(\insertshortinstitute)
    \usebeamerfont{section in head/foot}{\bf\insertshorttitle}
  \end{beamercolorbox}%
  \begin{beamercolorbox}[wd=.333333\paperwidth,ht=2.25ex,dp=1ex,center]{title in head/foot}%
    \usebeamerfont{section in head/foot}{\bf\insertsectionhead}
  \end{beamercolorbox}%
  \begin{beamercolorbox}[wd=.333333\paperwidth,ht=2.25ex,dp=1ex,right]{date in head/foot}%
    % \usebeamerfont{date in head/foot}\insertshortdate{}\hspace*{2em}
    \insertframenumber{} / \inserttotalframenumber\hspace*{2ex} 
  \end{beamercolorbox}}%
  \vskip0pt%
}


\definecolor{mygrey}{rgb}{0.5,0.5,0.5}
\setlength{\parindent}{0cm}
%\DeclareTextFontCommand{\helvetica}{\fontfamily{phv}\selectfont}

% background beamer
\definecolor{couleurhaut}{rgb}{0.85,0.9,1}  % creme
\definecolor{couleurmilieu}{rgb}{1,1,1}  % vert pale
\definecolor{couleurbas}{rgb}{0.85,0.9,1}  % blanc
\setbeamertemplate{background canvas}[vertical shading]%
[top=couleurhaut,middle=couleurmilieu,midpoint=0.4,bottom=couleurbas] 
%[top=fondtitre!05,bottom=fondtitre!60]



\makeatletter
\setbeamertemplate{theorem begin}
{%
  \begin{\inserttheoremblockenv}
  {%
    \inserttheoremheadfont
    \inserttheoremname
    \inserttheoremnumber
    \ifx\inserttheoremaddition\@empty\else\ (\inserttheoremaddition)\fi%
    \inserttheorempunctuation
  }%
}
\setbeamertemplate{theorem end}{\end{\inserttheoremblockenv}}

\newenvironment{theoreme}[1][]{%
   \setbeamercolor{block title}{fg=structure,bg=structure!40}
   \setbeamercolor{block body}{fg=black,bg=structure!10}
   \begin{block}{{\bf Th\'eor\`eme }#1}
}{%
   \end{block}%
}


\newenvironment{proposition}[1][]{%
   \setbeamercolor{block title}{fg=structure,bg=structure!40}
   \setbeamercolor{block body}{fg=black,bg=structure!10}
   \begin{block}{{\bf Proposition }#1}
}{%
   \end{block}%
}

\newenvironment{corollaire}[1][]{%
   \setbeamercolor{block title}{fg=structure,bg=structure!40}
   \setbeamercolor{block body}{fg=black,bg=structure!10}
   \begin{block}{{\bf Corollaire }#1}
}{%
   \end{block}%
}

\newenvironment{mydefinition}[1][]{%
   \setbeamercolor{block title}{fg=structure,bg=structure!40}
   \setbeamercolor{block body}{fg=black,bg=structure!10}
   \begin{block}{{\bf Définition} #1}
}{%
   \end{block}%
}

\newenvironment{lemme}[0]{%
   \setbeamercolor{block title}{fg=structure,bg=structure!40}
   \setbeamercolor{block body}{fg=black,bg=structure!10}
   \begin{block}{\bf Lemme}
}{%
   \end{block}%
}

\newenvironment{remarque}[1][]{%
   \setbeamercolor{block title}{fg=black,bg=structure!20}
   \setbeamercolor{block body}{fg=black,bg=structure!5}
   \begin{block}{Remarque #1}
}{%
   \end{block}%
}


\newenvironment{exemple}[1][]{%
   \setbeamercolor{block title}{fg=black,bg=structure!20}
   \setbeamercolor{block body}{fg=black,bg=structure!5}
   \begin{block}{{\bf Exemple }#1}
}{%
   \end{block}%
}


\newenvironment{miniexercice}[0]{%
   \setbeamercolor{block title}{fg=structure,bg=structure!20}
   \setbeamercolor{block body}{fg=black,bg=structure!5}
   \begin{block}{Mini-exercices}
}{%
   \end{block}%
}


\newenvironment{tp}[0]{%
   \setbeamercolor{block title}{fg=structure,bg=structure!40}
   \setbeamercolor{block body}{fg=black,bg=structure!10}
   \begin{block}{\bf Travaux pratiques}
}{%
   \end{block}%
}
\newenvironment{exercicecours}[1][]{%
   \setbeamercolor{block title}{fg=structure,bg=structure!40}
   \setbeamercolor{block body}{fg=black,bg=structure!10}
   \begin{block}{{\bf Exercice }#1}
}{%
   \end{block}%
}
\newenvironment{algo}[1][]{%
   \setbeamercolor{block title}{fg=structure,bg=structure!40}
   \setbeamercolor{block body}{fg=black,bg=structure!10}
   \begin{block}{{\bf Algorithme}\hfill{\color{gray}\texttt{#1}}}
}{%
   \end{block}%
}


\setbeamertemplate{proof begin}{
   \setbeamercolor{block title}{fg=black,bg=structure!20}
   \setbeamercolor{block body}{fg=black,bg=structure!5}
   \begin{block}{{\footnotesize Démonstration}}
   \footnotesize
   \smallskip}
\setbeamertemplate{proof end}{%
   \end{block}}
\setbeamertemplate{qed symbol}{\openbox}


\makeatother
\usecolortheme[RGB={56,98,238}]{structure}
 
% Commande spécifique à ce chapitre
\newcommand{\Pass}{\mathop{\text{P}}\nolimits}
   
%%%%%%%%%%%%%%%%%%%%%%%%%%%%%%%%%%%%%%%%%%%%%%%%%%%%%%%%%%%%%
%%%%%%%%%%%%%%%%%%%%%%%%%%%%%%%%%%%%%%%%%%%%%%%%%%%%%%%%%%%%%


\begin{document}


\title{{\bf Matrices et applications linéaires}}
\subtitle{Changement de bases}

\begin{frame}
  
  \debutmontitre

  \pause

{\footnotesize
\hfill
\setbeamercovered{transparent=50}
\begin{minipage}{0.6\textwidth}
  \begin{itemize}
    \item<3-> Application linéaire, matrice, vecteur
    \item<4-> Matrice de passage d'une base à une autre
    \item<5-> Formule de changement de base
    \item<6-> Matrices semblables
  \end{itemize}
\end{minipage}
}

\end{frame}

\setcounter{framenumber}{0}


%%%%%%%%%%%%%%%%%%%%%%%%%%%%%%%%%%%%%%%%%%%%%%%%%%%%%%%%%%%%%%%%
\section{Application linéaire, matrice, vecteur}

\begin{frame}

\begin{itemize}
  \item Soit $\mathcal{B} = (e_1,e_2, \dots ,e_p )$ une base d'un espace vectoriel $E$
  \pause
  \item Pour  $x \in E$, $x=x_1e_1+x_2e_2+\dots +x_p e_p$
\pause
  \item La matrice de $x$ dans $\mathcal{B}$ est $\Mat_\mathcal{B} (x) = 
  \left(\begin{smallmatrix}x_1\cr x_2\cr \vdots \cr x_p\end{smallmatrix}\right)_\mathcal{B}$
\end{itemize}

\pause
\vspace*{-1ex}
\begin{proposition}
\label{prop:matetapplin}
$f : E \to F$ \pause\quad  $y=f (x)$ \quad \pause$\mathcal{B}$ base de $E$ \quad\pause $\mathcal{B}'$ base de $F$
\vspace*{-2ex}
\pause
$$A = \Mat_{\mathcal{B},\mathcal{B}'} (f) 
\quad 
\pause
X = \Mat_{\mathcal{B}} (x) = 
\left(\begin{smallmatrix}x_1\cr x_2\cr \vdots \cr x_p\end{smallmatrix}\right)_\mathcal{B}
\quad 
\pause
Y = \Mat_{\mathcal{B}'} (y) = 
  \left(\begin{smallmatrix}y_1\cr y_2\cr \vdots \cr y_n\end{smallmatrix}\right)_{\mathcal{B}'}
$$
\vspace*{-2ex}
\pause
Alors, si $y = f(x)$, on a 
\vspace*{-1ex}
\mybox{$Y = AX$}
\vspace*{-3ex}
\pause
\mybox{$\Mat_{\mathcal{B}'} \big( f(x) \big) 
= \Mat_{\mathcal{B},\mathcal{B}'} (f) \times \Mat_{\mathcal{B}} (x)$}
\end{proposition}
\end{frame}


\begin{frame}
\begin{exemple}
\begin{itemize}
  \item $E$ un espace vectoriel de dimension $3$ de base $\mathcal{B}=(e_1,e_2,e_3)$
  \pause
  \item Soit $f : E \to E$ avec $A = \Mat_{\mathcal{B}} (f) =
\begin{pmatrix}
1&2&1\cr
2&3&1\cr
1&1&0\cr
\end{pmatrix}$
\end{itemize}
\pause\evidence{Quel est le noyau de $f$ ?}
\vspace*{-2ex}
\small
\pause
$$
x  \in \Ker f  \pause\iff 
f(x)= 0_E \pause\iff  \Mat_{\mathcal{B}} \big(f (x) \big) =
\begin{pmatrix}
0\cr
0\cr
0\end{pmatrix} 
\pause\iff 
A X =
\begin{pmatrix}
0\cr
0\cr
0\end{pmatrix}
$$
$$
\pause\iff
A
\begin{pmatrix}
x_1\cr
x_2\cr
x_3\end{pmatrix} =
 \begin{pmatrix}
0\cr
0\cr
0\end{pmatrix}
\pause\iff  \left \{
\begin{matrix}
x_1& +&2x_2&+&x_3&=&0\cr
2x_1& +&3x_2&+&x_3&=&0\cr
x_1&+&x_2&&&=&0\end{matrix}\right .
$$
\pause
$$\Ker f\pause\!=\!\left\{x \in E \mid 
\begin{array}{c}
x_1 + 2x_2+x_3 = 0 \\
\text{ et } x_2+x_3=0  
\end{array}
\!\right\}
\pause\!\!=\!\left\{ \left(\begin{smallmatrix}t\\-t\\t\end{smallmatrix}\right)_{\mathcal{B}} \mid t \in \Kk \right\} 
\pause= \Vect\left( \left(\begin{smallmatrix}1\\-1\\1\end{smallmatrix}\right)_{\mathcal{B}}\right)$$

\end{exemple}
\end{frame}

\begin{frame}

\begin{exemple}

\begin{itemize}
  \item Soit $E$ un espace vectoriel de dimension $3$ ayant une base $\mathcal{B}=(e_1,e_2,e_3)$
  \item Soit $f : E \to E$ avec $A = \Mat_{\mathcal{B}} (f) =
\begin{pmatrix}
1&2&1\cr
2&3&1\cr
1&1&0\cr
\end{pmatrix}$
\end{itemize}

\evidence{Quelle est l'image de $f$ ?}

\pause

\begin{itemize}
  \item $\Ker f$ est de dimension $1$
  \pause
  \item Théorème du rang : $\dim \Im f = \dim E - \dim \Ker f = 2$
  \pause
  \item Les deux premiers vecteurs de la matrice $A$ étant linéairement indépendants, ils engendrent
$\Im f$
\end{itemize}

\pause

$$\Im f = \Vect \left(
\left(\begin{smallmatrix}1\\2\\1\end{smallmatrix}\right)_{\mathcal{B}},
\left(\begin{smallmatrix}2\\3\\1\end{smallmatrix}\right)_{\mathcal{B}}
\right)$$
\end{exemple}
\end{frame}


%%%%%%%%%%%%%%%%%%%%%%%%%%%%%%%%%%%%%%%%%%%%%%%%%%%%%%%%%%%%%%%%
\section{Matrice de passage d'une base à une autre}

\begin{frame}

\begin{itemize}
  \item Soit $E$ un espace vectoriel de dimension $n$
  
\pause

  \item Soient $\mathcal{B}$ et $\mathcal{B}'$ deux bases de $E$
\end{itemize}

\pause

\begin{mydefinition}
La \defi{matrice de passage}  de la base $\mathcal{B}$ vers la base 
$\mathcal{B}'$, notée $\Pass_{\mathcal{B},\mathcal{B'}}$,
est la matrice dont la $j$-ème colonne est formée des coordonnées du $j$-ème 
vecteur de la base $\mathcal{B}'$, 
par rapport à la base $\mathcal{B}$
\end{mydefinition}

\pause

\mybox{
\begin{minipage}{0.8\textwidth}
\center
La matrice de passage $\Pass_{\mathcal{B},\mathcal{B'}}$ est constituée -\,en colonne\;- des
coordonnées des vecteurs de la nouvelle base $\mathcal{B}'$
exprimés dans l’ancienne base $\mathcal{B}$ 
\end{minipage}
}

\end{frame}


\begin{frame}
\begin{exemple}
Dans $\Rr^2$ on considère :
\vspace*{-1ex}
$$
e_1 = \begin{pmatrix}1\\0\end{pmatrix} \qquad
e_2 = \begin{pmatrix}1\\1\end{pmatrix} \qquad\qquad
\pause
\epsilon_1 = \begin{pmatrix}1\\2\end{pmatrix} \qquad
\epsilon_2 = \begin{pmatrix}5\\4\end{pmatrix}$$
\vspace*{-2ex}
\pause
\begin{itemize}
  \item $\mathcal{B} = (e_1,e_2)$ est une base
  \pause
  \item $\mathcal{B}'=(\epsilon_1,\epsilon_2)$ est une base
\end{itemize}

\pause

Quelle est la matrice de passage de la base $\mathcal{B}$ vers 
la base $\mathcal{B}'$ ?

\pause
\evidence{Méthode} Il faut exprimer $\epsilon_1$ et $\epsilon_2$ en fonction de $(e_1,e_2)$

\pause

$$\epsilon_1 = -e_1+2e_2 \pause =  \begin{pmatrix}-1\\2\end{pmatrix}_{\mathcal{B}}
\qquad
\pause
\epsilon_2 = e_1+4e_2 \pause =  \begin{pmatrix}1\\4\end{pmatrix}_{\mathcal{B}}$$
\pause
La matrice de passage est donc :
\vspace*{-2ex}
$$\Pass_{\mathcal{B},\mathcal{B'}} = 
\begin{pmatrix}-1&1\\2&4\end{pmatrix}$$
\end{exemple}
\end{frame}


\begin{frame}
\begin{proposition}
La matrice de passage $\Pass_{\mathcal{B},\mathcal{B}'}$
est la matrice associée à l'identité $\id_E :  (E, \mathcal{B}') \to (E,\mathcal{B})$

\mybox{$\Pass_{\mathcal{B},\mathcal{B'}} = \Mat_{\mathcal{B}',\mathcal{B}} (\id_E) $}
\end{proposition}

\pause

\begin{proposition}
\label{prop:chgtbase}
\begin{enumerate}
  \item Une matrice de passage est inversible et
  \myboxinline{ $\Pass_{\mathcal{B}',\mathcal{B}} = \big( \Pass_{\mathcal{B},\mathcal{B}'} \big)^{-1}$}
  
 \bigskip
 \pause
 
  \item Si $\mathcal{B}$, $\mathcal{B}'$ et $\mathcal{B}''$ sont trois bases
  \myboxinline{ $\Pass_{\mathcal{B},\mathcal{B}''} = \Pass_{\mathcal{B},\mathcal{B}'}
  \times \Pass_{\mathcal{B}',\mathcal{B}''}$}
\end{enumerate}
\end{proposition}


\end{frame}


\begin{frame}
\begin{exemple}
\label{ex:matpassag}
Soit $E = \Rr^3$ avec sa base canonique $\mathcal{B}$
\pause
$$\mathcal{B}_1 = 
\left(
\begin{pmatrix} 1\\1\\0\end{pmatrix},
\begin{pmatrix} 0 \\ -1 \\ 0\end{pmatrix},
\begin{pmatrix} 3\\2\\-1\end{pmatrix}
\right)
\pause\quad \text{et} \quad
\mathcal{B}_2 = 
\left(
\begin{pmatrix} 1\\-1\\0\end{pmatrix},
\begin{pmatrix} 0 \\ 1 \\ 0\end{pmatrix},
\begin{pmatrix} 0\\0\\-1\end{pmatrix}
\right)$$
\pause
Quelle est la matrice de passage de $\mathcal{B}_1$ vers $\mathcal{B}_2$ ?

\pause
\begin{itemize}

  \item $\Pass_{\mathcal{B}, \mathcal{B}_1} = 
\begin{pmatrix}
1 & 0 & 3\\
1 & -1 & 2\\
0 & 0 &-1
\end{pmatrix}
\pause
\quad \text{ et } \quad 
\Pass_{\mathcal{B}, \mathcal{B}_2} = 
\begin{pmatrix}
1 & 0 & 0\\
-1 & 1 & 0\\
0 & 0 &-1
\end{pmatrix}$

\pause
  \item $\Pass_{\mathcal{B}, \mathcal{B}_2} 
  = \Pass_{\mathcal{B}, \mathcal{B}_1} \times \Pass_{\mathcal{B}_1, \mathcal{B}_2}$
  \pause
  \quad donc \quad
  $\Pass_{\mathcal{B}_1, \mathcal{B}_2} 
  =  \Pass_{\mathcal{B}, \mathcal{B}_1}^{-1} \times \Pass_{\mathcal{B}, \mathcal{B}_2}$

\pause
  \item  $\Pass_{\mathcal{B}_1, \mathcal{B}_2} =
\begin{pmatrix}
1 & 0 & 3\\
1 & -1 & 2\\
0 & 0 &-1
\end{pmatrix}^{-1}
\times 
\begin{pmatrix}
1 & 0 & 0\\
-1 & 1 & 0\\
0 & 0 &-1
\end{pmatrix}
\pause
=  \begin{pmatrix}
1 & 0 & -3\\
2 & -1 & -1\\
0 & 0 & 1
\end{pmatrix}$
  
\end{itemize}

\end{exemple}
\end{frame}


\begin{frame}

\begin{itemize}


  \item $\mathcal{B}=(e_1, e_2, \ldots ,e_n)$ et 
$\mathcal{B}' = (e'_1, e'_2, \ldots ,e'_n)$ deux bases de $E$

\pause

  \item $\Pass_{\mathcal{B},\mathcal{B}'}$ la matrice de passage de 
la base $\mathcal{B}$ vers la base $\mathcal{B}'$

\pause
  
  \item Pour $x \in E$, $x=\displaystyle\sum_{i=1}^n x_ie_i$\pause, on note $X \!=\! \Mat_\mathcal{B} (x) = \!
\left(\begin{smallmatrix}x_1\cr x_2\cr \vdots \cr x_n
\end{smallmatrix}\right)_{\!\!\!\mathcal{B}}$

\pause

  \item Il s'écrit aussi $x=\displaystyle\sum_{i=1}^n x'_ie'_i$ \ \pause et on note $X' = \Mat_{\mathcal{B}'} (x) = 
\left(\begin{smallmatrix}x'_1\cr x'_2\cr \vdots \cr x'_n
\end{smallmatrix}\right)_{\!\!\!\mathcal{B}'}$
\end{itemize}
\vspace*{-1ex}
\pause
\begin{proposition}
\mybox{$X = \Pass_{\mathcal{B},\mathcal{B}'} \times X'$}
\end{proposition}
\vspace*{-1ex}
\pause
\begin{proof}
$\Pass_{\mathcal{B},\mathcal{B}'}$ est la matrice de $\id_E : (E,\mathcal{B}')\to (E,\mathcal{B})$ 

\pause
\smallskip

\centerline{$X 
\pause
= \Mat_\mathcal{B} (x) 
\pause
= \Mat_{\mathcal{B}} \big( \id_E(x) \big)
\pause
= \Mat_{\mathcal{B}',\mathcal{B}} (\id_E) \times \Mat_{\mathcal{B}'} (x)
\pause
= \Pass_{\mathcal{B},\mathcal{B}'} \times X'$}

\end{proof}

\end{frame}

%%%%%%%%%%%%%%%%%%%%%%%%%%%%%%%%%%%%%%%%%%%%%%%%%%%%%%%%%%%%%%%%
\section{Formule de changement de base}

% \begin{frame}
% 
% \evidence{Formule de changement de base}
% 
% \begin{itemize}
%   \item Soient $E$ et $F$ deux $\Kk$-espaces vectoriels de dimension finie
%   
%   \item Soit $f : E \to F$ une application linéaire  
%   
%   \item Soient $\mathcal{B}_E$, $\mathcal{B}'_E$ deux bases de $E$
%   
%   \item Soient $\mathcal{B}_F$, $\mathcal{B}'_F$ deux bases de $F$
%   
%   \item Soit $P = \Pass_{\mathcal{B}_E,\mathcal{B}'_E}$ la matrice de passage de $\mathcal{B}_E$
%   à $\mathcal{B}'_E$
%   
%   \item Soit $Q = \Pass_{\mathcal{B}_F,\mathcal{B}'_F}$ la matrice de passage de $\mathcal{B}_F$
%   à $\mathcal{B}_F'$
%   
%   \item Soit $A = \Mat_{\mathcal{B}_E,\mathcal{B}_F} (f)$ la matrice de l'application linéaire $f$ de la base
%   $\mathcal{B}_E$ vers la base $\mathcal{B}_F$
%   
%   \item Soit $B = \Mat_{\mathcal{B}'_E,\mathcal{B}'_F} (f)$ la matrice de l'application linéaire $f$ de la base
%   $\mathcal{B}'_E$ vers la base $\mathcal{B}'_F$
% \end{itemize}
% 
% 
% \begin{theoreme}
% \label{th:changementbase}
% \mybox{$B = Q^{-1} A P$}
% \end{theoreme}
% 
% \end{frame}


\begin{frame}

\evidence{Formule de changement de base}
\pause
\begin{itemize} 
  \item Soit $f : E \to E$ une application linéaire
  \pause
  \item Soient $\mathcal{B}$, $\mathcal{B}'$ deux bases de $E$
  \pause
  \item Soit $A = \Mat_{\mathcal{B}} (f)$ 
  la matrice de l'application $f$ dans la base 
  $\mathcal{B}$
  \pause
  \item Soit $B = \Mat_{\mathcal{B}'} (f)$ 
  la matrice de l'application $f$ dans
  la base $\mathcal{B}'$ 
  \pause
  \item Soit $P = \Pass_{\mathcal{B},\mathcal{B}'}$ 
  la matrice de passage de $\mathcal{B}$ à $\mathcal{B}'$
\end{itemize}

\pause

\begin{theoreme}
\label{cor:changementbase}
\mybox{$B = P^{-1} A P$}
\end{theoreme}
\end{frame}


\begin{frame}
\begin{exemple}
$$\mathcal{B}_1 = 
\left(
\left(\begin{smallmatrix} 1\\1\\0\end{smallmatrix}\right),
\left(\begin{smallmatrix} 0 \\ -1 \\ 0\end{smallmatrix}\right),
\left(\begin{smallmatrix} 3\\2\\-1\end{smallmatrix}\right)
\right)
\quad \text{ et } \quad
\mathcal{B}_2 = 
\left(
\left(\begin{smallmatrix} 1\\-1\\0\end{smallmatrix}\right),
\left(\begin{smallmatrix} 0 \\ 1 \\ 0\end{smallmatrix}\right),
\left(\begin{smallmatrix} 0\\0\\-1\end{smallmatrix}\right)
\right)
$$

\bigskip
\pause

Soit $f \!:\! \Rr^3 \to \Rr^3$ l'endomorphisme dont
la matrice dans la base $\mathcal{B}_1$ est $A$
$$A = \Mat_{\mathcal{B}_1}(f)
=\begin{pmatrix}
1 & 0 & -6\\
-2 & 2 & -7\\
0 & 0 & 3
\end{pmatrix}
\qquad
\pause
\uncover<4->{P = \Pass_{\mathcal{B}_1, \mathcal{B}_2} = 
\begin{pmatrix}
1 & 0 & -3\\
2 & -1 & -1\\
0 & 0 & 1
\end{pmatrix}}$$

\medskip


Que vaut la matrice de $f$ dans la base $\mathcal{B}_2$, $B=\Mat_{\mathcal{B}_2}(f)$ ?
\pause\pause
\small
  $$B \pause\!=\! P^{-1} A P \pause\!=\! 
\begin{pmatrix}
1 & 0 & 3\\
2 & -1 & 5\\
0 & 0 & 1
\end{pmatrix} \!\times\!
\begin{pmatrix}
1 & 0 & -6\\
-2 & 2 & -7\\
0 & 0 & 3
\end{pmatrix}
\!\times \!
\begin{pmatrix}
1 & 0 & -3\\
2 & -1 & -1\\
0 & 0 & 1
\end{pmatrix}
\pause\!=\!
\begin{pmatrix}
1 & 0 & 0\\
0 & 2 & 0\\
0 & 0 & 3
\end{pmatrix}$$ 
\end{exemple}
\end{frame}

%%%%%%%%%%%%%%%%%%%%%%%%%%%%%%%%%%%%%%%%%%%%%%%%%%%%%%%%%%%%%%%%
\section{Matrices semblables}

\begin{frame}
Soient $A$ et $B$ deux matrices de $M_n(\Kk)$
\begin{mydefinition}
$B$ est \defi{semblable} à la matrice $A$ s'il existe une
matrice inversible $P \in M_n(\Kk)$ telle que
$B=P^{-1}AP$
\end{mydefinition}

\pause

La relation << être semblable >> est une relation d'équivalence \pause:
\begin{proposition}
\begin{itemize}
  \item \evidence{réflexivité} : une matrice $A$ est semblable à elle-même
  \pause
  \item \evidence{symétrie} : si $A$ est semblable à $B$, 
  alors $B$ est semblable à $A$
  \pause
  \item \evidence{transitivité} : si $A$ est semblable à $B$, 
  et $B$ est semblable à $C$, alors $A$ est semblable à $C$
\end{itemize}
\end{proposition}

\pause

\begin{corollaire}
Deux matrices semblables représentent le même endomorphisme, mais exprimé dans
des bases différentes
\end{corollaire}
\end{frame}


%%%%%%%%%%%%%%%%%%%%%%%%%%%%%%%%%%%%%%%%%%%%%%%%%%%%%%%%%%%%%%%%
\section{Mini-exercices}

\begin{frame}

\begin{miniexercice}
Soit $f : \Rr^2 \to \Rr^2$ définie par $f(x,y) = (2x+y,3x-2y)$,
Soit $v = \left(\begin{smallmatrix}3\\-4\end{smallmatrix}\right) \in \Rr^2$
avec ses coordonnées dans la base canonique $\mathcal{B}_0$ de $\Rr^2$.
Soit $\mathcal{B}_1 = \left( 
\left(\begin{smallmatrix}3\\2\end{smallmatrix}\right),
\left(\begin{smallmatrix}2\\2\end{smallmatrix}\right)
\right)$ une autre base de $\Rr^2$.
\begin{enumerate}
  \item Calculer la matrice de $f$ dans la base canonique.

  \item Calculer les coordonnées de $f(v)$ dans la base canonique.
  
  \item Calculer la matrice de passage de $\mathcal{B}_0$ à $\mathcal{B}_1$.
  
  \item En déduire les coordonnées de $v$ dans la base $\mathcal{B}_1$,
  et de $f(v)$ dans la base $\mathcal{B}_1$. 
  
  \item Calculer la matrice de $f$ dans la base $\mathcal{B}_1$.
\end{enumerate}
  

Même exercice dans $\Rr^3$ avec $f : \Rr^3 \to \Rr^3$, $f(x,y,z) = (x-2y,y-2z,z-2x)$,
$v = \left(\begin{smallmatrix}3\\-2\\1\end{smallmatrix}\right) \in \Rr^3$
et $\mathcal{B}_1 = \left( 
\left(\begin{smallmatrix}0\\1\\2\end{smallmatrix}\right),
\left(\begin{smallmatrix}2\\0\\1\end{smallmatrix}\right),
\left(\begin{smallmatrix}1\\2\\0\end{smallmatrix}\right)
\right)$.
  
\end{miniexercice}

\end{frame}

\end{document}
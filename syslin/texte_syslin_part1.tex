
%%%%%%%%%%%%%%%%%% PREAMBULE %%%%%%%%%%%%%%%%%%


\documentclass[12pt]{article}

\usepackage{amsfonts,amsmath,amssymb,amsthm}
\usepackage[utf8]{inputenc}
\usepackage[T1]{fontenc}
\usepackage[francais]{babel}


% packages
\usepackage{amsfonts,amsmath,amssymb,amsthm}
\usepackage[utf8]{inputenc}
\usepackage[T1]{fontenc}
%\usepackage{lmodern}

\usepackage[francais]{babel}
\usepackage{fancybox}
\usepackage{graphicx}

\usepackage{float}

%\usepackage[usenames, x11names]{xcolor}
\usepackage{tikz}
\usepackage{datetime}

\usepackage{mathptmx}
%\usepackage{fouriernc}
%\usepackage{newcent}
\usepackage[mathcal,mathbf]{euler}

%\usepackage{palatino}
%\usepackage{newcent}


% Commande spéciale prompteur

%\usepackage{mathptmx}
%\usepackage[mathcal,mathbf]{euler}
%\usepackage{mathpple,multido}

\usepackage[a4paper]{geometry}
\geometry{top=2cm, bottom=2cm, left=1cm, right=1cm, marginparsep=1cm}

\newcommand{\change}{{\color{red}\rule{\textwidth}{1mm}\\}}

\newcounter{mydiapo}

\newcommand{\diapo}{\newpage
\hfill {\normalsize  Diapo \themydiapo \quad \texttt{[\jobname]}} \\
\stepcounter{mydiapo}}


%%%%%%% COULEURS %%%%%%%%%%

% Pour blanc sur noir :
%\pagecolor[rgb]{0.5,0.5,0.5}
% \pagecolor[rgb]{0,0,0}
% \color[rgb]{1,1,1}



%\DeclareFixedFont{\myfont}{U}{cmss}{bx}{n}{18pt}
\newcommand{\debuttexte}{
%%%%%%%%%%%%% FONTES %%%%%%%%%%%%%
\renewcommand{\baselinestretch}{1.5}
\usefont{U}{cmss}{bx}{n}
\bfseries

% Taille normale : commenter le reste !
%Taille Arnaud
%\fontsize{19}{19}\selectfont

% Taille Barbara
%\fontsize{21}{22}\selectfont

%Taille François
%\fontsize{25}{30}\selectfont

%Taille Pascal
%\fontsize{25}{30}\selectfont

%Taille Laura
%\fontsize{30}{35}\selectfont


%\myfont
%\usefont{U}{cmss}{bx}{n}

%\Huge
%\addtolength{\parskip}{\baselineskip}
}


% \usepackage{hyperref}
% \hypersetup{colorlinks=true, linkcolor=blue, urlcolor=blue,
% pdftitle={Exo7 - Exercices de mathématiques}, pdfauthor={Exo7}}


%section
% \usepackage{sectsty}
% \allsectionsfont{\bf}
%\sectionfont{\color{Tomato3}\upshape\selectfont}
%\subsectionfont{\color{Tomato4}\upshape\selectfont}

%----- Ensembles : entiers, reels, complexes -----
\newcommand{\Nn}{\mathbb{N}} \newcommand{\N}{\mathbb{N}}
\newcommand{\Zz}{\mathbb{Z}} \newcommand{\Z}{\mathbb{Z}}
\newcommand{\Qq}{\mathbb{Q}} \newcommand{\Q}{\mathbb{Q}}
\newcommand{\Rr}{\mathbb{R}} \newcommand{\R}{\mathbb{R}}
\newcommand{\Cc}{\mathbb{C}} 
\newcommand{\Kk}{\mathbb{K}} \newcommand{\K}{\mathbb{K}}

%----- Modifications de symboles -----
\renewcommand{\epsilon}{\varepsilon}
\renewcommand{\Re}{\mathop{\text{Re}}\nolimits}
\renewcommand{\Im}{\mathop{\text{Im}}\nolimits}
%\newcommand{\llbracket}{\left[\kern-0.15em\left[}
%\newcommand{\rrbracket}{\right]\kern-0.15em\right]}

\renewcommand{\ge}{\geqslant}
\renewcommand{\geq}{\geqslant}
\renewcommand{\le}{\leqslant}
\renewcommand{\leq}{\leqslant}

%----- Fonctions usuelles -----
\newcommand{\ch}{\mathop{\mathrm{ch}}\nolimits}
\newcommand{\sh}{\mathop{\mathrm{sh}}\nolimits}
\renewcommand{\tanh}{\mathop{\mathrm{th}}\nolimits}
\newcommand{\cotan}{\mathop{\mathrm{cotan}}\nolimits}
\newcommand{\Arcsin}{\mathop{\mathrm{Arcsin}}\nolimits}
\newcommand{\Arccos}{\mathop{\mathrm{Arccos}}\nolimits}
\newcommand{\Arctan}{\mathop{\mathrm{Arctan}}\nolimits}
\newcommand{\Argsh}{\mathop{\mathrm{Argsh}}\nolimits}
\newcommand{\Argch}{\mathop{\mathrm{Argch}}\nolimits}
\newcommand{\Argth}{\mathop{\mathrm{Argth}}\nolimits}
\newcommand{\pgcd}{\mathop{\mathrm{pgcd}}\nolimits} 

\newcommand{\Card}{\mathop{\text{Card}}\nolimits}
\newcommand{\Ker}{\mathop{\text{Ker}}\nolimits}
\newcommand{\id}{\mathop{\text{id}}\nolimits}
\newcommand{\ii}{\mathrm{i}}
\newcommand{\dd}{\mathrm{d}}
\newcommand{\Vect}{\mathop{\text{Vect}}\nolimits}
\newcommand{\Mat}{\mathop{\mathrm{Mat}}\nolimits}
\newcommand{\rg}{\mathop{\text{rg}}\nolimits}
\newcommand{\tr}{\mathop{\text{tr}}\nolimits}
\newcommand{\ppcm}{\mathop{\text{ppcm}}\nolimits}

%----- Structure des exercices ------

\newtheoremstyle{styleexo}% name
{2ex}% Space above
{3ex}% Space below
{}% Body font
{}% Indent amount 1
{\bfseries} % Theorem head font
{}% Punctuation after theorem head
{\newline}% Space after theorem head 2
{}% Theorem head spec (can be left empty, meaning ‘normal’)

%\theoremstyle{styleexo}
\newtheorem{exo}{Exercice}
\newtheorem{ind}{Indications}
\newtheorem{cor}{Correction}


\newcommand{\exercice}[1]{} \newcommand{\finexercice}{}
%\newcommand{\exercice}[1]{{\tiny\texttt{#1}}\vspace{-2ex}} % pour afficher le numero absolu, l'auteur...
\newcommand{\enonce}{\begin{exo}} \newcommand{\finenonce}{\end{exo}}
\newcommand{\indication}{\begin{ind}} \newcommand{\finindication}{\end{ind}}
\newcommand{\correction}{\begin{cor}} \newcommand{\fincorrection}{\end{cor}}

\newcommand{\noindication}{\stepcounter{ind}}
\newcommand{\nocorrection}{\stepcounter{cor}}

\newcommand{\fiche}[1]{} \newcommand{\finfiche}{}
\newcommand{\titre}[1]{\centerline{\large \bf #1}}
\newcommand{\addcommand}[1]{}
\newcommand{\video}[1]{}

% Marge
\newcommand{\mymargin}[1]{\marginpar{{\small #1}}}



%----- Presentation ------
\setlength{\parindent}{0cm}

%\newcommand{\ExoSept}{\href{http://exo7.emath.fr}{\textbf{\textsf{Exo7}}}}

\definecolor{myred}{rgb}{0.93,0.26,0}
\definecolor{myorange}{rgb}{0.97,0.58,0}
\definecolor{myyellow}{rgb}{1,0.86,0}

\newcommand{\LogoExoSept}[1]{  % input : echelle
{\usefont{U}{cmss}{bx}{n}
\begin{tikzpicture}[scale=0.1*#1,transform shape]
  \fill[color=myorange] (0,0)--(4,0)--(4,-4)--(0,-4)--cycle;
  \fill[color=myred] (0,0)--(0,3)--(-3,3)--(-3,0)--cycle;
  \fill[color=myyellow] (4,0)--(7,4)--(3,7)--(0,3)--cycle;
  \node[scale=5] at (3.5,3.5) {Exo7};
\end{tikzpicture}}
}



\theoremstyle{definition}
%\newtheorem{proposition}{Proposition}
%\newtheorem{exemple}{Exemple}
%\newtheorem{theoreme}{Théorème}
\newtheorem{lemme}{Lemme}
\newtheorem{corollaire}{Corollaire}
%\newtheorem*{remarque*}{Remarque}
%\newtheorem*{miniexercice}{Mini-exercices}
%\newtheorem{definition}{Définition}




%definition d'un terme
\newcommand{\defi}[1]{{\color{myorange}\textbf{\emph{#1}}}}
\newcommand{\evidence}[1]{{\color{blue}\textbf{\emph{#1}}}}



 %----- Commandes divers ------

\newcommand{\codeinline}[1]{\texttt{#1}}

%%%%%%%%%%%%%%%%%%%%%%%%%%%%%%%%%%%%%%%%%%%%%%%%%%%%%%%%%%%%%
%%%%%%%%%%%%%%%%%%%%%%%%%%%%%%%%%%%%%%%%%%%%%%%%%%%%%%%%%%%%%



\begin{document}

\debuttexte


%%%%%%%%%%%%%%%%%%%%%%%%%%%%%%%%%%%%%%%%%%%%%%%%%%%%%%%%%%%
\diapo

Les systèmes linéaires interviennent dans de nombreux contextes 
d'applications car ils forment la base calculatoire 
de l'algèbre linéaire. Ils permettent également de traiter 
une bonne partie de la théorie de l'algèbre linéaire en dimension finie.

\change

Le plan de cette première partie sur les systèmes linéaires est le suivant :

\change

nous commençons par l'exemple du système donnant l'ensemble intersection de  deux droites dans le plan,

\change

puis nous passerons à la méthode de résolution par substitution,

\change

nous examinerons ensuite l'exemple du système donnant l'ensemble intersection de deux plans dans l'espace,

\change

avant d'expliquer la méthode de résolution dite de Cramer,

\change

pour terminer par le lien entre résolution de systèmes linéaires et inversion de matrices. 





%%%%%%%%%%%%%%%%%%%%%%%%%%%%%%%%%%%%%%%%%%%%%%%%%%%%%%%%%%
\diapo

Commençons par le système le plus simple, celui à deux équations et à deux inconnues. Ce type de système apparaît lorsque l'on étudie l'intersection de deux droites dans le plan. On rappelle que l'équation d'une droite est de la forme
$$a x + b y = e$$

où $a, b$ et $e$ sont des constantes.

\change

On dit que cette équation est *linéaire* en les variables \(x\) et \(y\) : 
Il n y a pas de de carrés ou de cubes, par exemple.

\change

On fixe à présent deux droites \(D_1\) et \(D_2\) et on cherche à déterminer leur intersection. 

\change

Un point de coordonnées \((x,y)\) appartient à cette intersection précisément 
lorsqu'il vérifie simultanément les deux équations des deux droites, 
ce qui se traduit par le système suivant :


[[dire *et* entre les deux équations]]

\begin{equation}
\left\{\begin{array}{rcl} 
a x + b y & = & e\\
c x + d y & = & f 
\end{array}\right.  
\tag{$S$}  
\label{eq:syslin1}  
\end{equation}



%%%%%%%%%%%%%%%%%%%%%%%%%%%%%%%%%%%%%%%%%%%%%%%%%%%%%%%%%%%
\diapo

Il y a différentes configurations possibles.

Dans un premier cas, 
les droites $D_1$ et $D_2$ se coupent en un seul point. 
Dans ce cas le système à deux inconnues et deux équations a une seule solution.


\change

Dans le deuxième cas, les droites $D_1$ et $D_2$ sont parallèles et distinctes. 
Alors le système linéaire n'a pas de solution. 


\change

Enfin dans le dernier cas, les droites $D_1$ et $D_2$ sont confondues. 
Alors le système linéaire a une infinité de solutions.  


%%%%%%%%%%%%%%%%%%%%%%%%%%%%%%%%%%%%%%%%%%%%%%%%%%%%%%%%%%
\diapo

Pour savoir s'il existe une ou plusieurs solutions à un système linéaire, 
et les calculer, une première méthode est la substitution.
Nous allons illustrer cette méthode sur le système suivant :
\begin{equation}
\left\{\begin{array}{rcl} 
3 x + 2 y & = & 1\\
2 x - 7 y & = & -2 
\end{array}\right.   
\end{equation}


\change

On exprime la variable \(y\) en fonction de \(x\) grâce à la première équation, 
puis on remplace \(y\) par son expression \(\frac{1}{2}- \frac{3}{2}x \) 
dans la deuxième équation :

\change

La deuxième équation ne contient plus que la variable \(x\) et peut donc être résolue, ce qui donne dans un premier temps :

$(2+7\times\frac32)x   =  -2 +\frac72$

\change

Un calcul direct donne finalement \(x=\frac{3}{25} \) 


\change

Il suffit de remplacer \(x\) par sa valeur \(\frac{3}{25} \) dans la première équation 
pour obtenir celle de \(y\), qui est \(\frac{8}{25} \): 


\change

On a finalement obtenu l'ensemble des solutions du système, qui est un singleton, c'est-à-dire un ensemble qui possède un unique élément, ici le couple \(\left(\frac{3}{25},\frac{8}{25}\right)\) .





%%%%%%%%%%%%%%%%%%%%%%%%%%%%%%%%%%%%%%%%%%%%%%%%%%%%%%%%%%
\diapo

On rappelle que dans l'espace $(Oxyz)$, une équation linéaire du type
$$a x + b y  + c z = d$$
où $(a,b,c)$ sont des constantes non toutes nulles, est l'équation d'un plan.

\change

Un système à deux équations et à trois inconnues représente donc l'ensemble des coordonnées des points dans l'intersection de deux plans dans l'espace.

\change

Géométriquement, on voit qu'il y a trois possibilités :

\begin{itemize}
	\item premier cas : les deux plans sont strictement parallèles et il n'y a aucune solution au système,
	\item deuxième cas : les deux plans sont confondus et il y a une infinité de solutions au système,
	\item enfin, troisième cas : les deux plans se coupent en une droite et il y a aussi une infinité de solutions au système.
\end{itemize}
%%%%%%%%%%%%%%%%%%%%%%%%%%%%%%%%%%%%%%%%%%%%%%%%%%%%%%%%%%%
\diapo

Regardons brièvement trois exemples.

\begin{itemize}
	\item Ce premier système n'a pas de solution, car en divisant par $2$ la seconde équation, 
	on obtient deux équations  incompatibles, vu que $7 \neq -\frac{1}{2}$. 
	Géométriquement, nous sommes dans le cas de deux plans strictement parallèles.
		
	\change
	
	\item Dans le deuxième système, les deux équations sont équivalentes : 
	on passe de la première à la deuxième en multipliant par  \(2\). 
	En résolvant la première équation en $z$ par rapport à $x$ et $y$, 
	on obtient un ensemble de solutions dépendant de deux paramètres. 
	Nous sommes ici dans la situation de deux plans confondus.
		
	\change
	
\item On peut résoudre le troisième système en prenant 	\(x\) 
comme paramètre et en résolvant le système correspondant à deux 
équations et aux deux inconnues \(y\) et \(z\), par exemple par substitution.  
On obtient un ensemble de solutions dépendant du paramètre \(x\). 
Il s'agit d'une description paramétrique de la droite intersection des deux plans dont on est parti.
\end{itemize}

Il faut enfin remarquer qu'on pourrait faire une analyse tout à fait similaire pour les systèmes à *trois* équations et trois inconnues \(x\), \(y\) et \(z\). Le cas le plus fréquent est celui où les trois plans correspondants ont pour intersection un point, ce qui donne un ensemble de solutions à un élément. Faites la liste des cas possibles, c'est un bon exercice !

%%%%%%%%%%%%%%%%%%%%%%%%%%%%%%%%%%%%%%%%%%%%%%%%%%%%%%%%%%
\diapo

On va à présent expliquer la méthode de résolution dite de Cramer, dans le cas des systèmes à deux équations et à deux inconnues.

\change

On adopte la notation suivante pour le déterminant $ad-bc$ du système : les colonnes sont celles des coefficients du système, les barres latérales signifient que l'on prend le déterminant.

\change

Si le déterminant du système est non nul, le système a une solution unique, donnée par les formules suivantes, 
qui sont des fractions de déterminants.

$$x=\frac{ed-bf}{ad-bc} \qquad y=\frac{af-ec}{ad-bc}$$

Notez que le dénominateur égale le déterminant du système pour les deux coordonnées et est donc non nul.
Pour le numérateur de la première coordonnée $x$, on remplace la première colonne par le second membre ;
pour la deuxième coordonnée $y$, c'est la deuxième colonne que l'on remplace par le second membre.


%%%%%%%%%%%%%%%%%%%%%%%%%%%%%%%%%%%%%%%%%%%%%%%%%%%%%%%%%%%
\diapo

La méthode de Cramer est bien adaptée à la résolution de systèmes avec un (ou plusieurs) paramètre(s), voici un exemple,
avec un paramètre réel $t$.

\change

Le déterminant du système vaut $t^2+6$, qui est strictement positif si $t$ est réel, et donc ne s'annule pas.

\change

En appliquant la méthode de Cramer, on obtient immédiatement les valeurs de $x$ et $y$. 
Dans les deux cas, le dénominateur est le déterminant du système $t^2+6$. 
Pour la première coordonnée $x$, on calcule le dénominateur en remplaçant dans le 
déterminant du système la première colonne par la colonne formée par le second membre $(1,1)$. Pour la deuxième coordonnée $y$, on procède de même avec la deuxième colonne.

\change

En résumé, pour un $t$ fixé, l'ensemble des solutions est un singleton.

%%%%%%%%%%%%%%%%%%%%%%%%%%%%%%%%%%%%%%%%%%%%%%%%%%%%%%%%%%%
\diapo

Expliquons une méthode de résolution reposant sur l'inversion des matrices. Tout système linéaire admet une traduction matricielle. Voici comme cela s'exprime pour des systèmes à deux équations et à deux inconnues. Le système suivant : 

$$
\left\{\begin{array}{rcl} 
a x + b y & = & e\\
c x + d y & = & f 
\end{array}\right.  
$$


\change

est équivalent à l'égalité matricielle $AX=Y$, où $A$ est la matrice $(a,b,c,d)$
des coefficients du système, $X$ est le vecteur colonne des inconnues 
$(x,y)$, $AX$ est le produit matriciel de $A$ et de $X$, donc un 
vecteur à une colonne et deux lignes, enfin $Y$ est le vecteur colonne 
donné par le second membre. Résoudre le système revient à résoudre cette 
équation matricielle avec comme inconnue le vecteur colonne $X$.


\change

Pour résoudre cette équation matricielle, on peut utiliser l'inverse de la matrice $A$, 
qui existe si son déterminant est non nul, et qui prend la forme suivante. 
$$A^{-1} = \frac{1}{ad-bc} \begin{pmatrix} d & -b \\ -c & a \end{pmatrix}$$
La matrice inverse d'une matrice $2x2$ est donc $1$ sur le déterminant fois la matrice 
obtenue en échangeant les coefficients sur la diagonale principale, ici $a$ et $d$, 
et en changeant les signes les coefficients sur l'autre diagonale, ici \(b\) et \(c\).


\change

Alors la solution de l'équation matricielle, et donc celle du système de départ, 
est donnée par l'égalité : $X = A^{-1} Y$ : le produit matriciel de $A^{-1}$ 
par le second membre $Y$.

%%%%%%%%%%%%%%%%%%%%%%%%%%%%%%%%%%%%%%%%%%%%%%%%%%%%%%%%%%%
\diapo

On va illustrer la méthode de résolution par inversion de matrice sur le système à paramètre suivant.

\change

Le déterminant du système se calcule immédiatement, c'est $t^2-1$.

\change

Dans le premier cas, on suppose que ce déterminant est non nul. 

\change

Alors la matrice $A$ des coefficients du système est inversible et son inverse est donnée par la formule que nous venons de voir : $1$ sur le déterminant fois la matrice obtenue en échangeant les coefficients sur la diagonale principale $1$ et $t^2$, et en changeant le signe des coefficients sur l'autre diagonale $1$ et $1$.

\change

On obtient le vecteur colonne solution $X$ en multipliant l'inverse de $A$ 
par le vecteur colonne $Y$ correspondant au second membre. 
Un calcul direct (lorsqu'on connait le produit matriciel !) 
donne que $X$ vaut $\frac{1}{t^2-1}\begin{pmatrix} t^2-t\\ t-1 \end{pmatrix}$
qui se simplifie ainsi.

\change

On a donc déterminé l'ensemble solution, qui est donc le singleton formé par le couple $\left(\frac{t}{t+1},\frac{1}{t+1}\right)$.



%%%%%%%%%%%%%%%%%%%%%%%%%%%%%%%%%%%%%%%%%%%%%%%%%%%%%%%%%%%
\diapo

Examinons un deuxième cas où $t=+1$. 

\change
Alors le système se réduit à deux fois la même équation $x+y=1$, 

\change
et l'ensemble solution, 

\change
exprimé en fonction du paramètre $x$, est l'ensemble des couples $(x,1-x)$.

\change

Enfin, dans le troisième et dernier cas $t=-1$, 

\change
le système est constitué 

\change
des deux équations contradictoires $x+y=1$ et $x+y=-1$, 

\change
donc n'a pas de solution.

%%%%%%%%%%%%%%%%%%%%%%%%%%%%%%%%%%%%%%%%%%%%%%%%%%%%%%%%%%%
\diapo

Voici quelques exercices sur les systèmes linéaires pour vous entrainer sur les méthodes que nous venons de voir !


\end{document}

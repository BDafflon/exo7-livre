
%%%%%%%%%%%%%%%%%% PREAMBULE %%%%%%%%%%%%%%%%%%


\documentclass[12pt]{article}

\usepackage{amsfonts,amsmath,amssymb,amsthm}
\usepackage[utf8]{inputenc}
\usepackage[T1]{fontenc}
\usepackage[francais]{babel}


% packages
\usepackage{amsfonts,amsmath,amssymb,amsthm}
\usepackage[utf8]{inputenc}
\usepackage[T1]{fontenc}
%\usepackage{lmodern}

\usepackage[francais]{babel}
\usepackage{fancybox}
\usepackage{graphicx}

\usepackage{float}

%\usepackage[usenames, x11names]{xcolor}
\usepackage{tikz}
\usepackage{datetime}

\usepackage{mathptmx}
%\usepackage{fouriernc}
%\usepackage{newcent}
\usepackage[mathcal,mathbf]{euler}

%\usepackage{palatino}
%\usepackage{newcent}


% Commande spéciale prompteur

%\usepackage{mathptmx}
%\usepackage[mathcal,mathbf]{euler}
%\usepackage{mathpple,multido}

\usepackage[a4paper]{geometry}
\geometry{top=2cm, bottom=2cm, left=1cm, right=1cm, marginparsep=1cm}

\newcommand{\change}{{\color{red}\rule{\textwidth}{1mm}\\}}

\newcounter{mydiapo}

\newcommand{\diapo}{\newpage
\hfill {\normalsize  Diapo \themydiapo \quad \texttt{[\jobname]}} \\
\stepcounter{mydiapo}}


%%%%%%% COULEURS %%%%%%%%%%

% Pour blanc sur noir :
%\pagecolor[rgb]{0.5,0.5,0.5}
% \pagecolor[rgb]{0,0,0}
% \color[rgb]{1,1,1}



%\DeclareFixedFont{\myfont}{U}{cmss}{bx}{n}{18pt}
\newcommand{\debuttexte}{
%%%%%%%%%%%%% FONTES %%%%%%%%%%%%%
\renewcommand{\baselinestretch}{1.5}
\usefont{U}{cmss}{bx}{n}
\bfseries

% Taille normale : commenter le reste !
%Taille Arnaud
%\fontsize{19}{19}\selectfont

% Taille Barbara
%\fontsize{21}{22}\selectfont

%Taille François
%\fontsize{25}{30}\selectfont

%Taille Pascal
%\fontsize{25}{30}\selectfont

%Taille Laura
%\fontsize{30}{35}\selectfont


%\myfont
%\usefont{U}{cmss}{bx}{n}

%\Huge
%\addtolength{\parskip}{\baselineskip}
}


% \usepackage{hyperref}
% \hypersetup{colorlinks=true, linkcolor=blue, urlcolor=blue,
% pdftitle={Exo7 - Exercices de mathématiques}, pdfauthor={Exo7}}


%section
% \usepackage{sectsty}
% \allsectionsfont{\bf}
%\sectionfont{\color{Tomato3}\upshape\selectfont}
%\subsectionfont{\color{Tomato4}\upshape\selectfont}

%----- Ensembles : entiers, reels, complexes -----
\newcommand{\Nn}{\mathbb{N}} \newcommand{\N}{\mathbb{N}}
\newcommand{\Zz}{\mathbb{Z}} \newcommand{\Z}{\mathbb{Z}}
\newcommand{\Qq}{\mathbb{Q}} \newcommand{\Q}{\mathbb{Q}}
\newcommand{\Rr}{\mathbb{R}} \newcommand{\R}{\mathbb{R}}
\newcommand{\Cc}{\mathbb{C}} 
\newcommand{\Kk}{\mathbb{K}} \newcommand{\K}{\mathbb{K}}

%----- Modifications de symboles -----
\renewcommand{\epsilon}{\varepsilon}
\renewcommand{\Re}{\mathop{\text{Re}}\nolimits}
\renewcommand{\Im}{\mathop{\text{Im}}\nolimits}
%\newcommand{\llbracket}{\left[\kern-0.15em\left[}
%\newcommand{\rrbracket}{\right]\kern-0.15em\right]}

\renewcommand{\ge}{\geqslant}
\renewcommand{\geq}{\geqslant}
\renewcommand{\le}{\leqslant}
\renewcommand{\leq}{\leqslant}

%----- Fonctions usuelles -----
\newcommand{\ch}{\mathop{\mathrm{ch}}\nolimits}
\newcommand{\sh}{\mathop{\mathrm{sh}}\nolimits}
\renewcommand{\tanh}{\mathop{\mathrm{th}}\nolimits}
\newcommand{\cotan}{\mathop{\mathrm{cotan}}\nolimits}
\newcommand{\Arcsin}{\mathop{\mathrm{Arcsin}}\nolimits}
\newcommand{\Arccos}{\mathop{\mathrm{Arccos}}\nolimits}
\newcommand{\Arctan}{\mathop{\mathrm{Arctan}}\nolimits}
\newcommand{\Argsh}{\mathop{\mathrm{Argsh}}\nolimits}
\newcommand{\Argch}{\mathop{\mathrm{Argch}}\nolimits}
\newcommand{\Argth}{\mathop{\mathrm{Argth}}\nolimits}
\newcommand{\pgcd}{\mathop{\mathrm{pgcd}}\nolimits} 

\newcommand{\Card}{\mathop{\text{Card}}\nolimits}
\newcommand{\Ker}{\mathop{\text{Ker}}\nolimits}
\newcommand{\id}{\mathop{\text{id}}\nolimits}
\newcommand{\ii}{\mathrm{i}}
\newcommand{\dd}{\mathrm{d}}
\newcommand{\Vect}{\mathop{\text{Vect}}\nolimits}
\newcommand{\Mat}{\mathop{\mathrm{Mat}}\nolimits}
\newcommand{\rg}{\mathop{\text{rg}}\nolimits}
\newcommand{\tr}{\mathop{\text{tr}}\nolimits}
\newcommand{\ppcm}{\mathop{\text{ppcm}}\nolimits}

%----- Structure des exercices ------

\newtheoremstyle{styleexo}% name
{2ex}% Space above
{3ex}% Space below
{}% Body font
{}% Indent amount 1
{\bfseries} % Theorem head font
{}% Punctuation after theorem head
{\newline}% Space after theorem head 2
{}% Theorem head spec (can be left empty, meaning ‘normal’)

%\theoremstyle{styleexo}
\newtheorem{exo}{Exercice}
\newtheorem{ind}{Indications}
\newtheorem{cor}{Correction}


\newcommand{\exercice}[1]{} \newcommand{\finexercice}{}
%\newcommand{\exercice}[1]{{\tiny\texttt{#1}}\vspace{-2ex}} % pour afficher le numero absolu, l'auteur...
\newcommand{\enonce}{\begin{exo}} \newcommand{\finenonce}{\end{exo}}
\newcommand{\indication}{\begin{ind}} \newcommand{\finindication}{\end{ind}}
\newcommand{\correction}{\begin{cor}} \newcommand{\fincorrection}{\end{cor}}

\newcommand{\noindication}{\stepcounter{ind}}
\newcommand{\nocorrection}{\stepcounter{cor}}

\newcommand{\fiche}[1]{} \newcommand{\finfiche}{}
\newcommand{\titre}[1]{\centerline{\large \bf #1}}
\newcommand{\addcommand}[1]{}
\newcommand{\video}[1]{}

% Marge
\newcommand{\mymargin}[1]{\marginpar{{\small #1}}}



%----- Presentation ------
\setlength{\parindent}{0cm}

%\newcommand{\ExoSept}{\href{http://exo7.emath.fr}{\textbf{\textsf{Exo7}}}}

\definecolor{myred}{rgb}{0.93,0.26,0}
\definecolor{myorange}{rgb}{0.97,0.58,0}
\definecolor{myyellow}{rgb}{1,0.86,0}

\newcommand{\LogoExoSept}[1]{  % input : echelle
{\usefont{U}{cmss}{bx}{n}
\begin{tikzpicture}[scale=0.1*#1,transform shape]
  \fill[color=myorange] (0,0)--(4,0)--(4,-4)--(0,-4)--cycle;
  \fill[color=myred] (0,0)--(0,3)--(-3,3)--(-3,0)--cycle;
  \fill[color=myyellow] (4,0)--(7,4)--(3,7)--(0,3)--cycle;
  \node[scale=5] at (3.5,3.5) {Exo7};
\end{tikzpicture}}
}



\theoremstyle{definition}
%\newtheorem{proposition}{Proposition}
%\newtheorem{exemple}{Exemple}
%\newtheorem{theoreme}{Théorème}
\newtheorem{lemme}{Lemme}
\newtheorem{corollaire}{Corollaire}
%\newtheorem*{remarque*}{Remarque}
%\newtheorem*{miniexercice}{Mini-exercices}
%\newtheorem{definition}{Définition}




%definition d'un terme
\newcommand{\defi}[1]{{\color{myorange}\textbf{\emph{#1}}}}
\newcommand{\evidence}[1]{{\color{blue}\textbf{\emph{#1}}}}



 %----- Commandes divers ------

\newcommand{\codeinline}[1]{\texttt{#1}}

%%%%%%%%%%%%%%%%%%%%%%%%%%%%%%%%%%%%%%%%%%%%%%%%%%%%%%%%%%%%%
%%%%%%%%%%%%%%%%%%%%%%%%%%%%%%%%%%%%%%%%%%%%%%%%%%%%%%%%%%%%%



\begin{document}

\debuttexte

%%%%%%%%%%%%%%%%%%%%%%%%%%%%%%%%%%%%%%%%%%%%%%%%%%%%%%%%%%%
\diapo

\change

On attribue à César l'un des premiers protocoles de cryptage d'informations.

\change

Nous verrons son principe ->

\change

qui repose sur de l'arithmétique élémentaire : les calculs modulaires.

\change

Cela nous permettra de chiffrer (et déchiffrer) des messages

\change

Malheureusement nous verrons que ce procédé est trop simple pour apporter un niveau de sécurité suffisant.

\change 

Nous terminons par une mise en oeuvre algorithmique et informatique.


%%%%%%%%%%%%%%%%%%%%%%%%%%%%%%%%%%%%%%%%%%%%%%%%%%%%%%%%%%%
\diapo

Personne ne peut affirmer que César ait lui-même prononcé la phrase imprononçable écrite ici !

\change

Mais les historiens lui attribue l'un des premiers protocoles cryptographiques.

En tant que chef des armées, César ne tenait pas à ce que ses ordres soient interceptés et utilisés par ses ennemis avant même d'arriver sur les lieux de bataille.

Il avait donc recours à un procédé qui rendait ses message inintelligibles. On parlerait aujourd'hui d'un protocole de chiffrement.

\change

Ce procédé était en fait un simple décalage des lettres de l'alphabet : 

\change

pour crypter un message, on fait correspondre à la lettre $A$ du message initial la lettre $D$ du message transmis, à $B$  $E$, $C$  $F$ etc...


\change


$\equiv \equiv \equiv$ Voici une figure avec l'alphabet d'origine en ROUGE (qui correspond au texte que l'on veut cacher), en correspondance avec l'alphabet 
en VERT (destiné à être transmis au risque d'être intercepté par l'ennemi). On retrouve le fait que A en rouge devient un D en vert 

||| etc...


%%%%%%%%%%%%%%%%%%%%%%%%%%%%%%%%%%%%%%%%%%%%%%%%%%%%%%%%%%%
\diapo

Pour prendre en compte le décalage des dernières lettres de l'alphabet, il est plus judicieux de représenter les alphabets sur deux anneaux.

\change

On parle alors d'un \defi{décalage circulaire} des lettres de l'alphabet.

\change


Pour déchiffrer le message de César, il suffit de lire cette correspondance dans le sens inverse, c'est-à-dire de l'alphabet en VERT à celui en ROUGE.

\change

$\equiv \equiv \equiv$ $D$ se déchiffre en $A$, $E$ en $B$,...

\change

Reprenons la phrase citée précédemment. Le $D$ se déchiffre en $A$, le $O$ en $L$, etc... 

\change
ce qui conduit à :

\centerline{ALEA \  JACTA \  EST}


||| qui se traduit du latin par \og Les dés sont jetés \fg.



%%%%%%%%%%%%%%%%%%%%%%%%%%%%%%%%%%%%%%%%%%%%%%%%%%%%%%%%%%%
\diapo

Depuis le début, nous parlons de chiffrement et de déchiffrement. 

La raison en est que nous pouvons décrire mathématiquement ces procédés avec des nombres et des opérations sur les nombres et ainsi confier aux ordinateurs toutes ces opérations.

Nous nous limiterons dans ce cours à utiliser le codage des 26 lettres de l'alphabet en les mettant en correspondance avec les entiers de 0 à 25.

\change

$\equiv \equiv \equiv$ En termes mathématiques, nous définissons une bijection $f$ 
qui va de l'ensemble des 26 lettres de l'alphabet dans l'ensemble des entiers de $0$ à $25$

et qui à $A$ associe $0$

à $B$ associe $1$

et à $Z$ associe $25$

\change

Par exemple la suite de lettres  
"A L E A" 
devient la suite de nombres 
"0 12 4 0".

\change
 
||| Le chiffrement de César est un cas particulier de \defi{chiffrement mono-alphabétique}, 
c'est-à-dire une procédé au cours duquel cha\-que lettre du message initial subit une transformation pour donner une lettre du message crypté.

\change

Nous allons tout de suite voir l'intérêt de transformer les lettres en nombres. 
En effet, le chiffrement de César correspond à une opération mathématique très simple.

%%%%%%%%%%%%%%%%%%%%%%%%%%%%%%%%%%%%%%%%%%%%%%%%%%%%%%%%%%%
\diapo

Soit un entier $n \ge 2$ fixé.

Nous dirons que \defi{$a$ est congru à $b$ modulo $n$},
si $n$ divise la différence des entiers $b$ et $a$.

\change

$\equiv \equiv \equiv$ On note alors 
$$a \equiv b \pmod n$$ 


\change

|||  Dans le cadre de notre travail, nous compterons modulo $26$ (les $26$ lettres de l'alphabet).

On pose donc $n=26$. 

Par exemple $28 \equiv  2 \pmod {26}$,

\change

$\equiv \equiv \equiv$ car $28-2=26$ est bien divisible par $26$. 

\change

De même $85 = 26+ 59$ donc $85 \equiv  59 \pmod {26}$

\change
Mais on a également $85 = 3 \times 26+ 7$ donc $85-7$ est divisible par $26$ ce qui fait que 
$85 \equiv  7 \pmod {26}$. 

\change

On note $\Zz/26 \Zz$ l'ensemble de tous les éléments de $\Zz$ modulo $26$. 

\change

Cet ensemble peut par exemple être représenté par les $26$ éléments $\{0,1,2,\ldots, 25\}$. 

\change

En effet, puisque l'on compte modulo $26$, les premiers entiers sont en quelque sorte eux-mêmes
$0,\ 1,\ 2,\ \ldots,\  25,$

\change

 puis  $26\equiv 0,\  27 \equiv  1,\  28 \equiv  2,  \ldots$

\change
 
 et ainsi de suite $52\equiv 0,\  53\equiv 1, \ldots$ pour les entiers positifs

\change
 
et de même $-1\equiv 25$, $-2\equiv 24$,... pour les entiers négatifs

\change

Plus généralement $\Zz / n \Zz$ contient exactement $n$ éléments.

\change

||| Pour un entier $a \in \Zz$ quelconque, son \defi{représentant}
que nous choisissons ici dans l'ensemble $\{0,1,2,\ldots, n-1\}$ s'obtient comme le reste $k$
de la division euclidienne de $a$ par $n$ :


$a = bn + k$.

\change

Ce qui permet d'avoir $a \equiv  k \pmod n$ et $0 \le k < n$.


%%%%%%%%%%%%%%%%%%%%%%%%%%%%%%%%%%%%%%%%%%%%%%%%%%%%%%%%%%%
\diapo

De façon naturelle l'addition et la multiplication d'entiers se transposent dans $\Zz/n\Zz$.

Commençons par l'addition.

\change

Pour $a,b \in \Zz/n\Zz$, on associe $a+b \in \Zz/n\Zz$.

\change
Par exemple dans $\Zz/26\Zz$, $15+13$ égale $2$.

\change
$\equiv \equiv \equiv$ En effet $15+13 = 28 \equiv 2 \pmod{26}$.

\change
Autre exemple : que vaut $133+64$ ?

\change
$133+64=197=7\times 26 +15 \equiv 15 \pmod {26}$.

\change
Mais on pourrait procéder différemment.

Tout d'abord trouver le représentant de 133 modulo 26 :

$133 = 5 \times 26 + 3 \equiv 3 \pmod {26}$

\change
et de même $64 = 2 \times 26 + 12 \equiv 12 \pmod{26}$.

\change
Et maintenant :
$133 + 64 \equiv 3 + 12 \equiv 15 \pmod{26}$.


\change
||| On fait de même pour la multiplication :
pour $a,b \in \Zz/n\Zz$, on associe $a \times b \in \Zz/n\Zz$.

\change
$\equiv \equiv \equiv$ Par exemple $3 \times 12$ donne $36$ donc $10$ modulo $26$.

\change
Autre exemple : que vaut $3 \times 27$ ?

$3 \times 27 = 81 = 3 \times 26 + 3 \equiv 3 \pmod{26}$.

\change
Une autre façon de voir la même opération est d'écrire d'abord 
$27 \equiv 1 \pmod{26}$ puis $3 \times 27 \equiv 3 \times 1 \equiv 3 \pmod{26}$.

||| Ces exemples soulignent que les opérations d'addition et de multiplication sur les entiers 
sont bien compatibles avec les opérations d'addition et de multiplication sur les entiers modulaires.

%%%%%%%%%%%%%%%%%%%%%%%%%%%%%%%%%%%%%%%%%%%%%%%%%%%%%%%%%%%
\diapo

Le chiffrement de César peut maintenant être vu comme une simple addition dans $\Zz/26\Zz$ !

Fixons un entier $k$ qui est le décalage (nous avions choisi $k=3$ dans l'exemple) 
et définissons la  \defi{fonction de chiffrement de César de décalage $k$}
qui va de l'ensemble $\Zz/26 \Zz$ dans lui-même :

$\equiv \equiv \equiv$ cette fonction associe à $x$ le nombre $x+k$. 

Comme l'ensemble d'arrivée est $\Zz/26 \Zz$ il faut bien comprendre que cela revient à calculer  $x+k$ modulo $26$.

\change

Par exemple, pour un décalage de $3$ lettres comme précédemment : $C_3(0)=3$, $C_3(1)=4$\ldots

\change 

||| Pour déchiffrer, rien de plus simple ! Il suffit d'aller dans l'autre sens, c'est-à-dire ici de soustraire.

$\equiv \equiv \equiv$ A $x$ on associe $x-k$.

\change

En effet, si $1$ a été chiffré en $4$, par la fonction $C_3$ alors 

$D_3(4)=4-3=1$. 

||| On retrouve le nombre initial.

%%%%%%%%%%%%%%%%%%%%%%%%%%%%%%%%%%%%%%%%%%%%%%%%%%%%%%%%%%%
\diapo


Mathématiquement, $D_k$ est la bijection réciproque de $C_k$, 

\change

ce qui implique que pour tout 
$x \in \Zz/26 \Zz$ :

$\equiv \equiv \equiv$ $D_k \big( C_k(x) \big) = x$

\change

En d'autres termes, si $x$ est un nombre, on applique la fonction de chiffrement 
pour obtenir le nombre crypté $y=C_k(x)$ ; ensuite 
la fonction de déchiffrement fait bien ce que l'on attend d'elle 
$D_k(y)=x$, 

||| on retrouve le nombre initial $x$.



%%%%%%%%%%%%%%%%%%%%%%%%%%%%%%%%%%%%%%%%%%%%%%%%%%%%%%%%%%%
\diapo

Voici le principe du chiffrement : 

\change
Alice souhaite envoyer des messages secrets à Bruno.

\change
Ils se sont d'abord mis d'accord sur une clé secrète $k$, supposons par exemple que ce soit maintenant $k=11$.

\change
Le message qu'Alice souhaite envoyer est "COUCOU"

\change
$\equiv \equiv \equiv$ Elle transforme d'abord les lettres de "COUCOU" en une liste de nombres $C$ devient $2$, $O$ devient $14$ et $U$ devient $20$.

\change
Alice applique la fonction de chiffrement $C_{11}$ définie par 

$C_{11}(x)=x+11$ à chacun de ces nombres :

$2$ devient $2+11=13$, 

$14+11=25$ 

et $20+11=31 \equiv 5$ modulo 26.

\change
ce qui correspond au mot crypté "NZFNZF".

Elle transmet le mot crypté à Bruno, 


\change
qui selon le même principe 
applique la fonction de déchiffrement $D_{11}$ définie par 

$D_{11}(x)=x-11$.

Et trouvera bien sûr le message "COUCOU".

\change 

Tout ceci est représenté par le schéma que voici

la transformation de lettre en nombre, le chiffrement, la transformation de nombre en lettre


la transmission

et ici le déchiffrement.

% \change
% 
% Un exemple classique est le "rot13"  (pour rotation par un décalage de $13$) :
% $$C_{13}(x) = x +13$$
% et comme $-13=13 \pmod{26}$ alors $D_{13}(x)=x+13$.
% La fonction de déchiffrement est la même que la fonction de chiffrement !
% 
% Exemple : déchiffrez le mot "PRFNE". ["CESAR"]
% 
% \end{exemple}

%%%%%%%%%%%%%%%%%%%%%%%%%%%%%%%%%%%%%%%%%%%%%%%%%%%%%%%%%%%
\diapo

Combien existe-t-il de chiffrements de César différents ?

\change

Dans l'absolu, le paramètre $k$ de décalage peut prendre 26 va\-leurs différentes : il y a donc $26$ fonctions $C_k$ différentes, $k=0,1,\ldots,25$. 

Même si le décalage de 0 lettre n'a que peu d'intérêt crypto\-graphique !!

\change

Du point de vue des entiers ou des entiers modulaires, $k$ appartient à $\Zz / 26 \Zz$, 

\change

On retrouve ici le fait que les fonctions $C_{29}$ et $C_3$ sont identiques car $29$ est congru à $3$ modulo $26$.

\change

Le décalage $k$ s'appelle la \defi{clé de chiffrement}, c'est l'information nécessaire pour crypter le message.

\change

Il y a donc $26$ clés différentes et l'\defi{espace des clés} est
$\Zz/26\Zz$.


\change

Il est clair que ce protocole de chiffrement est d'une sécurité très faible.


Si Alice envoie un message crypté à Bruno et que Chloé intercepte ce message, 
il lui sera facile de le décrypter même sans connaître la clé de chiffrement.

En effet, Chloé peut déchiffrer le message intercepté selon les $26$ façons différentes et reconnaître parmi 
les $26$ résultats celui qui donne un texte compréhensible. 



%%%%%%%%%%%%%%%%%%%%%%%%%%%%%%%%%%%%%%%%%%%%%%%%%%%%%%%%%%%
\diapo

Les ordinateurs ont révolutionnés la cryptographie 

- tant sur le plan de l'automatisation du chiffrement des messages à envoyer 

- que sur celui des attaques multiples sur les messages chiffrés interceptés.

Nous montrons ici, à l'aide du langage Python comment programmer 
le chiffrement de César. 

Tout d'abord la fonction de chiffrement se programme en une seule ligne :

$\equiv \equiv \equiv$ Ici \codeinline{x} est un nombre de $\{0,1,\ldots,25\}$ et \codeinline{k}
est le décalage : ce sont les arguments de la fonction.

 \codeinline{(x+k)\%26} renvoie le reste modulo $26$ de la somme \codeinline{(x+k)}.

\change

||| Pour le déchiffrement, c'est aussi simple : on soustrait le décalage.




%%%%%%%%%%%%%%%%%%%%%%%%%%%%%%%%%%%%%%%%%%%%%%%%%%%%%%%%%%%
\diapo


Pour chiffrer un mot ou un phrase, il n'y a pas de problèmes théoriques, seulement 
des difficultés techniques.

~

Nous présentons ici la fonction qui crypte un mot avec un certain décalage.

~

Un mot est une chaîne de caractères (choisis parmi les 26 lettres de l'alphabet) qui en fait se comporte comme une liste.

~

$\equiv \equiv \equiv$   la boucle \codeinline{for lettre in mot:} permet de parcourir une à une les lettres du mot

  
||| Pour transformer une lettre en un nombre, on utilise la fonction \codeinline{ord} qui à chaque caractère 
  associe son code Ascii qui est un entier. Par exemple : \codeinline{ord(A)} vaut $65$, \codeinline{ord(B)} vaut $66$... 
  
~

Si nous souhaitons que la lettre A corresponde à l'entier 0, il suffit de considérer pour chaque lettre l'entier 

$\equiv \equiv \equiv$  \codeinline{(ord(lettre) - 65)} 

qui pour les lettres de $A$ à $Z$ sera un entier entre $0$ et $25$.

~

On peut maintenant appliquer la fonction de chiffrement, qui je vous le rappelle renvoie juste  nb+k modulo 26
 
~
 
La transformation des nombres en caractères se fait par la fonction \codeinline{chr}

~
  
On ajoute la lettre à la liste des lettres cryptées avec \codeinline{append}
  
Enfin on transforme cette liste en une chaîne de caractères 

~

||| qui est notre résultat : le mot crypté.
 
\end{document}

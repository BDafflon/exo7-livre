
%%%%%%%%%%%%%%%%%% PREAMBULE %%%%%%%%%%%%%%%%%%

\documentclass[aspectratio=169,utf8]{beamer}
%\documentclass[aspectratio=169,handout]{beamer}

\usetheme{Boadilla}
%\usecolortheme{seahorse}
%\usecolortheme[RGB={245,66,24}]{structure}
\useoutertheme{infolines}

% packages
\usepackage{amsfonts,amsmath,amssymb,amsthm}
\usepackage[utf8]{inputenc}
\usepackage[T1]{fontenc}
\usepackage{lmodern}

\usepackage[francais]{babel}
\usepackage{fancybox}
\usepackage{graphicx}

\usepackage{float}
\usepackage{xfrac}

%\usepackage[usenames, x11names]{xcolor}
\usepackage{pgfplots}
\usepackage{datetime}


% ----------------------------------------------------------------------
% Pour les images
\usepackage{tikz}
\usetikzlibrary{calc,shadows,arrows.meta,patterns,matrix}

\newcommand{\tikzinput}[1]{\input{figures/#1.tikz}}
% --- les figures avec échelle éventuel
\newcommand{\myfigure}[2]{% entrée : échelle, fichier(s) figure à inclure
\begin{center}\small%
\tikzstyle{every picture}=[scale=1.0*#1]% mise en échelle + 0% (automatiquement annulé à la fin du groupe)
#2%
\end{center}}



%-----  Package unités -----
\usepackage{siunitx}
\sisetup{locale = FR,detect-all,per-mode = symbol}

%\usepackage{mathptmx}
%\usepackage{fouriernc}
%\usepackage{newcent}
%\usepackage[mathcal,mathbf]{euler}

%\usepackage{palatino}
%\usepackage{newcent}
% \usepackage[mathcal,mathbf]{euler}



% \usepackage{hyperref}
% \hypersetup{colorlinks=true, linkcolor=blue, urlcolor=blue,
% pdftitle={Exo7 - Exercices de mathématiques}, pdfauthor={Exo7}}


%section
% \usepackage{sectsty}
% \allsectionsfont{\bf}
%\sectionfont{\color{Tomato3}\upshape\selectfont}
%\subsectionfont{\color{Tomato4}\upshape\selectfont}

%----- Ensembles : entiers, reels, complexes -----
\newcommand{\Nn}{\mathbb{N}} \newcommand{\N}{\mathbb{N}}
\newcommand{\Zz}{\mathbb{Z}} \newcommand{\Z}{\mathbb{Z}}
\newcommand{\Qq}{\mathbb{Q}} \newcommand{\Q}{\mathbb{Q}}
\newcommand{\Rr}{\mathbb{R}} \newcommand{\R}{\mathbb{R}}
\newcommand{\Cc}{\mathbb{C}} 
\newcommand{\Kk}{\mathbb{K}} \newcommand{\K}{\mathbb{K}}

%----- Modifications de symboles -----
\renewcommand{\epsilon}{\varepsilon}
\renewcommand{\Re}{\mathop{\text{Re}}\nolimits}
\renewcommand{\Im}{\mathop{\text{Im}}\nolimits}
%\newcommand{\llbracket}{\left[\kern-0.15em\left[}
%\newcommand{\rrbracket}{\right]\kern-0.15em\right]}

\renewcommand{\ge}{\geqslant}
\renewcommand{\geq}{\geqslant}
\renewcommand{\le}{\leqslant}
\renewcommand{\leq}{\leqslant}
\renewcommand{\epsilon}{\varepsilon}

%----- Fonctions usuelles -----
\newcommand{\ch}{\mathop{\text{ch}}\nolimits}
\newcommand{\sh}{\mathop{\text{sh}}\nolimits}
\renewcommand{\tanh}{\mathop{\text{th}}\nolimits}
\newcommand{\cotan}{\mathop{\text{cotan}}\nolimits}
\newcommand{\Arcsin}{\mathop{\text{arcsin}}\nolimits}
\newcommand{\Arccos}{\mathop{\text{arccos}}\nolimits}
\newcommand{\Arctan}{\mathop{\text{arctan}}\nolimits}
\newcommand{\Argsh}{\mathop{\text{argsh}}\nolimits}
\newcommand{\Argch}{\mathop{\text{argch}}\nolimits}
\newcommand{\Argth}{\mathop{\text{argth}}\nolimits}
\newcommand{\pgcd}{\mathop{\text{pgcd}}\nolimits} 


%----- Commandes divers ------
\newcommand{\ii}{\mathrm{i}}
\newcommand{\dd}{\text{d}}
\newcommand{\id}{\mathop{\text{id}}\nolimits}
\newcommand{\Ker}{\mathop{\text{Ker}}\nolimits}
\newcommand{\Card}{\mathop{\text{Card}}\nolimits}
\newcommand{\Vect}{\mathop{\text{Vect}}\nolimits}
\newcommand{\Mat}{\mathop{\text{Mat}}\nolimits}
\newcommand{\rg}{\mathop{\text{rg}}\nolimits}
\newcommand{\tr}{\mathop{\text{tr}}\nolimits}


%----- Structure des exercices ------

\newtheoremstyle{styleexo}% name
{2ex}% Space above
{3ex}% Space below
{}% Body font
{}% Indent amount 1
{\bfseries} % Theorem head font
{}% Punctuation after theorem head
{\newline}% Space after theorem head 2
{}% Theorem head spec (can be left empty, meaning ‘normal’)

%\theoremstyle{styleexo}
\newtheorem{exo}{Exercice}
\newtheorem{ind}{Indications}
\newtheorem{cor}{Correction}


\newcommand{\exercice}[1]{} \newcommand{\finexercice}{}
%\newcommand{\exercice}[1]{{\tiny\texttt{#1}}\vspace{-2ex}} % pour afficher le numero absolu, l'auteur...
\newcommand{\enonce}{\begin{exo}} \newcommand{\finenonce}{\end{exo}}
\newcommand{\indication}{\begin{ind}} \newcommand{\finindication}{\end{ind}}
\newcommand{\correction}{\begin{cor}} \newcommand{\fincorrection}{\end{cor}}

\newcommand{\noindication}{\stepcounter{ind}}
\newcommand{\nocorrection}{\stepcounter{cor}}

\newcommand{\fiche}[1]{} \newcommand{\finfiche}{}
\newcommand{\titre}[1]{\centerline{\large \bf #1}}
\newcommand{\addcommand}[1]{}
\newcommand{\video}[1]{}

% Marge
\newcommand{\mymargin}[1]{\marginpar{{\small #1}}}

\def\noqed{\renewcommand{\qedsymbol}{}}


%----- Presentation ------
\setlength{\parindent}{0cm}

%\newcommand{\ExoSept}{\href{http://exo7.emath.fr}{\textbf{\textsf{Exo7}}}}

\definecolor{myred}{rgb}{0.93,0.26,0}
\definecolor{myorange}{rgb}{0.97,0.58,0}
\definecolor{myyellow}{rgb}{1,0.86,0}

\newcommand{\LogoExoSept}[1]{  % input : echelle
{\usefont{U}{cmss}{bx}{n}
\begin{tikzpicture}[scale=0.1*#1,transform shape]
  \fill[color=myorange] (0,0)--(4,0)--(4,-4)--(0,-4)--cycle;
  \fill[color=myred] (0,0)--(0,3)--(-3,3)--(-3,0)--cycle;
  \fill[color=myyellow] (4,0)--(7,4)--(3,7)--(0,3)--cycle;
  \node[scale=5] at (3.5,3.5) {Exo7};
\end{tikzpicture}}
}


\newcommand{\debutmontitre}{
  \author{} \date{} 
  \thispagestyle{empty}
  \hspace*{-10ex}
  \begin{minipage}{\textwidth}
    \titlepage  
  \vspace*{-2.5cm}
  \begin{center}
    \LogoExoSept{2.5}
  \end{center}
  \end{minipage}

  \vspace*{-0cm}
  
  % Astuce pour que le background ne soit pas discrétisé lors de la conversion pdf -> png
\begin{tikzpicture}
        \fill[opacity=0,green!60!black] (0,0)--++(0,0)--++(0,0)--++(0,0)--cycle; 
\end{tikzpicture}

% toc S'affiche trop tot :
% \tableofcontents[hideallsubsections, pausesections]
}

\newcommand{\finmontitre}{
  \end{frame}
  \setcounter{framenumber}{0}
} % ne marche pas pour une raison obscure

%----- Commandes supplementaires ------

% \usepackage[landscape]{geometry}
% \geometry{top=1cm, bottom=3cm, left=2cm, right=10cm, marginparsep=1cm
% }
% \usepackage[a4paper]{geometry}
% \geometry{top=2cm, bottom=2cm, left=2cm, right=2cm, marginparsep=1cm
% }

%\usepackage{standalone}


% New command Arnaud -- november 2011
\setbeamersize{text margin left=24ex}
% si vous modifier cette valeur il faut aussi
% modifier le decalage du titre pour compenser
% (ex : ici =+10ex, titre =-5ex

\theoremstyle{definition}
%\newtheorem{proposition}{Proposition}
%\newtheorem{exemple}{Exemple}
%\newtheorem{theoreme}{Théorème}
%\newtheorem{lemme}{Lemme}
%\newtheorem{corollaire}{Corollaire}
%\newtheorem*{remarque*}{Remarque}
%\newtheorem*{miniexercice}{Mini-exercices}
%\newtheorem{definition}{Définition}

% Commande tikz
\usetikzlibrary{calc}
\usetikzlibrary{patterns,arrows}
\usetikzlibrary{matrix}
\usetikzlibrary{fadings} 

%definition d'un terme
\newcommand{\defi}[1]{{\color{myorange}\textbf{\emph{#1}}}}
\newcommand{\evidence}[1]{{\color{blue}\textbf{\emph{#1}}}}
\newcommand{\assertion}[1]{\emph{\og#1\fg}}  % pour chapitre logique
%\renewcommand{\contentsname}{Sommaire}
\renewcommand{\contentsname}{}
\setcounter{tocdepth}{2}



%------ Encadrement ------

\usepackage{fancybox}


\newcommand{\mybox}[1]{
\setlength{\fboxsep}{7pt}
\begin{center}
\shadowbox{#1}
\end{center}}

\newcommand{\myboxinline}[1]{
\setlength{\fboxsep}{5pt}
\raisebox{-10pt}{
\shadowbox{#1}
}
}

%--------------- Commande beamer---------------
\newcommand{\beameronly}[1]{#1} % permet de mettre des pause dans beamer pas dans poly


\setbeamertemplate{navigation symbols}{}
\setbeamertemplate{footline}  % tiré du fichier beamerouterinfolines.sty
{
  \leavevmode%
  \hbox{%
  \begin{beamercolorbox}[wd=.333333\paperwidth,ht=2.25ex,dp=1ex,center]{author in head/foot}%
    % \usebeamerfont{author in head/foot}\insertshortauthor%~~(\insertshortinstitute)
    \usebeamerfont{section in head/foot}{\bf\insertshorttitle}
  \end{beamercolorbox}%
  \begin{beamercolorbox}[wd=.333333\paperwidth,ht=2.25ex,dp=1ex,center]{title in head/foot}%
    \usebeamerfont{section in head/foot}{\bf\insertsectionhead}
  \end{beamercolorbox}%
  \begin{beamercolorbox}[wd=.333333\paperwidth,ht=2.25ex,dp=1ex,right]{date in head/foot}%
    % \usebeamerfont{date in head/foot}\insertshortdate{}\hspace*{2em}
    \insertframenumber{} / \inserttotalframenumber\hspace*{2ex} 
  \end{beamercolorbox}}%
  \vskip0pt%
}


\definecolor{mygrey}{rgb}{0.5,0.5,0.5}
\setlength{\parindent}{0cm}
%\DeclareTextFontCommand{\helvetica}{\fontfamily{phv}\selectfont}

% background beamer
\definecolor{couleurhaut}{rgb}{0.85,0.9,1}  % creme
\definecolor{couleurmilieu}{rgb}{1,1,1}  % vert pale
\definecolor{couleurbas}{rgb}{0.85,0.9,1}  % blanc
\setbeamertemplate{background canvas}[vertical shading]%
[top=couleurhaut,middle=couleurmilieu,midpoint=0.4,bottom=couleurbas] 
%[top=fondtitre!05,bottom=fondtitre!60]



\makeatletter
\setbeamertemplate{theorem begin}
{%
  \begin{\inserttheoremblockenv}
  {%
    \inserttheoremheadfont
    \inserttheoremname
    \inserttheoremnumber
    \ifx\inserttheoremaddition\@empty\else\ (\inserttheoremaddition)\fi%
    \inserttheorempunctuation
  }%
}
\setbeamertemplate{theorem end}{\end{\inserttheoremblockenv}}

\newenvironment{theoreme}[1][]{%
   \setbeamercolor{block title}{fg=structure,bg=structure!40}
   \setbeamercolor{block body}{fg=black,bg=structure!10}
   \begin{block}{{\bf Th\'eor\`eme }#1}
}{%
   \end{block}%
}


\newenvironment{proposition}[1][]{%
   \setbeamercolor{block title}{fg=structure,bg=structure!40}
   \setbeamercolor{block body}{fg=black,bg=structure!10}
   \begin{block}{{\bf Proposition }#1}
}{%
   \end{block}%
}

\newenvironment{corollaire}[1][]{%
   \setbeamercolor{block title}{fg=structure,bg=structure!40}
   \setbeamercolor{block body}{fg=black,bg=structure!10}
   \begin{block}{{\bf Corollaire }#1}
}{%
   \end{block}%
}

\newenvironment{mydefinition}[1][]{%
   \setbeamercolor{block title}{fg=structure,bg=structure!40}
   \setbeamercolor{block body}{fg=black,bg=structure!10}
   \begin{block}{{\bf Définition} #1}
}{%
   \end{block}%
}

\newenvironment{lemme}[0]{%
   \setbeamercolor{block title}{fg=structure,bg=structure!40}
   \setbeamercolor{block body}{fg=black,bg=structure!10}
   \begin{block}{\bf Lemme}
}{%
   \end{block}%
}

\newenvironment{remarque}[1][]{%
   \setbeamercolor{block title}{fg=black,bg=structure!20}
   \setbeamercolor{block body}{fg=black,bg=structure!5}
   \begin{block}{Remarque #1}
}{%
   \end{block}%
}


\newenvironment{exemple}[1][]{%
   \setbeamercolor{block title}{fg=black,bg=structure!20}
   \setbeamercolor{block body}{fg=black,bg=structure!5}
   \begin{block}{{\bf Exemple }#1}
}{%
   \end{block}%
}


\newenvironment{miniexercice}[0]{%
   \setbeamercolor{block title}{fg=structure,bg=structure!20}
   \setbeamercolor{block body}{fg=black,bg=structure!5}
   \begin{block}{Mini-exercices}
}{%
   \end{block}%
}


\newenvironment{tp}[0]{%
   \setbeamercolor{block title}{fg=structure,bg=structure!40}
   \setbeamercolor{block body}{fg=black,bg=structure!10}
   \begin{block}{\bf Travaux pratiques}
}{%
   \end{block}%
}
\newenvironment{exercicecours}[1][]{%
   \setbeamercolor{block title}{fg=structure,bg=structure!40}
   \setbeamercolor{block body}{fg=black,bg=structure!10}
   \begin{block}{{\bf Exercice }#1}
}{%
   \end{block}%
}
\newenvironment{algo}[1][]{%
   \setbeamercolor{block title}{fg=structure,bg=structure!40}
   \setbeamercolor{block body}{fg=black,bg=structure!10}
   \begin{block}{{\bf Algorithme}\hfill{\color{gray}\texttt{#1}}}
}{%
   \end{block}%
}


\setbeamertemplate{proof begin}{
   \setbeamercolor{block title}{fg=black,bg=structure!20}
   \setbeamercolor{block body}{fg=black,bg=structure!5}
   \begin{block}{{\footnotesize Démonstration}}
   \footnotesize
   \smallskip}
\setbeamertemplate{proof end}{%
   \end{block}}
\setbeamertemplate{qed symbol}{\openbox}


\makeatother
\usecolortheme[RGB={66,0,245}]{structure}

%%%%%%%%%%%%%%%%%%%%%%%%%%%%%%%%%%%%%%%%%%%%%%%%%%%%%%%%%%%%%
%%%%%%%%%%%%%%%%%%%%%%%%%%%%%%%%%%%%%%%%%%%%%%%%%%%%%%%%%%%%%


\begin{document}

\title{{\bf Leçons de choses}}
\subtitle{Formules de trigonométrie : sinus, cosinus, tangente}

\begin{frame}
  
  \debutmontitre

%   \pause
% 
% {\footnotesize
% \hfill
% \setbeamercovered{transparent=50}
% \begin{minipage}{0.6\textwidth}
%   \begin{itemize}
%     \item<3-> Définition
%     \item<4-> Exemples
%     \item<5-> Puissance
%     \item<6-> Matrices $2\times 2$
%   \end{itemize}
% \end{minipage}
% }

\end{frame}

\setcounter{framenumber}{0}



%%%%%%%%%%%%%%%%%%%%%%%%%%%%%%%%%%%%%%%%%%%%%%%%%%%%%%%%%%%%%%%%


\section{Le cercle trigonométrique}


% \setbeamertemplate{background canvas}[vertical shading]%
% [top=white,middle=white,midpoint=0.4,bottom=white] 
\begin{frame}
\myfigure{0.8}{
\tikzinput{fig_lecons07ter}
}
\end{frame}
% \setbeamertemplate{background canvas}[vertical shading]%
% [top=couleurhaut,middle=couleurmilieu,midpoint=0.4,bottom=couleurbas] 

\begin{frame}

\hspace*{-2em}
\begin{minipage}{0.7\textwidth}
\myfigure{0.8}{
\tikzinput{fig_lecons02}
}
\end{minipage}
\pause\pause
\begin{minipage}{0.29\textwidth}
\begin{align*}
& \cos^2 x + \sin^2 x = 1 \\
& \cos(x+2\pi)=\cos x \\
& \sin(x+2\pi)=\sin x \\
\end{align*}  

\end{minipage}
\end{frame}


\begin{frame}

\begin{minipage}{0.60\textwidth}
\myfigure{1}{
\tikzinput{fig_lecons06}
}
\end{minipage}
\begin{minipage}{0.39\textwidth}
\begin{align*}
\cos (-x) &= \cos x \\
\sin (-x) &= -\sin x \\  
\end{align*} 
\end{minipage}
\end{frame}

\begin{frame}
\begin{minipage}{0.32\textwidth}
\begin{align*}
\cos (\pi + x) &= -\cos x \\
\sin (\pi + x) &= -\sin x \\  
\end{align*}  
\end{minipage}
\begin{minipage}{0.32\textwidth}
\begin{align*}
\cos (\pi - x) &= -\cos x \\
\sin (\pi - x) &= \sin x \\  
\end{align*}  
\end{minipage}
\begin{minipage}{0.32\textwidth}
\begin{align*}
\cos (\frac\pi2 - x) &= \sin x \\
\sin (\frac\pi2 - x) &= \cos x \\  
\end{align*}  
\end{minipage}
\myfigure{0.5}{
\tikzinput{fig_lecons04bis}\quad
\tikzinput{fig_lecons03bis}\quad 
\tikzinput{fig_lecons05bis}
}

  
\end{frame}


\begin{frame}
\vfil
\begin{minipage}{0.5\textwidth}
{
\renewcommand{\arraystretch}{2.5}
\footnotesize
$$
\begin{array}{c|*{5}{c}}
   x      & \  0 \ & \ \dfrac\pi6 \ & \ \ \dfrac\pi 4\  \ 
& \ \dfrac \pi 3\  &\  \dfrac \pi 2\  \\
\hline
\cos x  \ & 1 & \dfrac{\sqrt3}{2} & \dfrac{\sqrt2}{2} & \dfrac12 & 0 \\
\hline
\sin x  \ & 0 &\dfrac{1}{2} & \dfrac{\sqrt2}{2} & \dfrac{\sqrt3}{2} & 1\\
\hline
\tan x  \ & 0 & \dfrac{1}{\sqrt{3}} & 1 & \sqrt{3} & 
\end{array}
$$
}   
\end{minipage}
\begin{minipage}{0.49\textwidth}
\myfigure{0.95}{
\tikzinput{fig_lecons08}
}  
\end{minipage}




\end{frame}




%--------------------------------------------------------
\section{Les fonctions sinus, cosinus, tangente}

\begin{frame}

\myfigure{0.5}{
\tikzinput{fig_lecons10}
}

\pause

\myfigure{1.2}{
\tikzinput{fig_lecons11}
}

\pause
\vspace*{-3ex}
$$\cos'x= -\sin x \qquad \sin'x=\cos x$$
\end{frame}


\begin{frame}
\vspace*{2ex}
\hspace*{-2em}$\tan x = \dfrac{\sin x}{\cos x}$, 
{\footnotesize $x \notin\{\ldots, -\frac\pi2, \frac\pi2, \frac{3\pi}{2}, \frac{5\pi}{2},\ldots  \}$}
\quad  $\tan' x = 1+\tan^2x=\dfrac{1}{\cos^2x}$


\myfigure{0.6}{
\tikzinput{fig_lecons12}
}  

\end{frame}









%--------------------------------------------------------
\section{Les formules d'additions}


\begin{frame}
\begin{align*}
\cos(a+b) &= \cos a \cdot \cos b - \sin a \cdot \sin b \\
\sin(a+b) &= \sin a\cdot \cos b  +  \sin b\cdot\cos a \\
\tan (a+b) &=\dfrac{\tan a + \tan b}{1-\tan a \cdot \tan b}\\
\end{align*}
\pause\vspace*{-3ex}
\begin{align*}
\cos 2a &= 2\cos^2a-1\\
    &= 1-2\sin^2a\\
    &=\cos^2a-\sin^2a\\[3mm]
\sin 2a &= 2\sin a\cdot \cos a\\[3mm]
\tan 2a &= \frac{2\tan a}{1-\tan^2 a}
\end{align*}  
\end{frame}



%--------------------------------------------------------
\section{Les autres formules}


\begin{frame}

\begin{align*}
\cos a\cdot\cos b &= \frac{1}{2}\big[ \cos(a+b)+\cos(a-b)\big]\\
\sin a\cdot\sin b &= \frac{1}{2}\big[ \cos(a-b)-\cos(a+b)\big]\\
\sin a\cdot\cos b &= \frac{1}{2}\big[ \sin(a+b)+\sin(a-b)\big]
\end{align*}

\pause
\bigskip

\begin{align*}
\cos p+\cos q &= 2\cos \frac{p+q}{2}\cdot\cos\frac{p-q}{2}\\
\sin p+\sin q &= 2\sin \frac{p+q}{2}\cdot\cos\frac{p-q}{2}\\
\end{align*}  
\end{frame}


\begin{frame}
Formules de la <<tangente de l'arc moitié>>
\begin{align*}
  t&=\tan \frac{x}{2}\quad \implies \quad 
\begin{cases}
    \cos x = \frac {1-t^2}{1+t^2} \\
    \sin x = \frac{2t}{1+t^2} \\
    \tan x = \frac{2t}{1-t^2} \\
\end{cases}
\end{align*}

\bigskip

Changement de variable pour les intégrales  $dx=\dfrac{2\,dt}{1+t^2}$
\end{frame}






%--------------------------------------------------------
\section{Mini-exercices}


\begin{frame}
\begin{miniexercice}
\begin{enumerate}
  \item Montrer que $1+\tan^2x=\frac{1}{\cos^2x}$.
  \item Montrer la formule d'addition de $\tan(a+b)$.
  \item Prouver la formule pour $\cos a\cdot\cos b$.
  \item Prouver la formule pour $\cos p+\cos q$.
  \item Prouver la formule : $\sin x = \dfrac{2\tan \frac{x}{2}}{1+(\tan \frac{x}{2})^2}$.
  \item Montrer que $\cos \frac{\pi}{8}= \frac 12\sqrt{ \sqrt{2} + 2}$. Calculer $\cos \frac{\pi}{16}$,
$\cos \frac{\pi}{32}$,\ldots
  \item Exprimer $\cos(3x)$ en fonction $\cos x$ ; $\sin(3x)$ en fonction $\sin x$ ;
$\tan(3x)$ en fonction $\tan x$.
\end{enumerate}
\end{miniexercice}
\end{frame}




\end{document}
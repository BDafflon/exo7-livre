
%%%%%%%%%%%%%%%%%% PREAMBULE %%%%%%%%%%%%%%%%%%


\documentclass[12pt]{article}

\usepackage{amsfonts,amsmath,amssymb,amsthm}
\usepackage[utf8]{inputenc}
\usepackage[T1]{fontenc}
\usepackage[francais]{babel}


% packages
\usepackage{amsfonts,amsmath,amssymb,amsthm}
\usepackage[utf8]{inputenc}
\usepackage[T1]{fontenc}
%\usepackage{lmodern}

\usepackage[francais]{babel}
\usepackage{fancybox}
\usepackage{graphicx}

\usepackage{float}

%\usepackage[usenames, x11names]{xcolor}
\usepackage{tikz}
\usepackage{datetime}

\usepackage{mathptmx}
%\usepackage{fouriernc}
%\usepackage{newcent}
\usepackage[mathcal,mathbf]{euler}

%\usepackage{palatino}
%\usepackage{newcent}


% Commande spéciale prompteur

%\usepackage{mathptmx}
%\usepackage[mathcal,mathbf]{euler}
%\usepackage{mathpple,multido}

\usepackage[a4paper]{geometry}
\geometry{top=2cm, bottom=2cm, left=1cm, right=1cm, marginparsep=1cm}

\newcommand{\change}{{\color{red}\rule{\textwidth}{1mm}\\}}

\newcounter{mydiapo}

\newcommand{\diapo}{\newpage
\hfill {\normalsize  Diapo \themydiapo \quad \texttt{[\jobname]}} \\
\stepcounter{mydiapo}}


%%%%%%% COULEURS %%%%%%%%%%

% Pour blanc sur noir :
%\pagecolor[rgb]{0.5,0.5,0.5}
% \pagecolor[rgb]{0,0,0}
% \color[rgb]{1,1,1}



%\DeclareFixedFont{\myfont}{U}{cmss}{bx}{n}{18pt}
\newcommand{\debuttexte}{
%%%%%%%%%%%%% FONTES %%%%%%%%%%%%%
\renewcommand{\baselinestretch}{1.5}
\usefont{U}{cmss}{bx}{n}
\bfseries

% Taille normale : commenter le reste !
%Taille Arnaud
%\fontsize{19}{19}\selectfont

% Taille Barbara
%\fontsize{21}{22}\selectfont

%Taille François
%\fontsize{25}{30}\selectfont

%Taille Pascal
%\fontsize{25}{30}\selectfont

%Taille Laura
%\fontsize{30}{35}\selectfont


%\myfont
%\usefont{U}{cmss}{bx}{n}

%\Huge
%\addtolength{\parskip}{\baselineskip}
}


% \usepackage{hyperref}
% \hypersetup{colorlinks=true, linkcolor=blue, urlcolor=blue,
% pdftitle={Exo7 - Exercices de mathématiques}, pdfauthor={Exo7}}


%section
% \usepackage{sectsty}
% \allsectionsfont{\bf}
%\sectionfont{\color{Tomato3}\upshape\selectfont}
%\subsectionfont{\color{Tomato4}\upshape\selectfont}

%----- Ensembles : entiers, reels, complexes -----
\newcommand{\Nn}{\mathbb{N}} \newcommand{\N}{\mathbb{N}}
\newcommand{\Zz}{\mathbb{Z}} \newcommand{\Z}{\mathbb{Z}}
\newcommand{\Qq}{\mathbb{Q}} \newcommand{\Q}{\mathbb{Q}}
\newcommand{\Rr}{\mathbb{R}} \newcommand{\R}{\mathbb{R}}
\newcommand{\Cc}{\mathbb{C}} 
\newcommand{\Kk}{\mathbb{K}} \newcommand{\K}{\mathbb{K}}

%----- Modifications de symboles -----
\renewcommand{\epsilon}{\varepsilon}
\renewcommand{\Re}{\mathop{\text{Re}}\nolimits}
\renewcommand{\Im}{\mathop{\text{Im}}\nolimits}
%\newcommand{\llbracket}{\left[\kern-0.15em\left[}
%\newcommand{\rrbracket}{\right]\kern-0.15em\right]}

\renewcommand{\ge}{\geqslant}
\renewcommand{\geq}{\geqslant}
\renewcommand{\le}{\leqslant}
\renewcommand{\leq}{\leqslant}

%----- Fonctions usuelles -----
\newcommand{\ch}{\mathop{\mathrm{ch}}\nolimits}
\newcommand{\sh}{\mathop{\mathrm{sh}}\nolimits}
\renewcommand{\tanh}{\mathop{\mathrm{th}}\nolimits}
\newcommand{\cotan}{\mathop{\mathrm{cotan}}\nolimits}
\newcommand{\Arcsin}{\mathop{\mathrm{Arcsin}}\nolimits}
\newcommand{\Arccos}{\mathop{\mathrm{Arccos}}\nolimits}
\newcommand{\Arctan}{\mathop{\mathrm{Arctan}}\nolimits}
\newcommand{\Argsh}{\mathop{\mathrm{Argsh}}\nolimits}
\newcommand{\Argch}{\mathop{\mathrm{Argch}}\nolimits}
\newcommand{\Argth}{\mathop{\mathrm{Argth}}\nolimits}
\newcommand{\pgcd}{\mathop{\mathrm{pgcd}}\nolimits} 

\newcommand{\Card}{\mathop{\text{Card}}\nolimits}
\newcommand{\Ker}{\mathop{\text{Ker}}\nolimits}
\newcommand{\id}{\mathop{\text{id}}\nolimits}
\newcommand{\ii}{\mathrm{i}}
\newcommand{\dd}{\mathrm{d}}
\newcommand{\Vect}{\mathop{\text{Vect}}\nolimits}
\newcommand{\Mat}{\mathop{\mathrm{Mat}}\nolimits}
\newcommand{\rg}{\mathop{\text{rg}}\nolimits}
\newcommand{\tr}{\mathop{\text{tr}}\nolimits}
\newcommand{\ppcm}{\mathop{\text{ppcm}}\nolimits}

%----- Structure des exercices ------

\newtheoremstyle{styleexo}% name
{2ex}% Space above
{3ex}% Space below
{}% Body font
{}% Indent amount 1
{\bfseries} % Theorem head font
{}% Punctuation after theorem head
{\newline}% Space after theorem head 2
{}% Theorem head spec (can be left empty, meaning ‘normal’)

%\theoremstyle{styleexo}
\newtheorem{exo}{Exercice}
\newtheorem{ind}{Indications}
\newtheorem{cor}{Correction}


\newcommand{\exercice}[1]{} \newcommand{\finexercice}{}
%\newcommand{\exercice}[1]{{\tiny\texttt{#1}}\vspace{-2ex}} % pour afficher le numero absolu, l'auteur...
\newcommand{\enonce}{\begin{exo}} \newcommand{\finenonce}{\end{exo}}
\newcommand{\indication}{\begin{ind}} \newcommand{\finindication}{\end{ind}}
\newcommand{\correction}{\begin{cor}} \newcommand{\fincorrection}{\end{cor}}

\newcommand{\noindication}{\stepcounter{ind}}
\newcommand{\nocorrection}{\stepcounter{cor}}

\newcommand{\fiche}[1]{} \newcommand{\finfiche}{}
\newcommand{\titre}[1]{\centerline{\large \bf #1}}
\newcommand{\addcommand}[1]{}
\newcommand{\video}[1]{}

% Marge
\newcommand{\mymargin}[1]{\marginpar{{\small #1}}}



%----- Presentation ------
\setlength{\parindent}{0cm}

%\newcommand{\ExoSept}{\href{http://exo7.emath.fr}{\textbf{\textsf{Exo7}}}}

\definecolor{myred}{rgb}{0.93,0.26,0}
\definecolor{myorange}{rgb}{0.97,0.58,0}
\definecolor{myyellow}{rgb}{1,0.86,0}

\newcommand{\LogoExoSept}[1]{  % input : echelle
{\usefont{U}{cmss}{bx}{n}
\begin{tikzpicture}[scale=0.1*#1,transform shape]
  \fill[color=myorange] (0,0)--(4,0)--(4,-4)--(0,-4)--cycle;
  \fill[color=myred] (0,0)--(0,3)--(-3,3)--(-3,0)--cycle;
  \fill[color=myyellow] (4,0)--(7,4)--(3,7)--(0,3)--cycle;
  \node[scale=5] at (3.5,3.5) {Exo7};
\end{tikzpicture}}
}



\theoremstyle{definition}
%\newtheorem{proposition}{Proposition}
%\newtheorem{exemple}{Exemple}
%\newtheorem{theoreme}{Théorème}
\newtheorem{lemme}{Lemme}
\newtheorem{corollaire}{Corollaire}
%\newtheorem*{remarque*}{Remarque}
%\newtheorem*{miniexercice}{Mini-exercices}
%\newtheorem{definition}{Définition}




%definition d'un terme
\newcommand{\defi}[1]{{\color{myorange}\textbf{\emph{#1}}}}
\newcommand{\evidence}[1]{{\color{blue}\textbf{\emph{#1}}}}



 %----- Commandes divers ------

\newcommand{\codeinline}[1]{\texttt{#1}}

%%%%%%%%%%%%%%%%%%%%%%%%%%%%%%%%%%%%%%%%%%%%%%%%%%%%%%%%%%%%%
%%%%%%%%%%%%%%%%%%%%%%%%%%%%%%%%%%%%%%%%%%%%%%%%%%%%%%%%%%%%%



\begin{document}

\debuttexte

%%%%%%%%%%%%%%%%%%%%%%%%%%%%%%%%%%%%%%%%%%%%%%%%%%%%%%%%%%%
\diapo


Dans cette leçon nous allons revoir les formulaires de trigonométrie classique
pour les fonctions sinus, cosinus, tangente, dites fonctions ``circulaires'' et nous allons 
les comparer aux formules pour les fonctions hyperboliques : cosinus hyperbolique, sinus hyperbolique
et tangente hyperbolique.

\change

Rappelons en premier lieu la formule fondamentale 
$\cos^2 x + \sin^2 x = 1$.


\change

Passons aux formules d'addition

tout d'abord 

$\cos(a+b)=\cos a\cdot\cos b - \sin a\cdot\sin b$

\change

puis 
$\sin(a+b)=\sin a\cdot\cos b  +  \sin b\cdot\cos a$

et enfin 

$\tan (a+b)=\frac{\tan a + \tan b}{1-\tan a\cdot\tan b}$

Vous devez déjà connaître toutes ces formules.


\change

Voyons leur contrepartie hyperbolique.


La formule fondamentale s'écrit maintenant
$\ch^2 x - \sh^2 x = 1$

\change

Quant aux formules d'additions elles deviennent :

$\ch(a+b)=\ch a\cdot\ch b + \sh a\cdot\sh b$

\change

$\sh(a+b)=\sh a\cdot\ch b  +  \sh b\cdot\ch a$

\change

$\tanh (a+b)=\frac{\tanh a + \tanh b}{1+\tanh a\cdot\tanh b}$



%%%%%%%%%%%%%%%%%%%%%%%%%%%%%%%%%%%%%%%%%%%%%%%%%%%%%%%%%%%
\diapo

Vous avez bien sûr noté qu'il y a une grande similitude entre les formules pour
cosinus, sinus, tangente et leur version hyperbolique.

Je vais vous montrer une façon de passer des formules classiques aux formules hyperboliques.

\change

A chaque fois que cosinus apparaît dans la formule nous le remplaçons par cosinus hyperbolique.

Alors qu'à chaque fois qu'apparaît sinus nous le remplaçons par $i$ fois sinus hyperbolique.

Ici $i$ est le nombre complexe avec $i^2=-1$.

\change

Voyons un exemple.

Partons de la formule $\cos^2 x + \sin^2 x = 1$.

\change

On remplace $\cos^2 x$ par $\ch^2 x$.

Et on remplace $\sin^2 x$
par $(i\sh x)^2$

\change

Mais $i^2=-1$ donc la formule que l'on obtient est 
bien $\ch^2 x - \sh^2 x = 1$.


\change

Faisons la même chose avec la formule pour $\cos(a+b)$.

\change

On remplace les $\cos$ par $\ch$ et les $\sin$ par $i\sh$.

\change

Encore une fois $i^2=-1$ et l'on obtient la formule :

$\ch(a+b)=\ch a\cdot\ch b + \sh a\cdot\sh b$.



\change

Dernier exemple avec la formule d'addition des sinus.

\change

Lorsque l'on effectue la substitution.
On obtient cette égalité.

Dans laquelle on simplifie par $i$ de part et d'autre de l'égalité

\change

afin d'obtenir la formule correcte.
 

%%%%%%%%%%%%%%%%%%%%%%%%%%%%%%%%%%%%%%%%%%%%%%%%%%%%%%%%%%%
\diapo

Une fois que l'on connaît les formules d'addition on en déduit immédiatement les formules
pour $\cos(2a)$, $\sin(2a)$, etc.

Revoyons les rapidement :

Tout d'abord $\cos 2a = 2\,\cos^2a-1$

\change

mais aussi $\cos 2a =1-2\,\sin^2a$

\change

ou encore $\cos 2a =\cos^2a-\sin^2a$.

\change

Ensuite $\sin 2a = 2\,\sin a\cdot\cos a$

\change

Et enfin
$\tan 2a = \displaystyle\frac{2\,\tan a}{1-\tan^2 a}$.

\change

Passons aux formules hyperboliques.

$\ch 2a = 2\,\ch^2a-1$

qui vaut aussi 
$1+2\,\sh^2a$

ou $\ch^2a+\sh^2a$.


\change

$\sh 2a = 2\,\sh a\cdot\ch a$.

\change

$\tanh 2a = \displaystyle\frac{2\,\tanh a}{1+\tanh^2 a}$


De nouveau vous passez des formules classiques aux formules hyperboliques
en remplaçant $\cos$ par $\ch$ et $\sin$ par $i\sh$.


%%%%%%%%%%%%%%%%%%%%%%%%%%%%%%%%%%%%%%%%%%%%%%%%%%%%%%%%%%%
\diapo

Voici ensuite les formules pour les produits de deux sinus ou deux cosinus.

Par exemple 
$\cos a\cdot\cos b = \frac{1}{2}\,\big[ \cos(a+b)+\cos(a-b)\big]$
Cette formule se déduit de la formule
de $\cos(a+b)$ et $\cos(a-b)$.

\change

Il y a aussi les formules pour la sommes de deux cosinus ou deux sinus.
Par exemple 
$\cos p+\cos q = 2\,\cos \frac{p+q}{2}\cdot\cos\frac{p-q}{2}$.


\change

Pour les cosinus et sinus hyperboliques il existe des formules similaires.

\change

Qui se retrouvent aussi en effectuant la substitution vue précédemment.


%%%%%%%%%%%%%%%%%%%%%%%%%%%%%%%%%%%%%%%%%%%%%%%%%%%%%%%%%%%
\diapo

Voici les formules de la <<tangente de l'arc moitié>>.

En posant $t=\tan \frac{x}{2}$ alors 
$\cos x$ s'écrit $\frac {1-t^2}{1+t^2}$

$\sin x = \frac{2t}{1+t^2}$

$\tan x = \frac{2t}{1-t^2}$


\change

Et pour la version hyperbolique on pose
$t=\tanh \frac{x}{2}$ et alors on a :

$\ch x = \frac {1+t^2}{1-t^2}$

$\sh x = \frac{2t}{1-t^2}$

$\tanh x = \frac{2t}{1+t^2}$.


%%%%%%%%%%%%%%%%%%%%%%%%%%%%%%%%%%%%%%%%%%%%%%%%%%%%%%%%%%%
\diapo

On termine par les formules pour les dérivées.

Bien sûr vous savez que 

$\cos'x= -\sin x$

$\sin'x=\cos x$

$\tan' x = 1+\tan^2x=\frac{1}{\cos^2x}$.

\change

Nous allons voir les dérivées de fonctions hyperboliques.
Mais attention pour les dérivées on ne peut plus passer
d'une formule à l'autre par la substitution.

\change

Les formules sont :

$\ch'x= \sh x$

$\sh'x=\ch x$

$\tanh' x = 1-\tanh^2x=\frac{1}{\ch^2x}$

\change

On conclue avec les dérivées des fonctions réciproques.

$\text{Arccos}'x=\frac{-1}{\sqrt{1-x^2}}$

$\text{Arcsin}'x=\frac{1}{\sqrt{1-x^2}}$

$\text{Arctan}'x=\frac{1}{1+x^2}$

\change

Et 

$\text{Argch}'x=\frac{1}{\sqrt{x^2-1}}$


$\text{Argsh}'x=\frac{1}{\sqrt{x^2+1}}$

$\text{Argth}'x=\frac{1}{1-x^2}$


\end{document}
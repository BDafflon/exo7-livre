
%%%%%%%%%%%%%%%%%% PREAMBULE %%%%%%%%%%%%%%%%%%


\documentclass[12pt]{article}

\usepackage{amsfonts,amsmath,amssymb,amsthm}
\usepackage[utf8]{inputenc}
\usepackage[T1]{fontenc}
\usepackage[francais]{babel}


% packages
\usepackage{amsfonts,amsmath,amssymb,amsthm}
\usepackage[utf8]{inputenc}
\usepackage[T1]{fontenc}
%\usepackage{lmodern}

\usepackage[francais]{babel}
\usepackage{fancybox}
\usepackage{graphicx}

\usepackage{float}

%\usepackage[usenames, x11names]{xcolor}
\usepackage{tikz}
\usepackage{datetime}

\usepackage{mathptmx}
%\usepackage{fouriernc}
%\usepackage{newcent}
\usepackage[mathcal,mathbf]{euler}

%\usepackage{palatino}
%\usepackage{newcent}


% Commande spéciale prompteur

%\usepackage{mathptmx}
%\usepackage[mathcal,mathbf]{euler}
%\usepackage{mathpple,multido}

\usepackage[a4paper]{geometry}
\geometry{top=2cm, bottom=2cm, left=1cm, right=1cm, marginparsep=1cm}

\newcommand{\change}{{\color{red}\rule{\textwidth}{1mm}\\}}

\newcounter{mydiapo}

\newcommand{\diapo}{\newpage
\hfill {\normalsize  Diapo \themydiapo \quad \texttt{[\jobname]}} \\
\stepcounter{mydiapo}}


%%%%%%% COULEURS %%%%%%%%%%

% Pour blanc sur noir :
%\pagecolor[rgb]{0.5,0.5,0.5}
% \pagecolor[rgb]{0,0,0}
% \color[rgb]{1,1,1}



%\DeclareFixedFont{\myfont}{U}{cmss}{bx}{n}{18pt}
\newcommand{\debuttexte}{
%%%%%%%%%%%%% FONTES %%%%%%%%%%%%%
\renewcommand{\baselinestretch}{1.5}
\usefont{U}{cmss}{bx}{n}
\bfseries

% Taille normale : commenter le reste !
%Taille Arnaud
%\fontsize{19}{19}\selectfont

% Taille Barbara
%\fontsize{21}{22}\selectfont

%Taille François
%\fontsize{25}{30}\selectfont

%Taille Pascal
%\fontsize{25}{30}\selectfont

%Taille Laura
%\fontsize{30}{35}\selectfont


%\myfont
%\usefont{U}{cmss}{bx}{n}

%\Huge
%\addtolength{\parskip}{\baselineskip}
}


% \usepackage{hyperref}
% \hypersetup{colorlinks=true, linkcolor=blue, urlcolor=blue,
% pdftitle={Exo7 - Exercices de mathématiques}, pdfauthor={Exo7}}


%section
% \usepackage{sectsty}
% \allsectionsfont{\bf}
%\sectionfont{\color{Tomato3}\upshape\selectfont}
%\subsectionfont{\color{Tomato4}\upshape\selectfont}

%----- Ensembles : entiers, reels, complexes -----
\newcommand{\Nn}{\mathbb{N}} \newcommand{\N}{\mathbb{N}}
\newcommand{\Zz}{\mathbb{Z}} \newcommand{\Z}{\mathbb{Z}}
\newcommand{\Qq}{\mathbb{Q}} \newcommand{\Q}{\mathbb{Q}}
\newcommand{\Rr}{\mathbb{R}} \newcommand{\R}{\mathbb{R}}
\newcommand{\Cc}{\mathbb{C}} 
\newcommand{\Kk}{\mathbb{K}} \newcommand{\K}{\mathbb{K}}

%----- Modifications de symboles -----
\renewcommand{\epsilon}{\varepsilon}
\renewcommand{\Re}{\mathop{\text{Re}}\nolimits}
\renewcommand{\Im}{\mathop{\text{Im}}\nolimits}
%\newcommand{\llbracket}{\left[\kern-0.15em\left[}
%\newcommand{\rrbracket}{\right]\kern-0.15em\right]}

\renewcommand{\ge}{\geqslant}
\renewcommand{\geq}{\geqslant}
\renewcommand{\le}{\leqslant}
\renewcommand{\leq}{\leqslant}

%----- Fonctions usuelles -----
\newcommand{\ch}{\mathop{\mathrm{ch}}\nolimits}
\newcommand{\sh}{\mathop{\mathrm{sh}}\nolimits}
\renewcommand{\tanh}{\mathop{\mathrm{th}}\nolimits}
\newcommand{\cotan}{\mathop{\mathrm{cotan}}\nolimits}
\newcommand{\Arcsin}{\mathop{\mathrm{Arcsin}}\nolimits}
\newcommand{\Arccos}{\mathop{\mathrm{Arccos}}\nolimits}
\newcommand{\Arctan}{\mathop{\mathrm{Arctan}}\nolimits}
\newcommand{\Argsh}{\mathop{\mathrm{Argsh}}\nolimits}
\newcommand{\Argch}{\mathop{\mathrm{Argch}}\nolimits}
\newcommand{\Argth}{\mathop{\mathrm{Argth}}\nolimits}
\newcommand{\pgcd}{\mathop{\mathrm{pgcd}}\nolimits} 

\newcommand{\Card}{\mathop{\text{Card}}\nolimits}
\newcommand{\Ker}{\mathop{\text{Ker}}\nolimits}
\newcommand{\id}{\mathop{\text{id}}\nolimits}
\newcommand{\ii}{\mathrm{i}}
\newcommand{\dd}{\mathrm{d}}
\newcommand{\Vect}{\mathop{\text{Vect}}\nolimits}
\newcommand{\Mat}{\mathop{\mathrm{Mat}}\nolimits}
\newcommand{\rg}{\mathop{\text{rg}}\nolimits}
\newcommand{\tr}{\mathop{\text{tr}}\nolimits}
\newcommand{\ppcm}{\mathop{\text{ppcm}}\nolimits}

%----- Structure des exercices ------

\newtheoremstyle{styleexo}% name
{2ex}% Space above
{3ex}% Space below
{}% Body font
{}% Indent amount 1
{\bfseries} % Theorem head font
{}% Punctuation after theorem head
{\newline}% Space after theorem head 2
{}% Theorem head spec (can be left empty, meaning ‘normal’)

%\theoremstyle{styleexo}
\newtheorem{exo}{Exercice}
\newtheorem{ind}{Indications}
\newtheorem{cor}{Correction}


\newcommand{\exercice}[1]{} \newcommand{\finexercice}{}
%\newcommand{\exercice}[1]{{\tiny\texttt{#1}}\vspace{-2ex}} % pour afficher le numero absolu, l'auteur...
\newcommand{\enonce}{\begin{exo}} \newcommand{\finenonce}{\end{exo}}
\newcommand{\indication}{\begin{ind}} \newcommand{\finindication}{\end{ind}}
\newcommand{\correction}{\begin{cor}} \newcommand{\fincorrection}{\end{cor}}

\newcommand{\noindication}{\stepcounter{ind}}
\newcommand{\nocorrection}{\stepcounter{cor}}

\newcommand{\fiche}[1]{} \newcommand{\finfiche}{}
\newcommand{\titre}[1]{\centerline{\large \bf #1}}
\newcommand{\addcommand}[1]{}
\newcommand{\video}[1]{}

% Marge
\newcommand{\mymargin}[1]{\marginpar{{\small #1}}}



%----- Presentation ------
\setlength{\parindent}{0cm}

%\newcommand{\ExoSept}{\href{http://exo7.emath.fr}{\textbf{\textsf{Exo7}}}}

\definecolor{myred}{rgb}{0.93,0.26,0}
\definecolor{myorange}{rgb}{0.97,0.58,0}
\definecolor{myyellow}{rgb}{1,0.86,0}

\newcommand{\LogoExoSept}[1]{  % input : echelle
{\usefont{U}{cmss}{bx}{n}
\begin{tikzpicture}[scale=0.1*#1,transform shape]
  \fill[color=myorange] (0,0)--(4,0)--(4,-4)--(0,-4)--cycle;
  \fill[color=myred] (0,0)--(0,3)--(-3,3)--(-3,0)--cycle;
  \fill[color=myyellow] (4,0)--(7,4)--(3,7)--(0,3)--cycle;
  \node[scale=5] at (3.5,3.5) {Exo7};
\end{tikzpicture}}
}



\theoremstyle{definition}
%\newtheorem{proposition}{Proposition}
%\newtheorem{exemple}{Exemple}
%\newtheorem{theoreme}{Théorème}
\newtheorem{lemme}{Lemme}
\newtheorem{corollaire}{Corollaire}
%\newtheorem*{remarque*}{Remarque}
%\newtheorem*{miniexercice}{Mini-exercices}
%\newtheorem{definition}{Définition}




%definition d'un terme
\newcommand{\defi}[1]{{\color{myorange}\textbf{\emph{#1}}}}
\newcommand{\evidence}[1]{{\color{blue}\textbf{\emph{#1}}}}



 %----- Commandes divers ------

\newcommand{\codeinline}[1]{\texttt{#1}}

%%%%%%%%%%%%%%%%%%%%%%%%%%%%%%%%%%%%%%%%%%%%%%%%%%%%%%%%%%%%%
%%%%%%%%%%%%%%%%%%%%%%%%%%%%%%%%%%%%%%%%%%%%%%%%%%%%%%%%%%%%%


\begin{document}

\debuttexte

%%%%%%%%%%%%%%%%%%%%%%%%%%%%%%%%%%%%%%%%%%%%%%%%%%%%%%%%%%%
\diapo

\change

Nous commençons ce chapitre sur les polynômes en mettant en place toutes
les notions :

\change

nous commençons par des définitions

\change

nous continuons avec l'addition, et la multiplication de polynômes

\change

Et on termine encore avec du vocabulaire. 


%%%%%%%%%%%%%%%%%%%%%%%%%%%%%%%%%%%%%%%%%%%%%%%%%%%%%%%%%%%
\diapo

Les polynômes sont des objets très simples mais aux propriétés extrêmement riches.
Vous savez déjà résoudre les équations de degré $2$ : $aX^2+bX+c=0$.

\change

Savez-vous que la résolution des équations de degré $3$, 
$aX^3+bX^2+cX+d=0$, a fait l'objet de luttes acharnées dans l'Italie 
du \textsc{\romannumeral 16}\textsuperscript{e} siècle ?
Un concours était organisé avec un prix pour chacune de trente équations 
de degré $3$ à résoudre.

\change

Un jeune italien, Tartaglia, trouve la formule générale des solutions
et résout les trente équations en une seule nuit ! Cette méthode que Tartaglia voulait garder secrète

\change

sera quand même publiée quelques années plus tard comme la \og méthode de Cardan\fg.

\change


Dans ces leçons sur les polynômes, après quelques définitions des concepts de base, 

nous allons étudier l'arithmétique des polynômes.

On continuera avec le théorème fondamental de l'algèbre :
\og Tout polynôme de degré $n$ admet $n$ racines complexes. \fg \ 

On terminera avec les fractions rationnelles.


%%%%%%%%%%%%%%%%%%%%%%%%%%%%%%%%%%%%%%%%%%%%%%%%%%%%%%%%%%%
\diapo

Un \defi{polynôme}  est une expression de la forme
$$P(X) = a_n X^n + a_{n-1} X^{n-1} + \cdots + a_2 X^2 + a_1 X + a_0$$


\change

$a_0,a_1,\ldots,a_n$ sont des éléments d'un corps $\Kk$.

Ces $a_i$ sont les \defi{coefficients} du polynôme.

\change

Pour nous le corps $\Kk$ sera soit $\Qq$ (le corps des nombres rationnels), 
soit $\Rr$ (le corps des nombres réels) ou 
$\Cc$ (le corps des nombres complexes).

\change

Un polynôme important est le polynôme nul, c'est le polynôme dont tous les coefficients sont nuls.

On le note $0$.


\change

Enfin $\Kk[X]$ ($K$ crochet $X$)

désigne l'ensemble de tous les polynômes $P(X)$.




%%%%%%%%%%%%%%%%%%%%%%%%%%%%%%%%%%%%%%%%%%%%%%%%%%%%%%%%%%%
\diapo


Le \defi{degré} d'un polynôme $P$ est le plus grand entier $i$ tel que $a_i\not=0$.

On note le degré par $\deg P$. 

\change


Voyons quelques cas particulier : 

Un polynôme constant est un polynôme de la forme $P=a_0$ (c-a-d les coefficient $a_1$, $a_2$,... sont tous nuls)

le degré d'un polynôme constant (non nul) est $0$

\change

Pour le degré du polynôme nul on pose par convention $\deg(0)=-\infty$.
(certains préfèrent dire que le degré du polynôme nul n'est pas défini).


\change

Il n'y a pas de difficultés :

 $X^3-5X+\frac 34$ est un polynôme de degré $3$.
 
 $X^n+1$ est un polynôme de degré $n$.
 
 $2$ est un polynôme constant, de degré $0$.  
 
 
%%%%%%%%%%%%%%%%%%%%%%%%%%%%%%%%%%%%%%%%%%%%%%%%%%%%%%%%%%%
\diapo

Partant d'un polynôme $P$ et d'un polynôme $Q$

On dit que le polynôme $P$ *égale* le polynôme $Q$ si
ils ont les mêmes coefficients.

C'est à dire $a_0=b_0$, $a_1=b_1$, $a_2=b_2$, etc.


\change

Repartant de deux polynômes $P$, $Q$ la somme $P+Q$ est un polynôme.
Le $i$-ème coefficient de $P+Q$ s'obtient en additionnant le $i$-ème coefficient de $P$ plus 
le $i$-ème coefficient de $Q$.


\change

La définition de la multiplication est plus délicate :

Partons d'un polynôme $P$ de degré $n$ avec des coeff. $a_i$

et d'un polynômes $Q$ de degré $m$ avec des coeff $b_j$.

Alors  $P \times Q$ est par définition 

le polynômes dont les coeff sont les $c_k$.

pour $k$ variant de $0$ à $n+m$.

Avec $c_k$ qui est la somme des $a_i \times b_j$ pour tous les indices $i,j$ tels que $i+j=k$.

\change

On termine par la multiplication d'un polynôme $P$ par un nombre $\lambda$.

 $\lambda \cdot P$ est le polynôme dont le $i$-ème coefficient
est $\lambda a_i$.

%%%%%%%%%%%%%%%%%%%%%%%%%%%%%%%%%%%%%%%%%%%%%%%%%%%%%%%%%%%
\diapo

Voyons ces opérations sur un exemple.


\change

La somme $P+Q$ se calcule ainsi :

le coefficient constant est la somme des deux coeff cst : c'est $d+ \gamma$.

Le coefficient de $X$  c'est $c+\beta$,

Pour le coeff de $X^2$ c'est $b+\alpha$

Pour le coeff de $X^3$ c'est seulement $a$ (car le coeff de $X^3$de $Q$ est nul).

\change

Passons à la multiplication : 

la formule de la multiplication est faites pour que le produit s'obtienne
en développant de la façon usuelle avec la relation $X^i \times X^j = X^{i+j}$.

Par exemple comment calcule-t-on le coeff de $X^3$.
Pour arriver au degré $3$ on peut faire
$3+0$ donc coeff de $X^3$ fois coeff de $X^0$ : $a \times \gamma$.

On a aussi $2+1=3$ donc $ b \times \beta$

et aussi $1+2$ donc $c \times \alpha$

On pourrait aussi faire $0+3$ mais comme le coeff de $X^3$ de $Q$ est nul alors cela fait $0$.


\change

Enfin quand est-ce que $P=Q$ ?

Il faut que les coeff de $P$ et $Q$ soient égaux

donc 

$P=Q$ si et seulement si $d=\gamma$ , $c=\beta$, $b=\alpha$,  et enfin $a=0$.


%%%%%%%%%%%%%%%%%%%%%%%%%%%%%%%%%%%%%%%%%%%%%%%%%%%%%%%%%%%
\diapo

L'addition et la multiplication se comportent sans problème :
 
 $0+P=P$ (le polynôme nul est l’élément neutre pour l'addition), 
 
 $P+Q=Q+P$ (l'addition est commutative),
 
 $(P+Q)+R=P+(Q+R)$ (l'addition est associative).
 
 \change
 
 Même chose pour la multiplication :
 
 $1\cdot P = P$(le polynôme cst = 1 est l’élément neutre pour la multiplication)
 
 $P\times Q=Q \times P$ (la multiplication est commutative)
 
 $(P \times Q) \times R=P \times (Q \times R)$ (la multiplication est associative)
 
 \change
 
 Enfin la multiplication est distributive vis à vis de l'addition :
 
 $P\times (Q+R)=P\times Q + P \times R$.

\change

Pour le degré il faut faire attention :
$\deg(P\times Q)=\deg P + \deg Q$

le degré d'un produit est la *somme* des degrés.


\change

Enfin pour la somme le $\deg(P+Q)$ est inférieur ou égal au maximum
de $\deg P, \deg Q$.

Attention il n'y a pas toujours égalité car le terme de haut plus degré de 
$P$ peut s'annuler avec le terme de plus haut degré de $Q$.

\change

On note [R indice n] $\Rr_n[X]$ l'ensemble des polynômes dont le degré est inférieur
ou égal à $n$.

Cette propriété [montrer degré] que 
si $P,Q \in \Rr_n[X]$ alors $P+Q$ est aussi dans  $\Rr_n[X]$.






%%%%%%%%%%%%%%%%%%%%%%%%%%%%%%%%%%%%%%%%%%%%%%%%%%%%%%%%%%%
\diapo

Complétons les définitions sur les polynômes.

 Les polynômes comportant un seul terme non nul (du type $a_kX^k$) sont
appelés \defi{monômes}.

Tout polynôme est donc une somme finie de monômes. 

\change
Considérons le polynômes $P=a_nX^n+a_{n-1}X^{n-1}+\cdots + a_1X+a_0,$ 

qui est vraiment de degré $n$ c-a-d que $a_n$ est non nul.

\change

On appelle \defi{terme dominant} le monôme $a_nX^n$. 

\change

Le coefficient $a_n$ est appelé le \defi{coefficient dominant} de $P$. 

\change

Si le coefficient dominant est $1$, on dit que $P$ est un \defi{polynôme
unitaire}.

\change

Considérons l'exemple de 

$P(X)=(X-1)(X^n+X^{n-1}+\cdots + X+1)$.

\change

On développe cette expression :
$X$ fois le second facteur

plus
$-1$ fois le second facteur.

Presque tout se simplifie,

il reste $X^{n+1} - 1$.

\change

$P(X)$ est donc un polynôme de degré $n+1$, il est unitaire et 
est somme de deux monômes : $X^{n+1}$ et $-1$.


%%%%%%%%%%%%%%%%%%%%%%%%%%%%%%%%%%%%%%%%%%%%%%%%%%%%%%%%%%%
\diapo


Voici quelques exercices simples pour commencer !




\end{document}

%%%%%%%%%%%%%%%%%% PREAMBULE %%%%%%%%%%%%%%%%%%

\documentclass[aspectratio=169,utf8]{beamer}
%\documentclass[aspectratio=169,handout]{beamer}

\usetheme{Boadilla}
%\usecolortheme{seahorse}
%\usecolortheme[RGB={245,66,24}]{structure}
\useoutertheme{infolines}

% packages
\usepackage{amsfonts,amsmath,amssymb,amsthm}
\usepackage[utf8]{inputenc}
\usepackage[T1]{fontenc}
\usepackage{lmodern}

\usepackage[francais]{babel}
\usepackage{fancybox}
\usepackage{graphicx}

\usepackage{float}
\usepackage{xfrac}

%\usepackage[usenames, x11names]{xcolor}
\usepackage{pgfplots}
\usepackage{datetime}


% ----------------------------------------------------------------------
% Pour les images
\usepackage{tikz}
\usetikzlibrary{calc,shadows,arrows.meta,patterns,matrix}

\newcommand{\tikzinput}[1]{\input{figures/#1.tikz}}
% --- les figures avec échelle éventuel
\newcommand{\myfigure}[2]{% entrée : échelle, fichier(s) figure à inclure
\begin{center}\small%
\tikzstyle{every picture}=[scale=1.0*#1]% mise en échelle + 0% (automatiquement annulé à la fin du groupe)
#2%
\end{center}}



%-----  Package unités -----
\usepackage{siunitx}
\sisetup{locale = FR,detect-all,per-mode = symbol}

%\usepackage{mathptmx}
%\usepackage{fouriernc}
%\usepackage{newcent}
%\usepackage[mathcal,mathbf]{euler}

%\usepackage{palatino}
%\usepackage{newcent}
% \usepackage[mathcal,mathbf]{euler}



% \usepackage{hyperref}
% \hypersetup{colorlinks=true, linkcolor=blue, urlcolor=blue,
% pdftitle={Exo7 - Exercices de mathématiques}, pdfauthor={Exo7}}


%section
% \usepackage{sectsty}
% \allsectionsfont{\bf}
%\sectionfont{\color{Tomato3}\upshape\selectfont}
%\subsectionfont{\color{Tomato4}\upshape\selectfont}

%----- Ensembles : entiers, reels, complexes -----
\newcommand{\Nn}{\mathbb{N}} \newcommand{\N}{\mathbb{N}}
\newcommand{\Zz}{\mathbb{Z}} \newcommand{\Z}{\mathbb{Z}}
\newcommand{\Qq}{\mathbb{Q}} \newcommand{\Q}{\mathbb{Q}}
\newcommand{\Rr}{\mathbb{R}} \newcommand{\R}{\mathbb{R}}
\newcommand{\Cc}{\mathbb{C}} 
\newcommand{\Kk}{\mathbb{K}} \newcommand{\K}{\mathbb{K}}

%----- Modifications de symboles -----
\renewcommand{\epsilon}{\varepsilon}
\renewcommand{\Re}{\mathop{\text{Re}}\nolimits}
\renewcommand{\Im}{\mathop{\text{Im}}\nolimits}
%\newcommand{\llbracket}{\left[\kern-0.15em\left[}
%\newcommand{\rrbracket}{\right]\kern-0.15em\right]}

\renewcommand{\ge}{\geqslant}
\renewcommand{\geq}{\geqslant}
\renewcommand{\le}{\leqslant}
\renewcommand{\leq}{\leqslant}
\renewcommand{\epsilon}{\varepsilon}

%----- Fonctions usuelles -----
\newcommand{\ch}{\mathop{\text{ch}}\nolimits}
\newcommand{\sh}{\mathop{\text{sh}}\nolimits}
\renewcommand{\tanh}{\mathop{\text{th}}\nolimits}
\newcommand{\cotan}{\mathop{\text{cotan}}\nolimits}
\newcommand{\Arcsin}{\mathop{\text{arcsin}}\nolimits}
\newcommand{\Arccos}{\mathop{\text{arccos}}\nolimits}
\newcommand{\Arctan}{\mathop{\text{arctan}}\nolimits}
\newcommand{\Argsh}{\mathop{\text{argsh}}\nolimits}
\newcommand{\Argch}{\mathop{\text{argch}}\nolimits}
\newcommand{\Argth}{\mathop{\text{argth}}\nolimits}
\newcommand{\pgcd}{\mathop{\text{pgcd}}\nolimits} 


%----- Commandes divers ------
\newcommand{\ii}{\mathrm{i}}
\newcommand{\dd}{\text{d}}
\newcommand{\id}{\mathop{\text{id}}\nolimits}
\newcommand{\Ker}{\mathop{\text{Ker}}\nolimits}
\newcommand{\Card}{\mathop{\text{Card}}\nolimits}
\newcommand{\Vect}{\mathop{\text{Vect}}\nolimits}
\newcommand{\Mat}{\mathop{\text{Mat}}\nolimits}
\newcommand{\rg}{\mathop{\text{rg}}\nolimits}
\newcommand{\tr}{\mathop{\text{tr}}\nolimits}


%----- Structure des exercices ------

\newtheoremstyle{styleexo}% name
{2ex}% Space above
{3ex}% Space below
{}% Body font
{}% Indent amount 1
{\bfseries} % Theorem head font
{}% Punctuation after theorem head
{\newline}% Space after theorem head 2
{}% Theorem head spec (can be left empty, meaning ‘normal’)

%\theoremstyle{styleexo}
\newtheorem{exo}{Exercice}
\newtheorem{ind}{Indications}
\newtheorem{cor}{Correction}


\newcommand{\exercice}[1]{} \newcommand{\finexercice}{}
%\newcommand{\exercice}[1]{{\tiny\texttt{#1}}\vspace{-2ex}} % pour afficher le numero absolu, l'auteur...
\newcommand{\enonce}{\begin{exo}} \newcommand{\finenonce}{\end{exo}}
\newcommand{\indication}{\begin{ind}} \newcommand{\finindication}{\end{ind}}
\newcommand{\correction}{\begin{cor}} \newcommand{\fincorrection}{\end{cor}}

\newcommand{\noindication}{\stepcounter{ind}}
\newcommand{\nocorrection}{\stepcounter{cor}}

\newcommand{\fiche}[1]{} \newcommand{\finfiche}{}
\newcommand{\titre}[1]{\centerline{\large \bf #1}}
\newcommand{\addcommand}[1]{}
\newcommand{\video}[1]{}

% Marge
\newcommand{\mymargin}[1]{\marginpar{{\small #1}}}

\def\noqed{\renewcommand{\qedsymbol}{}}


%----- Presentation ------
\setlength{\parindent}{0cm}

%\newcommand{\ExoSept}{\href{http://exo7.emath.fr}{\textbf{\textsf{Exo7}}}}

\definecolor{myred}{rgb}{0.93,0.26,0}
\definecolor{myorange}{rgb}{0.97,0.58,0}
\definecolor{myyellow}{rgb}{1,0.86,0}

\newcommand{\LogoExoSept}[1]{  % input : echelle
{\usefont{U}{cmss}{bx}{n}
\begin{tikzpicture}[scale=0.1*#1,transform shape]
  \fill[color=myorange] (0,0)--(4,0)--(4,-4)--(0,-4)--cycle;
  \fill[color=myred] (0,0)--(0,3)--(-3,3)--(-3,0)--cycle;
  \fill[color=myyellow] (4,0)--(7,4)--(3,7)--(0,3)--cycle;
  \node[scale=5] at (3.5,3.5) {Exo7};
\end{tikzpicture}}
}


\newcommand{\debutmontitre}{
  \author{} \date{} 
  \thispagestyle{empty}
  \hspace*{-10ex}
  \begin{minipage}{\textwidth}
    \titlepage  
  \vspace*{-2.5cm}
  \begin{center}
    \LogoExoSept{2.5}
  \end{center}
  \end{minipage}

  \vspace*{-0cm}
  
  % Astuce pour que le background ne soit pas discrétisé lors de la conversion pdf -> png
\begin{tikzpicture}
        \fill[opacity=0,green!60!black] (0,0)--++(0,0)--++(0,0)--++(0,0)--cycle; 
\end{tikzpicture}

% toc S'affiche trop tot :
% \tableofcontents[hideallsubsections, pausesections]
}

\newcommand{\finmontitre}{
  \end{frame}
  \setcounter{framenumber}{0}
} % ne marche pas pour une raison obscure

%----- Commandes supplementaires ------

% \usepackage[landscape]{geometry}
% \geometry{top=1cm, bottom=3cm, left=2cm, right=10cm, marginparsep=1cm
% }
% \usepackage[a4paper]{geometry}
% \geometry{top=2cm, bottom=2cm, left=2cm, right=2cm, marginparsep=1cm
% }

%\usepackage{standalone}


% New command Arnaud -- november 2011
\setbeamersize{text margin left=24ex}
% si vous modifier cette valeur il faut aussi
% modifier le decalage du titre pour compenser
% (ex : ici =+10ex, titre =-5ex

\theoremstyle{definition}
%\newtheorem{proposition}{Proposition}
%\newtheorem{exemple}{Exemple}
%\newtheorem{theoreme}{Théorème}
%\newtheorem{lemme}{Lemme}
%\newtheorem{corollaire}{Corollaire}
%\newtheorem*{remarque*}{Remarque}
%\newtheorem*{miniexercice}{Mini-exercices}
%\newtheorem{definition}{Définition}

% Commande tikz
\usetikzlibrary{calc}
\usetikzlibrary{patterns,arrows}
\usetikzlibrary{matrix}
\usetikzlibrary{fadings} 

%definition d'un terme
\newcommand{\defi}[1]{{\color{myorange}\textbf{\emph{#1}}}}
\newcommand{\evidence}[1]{{\color{blue}\textbf{\emph{#1}}}}
\newcommand{\assertion}[1]{\emph{\og#1\fg}}  % pour chapitre logique
%\renewcommand{\contentsname}{Sommaire}
\renewcommand{\contentsname}{}
\setcounter{tocdepth}{2}



%------ Encadrement ------

\usepackage{fancybox}


\newcommand{\mybox}[1]{
\setlength{\fboxsep}{7pt}
\begin{center}
\shadowbox{#1}
\end{center}}

\newcommand{\myboxinline}[1]{
\setlength{\fboxsep}{5pt}
\raisebox{-10pt}{
\shadowbox{#1}
}
}

%--------------- Commande beamer---------------
\newcommand{\beameronly}[1]{#1} % permet de mettre des pause dans beamer pas dans poly


\setbeamertemplate{navigation symbols}{}
\setbeamertemplate{footline}  % tiré du fichier beamerouterinfolines.sty
{
  \leavevmode%
  \hbox{%
  \begin{beamercolorbox}[wd=.333333\paperwidth,ht=2.25ex,dp=1ex,center]{author in head/foot}%
    % \usebeamerfont{author in head/foot}\insertshortauthor%~~(\insertshortinstitute)
    \usebeamerfont{section in head/foot}{\bf\insertshorttitle}
  \end{beamercolorbox}%
  \begin{beamercolorbox}[wd=.333333\paperwidth,ht=2.25ex,dp=1ex,center]{title in head/foot}%
    \usebeamerfont{section in head/foot}{\bf\insertsectionhead}
  \end{beamercolorbox}%
  \begin{beamercolorbox}[wd=.333333\paperwidth,ht=2.25ex,dp=1ex,right]{date in head/foot}%
    % \usebeamerfont{date in head/foot}\insertshortdate{}\hspace*{2em}
    \insertframenumber{} / \inserttotalframenumber\hspace*{2ex} 
  \end{beamercolorbox}}%
  \vskip0pt%
}


\definecolor{mygrey}{rgb}{0.5,0.5,0.5}
\setlength{\parindent}{0cm}
%\DeclareTextFontCommand{\helvetica}{\fontfamily{phv}\selectfont}

% background beamer
\definecolor{couleurhaut}{rgb}{0.85,0.9,1}  % creme
\definecolor{couleurmilieu}{rgb}{1,1,1}  % vert pale
\definecolor{couleurbas}{rgb}{0.85,0.9,1}  % blanc
\setbeamertemplate{background canvas}[vertical shading]%
[top=couleurhaut,middle=couleurmilieu,midpoint=0.4,bottom=couleurbas] 
%[top=fondtitre!05,bottom=fondtitre!60]



\makeatletter
\setbeamertemplate{theorem begin}
{%
  \begin{\inserttheoremblockenv}
  {%
    \inserttheoremheadfont
    \inserttheoremname
    \inserttheoremnumber
    \ifx\inserttheoremaddition\@empty\else\ (\inserttheoremaddition)\fi%
    \inserttheorempunctuation
  }%
}
\setbeamertemplate{theorem end}{\end{\inserttheoremblockenv}}

\newenvironment{theoreme}[1][]{%
   \setbeamercolor{block title}{fg=structure,bg=structure!40}
   \setbeamercolor{block body}{fg=black,bg=structure!10}
   \begin{block}{{\bf Th\'eor\`eme }#1}
}{%
   \end{block}%
}


\newenvironment{proposition}[1][]{%
   \setbeamercolor{block title}{fg=structure,bg=structure!40}
   \setbeamercolor{block body}{fg=black,bg=structure!10}
   \begin{block}{{\bf Proposition }#1}
}{%
   \end{block}%
}

\newenvironment{corollaire}[1][]{%
   \setbeamercolor{block title}{fg=structure,bg=structure!40}
   \setbeamercolor{block body}{fg=black,bg=structure!10}
   \begin{block}{{\bf Corollaire }#1}
}{%
   \end{block}%
}

\newenvironment{mydefinition}[1][]{%
   \setbeamercolor{block title}{fg=structure,bg=structure!40}
   \setbeamercolor{block body}{fg=black,bg=structure!10}
   \begin{block}{{\bf Définition} #1}
}{%
   \end{block}%
}

\newenvironment{lemme}[0]{%
   \setbeamercolor{block title}{fg=structure,bg=structure!40}
   \setbeamercolor{block body}{fg=black,bg=structure!10}
   \begin{block}{\bf Lemme}
}{%
   \end{block}%
}

\newenvironment{remarque}[1][]{%
   \setbeamercolor{block title}{fg=black,bg=structure!20}
   \setbeamercolor{block body}{fg=black,bg=structure!5}
   \begin{block}{Remarque #1}
}{%
   \end{block}%
}


\newenvironment{exemple}[1][]{%
   \setbeamercolor{block title}{fg=black,bg=structure!20}
   \setbeamercolor{block body}{fg=black,bg=structure!5}
   \begin{block}{{\bf Exemple }#1}
}{%
   \end{block}%
}


\newenvironment{miniexercice}[0]{%
   \setbeamercolor{block title}{fg=structure,bg=structure!20}
   \setbeamercolor{block body}{fg=black,bg=structure!5}
   \begin{block}{Mini-exercices}
}{%
   \end{block}%
}


\newenvironment{tp}[0]{%
   \setbeamercolor{block title}{fg=structure,bg=structure!40}
   \setbeamercolor{block body}{fg=black,bg=structure!10}
   \begin{block}{\bf Travaux pratiques}
}{%
   \end{block}%
}
\newenvironment{exercicecours}[1][]{%
   \setbeamercolor{block title}{fg=structure,bg=structure!40}
   \setbeamercolor{block body}{fg=black,bg=structure!10}
   \begin{block}{{\bf Exercice }#1}
}{%
   \end{block}%
}
\newenvironment{algo}[1][]{%
   \setbeamercolor{block title}{fg=structure,bg=structure!40}
   \setbeamercolor{block body}{fg=black,bg=structure!10}
   \begin{block}{{\bf Algorithme}\hfill{\color{gray}\texttt{#1}}}
}{%
   \end{block}%
}


\setbeamertemplate{proof begin}{
   \setbeamercolor{block title}{fg=black,bg=structure!20}
   \setbeamercolor{block body}{fg=black,bg=structure!5}
   \begin{block}{{\footnotesize Démonstration}}
   \footnotesize
   \smallskip}
\setbeamertemplate{proof end}{%
   \end{block}}
\setbeamertemplate{qed symbol}{\openbox}


\makeatother
\usecolortheme[RGB={142,35,35}]{structure}

%%%%%%%%%%%%%%%%%%%%%%%%%%%%%%%%%%%%%%%%%%%%%%%%%%%%%%%%%%%%%
%%%%%%%%%%%%%%%%%%%%%%%%%%%%%%%%%%%%%%%%%%%%%%%%%%%%%%%%%%%%%

\begin{document}


\title{{\bf Polynômes}}
\subtitle{Racine d'un polynôme, factorisation}

\begin{frame}
  
  \debutmontitre

  \pause

{\footnotesize
\hfill
\setbeamercovered{transparent=50}
\begin{minipage}{0.6\textwidth}
  \begin{itemize}
    \item<3-> Racines d'un polynôme
    \item<4-> Théorème de d'Alembert-Gauss
    \item<5-> Polynômes irréductibles
    \item<6-> Théorème de factorisation
    \item<7-> Factorisation dans $\Cc[X]$ et $\Rr[X]$   
  \end{itemize}
\end{minipage}
}

\end{frame}

\setcounter{framenumber}{0}


%%%%%%%%%%%%%%%%%%%%%%%%%%%%%%%%%%%%%%%%%%%%%%%%%%%%%%%%%%%%%%%%



%---------------------------------------------------------------
\section{Racines d'un polynôme}

\begin{frame}

Soit $P\in\Kk[X]$ et $\alpha\in\Kk$
\begin{mydefinition}
\centerline{$\alpha$ est une \defi{racine} (ou un \defi{zéro}) de $P$ si $P(\alpha)=0$}
\end{mydefinition}

\bigskip
\pause


\begin{proposition}
\centerline{$P(\alpha)=0 \quad \iff \quad X-\alpha \text{ divise } P$}
\end{proposition}
\pause

\begin{proof}
\begin{itemize}
  \item Division euclidienne de $P$ par $X-\alpha$ 
\pause
  \item $P=Q\cdot(X-\alpha)+R$ \pause où $\deg R < \deg (X-\alpha) =1$
\pause
  \item $R$ est constant
\pause  
  \item $P(\alpha)=0 \iff R(\alpha) =0 \pause \iff R=0 \pause \iff P=Q\cdot(X-\alpha) \pause \iff X-\alpha | P$
\end{itemize}
\end{proof}
\end{frame}


\begin{frame}
\begin{mydefinition}
\begin{itemize}
  \item $\alpha$ est une \defi{racine de multiplicité $k$}
 de $P$ si $(X-\alpha )^k$ divise $P$ alors que $(X- \alpha )^{k+1}$ ne divise pas $P$
\pause
  \item Lorsque $k=1$ on parle d'une \defi{racine simple}
\pause 
  \item Lorsque $k=2$ d'une \defi{racine double}
\end{itemize}
\end{mydefinition}

\pause

\begin{proposition}
\label{prop:racmul}
Il y a équivalence entre :
\begin{itemize}
  \item[(i)] $\alpha$ est une racine de multiplicité $k$ de $P$
\pause
  \item[(ii)] Il existe  $Q \in\Kk[X]$ tel que $P=(X-\alpha)^kQ,$ avec $Q(\alpha) \neq 0$
\pause
  \item[(iii)] $P(\alpha)= P'(\alpha)=\cdots=P^{(k-1)}(\alpha)=0$ et $P^{(k)}(\alpha) \neq 0$
\end{itemize}
\end{proposition}

\end{frame}


%---------------------------------------------------------------
\section{Théorème de d'Alembert-Gauss}

\begin{frame}

\begin{theoreme}[de d'Alembert-Gauss]  
\begin{itemize}
  \item Tout polynôme à coefficients complexes de degré $n \ge 1$  
a au moins une racine dans $\Cc$
\pause
  \item Il admet exactement $n$ racines si on compte chaque racine
avec multiplicité
\end{itemize}
\end{theoreme}

\pause

\begin{exemple}
$P(X)=3X^3-2X^2+6X-4$
\pause
\begin{itemize}
  \item sur $\Qq$ ou $\Rr$ 
  \begin{itemize}
    \item $P$ a une seule racine (qui est simple) $\alpha = \frac23$
\pause
    \item $P(X)=3(X-\frac23)(X^2+2)$
    \end{itemize}
\pause
  \item sur $\Cc$
  \begin{itemize} 
    \item $P$ a $3$ racines simples
\pause 
    \item $P(X)=3(X-\frac23)(X-\ii\sqrt2)(X+\ii\sqrt2)$
  \end{itemize}
\end{itemize}

\end{exemple}

\end{frame}


\begin{frame}
\begin{exemple}
Soit $P(X)=aX^2+bX+c$ à coefficients réels : $a,b,c \in \Rr$ et $a\neq 0$
\pause
\begin{itemize}
  \item Si $\Delta = b^2-4ac > 0$ alors $P$ admet $2$ racines réelles distinctes 
  \mybox{$\frac{-b+\sqrt{\Delta}}{2a}$ \ \ et \ \ $\frac{-b-\sqrt{\Delta}}{2a}$}
\pause
  \item Si $\Delta < 0$ alors $P$ admet $2$ racines complexes distinctes 
  \mybox{$\frac{-b+\ii\sqrt{|\Delta|}}{2a}$ \ \ et \ \ $\frac{-b-\ii\sqrt{|\Delta|}}{2a}$}
\pause
  \item Si $\Delta = 0$ alors $P$ admet une racine réelle double \myboxinline{$\frac{-b}{2a}$}
\end{itemize}
\end{exemple}
\end{frame}


\begin{frame}
\begin{exemple}
$P(X)=X^n-1$ admet $n$ racines distinctes
\pause
\begin{itemize}
  \item D'Alembert-Gauss : $P$  admet $n$ racines comptées avec multiplicité
\pause
  \item Montrer que ce sont des racines simples
\pause
  \begin{itemize}
    \item Par l'absurde
\pause    
    \item Supposons que $\alpha \in \Cc$ soit une racine de multiplicité $\ge 2$
\pause
    \item \uncover<6->{$P(\alpha)=0$} \uncover<8->{$\implies  \alpha^n-1=0$} \uncover<9->{$\implies \alpha \neq 0$}

    \item \uncover<7->{$P'(\alpha)=0$}  \uncover<10->{$\implies n\alpha^{n-1}=0$} \uncover<11->{$\implies \alpha = 0$}
\pause \pause \pause \pause \pause \pause
    \item Contradiction
  \end{itemize}
\pause 
  \item Donc toutes les racines sont simples
\pause 
  \item Ainsi les $n$ racines sont distinctes
\end{itemize}
\end{exemple}


\end{frame}

%---------------------------------------------------------------
\section{Polynômes irréductibles}

\begin{frame}
Soit  $P \in\Kk[X]$ de degré $\ge 1$
\begin{mydefinition}
$P$ est \defi{irréductible} si pour tout $Q \in\Kk[X]$ divisant $P$, alors
\begin{itemize}
  \item soit $Q \in\Kk^*$
  \item soit il existe $\lambda \in\Kk^*$ tel que $Q=\lambda P$
\end{itemize}
\end{mydefinition}

\pause

$P$ est \defi{réductible} : $P=A\times B$, avec $A, B \in \Kk[X]$,
 $\deg A \ge 1$, $\deg B \ge 1$

\medskip
\pause 

\begin{exemple}
\begin{itemize}
  \item $X^2-1=(X-1)(X+1)\in\Rr[X]$ est réductible
\pause 
  \item $X^2+1=(X-\ii)(X+\ii)$ est réductible dans $\Cc[X]$ mais est irréductible dans $\Rr[X]$
\pause 
  \item $X^2-2=(X-\sqrt2)(X+\sqrt2)$ est réductible dans $\Rr[X]$ mais est irréductible dans $\Qq[X]$ 
\end{itemize}
\end{exemple}

\end{frame}

%---------------------------------------------------------------
\section{Théorème de factorisation}

\begin{frame}
\begin{theoreme}
\begin{enumerate}
  \item Tout polynôme non constant $A\in\Kk[X]$ s'écrit comme un produit de polynômes
irréductibles unitaires :
$$A= \lambda  P_1^{k_1}P_2^{k_2} \cdots P_r^{k_r}$$
 où $\lambda\in\Kk^*$, $r \in \Nn^*$, $k_i \in \Nn^*$ \\
et les $P_i$ sont des polynômes irréductibles distincts
\pause   
  \item De plus cette décomposition est unique à l'ordre près des facteurs
\end{enumerate}
\end{theoreme}

\end{frame}


%---------------------------------------------------------------
\section{Factorisation dans $\Cc[X]$ et $\Rr[X]$}

\begin{frame}

\begin{theoreme}
Les polynômes irréductibles de $\Cc[X]$ sont les polynômes de degré $1$
\end{theoreme}

\pause 
$$P=\lambda (X-\alpha_1)^{k_1}(X- \alpha_2)^{k_2}\cdots(X- \alpha_r)^{k_r}$$
\end{frame}


\begin{frame}

\begin{theoreme}
Les polynômes irréductibles de $\Rr[X]$
sont les polynômes de degré $1$ ainsi que les polynômes de degré $2$ ayant
un discriminant $\Delta<0$
\end{theoreme}

\pause 

$$P=\lambda(X-\alpha_1)^{k_1}(X-\alpha_2)^{k_2}\cdots(X-\alpha_r)^{k_r}
Q_1^{\ell_1}\cdots Q_s^{\ell_s}$$

\pause

\begin{itemize}
  \item $\alpha_i$ sont exactement les racines réelles distinctes, de multiplicité $k_i$
\pause  
  \item les $Q_i$ sont des polynômes irréductibles de degré $2$ :
$Q_i=X^2+\beta_iX+\gamma_i$ avec $\Delta = \beta_i^2-4\gamma_i<0$
\end{itemize}

\bigskip
\pause

\begin{exemple}
\begin{itemize}
  \item $P(X)=2X^4(X-1)^3(X^2+1)^2(X^2+X+1)$ est déjà décomposé en facteurs irréductibles dans $\Rr[X]$
\pause
  \item $P(X)=2X^4(X-1)^3(X-\ii)^2(X+\ii)^2(X-j)(X-j^2)$ dans $\Cc[X]$, avec $j=e^{\frac{2\ii\pi}{3}}=\frac{-1+\ii\sqrt3}{2}$
\end{itemize}
\end{exemple}

\end{frame}


\begin{frame}
\begin{exemple}
$P(X)=X^4+1$ 
\pause
\begin{itemize}
  \item Sur $\Cc$ 
  \pause
  \begin{itemize}
  \setlength{\itemsep}{6pt} 
    \item $P(X)=(X^2+\ii)(X^2-\ii)$
\pause
    \item Racines de $P$ : racines carrées complexes de $\ii$ et $-\ii$
\pause
    \item $P(X)=\big(X-\tfrac{\sqrt2}{2}(1+\ii)\big)\big(X+\tfrac{\sqrt2}{2}(1+\ii)\big)\big(X-\tfrac{\sqrt2}{2}(1-\ii)\big)
\big(X+\tfrac{\sqrt2}{2}(1-\ii)\big)$
  \end{itemize} 
  
\pause
\medskip

  \item Sur $\Rr$
\pause
  \begin{itemize}
  \setlength{\itemsep}{6pt} 
    \item Pour un polynôme à coefficient réels, si $\alpha$ est racine alors $\bar \alpha$ aussi
\pause
    \item {\scriptsize $P \!=\!\left[\big(X-\tfrac{\sqrt2}{2}(1+\ii)\big)\big(X-\tfrac{\sqrt2}{2}(1-\ii)\big)\right]
\left[\big(X+\tfrac{\sqrt2}{2}(1+\ii)\big)\big(X+\tfrac{\sqrt2}{2}(1-\ii)\big)\right]$}
\pause
    \item $P(X)= \big[X^2+\sqrt2X+1\big]\big[X^2-\sqrt2X+1\big]$
  \end{itemize}
\end{itemize} 
\end{exemple}

\end{frame}



%%%%%%%%%%%%%%%%%%%%%%%%%%%%%%%%%%%%%%%%%%%%%%%%%%%%%%%%%%%%%%%%
\section{Mini-exercices}

\begin{frame}

\begin{miniexercice}
\begin{enumerate}
  \item Trouver un polynôme $P(X) \in \Zz[X]$ de degré minimal tel que : $\frac 12$ soit une racine simple,
$\sqrt 2$ soit une racine double et $\ii$ soit une racine triple.

  \item Montrer cette partie de la proposition :
\og$P(\alpha)=0$ et $P'(\alpha)=0$  $\iff$ $\alpha$ est une racine de multiplicité $\ge 2$\fg.

  \item Montrer que pour $P\in\Cc [X]$ :
\og$P$ admet une racine de multiplicité $\ge 2$ $\iff$ $P$ et $P'$ ne sont pas premiers entre eux\fg.

  \item Factoriser $P(X) = (2X^2+X-2)^2(X^4-1)^3$ et $Q(X)=3(X^2-1)^2(X^2-X+\frac14)$ dans $\Cc[X]$.
En déduire leur pgcd et leur ppcm. Mêmes questions dans $\Rr[X]$.

  \item Si $\pgcd(A,B)=1$ montrer que $\pgcd(A+B,A\times B)=1$.

  \item Soit $P\in \Rr[X]$ et $\alpha \in \Cc\setminus\Rr$ tel que $P(\alpha)=0$.
Vérifier que $P(\bar \alpha)=0$. Montrer que $(X-\alpha)(X-\bar\alpha)$ est un polynôme irréductible de $\Rr[X]$
et qu'il divise $P$ dans $\Rr[X]$.
\end{enumerate}
\end{miniexercice}

\end{frame}

\end{document}
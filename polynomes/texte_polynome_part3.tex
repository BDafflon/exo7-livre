
%%%%%%%%%%%%%%%%%% PREAMBULE %%%%%%%%%%%%%%%%%%


\documentclass[12pt]{article}

\usepackage{amsfonts,amsmath,amssymb,amsthm}
\usepackage[utf8]{inputenc}
\usepackage[T1]{fontenc}
\usepackage[francais]{babel}


% packages
\usepackage{amsfonts,amsmath,amssymb,amsthm}
\usepackage[utf8]{inputenc}
\usepackage[T1]{fontenc}
%\usepackage{lmodern}

\usepackage[francais]{babel}
\usepackage{fancybox}
\usepackage{graphicx}

\usepackage{float}

%\usepackage[usenames, x11names]{xcolor}
\usepackage{tikz}
\usepackage{datetime}

\usepackage{mathptmx}
%\usepackage{fouriernc}
%\usepackage{newcent}
\usepackage[mathcal,mathbf]{euler}

%\usepackage{palatino}
%\usepackage{newcent}


% Commande spéciale prompteur

%\usepackage{mathptmx}
%\usepackage[mathcal,mathbf]{euler}
%\usepackage{mathpple,multido}

\usepackage[a4paper]{geometry}
\geometry{top=2cm, bottom=2cm, left=1cm, right=1cm, marginparsep=1cm}

\newcommand{\change}{{\color{red}\rule{\textwidth}{1mm}\\}}

\newcounter{mydiapo}

\newcommand{\diapo}{\newpage
\hfill {\normalsize  Diapo \themydiapo \quad \texttt{[\jobname]}} \\
\stepcounter{mydiapo}}


%%%%%%% COULEURS %%%%%%%%%%

% Pour blanc sur noir :
%\pagecolor[rgb]{0.5,0.5,0.5}
% \pagecolor[rgb]{0,0,0}
% \color[rgb]{1,1,1}



%\DeclareFixedFont{\myfont}{U}{cmss}{bx}{n}{18pt}
\newcommand{\debuttexte}{
%%%%%%%%%%%%% FONTES %%%%%%%%%%%%%
\renewcommand{\baselinestretch}{1.5}
\usefont{U}{cmss}{bx}{n}
\bfseries

% Taille normale : commenter le reste !
%Taille Arnaud
%\fontsize{19}{19}\selectfont

% Taille Barbara
%\fontsize{21}{22}\selectfont

%Taille François
%\fontsize{25}{30}\selectfont

%Taille Pascal
%\fontsize{25}{30}\selectfont

%Taille Laura
%\fontsize{30}{35}\selectfont


%\myfont
%\usefont{U}{cmss}{bx}{n}

%\Huge
%\addtolength{\parskip}{\baselineskip}
}


% \usepackage{hyperref}
% \hypersetup{colorlinks=true, linkcolor=blue, urlcolor=blue,
% pdftitle={Exo7 - Exercices de mathématiques}, pdfauthor={Exo7}}


%section
% \usepackage{sectsty}
% \allsectionsfont{\bf}
%\sectionfont{\color{Tomato3}\upshape\selectfont}
%\subsectionfont{\color{Tomato4}\upshape\selectfont}

%----- Ensembles : entiers, reels, complexes -----
\newcommand{\Nn}{\mathbb{N}} \newcommand{\N}{\mathbb{N}}
\newcommand{\Zz}{\mathbb{Z}} \newcommand{\Z}{\mathbb{Z}}
\newcommand{\Qq}{\mathbb{Q}} \newcommand{\Q}{\mathbb{Q}}
\newcommand{\Rr}{\mathbb{R}} \newcommand{\R}{\mathbb{R}}
\newcommand{\Cc}{\mathbb{C}} 
\newcommand{\Kk}{\mathbb{K}} \newcommand{\K}{\mathbb{K}}

%----- Modifications de symboles -----
\renewcommand{\epsilon}{\varepsilon}
\renewcommand{\Re}{\mathop{\text{Re}}\nolimits}
\renewcommand{\Im}{\mathop{\text{Im}}\nolimits}
%\newcommand{\llbracket}{\left[\kern-0.15em\left[}
%\newcommand{\rrbracket}{\right]\kern-0.15em\right]}

\renewcommand{\ge}{\geqslant}
\renewcommand{\geq}{\geqslant}
\renewcommand{\le}{\leqslant}
\renewcommand{\leq}{\leqslant}

%----- Fonctions usuelles -----
\newcommand{\ch}{\mathop{\mathrm{ch}}\nolimits}
\newcommand{\sh}{\mathop{\mathrm{sh}}\nolimits}
\renewcommand{\tanh}{\mathop{\mathrm{th}}\nolimits}
\newcommand{\cotan}{\mathop{\mathrm{cotan}}\nolimits}
\newcommand{\Arcsin}{\mathop{\mathrm{Arcsin}}\nolimits}
\newcommand{\Arccos}{\mathop{\mathrm{Arccos}}\nolimits}
\newcommand{\Arctan}{\mathop{\mathrm{Arctan}}\nolimits}
\newcommand{\Argsh}{\mathop{\mathrm{Argsh}}\nolimits}
\newcommand{\Argch}{\mathop{\mathrm{Argch}}\nolimits}
\newcommand{\Argth}{\mathop{\mathrm{Argth}}\nolimits}
\newcommand{\pgcd}{\mathop{\mathrm{pgcd}}\nolimits} 

\newcommand{\Card}{\mathop{\text{Card}}\nolimits}
\newcommand{\Ker}{\mathop{\text{Ker}}\nolimits}
\newcommand{\id}{\mathop{\text{id}}\nolimits}
\newcommand{\ii}{\mathrm{i}}
\newcommand{\dd}{\mathrm{d}}
\newcommand{\Vect}{\mathop{\text{Vect}}\nolimits}
\newcommand{\Mat}{\mathop{\mathrm{Mat}}\nolimits}
\newcommand{\rg}{\mathop{\text{rg}}\nolimits}
\newcommand{\tr}{\mathop{\text{tr}}\nolimits}
\newcommand{\ppcm}{\mathop{\text{ppcm}}\nolimits}

%----- Structure des exercices ------

\newtheoremstyle{styleexo}% name
{2ex}% Space above
{3ex}% Space below
{}% Body font
{}% Indent amount 1
{\bfseries} % Theorem head font
{}% Punctuation after theorem head
{\newline}% Space after theorem head 2
{}% Theorem head spec (can be left empty, meaning ‘normal’)

%\theoremstyle{styleexo}
\newtheorem{exo}{Exercice}
\newtheorem{ind}{Indications}
\newtheorem{cor}{Correction}


\newcommand{\exercice}[1]{} \newcommand{\finexercice}{}
%\newcommand{\exercice}[1]{{\tiny\texttt{#1}}\vspace{-2ex}} % pour afficher le numero absolu, l'auteur...
\newcommand{\enonce}{\begin{exo}} \newcommand{\finenonce}{\end{exo}}
\newcommand{\indication}{\begin{ind}} \newcommand{\finindication}{\end{ind}}
\newcommand{\correction}{\begin{cor}} \newcommand{\fincorrection}{\end{cor}}

\newcommand{\noindication}{\stepcounter{ind}}
\newcommand{\nocorrection}{\stepcounter{cor}}

\newcommand{\fiche}[1]{} \newcommand{\finfiche}{}
\newcommand{\titre}[1]{\centerline{\large \bf #1}}
\newcommand{\addcommand}[1]{}
\newcommand{\video}[1]{}

% Marge
\newcommand{\mymargin}[1]{\marginpar{{\small #1}}}



%----- Presentation ------
\setlength{\parindent}{0cm}

%\newcommand{\ExoSept}{\href{http://exo7.emath.fr}{\textbf{\textsf{Exo7}}}}

\definecolor{myred}{rgb}{0.93,0.26,0}
\definecolor{myorange}{rgb}{0.97,0.58,0}
\definecolor{myyellow}{rgb}{1,0.86,0}

\newcommand{\LogoExoSept}[1]{  % input : echelle
{\usefont{U}{cmss}{bx}{n}
\begin{tikzpicture}[scale=0.1*#1,transform shape]
  \fill[color=myorange] (0,0)--(4,0)--(4,-4)--(0,-4)--cycle;
  \fill[color=myred] (0,0)--(0,3)--(-3,3)--(-3,0)--cycle;
  \fill[color=myyellow] (4,0)--(7,4)--(3,7)--(0,3)--cycle;
  \node[scale=5] at (3.5,3.5) {Exo7};
\end{tikzpicture}}
}



\theoremstyle{definition}
%\newtheorem{proposition}{Proposition}
%\newtheorem{exemple}{Exemple}
%\newtheorem{theoreme}{Théorème}
\newtheorem{lemme}{Lemme}
\newtheorem{corollaire}{Corollaire}
%\newtheorem*{remarque*}{Remarque}
%\newtheorem*{miniexercice}{Mini-exercices}
%\newtheorem{definition}{Définition}




%definition d'un terme
\newcommand{\defi}[1]{{\color{myorange}\textbf{\emph{#1}}}}
\newcommand{\evidence}[1]{{\color{blue}\textbf{\emph{#1}}}}



 %----- Commandes divers ------

\newcommand{\codeinline}[1]{\texttt{#1}}

%%%%%%%%%%%%%%%%%%%%%%%%%%%%%%%%%%%%%%%%%%%%%%%%%%%%%%%%%%%%%
%%%%%%%%%%%%%%%%%%%%%%%%%%%%%%%%%%%%%%%%%%%%%%%%%%%%%%%%%%%%%


\begin{document}

\debuttexte

%%%%%%%%%%%%%%%%%%%%%%%%%%%%%%%%%%%%%%%%%%%%%%%%%%%%%%%%%%%
\diapo

\change

Voici une leçon importante dans laquelle nous abordons beaucoup de notions : 

\change

Les racines d'un polynôme

\change

le théorème de d'Alembert-Gauss

\change

La définition de Polynômes irréductibles

\change

et le Théorème de factorisation

\change

Que l'on détaillera dans le cas des polynômes sur $\Cc$ ou sur $\Rr$.

%%%%%%%%%%%%%%%%%%%%%%%%%%%%%%%%%%%%%%%%%%%%%%%%%%%%%%%%%%%
\diapo

Soit $P$ un polynôme à coefficient dans $\Kk$ où $\Kk$ est l'un des corps $\Qq, \Rr, \Cc$.

 On dit que
$\alpha$ est une \defi{racine}  de $P$ si $P(\alpha)=0$.

On dit aussi que $\alpha$ est un \defi{zéro} de $P$.


\change

Une relation importante est que 
$P(\alpha)=0$ ssi  $X-\alpha \text{ divise } P$

\change

La preuve est la suivante on écrit la division euclidienne de $P$ par $X-\alpha$ 

\change

on obtient  $P=Q\cdot(X-\alpha)+R$ 

\change

où $R$ est un polynome de degré strictement inférieur au diviseur donc $\deg R < 1$. 

\change

Mais un polynome de degre $<1$ est un polynome constant.


\change

Maintenant par l'écriture de la division euclidienne  évaluée en $\alpha$
$P(\alpha)=0 \iff R(\alpha) =0 $

comme $R$ est un polynôme constant 

\change

$R(\alpha) =0$
équivaut à ce que $R$ est le polynôme cst  $=0$.

\change

Mais toujours par l'écriture de la division euclidienne
$R=0 \iff P = Q(X-\alpha)$ 

\change

c-a-d $X-\alpha$ divise $P$.  

%%%%%%%%%%%%%%%%%%%%%%%%%%%%%%%%%%%%%%%%%%%%%%%%%%%%%%%%%%%
\diapo

On dit que $\alpha$ est une \defi{racine de multiplicité $k$}
 de $P$ si $(X-\alpha )^k$ divise $P$ alors que $(X- \alpha )^{k+1}$ ne divise pas $P$.

On dit aussi que $\alpha$ est une \defi{racine d'ordre $k$}.

\change

si $k=1$ $\alpha$ est une \defi{racine simple}, 

\change

si $k=2$ $\alpha$ est une \defi{racine double}, etc.

\change


Voici une façon de calculer la multiplicité d'une racine :


\change


les trois proposition suivantes sont équivalentes [[tout afficher]]

\change

\change


$\alpha$ est une racine de multiplicité $k$ de $P$.

équivaut

Il existe  $Q \in\Kk[X]$ tel que $P=(X-\alpha)^kQ,$ avec $Q(\alpha) \neq 0$.


équivaut à

$P(\alpha)= 0$, $P'(\alpha)=0$ $\cdots$ $P^{(k-1)}(\alpha)=0$ et $P^{(k)}(\alpha) \neq 0$.

Cette dernière assertion est très utile dans la pratique,

on calcule $P(\alpha)$, $P'(\alpha)$, $P''(\alpha)$
jusqu'à ce que l'on obtienne une dérivée $k$-ème non nulle.


%%%%%%%%%%%%%%%%%%%%%%%%%%%%%%%%%%%%%%%%%%%%%%%%%%%%%%%%%%%
\diapo

Voici le théorème fondamental de l'algèbre : le théorème de d'Alembert-Gauss.

Tout polynôme à coefficients complexes de degré $n \ge 1$  
admet au moins une racine dans $\Cc$.

\change

Et on peut être plus précis : il y a exactement $n$ racines 
si on compte chaque racine
avec sa multiplicité.

(une racine simple compte pour $1$, une racine double pour $2$, etc.)




\change

Soit $P(X)=3X^3-2X^2+6X-4$.


\change

Considéré comme un polynôme à coefficients dans $\Qq$ ou $\Rr$,
$P$ a une racine  $\alpha = \frac23$

\change

et il se décompose en
$P(X)=3(X-\frac23)(X^2+2)$.

Il n'y a qu'une seule racine réelle

et en plus $\frac23$ est une racine simple (l'exposant est $1$).

\change

Par contre si on considère maintenant aussi les racines complexes

\change

alors $P(X)=3(X-\frac23)(X-\ii\sqrt2)(X+\ii\sqrt2)$

donc admets trois racines : $\frac23$ bien sûr,

mais aussi les nombres complexes $+ \ii \sqrt 2$ et $- \ii \sqrt 2$.
Ce sont des racines simples.

$P$ est un polynôme de degré $3$ et admet exactement $3$ racines, mais c'est valable sur $\Cc$.


Pour un corps $\Kk$ autre que les nombres complexes,
on peut seulement affirmer 
$P$ admet *au plus* $n$ racines dans $\Kk$.

 

%%%%%%%%%%%%%%%%%%%%%%%%%%%%%%%%%%%%%%%%%%%%%%%%%%%%%%%%%%%
\diapo

Revoyons un exemple que vous connaissez bien : 
les racines d'un polynôme de degré $2$ :

Soit $P(X)=aX^2+bX+c$ un polynôme de degré $2$ à coefficients réels,
c-a-d $a,b,c \in \Rr$ et $a\neq 0$.


\change

Si $\Delta = b^2-4ac > 0$ alors $P$ admet $2$ racines réelles distinctes $\frac{-b+\sqrt{\Delta}}{2a}$
et $\frac{-b-\sqrt{\Delta}}{2a}$

\change

Si $\Delta < 0$ alors $P$ n'a pas de racines réelles mais admet $2$ racines complexes distinctes qui sont
$\frac{-b+\ii\sqrt{|\Delta|}}{2a}$
et $\frac{-b-\ii\sqrt{|\Delta|}}{2a}$

\change

Si $\Delta = 0$ alors $P$ admet une racine réelle double $\frac{-b}{2a}$.
En tenant compte des multiplicités on a donc toujours exactement $2$ racines.



%%%%%%%%%%%%%%%%%%%%%%%%%%%%%%%%%%%%%%%%%%%%%%%%%%%%%%%%%%%
\diapo

Pour le polynôme $P(X)=X^n-1$ montrons qu'il n'y a pas deux racines égales,
sans les calculer.

\change

Sachant que $P$ est de degré $n$ alors par le théorème de d'Alembert-Gauss
on sait qu'il admet $n$ racines comptées avec multiplicité. 

\change

Il s'agit donc maintenant de montrer que ce sont des racines simples.

\change

Raisonnons par l'absurde et 

\change

Supposons qu'il existe une racine $\alpha$ de multiplicité $\ge 2$

\change

Cela signifie exactement que  $P(\alpha)=0$

\change

et que $P'(\alpha)=0$

\change

$P(\alpha)=0$ implique $\alpha^n-1=0$ 

\change

et donc $\alpha$ ne peut pas être nul.

\change

Mais d'autre part $P'(X) = n X^{n-1}$ donc
$P'(\alpha)=0$ 
implique $n\alpha^{n-1}=0$ 

\change

et donc $\alpha=0$,


\change

On obtient une contradiction.


\change

Ce qui implique que toutes les racines sont simples.

\change

et ainsi les $n$ racines sont distinctes.

Sur cet exemple particulier on aurait aussi pu calculer les racines 
qui sont ici les racines $n$-ième de l'unité.

%%%%%%%%%%%%%%%%%%%%%%%%%%%%%%%%%%%%%%%%%%%%%%%%%%%%%%%%%%%
\diapo

Nous dirons d'un polynômes $P$ a coefficient dans $\Kk$ qu'il est irréductible 
si ses seuls diviseurs sont soit des constantes, soit le polynome lui même.

Plus précisément si $Q$ est un polynôme à coefficient dans $\Kk$ qui divise le polynôme irrédéuctible $P$
alors

soit $Q$ est un polynômes constant,

soit $Q$ égale $P$ a une constante multiplicative près.


Notez déjà l'analogie avec les nombres premiers :
un nombre est premiers si ces diviseurs sont soit $1$, soit le nombre lui même.


\change

Un polynôme est donc réductible s'il se décompose en produit de deux 
 polynômes $A \times B$ de $\Kk[X]$ de degré $\ge 1$.
 
 
\change

Passons aux exemples :

$X^2-1$ est réductible car il s'écrit $(X-1)(X+1)$.


\change

$X^2+1=(X-\ii)(X+\ii)$ est réductible en tant que polynôme sur les nombres complexes.

Mais en tant que polynômes sur $\Rr$ il n'existe pas une telle décomposition : c'est un polynômes irréductible
dans $\Rr[X]$.


Etre irréductible ou pas dépend donc du corps $\Qq$, $\Rr$ ou $\Cc$ que l'on considère.

\change

Par exemple 
$X^2-2=(X-\sqrt2)(X+\sqrt2)$ est réductible dans $\Rr[X]$ mais cette décomposition n'est pas à coefficients rationnel et en fait
$X^2-2$ est irréductible dans $\Qq[X]$.  



%%%%%%%%%%%%%%%%%%%%%%%%%%%%%%%%%%%%%%%%%%%%%%%%%%%%%%%%%%%
\diapo

Les polynômes irréductibles sont les briques élémentaires qui permettent de reconstituer
tous les polynômes :


Tout polynôme $A(X)$ s'écrit comme un produit de polynômes
irréductibles :

Il s'agit bien sûr de l'analogue de la décomposition d'un nombre en facteurs premiers.
Les polynômes correspondent aux entiers et les polynômes irréductibles aux nombres premiers.

Théorème : 
Tout polynôme $A$ s'écrit comme un produit de polynômes
irréductibles unitaires :
$$A= \lambda  P_1^{k_1}P_2^{k_2} \cdots P_r^{k_r}$$
 où $\lambda$ est une constante, $k_i$ sont des entiers, qui sont les multiplicités
et les $P_i$ sont des polynômes irréductibles distincts.


\change


Comme dans le cas des entiers cette écriture va être unique à l'ordre près :

Si l'on suppose les polynômes irréductibles unitaires et distincts
alors cette écriture est unique 
(unicité des $P_i$, du coefficient $\lambda$,
du nombre de polynômes $r$,
 des exposants $k_i$).




%%%%%%%%%%%%%%%%%%%%%%%%%%%%%%%%%%%%%%%%%%%%%%%%%%%%%%%%%%%
\diapo


Ce théorème important est une reformulation immédiate du théorème de d'Alembert-Gauss.  

Théorème : Les polynômes irréductibles de $\Cc[X]$ sont les polynômes de degré $1$. 

\change

Donc pour $P\in\Cc[X]$ la factorisation s'écrit 
$P=\lambda (X-\alpha_1)^{k_1}(X- \alpha_2)^{k_2}\cdots(X- \alpha_r)^{k_r},$ 
où $\alpha_1,...,\alpha_r$ sont les racines distinctes de $P$ et
$k_1,...,k_r$ sont leurs multiplicités. 


%%%%%%%%%%%%%%%%%%%%%%%%%%%%%%%%%%%%%%%%%%%%%%%%%%%%%%%%%%%
\diapo

Passons au cas des polynômes à coefficients réels.

Les polynômes irréductibles de $\Rr[X]$
sont d'une part les polynômes de degré $1$ 

mais aussi les polynômes de degré $2$ ayant
un discriminant $\Delta$ strictement négatif.

\change

la factorisation d'un polynome $P$ à coefficient réel s'écrit
$P=\lambda(X-\alpha_1)^{k_1}(X-\alpha_2)^{k_2}\cdots(X-\alpha_r)^{k_r}
Q_1^{\ell_1}\cdots Q_s^{\ell_s},$

\change

les $\alpha_i$ sont toutes les racines réelles 

et $k_i$ la multiplicité de chaque racine

\change
les $Q_i$ sont des polynômes de degré $2$
ayant un discriminant strictement négatif.

Si $Q_i=X^2+\beta_iX+\gamma_i$ alors $\Delta = \beta_i^2-4\gamma_i<0$


\change

Par exemple le polynôme $2X^4(X-1)^3(X^2+1)^2(X^2+X+1)$ 
est déjà décomposé en facteurs irréductibles dans $\Rr[X]$

car $X^2+1$ et $X^2+X+1$ ont un discriminant $<0$ donc n'ont pas de racines réelles.

\change

Par contre si on considère $P$ comme un polynôme à coefficients complexes
alors il faut factoriser  $X^2+1$ et $X^2+X+1$

pour  obtenir la factorisation :
$2X^4(X-1)^3(X-\ii)^2(X+\ii)^2(X-j)(X-j^2)$ 
où $j$ est le nombre complexe $e^{\frac{2\ii\pi}{3}}$

%%%%%%%%%%%%%%%%%%%%%%%%%%%%%%%%%%%%%%%%%%%%%%%%%%%%%%%%%%%
\diapo

Soit $P(X)=X^4+1$. On cherche quelle est sa décomposition en facteurs irréductibles

\change

Sur $\Cc$ tout d'abord.

\change

On peut dans un premier temps reconnaître l'identité remarquable $a^2-b^2$ qui permet de
décomposer $P(X)=(X^2+\ii)(X^2-\ii)$.

\change

Mais ce n'est pas fini on doit se ramener à des facteurs de degré $1$.

On voit que les racines du polynôme $P$ sont les racines carrées complexes de $\ii$ et de $-\ii$.

\change

On calcule ses racines pour trouver cette factorisation de $P$.


\change

Passons  Sur $\Rr$. 

\change

Il et facile de montrer que pour un polynôme à coefficient *réels*
, si $\alpha$ est une racine
alors $\bar \alpha$ aussi. 

\change

Dans la décomposition ci-dessus on regroupe les facteurs ayant des racines conjuguées, 

\change

de cette façon :

$(X-\alpha)(X-\bar \alpha)$ est un polynôme à coefficients réels.


Ce qui donne la factorisation dans $\Rr[X]$.

%%%%%%%%%%%%%%%%%%%%%%%%%%%%%%%%%%%%%%%%%%%%%%%%%%%%%%%%%%%
\diapo

Entraînez-vous avec ces exercices !


\end{document}
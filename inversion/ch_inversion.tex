\documentclass[11pt,class=report,crop=false]{standalone}
\usepackage[screen]{../exo7book}


% Commandes spécifiques à ce chapitre
\newcommand{\inversion}{i}
\renewcommand{\vec}[1]{\overrightarrow{#1}}

\begin{document}

%====================================================================
\chapitre{L'inversion}
%====================================================================

L'inversion est une transformation remarquable du plan. Par exemple elle peut changer une droite en un cercle
et un cercle en une droite. Nous étudions ici quelques propriétés de l'inversion. Cela nous permettra de créer un dispositif mécanique qui transforme un mouvement circulaire en un mouvement rectiligne. Nous montrerons enfin que toute construction à la règle et au compas peut s'effectuer au compas seulement.

%%%%%%%%%%%%%%%%%%%%%%%%%%%%%%%%%%%%%%%%%%%%%%%%%%%%%%%%%%%%%%%%
\section{Cercle-droite}

%---------------------------------------------------------------
\subsection{\'Equation complexe d'une droite}

Soit 
$$ax+by=c$$ 
l'équation réelle d'une droite $\mathcal{D}$ : $a,b,c$ sont des nombres réels
($a$ et $b$ n'étant pas nuls en même temps) d'inconnues $(x,y) \in \Rr^2$.



\'Ecrivons $z=x+\ii y \in \Cc$, alors 
$$x = \frac{z+\bar z}{2}, \quad y = \frac{z - \bar z}{2i},$$
donc $\mathcal{D}$ a aussi pour équation
$a(z+\bar z) -\ii b(z-\bar z)=2c$ ou encore $(a-\ii b)z+(a+\ii b)\bar z = 2c$.
Posons $\omega = a+\ii b \in \Cc^*$ et $k=2c \in \Rr$ alors nous obtenons\\
\mybox{
\begin{minipage}{0.7\textwidth}
l'\defi{équation complexe} d'une droite est : \\
\centerline{$\bar \omega z + \omega \bar z = k$} \\
où $\omega \in \Cc^*$ et $k\in \Rr$. 
\end{minipage}
}

\myfigure{1}{
\tikzinput{fig_inversion01}
}
%---------------------------------------------------------------
\subsection{\'Equation complexe d'un cercle}

Soit $\mathcal{C}(\Omega,r)$ le cercle de centre $\Omega$ et de rayon $r$.
C'est l'ensemble des points $M$ tels que $d(\Omega,M)=r$. Si l'on note $\omega$ l'affixe
de $\Omega$ et $z$ l'affixe de $M$. Nous obtenons:
$$d(\Omega,M) = r \iff  |z-\omega|=r \iff |z-\omega|^2=r^2 \iff (z-\omega)\overline{(z-\omega)}=r^2$$
en développant nous trouvons que\\
\mybox{
\begin{minipage}{0.7\textwidth}
l'\defi{équation complexe} du cercle centré en $\Omega(\omega)$ et de rayon $r$ est: \\
\centerline{$z\bar z - \bar \omega z - \omega \bar z = r^2-|\omega|^2$} \\
où $\omega \in \Cc$ et $r\in \Rr$. 
\end{minipage}
}

\myfigure{1}{
\tikzinput{fig_inversion02}
}

%---------------------------------------------------------------
\subsection{Les cercles-droites}

Les deux paragraphes précédents conduisent à la définition suivante.
\begin{proposition}
 Un \defi{cercle-droite} est un ensemble de points $M$ du plan, d'affixe $z$,
 tel que
$$az\bar z - \bar \omega z - \omega \bar z = k$$
 où $a,k \in \Rr$, $\omega \in \Cc$ sont donnés.
 \begin{itemize}
  \item Si $a=0$ un cercle-droite est une droite.
  \item Si $a\neq 0$ un cercle-droite est un cercle.
 \end{itemize}
\end{proposition}


\begin{exemple}
Le cercle $\mathcal{C}_r = \mathcal{C}(\Omega(0,r),r)$ a pour équation
$z\bar z - \bar \omega_r z - \omega_r \bar z = r^2-|\omega_r|^2$
avec son centre d'affixe $\omega_r = 0+\ii r$. Cette équation s'écrit aussi
$z\bar z +\ii rz-\ii r\bar z =0$ ou encore $z-\bar z + \frac{z\bar z}{\ii r} = 0$.
On fait tendre $r$  vers l'infini : le rayon tend vers l'infini et le centre s'éloigne indéfiniment, cependant le cercle passe toujours par l'origine. 
\`A la limite l'équation devient
$z-\bar z = 0$, qui est l'équation d'une droite et plus précisément de l'axe des abscisses.
Une droite peut-être vue comme un cercle dont le centre est à l'infini.
\myfigure{1}{
\tikzinput{fig_inversion03}
}
\end{exemple}



%%%%%%%%%%%%%%%%%%%%%%%%%%%%%%%%%%%%%%%%%%%%%%%%%%%%%%%%%%%%%%%%
\section{L'inversion}

%---------------------------------------------------------------
\subsection{Définition géométrique}

Soit le cercle $\mathcal{C} = \mathcal{C}(\Omega,r)$. L'\defi{inversion} est l'application
du plan privé de $\Omega$ dans lui-même, qui à un point $M$ associe un point $M'$ tel que :
\begin{itemize}
  \item $M' \in [\Omega M)$,
  \item $\Omega M \cdot \Omega M' = r^2$.
\end{itemize}

\myfigure{1}{
\tikzinput{fig_inversion04}
}

La première condition impose que $M'$ est sur la demi-droite issue de $\Omega$ passant par $M$,
la deuxième condition lie les distances de $M$ et $M'$ à $\Omega$.

Le point $\Omega$ est le \defi{centre} de l'inversion, le nombre
$r^2$ est sa \defi{puissance}, $\mathcal{C}(\Omega,r)$ est le \defi{cercle d'inversion}.

Voici quelques propriétés élémentaires ($\mathcal{P}$ désigne le plan) :
\begin{proposition}
Soit $\inversion : \mathcal{P} \setminus \{\Omega\} \to \mathcal{P} \setminus \{\Omega\}$ une inversion de centre $\Omega$ et de puissance $r^2$.
\begin{enumerate}
  \item Chaque point du cercle $\mathcal{C}(\Omega,r)$ est invariant par $\inversion$ :
  $M \in \mathcal{C}(\Omega,r) \implies \inversion(M) = M$.
  
  \item L'inversion $\inversion$ est une bijection. 
  C'est même une involution : pour tout point $M \in \mathcal{P} \setminus \{\Omega\}$, $\inversion(\inversion(M)) = M$.
\end{enumerate}
\end{proposition}

Le fait que $M \mapsto \inversion(M)$ soit une involution se formule aussi ainsi : si $M'=\inversion(M)$ alors 
$M = \inversion(M')$.


\begin{exemple}
Soit $\inversion$ l'inversion de centre l'origine et de puissance $r^2=4$. 
Nous représentons des points $M_k$ ainsi que leur image $M'_k=\inversion(M_k)$.
Comme l'inversion est involutive, nous avons aussi $M_k=\inversion(M'_k)$.
Il est important de noter que \emph{l'inversion ne préservent pas les longueurs}.
\myfigure{1}{
\tikzinput{fig_inversion05}
}
Par exemple, comparez les distance $M_1M_4$ et $M'_1 M'_4$.
Voir l'exercice \ref{exo:inversion_ptoleme} pour une formule.
\end{exemple}

\begin{proof}~
\begin{enumerate}
  \item Soit $M \in \mathcal{C}(\Omega,r)$ et notons $M' = \inversion(M)$.
  La relation entre les distances s'écrit $\Omega M \cdot \Omega M' = r^2$.
  Mais comme $\Omega M=r$ alors nous avons aussi $\Omega M' = r$. Comme $M$ et $M'$
  sont sur la même demi-droite issue de $\Omega$ alors $M=M'$.
  
  \item $M \in \mathcal{P} \setminus \{\Omega\}$. Notons $M' = \inversion(M)$ et $M'' = \inversion(M')$.
  Comme $M'' \in [\Omega M')$ et $M' \in [\Omega M)$ alors $M''$ appartient à la demi-droite $[\Omega M)$.
  Les relations entre les distances sont d'une part $\Omega M \cdot \Omega M' = r^2$
  et $\Omega M' \cdot \Omega M'' = r^2$. D'où les égalités $\Omega M \cdot \Omega M' = \Omega M' \cdot \Omega M''$,
  puis $\Omega M=\Omega M''$. Comme $M$ et $M''$ sont sur la même demi-droite issue de $\Omega$ alors $M=M''$.
  
  Le bilan est le suivant $\inversion\big(\inversion(M)\big) = M$. L'application $M \mapsto \inversion(M)$ est donc une involution.
  En particulier c'est une bijection.
\end{enumerate}

\end{proof}
%---------------------------------------------------------------
\subsection{\'Ecriture complexe}

Considérons les points et leur affixes $\Omega(\omega)$, $M(z)$, $M'(z')$.
Nous allons transformer la relation $M'= \inversion (M)$ en une condition entre $z$ et $z'$.
La première condition $M' \in [\Omega M)$ s'écrit $z'-\omega= \lambda(z-\omega)$ avec $\lambda \in \Rr$ 
et $\lambda \ge 0$.
La deuxième condition $\Omega M \cdot \Omega M' = r^2$ devient en écriture complexe 
$|z-\omega|\cdot|z'-\omega| = r^2$, ce qui donne à l'aide de la première condition
$\lambda |z-\omega|^2 = r^2$ donc $\lambda = \frac{r^2}{|z-\omega|^2}$.
Nous exprimons alors $z'$ comme une fonction de $z$ :
$$z'= \omega + r^2 \frac{z-\omega}{|z-\omega|^2} = \omega + \frac{r^2}{\,\overline{z-\omega}\,}.$$
\myfigure{1}{
\tikzinput{fig_inversion06}
}

Ceci nous permet de donner la définition complexe de l'inversion :
\mybox{
\begin{minipage}{0.7\textwidth}
L'\defi{inversion} est l'application $\inversion : \Cc\cup\{ \infty\} \to \Cc\cup\{ \infty\}$ définie par
$\inversion(z) = \omega + \frac{r^2}{\,\overline{z-\omega}\,}$ pour $z \in \Cc \setminus \{\omega\}$
et prolongée par $\inversion(\omega) = \infty$ et $\inversion(\infty) = \omega$.
\end{minipage}
}

\begin{exemple}
L'inversion de cercle $\mathcal{C}(O,1)$ a pour écriture complexe $\inversion(z)=1/ \bar z$,
que l'on prolonge en $\inversion(0) = \infty$ et $\inversion(\infty)=0$.
\end{exemple}


%---------------------------------------------------------------
\subsection{Inversion et cercle-droite}

\begin{theoreme}
L'image d'un cercle-droite par une inversion est un cercle-droite. 
\end{theoreme}


Plus précisément nous allons montrer que si $\inversion$ est l'inversion
de cercle $\mathcal{C}(\Omega,r)$ alors :
\begin{itemize}
  \item L'image d'une droite passant par $\Omega$ est elle-même.
  
  \item L'image d'une droite ne passant pas par $\Omega$ est un cercle passant par $\Omega$.
  
  \item L'image d'un cercle passant par $\Omega$ est une droite ne passant pas par $\Omega$.
  
  \item L'image d'un cercle ne passant pas par $\Omega$ est une cercle ne passant pas par $\Omega$.
\end{itemize}

Les trois premiers cas sont sur la figure de gauche, le dernier sur la figure de droite. 
\myfigure{1}{
\tikzinput{fig_inversion07} 
\tikzinput{fig_inversion08}
}

\begin{proof}
Remarquons tout d'abord que pour une translation l'image d'une droite est une droite et l'image d'un cercle est un cercle.
Il en va de même pour les homothéties.

Donc par une translation, nous nous ramenons à démontrer la proposition dans le cas où le centre de l'inversion est situé à l'origine du plan complexe. Par une homothétie nous supposons même
que le cercle d'inversion est de rayon $1$.
Après ces deux réductions nous nous sommes ramenés au cas où l'inversion a pour écriture complexe:
$$\inversion(z) = \frac{1}{\bar z}.$$

Soit maintenant $\mathcal{C}$ un cercle-droite d'équation
$az\bar z - \bar \omega z - \omega \bar z = k$
($a,k\in \Rr$, $\omega \in \Cc$).
Soit $M(z)$ un point du plan (d'affixe $z$) et notons
$M'$ l'image de $M$ par notre inversion
qui sera donc d'affixe $z' = \inversion(z) = \frac{1}{\bar z}$.

\begin{align*}
 M(z) \in \mathcal{C} 
   &\iff  az\bar z - \bar \omega z - \omega \bar z = k \\
   &\iff a - \bar \omega\frac{1}{\bar z} - \omega \frac 1 z = \frac{k}{ z\bar z} \quad \text{ en divisant par } z\bar z \\
   &\iff a - \bar \omega z' - \omega \bar z' = k z' \bar z' \quad \text{ car } z' = 1/\bar z\\
   &\iff k z' \bar z' + \bar \omega z' + \omega \bar z' = a \\
\end{align*}
Mais la dernière ligne est l'équation d'un autre cercle-droite $\mathcal{C}'$.
Bilan $M(z) \in \mathcal{C}$ si et seulement si $\inversion(M) \in \mathcal{C}'$.
Autrement dit l'image du cercle-droite $\mathcal{C}$ est le cercle-droite $\mathcal{C}'$.

Il suffit de regarder les équations pour obtenir les différents cas. Par exemple si notre cercle-droite passe par l'origine (c'est le cas lorsque $k=0$) il faut traiter le cas $z=0$ à part ; il faut se rappeler notre convention $\inversion(0) = \infty$. Dans ce cas l'équation obtenue pour $\mathcal{C}'$ est celle d'une droite.


\end{proof}

\begin{remarque*}
\sauteligne
\begin{itemize}
 \item L'image d'une droite $\mathcal{D}$ passant par le centre d'une l'inversion $\inversion$ est la droite elle même :
 $\inversion(\mathcal{D})=\mathcal{D}$. La droite est \defi{invariante globalement}. Un point de la droite
 est envoyé sur un autre point de la droite.
 Par contre chaque point du cercle d'inversion est conservé par l'inversion : c'est l'\defi{invariance point par point}.
 Si $\mathcal{C}$ est le cercle d'inversion, cela s'écrit :
 $$\forall P \in \mathcal{C} \quad \inversion(P)=P.$$
 La droite $\mathcal{D}$ passant par l'origine \emph{n'est pas} invariante point par point.
 
 \item Même si l'image d'un cercle $\mathcal{C}$ de centre $O$ est un cercle $\mathcal{C}' = \inversion(\mathcal{C})$, cependant 
 $\inversion(O)$ \emph{n'est pas} le centre de $\mathcal{C}'$. Voir la figure.
\myfigure{1}{
\tikzinput{fig_inversion09}
}
 
\end{itemize}
\end{remarque*}




%---------------------------------------------------------------
\subsection{Inversion et cocyclicité}


\begin{proposition}
\label{prop:cocycl}
Soient $\inversion$ un inversion et $M, N$ deux points du plan.
Les points $M$, $N$, $\inversion(M)$, $\inversion(N)$ sont cocycliques (ou alignés).
\end{proposition}
\myfigure{1}{
\tikzinput{fig_inversion10}
}

C'est un résultat important et utile qui est démontré dans l'exercice \ref{exo:inversion_cocyclicite}.


%%%%%%%%%%%%%%%%%%%%%%%%%%%%%%%%%%%%%%%%%%%%%%%%%%%%%%%%%%%%%%%%
\section{Les homographies}

%---------------------------------------------------------------
\subsection{Définition}

Une \defi{homographie} est une application $h : \Cc \cup \{\infty\} \to \Cc \cup \{\infty\}$
définie par 
$$h(z) = \frac{az+b}{cz+d}, \qquad h(\infty)=\frac ac, \qquad h(-d/c)=\infty,$$
avec $a,b,c,d \in \Cc$ tels que $ad-bc \neq 0$.

\begin{proposition}
\label{prop:dechomo}
Une homographie est la composée d'une inversion $z \mapsto 1/\bar z$, d'une réflexion $z\mapsto \bar z$,
de translations $z\mapsto z+\alpha$ et de rotation-homothéties $z\mapsto \lambda z$ ($\alpha, \lambda \in \Cc$).
\end{proposition}

En particulier une homographie est une application bijective.

\begin{proof}
Tout d'abord par la composition d'une rotation, d'une homothétie et d'une translation nous définissons
$h_1(z) = cz+d$. Puis $h_2(z)= \frac 1z$ est la composée d'une inversion $z \mapsto \frac{1}{\bar z}$
et d'une réflexion $z \mapsto \bar z$. Nous obtenons donc 
$h_2 \circ h_1(z) = \frac{1}{cz+d}$.
Posons $h_3(z)= \alpha z + \beta$ (encore la composition d'une rotation, d'une homothétie et d'une translation)
alors
$$h_3\circ h_2 \circ h_1(z) =  \frac{\alpha}{cz+d} + \beta = \frac{\beta c z + \beta d + \alpha}{cz+d}=\frac{az+b}{cz+d} $$
si l'on a choisi $\beta = \frac{a}{c}$ et $\alpha = b - \frac a c d$.
\end{proof}

\begin{corollaire}
L'image par une homographie d'un cercle-droite est un cercle-droite.
\end{corollaire}

\begin{proof}
L'image d'une droite par une translation est une droite. De même l'image d'une cercle par une translation est un cercle.
Il en va de même pour les rotations, pour les homothéties et pour les réflexions.
L'image d'un cercle par une inversion est un cercle ou une droite, l'image d'une droite
par une inversion est un cercle ou une droite. Par composition
l'image d'une cercle-droite par une homographie est un cercle-droite.
\end{proof}




%---------------------------------------------------------------
\subsection{Homographie et angles}

\begin{theoreme}
\label{th:angle}
Les homographies préservent les angles orientés. 
\end{theoreme}

Voyons d'abord ce que cela signifie. Soient deux courbes $\mathcal{C}_1$ et $\mathcal{C}_2$
qui s'intersectent en un point $P$. Soit $\alpha$ l'angle formé par les deux tangentes à $\mathcal{C}_1$ et $\mathcal{C}_2$ en $P$.
Soit $h$ notre homographie $h$ ; notons $\mathcal{C}'_1 = h(\mathcal{C}_1)$,
$\mathcal{C}'_2 = h(\mathcal{C}_2)$, et $P' = h(P)$ qui appartient à l'intersection
de $\mathcal{C}'_1$ et $\mathcal{C}'_2$. Alors les deux tangentes à $\mathcal{C}_1'$ et $\mathcal{C}_2'$ en $P$
forment le même angle $\alpha$.
\myfigure{1}{
\tikzinput{fig_inversion23}
}

Dans la pratique le corollaire suivant est très utile:
\begin{corollaire}
Soit $h$ une homographie. Si deux courbes $\mathcal{C}_1$ et $\mathcal{C}_2$ sont tangentes
en un point $M$ (resp. perpendiculaires en $M$) alors les courbes
$h(\mathcal{C}_1)$ et $h(\mathcal{C}_2)$ sont tangentes en $h(M)$
(resp. perpendiculaires en $h(M)$).
\end{corollaire}

En fait, on prouve en même temps que:
\begin{corollaire}
Soit $\inversion$ une inversion. Si deux courbes $\mathcal{C}_1$ et $\mathcal{C}_2$ sont tangentes
en un point $M$ (resp. perpendiculaires en $M$) alors les courbes
$\inversion(\mathcal{C}_1)$ et $\inversion(\mathcal{C}_2)$ sont tangentes en $h(M)$
(resp. perpendiculaires en $\inversion(M)$).
\end{corollaire}



\begin{proof}[Preuve du théorème \ref{th:angle}]
~
\begin{itemize}
  \item Encore une fois nous allons ramener le problème à l'étude d'une inversion.
 En effet les homothéties, translations, rotations préservent les angles orientés
 alors qu'une réflexion préserve les angles mais change l'orientation. Par la 
 proposition \ref{prop:dechomo}, il suffit donc de montrer qu'une inversion préserve aussi les angles
 mais change l'orientation.
 
  \item On se donne une courbe $\mathcal{C}$ et un point $M\in \mathcal{C}$, notons
  $T$ la tangente à $M$ en $\mathcal{C}$ (nous supposons donc qu'il existe une tangente en ce point).
  Soit $N$ un autre point de $\mathcal{C}$. Soit $\inversion$ une inversion. On note
  $\mathcal{C}'= \inversion(\mathcal{C})$, $M' = \inversion(M)$, $N'=\inversion(N)$ et on appelle
  $T'$ la tangente à $\mathcal{C}'$ en $M'$. (Attention $T'$ n'est pas égal à $\inversion(T)$,
  de toute façon $\inversion(T)$ n'est pas nécessairement une droite...)

  
\myfigure{0.7}{
\tikzinput{fig_inversion24ter}
}

  
  \item D'après la proposition \ref{prop:cocycl} les points $M,N,M',N'$ sont cocycliques,
  donc par le théorème de l'angle inscrit, les angles $(\vec{MN},\vec{MN'})$ et $(\vec{M'N},\vec{M'N'})$ sont égaux.
  
\myfigure{0.7}{
\tikzinput{fig_inversion24bis}
}

  
  \item Faisons tendre le point $N$ vers le point $M$ alors la droite définie par le vecteur $\vec{MN}$
  et passant par $M$ tend vers la tangente $T$. 

\myfigure{0.7}{
\tikzinput{fig_inversion26}
}    
  
  \item 
  De plus $N'$ tend vers $M'$ et la droite de vecteur $\vec{M'N'}$
  et passant par $M'$ tend vers $T'$. \`A la limite on obtient l'égalité des angles:
  $(T,\vec{MM'}) = (\vec{M'M},T')$. (Cet angle est noté $\theta$ sur la figure ci-dessous.)
  

  \item  En conséquence les tangentes $T$ et $T'$ sont symétriques l'une de l'autre par la réflexion d'axe $\Delta$, la médiatrice de $[MM']$.
\myfigure{0.8}{
\tikzinput{fig_inversion24}
}  
  
  \item Si maintenant $\mathcal{C}_1$ et $\mathcal{C}_2$ sont deux courbes qui s'intersectent en $M$, l'angle entre
  les deux tangentes $T_1$ et $T_2$ étant $\alpha$, alors par la réflexion d'axe $\Delta$, l'angle entre les deux tangentes $T_1'$ et $T_2'$ en $M' = \inversion(M)$ est $-\alpha$.
\myfigure{0.7}{
\tikzinput{fig_inversion25}
}  
\end{itemize}

\end{proof}


%%%%%%%%%%%%%%%%%%%%%%%%%%%%%%%%%%%%%%%%%%%%%%%%%%%%%%%%%%%%%%%%
\section{Dispositifs mécaniques}

%---------------------------------------------------------------
\subsection{La courbe de Watt}

Le but est de transformer un mouvement circulaire en mouvement rectiligne
(ou l'inverse). Une solution simple est d'utiliser une bielle et un piston.
Le problème est que le coulissage génère des frottements au niveau du piston.
\myfigure{1}{
\tikzinput{fig_inversion_piston}
}


\bigskip

L'ingénieur James Watt améliora le dispositif en inventant un mécanisme
qui permet d'obtenir une portion presque rectiligne à partir d'un mouvement circulaire.

\myfigure{0.7}{
\tikzinput{fig_inversion_watt}
}

Les points $A$ et $B$ sont fixes. Autour de chacun de ces points est attachée une barre tournante, l'une se termine en $P$, l'autre en $Q$. Ces deux barres sont reliées par une troisième barre qui va de $P$ à $Q$. 
On note $M$ le milieu de $[PQ]$. Lorsque l'on fait tourné la barre $[AP]$ alors les trois barres bougent et le points $M$ décrit une \defi{courbe de Watt}. Une portion de cette courbe (autour de l'auto-intersection) approxime assez bien une portion rectiligne.

Nous allons voir cependant que l'on sait résoudre de façon exacte ce problème : c'est l'inverseur de Peaucellier.

%---------------------------------------------------------------
\subsection{L'inverseur de Peaucellier}

La construction est basée sur le résultat suivant :
\begin{theoreme}
Soit la configuration suivante avec $\Omega A = \Omega B = R$
et $AMBM'$ un losange de côté $r$.
Alors $M'$ est l'image de $M$ par l'inversion de centre $\Omega$
et de puissance $R^2-r^2$.
\myfigure{0.8}{
\tikzinput{fig_inversion11}}
\end{theoreme}

Ce théorème implique donc par inversion le résultat voulu :
\begin{corollaire}
Si $M$ parcourt un cercle passant par $\Omega$ alors
$M'$ parcourt une droite.
\end{corollaire}


Il existe d'autres dispositifs mécaniques qui transforment
un cercle en une droite, voir par exemple l'inverseur de Hart
dans l'exercice 8.


\begin{proof}[Preuve du théorème]
Tout d'abord $M,M',\Omega$ sont sur la médiatrice du segment $[AB]$.
Donc $M' \in (\Omega M)$. De plus si l'on suppose $R>r$ alors
$M' \in [\Omega M)$.

Calculons maintenant $\Omega M \cdot \Omega M'$.
Soit $I$ le centre du losange $AMBM'$.
Avec la configuration de la figure ci-dessous on a $\Omega M= \Omega I - IM$ et $\Omega M' = \Omega I + IM'= \Omega I + IM$.
Donc $\Omega M \cdot \Omega M' = (\Omega I - IM) (\Omega I + IM) = \Omega I ^2 - IM^2.$
Par le théorème de Pythagore $\Omega I^2 = \Omega A^2 - IA^2$ et 
$IM^2 = AM^2 - IA^2$. Donc $\Omega M\cdot \Omega M'= \Omega A^2 - AM^2 = R^2 - r^2$.

\myfigure{1}{
\tikzinput{fig_inversion13}
}

Nous avons montré que $M'$ est l'image de $M$ par l'inversion de centre $\Omega$
et de puissance $R^2 - r^2$.
\end{proof}

Ainsi lorsque le point $M$ décrit une portion du cercle (en bleu sur la figure), 
le point $M'$ décrit une portion de droite (en rouge). L'autre cercle (en orange) est le cercle d'inversion.
Pour la réalisation mécanique : il faut $6$ barres rigides (en noires), articulées à chaque jonction.
Les points $A$, $B$, $M$, $M'$ sont libres de mouvement, mais le point $\Omega$ est fixe.
\myfigure{1}{
\tikzinput{fig_inversion27}
}

Il existe d'autres dispositifs mécaniques qui transforment
un cercle en une droite, voir par exemple l'inverseur de Hart
dans l'exercice \ref{exo:inversion_hart}.

%---------------------------------------------------------------
\subsection{Théorème de Kempe}

Il existe en fait un théorème plus général.
\begin{theoreme}
Soit $P(x,y) \in \Rr[x,y]$ un polynôme de deux variables.
Pour n'importe qu'elle partie bornée de la courbe $\mathcal{C}$ définie par l'équation $(P(x,y)=0)$, 
il existe un dispositif 
mécanique qui permet de la tracer.
\end{theoreme}


%%%%%%%%%%%%%%%%%%%%%%%%%%%%%%%%%%%%%%%%%%%%%%%%%%%%%%%%%%%%%%%%
\section{Construction au compas seulement}

%---------------------------------------------------------------
\subsection{Problème de Napoléon}

Traçons un cercle, puis effaçons son centre. Il est facile de retrouver le centre avec une règle
et un compas. Faites-le ! Oublions maintenant la règle. 

\defi{Problème de Napoléon.}  \`A l'aide du compas seulement, 
tracer le centre d'un cercle dont on connaît uniquement le contour.

La solution, qui n'a rien d'évidente, utilise l'inversion et se décompose en plusieurs étapes:
\myfigure{1}{\tikzinput{fig_inversion14}}
\begin{enumerate}
  \item Soit $\mathcal{C}_0$ le cercle dont on souhaite trouver le centre. Choisir un point $A$ sur
  $\mathcal{C}_0$ et prendre un écartement quelconque de compas (mais ni trop grand, ni trop petit :
  entre une demie fois et deux fois le rayon --inconnu-- du cercle $\mathcal{C}_0$)

 
  \item Placer la pointe du compas en $A$ et tracer le cercle $\mathcal{C}$. Ce cercle coupe $\mathcal{C}_0$ en deux points, notés
  $B$ et $C$.
\myfigure{1}{\tikzinput{fig_inversion15}}
 
  \item Construire $A'$ le symétrique de $A$ par rapport à $(BC)$ : pour cela, tracer
  les cercles de centre $B$ (puis de centre $C$) passant par $A$. Ces deux cercles se coupent en $A$ et $A'$.
\myfigure{1}{\tikzinput{fig_inversion16}}
  
  \item Construire l'image de $A'$ par l'inversion $\inversion$ de centre $A$ et de cercle $\mathcal{C}$.
  Pour cela on trace le cercle de centre $A'$ passant par $A$ ; il recoupe $\mathcal{C}$ en $D$ et $E$.
  Les cercles de centre $D$, puis de centre $E$, passant par $A$ se coupent en $A$ et en $A''=\inversion(A)$.
  
  
  \item $A''$ est le centre de $\mathcal{C}_0$. 
\myfigure{1}{  
  \tikzinput{fig_inversion17}\qquad\qquad   
  \tikzinput{fig_inversion18}
  }
\end{enumerate}


\begin{remarque*}
Notez, et ce sera important pour la suite, que si on connaît seulement trois points $A,B,C$
formant un triangle isocèle, alors la construction précédente permet de retrouver le contour du cercle $\mathcal{C}_0$ 
passant par $A,B,C$ et son centre.
\end{remarque*}

%---------------------------------------------------------------
\subsection{Preuve}

Dans une première étape nous montrons que $\inversion(A')$ est le centre
de $\mathcal{C}_0$ où $\inversion$ est l'inversion de centre $A$ et de cercle d'inversion
$\mathcal{C}$.

\begin{itemize}
 \item L'image de la droite $(BC)$ par l'inversion est un cercle passant par $B$, $C$ (qui sont invariant par $\inversion$) et $A$ (qui est le centre de $\inversion$), c'est donc bien $\mathcal{C}_0$.
 
 \item Notons $I$ le milieu de $[AA']$, c'est aussi le milieu de $[BC]$.
 Notons $F$ le point de $\mathcal{C}_0$ diamétralement opposé à $A$.
 
 \item L'image de $I$ par l'inversion est $F$, en effet d'une part $I \in (BC)$ donc
 $\inversion(I)$ est dans l'image de $(BC)$ qui est $\mathcal{C}_0$, d'autre part $\inversion(I)$ est sur la demi-droite $[AI)$. 
 Donc $AI \cdot AF = r^2$,
 ici $r$ désigne le rayon de $\mathcal{C}$.
 
 \item Notons $A''=\inversion(A')$ alors $AA' \cdot AA''= r^2$. Mais comme $AA' = 2 AI$
 alors $2 AI\cdot AA''= r^2$. Avec l'égalité du point précédent alors $AA'' = AF/2$, donc $A''$ est le milieu d'un diamètre, c'est bien le centre de $\mathcal{C}_0$.
\end{itemize}
\myfigure{1}{  
  \tikzinput{fig_inversion28}
  }

La dernière étape de notre construction est justifiée dans le paragraphe suivant.

%---------------------------------------------------------------
\subsection{Construction de l'inverse d'un point au compas seul}

\'Etant donné un cercle $\mathcal{C}$ de centre $\Omega$ et un point $M$
extérieur à $\mathcal{C}$, nous allons construire, avec le compas seulement, 
l'image de $M$ par l'inversion $\inversion$ de cercle $\mathcal{C}$.
\begin{itemize}
  \item Tracer le cercle $\mathcal{C}'$ de centre $M$ passant par $\Omega$,
  il recoupe $\mathcal{C}$ en $A$ et $B$.
  
  \item Tracer les cercles de centre $A$, puis de centre $B$, passant par 
  $\Omega$. Ils se coupent en $\Omega$ et en $M' = \inversion(M)$.
\end{itemize}
\myfigure{1}{
\tikzinput{fig_inversion20}
}


\begin{proof}
Le théorème de Pythagore dans le triangle $AIM$ donne
$AM ^2 - AI^2 = IM^2 = (\Omega M - I \Omega)^2$, or $AM=\Omega M$
donc $\Omega M^2 - AI^2 = \Omega M^2 + I\Omega^2 - 2 \Omega M \cdot I\Omega$.
Alors $\Omega M \cdot \Omega M' = \Omega M \cdot(2I\Omega) = I\Omega^2 + AI^2$.
Par le théorème de Pythagore, cette fois dans le triangle $AI\Omega$,
nous obtenons $\Omega M \cdot \Omega M' = A\Omega ^2 = r^2$. 
Comme en plus $M$, $M'$, $\Omega$ sont alignés (car sur la médiatrice de $[AB]$) alors
$M' = \inversion(M)$.
\end{proof}
\myfigure{1}{
\tikzinput{fig_inversion19}
}


\begin{remarque*}
Deux cercles, dont l'un est l'image de l'autre par une inversion sont homothétiques,
par une homothétie dont le centre est le centre de l'inversion. Par contre le rapport
dépend des cercles considérés.
\myfigure{1}{
\tikzinput{fig_inversion21}
}
\end{remarque*}


%---------------------------------------------------------------
\subsection{Théorème de Mohr-Masheroni}

\begin{theoreme}
Toute construction possible à la règle et au compas
est possible au compas seulement.
\end{theoreme}

Bien sûr il faut quand même exclure la tracé effectif des droites.

Un cercle c'est la donnée de deux points : son centre et un point de sa circonférence.
Une droite c'est la donnée de deux points distincts.

Une \defi{construction à la règle et au compas}, c'est partir de plusieurs points sur une
feuille ; vous pouvez maintenant tracer d’autres points, à partir de cercles et de droites en respectant les conditions
suivantes :
\begin{itemize}
  \item[(i)] vous pouvez tracer une droite entre deux points déjà construits,
  \item[(ii)]vous pouvez tracer un cercle dont le centre est un point construit et qui passe par un autre point construit,
  \item[(iii)] vous pouvez utiliser les points obtenus comme intersection de deux droites tracées, ou bien intersections d’une
droite et d’un cercle tracé, ou bien intersections de deux cercles tracés,
  \item[(iv)] vous pouvez utiliser les points obtenus comme intersections de deux cercles tracés.
\end{itemize}

Une \defi{construction au compas seul} c'est le même principe mais avec seulement les conditions
(ii) et (iv) !

\bigskip

Avant d'étudier la preuve, commençons par une série de constructions très faciles avec une règle et un compas mais plus subtiles sans la règle.

\textbf{Construction du symétrique de $C$ par rapport à la droite $(AB)$.}

Pour cela, tracer le cercle
centré en $A$ passant par $B$ et aussi le cercle centré en $B$ passant par $A$ (c'est deux fois la construction (ii)). Ces deux cercles s’intersectent en deux
points $C$, $C'$ (construction (iv)) ; 
$C'$ est le symétrique recherché.

Comme les points $C$, $C'$ sont à la même distance de $A$ que de $B$ alors on a aussi construit la médiatrice de $[AB]$.
Bien sûr on n'a pas tracer tous les points de cette médiatrices, mais seulement deux points qui la déterminent.

\myfigure{0.8}{
\tikzinput{fig_inversion29}
}

\bigskip
\textbf{Construction du symétrique de $B$ par rapport au point $A$.}

La construction est basée sur le tracé de la rosace : tracer le cercle $\mathcal{C}$
centré en $A$ passant par $B$, puis le cercle centré en $B$ passant par $A$, 
ils se coupent par exemple en $C$. Tracé le cercle centré en $C$ et passant par $B$,
il recoupe $\mathcal{C}$ en $D$ ; le cercle centré en $D$ recoupe $\mathcal{C}$ en $B'$ qui est le symétrique de
$B$ par rapport à $A$. 


\myfigure{1}{
\tikzinput{fig_inversion30}
}

\begin{remarque*}
Noter que pour l'instant nous n'avons pas le droit de relever la pointe du compas pour reporter une distance.
On peut seulement tracer un cercle dont on connaît un centre et un point de sa circonférence.
\end{remarque*}

\bigskip
\textbf{Construction du milieu d'un segment $[AB]$.}

Ce n'est pas facile du tout :
\begin{itemize}
  \item Tracer le cercle $\mathcal{C}$ de centre $A$ passant par $B$.
  \item Tracer, au compas seul, $A'$, le symétrique de $A$ par rapport au point $B$.
  \item Tracer, au compas seul, l'inverse $I = \inversion(A')$ par l'inversion $\inversion$ de cercle $\mathcal{C}$.
\end{itemize}

\myfigure{0.8}{
\tikzinput{fig_inversion33}
}

Preuve que cette construction est correcte. 
Soit $r$ le rayon du cercle $\mathcal{C}$. 
D'une part $AB=r$ donc $AA' = 2r$.
D'autre part, par la définition de l'inversion, $AI \cdot AA' = r^2$. 
Ainsi $AI = \frac{r^2}{2r} = \frac{r}{2} = \frac{AB}{2}$.
Conclusion $I$ est bien le milieu de $AB$.


\bigskip
\bigskip

Passons maintenant à la preuve du théorème de Mohr-Masheroni.
Comme les opérations (ii) et (iv) sont les seules autorisées il nous faut montrer comment réaliser 
l'opération (i) puis l'opération (iii) uniquement avec le compas.

\textbf{Intersection entre un cercle et une droite au compas seul.}

La droite est déterminée par deux points $A$ et $B$.
Le cercle $\mathcal{C}$ est donné par son centre $O$ et un point $C$ de sa circonférence.
La construction est très simple : on trace le symétrique $\mathcal{C'}$ de $\mathcal{C}$
par rapport à la droite $(AB)$, c'est le cercle centré en $O'=s_{(AB)}(O)$ passant par 
$C'=s_{(AB)}(C)$.

Les deux cercles $\mathcal{C}$ et $\mathcal{C'}$ ont les mêmes points d'intersection
que le cercle $\mathcal{C}$ et la droite $(AB)$.

\myfigure{1}{
\tikzinput{fig_inversion31}
}




\bigskip

\textbf{Intersection de deux droites au compas seul.}

C'est la construction la plus difficile.
Une première droite est donnée par les deux points $A$ et $B$, une seconde par deux $C$ et $D$.
On suppose que les droites $(AB)$ et $(CD)$ ne sont pas perpendiculaires (à vous de chercher comment faire si elles le sont !).

On trace au compas seul le symétrique $A'$ (resp. $B'$) de $A$ (resp. $B$) par rapport à $(CD)$.
On recommence : on trace le symétrique $C'$ (resp. $D'$) de $C$ (resp. $D$)  par rapport à $(A'B')$.
Et encore une dernière fois : on trace le symétrique $A''$ (resp. $B''$) de $A'$ (resp. $B'$) par rapport à $(C'D')$.

Notons $O$ le point d'intersection de $(AB)$ et $(CD)$ que l'on veut tracer.
Le points $A,A',A''$ sont situés à la même distance de $O$, en plus les distance
$AA'$ et $A'A''$ sont égales. Justification : 
$A'$ est aussi l'image de $A$ par la rotation d'angle $2\alpha$ ou $\alpha$ est l'angle entre les deux droites.
Idem : pour $A''$ par rapport à $A$,...

Notons $\mathcal{C}_0$ le cercle passant par $A$, $A'$ et $A''$.
Alors d'une part son centre est $O$ et d'autre part
par la construction du problème de Napoléon et à l'aide de l'inversion, on sait 
construire au compas seul ce centre $O$ et le cercle $\mathcal{C}_0$ 
(voir la remarque à la fin de la construction de Napoléon).


\myfigure{1}{
\tikzinput{fig_inversion32}
}

%%%%%%%%%%%%%%%%%%%%%%%%%%%%%%%%%%%%%%%%%%%%%%%%%%%%%%%%%%%%%%%%
\section{Exercices}

% Autres idées :
% \begin{enumerate}
%  \item inversion comme composé d'homothéties,
%  \item cercles tangents,
%  \item cercles orthogonaux,
%  \item puissance d'un point,
%  \item systèmes élémentaires pour Kempe,
%  \item théorème de Mohr-Mascheroni.
% \end{enumerate}


%%%%%%%%%%%%%%%%%%%%%%%%%%%%%%%%%%%%%%%%%%%%%%%%%
\subsection{L'inversion}


\begin{exercicecours}[Cercle et droite]
\sauteligne
\begin{enumerate}
 
 \item Donner l'équation complexe du cercle $\mathcal{C}$ de centre $2+\ii$ et de rayon $\sqrt{5}$.
 Donner une paramétrisation polaire du cercle $\mathcal{C}'$ de centre $3\ii$ et de rayon $1$.
 Trouver l'intersection de $\mathcal{C}$ et  $\mathcal{C}'$.
 
 \item Soit $\mathcal{C}$ le cercle de centre $1+\ii$ et de rayon $1$ 
 et $\mathcal{D}$ la droite passant par les points d'affixe $1$ et $1+\ii$.
 Déterminer l'image de  $\mathcal{C}$ et $\mathcal{D}$ par chacune des transformations suivantes:
   \begin{enumerate}
     \item $z \mapsto 2 z e^{-\frac{\ii\pi}{4}} + \ii$ ;
     \item $z \mapsto (1+\ii)\overline{(z+\ii)}$.
   \end{enumerate}
   
\myfigure{1}{\tikzinput{fig_inversion_exo_cercle}}   



 \item Donner l'équation complexe de la droite $(AB)$ passant par les points d'affixe $\alpha \in \Cc$ et $\beta\in \Cc$.
 
 \item \'Etant fixés $\omega \in \Cc^*$, $k\in \Rr$, trouver l'équation complexe des droites perpendiculaires à la droite d'équation $\bar \omega z + \omega \bar z = k$.

\end{enumerate}
\end{exercicecours}


\begin{exercicecours}[Inversion]

Soit l'inversion $\inversion$ de centre $O$ et de rayon $1$ définie par $\inversion(z) = \frac{1}{\bar z}$.
Déterminer les images, par $\inversion$, de chacune des figures suivantes :

\myfigure{1}{\tikzinput{fig_inversion_exo_inversion}}

\begin{enumerate}
 \item la droite $\mathcal{D}_1$ d'équation réelle $(y=x)$ ;
 \item la droite $\mathcal{D}_2$ d'équation réelle $(y=x+1)$;
 \item le cercle $\mathcal{C}_1$ de centre $(0,2)$ et de rayon $2$;
 \item le cercle $\mathcal{C}_2$ de centre $(0,2)$ et de rayon $1$.
\end{enumerate}


Mêmes questions avec l'inversion $\inversion(z) = 2\ii + \frac{1}{\,\overline{z-2\ii}\,}$.

\emph{Indications :} Deux méthodes sont possibles : un calcul avec les équations complexes
ou alors d'abord reconnaître (à l'aide du cours) la nature de l'image et la calculer.
\end{exercicecours}


\begin{exercicecours}[Inversion et cocyclicité]
\label{exo:inversion_cocyclicite}
\sauteligne
\begin{enumerate}
  \item Soient $A, B, C, D$ quatre points d'affixe $a,b,c,d$.
Montrer que $A, B, C, D$ sont cocyliques ou alignés si
et seulement si  
$$\frac{\frac{d-a}{d-b}}{\frac{c-a}{c-b}} \in \Rr.$$

\emph{Indication.} 
Utiliser le théorème de l'angle inscrit.

  \item Soit $\inversion$ une inversion, $M,N$ deux points du plan distincts du centre de l'inversion.
  Montrer que $M, N, \inversion(M), \inversion(N)$ sont cocyliques ou alignés.
  
\emph{Indication.} Choisir le repère de telle sorte que l'inversion s'écrive $\inversion(z)=  \frac{1}{\bar z}$.

  \item \textbf{Application 1.} Soit $\inversion$ une inversion de centre $\Omega$. Soient $M, N$ deux points du plan
  (avec $\Omega$, $M$, $N$ distincts).
  Supposons connus et placés les quatre points distincts $\Omega$, $M$, $N$ et $\inversion(M)$, construire à la règle et au compas le point $\inversion(N)$.
  
  \item \textbf{Application 2.} Soit $\inversion$ une inversion de centre $\Omega$. Soient $M$, $\inversion(M)$ donnés
    (avec $\Omega$, $M$, $\inversion(M)$ distincts). 
    Construire à la règle et au compas le cercle invariant par l'inversion $\inversion$.
    
  \item \textbf{Application 3.} Montrer que si un cercle contient un point et son inverse alors il est globalement invariant.
\end{enumerate}
\end{exercicecours}


\begin{exercicecours}[Théorème de Ptolémé]
\label{exo:inversion_ptoleme}

% source et correction
%http://debart.pagesperso-orange.fr/geoplan/config_cercle_classique.html#ch3

``Les sommets d'un quadrilatère convexe sont cocycliques si et seulement si la somme des produits des côtés opposés est égale au produit des diagonales.''
En d'autres termes si et seulement si : 
$$AB \times CD + AD\times CB = AC \times BD.$$


\myfigure{1}{\tikzinput{fig_inversion_exo_ptoleme}}

\begin{enumerate}
 \item \textbf{Sens direct.}
 Supposons que $A, B, C, D$ appartiennent à un même cercle.
 Soit $I$ le point de $[AC]$ tel qu'on ait l'égalité des angles : $\widehat{ABI} = \widehat{CBD}$.
   \begin{enumerate}
     \item Montrer que les triangles $CBD$ et $IBA$ sont semblables (trouver une autre égalité d'angle) ; en déduire
     $AB \times CD = IA \times BD$.
     \item Montrer aussi que les triangles $ABD$ et $IBC$ sont semblables ; en déduire que $AD \times BC = IC \times BD$.
     \item Conclure.
   \end{enumerate}  
   
   
 \item \textbf{Préliminaire à la réciproque.}
 Soit $\inversion$  une inversion de centre $\Omega$ et de rapport $r^2$, soient $M, N$ deux points du plan et
   $M'=\inversion(M)$, $N'=\inversion(N)$ leurs images. Montrer la relation entre les distances:
   $$M'N' = \frac{r^2 MN}{\Omega M \times \Omega N}.$$
   \emph{Indications.} On pourra supposer que $\Omega$ est l'origine du plan, puis faire
   les calculs avec l'écriture complexe de $\inversion$.
   
 \item \textbf{Réciproque.}  
 Soient $A,B,C,D$ quatre points vérifiant $AB \times CD + AD\times CB = AC \times BD.$
  Soit $\inversion$ l'inversion de centre $D$ et d'un rapport $r^2$ fixé. Soit $A'=\inversion(A)$, $B'=\inversion(B)$, 
  $C'=\inversion(C)$.
  \begin{enumerate}
   \item Calculer $A'B'$, $B'C'$ et $A'C'$ et montrer que $A'B'+B'C'=A'C'$. Qu'en déduire pour $A', B', C'$ ?
   \item En déduire que $A,B,C,D$ sont cocyliques.
 \end{enumerate}  
\end{enumerate}
\end{exercicecours}



%%%%%%%%%%%%%%%%%%%%%%%%%%%%%%%%%%%%%%%%%%%%%%%%%
\subsection{Homographie}

\begin{exercicecours}[Homographie et birapport]

Soient $A, B, C, D$ quatre points d'affixe $a,b,c,d$.
Le \emph{birapport} de ces quatre points est 
$$[a:b:c:d] = \frac{\frac{d-a}{d-b}}{\frac{c-a}{c-b}}.$$

\begin{enumerate}
 \item Montrer que les homographies préservent le birapport (c'est-à-dire:
 $[h(a):h(b):h(c):h(d)]=[a:b:c:d]$, pour toute homographie $h$ et tous quadruplets).
 
 \emph{Indication.} Se ramener à l'étude de chacune des transformations $z \mapsto z+\alpha$,
 $z \mapsto \lambda \cdot z$, $z \mapsto \frac{1}{z}$.
 
 \item En déduire que l'image d'un cercle-droite par une homographie est un cercle-droite.
 
  \emph{Indication.} Utiliser le critère de cocylicité.
 
 \item \emph{Exemple.} Soit $h(z) = \frac{z-1}{z-\ii}$. Soit $\mathcal{C}$
 le cercle de centre $1 + \ii$ et de rayon $1$. 
 \`A l'aide de l'image de trois points, trouver l'image de $\mathcal{C}$.
 
\end{enumerate}
\end{exercicecours}


\begin{exercicecours}[Homographie et birapport (bis)]
Soient $a,b,c,d \in \Cc$ distincts.
\begin{enumerate}
 \item Montrer qu'il existe une unique homographie $h$ tel que $h(a)=\infty$, $h(b)=0$,
 $h(c)=1$. 
 
 \item Montrer que $h(d) = [a:b:c:d]$.
\end{enumerate}
\end{exercicecours}

\begin{exercicecours}[Homographie et matrices]
Notons $\mathcal{H}$ l'ensemble des homographies définies par $h(z)= \frac{az+b}{cz+d}$ avec $a,b,c,d \in \Cc$ et $ad-bc \neq 0$.
Notons $GL_2(\Cc)$ l'ensemble des matrices $\left(\begin{smallmatrix} a & b \\ c & d \end{smallmatrix}\right)$ avec
$a,b,c,d \in \Cc$ et $ad-bc \neq 0$.
\begin{enumerate}
 \item Montrer que $(GL_2(\Cc),\times)$ est un groupe.
 \item Montrer que $(\mathcal{H},\circ)$ est un groupe.
 \item Soit $\Phi : GL_2(\Cc) \longrightarrow \mathcal{H}$ l'application qui à 
 $\left(\begin{smallmatrix} a & b \\ c & d \end{smallmatrix}\right)$ associe l'homographie $h$ définie par $h(z)= \frac{az+b}{cz+d}$.
 Montrer que $\Phi$ définie un morphisme du groupe $(GL_2(\Cc),\times)$ vers le groupe $(\mathcal{H},\circ)$.
 \item Calculer le noyau de $\Phi$.
\end{enumerate}
\end{exercicecours}


%%%%%%%%%%%%%%%%%%%%%%%%%%%%%%%%%%%%%%%%%%%%%%%%%
\subsection{Dispositifs mécaniques}


\begin{exercicecours}[Inverseur de Hart]
\label{exo:inversion_hart}
L'inverseur de Hart est un dispositif mécanique constitué de quatre tiges articulées avec
$AB=CD$, $AC=BD$. Soient $\Omega, P, P'$ des points des tiges tels que:
$\overrightarrow{B\Omega} = k \overrightarrow{BA}$, $\overrightarrow{BP} = k \overrightarrow{BD}$,
 $\overrightarrow{CP'} = k \overrightarrow{CA}$ (avec $0< k < 1$ fixé).

\begin{enumerate}
 \item Montrer que le quadrilatère $ABCD$ est un trapèze dont les sommets sont cocycliques.
 
 \item \`A l'aide du théorème de Ptolémé calculer $AD\cdot BC$. En déduire une valeur de $\Omega P\cdot \Omega P'$.
 Montrer que $P'$ est l'image de  $P$ par une inversion que l'on précisera.
 
 \item En déduire un dispositif mécanique qui transforme un cercle en droite.
\end{enumerate} 

\myfigure{1}{\tikzinput{fig_inversion_exo_hart}}
\end{exercicecours}


% %%%%%%%%%%%%%%%%%%%%%%%%%%%%%%%%%%%%%%%%%%%%%%%%%
% \section*{Construction au compas seulement}
% 
% \exercice{}
% \enonce[Théorème de Mohr-Mascheroni]
% ``Toute construction à la règle et au compas est possible au compas seulement.''
% Sauf le tracé effectif des droites !
% Une construction à la règle et au compas est une succession d'étapes élémentaires :
% \begin{enumerate}
%  \item[(i)] Tracer un cercle connaissant son centre et son rayon.
%  \item[(ii)] Tracer l'intersection de deux cercles.
%  \item[(iii)] Tracer l'intersection d'une droite et d'un cercle.
%  \item[(iv)] Tracer l'intersection de deux cercles.
% \end{enumerate}
% Un cercle est la donnée d'un centre et d'un rayon (mais de toute façon 
% par la construction de Napoléon, connaissant le contour
% d'un cercle on peut retrouver son centre et son rayon). Les étapes (i) et (ii) sont donc claires.
% Une droite est la donnée de deux points seulement.  Prouvons (iii) et (iv).
% \begin{enumerate}
%  \item 
%  \item
%  \item Donner une construction au compas seul pour (iv).
% \end{enumerate}
% \finenonce
% 
% \finexercice

\auteurs{Arnaud Bodin}

\finchapitre
\end{document}



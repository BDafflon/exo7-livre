%%%%%%%%%%%%%%%%%% PREAMBULE %%%%%%%%%%%%%%%%%%

\documentclass[11pt]{book}

%----- Principaux packages -----
\usepackage{amsfonts,amsmath,amssymb}
\usepackage[utf8]{inputenc}
\usepackage[T1]{fontenc}
\usepackage[francais]{babel}
\usepackage{fancybox}
\usepackage{graphicx}
\usepackage{float}
\usepackage[usenames, x11names]{xcolor}
\usepackage{fouriernc}
\usepackage{ifthen}
\usepackage{titletoc} % Required for manipulating the tables of contents

%----- Ensembles : entiers, reels, complexes -----
\newcommand{\Nn}{\mathbb{N}} \newcommand{\N}{\mathbb{N}}
\newcommand{\Zz}{\mathbb{Z}} \newcommand{\Z}{\mathbb{Z}}
\newcommand{\Qq}{\mathbb{Q}} \newcommand{\Q}{\mathbb{Q}}
\newcommand{\Rr}{\mathbb{R}} \newcommand{\R}{\mathbb{R}}
\newcommand{\Cc}{\mathbb{C}} \newcommand{\C}{\mathbb{C}}
\newcommand{\Kk}{\mathbb{K}} \newcommand{\K}{\mathbb{K}}

%----- Modifications de symboles -----
\renewcommand{\epsilon}{\varepsilon}
\renewcommand{\Re}{\mathop{\text{Re}}\nolimits}
\renewcommand{\Im}{\mathop{\text{Im}}\nolimits}
\renewcommand{\ge}{\geqslant} \renewcommand{\geq}{\geqslant}
\renewcommand{\le}{\leqslant} \renewcommand{\leq}{\leqslant}


%----- Fonctions usuelles -----
\newcommand{\ch}{\mathop{\mathrm{ch}}\nolimits}
\newcommand{\sh}{\mathop{\mathrm{sh}}\nolimits}
\renewcommand{\tanh}{\mathop{\mathrm{th}}\nolimits}
\newcommand{\cotan}{\mathop{\mathrm{cotan}}\nolimits}
\newcommand{\Arcsin}{\mathop{\mathrm{arcsin}}\nolimits}
\newcommand{\Arccos}{\mathop{\mathrm{arccos}}\nolimits}
\newcommand{\Arctan}{\mathop{\mathrm{arctan}}\nolimits}
\newcommand{\Argsh}{\mathop{\mathrm{argsh}}\nolimits}
\newcommand{\Argch}{\mathop{\mathrm{argch}}\nolimits}
\newcommand{\Argth}{\mathop{\mathrm{argth}}\nolimits}
\newcommand{\pgcd}{\mathop{\mathrm{pgcd}}\nolimits} 

 %----- Commandes divers ------
\newcommand{\ii}{\mathrm{i}}
\newcommand{\dd}{\mathrm{d}}
\newcommand{\Ker}{\mathop{\text{Ker}}\nolimits}
\newcommand{\id}{\mathop{\text{id}}\nolimits}
\newcommand{\Card}{\mathop{\text{Card}}\nolimits}
\newcommand{\Vect}{\mathop{\text{Vect}}\nolimits}
\newcommand{\Mat}{\mathop{\mathrm{Mat}}\nolimits}
\newcommand{\rg}{\mathop{\text{rg}}\nolimits}
\newcommand{\tr}{\mathop{\text{tr}}\nolimits}
\newcommand{\ppcm}{\mathop{\text{ppcm}}\nolimits}


%----- Definition d'un terme -----
\newcommand{\defi}[1]{{\color{myorange}\textbf{\emph{#1}}}}
\newcommand{\evidence}[1]{{\color{blue}\textbf{\emph{#1}}}}
\newcommand{\assertion}[1]{{\og\emph{#1}\fg}} % pour chapitre logique


%-----  Package liens hypertexts ----- 
\usepackage{hyperref}
\hypersetup{colorlinks=true, linkcolor=blue, urlcolor=blue,
pdftitle={Exo7 - Cours de mathématiques}, pdfauthor={Exo7}}

%----- Commandes tikz -----
\usepackage{tikz}
\usepackage{pgfplots}
%\pgfplotsset{compat=newest}
%\pgfplotsset{compat=2.10}
\usetikzlibrary{calc}
\usetikzlibrary{shadows}
\usetikzlibrary{arrows}
\usetikzlibrary{patterns}
\usetikzlibrary{matrix}

%-----  Multi colonnes- ---- 
\usepackage{multicol}
\setlength{\columnseprule}{0.2mm}

%-----  Package d'importation ----- 
\usepackage{import}

%-----  Package unités -----
\usepackage{siunitx}
\sisetup{locale = FR,detect-all,per-mode = symbol}
%\sisetup{round-mode = places}

%----- Format de la page ------
\usepackage[a4paper]{geometry}
\geometry{top=2.5cm, bottom=2cm, left=2.5cm, right=2.5cm, marginparsep=1cm}
\setlength{\parindent}{0cm}


% Espace interligne (+ 10%)
\renewcommand{\baselinestretch}{1.1}

% Microtype
%\frenchbsetup{AutoSpacePunctuation=false}
\usepackage[babel=true, activate={true,nocompatibility},final,tracking=true,spacing=true,factor=1100,stretch=10,shrink=10]{microtype}
% activate={true,nocompatibility} - activate protrusion and expansion
% final - enable microtype; use "draft" to disable
% tracking=true, kerning=true, spacing=true - activate these techniques
% factor=1100 - add 10% to the protrusion amount (default is 1000)
% stretch=10, shrink=10 - reduce stretchability/shrinkability (default is 20/20)
% Overfull box

\emergencystretch=1.5em

% Police des titres
\newcommand{\policeheader}{\color{Ivory4}\fontencoding{T1}\fontfamily{fvs}\fontseries{b}\fontshape{n}\selectfont}
\newcommand{\policechapter}{\color{Red3}\fontencoding{T1}\fontfamily{fvs}\fontseries{bx}\fontshape{n}\selectfont}
\newcommand{\policesection}{\color{Tomato3}\fontencoding{T1}\fontfamily{fvs}\fontseries{bx}\fontshape{n}\selectfont}
\newcommand{\policesubsection}{\color{Tomato4}\fontencoding{T1}\fontfamily{fvs}\fontseries{b}\fontshape{n}\selectfont}


% Haut (et pied) de page
\usepackage{fancyhdr}

\fancypagestyle{plain}{%
\fancyhf{} % clear all header and footer fields
\fancyfoot[C]{\fcolorbox{white}{Ivory4}{\policeheader\color{white}\thepage}} % except the center
\renewcommand{\headrulewidth}{0pt}
\renewcommand{\footrulewidth}{0pt}}

\pagestyle{fancy}
\renewcommand{\chaptermark}[1]{\markboth{#1}{}}
%\lhead{\policeheader\leftmark}
%\chead{}
%\rhead{\fcolorbox{white}{Ivory4}{\policeheader\color{white}\thepage}}
\fancyhead[LO,RE]{\policeheader\leftmark}
\fancyhead[RO,LE]{\fcolorbox{white}{Ivory4}{\policeheader\color{white}\thepage}}
\lfoot{}
\cfoot{}
\rfoot{}

\renewcommand{\headrulewidth}{1pt}
\renewcommand{\headrule}{{\color{Ivory4}%
\hrule width\headwidth height\headrulewidth \vskip-\headrulewidth}}


%----- New Style sections ----- 
\usepackage{sectsty}
\sectionfont{\policesection}
\subsectionfont{\policesubsection}
\subsubsectionfont{\policesubsection}

\makeatletter
\renewcommand{\thesection}{\@arabic\c@section}
% \renewcommand{\thechapter}{}
% \renewcommand{\chaptername}{}
\makeatother

% Numérotation dans la marge + point après numéro
\makeatletter \def\@seccntformat#1{\llap{\csname the#1\endcsname.\ }} \makeatother


%----- New Chapter Title ----- 
\usepackage[explicit]{titlesec}
\newcommand*\chapterlabel{}
\titleformat{\chapter}
  {\gdef\chapterlabel{}
   \Large\policechapter}
  {\gdef\chapterlabel{\thechapter\quad }}{0pt}
  {\begin{tikzpicture}[remember picture,overlay]
    \node[yshift=-5cm] at (current page.north west)
      {\begin{tikzpicture}[remember picture, overlay]
        \draw[fill=Ivory2,Ivory2] (0,0) rectangle
          (\paperwidth,3cm);
        \node[anchor=east,xshift=.9\paperwidth,rectangle,
              rounded corners=18pt,inner sep=11pt,
              fill=Firebrick1]
              {\color{white}\chapterlabel#1};
       \end{tikzpicture}
      };
      \node[yshift=-3.5cm,xshift=2cm] at (current page.north west)      
      {{\normalsize\LogoExoSept{2.5}}
       };
   \end{tikzpicture}
  }
\titlespacing*{\chapter}{0pt}{150pt}{-80pt}


%Link to video Youtube

% variable myvideo : 0 no video, otherwise youtube reference
\newcommand{\video}[1]{\def\myvideo{#1}}
\newcommand{\insertvideo}[2]{\video{#1}%
{\small\texttt{\href{http://www.youtube.com/watch?v=\myvideo}{Vidéo $\blacksquare$ #2}}}}

% Liens vers les fiches d'exercices
\newcommand{\mafiche}[1]{\def\mymafiche{#1}}
\newcommand{\insertfiche}[2]{\mafiche{#1}%
{\small\texttt{\href{http://exo7.emath.fr/ficpdf/\mymafiche}{Exercices $\blacklozenge$ #2}}}}



%----- Sommaire et mini-sommaires ------ from latextemplate
% Package Titletoc


\contentsmargin{0cm} % Removes the default margin
% Chapter text styling
\titlecontents{chapter}[1.25cm] % Indentation
{\addvspace{10pt}\large%
\color{Red3}\fontencoding{T1}\fontfamily{fvs}\fontseries{m}\fontshape{n}\selectfont
} % Spacing and font options for chapters
{\contentslabel[\Large\thecontentslabel]{1.25cm}} % Chapter number
{}  
{\;\titlerule*[.5pc]{.}\;\thecontentspage} % Page number

% Section text styling
\titlecontents{section}[1.25cm] % Indentation
{\addvspace{2pt}%
\color{Tomato3}\fontencoding{T1}\fontfamily{fvs}\fontseries{m}\fontshape{n}\selectfont
} % Spacing and font options for sections
{\contentslabel[\thecontentslabel]{1.25cm}} % Section number
{}
{\;\titlerule*[.5pc]{.}\;\thecontentspage} % Page number
[]

% Subsection text styling
% \titlecontents{subsection}[1.25cm] % Indentation
% {\addvspace{1pt}\small%
% \color{Tomato4}\fontencoding{T1}\fontfamily{fvs}\fontseries{m}\fontshape{n}\selectfont
% } % Spacing and font options for subsections
% {\contentslabel[\thecontentslabel]{1.25cm}} % Subsection number
% {}
% {\sffamily\;\titlerule*[.5pc]{.}\;\thecontentspage} % Page number
% [] 

%----- Mini-sommaires ------ from latextemplate

% Section text styling
\titlecontents{lsection}[0.7em]{\color{Tomato3}\small\sffamily}{\contentslabel[\thecontentslabel]{1em}}{}{}

% Subsection text styling
\titlecontents{lsubsection}[1em]{\normalfont\footnotesize\sffamily}{\contentslabel[\thecontentslabel]{1.25cm}}{}{}


%----- Logo Exo7 ------
\definecolor{myred}{rgb}{0.93,0.26,0}
\definecolor{myorange}{rgb}{0.97,0.58,0}
\definecolor{myyellow}{rgb}{1,0.86,0}

\newcommand{\LogoExoSept}[1]{  % input : echelle
{\usefont{U}{cmss}{bx}{n}
\begin{tikzpicture}[scale=0.1*#1,transform shape]
  \fill[color=myorange] (0,0)--(4,0)--(4,-4)--(0,-4)--cycle;
  \fill[color=myred] (0,0)--(0,3)--(-3,3)--(-3,0)--cycle;
  \fill[color=myyellow] (4,0)--(7,4)--(3,7)--(0,3)--cycle;
  \node[scale=5] at (3.5,3.5) {Exo7};
\end{tikzpicture}}
}

%------ Titre livre -------------
\newcommand{\montitre}[1]{
\thispagestyle{empty}
~ \vfil
\begin{center}
{\Huge \policechapter #1}  \\
\vspace{2cm}
\LogoExoSept{5}
\end{center}
\addtocontents{toc}{\setcounter{tocdepth}{1}}
{\footnotesize
\bigskip
\tableofcontents
}
\finsommaire
\pagestyle{fancy}}

\newcommand{\finsommaire}{
\vfill\par\href{http://www.unisciel.fr/}{\includegraphics[scale=1]{logo_unisciel.png}}
\hfill\hspace*{9ex}\begin{minipage}{0.5\textwidth}\vspace*{-5.5ex}\center Cours et exercices 
de maths \\ \texttt{\href{http://exo7.emath.fr}{exo7.emath.fr}}\end{minipage}
\hfill\href{http://www.univ-lille1.fr/}{\includegraphics[scale=0.15]{logo_lille1_new.png}}
% \vspace*{-3.5ex}
\centerline{Licence Creative Commons \ -- BY-NC-SA -- \ 3.0 FR}
\newpage}

%%%%% SUITE A COMMENTER SI CHAPITRE SEUL %%%%%

%------ Chapitre dans livre -------------

% \newcommand{\chapitre}[1]{          % pour chapitre dans livre
% \chapter{#1}%\vspace*{-21.5ex}
% \thispagestyle{empty}
% \startcontents
% \printcontents{l}{1}{\addtocontents{ptc}{\setcounter{tocdepth}{1}}}
% 
% %\LogoExoSept{1.6} \vspace*{5ex}
% %{\small\minitoc}
% \vspace*{2ex} 
% }
% 
% \newcommand{\finchapitre}{}  % pour chapitre dans livre

%----------------------------------

%%%%% SUITE A COMMENTER SI LIVRE %%%%%

%------ Chapitre seul -------------

\newcommand{\chapitre}[1]{          % pour chapitre seul
\begin{document}
\chapter*{#1}
%\tableofcontents
}

\newcommand{\finchapitre}{\end{document}} % pour chapitre seul


%%%%% FIN  %%%%%
%----------------------------------

%----- Personnalisation tiret itemize -----
\renewcommand{\FrenchLabelItem}{{\bf--}}  % Evite confusion avec signe -

%----- Personnalisation pour les theoremes,... -----

%\usepackage[babel=true,kerning=true]{microtype} % to avoid conflict tikz/babel
\usepackage[framemethod=tikz]{mdframed}

%---- Theorem style ----
\mdfdefinestyle{theoremestyle}{%
linecolor=Tomato3,
middlelinewidth=2pt,%
roundcorner=5pt,
frametitlerule=false,%
frametitlefont=\bfseries,
frametitlealignment=\raggedright,
theoremseparator={.},
apptotikzsetting={\tikzset{mdfframetitlebackground/.append style={%
shade,left color=DarkOliveGreen4!30, right color=DarkOliveGreen4!10}}},
apptotikzsetting={\tikzset{mdfbackground/.append style={%
shade,left color=DarkOliveGreen3!30, right color=DarkOliveGreen3!10}}},
frametitlerulecolor=green!60,
frametitlerulewidth=1pt,
innertopmargin=0.6\topskip,
splittopskip=2pt, splitbottomskip=2pt,
skipabove=5pt, skipbelow=5pt,
needspace=10\baselineskip,
}
\mdtheorem[style=theoremestyle]{theoreme}{Théorème}
\mdtheorem[style=theoremestyle]{proposition}{Proposition}
\mdtheorem[style=theoremestyle]{propriete}{Propriété}
\mdtheorem[style=theoremestyle]{lemme}{Lemme}
\mdtheorem[style=theoremestyle]{corollaire}{Corollaire}



     
%---- Definition style ----
\mdfdefinestyle{definitionstyle}{%
linecolor=Tomato3,
middlelinewidth=2pt,%
rightline=false,
leftline=true,
topline=false,
bottomline=false,
%roundcorner=5pt,
frametitleaboveskip=3pt,
frametitlebelowskip=3pt,
frametitlerule=false,%
frametitlefont=\bfseries,
frametitlealignment=\raggedright,
theoremseparator={.},
apptotikzsetting={\tikzset{mdfframetitlebackground/.append style={%
shade,left color=DarkOliveGreen4!20, right color=DarkOliveGreen4!5}}},
apptotikzsetting={\tikzset{mdfbackground/.append style={%
shade,left color=DarkOliveGreen3!20, right color=DarkOliveGreen3!5}}},
frametitlerulecolor=green!60,
frametitlerulewidth=1pt,
innertopmargin=0.6\topskip,
splittopskip=2pt, splitbottomskip=2pt,
skipabove=5pt, skipbelow=5pt,
needspace=7\baselineskip,
}
\mdtheorem[style=definitionstyle]{definition}{Définition}

     
%---- Definition style ----
\mdfdefinestyle{exemplestyle}{%
linecolor=Tomato3,
middlelinewidth=2pt,%
rightline=false,
leftline=true,
topline=false,
bottomline=false,
%roundcorner=5pt,
frametitleaboveskip=3pt,
frametitlebelowskip=3pt,
frametitlerule=false,%
frametitlefont=\bfseries,
frametitlealignment=\raggedright,
theoremseparator={.},
apptotikzsetting={\tikzset{mdfframetitlebackground/.append style={%
shade,left color=DarkOliveGreen4!20, right color=white}}},
apptotikzsetting={\tikzset{mdfbackground/.append style={%
shade,left color=white, right color=white}}},
frametitlerulecolor=green!60,
frametitlerulewidth=1pt,
innertopmargin=0.6\topskip,
splittopskip=2pt, splitbottomskip=2pt,
skipabove=5pt, skipbelow=5pt,
needspace=7\baselineskip,
}

\mdtheorem[style=exemplestyle]{remarque}{Remarque}
\mdtheorem[style=exemplestyle]{exemple}{Exemple}
\mdtheorem[style=exemplestyle]{tp}{Travaux pratiques}
\mdtheorem[style=exemplestyle]{exercicecours}{Exercice}

%---- Definition style ----
\mdfdefinestyle{proofstyle}{%
middlelinewidth=2pt,%
leftmargin=12pt,
linecolor=Ivory4,
rightline=false,
leftline=true,
topline=false,
bottomline=false,
font=\small,
%roundcorner=5pt,
frametitleaboveskip=3pt,
frametitlebelowskip=3pt,
frametitlerule=false,%
frametitlefont=\bfseries,
frametitlealignment=\raggedright,
theoremseparator={}, %theoremseparator={.},
apptotikzsetting={\tikzset{mdfframetitlebackground/.append style={%
shade,left color=DarkOliveGreen4!20, right color=white}}},
apptotikzsetting={\tikzset{mdfbackground/.append style={%
shade,left color=DarkOliveGreen3!10, right color=white}}},
frametitlerulecolor=green!60,
frametitlerulewidth=1pt,
innertopmargin=0.6\topskip,
splittopskip=2pt, splitbottomskip=2pt,
skipabove=5pt, skipbelow=5pt,
needspace=7\baselineskip,
}
\mdtheorem[style=proofstyle]{proof}{Démonstration}
\renewcommand\theproof{} % No numeratotion to proof


%---- Mini-exercices style ----
\mdfdefinestyle{miniexercicesstyle}{%
linecolor=Ivory4,
middlelinewidth=2pt,%
roundcorner=0pt,
frametitlerule=false,%
frametitlefont=\bfseries,
frametitlealignment=\raggedright,
theoremseparator={.},
apptotikzsetting={\tikzset{mdfframetitlebackground/.append style={%
shade,left color=DarkOliveGreen4!30, right color=DarkOliveGreen4!10}}},
apptotikzsetting={\tikzset{mdfbackground/.append style={%
shade,left color=DarkOliveGreen3!30, right color=DarkOliveGreen3!5}}},
frametitlerulecolor=green!60,
frametitlerulewidth=0pt,
innertopmargin=0.6\topskip,
splittopskip=2pt, splitbottomskip=2pt,
skipabove=25pt, skipbelow=5pt,
needspace=7\baselineskip,
}
\mdtheorem[style=miniexercicesstyle]{miniexercices}{Mini-exercices}
\renewcommand\theminiexercices{} % No numeratotion to mini-exercices

%---- Algo style ----
\mdfdefinestyle{algostyle}{%
linecolor=DarkOliveGreen3,
middlelinewidth=2pt,%
roundcorner=0pt,
rightline=false,
leftline=true,
topline=false,
bottomline=false,
frametitlerule=false,%
frametitlefont=\bfseries,
frametitlealignment=\raggedright,
theoremseparator={},
apptotikzsetting={\tikzset{mdfframetitlebackground/.append style={%
shade,left color=DarkOliveGreen4!30, right color=white}}},
apptotikzsetting={\tikzset{mdfbackground/.append style={%
shade,left color=white, right color=white}}},
frametitlerulecolor=green!60,
frametitlerulewidth=1pt,
innertopmargin=0.6\topskip,
splittopskip=2pt, splitbottomskip=2pt,
skipabove=2pt, skipbelow=5pt,
needspace=7\baselineskip,
}
\mdtheorem[style=algostyle]{algo}{Code}
\renewcommand\thealgo{} % No numeratotion to algo


%---- Auteurs style ----
\mdfdefinestyle{auteurstyle}{%
linecolor=black!50,
middlelinewidth=0pt,%
roundcorner=5pt,
frametitlerule=false,%
frametitlefont=\bfseries,
fontcolor=red,
frametitlealignment=\raggedright,
theoremseparator={.},
apptotikzsetting={\tikzset{mdfframetitlebackground/.append style={%
shade,left color=black!15, right color=black!1}}},
apptotikzsetting={\tikzset{mdfbackground/.append style={%
shade,left color=black!1, right color=black!5}}},
frametitlerulecolor=green!60,
frametitlerulewidth=1pt,
innertopmargin=0.6\topskip,
splittopskip=2pt, splitbottomskip=2pt,
skipabove=5pt, skipbelow=0pt,
needspace=7\baselineskip,
}
\mdtheorem[style=auteurstyle]{auteur}{\color{black!80} Auteurs}
\newcommand{\auteurs}[1]{
\vfill
\begin{auteur*}
\color{black!70}
#1  
\end{auteur*}}



%----- Commandes anti-beamer -----
\newcommand{\pause}{}  % permet de mettre des \pause dans beamer pas dans poly
\newcommand{\beameronly}[1]{}

%------ Figures ------
\def\myscale{1} % par défaut 
\newcommand{\myfigure}[2]{  % entrée : echelle, fichier figure
\def\myscale{#1}\begin{center}\footnotesize{#2}\end{center}}


%------ Encadrement des formules ------
 \usepackage{fancybox}
% %\setlength{\fboxsep}{7pt}
\newcommand{\mybox}[1]{\begin{center}\shadowbox{#1}\end{center}}
\newcommand{\myboxinline}[1]{\raisebox{-2ex}{\shadowbox{#1}}}

% \tikzstyle{myboxstyle} = [draw=black, ultra thick, fill=white,
%     rectangle, drop shadow={
%                         top color=gray,
%                         bottom color=white,
%                         fill=gray,
%                         opacity=0.4,
%                         shadow xshift=3pt,
%                         shadow yshift=-3pt
%                         }, inner sep=10pt, inner ysep=10pt]
% \newmdenv[tikzsetting={fill=green!20},
%           roundcorner=10pt,shadow=true]{myshadowbox}
% \newcommand{\mybox}[1]{\begin{center}
% \begin{tikzpicture}\node[myboxstyle] (box) {#1};\end{tikzpicture}\end{center}}
% \newcommand{\myboxinline}[1]{\begin{tikzpicture}\node[myboxstyle] (box) {#1};\end{tikzpicture}}

% %----- Structure des exercices ------
% \newtheoremstyle{styleexo}% name
% {2ex}% Space above
% {3ex}% Space below
% {}% Body font
% {}% Indent amount 1
% {\bfseries} % Theorem head font
% {}% Punctuation after theorem head
% {\newline}% Space after theorem head 2
% {}% Theorem head spec (can be left empty, meaning ‘normal’)
% 
% \theoremstyle{styleexo}
% \newtheorem{exo}{Exercice}
% \newtheorem{ind}{Indications}
% \newtheorem{cor}{Correction}
% 
% \newcommand{\exercice}[1]{} \newcommand{\finexercice}{}
% %\newcommand{\exercice}[1]{{\tiny\texttt{#1}}\vspace{-2ex}} % pour afficher le numero absolu, l'auteur...
% \newcommand{\enonce}{\begin{exo}} \newcommand{\finenonce}{\end{exo}}
% \newcommand{\indication}{\begin{ind}} \newcommand{\finindication}{\end{ind}}
% \newcommand{\correction}{\begin{cor}} \newcommand{\fincorrection}{\end{cor}}
% \newcommand{\noindication}{\stepcounter{ind}}
% \newcommand{\nocorrection}{\stepcounter{cor}}
% \newcommand{\fiche}[1]{} \newcommand{\finfiche}{}
% \newcommand{\titre}[1]{\centerline{\large \bf #1}}
% \newcommand{\addcommand}[1]{}
% 
% 
%  
 
%------ Algorithmes ------

\newcommand{\Python}{\texttt{Python}}
\newcommand{\Sage}{\texttt{Sage}}

% Pour afficher du code
\usepackage{listingsutf8}

\lstset{
  language=Python,
  upquote=true,
  columns=flexible,
  keepspaces=true,
  basicstyle=\ttfamily,
  commentstyle=\color{gray},
  showspaces=false,
  showstringspaces=false
}

% \makeatletter
% \def\lst@outputspace{\lst@bkgcolor\empty\color{white}}
% \makeatother

\makeatletter
\def\lst@outputspace{{\ifx\lst@bkgcolor\empty\color{white}\else
\lst@bkgcolor\fi\lst@visiblespace}}
\lst@keepspacestrue
\lst@keepspacestrue 
\makeatother

% Code inline
\newcommand{\codeinline}[1]{\lstinline!#1!}

% Long code
\newcommand{\insertcode}[2]{
\begin{algo}[{\sl #2}]
\lstinputlisting[inputencoding=utf8/latin1]{../#1}  
\end{algo}
}

% \newcommand{\insertcodebis}[2]{
% \begin{algo}[#2]
% \lstinputlisting[firstline=10,inputencoding=utf8/latin1]{../#1}  
% \end{algo}
% }
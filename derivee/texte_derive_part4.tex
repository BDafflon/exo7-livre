
%%%%%%%%%%%%%%%%%% PREAMBULE %%%%%%%%%%%%%%%%%%


\documentclass[12pt]{article}

\usepackage{amsfonts,amsmath,amssymb,amsthm}
\usepackage[utf8]{inputenc}
\usepackage[T1]{fontenc}
\usepackage[francais]{babel}


% packages
\usepackage{amsfonts,amsmath,amssymb,amsthm}
\usepackage[utf8]{inputenc}
\usepackage[T1]{fontenc}
%\usepackage{lmodern}

\usepackage[francais]{babel}
\usepackage{fancybox}
\usepackage{graphicx}

\usepackage{float}

%\usepackage[usenames, x11names]{xcolor}
\usepackage{tikz}
\usepackage{datetime}

\usepackage{mathptmx}
%\usepackage{fouriernc}
%\usepackage{newcent}
\usepackage[mathcal,mathbf]{euler}

%\usepackage{palatino}
%\usepackage{newcent}


% Commande spéciale prompteur

%\usepackage{mathptmx}
%\usepackage[mathcal,mathbf]{euler}
%\usepackage{mathpple,multido}

\usepackage[a4paper]{geometry}
\geometry{top=2cm, bottom=2cm, left=1cm, right=1cm, marginparsep=1cm}

\newcommand{\change}{{\color{red}\rule{\textwidth}{1mm}\\}}

\newcounter{mydiapo}

\newcommand{\diapo}{\newpage
\hfill {\normalsize  Diapo \themydiapo \quad \texttt{[\jobname]}} \\
\stepcounter{mydiapo}}


%%%%%%% COULEURS %%%%%%%%%%

% Pour blanc sur noir :
%\pagecolor[rgb]{0.5,0.5,0.5}
% \pagecolor[rgb]{0,0,0}
% \color[rgb]{1,1,1}



%\DeclareFixedFont{\myfont}{U}{cmss}{bx}{n}{18pt}
\newcommand{\debuttexte}{
%%%%%%%%%%%%% FONTES %%%%%%%%%%%%%
\renewcommand{\baselinestretch}{1.5}
\usefont{U}{cmss}{bx}{n}
\bfseries

% Taille normale : commenter le reste !
%Taille Arnaud
%\fontsize{19}{19}\selectfont

% Taille Barbara
%\fontsize{21}{22}\selectfont

%Taille François
%\fontsize{25}{30}\selectfont

%Taille Pascal
%\fontsize{25}{30}\selectfont

%Taille Laura
%\fontsize{30}{35}\selectfont


%\myfont
%\usefont{U}{cmss}{bx}{n}

%\Huge
%\addtolength{\parskip}{\baselineskip}
}


% \usepackage{hyperref}
% \hypersetup{colorlinks=true, linkcolor=blue, urlcolor=blue,
% pdftitle={Exo7 - Exercices de mathématiques}, pdfauthor={Exo7}}


%section
% \usepackage{sectsty}
% \allsectionsfont{\bf}
%\sectionfont{\color{Tomato3}\upshape\selectfont}
%\subsectionfont{\color{Tomato4}\upshape\selectfont}

%----- Ensembles : entiers, reels, complexes -----
\newcommand{\Nn}{\mathbb{N}} \newcommand{\N}{\mathbb{N}}
\newcommand{\Zz}{\mathbb{Z}} \newcommand{\Z}{\mathbb{Z}}
\newcommand{\Qq}{\mathbb{Q}} \newcommand{\Q}{\mathbb{Q}}
\newcommand{\Rr}{\mathbb{R}} \newcommand{\R}{\mathbb{R}}
\newcommand{\Cc}{\mathbb{C}} 
\newcommand{\Kk}{\mathbb{K}} \newcommand{\K}{\mathbb{K}}

%----- Modifications de symboles -----
\renewcommand{\epsilon}{\varepsilon}
\renewcommand{\Re}{\mathop{\text{Re}}\nolimits}
\renewcommand{\Im}{\mathop{\text{Im}}\nolimits}
%\newcommand{\llbracket}{\left[\kern-0.15em\left[}
%\newcommand{\rrbracket}{\right]\kern-0.15em\right]}

\renewcommand{\ge}{\geqslant}
\renewcommand{\geq}{\geqslant}
\renewcommand{\le}{\leqslant}
\renewcommand{\leq}{\leqslant}

%----- Fonctions usuelles -----
\newcommand{\ch}{\mathop{\mathrm{ch}}\nolimits}
\newcommand{\sh}{\mathop{\mathrm{sh}}\nolimits}
\renewcommand{\tanh}{\mathop{\mathrm{th}}\nolimits}
\newcommand{\cotan}{\mathop{\mathrm{cotan}}\nolimits}
\newcommand{\Arcsin}{\mathop{\mathrm{Arcsin}}\nolimits}
\newcommand{\Arccos}{\mathop{\mathrm{Arccos}}\nolimits}
\newcommand{\Arctan}{\mathop{\mathrm{Arctan}}\nolimits}
\newcommand{\Argsh}{\mathop{\mathrm{Argsh}}\nolimits}
\newcommand{\Argch}{\mathop{\mathrm{Argch}}\nolimits}
\newcommand{\Argth}{\mathop{\mathrm{Argth}}\nolimits}
\newcommand{\pgcd}{\mathop{\mathrm{pgcd}}\nolimits} 

\newcommand{\Card}{\mathop{\text{Card}}\nolimits}
\newcommand{\Ker}{\mathop{\text{Ker}}\nolimits}
\newcommand{\id}{\mathop{\text{id}}\nolimits}
\newcommand{\ii}{\mathrm{i}}
\newcommand{\dd}{\mathrm{d}}
\newcommand{\Vect}{\mathop{\text{Vect}}\nolimits}
\newcommand{\Mat}{\mathop{\mathrm{Mat}}\nolimits}
\newcommand{\rg}{\mathop{\text{rg}}\nolimits}
\newcommand{\tr}{\mathop{\text{tr}}\nolimits}
\newcommand{\ppcm}{\mathop{\text{ppcm}}\nolimits}

%----- Structure des exercices ------

\newtheoremstyle{styleexo}% name
{2ex}% Space above
{3ex}% Space below
{}% Body font
{}% Indent amount 1
{\bfseries} % Theorem head font
{}% Punctuation after theorem head
{\newline}% Space after theorem head 2
{}% Theorem head spec (can be left empty, meaning ‘normal’)

%\theoremstyle{styleexo}
\newtheorem{exo}{Exercice}
\newtheorem{ind}{Indications}
\newtheorem{cor}{Correction}


\newcommand{\exercice}[1]{} \newcommand{\finexercice}{}
%\newcommand{\exercice}[1]{{\tiny\texttt{#1}}\vspace{-2ex}} % pour afficher le numero absolu, l'auteur...
\newcommand{\enonce}{\begin{exo}} \newcommand{\finenonce}{\end{exo}}
\newcommand{\indication}{\begin{ind}} \newcommand{\finindication}{\end{ind}}
\newcommand{\correction}{\begin{cor}} \newcommand{\fincorrection}{\end{cor}}

\newcommand{\noindication}{\stepcounter{ind}}
\newcommand{\nocorrection}{\stepcounter{cor}}

\newcommand{\fiche}[1]{} \newcommand{\finfiche}{}
\newcommand{\titre}[1]{\centerline{\large \bf #1}}
\newcommand{\addcommand}[1]{}
\newcommand{\video}[1]{}

% Marge
\newcommand{\mymargin}[1]{\marginpar{{\small #1}}}



%----- Presentation ------
\setlength{\parindent}{0cm}

%\newcommand{\ExoSept}{\href{http://exo7.emath.fr}{\textbf{\textsf{Exo7}}}}

\definecolor{myred}{rgb}{0.93,0.26,0}
\definecolor{myorange}{rgb}{0.97,0.58,0}
\definecolor{myyellow}{rgb}{1,0.86,0}

\newcommand{\LogoExoSept}[1]{  % input : echelle
{\usefont{U}{cmss}{bx}{n}
\begin{tikzpicture}[scale=0.1*#1,transform shape]
  \fill[color=myorange] (0,0)--(4,0)--(4,-4)--(0,-4)--cycle;
  \fill[color=myred] (0,0)--(0,3)--(-3,3)--(-3,0)--cycle;
  \fill[color=myyellow] (4,0)--(7,4)--(3,7)--(0,3)--cycle;
  \node[scale=5] at (3.5,3.5) {Exo7};
\end{tikzpicture}}
}



\theoremstyle{definition}
%\newtheorem{proposition}{Proposition}
%\newtheorem{exemple}{Exemple}
%\newtheorem{theoreme}{Théorème}
\newtheorem{lemme}{Lemme}
\newtheorem{corollaire}{Corollaire}
%\newtheorem*{remarque*}{Remarque}
%\newtheorem*{miniexercice}{Mini-exercices}
%\newtheorem{definition}{Définition}




%definition d'un terme
\newcommand{\defi}[1]{{\color{myorange}\textbf{\emph{#1}}}}
\newcommand{\evidence}[1]{{\color{blue}\textbf{\emph{#1}}}}



 %----- Commandes divers ------

\newcommand{\codeinline}[1]{\texttt{#1}}

%%%%%%%%%%%%%%%%%%%%%%%%%%%%%%%%%%%%%%%%%%%%%%%%%%%%%%%%%%%%%
%%%%%%%%%%%%%%%%%%%%%%%%%%%%%%%%%%%%%%%%%%%%%%%%%%%%%%%%%%%%%



\begin{document}

\debuttexte

%%%%%%%%%%%%%%%%%%%%%%%%%%%%%%%%%%%%%%%%%%%%%%%%%%%%%%%%%%%
\diapo

\change

Nous terminons ce chapitre sur la dérivation par un théorème important 
à la fois d'un point de vie théorique mais aussi pratique : le théorème des accroissements finis.

\change

Après avoir vu l'énoncé et la démonstration 

\change

nous l'appliquerons pour montrer qu'une fonction est croissante si et seulement si sa dérivée est positive

\change

Nous en déduirons aussi l'inégalité des accroissements finis.

\change

Nous terminons avec la règle de L'Hospital qui permet de calculer certaines limites.

%%%%%%%%%%%%%%%%%%%%%%%%%%%%%%%%%%%%%%%%%%%%%%%%%%%%%%%%%%%
\diapo

Voici le théorème des accroissements finis

Soit $f:[a,b] \to \Rr$ une fonction continue sur l'intervalle fermé $[a,b]$ 
et dérivable sur l'intervalle ouvert $]a,b[$.

Alors il existe une valeur $c\in]a,b[$ telle que 
$f(b)-f(a)= f'(c) \; (b-a)$

\change

Voici l'interprétation géométrique de cet énoncé : 
il existe au moins un point du graphe de $f$ où la tangente est 
parallèle à la droite $(AB)$ où $A=(a,f(a))$ et $B=(b,f(b))$.


%%%%%%%%%%%%%%%%%%%%%%%%%%%%%%%%%%%%%%%%%%%%%%%%%%%%%%%%%%%
\diapo

Passons à la preuve du théorème des accroissements finis ;

remarquons que les hypothèses sont similaires à celles du théorème de Rolle.

Et en fait c'est ce théorème que nous appliquer.

\change

Posons $\ell= \frac{f(b)-f(a)}{b-a}$ et définissons une nouvelle fonction
par $g(x) = f(x) - \ell \cdot (x-a)$.

\change

En premier lieu  $g(a)=f(a)$, 

\change

Mais on a aussi $g(b)=f(b)- \frac{f(b)-f(a)}{b-a} \cdot (b-a) = f(a)$.

\change

Remarquons que --comme $f$--
$g$ est une fonction continue sur l'intervalle fermé $[a,b]$ 
et dérivable sur l'intervalle ouvert $]a,b[$.


D'autre part $g$ prend la même valeur en $a$ et $b$
alors par le théorème de Rolle, 
il existe $c \in ]a,b[$ tel que $g'(c) =0$.

\change

Mais $g'(x) = f'(x) - \ell$.

\change

Ce qui donne $f'(c)= \frac{f(b)-f(a)}{b-a}$.  


%%%%%%%%%%%%%%%%%%%%%%%%%%%%%%%%%%%%%%%%%%%%%%%%%%%%%%%%%%%
\diapo

\change

Appliquons ceci afin de prouver un des résultats favoris des étudiants :

Pour une fonction dérivable :
$f'(x) \ge 0 \quad \iff \quad$ $f$ est croissante 

\change

De même 

 $f'(x) \le 0 \quad \iff \quad$ $f$ est décroissante ;

\change

Et par conséquent 

$f$ est constante ssi $f'(x) = 0$ pour tout $x$

\change

On a des résultat un peu plus précis si $f'(x) > 0$
alors  $f$ est strictement croissante ;

(mais la réciproque est fausse : par exemple 
la fonction $x^3$ est strictement croissante et pourtant sa dérivée s'annule en $0$.)

\change

Et de même si $f'(x) < 0$ alors $f$ est strictement décroissante.



%%%%%%%%%%%%%%%%%%%%%%%%%%%%%%%%%%%%%%%%%%%%%%%%%%%%%%%%%%%
\diapo

Prouvons l'équivalence 

$f'(x) \ge 0 \quad \iff \quad$ $f$ est croissante 

\change

Pour le sens direct on suppose la dérivée positive.

Et on prend deux éléments $x,y \in ]a,b[$ avec $x\le y$. 

\change

Alors par le théorème des accroissements finis, il existe $c\in]x,y[$ tel que
$f(x)-f(y) = f'(c) (x-y)$. 

\change

Mais comme $f'(c) \ge 0$ et que $x-y \le 0$ alors on a $f(x)-f(y) \le 0$.

\change

Cela implique que $f(x) \le f(y)$. 

Ceci étant vrai pour tout $x,y$ alors $f$ est croissante.

\change

Passons à la réciproque.

On suppose cette fois ci que  $f$ est croissante et on on fixe $x\in]a,b[$.

\change

Pour tout $y > x$ on a $y-x>0$ et $f(y)-f(x)\ge 0$, 

\change


Ainsi le taux d'accroissement $\frac{f(y)-f(x)}{y-x}$ est positif. 


\change

Lorsque l'on fait tendre $y$ vers $x$  
alors ce taux d'accroissement 
tend vers la dérivée de $f$ en $x$.

Le taux d'accroissement étant toujours positif la limite $f'(x)$ est positive ou nulle.


%%%%%%%%%%%%%%%%%%%%%%%%%%%%%%%%%%%%%%%%%%%%%%%%%%%%%%%%%%%
\diapo

Une application importante du théorème des accroissements finis est l'inégalité des accroissements finis :

On part d'une fonction $f : I \to \Rr$ définie et dérivable sur un intervalle ouvert $I$.

On suppose qu'il existe une constante $M$ tel que pour tout 
$x \in I$, $\big|f'(x)\big| \le M$ (c'est-à-dire que la dérivée est bornée par $M$)



alors
$\forall x,y \in I \qquad \big| f(x)-f(y) \big| \le M |x-y|$

\change

Pour la preuve on fixe $x,y \in I$, 

\change

Par le théorème des accroissements finis 
il existe alors $c$ entre $x$ et $y$ tel que $f(x)-f(y)=f'(c)(x-y)$ 

\change

Mais comme $|f'(c)| \le M$ alors cette égalité devient l'inégalité 
$\big| f(x)-f(y) \big| \le M |x-y|$. Comme nous le voulions !



%%%%%%%%%%%%%%%%%%%%%%%%%%%%%%%%%%%%%%%%%%%%%%%%%%%%%%%%%%%
\diapo

Appliquons le théorème des accroissements finis ou plus précisément l'inégalité des accroissements finis
à la fonctions $\sin(x)$. 


\change

La dérivée est $\cos x$ elle est donc bornée (an valeur absolue) par $1$ quelque soit $x\in \Rr$.

\change

L'inégalité des accroissements finis s'écrit alors :

\change

$$\text{pour tous } x,y \in \Rr \qquad |\sin x - \sin y | \le |x-y|.$$

\change

En particulier si l'on fixe $y=0$ alors
on obtient 
$|\sin x| \le |x|$

\change



Sur le dessin cela se traduit ainsi le graphe de $\sin x$ est compris entre les droites 
d'équation $y=+x$ et $y=-x$. Même chose pour le graphe de $-\sin x$.

Ce qui est un résultat particulièrement intéressant pour $x$ proche de $0$.


%%%%%%%%%%%%%%%%%%%%%%%%%%%%%%%%%%%%%%%%%%%%%%%%%%%%%%%%%%%
\diapo

On termine avec une autre application du théorème des accroissements finis : 
la règle de l'Hospital
qui permet de calculer certaines limites.

On part de deux fonctions dérivables
 $f,g$  et  un réel $x_0$.
 
\change

Voici la règle de l'Hospital :

Si  $\frac{f'(x)}{g'(x)}$ admet une limite $\ell$ fini lorsque $x\to x_0$

alors $\frac{f(x)}{g(x)}$ tend aussi vers $\ell$, lorsque $x\to x_0$


\change

Pour que cela marche il faut les hypothèses suivantes


nos deux fonctions $f$ et $g$ s'annule en $x_0$.

et en dehors de $x_0$ la dérivée de $g$ ne s'annule pas.


Notez qu'au départ $f$ et $g$ s'annulent en $x_0$ donc le quotient
$\frac{f(x)}{g(x)}$ est une forme indéterminée.

La règle de l'Hospital permet de lever cette indétermination :
Si le quotient des dérivées tend vers $\ell$ alors le quotient des fonctions tend aussi  
vers $\ell$.




%%%%%%%%%%%%%%%%%%%%%%%%%%%%%%%%%%%%%%%%%%%%%%%%%%%%%%%%%%%
\diapo

Nous allons mettre en oeuvre la règle de l'Hospital pour calculer 
la limite en $1$ du quotient $\frac{\ln(x^2+x-1)}{\ln(x)}$.

\change

Commençons par le numérateur 
on pose  $f(x)=\ln(x^2+x-1)$ : en $1$, $f(1)=0$

et on calcule la dérivée  $f'(x)=\frac{2x+1}{x^2+x-1}$,

\change

Même chose pour le dénominateur $g(x)=\ln(x)$, on a $g(1)=0$ et $g'(x)=\frac 1x$, 

\change

Vérifions que l'on peut appliquer la règle de l'Hospital :

Prenons pour intervalle $I=]0,1]$ et $x_0=1$, 

on a déjà vu que $f$ et $g$ s'annule en $1$, nous avons bien une forme indéterminée.

De plus $g'(x)=1/x$ ne s'annule pas.

\change

Partons du quotient des dérivées : $\frac{f'(x)}{g'(x)}$

\change

cela vaut $\frac{2x+1}{x^2+x-1}$ diviser par $1/x$ donc $ \times x$

\change

ce qui donne $\frac{2x^2+x}{x^2+x-1}$

Ce n'est pas une forme indéterminée.

\change

Lorsque $x\to 1$ le numérateur tend vers $3$ et le dénominateur tend vers $1$.

Donc $\frac{f'(x)}{g'(x)}$ tend vers $3$.

\change

Par la règle de l'Hospital alors $\frac{f(x)}{g(x)} \xrightarrow[x\to 1]{} 3.$

%%%%%%%%%%%%%%%%%%%%%%%%%%%%%%%%%%%%%%%%%%%%%%%%%%%%%%%%%%%
\diapo

Pour terminer, voici la série de mini-exercices !

\end{document}
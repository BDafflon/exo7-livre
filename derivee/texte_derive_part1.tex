
%%%%%%%%%%%%%%%%%% PREAMBULE %%%%%%%%%%%%%%%%%%


\documentclass[12pt]{article}

\usepackage{amsfonts,amsmath,amssymb,amsthm}
\usepackage[utf8]{inputenc}
\usepackage[T1]{fontenc}
\usepackage[francais]{babel}


% packages
\usepackage{amsfonts,amsmath,amssymb,amsthm}
\usepackage[utf8]{inputenc}
\usepackage[T1]{fontenc}
%\usepackage{lmodern}

\usepackage[francais]{babel}
\usepackage{fancybox}
\usepackage{graphicx}

\usepackage{float}

%\usepackage[usenames, x11names]{xcolor}
\usepackage{tikz}
\usepackage{datetime}

\usepackage{mathptmx}
%\usepackage{fouriernc}
%\usepackage{newcent}
\usepackage[mathcal,mathbf]{euler}

%\usepackage{palatino}
%\usepackage{newcent}


% Commande spéciale prompteur

%\usepackage{mathptmx}
%\usepackage[mathcal,mathbf]{euler}
%\usepackage{mathpple,multido}

\usepackage[a4paper]{geometry}
\geometry{top=2cm, bottom=2cm, left=1cm, right=1cm, marginparsep=1cm}

\newcommand{\change}{{\color{red}\rule{\textwidth}{1mm}\\}}

\newcounter{mydiapo}

\newcommand{\diapo}{\newpage
\hfill {\normalsize  Diapo \themydiapo \quad \texttt{[\jobname]}} \\
\stepcounter{mydiapo}}


%%%%%%% COULEURS %%%%%%%%%%

% Pour blanc sur noir :
%\pagecolor[rgb]{0.5,0.5,0.5}
% \pagecolor[rgb]{0,0,0}
% \color[rgb]{1,1,1}



%\DeclareFixedFont{\myfont}{U}{cmss}{bx}{n}{18pt}
\newcommand{\debuttexte}{
%%%%%%%%%%%%% FONTES %%%%%%%%%%%%%
\renewcommand{\baselinestretch}{1.5}
\usefont{U}{cmss}{bx}{n}
\bfseries

% Taille normale : commenter le reste !
%Taille Arnaud
%\fontsize{19}{19}\selectfont

% Taille Barbara
%\fontsize{21}{22}\selectfont

%Taille François
%\fontsize{25}{30}\selectfont

%Taille Pascal
%\fontsize{25}{30}\selectfont

%Taille Laura
%\fontsize{30}{35}\selectfont


%\myfont
%\usefont{U}{cmss}{bx}{n}

%\Huge
%\addtolength{\parskip}{\baselineskip}
}


% \usepackage{hyperref}
% \hypersetup{colorlinks=true, linkcolor=blue, urlcolor=blue,
% pdftitle={Exo7 - Exercices de mathématiques}, pdfauthor={Exo7}}


%section
% \usepackage{sectsty}
% \allsectionsfont{\bf}
%\sectionfont{\color{Tomato3}\upshape\selectfont}
%\subsectionfont{\color{Tomato4}\upshape\selectfont}

%----- Ensembles : entiers, reels, complexes -----
\newcommand{\Nn}{\mathbb{N}} \newcommand{\N}{\mathbb{N}}
\newcommand{\Zz}{\mathbb{Z}} \newcommand{\Z}{\mathbb{Z}}
\newcommand{\Qq}{\mathbb{Q}} \newcommand{\Q}{\mathbb{Q}}
\newcommand{\Rr}{\mathbb{R}} \newcommand{\R}{\mathbb{R}}
\newcommand{\Cc}{\mathbb{C}} 
\newcommand{\Kk}{\mathbb{K}} \newcommand{\K}{\mathbb{K}}

%----- Modifications de symboles -----
\renewcommand{\epsilon}{\varepsilon}
\renewcommand{\Re}{\mathop{\text{Re}}\nolimits}
\renewcommand{\Im}{\mathop{\text{Im}}\nolimits}
%\newcommand{\llbracket}{\left[\kern-0.15em\left[}
%\newcommand{\rrbracket}{\right]\kern-0.15em\right]}

\renewcommand{\ge}{\geqslant}
\renewcommand{\geq}{\geqslant}
\renewcommand{\le}{\leqslant}
\renewcommand{\leq}{\leqslant}

%----- Fonctions usuelles -----
\newcommand{\ch}{\mathop{\mathrm{ch}}\nolimits}
\newcommand{\sh}{\mathop{\mathrm{sh}}\nolimits}
\renewcommand{\tanh}{\mathop{\mathrm{th}}\nolimits}
\newcommand{\cotan}{\mathop{\mathrm{cotan}}\nolimits}
\newcommand{\Arcsin}{\mathop{\mathrm{Arcsin}}\nolimits}
\newcommand{\Arccos}{\mathop{\mathrm{Arccos}}\nolimits}
\newcommand{\Arctan}{\mathop{\mathrm{Arctan}}\nolimits}
\newcommand{\Argsh}{\mathop{\mathrm{Argsh}}\nolimits}
\newcommand{\Argch}{\mathop{\mathrm{Argch}}\nolimits}
\newcommand{\Argth}{\mathop{\mathrm{Argth}}\nolimits}
\newcommand{\pgcd}{\mathop{\mathrm{pgcd}}\nolimits} 

\newcommand{\Card}{\mathop{\text{Card}}\nolimits}
\newcommand{\Ker}{\mathop{\text{Ker}}\nolimits}
\newcommand{\id}{\mathop{\text{id}}\nolimits}
\newcommand{\ii}{\mathrm{i}}
\newcommand{\dd}{\mathrm{d}}
\newcommand{\Vect}{\mathop{\text{Vect}}\nolimits}
\newcommand{\Mat}{\mathop{\mathrm{Mat}}\nolimits}
\newcommand{\rg}{\mathop{\text{rg}}\nolimits}
\newcommand{\tr}{\mathop{\text{tr}}\nolimits}
\newcommand{\ppcm}{\mathop{\text{ppcm}}\nolimits}

%----- Structure des exercices ------

\newtheoremstyle{styleexo}% name
{2ex}% Space above
{3ex}% Space below
{}% Body font
{}% Indent amount 1
{\bfseries} % Theorem head font
{}% Punctuation after theorem head
{\newline}% Space after theorem head 2
{}% Theorem head spec (can be left empty, meaning ‘normal’)

%\theoremstyle{styleexo}
\newtheorem{exo}{Exercice}
\newtheorem{ind}{Indications}
\newtheorem{cor}{Correction}


\newcommand{\exercice}[1]{} \newcommand{\finexercice}{}
%\newcommand{\exercice}[1]{{\tiny\texttt{#1}}\vspace{-2ex}} % pour afficher le numero absolu, l'auteur...
\newcommand{\enonce}{\begin{exo}} \newcommand{\finenonce}{\end{exo}}
\newcommand{\indication}{\begin{ind}} \newcommand{\finindication}{\end{ind}}
\newcommand{\correction}{\begin{cor}} \newcommand{\fincorrection}{\end{cor}}

\newcommand{\noindication}{\stepcounter{ind}}
\newcommand{\nocorrection}{\stepcounter{cor}}

\newcommand{\fiche}[1]{} \newcommand{\finfiche}{}
\newcommand{\titre}[1]{\centerline{\large \bf #1}}
\newcommand{\addcommand}[1]{}
\newcommand{\video}[1]{}

% Marge
\newcommand{\mymargin}[1]{\marginpar{{\small #1}}}



%----- Presentation ------
\setlength{\parindent}{0cm}

%\newcommand{\ExoSept}{\href{http://exo7.emath.fr}{\textbf{\textsf{Exo7}}}}

\definecolor{myred}{rgb}{0.93,0.26,0}
\definecolor{myorange}{rgb}{0.97,0.58,0}
\definecolor{myyellow}{rgb}{1,0.86,0}

\newcommand{\LogoExoSept}[1]{  % input : echelle
{\usefont{U}{cmss}{bx}{n}
\begin{tikzpicture}[scale=0.1*#1,transform shape]
  \fill[color=myorange] (0,0)--(4,0)--(4,-4)--(0,-4)--cycle;
  \fill[color=myred] (0,0)--(0,3)--(-3,3)--(-3,0)--cycle;
  \fill[color=myyellow] (4,0)--(7,4)--(3,7)--(0,3)--cycle;
  \node[scale=5] at (3.5,3.5) {Exo7};
\end{tikzpicture}}
}



\theoremstyle{definition}
%\newtheorem{proposition}{Proposition}
%\newtheorem{exemple}{Exemple}
%\newtheorem{theoreme}{Théorème}
\newtheorem{lemme}{Lemme}
\newtheorem{corollaire}{Corollaire}
%\newtheorem*{remarque*}{Remarque}
%\newtheorem*{miniexercice}{Mini-exercices}
%\newtheorem{definition}{Définition}




%definition d'un terme
\newcommand{\defi}[1]{{\color{myorange}\textbf{\emph{#1}}}}
\newcommand{\evidence}[1]{{\color{blue}\textbf{\emph{#1}}}}



 %----- Commandes divers ------

\newcommand{\codeinline}[1]{\texttt{#1}}

%%%%%%%%%%%%%%%%%%%%%%%%%%%%%%%%%%%%%%%%%%%%%%%%%%%%%%%%%%%%%
%%%%%%%%%%%%%%%%%%%%%%%%%%%%%%%%%%%%%%%%%%%%%%%%%%%%%%%%%%%%%



\begin{document}

\debuttexte

%%%%%%%%%%%%%%%%%%%%%%%%%%%%%%%%%%%%%%%%%%%%%%%%%%%%%%%%%%%
\diapo

\change

\change

Nous commençons ce chapitre consacré aux dérivées par la définition
de la dérivée d'une fonction et des exemples

\change

Géométriquement la calcul de la dérivée correspond à la pente de la tangente.

\change

Nous terminons avec d'autres écritures de la dérivée, utiles pour les applications.

%%%%%%%%%%%%%%%%%%%%%%%%%%%%%%%%%%%%%%%%%%%%%%%%%%%%%%%%%%%
\diapo

Motivons d'abord cette notion de dérivée.

Nous souhaitons calculer $\sqrt{1,01}$ ou du moins en trouver une valeur approchée.

\change

Comme $1,01$ est proche de $1$ et que $\sqrt{1}=1$ on se doute bien que $\sqrt{1,01}$
sera proche de $1$. 

Justifions cela !

\change

Si l'on appelle $f$ la fonction définie par $f(x)=\sqrt{x}$, alors la fonction $f$ est une fonction 
continue en $x_0=1$. 

\change

La continuité nous affirme que pour $x$ suffisamment proche de $x_0$,  
$f(x)$ est proche de $f(x_0)$. 

Cela revient à dire que pour $x$ au voisinage de $x_0$ on approche $f(x)$
par la constante $f(x_0)$ .

\change

Nous pouvons faire mieux qu'approcher notre fonction par une droite horizontale ! 

Essayons avec une droite quelconque. Quelle droite se rapproche le plus du graphe de $f$ autour de $x_0$ ? 
Elle doit passer par le point $(x_0,f(x_0))$ et doit <<coller>> le plus possible au graphe : 

\change

Cette droite c'est la tangente au graphe en $x_0$.

\change

Une équation de la tangente est 
$$y = (x-x_0) f'(x_0) + f(x_0)$$
où $f'(x_0)$ désigne le nombre dérivé de $f$ en $x_0$.

\change


On sait que pour $f(x)=\sqrt x$, on a $f'(x)=\frac{1}{2\sqrt x}$.

\change

Une équation de la tangente en $x_0=1$ est donc $y=(x-1)\frac12+1$.

\change

Et donc pour $x$ proche de $1$ on a $f(x) \approx (x-1)\frac12+1$.

\change


Qu'est ce que cela donne pour notre calcul de $\sqrt{1,01}$ ?


On pose $x=1,01$ donc $f(x) \approx 1+\frac12(x-1) = 1 + \frac{0,01}{2}=1,005$.

\change

Et c'est effectivement une très bonne de approximation de $\sqrt{1,01}=1,00498\ldots$.


%%%%%%%%%%%%%%%%%%%%%%%%%%%%%%%%%%%%%%%%%%%%%%%%%%%%%%%%%%%
\diapo

Définissons la dérivée en un point.

Soit $f : I \to \Rr$ une fonction, où $I$ est un intervalle ouvert de $\Rr$. 

Fixons un $x_0$ dans l'intervalle.

\change

On dit que $f$ est \defi{dérivable en $x_0$} si le 
\evidence{taux d'accroissement} $\frac{f(x)-f(x_0)}{x-x_0}$ 
admet une limite finie lorsque $x$ tend vers $x_0$.

\change

Lorsque elle existe cette la limite s'appelle 
le \defi{nombre dérivé} de $f$ en $x_0$ et on le note $f'(x_0)$. 

\change

En résumé, si elle existe 
la dérivée de $f$ en $x_0$ est
$\displaystyle f'(x_0)= \lim_{x\to x_0} \frac{f(x)-f(x_0)}{x-x_0}$

%%%%%%%%%%%%%%%%%%%%%%%%%%%%%%%%%%%%%%%%%%%%%%%%%%%%%%%%%%%
\diapo

Nous disons que $f$ est \defi{dérivable sur l'intervalle $I$} si $f$ est dérivable en tout point $x_0 \in I$.

\change

Nous définissons alors une nouvelle fonction, la \defi{fonction dérivée},

qui a $x \mapsto f'(x)$ le nombre dérivée en $x$.

\change

On note cette fonction $f'$ ou $\frac{df}{dx}$.

\change

Voyons un exemple simple :

Considérons la fonction définie par $f(x)=x^2$  et montrons qu'elle est dérivable en tout point $x_0 \in \Rr$.

\change 

On part du taux d'accroissement $\frac{f(x)-f(x_0)}{x-x_0}$

\change 

qui vaut ici $\frac{x^2-x_0^2}{x-x_0}$

\change 

on factorise le numérateur par l'identité remarquable $x^2-x_0^2=(x-x_0)(x+x_0)$

\change 

ce qui permet de simplifier le taux d'accroissement en $x+x_0$

\change 

Lorsque $x \to x_0$ alors la limite existe et vaut $2x_0.$

\change

Ainsi on a montré que $f$ était dérivable en $x_0$ 
et que son nombre dérivée est $2x_0$


Ceci est vrai pour tout $x_0$ de $\Rr$
donc $f$ est une fonction dérivable sur $\Rr$ et sa dérivée est 
définie par $f'(x)=2x$.

%%%%%%%%%%%%%%%%%%%%%%%%%%%%%%%%%%%%%%%%%%%%%%%%%%%%%%%%%%%
\diapo

On enchaîne avec un exemple un peu plus compliqué.

Montrons que la dérivée de la fonction sinus est la fonction cosinus.

\change


Commençons par calculer le nombre dérivé en $x_0=0$.

\change

Nous savons $\frac{\sin x}{x} \to 1$

\change

Mais le taux d'accroissement en $0$ qui est
 $\frac{f(x)-f(0)}{x-0} $ est exactement $\frac{\sin x}{x}$ donc tend vers $1$.

\change

Cela prouve que $f$ est dérivable en $0$ et $f'(0)=1$.

\change

Passons au cas de $x_0$ quelconque.

\change

On se rappelle de la formule 
$\sin p-\sin q = 2\sin \frac{p-q}{2}\cdot\cos\frac{p+q}{2}$

\change

Calculons le taux d'accroissement 
$\frac{f(x)-f(x_0)}{x-x_0} = \frac{\sin x - \sin x_0}{x-x_0}$

avec la formule cela s'écrit aussi 
$\frac{\sin \frac{x-x_0}{2}}{\frac{x-x_0}{2}} \cdot \cos \frac{x+x_0}{2}$.

\change

Lorsque $x\to x_0$ alors d'une part
$\cos \frac{x+x_0}{2}\to \cos x_0$,

\change

et d'autre part
en posant $u=\frac{x-x_0}{2}$ alors $u\to 0$ et on a $\frac {\sin u}u \to 1$.

\change

Ainsi la taux d'accroissement tend vers $\cos x_0$ 

\change

et donc $f$ est dérivable et $f'(x)=\cos x$.



%%%%%%%%%%%%%%%%%%%%%%%%%%%%%%%%%%%%%%%%%%%%%%%%%%%%%%%%%%%
\diapo

La \defi{tangente} au point $(x_0,f(x_0))$ est par définition
la droite d'équation $y = (x-x_0) f'(x_0) + f(x_0)$.



Le fait que la pente de la tangente soit $f'(x_0)$ se justifie géométriquement de la façon suivante.


\change
La droite qui passe par les points $(x_0,f(x_0))$ et $(x,f(x))$
a pour coefficient directeur le taux d'accroissement $\frac{f(x)-f(x_0)}{x-x_0}$.

\change

Voici une telle droite.

Lorsque l'on rapproche $x$ de $x_0$,

\change

\change

alors le coefficient directeur de la droite
tend vers $f'(x_0)$.


\change

\change

Et à la limite la droite obtenue est bien la tangente !


%%%%%%%%%%%%%%%%%%%%%%%%%%%%%%%%%%%%%%%%%%%%%%%%%%%%%%%%%%%
\diapo

Voici deux autres formulations de la dérivabilité de $f$ en $x_0$.



Tout d'abord $f$ est dérivable en $x_0$ si et seulement si 
$\displaystyle \lim_{h\to 0} \frac{f(x_0+h)-f(x_0)}{h}$ existe et est finie.

On en fait juste remplacer $x$ par $x_0+h$.

\change

Une autre façon d'écrire la dérivabilité est la suivante :
$f$ est dérivable en $x_0$ si et seulement s'il existe $\ell \in \Rr$ (qui en fait sera $f'(x_0)$)
et une fonction $\epsilon : I \to \Rr$ telle que $\epsilon(x) \xrightarrow[x\to x_0]{} 0$ avec
$f(x)=f(x_0)+(x-x_0) \ell + (x-x_0) \epsilon(x).$

\change


Cette dernière formulation se justifie aisément :
si l'on passe $f(x_0)$ de l'autre coté et que l'on divise par $x-x_0$
Alors on obtient 
$\frac{f(x)-f(x_0)}{x-x_0} = \ell + \epsilon(x).$
Donc dire qu'il existe une limite $\ell$ équivaut à ce que $f$ soit dérivable en $x_0$.


%%%%%%%%%%%%%%%%%%%%%%%%%%%%%%%%%%%%%%%%%%%%%%%%%%%%%%%%%%%
\diapo



On applique ces formulations pour démontrer le résultat suivant.
Soit $I$ un intervalle ouvert et $f : I \to \Rr$ une fonction.

\change

Si $f$ est dérivable en $x_0$ alors $f$ est continue en $x_0$.

\change

Et bien sûr on en déduit que : Si $f$ est dérivable sur $I$ alors $f$ est continue sur $I$.

Donc être dérivable implique être continue.

\change



On suppose que $f$ dérivable en $x_0$ et on veut montrer qu'elle est aussi continue en ce point.

Nous allons utiliser la deuxième formulation de la dérivabilité :

$f(x)= f(x_0)+\underbrace{(x-x_0) \ell}_{\to 0} + \underbrace{(x-x_0) \epsilon(x)}_{\to 0}.$

\change

Mais lorsque $x\to x_0$ alors $x-x_0\to 0$ donc $(x-x_0) \ell \to 0$.
Et de même $(x-x_0) \epsilon(x)\to 0$.

\change

Ainsi $f(x) \to f(x_0)$

\change

Ce qui prouve que $f$ est continue en $x_0$.


%%%%%%%%%%%%%%%%%%%%%%%%%%%%%%%%%%%%%%%%%%%%%%%%%%%%%%%%%%%
\diapo

Attention aux erreurs : 

si une fonction est dérivable alors elle est continue. 

Mais attention la réciproque est \textbf{fausse} !


\change

Retenez l'exemple de la fonction valeur absolue.

\change

Elle est continue partout donc y compris en $0$

mais elle n'est pas dérivable en $0$.




\change

En effet, le taux d'accroissement de $|x|$ à l'origine vérifie :
$\frac{f(x)-f(0)}{x-0}$

\change

$ = \frac{|x|}{x}=$

\change

$+1  \text{ si } x>0 $

\change

$-1  \text{ si } x < 0$


Il y a bien une limite à droite (qui vaut $+1$), une limite à gauche (qui vaut $-1$) mais elles ne sont pas égales :
il n'y a pas de limite en $0$. Ainsi $f$ n'est pas dérivable en $0$.


Cela se lit aussi sur le dessin il y a une demi-tangente à droite, une demi-tangente à gauche
mais elles ont des directions différentes.



%%%%%%%%%%%%%%%%%%%%%%%%%%%%%%%%%%%%%%%%%%%%%%%%%%%%%%%%%%%
\diapo

Voici des exercices d'entraînement afin de vérifier vos connaissances 
avant de passer à la suite !



\end{document}
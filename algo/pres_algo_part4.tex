
%%%%%%%%%%%%%%%%%% PREAMBULE %%%%%%%%%%%%%%%%%%

\documentclass[aspectratio=169,utf8]{beamer}
%\documentclass[aspectratio=169,handout]{beamer}

\usetheme{Boadilla}
%\usecolortheme{seahorse}
%\usecolortheme[RGB={245,66,24}]{structure}
\useoutertheme{infolines}

% packages
\usepackage{amsfonts,amsmath,amssymb,amsthm}
\usepackage[utf8]{inputenc}
\usepackage[T1]{fontenc}
\usepackage{lmodern}

\usepackage[francais]{babel}
\usepackage{fancybox}
\usepackage{graphicx}

\usepackage{float}
\usepackage{xfrac}

%\usepackage[usenames, x11names]{xcolor}
\usepackage{pgfplots}
\usepackage{datetime}


% ----------------------------------------------------------------------
% Pour les images
\usepackage{tikz}
\usetikzlibrary{calc,shadows,arrows.meta,patterns,matrix}

\newcommand{\tikzinput}[1]{\input{figures/#1.tikz}}
% --- les figures avec échelle éventuel
\newcommand{\myfigure}[2]{% entrée : échelle, fichier(s) figure à inclure
\begin{center}\small%
\tikzstyle{every picture}=[scale=1.0*#1]% mise en échelle + 0% (automatiquement annulé à la fin du groupe)
#2%
\end{center}}



%-----  Package unités -----
\usepackage{siunitx}
\sisetup{locale = FR,detect-all,per-mode = symbol}

%\usepackage{mathptmx}
%\usepackage{fouriernc}
%\usepackage{newcent}
%\usepackage[mathcal,mathbf]{euler}

%\usepackage{palatino}
%\usepackage{newcent}
% \usepackage[mathcal,mathbf]{euler}



% \usepackage{hyperref}
% \hypersetup{colorlinks=true, linkcolor=blue, urlcolor=blue,
% pdftitle={Exo7 - Exercices de mathématiques}, pdfauthor={Exo7}}


%section
% \usepackage{sectsty}
% \allsectionsfont{\bf}
%\sectionfont{\color{Tomato3}\upshape\selectfont}
%\subsectionfont{\color{Tomato4}\upshape\selectfont}

%----- Ensembles : entiers, reels, complexes -----
\newcommand{\Nn}{\mathbb{N}} \newcommand{\N}{\mathbb{N}}
\newcommand{\Zz}{\mathbb{Z}} \newcommand{\Z}{\mathbb{Z}}
\newcommand{\Qq}{\mathbb{Q}} \newcommand{\Q}{\mathbb{Q}}
\newcommand{\Rr}{\mathbb{R}} \newcommand{\R}{\mathbb{R}}
\newcommand{\Cc}{\mathbb{C}} 
\newcommand{\Kk}{\mathbb{K}} \newcommand{\K}{\mathbb{K}}

%----- Modifications de symboles -----
\renewcommand{\epsilon}{\varepsilon}
\renewcommand{\Re}{\mathop{\text{Re}}\nolimits}
\renewcommand{\Im}{\mathop{\text{Im}}\nolimits}
%\newcommand{\llbracket}{\left[\kern-0.15em\left[}
%\newcommand{\rrbracket}{\right]\kern-0.15em\right]}

\renewcommand{\ge}{\geqslant}
\renewcommand{\geq}{\geqslant}
\renewcommand{\le}{\leqslant}
\renewcommand{\leq}{\leqslant}
\renewcommand{\epsilon}{\varepsilon}

%----- Fonctions usuelles -----
\newcommand{\ch}{\mathop{\text{ch}}\nolimits}
\newcommand{\sh}{\mathop{\text{sh}}\nolimits}
\renewcommand{\tanh}{\mathop{\text{th}}\nolimits}
\newcommand{\cotan}{\mathop{\text{cotan}}\nolimits}
\newcommand{\Arcsin}{\mathop{\text{arcsin}}\nolimits}
\newcommand{\Arccos}{\mathop{\text{arccos}}\nolimits}
\newcommand{\Arctan}{\mathop{\text{arctan}}\nolimits}
\newcommand{\Argsh}{\mathop{\text{argsh}}\nolimits}
\newcommand{\Argch}{\mathop{\text{argch}}\nolimits}
\newcommand{\Argth}{\mathop{\text{argth}}\nolimits}
\newcommand{\pgcd}{\mathop{\text{pgcd}}\nolimits} 


%----- Commandes divers ------
\newcommand{\ii}{\mathrm{i}}
\newcommand{\dd}{\text{d}}
\newcommand{\id}{\mathop{\text{id}}\nolimits}
\newcommand{\Ker}{\mathop{\text{Ker}}\nolimits}
\newcommand{\Card}{\mathop{\text{Card}}\nolimits}
\newcommand{\Vect}{\mathop{\text{Vect}}\nolimits}
\newcommand{\Mat}{\mathop{\text{Mat}}\nolimits}
\newcommand{\rg}{\mathop{\text{rg}}\nolimits}
\newcommand{\tr}{\mathop{\text{tr}}\nolimits}


%----- Structure des exercices ------

\newtheoremstyle{styleexo}% name
{2ex}% Space above
{3ex}% Space below
{}% Body font
{}% Indent amount 1
{\bfseries} % Theorem head font
{}% Punctuation after theorem head
{\newline}% Space after theorem head 2
{}% Theorem head spec (can be left empty, meaning ‘normal’)

%\theoremstyle{styleexo}
\newtheorem{exo}{Exercice}
\newtheorem{ind}{Indications}
\newtheorem{cor}{Correction}


\newcommand{\exercice}[1]{} \newcommand{\finexercice}{}
%\newcommand{\exercice}[1]{{\tiny\texttt{#1}}\vspace{-2ex}} % pour afficher le numero absolu, l'auteur...
\newcommand{\enonce}{\begin{exo}} \newcommand{\finenonce}{\end{exo}}
\newcommand{\indication}{\begin{ind}} \newcommand{\finindication}{\end{ind}}
\newcommand{\correction}{\begin{cor}} \newcommand{\fincorrection}{\end{cor}}

\newcommand{\noindication}{\stepcounter{ind}}
\newcommand{\nocorrection}{\stepcounter{cor}}

\newcommand{\fiche}[1]{} \newcommand{\finfiche}{}
\newcommand{\titre}[1]{\centerline{\large \bf #1}}
\newcommand{\addcommand}[1]{}
\newcommand{\video}[1]{}

% Marge
\newcommand{\mymargin}[1]{\marginpar{{\small #1}}}

\def\noqed{\renewcommand{\qedsymbol}{}}


%----- Presentation ------
\setlength{\parindent}{0cm}

%\newcommand{\ExoSept}{\href{http://exo7.emath.fr}{\textbf{\textsf{Exo7}}}}

\definecolor{myred}{rgb}{0.93,0.26,0}
\definecolor{myorange}{rgb}{0.97,0.58,0}
\definecolor{myyellow}{rgb}{1,0.86,0}

\newcommand{\LogoExoSept}[1]{  % input : echelle
{\usefont{U}{cmss}{bx}{n}
\begin{tikzpicture}[scale=0.1*#1,transform shape]
  \fill[color=myorange] (0,0)--(4,0)--(4,-4)--(0,-4)--cycle;
  \fill[color=myred] (0,0)--(0,3)--(-3,3)--(-3,0)--cycle;
  \fill[color=myyellow] (4,0)--(7,4)--(3,7)--(0,3)--cycle;
  \node[scale=5] at (3.5,3.5) {Exo7};
\end{tikzpicture}}
}


\newcommand{\debutmontitre}{
  \author{} \date{} 
  \thispagestyle{empty}
  \hspace*{-10ex}
  \begin{minipage}{\textwidth}
    \titlepage  
  \vspace*{-2.5cm}
  \begin{center}
    \LogoExoSept{2.5}
  \end{center}
  \end{minipage}

  \vspace*{-0cm}
  
  % Astuce pour que le background ne soit pas discrétisé lors de la conversion pdf -> png
\begin{tikzpicture}
        \fill[opacity=0,green!60!black] (0,0)--++(0,0)--++(0,0)--++(0,0)--cycle; 
\end{tikzpicture}

% toc S'affiche trop tot :
% \tableofcontents[hideallsubsections, pausesections]
}

\newcommand{\finmontitre}{
  \end{frame}
  \setcounter{framenumber}{0}
} % ne marche pas pour une raison obscure

%----- Commandes supplementaires ------

% \usepackage[landscape]{geometry}
% \geometry{top=1cm, bottom=3cm, left=2cm, right=10cm, marginparsep=1cm
% }
% \usepackage[a4paper]{geometry}
% \geometry{top=2cm, bottom=2cm, left=2cm, right=2cm, marginparsep=1cm
% }

%\usepackage{standalone}


% New command Arnaud -- november 2011
\setbeamersize{text margin left=24ex}
% si vous modifier cette valeur il faut aussi
% modifier le decalage du titre pour compenser
% (ex : ici =+10ex, titre =-5ex

\theoremstyle{definition}
%\newtheorem{proposition}{Proposition}
%\newtheorem{exemple}{Exemple}
%\newtheorem{theoreme}{Théorème}
%\newtheorem{lemme}{Lemme}
%\newtheorem{corollaire}{Corollaire}
%\newtheorem*{remarque*}{Remarque}
%\newtheorem*{miniexercice}{Mini-exercices}
%\newtheorem{definition}{Définition}

% Commande tikz
\usetikzlibrary{calc}
\usetikzlibrary{patterns,arrows}
\usetikzlibrary{matrix}
\usetikzlibrary{fadings} 

%definition d'un terme
\newcommand{\defi}[1]{{\color{myorange}\textbf{\emph{#1}}}}
\newcommand{\evidence}[1]{{\color{blue}\textbf{\emph{#1}}}}
\newcommand{\assertion}[1]{\emph{\og#1\fg}}  % pour chapitre logique
%\renewcommand{\contentsname}{Sommaire}
\renewcommand{\contentsname}{}
\setcounter{tocdepth}{2}



%------ Encadrement ------

\usepackage{fancybox}


\newcommand{\mybox}[1]{
\setlength{\fboxsep}{7pt}
\begin{center}
\shadowbox{#1}
\end{center}}

\newcommand{\myboxinline}[1]{
\setlength{\fboxsep}{5pt}
\raisebox{-10pt}{
\shadowbox{#1}
}
}

%--------------- Commande beamer---------------
\newcommand{\beameronly}[1]{#1} % permet de mettre des pause dans beamer pas dans poly


\setbeamertemplate{navigation symbols}{}
\setbeamertemplate{footline}  % tiré du fichier beamerouterinfolines.sty
{
  \leavevmode%
  \hbox{%
  \begin{beamercolorbox}[wd=.333333\paperwidth,ht=2.25ex,dp=1ex,center]{author in head/foot}%
    % \usebeamerfont{author in head/foot}\insertshortauthor%~~(\insertshortinstitute)
    \usebeamerfont{section in head/foot}{\bf\insertshorttitle}
  \end{beamercolorbox}%
  \begin{beamercolorbox}[wd=.333333\paperwidth,ht=2.25ex,dp=1ex,center]{title in head/foot}%
    \usebeamerfont{section in head/foot}{\bf\insertsectionhead}
  \end{beamercolorbox}%
  \begin{beamercolorbox}[wd=.333333\paperwidth,ht=2.25ex,dp=1ex,right]{date in head/foot}%
    % \usebeamerfont{date in head/foot}\insertshortdate{}\hspace*{2em}
    \insertframenumber{} / \inserttotalframenumber\hspace*{2ex} 
  \end{beamercolorbox}}%
  \vskip0pt%
}


\definecolor{mygrey}{rgb}{0.5,0.5,0.5}
\setlength{\parindent}{0cm}
%\DeclareTextFontCommand{\helvetica}{\fontfamily{phv}\selectfont}

% background beamer
\definecolor{couleurhaut}{rgb}{0.85,0.9,1}  % creme
\definecolor{couleurmilieu}{rgb}{1,1,1}  % vert pale
\definecolor{couleurbas}{rgb}{0.85,0.9,1}  % blanc
\setbeamertemplate{background canvas}[vertical shading]%
[top=couleurhaut,middle=couleurmilieu,midpoint=0.4,bottom=couleurbas] 
%[top=fondtitre!05,bottom=fondtitre!60]



\makeatletter
\setbeamertemplate{theorem begin}
{%
  \begin{\inserttheoremblockenv}
  {%
    \inserttheoremheadfont
    \inserttheoremname
    \inserttheoremnumber
    \ifx\inserttheoremaddition\@empty\else\ (\inserttheoremaddition)\fi%
    \inserttheorempunctuation
  }%
}
\setbeamertemplate{theorem end}{\end{\inserttheoremblockenv}}

\newenvironment{theoreme}[1][]{%
   \setbeamercolor{block title}{fg=structure,bg=structure!40}
   \setbeamercolor{block body}{fg=black,bg=structure!10}
   \begin{block}{{\bf Th\'eor\`eme }#1}
}{%
   \end{block}%
}


\newenvironment{proposition}[1][]{%
   \setbeamercolor{block title}{fg=structure,bg=structure!40}
   \setbeamercolor{block body}{fg=black,bg=structure!10}
   \begin{block}{{\bf Proposition }#1}
}{%
   \end{block}%
}

\newenvironment{corollaire}[1][]{%
   \setbeamercolor{block title}{fg=structure,bg=structure!40}
   \setbeamercolor{block body}{fg=black,bg=structure!10}
   \begin{block}{{\bf Corollaire }#1}
}{%
   \end{block}%
}

\newenvironment{mydefinition}[1][]{%
   \setbeamercolor{block title}{fg=structure,bg=structure!40}
   \setbeamercolor{block body}{fg=black,bg=structure!10}
   \begin{block}{{\bf Définition} #1}
}{%
   \end{block}%
}

\newenvironment{lemme}[0]{%
   \setbeamercolor{block title}{fg=structure,bg=structure!40}
   \setbeamercolor{block body}{fg=black,bg=structure!10}
   \begin{block}{\bf Lemme}
}{%
   \end{block}%
}

\newenvironment{remarque}[1][]{%
   \setbeamercolor{block title}{fg=black,bg=structure!20}
   \setbeamercolor{block body}{fg=black,bg=structure!5}
   \begin{block}{Remarque #1}
}{%
   \end{block}%
}


\newenvironment{exemple}[1][]{%
   \setbeamercolor{block title}{fg=black,bg=structure!20}
   \setbeamercolor{block body}{fg=black,bg=structure!5}
   \begin{block}{{\bf Exemple }#1}
}{%
   \end{block}%
}


\newenvironment{miniexercice}[0]{%
   \setbeamercolor{block title}{fg=structure,bg=structure!20}
   \setbeamercolor{block body}{fg=black,bg=structure!5}
   \begin{block}{Mini-exercices}
}{%
   \end{block}%
}


\newenvironment{tp}[0]{%
   \setbeamercolor{block title}{fg=structure,bg=structure!40}
   \setbeamercolor{block body}{fg=black,bg=structure!10}
   \begin{block}{\bf Travaux pratiques}
}{%
   \end{block}%
}
\newenvironment{exercicecours}[1][]{%
   \setbeamercolor{block title}{fg=structure,bg=structure!40}
   \setbeamercolor{block body}{fg=black,bg=structure!10}
   \begin{block}{{\bf Exercice }#1}
}{%
   \end{block}%
}
\newenvironment{algo}[1][]{%
   \setbeamercolor{block title}{fg=structure,bg=structure!40}
   \setbeamercolor{block body}{fg=black,bg=structure!10}
   \begin{block}{{\bf Algorithme}\hfill{\color{gray}\texttt{#1}}}
}{%
   \end{block}%
}


\setbeamertemplate{proof begin}{
   \setbeamercolor{block title}{fg=black,bg=structure!20}
   \setbeamercolor{block body}{fg=black,bg=structure!5}
   \begin{block}{{\footnotesize Démonstration}}
   \footnotesize
   \smallskip}
\setbeamertemplate{proof end}{%
   \end{block}}
\setbeamertemplate{qed symbol}{\openbox}


\makeatother
\usecolortheme[RGB={192,41,0}]{structure}

% Commande spécifique à ce chapitre

\newcommand{\Python}{\texttt{Python}}
\renewcommand{\evidence}[1]{{\color{blue}\textbf{#1}}}

\usepackage{textcomp}

\usepackage{listings}
\lstset{
  upquote=true,
  columns=flexible,
  keepspaces=true,
  basicstyle=\ttfamily,
  commentstyle=\color{gray},
  language=Python,
  showstringspaces=false,
  aboveskip=0em,  
  belowskip=0em,
  escapeinside=||
}

\lstset{
  literate={é}{{\'e}}1
           {è}{{\`e}}1
           {à}{{\`a}}1
}


\newcommand{\codeinline}[1]{\lstinline!#1!}


%%%%%%%%%%%%%%%%%%%%%%%%%%%%%%%%%%%%%%%%%%%%%%%%%%%%%%%%%%%%%
%%%%%%%%%%%%%%%%%%%%%%%%%%%%%%%%%%%%%%%%%%%%%%%%%%%%%%%%%%%%%


\begin{document}


\title{{\bf Algorithmes et mathématiques}}
\subtitle{Les réels}

\begin{frame}
  
  \debutmontitre

  \pause

{\footnotesize
\hfill
\setbeamercovered{transparent=50}
\begin{minipage}{0.6\textwidth}
  \begin{itemize}
    \item<3-> Constante $\gamma$ d'Euler
    \item<4-> $1000$ décimales de la constante d'Euler
    \item<5-> Un peu de réalité
  \end{itemize}
\end{minipage}
}

\end{frame}

\setcounter{framenumber}{0}


%%%%%%%%%%%%%%%%%%%%%%%%%%%%%%%%%%%%%%%%%%%%%%%%%%%%%%%%%%%%%%%%
\section{}

\subsection{Constante $\gamma$ d'Euler}





 
\begin{frame}
La \defi{suite harmonique}
$$H_n = 1+\frac{1}{2}+\frac{1}{3}+\cdots + \frac{1}{n}$$
\pause
\vspace*{-2ex}
$$u_n = H_n - \ln n$$
\pause
\centerline{$(u_n)$ admet une limite lorsque $n\to+\infty$}

\centerline{la \defi{constante $\gamma$ d'Euler}}

\bigskip

\pause

\begin{tp}
\begin{enumerate}
  \item Calculer les premières décimales de $\gamma$. Sachant que $u_n - \gamma \sim \frac{1}{2n}$, combien de
  décimales exactes peut-on espérer avoir obtenues ?
  
  \item On considère $v_n = H_n -\ln\big(n+\frac 12 + \frac{1}{24n} \big)$. Sachant $v_n - \gamma \sim -\frac{1}{48n^3}$, calculer davantage
  de décimales.
\end{enumerate}  
\end{tp}

\end{frame}


\begin{frame}[fragile]

\begin{algo}[euler.py (1)]
\begin{lstlisting}
def euler1(n):
    somme = 0
    for i in range(n,0,-1):
        somme = somme + 1/i
    return somme - log(n)
\end{lstlisting}  
\end{algo}

\pause

\begin{algo}[euler.py (2)]
\begin{lstlisting}
def euler2(n):
    somme = 0
    for i in range(n,0,-1):
        somme = somme + 1/i
    return somme - log(n+1/2+1/(24*n))
\end{lstlisting}  
\end{algo}

\end{frame}


%%%%%%%%%%%%%%%%%%%%%%%%%%%%%%%%%%%%%%%%%%%%%%%%%%%%%%%%%%%%%%%%
\section{$1000$ décimales de la constante d'Euler}

\begin{frame}

\hfill\hfill\textbf{Méthode de Bessel modifiée}

$$w_n = \frac{A_n}{B_n} - \ln n 
\qquad 
A_n =\sum_{k=1}^{E(\alpha n)} \left( \frac{n^k}{k!} \right)^2 H_k
 \qquad 
B_n =\sum_{k=0}^{E(\alpha n)} \left( \frac{n^k}{k!} \right)^2 
$$

\pause
\centerline{où $\alpha = 3.59112147...$ est la solution de $\alpha(\ln \alpha - 1)=1$}

\pause

$$|w_n - \gamma| \le \frac{C}{e^{4n}}$$

\pause

\begin{tp}
\begin{enumerate}
  \item Programmer cette méthode.
  \item Combien d'itérations faut-il pour obtenir $1000$ décimales ?
  \item Utiliser le module \codeinline{decimal} pour les calculer.
\end{enumerate}
\end{tp}
\end{frame}

\begin{frame}[fragile]

\begin{itemize}
  \item $N$ décimales si $\frac{C}{e^{4n}} \le \frac{1}{10^{N}}$
\pause
  \item Donc si $n \ge \frac{N \ln(10)+\ln(C)}{4}$
\pause
  \item Une itération de plus donne (à peu près) une décimale de plus
\end{itemize}

\pause

\begin{algo}[euler.py (3)]
\begin{lstlisting}
def euler3(n):
    alpha = 3.59112147
    N = floor(alpha*n)                # Borne des sommes
    A = 0 ; B = 0
    H = 0
    for k in range(1,N+1):
        c = ( (n**k)/factorial(k) ) ** 2   # Coefficient 
        H = H + 1/k                   # Somme harmonique
        A = A + c*H
        B = B + c
    return A/B - log(n)
\end{lstlisting}  
\end{algo}

\end{frame}

\begin{frame}
\begin{center}
{\footnotesize
\qquad \  $\gamma$ = 0,\hfill \hfill\  \\
  57721566490153286060651209008240243104215933593992 \ 35988057672348848677267776646709369470632917467495 \\
  14631447249807082480960504014486542836224173997644 \ 92353625350033374293733773767394279259525824709491 \\
  60087352039481656708532331517766115286211995015079 \ 84793745085705740029921354786146694029604325421519 \\
  05877553526733139925401296742051375413954911168510 \ 28079842348775872050384310939973613725530608893312 \\
  67600172479537836759271351577226102734929139407984 \ 30103417771778088154957066107501016191663340152278 \\
  93586796549725203621287922655595366962817638879272 \ 68013243101047650596370394739495763890657296792960 \\
  10090151251959509222435014093498712282479497471956 \ 46976318506676129063811051824197444867836380861749 \\
  45516989279230187739107294578155431600500218284409 \ 60537724342032854783670151773943987003023703395183 \\
  28690001558193988042707411542227819716523011073565 \ 83396734871765049194181230004065469314299929777956 \\
  93031005030863034185698032310836916400258929708909 \ 85486825777364288253954925873629596133298574739302\ \\
  }  
\end{center}
\end{frame}



%%%%%%%%%%%%%%%%%%%%%%%%%%%%%%%%%%%%%%%%%%%%%%%%%%%%%%%%%%%%%%%%
\section{Un peu de réalité}



%--------------------------------------------------------
\begin{frame}
Un \defi{nombre flottant} s'écrit
$$\underbrace{\pm 1,234567890123456789}_{\text{mantisse}} \ e\underbrace{\pm 123}_{\text{exposant}}$$
\pause
$$\pm 1,234\ldots \times 10^{\pm 123}$$

\pause

\begin{itemize}
  \item La \defi{mantisse} est un nombre décimal (positif ou négatif) appartenant à $[1,10[$
\pause
  \item L'exposant est un entier (positif ou négatif)
\end{itemize}

\pause

\begin{tp}
Poser $x=10^{-16}$, $y=x+1$, $z=y-1$. Que vaut $z$ pour \Python \ ?
\end{tp}

\end{frame}


\begin{frame}[fragile]

\begin{tp}
\begin{enumerate}
  \item Calculer l'exposant d'un nombre réel. Calculer la mantisse.
  \item Faire une fonction qui ne conserve que $6$ chiffres d'un nombre 
  ($6$ chiffres en tout : avant + après la virgule, exemple
  $123,456789$ devient $123,456$).
\end{enumerate}  
\end{tp}

\pause

\begin{algo}[reels.py (1)]
\begin{lstlisting}
precision = 6            # Nombre de décimales conservées
def tronquer(x):
    n = floor(log(x,10))                       # Exposant |\pause|
    m = floor( x * 10 ** (precision-1 - n))    # Mantisse |\pause|
    return m * 10 ** (-precision+1+n)    # Nombre tronqué
\end{lstlisting}  
\end{algo}

\pause

Exemple : $x = 123,456789 = 1,23... \times 10^2$ alors l'exposant est $n=2$

On décale vers la gauche $123\, 456,789$,
la partie entière  est $m=123\, 456$

On redécale vers la droite : $123,456$


\end{frame}


%--------------------------------------------------------
\begin{frame}

\hfill\hfill\textbf{Absorption}

\begin{tp}
\begin{enumerate}
  \item Calculer \codeinline{tronquer(1234.56 + 0.007)}.
  \item Expliquer.
\end{enumerate}  
\end{tp}

\pause
\bigskip

\begin{itemize}
  \item $1234,56$ et $0,007$ s'écrivent avec moins de $6$ décimales
\pause
  \item La somme $1234,567$ a besoin de $7$ décimales
\pause
  \item L'ordinateur ne retient que $1234,56$
\pause
  \item Le $0,007$ disparaît : il a été victime d'une \defi{absorption}
\end{itemize}

\end{frame}

%--------------------------------------------------------
\begin{frame}
  
\hfill\hfill\textbf{\'Elimination}

\begin{tp}
\begin{enumerate}
  \item Soient $x = 1234,8777$, $y = 1212,2222$. Calculer $x-y$ à la main.
  Comment se calcule la différence $x-y$ avec notre précision
  de $6$ chiffres ?
  \item Expliquer la différence.
\end{enumerate}  
\end{tp}

\pause

\begin{itemize}
  \item $x-y = 22,6555$ qui n'a que $6$ chiffres
\pause
  \item Mais l'ordinateur ne stocke pas $x$ mais \codeinline{tronquer(x)}, idem pour $y$
\pause
  \item Le calcul est \codeinline{tronquer(tronquer(x)-tronquer(y))}
\pause
  \item Il calcule donc $1234,87-1212,22=22,65$ et retourne $22,6500$
\pause
  \item Les $2$ derniers chiffres sont une pure invention
\pause
  \item Phénomène d'\defi{élimination}
\end{itemize}

\pause
\vspace*{-1ex}

\begin{center}
Ne pas confondre la \evidence{précision} d'affichage

(exemple : on calcule avec $10$ chiffres après la virgule) 

avec l'\evidence{exactitude} du résultat

(combien de décimales sont vraiment exactes ?)
\end{center}
\end{frame}

%--------------------------------------------------------
\begin{frame}

\hfill\hfill\textbf{Conversion binaire -- décimale}

\begin{tp}
Effectuer les commandes suivantes et constater !
\begin{enumerate}
  \item \codeinline{sum = 0} puis \codeinline{for i in range(10): sum = sum + 0.1}. Que vaut \codeinline{sum} ?
  \item \codeinline{0.1 + 0.1 == 0.2} \quad et \quad  \codeinline{0.1 + 0.1 + 0.1 == 0.3} 
  \item \codeinline{x = 0.2 ; print("0.2 en Python = \%.25f" \%x)}
\end{enumerate}  
\end{tp}

\pause

\begin{itemize}

  \item En écriture décimale, il est impossible de coder $1/3 = 0,3333\ldots$ avec un nombre fini
de chiffres après la virgule
\pause
  \item L'ordinateur ne stocke pas $0,1$ ni $0,2$ en mémoire
\pause
  \item Il stocke un nombre en écriture binaire qui s'en rapproche le plus
\pause  
  \item Il retourne l'écriture décimale qui se rapproche le plus
 \centerline{\codeinline{0.2000000000000000111022302}\ldots}
\end{itemize}

\end{frame}
 
%--------------------------------------------------------
\begin{frame}

\hfill\hfill\textbf{Somme des inverses des carrés}

\begin{tp}
\begin{enumerate}
  \item Faire une fonction qui calcule la somme $S_n = \frac{1}{1^2} + \frac{1}{2^2}+\frac{1}{3^2}+\cdots + \frac{1}{n^2}$.

  \item Faire une fonction qui calcule cette somme mais en utilisant seulement une écriture décimale à $6$ chiffres 
  (à l'aide de la fonction \codeinline{tronquer()} vue au-dessus).
  
  \item Reprendre cette dernière fonction, mais en commençant la somme par les plus petits termes.
  
  \item Comparez le deux dernières méthodes, justifier et conclure.
\end{enumerate}  
\end{tp}
\end{frame}



\begin{frame}[fragile]

\begin{algo}[reels.py (2)]
\begin{lstlisting}
def somme_inverse_carres_tronq(n):
  somme = 0
  for i in range(1,n+1):
      somme = tronquer(somme + tronquer(1/(i*i)))
  return somme
\end{lstlisting}  
\end{algo}

\pause

\begin{algo}[reels.py (3)]
\begin{lstlisting}
def somme_inverse_carres_tronq_inv(n):
  somme = 0
  for i in range(n,0,-1):
      somme = tronquer(somme + tronquer(1/(i*i)))
  return somme
\end{lstlisting}  
\end{algo}

\pause

Le vrai résultat est $\frac{\pi^2}{6}=1,64493\ldots$ 

\pause

Pour $n = 100\, 000$, le premier retourne $1,64038$ 

\pause

Pour $n = 100\, 000$, le second retourne $1,64490$

\end{frame}



%%%%%%%%%%%%%%%%%%%%%%%%%%%%%%%%%%%%%%%%%%%%%%%%%%%%%%%%%%%%%%%%
\section{Mini-exercices}

\begin{frame}
\small
\begin{miniexercice}
\vspace*{-1ex}
\begin{enumerate}
\setlength{\itemsep}{0pt}
  \item \'Ecrire une fonction qui approxime la constante $\alpha$ qui vérifie $\alpha (\ln \alpha -1)=1$.
  Pour cela poser $f(x) = x(\ln x - 1)-1$ et appliquer la méthode de Newton : fixer $u_0$ (par exemple ici $u_0=4$)
  et $u_{n+1} = u_n - \frac{f(u_n)}{f'(u_n)}$.
  
  \item Pour chacune des trois méthodes, calculer le nombre approximatif d'itérations nécessaires
  pour obtenir $100$ décimales de la constante $\gamma$ d'Euler.
  
  \item Notons $C_n = \frac{1}{4n} \sum_{k=0}^{2n} \frac{[(2k)!]^3}{(k!)^4(16n)^2k}$.
  La formule de Brent-McMillan affirme $\gamma = \frac{A_n}{B_n} - \frac{C_n}{B_n^2} - \ln n + O(\frac{1}{e^{8n}})$
  où cette fois les sommations pour $A_n$ et $B_n$ vont jusqu'à $E(\beta n)$ avec
  $\beta = 4,970625759\ldots$ la solution de $\beta(\ln \beta - 1)=3$. La notation $O(\frac{1}{e^{8n}})$ 
  indique que l'erreur est $\le \frac{C}{e^{8n}}$ pour une certaine constante $C$.
  Mettre en \oe uvre cette formule. En 1999 cette formule a permis de calculer $100$ millions de décimales.
  Combien a-t-il fallu d'itérations ?
 
  \item Faire une fonction qui renvoie le terme $u_n$ de la suite définie par $u_0 = \frac 13$
  et $u_{n+1} = 4 u_n -1$. Que vaut $u_{100}$ ? Faire l'étude mathématique et commenter.

\end{enumerate}
\end{miniexercice}

\end{frame}

\end{document}
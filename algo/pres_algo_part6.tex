
%%%%%%%%%%%%%%%%%% PREAMBULE %%%%%%%%%%%%%%%%%%

\documentclass[aspectratio=169,utf8]{beamer}
%\documentclass[aspectratio=169,handout]{beamer}

\usetheme{Boadilla}
%\usecolortheme{seahorse}
%\usecolortheme[RGB={245,66,24}]{structure}
\useoutertheme{infolines}

% packages
\usepackage{amsfonts,amsmath,amssymb,amsthm}
\usepackage[utf8]{inputenc}
\usepackage[T1]{fontenc}
\usepackage{lmodern}

\usepackage[francais]{babel}
\usepackage{fancybox}
\usepackage{graphicx}

\usepackage{float}
\usepackage{xfrac}

%\usepackage[usenames, x11names]{xcolor}
\usepackage{pgfplots}
\usepackage{datetime}


% ----------------------------------------------------------------------
% Pour les images
\usepackage{tikz}
\usetikzlibrary{calc,shadows,arrows.meta,patterns,matrix}

\newcommand{\tikzinput}[1]{\input{figures/#1.tikz}}
% --- les figures avec échelle éventuel
\newcommand{\myfigure}[2]{% entrée : échelle, fichier(s) figure à inclure
\begin{center}\small%
\tikzstyle{every picture}=[scale=1.0*#1]% mise en échelle + 0% (automatiquement annulé à la fin du groupe)
#2%
\end{center}}



%-----  Package unités -----
\usepackage{siunitx}
\sisetup{locale = FR,detect-all,per-mode = symbol}

%\usepackage{mathptmx}
%\usepackage{fouriernc}
%\usepackage{newcent}
%\usepackage[mathcal,mathbf]{euler}

%\usepackage{palatino}
%\usepackage{newcent}
% \usepackage[mathcal,mathbf]{euler}



% \usepackage{hyperref}
% \hypersetup{colorlinks=true, linkcolor=blue, urlcolor=blue,
% pdftitle={Exo7 - Exercices de mathématiques}, pdfauthor={Exo7}}


%section
% \usepackage{sectsty}
% \allsectionsfont{\bf}
%\sectionfont{\color{Tomato3}\upshape\selectfont}
%\subsectionfont{\color{Tomato4}\upshape\selectfont}

%----- Ensembles : entiers, reels, complexes -----
\newcommand{\Nn}{\mathbb{N}} \newcommand{\N}{\mathbb{N}}
\newcommand{\Zz}{\mathbb{Z}} \newcommand{\Z}{\mathbb{Z}}
\newcommand{\Qq}{\mathbb{Q}} \newcommand{\Q}{\mathbb{Q}}
\newcommand{\Rr}{\mathbb{R}} \newcommand{\R}{\mathbb{R}}
\newcommand{\Cc}{\mathbb{C}} 
\newcommand{\Kk}{\mathbb{K}} \newcommand{\K}{\mathbb{K}}

%----- Modifications de symboles -----
\renewcommand{\epsilon}{\varepsilon}
\renewcommand{\Re}{\mathop{\text{Re}}\nolimits}
\renewcommand{\Im}{\mathop{\text{Im}}\nolimits}
%\newcommand{\llbracket}{\left[\kern-0.15em\left[}
%\newcommand{\rrbracket}{\right]\kern-0.15em\right]}

\renewcommand{\ge}{\geqslant}
\renewcommand{\geq}{\geqslant}
\renewcommand{\le}{\leqslant}
\renewcommand{\leq}{\leqslant}
\renewcommand{\epsilon}{\varepsilon}

%----- Fonctions usuelles -----
\newcommand{\ch}{\mathop{\text{ch}}\nolimits}
\newcommand{\sh}{\mathop{\text{sh}}\nolimits}
\renewcommand{\tanh}{\mathop{\text{th}}\nolimits}
\newcommand{\cotan}{\mathop{\text{cotan}}\nolimits}
\newcommand{\Arcsin}{\mathop{\text{arcsin}}\nolimits}
\newcommand{\Arccos}{\mathop{\text{arccos}}\nolimits}
\newcommand{\Arctan}{\mathop{\text{arctan}}\nolimits}
\newcommand{\Argsh}{\mathop{\text{argsh}}\nolimits}
\newcommand{\Argch}{\mathop{\text{argch}}\nolimits}
\newcommand{\Argth}{\mathop{\text{argth}}\nolimits}
\newcommand{\pgcd}{\mathop{\text{pgcd}}\nolimits} 


%----- Commandes divers ------
\newcommand{\ii}{\mathrm{i}}
\newcommand{\dd}{\text{d}}
\newcommand{\id}{\mathop{\text{id}}\nolimits}
\newcommand{\Ker}{\mathop{\text{Ker}}\nolimits}
\newcommand{\Card}{\mathop{\text{Card}}\nolimits}
\newcommand{\Vect}{\mathop{\text{Vect}}\nolimits}
\newcommand{\Mat}{\mathop{\text{Mat}}\nolimits}
\newcommand{\rg}{\mathop{\text{rg}}\nolimits}
\newcommand{\tr}{\mathop{\text{tr}}\nolimits}


%----- Structure des exercices ------

\newtheoremstyle{styleexo}% name
{2ex}% Space above
{3ex}% Space below
{}% Body font
{}% Indent amount 1
{\bfseries} % Theorem head font
{}% Punctuation after theorem head
{\newline}% Space after theorem head 2
{}% Theorem head spec (can be left empty, meaning ‘normal’)

%\theoremstyle{styleexo}
\newtheorem{exo}{Exercice}
\newtheorem{ind}{Indications}
\newtheorem{cor}{Correction}


\newcommand{\exercice}[1]{} \newcommand{\finexercice}{}
%\newcommand{\exercice}[1]{{\tiny\texttt{#1}}\vspace{-2ex}} % pour afficher le numero absolu, l'auteur...
\newcommand{\enonce}{\begin{exo}} \newcommand{\finenonce}{\end{exo}}
\newcommand{\indication}{\begin{ind}} \newcommand{\finindication}{\end{ind}}
\newcommand{\correction}{\begin{cor}} \newcommand{\fincorrection}{\end{cor}}

\newcommand{\noindication}{\stepcounter{ind}}
\newcommand{\nocorrection}{\stepcounter{cor}}

\newcommand{\fiche}[1]{} \newcommand{\finfiche}{}
\newcommand{\titre}[1]{\centerline{\large \bf #1}}
\newcommand{\addcommand}[1]{}
\newcommand{\video}[1]{}

% Marge
\newcommand{\mymargin}[1]{\marginpar{{\small #1}}}

\def\noqed{\renewcommand{\qedsymbol}{}}


%----- Presentation ------
\setlength{\parindent}{0cm}

%\newcommand{\ExoSept}{\href{http://exo7.emath.fr}{\textbf{\textsf{Exo7}}}}

\definecolor{myred}{rgb}{0.93,0.26,0}
\definecolor{myorange}{rgb}{0.97,0.58,0}
\definecolor{myyellow}{rgb}{1,0.86,0}

\newcommand{\LogoExoSept}[1]{  % input : echelle
{\usefont{U}{cmss}{bx}{n}
\begin{tikzpicture}[scale=0.1*#1,transform shape]
  \fill[color=myorange] (0,0)--(4,0)--(4,-4)--(0,-4)--cycle;
  \fill[color=myred] (0,0)--(0,3)--(-3,3)--(-3,0)--cycle;
  \fill[color=myyellow] (4,0)--(7,4)--(3,7)--(0,3)--cycle;
  \node[scale=5] at (3.5,3.5) {Exo7};
\end{tikzpicture}}
}


\newcommand{\debutmontitre}{
  \author{} \date{} 
  \thispagestyle{empty}
  \hspace*{-10ex}
  \begin{minipage}{\textwidth}
    \titlepage  
  \vspace*{-2.5cm}
  \begin{center}
    \LogoExoSept{2.5}
  \end{center}
  \end{minipage}

  \vspace*{-0cm}
  
  % Astuce pour que le background ne soit pas discrétisé lors de la conversion pdf -> png
\begin{tikzpicture}
        \fill[opacity=0,green!60!black] (0,0)--++(0,0)--++(0,0)--++(0,0)--cycle; 
\end{tikzpicture}

% toc S'affiche trop tot :
% \tableofcontents[hideallsubsections, pausesections]
}

\newcommand{\finmontitre}{
  \end{frame}
  \setcounter{framenumber}{0}
} % ne marche pas pour une raison obscure

%----- Commandes supplementaires ------

% \usepackage[landscape]{geometry}
% \geometry{top=1cm, bottom=3cm, left=2cm, right=10cm, marginparsep=1cm
% }
% \usepackage[a4paper]{geometry}
% \geometry{top=2cm, bottom=2cm, left=2cm, right=2cm, marginparsep=1cm
% }

%\usepackage{standalone}


% New command Arnaud -- november 2011
\setbeamersize{text margin left=24ex}
% si vous modifier cette valeur il faut aussi
% modifier le decalage du titre pour compenser
% (ex : ici =+10ex, titre =-5ex

\theoremstyle{definition}
%\newtheorem{proposition}{Proposition}
%\newtheorem{exemple}{Exemple}
%\newtheorem{theoreme}{Théorème}
%\newtheorem{lemme}{Lemme}
%\newtheorem{corollaire}{Corollaire}
%\newtheorem*{remarque*}{Remarque}
%\newtheorem*{miniexercice}{Mini-exercices}
%\newtheorem{definition}{Définition}

% Commande tikz
\usetikzlibrary{calc}
\usetikzlibrary{patterns,arrows}
\usetikzlibrary{matrix}
\usetikzlibrary{fadings} 

%definition d'un terme
\newcommand{\defi}[1]{{\color{myorange}\textbf{\emph{#1}}}}
\newcommand{\evidence}[1]{{\color{blue}\textbf{\emph{#1}}}}
\newcommand{\assertion}[1]{\emph{\og#1\fg}}  % pour chapitre logique
%\renewcommand{\contentsname}{Sommaire}
\renewcommand{\contentsname}{}
\setcounter{tocdepth}{2}



%------ Encadrement ------

\usepackage{fancybox}


\newcommand{\mybox}[1]{
\setlength{\fboxsep}{7pt}
\begin{center}
\shadowbox{#1}
\end{center}}

\newcommand{\myboxinline}[1]{
\setlength{\fboxsep}{5pt}
\raisebox{-10pt}{
\shadowbox{#1}
}
}

%--------------- Commande beamer---------------
\newcommand{\beameronly}[1]{#1} % permet de mettre des pause dans beamer pas dans poly


\setbeamertemplate{navigation symbols}{}
\setbeamertemplate{footline}  % tiré du fichier beamerouterinfolines.sty
{
  \leavevmode%
  \hbox{%
  \begin{beamercolorbox}[wd=.333333\paperwidth,ht=2.25ex,dp=1ex,center]{author in head/foot}%
    % \usebeamerfont{author in head/foot}\insertshortauthor%~~(\insertshortinstitute)
    \usebeamerfont{section in head/foot}{\bf\insertshorttitle}
  \end{beamercolorbox}%
  \begin{beamercolorbox}[wd=.333333\paperwidth,ht=2.25ex,dp=1ex,center]{title in head/foot}%
    \usebeamerfont{section in head/foot}{\bf\insertsectionhead}
  \end{beamercolorbox}%
  \begin{beamercolorbox}[wd=.333333\paperwidth,ht=2.25ex,dp=1ex,right]{date in head/foot}%
    % \usebeamerfont{date in head/foot}\insertshortdate{}\hspace*{2em}
    \insertframenumber{} / \inserttotalframenumber\hspace*{2ex} 
  \end{beamercolorbox}}%
  \vskip0pt%
}


\definecolor{mygrey}{rgb}{0.5,0.5,0.5}
\setlength{\parindent}{0cm}
%\DeclareTextFontCommand{\helvetica}{\fontfamily{phv}\selectfont}

% background beamer
\definecolor{couleurhaut}{rgb}{0.85,0.9,1}  % creme
\definecolor{couleurmilieu}{rgb}{1,1,1}  % vert pale
\definecolor{couleurbas}{rgb}{0.85,0.9,1}  % blanc
\setbeamertemplate{background canvas}[vertical shading]%
[top=couleurhaut,middle=couleurmilieu,midpoint=0.4,bottom=couleurbas] 
%[top=fondtitre!05,bottom=fondtitre!60]



\makeatletter
\setbeamertemplate{theorem begin}
{%
  \begin{\inserttheoremblockenv}
  {%
    \inserttheoremheadfont
    \inserttheoremname
    \inserttheoremnumber
    \ifx\inserttheoremaddition\@empty\else\ (\inserttheoremaddition)\fi%
    \inserttheorempunctuation
  }%
}
\setbeamertemplate{theorem end}{\end{\inserttheoremblockenv}}

\newenvironment{theoreme}[1][]{%
   \setbeamercolor{block title}{fg=structure,bg=structure!40}
   \setbeamercolor{block body}{fg=black,bg=structure!10}
   \begin{block}{{\bf Th\'eor\`eme }#1}
}{%
   \end{block}%
}


\newenvironment{proposition}[1][]{%
   \setbeamercolor{block title}{fg=structure,bg=structure!40}
   \setbeamercolor{block body}{fg=black,bg=structure!10}
   \begin{block}{{\bf Proposition }#1}
}{%
   \end{block}%
}

\newenvironment{corollaire}[1][]{%
   \setbeamercolor{block title}{fg=structure,bg=structure!40}
   \setbeamercolor{block body}{fg=black,bg=structure!10}
   \begin{block}{{\bf Corollaire }#1}
}{%
   \end{block}%
}

\newenvironment{mydefinition}[1][]{%
   \setbeamercolor{block title}{fg=structure,bg=structure!40}
   \setbeamercolor{block body}{fg=black,bg=structure!10}
   \begin{block}{{\bf Définition} #1}
}{%
   \end{block}%
}

\newenvironment{lemme}[0]{%
   \setbeamercolor{block title}{fg=structure,bg=structure!40}
   \setbeamercolor{block body}{fg=black,bg=structure!10}
   \begin{block}{\bf Lemme}
}{%
   \end{block}%
}

\newenvironment{remarque}[1][]{%
   \setbeamercolor{block title}{fg=black,bg=structure!20}
   \setbeamercolor{block body}{fg=black,bg=structure!5}
   \begin{block}{Remarque #1}
}{%
   \end{block}%
}


\newenvironment{exemple}[1][]{%
   \setbeamercolor{block title}{fg=black,bg=structure!20}
   \setbeamercolor{block body}{fg=black,bg=structure!5}
   \begin{block}{{\bf Exemple }#1}
}{%
   \end{block}%
}


\newenvironment{miniexercice}[0]{%
   \setbeamercolor{block title}{fg=structure,bg=structure!20}
   \setbeamercolor{block body}{fg=black,bg=structure!5}
   \begin{block}{Mini-exercices}
}{%
   \end{block}%
}


\newenvironment{tp}[0]{%
   \setbeamercolor{block title}{fg=structure,bg=structure!40}
   \setbeamercolor{block body}{fg=black,bg=structure!10}
   \begin{block}{\bf Travaux pratiques}
}{%
   \end{block}%
}
\newenvironment{exercicecours}[1][]{%
   \setbeamercolor{block title}{fg=structure,bg=structure!40}
   \setbeamercolor{block body}{fg=black,bg=structure!10}
   \begin{block}{{\bf Exercice }#1}
}{%
   \end{block}%
}
\newenvironment{algo}[1][]{%
   \setbeamercolor{block title}{fg=structure,bg=structure!40}
   \setbeamercolor{block body}{fg=black,bg=structure!10}
   \begin{block}{{\bf Algorithme}\hfill{\color{gray}\texttt{#1}}}
}{%
   \end{block}%
}


\setbeamertemplate{proof begin}{
   \setbeamercolor{block title}{fg=black,bg=structure!20}
   \setbeamercolor{block body}{fg=black,bg=structure!5}
   \begin{block}{{\footnotesize Démonstration}}
   \footnotesize
   \smallskip}
\setbeamertemplate{proof end}{%
   \end{block}}
\setbeamertemplate{qed symbol}{\openbox}


\makeatother
\usecolortheme[RGB={192,41,0}]{structure}

% Commande spécifique à ce chapitre

\newcommand{\Python}{\texttt{Python}}
\renewcommand{\evidence}[1]{{\color{blue}\textbf{#1}}}

\usepackage{textcomp}

\usepackage{listings}
\lstset{
  upquote=true,
  columns=flexible,
  keepspaces=true,
  basicstyle=\ttfamily,
  commentstyle=\color{gray},
  language=Python,
  showstringspaces=false,
  aboveskip=0em,  
  belowskip=0em,
  escapeinside=||
}

\lstset{
  literate={é}{{\'e}}1
           {è}{{\`e}}1
           {à}{{\`a}}1
}


\newcommand{\codeinline}[1]{\lstinline!#1!}


%%%%%%%%%%%%%%%%%%%%%%%%%%%%%%%%%%%%%%%%%%%%%%%%%%%%%%%%%%%%%
%%%%%%%%%%%%%%%%%%%%%%%%%%%%%%%%%%%%%%%%%%%%%%%%%%%%%%%%%%%%%


\begin{document}


\title{{\bf Algorithmes et mathématiques}}
\subtitle{Polynômes -- Complexité d'un algorithme}

\begin{frame}
  
  \debutmontitre

  \pause

{\footnotesize
\hfill
\setbeamercovered{transparent=50}
\begin{minipage}{0.6\textwidth}
  \begin{itemize}
    \item<3-> Qu'est-ce qu'un algorithme ?
    \item<4-> Polynômes
    \item<5-> Algorithme de Karatsuba
  \end{itemize}
\end{minipage}
}

\end{frame}

\setcounter{framenumber}{0}


%%%%%%%%%%%%%%%%%%%%%%%%%%%%%%%%%%%%%%%%%%%%%%%%%%%%%%%%%%%%%%%%
\section{Qu'est-ce qu'un algorithme ?}

\begin{frame}
\hfill\hfill\textbf{Qu'est ce qu'un algorithme ?}
\bigskip
\pause
\begin{itemize}
  \item Succession d'instructions qui renvoie un résultat
\pause
  \item Le résultat retourné est \evidence{exact}
\pause
  \item En un \evidence{nombre fini d'étapes}
\end{itemize}

\pause
\bigskip
Rapide ou pas ?

\pause
\begin{itemize}
  \item Principal critère  : le temps de calcul
\pause
  \item La mémoire occupée
\pause
  \item Dépend du langage et de la machine utilisée
\pause
  \item Notion de \defi{complexité} \pause : nombre d'opérations élémentaires à faire
\pause
  \item Pour nous : nombre d'additions, de multiplications
\end{itemize}
\end{frame}




%%%%%%%%%%%%%%%%%%%%%%%%%%%%%%%%%%%%%%%%%%%%%%%%%%%%%%%%%%%%%%%%
\section{Polynômes}

\begin{frame}
\begin{tp}
On code un polynôme $a_0+a_1X+\cdots + a_n X^n$ sous la forme d'une liste $[a_0,a_1,\ldots,a_n]$.
\begin{enumerate}
  \item \'Ecrire une fonction correspondant à la somme de deux polynômes. Calculer la complexité 
  de cet algorithme (en terme du nombre d'additions sur les coefficients, en fonctions du degré des polynômes).
  
  \item \'Ecrire une fonction correspondant au produit de deux polynômes. Calculer la complexité 
  de cet algorithme (en terme du nombre d'additions et de multiplications sur les coefficients).
  
  \item \'Ecrire une fonction correspondant au quotient et au reste de la division euclidienne de $A$ par 
  $B$ où   $B$ est un polynôme unitaire (son coefficient de plus haut degré est $1$). 
  Majorer la complexité de cet algorithme (en terme du nombre d'additions et de multiplications sur les coefficients).
\end{enumerate}
  
\end{tp}
\end{frame}


\begin{frame}[fragile]

\begin{algo}[polynome.py (1)]
\begin{lstlisting}
def somme(A,B):           # si deg(A)=deg(B)
    C = []
    for i in range(0,degre(A)+1): 
        s = A[i]+B[i]
        C.append(s) 
\end{lstlisting}  
\end{algo}

\pause
\bigskip

\begin{itemize}
  \item \codeinline{degre(poly)}
\pause
  \item Complexité 
\pause
  \begin{itemize}
    \item $\deg A \le n$ et $\deg B \le n$ 
\pause
    \item addition des coefficients $a_i+b_i$
\pause
    \item $i$ variant de $0$ à $n$
\pause
    \item complexité est de $n+1$ additions (dans $\Zz$ ou $\Rr$)
  \end{itemize}
\end{itemize}

\end{frame}


\begin{frame}[fragile]

$$A(X) = \sum_{i=0}^m a_i X^i \qquad B(X) = \sum_{j=0}^n b_j X^j$$

\pause

$$C = A \times B = \sum_{k=0}^{m+n} c_k X^k  \pause \qquad c_k = \sum_{i+j=k} a_i \times b_j$$
\pause
\begin{algo}[polynome.py (2)]
\begin{lstlisting}
def produit(A,B):
    C = []
    for k in range(degre(A)+degre(B)+1): 
        s = 0
        for i in range(k+1):            
            if (i <= degre(A)) and (k-i <= degre(B)):
                s = s + A[i]*B[k-i]
        C.append(s)
    return C
\end{lstlisting}  
\end{algo} 
\end{frame}

\begin{frame}
Calcul de la complexité
\bigskip

\pause
\begin{itemize}
  \item Nombre de multiplications (dans $\Zz$ ou $\Rr$)
\pause
  \begin{itemize}
    \item $m = \deg A$ et $n = \deg B$
\pause
    \item il faut multiplier les $m+1$ coefficients de $A$ par les $n+1$ coefficients de $B$
\pause    
    \item il y a donc $(m+1)(n+1)$ multiplications
  \end{itemize}
\pause
  \item Nombre d'additions
\pause
  \begin{itemize}
    \item les coefficients de $A\times B$ sont $c_0 = a_0 b_0$, 
    $c_1 = a_0b_1+a_1b_0$, $c_2=a_2b_0+a_1b_1+a_2b_0$,\ldots
\pause
    \item le produit $A\times B$ est de degré $m+n$ donc a $m+n+1$ coefficients
\pause
    \item chaque addition regroupe deux termes
\pause
    \item partant de $(m+1)(n+1)$ produits nous devons arriver à $m+n+1$ coefficients
\pause
    \item il y a donc $(m+1)(n+1)-(m+n+1) = mn$ additions
  \end{itemize}
\pause
  \item Exemple : $\deg A=\deg B=n$ de l'ordre de $n^2$ multiplications
\end{itemize}
\end{frame}



\begin{frame}

$$A = 2X^4-X^3-2X^2+3X-1 \qquad B=X^2-X+1$$

\pause

\myfigure{1.2}{
\tikzinput{fig_algo05} 
}  

\pause
\vspace*{-2ex}

\begin{itemize}
  \item Quel monôme $P_1$ fait diminuer le degré de $A-P_1B$ ? C'est $2X^2$ 
\pause
  \item Poser $R_1=A-P_1B=X^3-4X^2+3X-1$, $Q_1 = 2X^2$
\pause
  \item Recommencer avec $R_1$ divisé par $B$
\pause
  \item Poser $R_2 = R_1-P_2B$ avec $P_2 = X$, $Q_2= Q_1+P_2$
\pause
  \item S'arrêter lorsque $\deg R_i < \deg B$
\end{itemize}

\end{frame}



\begin{frame}[fragile]

\begin{algo}[polynome.py (3)]
\begin{lstlisting}
def division(A,B):
    Q = [0]     # Quotient
    R = A       # Reste                            |\pause|
    while (degre(R) >= degre(B)):                  |\pause|
        P = monome(R[degre(R)],degre(R)-degre(B))  |\pause|
        R = somme(R,produit(-P,B))                 |\pause|
        Q = somme(Q,P)                             |\pause| 
    return Q,R
\end{lstlisting}  
\end{algo}

\end{frame}

%%%%%%%%%%%%%%%%%%%%%%%%%%%%%%%%%%%%%%%%%%%%%%%%%%%%%%%%%%%%%%%%
\section{Algorithme de Karatsuba}

\begin{frame}


\begin{itemize}
\item Calcul de $P\times Q$ \pause : \og diviser pour régner \fg, \pause algorithme de Karatsuba
\pause
  \item $P$ et $Q$ polynômes de degrés $< 2n$
\pause  
  

  \item $P = P_1 + P_2 \cdot X^n$ et $Q = Q_1 + Q_2 \cdot X^n$ \pause \quad $\deg P_i, Q_i < n$
\end{itemize}

\pause
\vspace*{-1ex}
\begin{tp}
\small\vspace*{-1ex}
\begin{enumerate}
\setlength{\itemsep}{0pt}
  \item \'Ecrire une formule qui réduit la multiplication des polynômes $P$ et $Q$ de degrés strictement inférieurs à $2n$ en
multiplications de polynômes de degrés strictement inférieurs à $n$.

  \item Programmer un algorithme récursif de multiplication qui utilise la formule précédente. Quelle est sa complexité ?
  
  \item On peut raffiner cette méthode avec la remarque suivante de Karatsuba : le terme intermédiaire de $P \cdot Q$ s’écrit
\centerline{$P_1 \cdot Q_2 + P_2 \cdot Q_1 = (P_1 + P_2 ) \cdot (Q_1 + Q_2 ) - P_1 Q_1 - P_2 Q_2$}
Comme on a déjà calculé $P_1Q_1$ et $P_2Q_2$, on échange deux multiplications et une addition (à gauche) 
contre une multiplication et quatre additions (à droite).
\'Ecrire une fonction qui réalise la multiplication de polynômes à la Karatsuba.
  
  \item Trouver la formule de récurrence qui définit la complexité de la multiplication de Karatsuba. Quelle est sa solution ?
\end{enumerate}
\vspace*{-2.5ex}
\end{tp}
\end{frame}


\begin{frame}[fragile]

\begin{itemize}
  \item $P = P_1 + P_2 \cdot X^n$ et $Q = Q_1 + Q_2 \cdot X^n$
\pause
  \item $P \times Q = (P_1 + X^n P_2 ) \cdot (Q_1 + X^n Q_2 ) 
  \pause = P_1 Q_1 + X^n \cdot (P_1 Q_2 + P_2 Q_1 ) + X^{2n} \cdot P_2 Q_2$
\pause
  \item $4$ multiplications $P_1 Q_1$ , $P_1 Q_2$ , $P_2 Q_1$ et $P_2 Q_2$ 
  entre polynômes de degrés $<n$
\pause  
  \item Multiplications par $X^n$ et $X^{2n}$ sont des ajouts de zéros
\pause  
  \item Complexité $C(n) = 4C(n/2)+O(n)$ ; se résout en $C(n) = O(n^2)$
\pause
  \item Karatsuba \pause $P_1 \cdot Q_2 + P_2 \cdot Q_1 = (P_1 + P_2 ) \cdot (Q_1 + Q_2 ) - P_1 Q_1 - P_2 Q_2$
 \pause 
  \item $\mathbf{3}$ multiplications $P_1 Q_1$, $P_2 Q_2$, $(P_1 + P_2 ) \cdot (Q_1 + Q_2 )$
\pause 
  \item Le découpage \codeinline{P1,P2 = decoupe(P,n)}
\pause  
  \item Décalage $X^n \cdot P$ : \codeinline{produit_monome(P,n)} 
\end{itemize}

\end{frame}


\begin{frame}[fragile]


\begin{algo}[polynome.py (4)]
\begin{lstlisting}
def produit_rapide(P,Q):
  p = degre(P) ; q = degre(Q)                                  |\pause|
  if (p == 0): return [P[0]*k for k in Q]   # init: P=cst       
  if (q == 0): return [Q[0]*k for k in P]   # init: Q=cst      |\pause|
  n = (max(p,q)+1)//2                       # demi-degré
  P1,P2 = decoupe(P,n)                      # decoupages
  Q1,Q2 = decoupe(Q,n)                                                 |\pause|
  P1Q1 = produit_rapide(P1,Q1)              # produits
  P2Q2 = produit_rapide(P2,Q2)                                         |\pause|
  P12Q12 = produit_rapide(somme(P1,P2),somme(Q1,Q2))                       |\pause|
  R1 = somme(P12Q12,somme(-P1Q1,-P2Q2))          |\pause|
  R1 = produit_monome(R1,n)                 # décalages                  |\pause|
  R2 = produit_monome(P2Q2,2*n)                                          |\pause|
  return somme(P1Q1,somme(R1,R2))           # résultat                    
\end{lstlisting}  
\end{algo}

\end{frame}


\begin{frame}
\begin{enumerate}
  \item Notons $C(n)$ la complexité de la multiplication entre deux polynômes de degrés $<n$

 \pause 
 
  \item
  \begin{itemize} 
    \item Trois appels récursifs
    \pause  
    \item Opérations linéaires : deux calculs de degrés, 
      deux découpes en $n/2$ puis des additions : 
      deux de taille $n/2$, une de taille $n$,
      une de taille $3n/2$ et une de taille $2n$ 
    \pause    
    \item $C(n) = 3 \cdot C(n/2) + \gamma n$   
  \end{itemize}
  
\pause   

  \item 
  \begin{itemize} 
    \item $\alpha_\ell= \frac{C(2^\ell)}{3^\ell}$ \pause \quad $\alpha_0 = C(1)=1$
\pause  
    \item $\alpha_\ell = \frac{C(2^\ell)}{3^\ell} = \frac{3 C(2^{\ell-1})}{3^\ell} + \frac{\gamma 2^\ell}{3^\ell} \pause =  \alpha_{\ell-1} + \gamma \left( \frac23  \right)^\ell$
\pause  
    \item $\alpha_\ell = \gamma \sum_{k=1}^\ell \left( \frac23  \right)^{k}+\alpha_0 
\pause  = 3\gamma\left( 1 -  \left( \frac23  \right)^{\ell+1} \right)+1-\gamma$
\pause  
    \item $C(n)\pause = C(2^\ell)\pause = 3^\ell \alpha_\ell \pause =\gamma(3^{\ell+1}-2^{\ell+1})+(1-\gamma)3^\ell\pause$
    
    \item $C(n) = O(3^\ell)\pause  = O(2^{\ell \frac{\ln 3}{\ln 2}})\pause  = O(n^\frac{\ln 3}{\ln 2})$

  \end{itemize}    
  
 \pause  
 
  \item La complexité de la multiplication de Karatsuba est 
  \mybox{$O(n^\frac{\ln 3}{\ln 2}) \simeq O(n^{1.585})$}
   
\end{enumerate}

\end{frame}



%%%%%%%%%%%%%%%%%%%%%%%%%%%%%%%%%%%%%%%%%%%%%%%%%%%%%%%%%%%%%%%%
\section{Mini-exercices}

\begin{frame}
\small
\begin{miniexercice}
\begin{enumerate}\setlength{\itemsep}{0pt}
  \item Faire une fonction qui renvoie le pgcd de deux polynômes.
  
  \item Comparer les complexités des deux méthodes suivantes pour évaluer un polynôme $P$ en une valeur $x_0 \in \Rr$ :
  $P(x_0)= a_0+a_1x_0+\cdots+a_{n-1}x_0^{n-1}+a_nx_0^n$ et $P(x_0)=a_0+x_0\bigg(a_1 + x_0\big(a_2+ \cdots + x_0(a_{n-1}+a_nx_0)\big)\bigg)$
  (méthode de Horner).
  
  \item Comment trouver le maximum d'une liste ? Montrer que votre méthode est de complexité minimale 
 (en terme du nombre de comparaisons).
  
  \item Soit $f : [a,b] \to \Rr$ une fonction continue vérifiant $f(a)\cdot f(b)\le 0$. Combien d'itérations de la méthode de dichotomie 
  sont nécessaires pour obtenir une racine de $f(x)=0$ avec une précision inférieure à $\epsilon$ ?

  \item Programmer plusieurs façons de calculer les coefficients du binôme de Newton 
  $\binom{n}{k}$ et les comparer.

  \item Trouver une méthode de calcul de $2^n$ qui utilise peu de multiplications. 
  On commencera par écrire $n$ en base $2$.
\end{enumerate}
\end{miniexercice}

\end{frame}

\end{document}
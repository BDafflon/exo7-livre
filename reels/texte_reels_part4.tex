
%%%%%%%%%%%%%%%%%% PREAMBULE %%%%%%%%%%%%%%%%%%


\documentclass[12pt]{article}

\usepackage{amsfonts,amsmath,amssymb,amsthm}
\usepackage[utf8]{inputenc}
\usepackage[T1]{fontenc}
\usepackage[francais]{babel}


% packages
\usepackage{amsfonts,amsmath,amssymb,amsthm}
\usepackage[utf8]{inputenc}
\usepackage[T1]{fontenc}
%\usepackage{lmodern}

\usepackage[francais]{babel}
\usepackage{fancybox}
\usepackage{graphicx}

\usepackage{float}

%\usepackage[usenames, x11names]{xcolor}
\usepackage{tikz}
\usepackage{datetime}

\usepackage{mathptmx}
%\usepackage{fouriernc}
%\usepackage{newcent}
\usepackage[mathcal,mathbf]{euler}

%\usepackage{palatino}
%\usepackage{newcent}


% Commande spéciale prompteur

%\usepackage{mathptmx}
%\usepackage[mathcal,mathbf]{euler}
%\usepackage{mathpple,multido}

\usepackage[a4paper]{geometry}
\geometry{top=2cm, bottom=2cm, left=1cm, right=1cm, marginparsep=1cm}

\newcommand{\change}{{\color{red}\rule{\textwidth}{1mm}\\}}

\newcounter{mydiapo}

\newcommand{\diapo}{\newpage
\hfill {\normalsize  Diapo \themydiapo \quad \texttt{[\jobname]}} \\
\stepcounter{mydiapo}}


%%%%%%% COULEURS %%%%%%%%%%

% Pour blanc sur noir :
%\pagecolor[rgb]{0.5,0.5,0.5}
% \pagecolor[rgb]{0,0,0}
% \color[rgb]{1,1,1}



%\DeclareFixedFont{\myfont}{U}{cmss}{bx}{n}{18pt}
\newcommand{\debuttexte}{
%%%%%%%%%%%%% FONTES %%%%%%%%%%%%%
\renewcommand{\baselinestretch}{1.5}
\usefont{U}{cmss}{bx}{n}
\bfseries

% Taille normale : commenter le reste !
%Taille Arnaud
%\fontsize{19}{19}\selectfont

% Taille Barbara
%\fontsize{21}{22}\selectfont

%Taille François
%\fontsize{25}{30}\selectfont

%Taille Pascal
%\fontsize{25}{30}\selectfont

%Taille Laura
%\fontsize{30}{35}\selectfont


%\myfont
%\usefont{U}{cmss}{bx}{n}

%\Huge
%\addtolength{\parskip}{\baselineskip}
}


% \usepackage{hyperref}
% \hypersetup{colorlinks=true, linkcolor=blue, urlcolor=blue,
% pdftitle={Exo7 - Exercices de mathématiques}, pdfauthor={Exo7}}


%section
% \usepackage{sectsty}
% \allsectionsfont{\bf}
%\sectionfont{\color{Tomato3}\upshape\selectfont}
%\subsectionfont{\color{Tomato4}\upshape\selectfont}

%----- Ensembles : entiers, reels, complexes -----
\newcommand{\Nn}{\mathbb{N}} \newcommand{\N}{\mathbb{N}}
\newcommand{\Zz}{\mathbb{Z}} \newcommand{\Z}{\mathbb{Z}}
\newcommand{\Qq}{\mathbb{Q}} \newcommand{\Q}{\mathbb{Q}}
\newcommand{\Rr}{\mathbb{R}} \newcommand{\R}{\mathbb{R}}
\newcommand{\Cc}{\mathbb{C}} 
\newcommand{\Kk}{\mathbb{K}} \newcommand{\K}{\mathbb{K}}

%----- Modifications de symboles -----
\renewcommand{\epsilon}{\varepsilon}
\renewcommand{\Re}{\mathop{\text{Re}}\nolimits}
\renewcommand{\Im}{\mathop{\text{Im}}\nolimits}
%\newcommand{\llbracket}{\left[\kern-0.15em\left[}
%\newcommand{\rrbracket}{\right]\kern-0.15em\right]}

\renewcommand{\ge}{\geqslant}
\renewcommand{\geq}{\geqslant}
\renewcommand{\le}{\leqslant}
\renewcommand{\leq}{\leqslant}

%----- Fonctions usuelles -----
\newcommand{\ch}{\mathop{\mathrm{ch}}\nolimits}
\newcommand{\sh}{\mathop{\mathrm{sh}}\nolimits}
\renewcommand{\tanh}{\mathop{\mathrm{th}}\nolimits}
\newcommand{\cotan}{\mathop{\mathrm{cotan}}\nolimits}
\newcommand{\Arcsin}{\mathop{\mathrm{Arcsin}}\nolimits}
\newcommand{\Arccos}{\mathop{\mathrm{Arccos}}\nolimits}
\newcommand{\Arctan}{\mathop{\mathrm{Arctan}}\nolimits}
\newcommand{\Argsh}{\mathop{\mathrm{Argsh}}\nolimits}
\newcommand{\Argch}{\mathop{\mathrm{Argch}}\nolimits}
\newcommand{\Argth}{\mathop{\mathrm{Argth}}\nolimits}
\newcommand{\pgcd}{\mathop{\mathrm{pgcd}}\nolimits} 

\newcommand{\Card}{\mathop{\text{Card}}\nolimits}
\newcommand{\Ker}{\mathop{\text{Ker}}\nolimits}
\newcommand{\id}{\mathop{\text{id}}\nolimits}
\newcommand{\ii}{\mathrm{i}}
\newcommand{\dd}{\mathrm{d}}
\newcommand{\Vect}{\mathop{\text{Vect}}\nolimits}
\newcommand{\Mat}{\mathop{\mathrm{Mat}}\nolimits}
\newcommand{\rg}{\mathop{\text{rg}}\nolimits}
\newcommand{\tr}{\mathop{\text{tr}}\nolimits}
\newcommand{\ppcm}{\mathop{\text{ppcm}}\nolimits}

%----- Structure des exercices ------

\newtheoremstyle{styleexo}% name
{2ex}% Space above
{3ex}% Space below
{}% Body font
{}% Indent amount 1
{\bfseries} % Theorem head font
{}% Punctuation after theorem head
{\newline}% Space after theorem head 2
{}% Theorem head spec (can be left empty, meaning ‘normal’)

%\theoremstyle{styleexo}
\newtheorem{exo}{Exercice}
\newtheorem{ind}{Indications}
\newtheorem{cor}{Correction}


\newcommand{\exercice}[1]{} \newcommand{\finexercice}{}
%\newcommand{\exercice}[1]{{\tiny\texttt{#1}}\vspace{-2ex}} % pour afficher le numero absolu, l'auteur...
\newcommand{\enonce}{\begin{exo}} \newcommand{\finenonce}{\end{exo}}
\newcommand{\indication}{\begin{ind}} \newcommand{\finindication}{\end{ind}}
\newcommand{\correction}{\begin{cor}} \newcommand{\fincorrection}{\end{cor}}

\newcommand{\noindication}{\stepcounter{ind}}
\newcommand{\nocorrection}{\stepcounter{cor}}

\newcommand{\fiche}[1]{} \newcommand{\finfiche}{}
\newcommand{\titre}[1]{\centerline{\large \bf #1}}
\newcommand{\addcommand}[1]{}
\newcommand{\video}[1]{}

% Marge
\newcommand{\mymargin}[1]{\marginpar{{\small #1}}}



%----- Presentation ------
\setlength{\parindent}{0cm}

%\newcommand{\ExoSept}{\href{http://exo7.emath.fr}{\textbf{\textsf{Exo7}}}}

\definecolor{myred}{rgb}{0.93,0.26,0}
\definecolor{myorange}{rgb}{0.97,0.58,0}
\definecolor{myyellow}{rgb}{1,0.86,0}

\newcommand{\LogoExoSept}[1]{  % input : echelle
{\usefont{U}{cmss}{bx}{n}
\begin{tikzpicture}[scale=0.1*#1,transform shape]
  \fill[color=myorange] (0,0)--(4,0)--(4,-4)--(0,-4)--cycle;
  \fill[color=myred] (0,0)--(0,3)--(-3,3)--(-3,0)--cycle;
  \fill[color=myyellow] (4,0)--(7,4)--(3,7)--(0,3)--cycle;
  \node[scale=5] at (3.5,3.5) {Exo7};
\end{tikzpicture}}
}



\theoremstyle{definition}
%\newtheorem{proposition}{Proposition}
%\newtheorem{exemple}{Exemple}
%\newtheorem{theoreme}{Théorème}
\newtheorem{lemme}{Lemme}
\newtheorem{corollaire}{Corollaire}
%\newtheorem*{remarque*}{Remarque}
%\newtheorem*{miniexercice}{Mini-exercices}
%\newtheorem{definition}{Définition}




%definition d'un terme
\newcommand{\defi}[1]{{\color{myorange}\textbf{\emph{#1}}}}
\newcommand{\evidence}[1]{{\color{blue}\textbf{\emph{#1}}}}



 %----- Commandes divers ------

\newcommand{\codeinline}[1]{\texttt{#1}}

%%%%%%%%%%%%%%%%%%%%%%%%%%%%%%%%%%%%%%%%%%%%%%%%%%%%%%%%%%%%%
%%%%%%%%%%%%%%%%%%%%%%%%%%%%%%%%%%%%%%%%%%%%%%%%%%%%%%%%%%%%%



\begin{document}

\debuttexte

%%%%%%%%%%%%%%%%%%%%%%%%%%%%%%%%%%%%%%%%%%%%%%%%%%%%%%%%%%%
\diapo


\change
Cette leçon sur la borne supérieure va être riche en définition.

\change

Nous allons d'abord définir ce qu'est le maximum et le minimum d'un ensemble,

\change

Puis la notion de majorants et de minorants

\change

Tout cela afin de définir le principale objet de cette leçon : la borne supérieure
et la borne inférieure.

Le résultat fondamental affirme que ces deux nombres existent dès qu'une partie est bornée.


\change

On termine par des quelques repère historique sur la construction des nombres réels.


%%%%%%%%%%%%%%%%%%%%%%%%%%%%%%%%%%%%%%%%%%%%%%%%%%%%%%%%%%%
\diapo


Soit $A$ une partie non vide de $\Rr$. Nous dirons que le réel $\alpha$ est un 
\defi{plus grand élément} de $A$ s'il vérifie les deux points suivants :

(1) $\alpha \in A$ 
et 
(2) $\forall x \in A \;\; x\leq \alpha$.

\change


S'il existe, ce plus grand élément est unique, on dira donc *le* plus grand élément

On le note  $\max A$.

\change

De façon symétrique on définit le plus petit élément, 

s'il existe c'est le réel $\alpha$ tel que $\alpha \in A$ et $\forall x \in A \;\; x \ge \alpha$.

\change

En fait le plus grand élément s'appelle souvent le \defi{maximum} 

et le plus petit élément, le \defi{minimum}.


Il faut garder à l'esprit que le plus grand élément ou le plus petit élément n'existent pas toujours.


\change

Considérons l'intervalle $[a,b]$ (fermé en $a$ et $b$, avec $a$ plus petit que $b$)

Son plus grand élément est $b$, en effet $b$ est bien un élément de l'intervalle

et il est plus grand que tous les autres éléments.

Quant à son plus petit élément c'est bien sûr $a$.

\change

Par contre l'intervalle ouvert $]a,b[$ n'a ni plus grand élément, ni plus petit élément.

S'il existait un plus grand élément $\alpha$, 

il devrait être plus grand que tout les éléments de l'intervalle

donc $\alpha$ serait plus grand que $b$, mais alors $\alpha$ ne serait pas dans l'intervalle.

\change

Enfin l'intervalle $[0,1[$ a pour plus petit élément $0$ mais n'a pas de plus grand élément.




%%%%%%%%%%%%%%%%%%%%%%%%%%%%%%%%%%%%%%%%%%%%%%%%%%%%%%%%%%%
\diapo

Voyons un exemple plus compliqué :

Définissons la partie  $A$ comme l'ensemble des réels de la forme $1-\frac{1}{n}$,
avec $n=1,2,3$, etc.

\change

Autrement dit si on note $u_n=1-\frac{1}{n}$ alors $A$ est l'ensemble des valeurs prises par la suite $u_n$.

\change



Voici une représentation graphique de $A$ sur la droite numérique :

pour $n=1$ $u_n$ vaut $0$, pour $n=2$, $u_n $ vaut $1/2$, pour $n=3$ $u_n$ vaut $2/3$ 

et lorsque $n$ croit, $u_n$ tend vers $1$ sans jamais l'atteindre.

$A$ est constitué d'une infinité de points, mais ce n'est pas un intervalle.


\change

Tout d'abord  $\min A=0$ :

\change

 Il y a deux choses à vérifier tout d'abord pour $n=1$, $u_n=0$ donc $0$ est bien un élément de notre partie $A$.

\change
Ensuite pour tout $n\ge 1$, les $u_n$ sont plus grand que $0$. 

C'est donc bien $0$ le plus petit élément de $A$.

\change


Nous allons maintenant montrer que $A$ n'a pas de plus grand élément 

\change

Nous raisonnons par l'absurde en supposant qu'il existe un plus grand élément que l'on note $\alpha$.

\change


On aurait alors $u_n \le \alpha$, pour tout $u_n$.

C'est-à-dire que $1-\frac{1}{n} \le \alpha$ 

\change 

Comme $\alpha \ge 1-\frac{1}{n}$ pour tout $n$ alors, 
à la limite, lorsque $n \to +\infty$ cela implique $\alpha \ge 1$.

\change

Mais par définition du plus grand élément $\alpha$ *doit* appartenir à la partie $A$.

\change

Cela signifie qu'il existe un entier $n_0$ tel que $\alpha = u_{n_0}$. 

\change

Mais alors $\alpha$ vaut $1-\frac{1}{n_0}$. 

On en déduit $\alpha$ est strictement plus petit que $1$.

\change

Ce qui entre en contradiction avec $\alpha \ge 1$. 

Ainsi notre hypothèse de départ qu'il existait un plus grand élément est fausse.

Donc $A$ n'a pas de plus grand élément.


%%%%%%%%%%%%%%%%%%%%%%%%%%%%%%%%%%%%%%%%%%%%%%%%%%%%%%%%%%%
\diapo

Passons à une notion différente celle de majorant.

 Un réel grand $M$ est un \defi{majorant} de la partie $A$ si $\forall x \in A \;\; x\leq M$. 

Notez bien la différence avec la définition de plus grand élément, ici on ne demande pas à grand $M$
d'appartenir à la partie $A$.

\change

De même un réel petit $m$ est un \defi{minorant} de $A$ si $\forall x \in A \;\; x\geq m$.

\change

Lorsqu'il existe un majorant on dit que la partie $A$ est \defi{majorée} 

et bien sûr elle est dite minorée s'il existe au moins un minorant.

\change

Quelques exemples :
 $3$ est un majorant de l'intervalle $]0,2[$ ;

\change

 $-7$ est un minorant l'intervalle  $]0,+\infty[$. Mais $-\pi$ est aussi un minorant, de même que $0$.

Par contre il n'y a pas de majorant.



Comme pour le minimum et maximum il n'existe pas toujours de majorant ni de minorant, 

mais ici en plus on n'a pas l'unicité.

\change

Lorsqu'il y a un majorant il y en a en fait une infinité :
par exemple le majorants de l'intervalle $[0,1[$ sont tous les éléments
supérieurs ou égaux à $1$. Les majorants sont exactement les éléments de l'intervalle $[1,+\infty[$

\change

De même les minorants sont tous les réels négatifs ou nuls ce sont donc les éléments de l'intervalle
 $]-\infty,0]$.

\change

Voici la représentation graphique : les majorants sont les éléments plus grands que tous les éléments de $A$,
les minorants sont les éléments plus petits que tous les éléments de $A$.

%%%%%%%%%%%%%%%%%%%%%%%%%%%%%%%%%%%%%%%%%%%%%%%%%%%%%%%%%%%
\diapo


Nous avons vu que même si une partie est bornée elle n'admet pas toujours de plus grand élément.
Pour remédier à ce problème nous allons définir la borne supérieur. 

Attention il faut bien avoir assimilé toutes les notions précédentes de cette leçon !

Soit $A$ une partie non vide de $\Rr$ et $\alpha$ un réel.

Le nombre réel $\alpha$ est la \defi{borne supérieure} de $A$ si d'une part $\alpha$ est un majorant de $A$ et 
d'autre part $\alpha$ est le plus petit des majorants. 

S'il existe on le notera $\sup A$.

\change

De même  $\alpha$ est la \defi{borne inférieure} de $A$ si $\alpha$ est un minorant de 
$A$ et si c'est le plus grand des minorants. 

S'il existe on le note $\inf A$.

\change

Voici quelques exemples :


Pour un intervalle fermé $[a,b]$ le $\sup A$ coincide avec le plus grand élément c'est $b$.

De même la borne inf est $a$.

Par contre pour  l'intervalle ouvert $]a,b[$ la borne sup est $b$, alors qu'il n'y a pas de plus grand élément.
En effet on ne demande pas que la borne sup appartienne à l'ensemble.


Si la partie n'est pas majorée alors il n'y a ni borne sup ni plus grand élément.

Pour cette même partie la borne inf est $0$.



%%%%%%%%%%%%%%%%%%%%%%%%%%%%%%%%%%%%%%%%%%%%%%%%%%%%%%%%%%%
\diapo


Voici le résultat le plus important de cette leçon :

Théorème : Toute partie de $\Rr$ non vide et majorée admet une borne supérieure.

\change

De la même façon : Toute partie de $\Rr$ non vide et minorée admet une borne inférieure.


C'est tout l'intérêt de la borne supérieure par rapport à la notion de plus grand élément,
dès qu'une partie est bornée elle admet toujours une borne supérieure et une borne inférieure. 


Ce qui n'est pas le cas pour le plus grand ou plus petit élément

\change

Etudions on détails la partie $A=]0,1]$ (ouvert en $0$ et fermé en $1$)

\change

 $\sup A=1$ 

\change

 en effet les majorants de $A$ sont tous les réels  supérieurs ou égaux à $1$.

\change

Donc le plus petit des majorants est $1$.

\change

  Pour la borne inf c'est $0$ :

\change

les minorants sont tous les les éléments réels inférieur ou égale à $0$,

\change

 donc le plus grand des minorants est $0$.

Gardez à l'esprit cet exemple : il y a une borne sup $1$ qui est aussi le plus grand élément

il y a une borne inf $0$, mais il n'y a pas de plus petit élément. 


%%%%%%%%%%%%%%%%%%%%%%%%%%%%%%%%%%%%%%%%%%%%%%%%%%%%%%%%%%%
\diapo

Voici une proposition assez délicate qui permet de caractériser la borne supérieure.

La borne supérieure de $A$ est l'unique réel $\sup A$ qui vérifie les deux points suivants :

Tout d'abord c'est un majorant de $A$, c-a-d : Si $x\in A$, alors $x\leq \sup A$,

Et ensuite : pour tout $y<\sup A$, il existe $x\in A$ tel que $y<x$. 

Autrement dit on peut approcher la borne sup d'aussi proche que l'on veut par des éléments de $A$,

c'est crucial lorsque la borne sup n'est pas un élément de $A$.

Cette proposition est utile dans les deux sens :
la borne sup vérifie les points (i) et (ii) et
réciproquement si un réel vérifie les assertion (i) et (ii) alors c'est la borne sup. 

\change

Reprenons l'exemple de la partie $A=\big\{ 1-\frac{1}{n}\big\}$.

\change

Nous avions vu que $\min A = 0$. Lorsque le plus petit élément d'une partie existe alors
la borne inférieure vaut ce plus petit élément : donc $\inf A= \min A=0$.

\change

Nous avions vu qu'il n'y avait pas de plus grand élément cependant nous allons montrer que 
$\sup A=1$. Une première méthode est d'utiliser directement la définition.

\change

 Soit $M$ un majorant de $A$ alors $M \ge 1-\frac 1n$, pour tout $n\ge 1$.
Donc à la limite $M \ge 1$. 

\change

Réciproquement si $M\ge 1$ alors $M$ est un majorant de $A$.
Donc les majorants sont les éléments de $[1,+\infty[$. 
Ainsi le plus petit des majorant est $1$ et donc $\sup A=1$.

\change


Une deuxième méthode est d'utiliser la caractérisation de la borne supérieure. 

\change

Tout d'abord si $x\in A$, alors $x\leq 1$ : $1$ est bien un majorant de $A$

\change

D'autre part pour tout $y< 1$, il existe $x\in A$ tel que $y<x$  

\change

 en effet prenons $n$ suffisamment grand tel que
$0<\frac 1n < 1-y$
 Alors on a $y < 1-\frac 1n < 1$. 

\change

Donc $x=1-\frac 1n$ est compris entre $y$ et $1$ et c'est un élément de $A$

Par la caractérisation de la borne supérieure, $\sup A=1$.


%%%%%%%%%%%%%%%%%%%%%%%%%%%%%%%%%%%%%%%%%%%%%%%%%%%%%%%%%%%
\diapo


Terminons par des remarques générales et historiques.


Les propriétés $\Rr1$, $\Rr2$, $\Rr3$ que nous avons vues dans les leçons précédentes et le théorème $\Rr4$ 
sont des propriétés intrinsèques à la construction de $\Rr$.


La construction de $\Rr$ est assez délicate et nous l'admettons.

\change

Il y a un grand saut entre les rationnels et les réels :
on peut donner un sens précis à l'assertion

 \og{} il y a beaucoup plus de nombres irrationnels que de nombres rationnels \fg{}, 

bien que ces deux ensembles soient infinis, et m\^eme denses dans $\Rr$.



D'autre part, la construction du corps des réels $\Rr$ est beaucoup plus récente 
que celle de $\Qq$ dans l'histoire des mathématiques.

\change

La construction de $\Rr$ devient une nécessité après l'introduction 
du calcul infinitésimal (par Newton et Leibniz vers 1670). 

 Auparavant l'existence de la borne sup allait de soi et était souvent confondue avec la notion de plus grand élément.

\change

Ce n'est pourtant que beaucoup plus tard, dans les années $1860$-$1870$ 
(donc assez récemment dans l'histoire des mathématiques) 

que deux constructions complètes de $\Rr$ sont données :

Une première construction se fait par les coupures de Dedekind,

une seconde par les suites de Cauchy. 


%%%%%%%%%%%%%%%%%%%%%%%%%%%%%%%%%%%%%%%%%%%%%%%%%%%%%%%%%%%
\diapo

Je vous laisse vous entraînez avec ces exercices, 
pour vérifier si vous avez bien compris le cours et assimilez tout le vocabulaire.

\end{document}
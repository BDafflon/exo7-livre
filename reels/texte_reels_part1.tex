
%%%%%%%%%%%%%%%%%% PREAMBULE %%%%%%%%%%%%%%%%%%


\documentclass[12pt]{article}

\usepackage{amsfonts,amsmath,amssymb,amsthm}
\usepackage[utf8]{inputenc}
\usepackage[T1]{fontenc}
\usepackage[francais]{babel}


% packages
\usepackage{amsfonts,amsmath,amssymb,amsthm}
\usepackage[utf8]{inputenc}
\usepackage[T1]{fontenc}
%\usepackage{lmodern}

\usepackage[francais]{babel}
\usepackage{fancybox}
\usepackage{graphicx}

\usepackage{float}

%\usepackage[usenames, x11names]{xcolor}
\usepackage{tikz}
\usepackage{datetime}

\usepackage{mathptmx}
%\usepackage{fouriernc}
%\usepackage{newcent}
\usepackage[mathcal,mathbf]{euler}

%\usepackage{palatino}
%\usepackage{newcent}


% Commande spéciale prompteur

%\usepackage{mathptmx}
%\usepackage[mathcal,mathbf]{euler}
%\usepackage{mathpple,multido}

\usepackage[a4paper]{geometry}
\geometry{top=2cm, bottom=2cm, left=1cm, right=1cm, marginparsep=1cm}

\newcommand{\change}{{\color{red}\rule{\textwidth}{1mm}\\}}

\newcounter{mydiapo}

\newcommand{\diapo}{\newpage
\hfill {\normalsize  Diapo \themydiapo \quad \texttt{[\jobname]}} \\
\stepcounter{mydiapo}}


%%%%%%% COULEURS %%%%%%%%%%

% Pour blanc sur noir :
%\pagecolor[rgb]{0.5,0.5,0.5}
% \pagecolor[rgb]{0,0,0}
% \color[rgb]{1,1,1}



%\DeclareFixedFont{\myfont}{U}{cmss}{bx}{n}{18pt}
\newcommand{\debuttexte}{
%%%%%%%%%%%%% FONTES %%%%%%%%%%%%%
\renewcommand{\baselinestretch}{1.5}
\usefont{U}{cmss}{bx}{n}
\bfseries

% Taille normale : commenter le reste !
%Taille Arnaud
%\fontsize{19}{19}\selectfont

% Taille Barbara
%\fontsize{21}{22}\selectfont

%Taille François
%\fontsize{25}{30}\selectfont

%Taille Pascal
%\fontsize{25}{30}\selectfont

%Taille Laura
%\fontsize{30}{35}\selectfont


%\myfont
%\usefont{U}{cmss}{bx}{n}

%\Huge
%\addtolength{\parskip}{\baselineskip}
}


% \usepackage{hyperref}
% \hypersetup{colorlinks=true, linkcolor=blue, urlcolor=blue,
% pdftitle={Exo7 - Exercices de mathématiques}, pdfauthor={Exo7}}


%section
% \usepackage{sectsty}
% \allsectionsfont{\bf}
%\sectionfont{\color{Tomato3}\upshape\selectfont}
%\subsectionfont{\color{Tomato4}\upshape\selectfont}

%----- Ensembles : entiers, reels, complexes -----
\newcommand{\Nn}{\mathbb{N}} \newcommand{\N}{\mathbb{N}}
\newcommand{\Zz}{\mathbb{Z}} \newcommand{\Z}{\mathbb{Z}}
\newcommand{\Qq}{\mathbb{Q}} \newcommand{\Q}{\mathbb{Q}}
\newcommand{\Rr}{\mathbb{R}} \newcommand{\R}{\mathbb{R}}
\newcommand{\Cc}{\mathbb{C}} 
\newcommand{\Kk}{\mathbb{K}} \newcommand{\K}{\mathbb{K}}

%----- Modifications de symboles -----
\renewcommand{\epsilon}{\varepsilon}
\renewcommand{\Re}{\mathop{\text{Re}}\nolimits}
\renewcommand{\Im}{\mathop{\text{Im}}\nolimits}
%\newcommand{\llbracket}{\left[\kern-0.15em\left[}
%\newcommand{\rrbracket}{\right]\kern-0.15em\right]}

\renewcommand{\ge}{\geqslant}
\renewcommand{\geq}{\geqslant}
\renewcommand{\le}{\leqslant}
\renewcommand{\leq}{\leqslant}

%----- Fonctions usuelles -----
\newcommand{\ch}{\mathop{\mathrm{ch}}\nolimits}
\newcommand{\sh}{\mathop{\mathrm{sh}}\nolimits}
\renewcommand{\tanh}{\mathop{\mathrm{th}}\nolimits}
\newcommand{\cotan}{\mathop{\mathrm{cotan}}\nolimits}
\newcommand{\Arcsin}{\mathop{\mathrm{Arcsin}}\nolimits}
\newcommand{\Arccos}{\mathop{\mathrm{Arccos}}\nolimits}
\newcommand{\Arctan}{\mathop{\mathrm{Arctan}}\nolimits}
\newcommand{\Argsh}{\mathop{\mathrm{Argsh}}\nolimits}
\newcommand{\Argch}{\mathop{\mathrm{Argch}}\nolimits}
\newcommand{\Argth}{\mathop{\mathrm{Argth}}\nolimits}
\newcommand{\pgcd}{\mathop{\mathrm{pgcd}}\nolimits} 

\newcommand{\Card}{\mathop{\text{Card}}\nolimits}
\newcommand{\Ker}{\mathop{\text{Ker}}\nolimits}
\newcommand{\id}{\mathop{\text{id}}\nolimits}
\newcommand{\ii}{\mathrm{i}}
\newcommand{\dd}{\mathrm{d}}
\newcommand{\Vect}{\mathop{\text{Vect}}\nolimits}
\newcommand{\Mat}{\mathop{\mathrm{Mat}}\nolimits}
\newcommand{\rg}{\mathop{\text{rg}}\nolimits}
\newcommand{\tr}{\mathop{\text{tr}}\nolimits}
\newcommand{\ppcm}{\mathop{\text{ppcm}}\nolimits}

%----- Structure des exercices ------

\newtheoremstyle{styleexo}% name
{2ex}% Space above
{3ex}% Space below
{}% Body font
{}% Indent amount 1
{\bfseries} % Theorem head font
{}% Punctuation after theorem head
{\newline}% Space after theorem head 2
{}% Theorem head spec (can be left empty, meaning ‘normal’)

%\theoremstyle{styleexo}
\newtheorem{exo}{Exercice}
\newtheorem{ind}{Indications}
\newtheorem{cor}{Correction}


\newcommand{\exercice}[1]{} \newcommand{\finexercice}{}
%\newcommand{\exercice}[1]{{\tiny\texttt{#1}}\vspace{-2ex}} % pour afficher le numero absolu, l'auteur...
\newcommand{\enonce}{\begin{exo}} \newcommand{\finenonce}{\end{exo}}
\newcommand{\indication}{\begin{ind}} \newcommand{\finindication}{\end{ind}}
\newcommand{\correction}{\begin{cor}} \newcommand{\fincorrection}{\end{cor}}

\newcommand{\noindication}{\stepcounter{ind}}
\newcommand{\nocorrection}{\stepcounter{cor}}

\newcommand{\fiche}[1]{} \newcommand{\finfiche}{}
\newcommand{\titre}[1]{\centerline{\large \bf #1}}
\newcommand{\addcommand}[1]{}
\newcommand{\video}[1]{}

% Marge
\newcommand{\mymargin}[1]{\marginpar{{\small #1}}}



%----- Presentation ------
\setlength{\parindent}{0cm}

%\newcommand{\ExoSept}{\href{http://exo7.emath.fr}{\textbf{\textsf{Exo7}}}}

\definecolor{myred}{rgb}{0.93,0.26,0}
\definecolor{myorange}{rgb}{0.97,0.58,0}
\definecolor{myyellow}{rgb}{1,0.86,0}

\newcommand{\LogoExoSept}[1]{  % input : echelle
{\usefont{U}{cmss}{bx}{n}
\begin{tikzpicture}[scale=0.1*#1,transform shape]
  \fill[color=myorange] (0,0)--(4,0)--(4,-4)--(0,-4)--cycle;
  \fill[color=myred] (0,0)--(0,3)--(-3,3)--(-3,0)--cycle;
  \fill[color=myyellow] (4,0)--(7,4)--(3,7)--(0,3)--cycle;
  \node[scale=5] at (3.5,3.5) {Exo7};
\end{tikzpicture}}
}



\theoremstyle{definition}
%\newtheorem{proposition}{Proposition}
%\newtheorem{exemple}{Exemple}
%\newtheorem{theoreme}{Théorème}
\newtheorem{lemme}{Lemme}
\newtheorem{corollaire}{Corollaire}
%\newtheorem*{remarque*}{Remarque}
%\newtheorem*{miniexercice}{Mini-exercices}
%\newtheorem{definition}{Définition}




%definition d'un terme
\newcommand{\defi}[1]{{\color{myorange}\textbf{\emph{#1}}}}
\newcommand{\evidence}[1]{{\color{blue}\textbf{\emph{#1}}}}



 %----- Commandes divers ------

\newcommand{\codeinline}[1]{\texttt{#1}}

%%%%%%%%%%%%%%%%%%%%%%%%%%%%%%%%%%%%%%%%%%%%%%%%%%%%%%%%%%%%%
%%%%%%%%%%%%%%%%%%%%%%%%%%%%%%%%%%%%%%%%%%%%%%%%%%%%%%%%%%%%%



\begin{document}

\debuttexte


%%%%%%%%%%%%%%%%%%%%%%%%%%%%%%%%%%%%%%%%%%%%%%%%%%%%%%%%%%%
\diapo

\change

Nous commençons ce chapitre sur les nombres réels par 
une première leçon consacrée aux propriétés des nombres rationnels.

\change

Nous allons rappeler ce qu'est l'écriture décimale des nombres.

\change

Puis nous allons voir qu'il existe des nombres irrationnels, comme $\sqrt{2}$.


%%%%%%%%%%%%%%%%%%%%%%%%%%%%%%%%%%%%%%%%%%%%%%%%%%%%%%%%%%
\diapo

Voici une introduction, non seulement à ce chapitre sur les nombres réels,
mais aussi aux premiers chapitres de ce cours d'analyse.

Aux temps des babyloniens le système de numération était en base $60$, 

\change

c'est-à-dire que tous 
les nombres étaient exprimés sous la forme $a+\frac{b}{60} + \frac{c}{60^2}+ \cdots$.


On peut imaginer que pour les applications pratiques c'était largement suffisant (par exemple
estimer la surface d'un champ, le diviser en deux parties égales, etc).

En langage moderne cela correspond à compter uniquement avec des nombres rationnels.

\change

Plus tard les pythagoriciens montrent que $\sqrt 2$ n'entre pas ce cadre là.

\change

C'est-à-dire que $\sqrt2$ ne peut s'écrire sous la forme $\frac pq$ avec $p$ et $q$ deux entiers.

C'est un double saut conceptuel : d'une part concevoir que $\sqrt 2$ est de nature différente 
mais surtout d'en donner une démonstration.

\change

Le fil rouge de ce cours va être un exemple très simple : le nombre $\sqrt{10}$.

\change

Dans ce premier chapitre sur les nombres réels 
vous allez apprendre à montrer que $\sqrt{10}$ n'est pas un nombre rationnel

\change

et ensuite à encadrer $\sqrt{10}$ entre deux entiers consécutifs.


A la fin de ce cours d'analyse vous saurez calculer les premières décimales de $\sqrt{10}$

\change

et même des centaines de décimales et certifier quelles sont exactes.

\change

Pour cela nous aurons besoin d'outils beaucoup plus sophistiqués que
nous étudierons tout au long de ce cours :

nous commencerons par les propriétés des nombres réels,

cela nous permettra d'étudier les suites et leur limite,

nous terminerons avec deux chapitres : les fonctions continues et les fonctions dérivables.

%%%%%%%%%%%%%%%%%%%%%%%%%%%%%%%%%%%%%%%%%%%%%%%%%%%%%%%%%%%
\diapo

Par définition, l'ensemble des \defi{nombres rationnels} est
\[
\Qq = \left\{ \frac{p}{q}  \mid p\in \Zz, q\in \Nn^*\right\}.
\]

\change

On a noté $\Nn^*$ pour $\Nn\setminus\left\{ 0 \right\}$.

\change

Par exemple : $\frac25$ ; $\frac{-7}{10}$ ; $\frac36$ qui s'écrit aussi $\frac12$. 

\change

Les nombres décimaux, c'est-à-dire les nombres de la forme $\frac{a}{10^n}$, 
avec $a\in \Zz$ et $n\in \Nn$, fournissent d'autres exemples 

\change

Tout d'abord $1,234=1234\times 10^{-3}=\frac{1234}{1000}$

\change

Ou encore $0,00345=345\times 10^{-5}=\frac{345}{100\,000}.$


%%%%%%%%%%%%%%%%%%%%%%%%%%%%%%%%%%%%%%%%%%%%%%%%%%%%%%%%%%%
\diapo

Cette proposition caractérise les nombres rationnels par leur écriture décimale :
Un nombre est rationnel si et seulement s'il admet une écriture décimale périodique ou finie.

\change

Expliquons les termes avoir une écriture décimale finie c'est avoir une écriture
décimale qui se termine à un certain rang après la virgule comme

$3/5$ qui s'écrit $0,6$.

\change

$1/3$ a lui une écriture infinie et périodique le $3$ se répète indéfiniment.

\change

On autorise aussi une écriture périodique à partir d'un certain rang, 
ici $325$ se répète indéfiniment, par la proposition c'est un nombre rationnel.


\change

Nous n'allons pas donner la démonstration mais le sens direct ($\implies$) repose
sur la division euclidienne.


Pour la réciproque ($\Longleftarrow$) voyons comment cela marche sur un exemple :
Montrons que $x = 12,34\,\underleftrightarrow{2021}\,\underleftrightarrow{2021}\ldots$ est un rationnel.



\change

L'idée est d'abord de faire apparaître la partie périodique juste après la virgule.
Ici la période commence deux chiffres après la virgule donc on multiplie par $100$ :
\begin{equation}
\label{eq:per1}
100x = 1234,\underleftrightarrow{2021}\,\underleftrightarrow{2021}\ldots
\end{equation}

\change

Maintenant on va décaler tout vers la gauche de la longueur d'une période, 
donc ici on multiplie encore par $10\,000$ pour décaler de $4$ chiffres :
\begin{equation}
\label{eq:per2} 
10\,000 \times 100  x = 1234 \, 2021,\underleftrightarrow{2021}\ldots
\end{equation}

\change

Les parties après la virgule de ces deux lignes 
sont les mêmes, donc si on les soustrait : -- la deuxième moins la première --
alors les parties décimales s'annulent :
$$10\,000 \times 100 x-100x=12\,342\,021-1234$$

\change

cela donne donc $999\,900x=12\,340\,787$ 


\change

Et on a bien écrit notre nombre sous la forme d'un quotient $p/q$ :

$\frac{12\,340\,787}{999\,900}$

C'est donc un nombre rationnel.


%%%%%%%%%%%%%%%%%%%%%%%%%%%%%%%%%%%%%%%%%%%%%%%%%%%%%%%%%%%
\diapo

Il existe des nombres qui ne sont pas rationnels, les \defi{irrationnels}.

Les nombres irrationnels apparaissent naturellement dans certaines figures géométriques :

par exemple la diagonale d'un carré de côté $1$ est le nombre irrationnel $\sqrt{2}$ ;

\change

la circonférence d'un cercle de rayon $\frac12$ est $\pi$ qui est également un nombre irrationnel.


%%%%%%%%%%%%%%%%%%%%%%%%%%%%%%%%%%%%%%%%%%%%%%%%%%%%%%%%%%%
\diapo

Nous allons prouver la proposition suivante : $\sqrt{2}$ n'est pas un nombre rationnel.

\change

Il est important de comprendre et savoir refaire cette démonstration !

\change

Le raisonnement se fait par l'absurde, 

c'est à dire que l'on suppose que $\sqrt2$ *est* un nombre rationnel.

Et nous devons trouver un contradiction.

\change

Comme on a supposé que $\sqrt2$ est rationnel alors il s'écrit
$\frac pq$, où $p$ et $q$ sont deux entiers

\change


de plus --ce sera important pour la suite-- 
on suppose que $p$ et $q$ sont premiers entre eux
(c'est-à-dire que la fraction $\frac pq$ est sous une écriture irréductible).

\change

En élevant au carré, l'égalité  $\sqrt{2}=\frac pq$ devient $2q^2=p^2$.


Cette dernière égalité est une égalité d'entiers. L'entier de gauche est pair, 
donc on en déduit que
$p^2$ est pair ; 

\change

en terme de divisibilité  $2$ divise $p^2$.

\change

Mais si $2$ divise $p^2$ alors $2$ divise $p$ (cela se prouve par facilement l'absurde).

\change

Donc il existe un entier $p'\in \Zz$ tel que $p=2p'$. 

\change

Repartons de l'égalité $2q^2=p^2$ et remplaçons $p$ par $2p'$. 

\change

Cela donne
$2q^2=4p'^2$. 

\change

Donc $q^2=2p'^2$. 

\change

Maintenant cela entraîne que $2$ divise $q^2$
et comme avant alors $2$ divise $q$.

\change

Nous avons prouvé que $2$ divise à la fois $p$ et $q$. 

\change

Cela rentre en contradiction avec le fait que $p$ et $q$ sont premiers
entre eux. 

\change

Notre hypothèse de départ est donc fausse :

\change

ainsi nous avons bien prouvé que $\sqrt2$ *n'est pas* un nombre rationnel.


%%%%%%%%%%%%%%%%%%%%%%%%%%%%%%%%%%%%%%%%%%%%%%%%%%%%%%%%%%%
\diapo

On représente souvent les nombres réels sur une \og{}droite numérique\fg{}, comme sur la figure suivante :

\change

Il est bon de connaître les premières décimales de certains réels 
$\sqrt{2}\simeq 1,4142\ldots$ \quad 
$\pi\simeq 3,14159265\ldots$ \quad $e\simeq 2,718\ldots$

\change

Enfin il est souvent pratique de rajouter les deux extrémités à la droite numérique.
On définit ainsi \[ \overline{\Rr}=\Rr\cup\{-\infty,\infty\}  \].





%%%%%%%%%%%%%%%%%%%%%%%%%%%%%%%%%%%%%%%%%%%%%%%%%%%%%%%%%%%
\diapo

Entraînez avec ces exercices, 
pour vérifier que vous avez bien compris le cours.

\end{document}

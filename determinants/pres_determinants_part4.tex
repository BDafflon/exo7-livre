
%%%%%%%%%%%%%%%%%% PREAMBULE %%%%%%%%%%%%%%%%%%

\documentclass[aspectratio=169,utf8]{beamer}
%\documentclass[aspectratio=169,handout]{beamer}

\usetheme{Boadilla}
%\usecolortheme{seahorse}
%\usecolortheme[RGB={245,66,24}]{structure}
\useoutertheme{infolines}

% packages
\usepackage{amsfonts,amsmath,amssymb,amsthm}
\usepackage[utf8]{inputenc}
\usepackage[T1]{fontenc}
\usepackage{lmodern}

\usepackage[francais]{babel}
\usepackage{fancybox}
\usepackage{graphicx}

\usepackage{float}
\usepackage{xfrac}

%\usepackage[usenames, x11names]{xcolor}
\usepackage{pgfplots}
\usepackage{datetime}


% ----------------------------------------------------------------------
% Pour les images
\usepackage{tikz}
\usetikzlibrary{calc,shadows,arrows.meta,patterns,matrix}

\newcommand{\tikzinput}[1]{\input{figures/#1.tikz}}
% --- les figures avec échelle éventuel
\newcommand{\myfigure}[2]{% entrée : échelle, fichier(s) figure à inclure
\begin{center}\small%
\tikzstyle{every picture}=[scale=1.0*#1]% mise en échelle + 0% (automatiquement annulé à la fin du groupe)
#2%
\end{center}}



%-----  Package unités -----
\usepackage{siunitx}
\sisetup{locale = FR,detect-all,per-mode = symbol}

%\usepackage{mathptmx}
%\usepackage{fouriernc}
%\usepackage{newcent}
%\usepackage[mathcal,mathbf]{euler}

%\usepackage{palatino}
%\usepackage{newcent}
% \usepackage[mathcal,mathbf]{euler}



% \usepackage{hyperref}
% \hypersetup{colorlinks=true, linkcolor=blue, urlcolor=blue,
% pdftitle={Exo7 - Exercices de mathématiques}, pdfauthor={Exo7}}


%section
% \usepackage{sectsty}
% \allsectionsfont{\bf}
%\sectionfont{\color{Tomato3}\upshape\selectfont}
%\subsectionfont{\color{Tomato4}\upshape\selectfont}

%----- Ensembles : entiers, reels, complexes -----
\newcommand{\Nn}{\mathbb{N}} \newcommand{\N}{\mathbb{N}}
\newcommand{\Zz}{\mathbb{Z}} \newcommand{\Z}{\mathbb{Z}}
\newcommand{\Qq}{\mathbb{Q}} \newcommand{\Q}{\mathbb{Q}}
\newcommand{\Rr}{\mathbb{R}} \newcommand{\R}{\mathbb{R}}
\newcommand{\Cc}{\mathbb{C}} 
\newcommand{\Kk}{\mathbb{K}} \newcommand{\K}{\mathbb{K}}

%----- Modifications de symboles -----
\renewcommand{\epsilon}{\varepsilon}
\renewcommand{\Re}{\mathop{\text{Re}}\nolimits}
\renewcommand{\Im}{\mathop{\text{Im}}\nolimits}
%\newcommand{\llbracket}{\left[\kern-0.15em\left[}
%\newcommand{\rrbracket}{\right]\kern-0.15em\right]}

\renewcommand{\ge}{\geqslant}
\renewcommand{\geq}{\geqslant}
\renewcommand{\le}{\leqslant}
\renewcommand{\leq}{\leqslant}
\renewcommand{\epsilon}{\varepsilon}

%----- Fonctions usuelles -----
\newcommand{\ch}{\mathop{\text{ch}}\nolimits}
\newcommand{\sh}{\mathop{\text{sh}}\nolimits}
\renewcommand{\tanh}{\mathop{\text{th}}\nolimits}
\newcommand{\cotan}{\mathop{\text{cotan}}\nolimits}
\newcommand{\Arcsin}{\mathop{\text{arcsin}}\nolimits}
\newcommand{\Arccos}{\mathop{\text{arccos}}\nolimits}
\newcommand{\Arctan}{\mathop{\text{arctan}}\nolimits}
\newcommand{\Argsh}{\mathop{\text{argsh}}\nolimits}
\newcommand{\Argch}{\mathop{\text{argch}}\nolimits}
\newcommand{\Argth}{\mathop{\text{argth}}\nolimits}
\newcommand{\pgcd}{\mathop{\text{pgcd}}\nolimits} 


%----- Commandes divers ------
\newcommand{\ii}{\mathrm{i}}
\newcommand{\dd}{\text{d}}
\newcommand{\id}{\mathop{\text{id}}\nolimits}
\newcommand{\Ker}{\mathop{\text{Ker}}\nolimits}
\newcommand{\Card}{\mathop{\text{Card}}\nolimits}
\newcommand{\Vect}{\mathop{\text{Vect}}\nolimits}
\newcommand{\Mat}{\mathop{\text{Mat}}\nolimits}
\newcommand{\rg}{\mathop{\text{rg}}\nolimits}
\newcommand{\tr}{\mathop{\text{tr}}\nolimits}


%----- Structure des exercices ------

\newtheoremstyle{styleexo}% name
{2ex}% Space above
{3ex}% Space below
{}% Body font
{}% Indent amount 1
{\bfseries} % Theorem head font
{}% Punctuation after theorem head
{\newline}% Space after theorem head 2
{}% Theorem head spec (can be left empty, meaning ‘normal’)

%\theoremstyle{styleexo}
\newtheorem{exo}{Exercice}
\newtheorem{ind}{Indications}
\newtheorem{cor}{Correction}


\newcommand{\exercice}[1]{} \newcommand{\finexercice}{}
%\newcommand{\exercice}[1]{{\tiny\texttt{#1}}\vspace{-2ex}} % pour afficher le numero absolu, l'auteur...
\newcommand{\enonce}{\begin{exo}} \newcommand{\finenonce}{\end{exo}}
\newcommand{\indication}{\begin{ind}} \newcommand{\finindication}{\end{ind}}
\newcommand{\correction}{\begin{cor}} \newcommand{\fincorrection}{\end{cor}}

\newcommand{\noindication}{\stepcounter{ind}}
\newcommand{\nocorrection}{\stepcounter{cor}}

\newcommand{\fiche}[1]{} \newcommand{\finfiche}{}
\newcommand{\titre}[1]{\centerline{\large \bf #1}}
\newcommand{\addcommand}[1]{}
\newcommand{\video}[1]{}

% Marge
\newcommand{\mymargin}[1]{\marginpar{{\small #1}}}

\def\noqed{\renewcommand{\qedsymbol}{}}


%----- Presentation ------
\setlength{\parindent}{0cm}

%\newcommand{\ExoSept}{\href{http://exo7.emath.fr}{\textbf{\textsf{Exo7}}}}

\definecolor{myred}{rgb}{0.93,0.26,0}
\definecolor{myorange}{rgb}{0.97,0.58,0}
\definecolor{myyellow}{rgb}{1,0.86,0}

\newcommand{\LogoExoSept}[1]{  % input : echelle
{\usefont{U}{cmss}{bx}{n}
\begin{tikzpicture}[scale=0.1*#1,transform shape]
  \fill[color=myorange] (0,0)--(4,0)--(4,-4)--(0,-4)--cycle;
  \fill[color=myred] (0,0)--(0,3)--(-3,3)--(-3,0)--cycle;
  \fill[color=myyellow] (4,0)--(7,4)--(3,7)--(0,3)--cycle;
  \node[scale=5] at (3.5,3.5) {Exo7};
\end{tikzpicture}}
}


\newcommand{\debutmontitre}{
  \author{} \date{} 
  \thispagestyle{empty}
  \hspace*{-10ex}
  \begin{minipage}{\textwidth}
    \titlepage  
  \vspace*{-2.5cm}
  \begin{center}
    \LogoExoSept{2.5}
  \end{center}
  \end{minipage}

  \vspace*{-0cm}
  
  % Astuce pour que le background ne soit pas discrétisé lors de la conversion pdf -> png
\begin{tikzpicture}
        \fill[opacity=0,green!60!black] (0,0)--++(0,0)--++(0,0)--++(0,0)--cycle; 
\end{tikzpicture}

% toc S'affiche trop tot :
% \tableofcontents[hideallsubsections, pausesections]
}

\newcommand{\finmontitre}{
  \end{frame}
  \setcounter{framenumber}{0}
} % ne marche pas pour une raison obscure

%----- Commandes supplementaires ------

% \usepackage[landscape]{geometry}
% \geometry{top=1cm, bottom=3cm, left=2cm, right=10cm, marginparsep=1cm
% }
% \usepackage[a4paper]{geometry}
% \geometry{top=2cm, bottom=2cm, left=2cm, right=2cm, marginparsep=1cm
% }

%\usepackage{standalone}


% New command Arnaud -- november 2011
\setbeamersize{text margin left=24ex}
% si vous modifier cette valeur il faut aussi
% modifier le decalage du titre pour compenser
% (ex : ici =+10ex, titre =-5ex

\theoremstyle{definition}
%\newtheorem{proposition}{Proposition}
%\newtheorem{exemple}{Exemple}
%\newtheorem{theoreme}{Théorème}
%\newtheorem{lemme}{Lemme}
%\newtheorem{corollaire}{Corollaire}
%\newtheorem*{remarque*}{Remarque}
%\newtheorem*{miniexercice}{Mini-exercices}
%\newtheorem{definition}{Définition}

% Commande tikz
\usetikzlibrary{calc}
\usetikzlibrary{patterns,arrows}
\usetikzlibrary{matrix}
\usetikzlibrary{fadings} 

%definition d'un terme
\newcommand{\defi}[1]{{\color{myorange}\textbf{\emph{#1}}}}
\newcommand{\evidence}[1]{{\color{blue}\textbf{\emph{#1}}}}
\newcommand{\assertion}[1]{\emph{\og#1\fg}}  % pour chapitre logique
%\renewcommand{\contentsname}{Sommaire}
\renewcommand{\contentsname}{}
\setcounter{tocdepth}{2}



%------ Encadrement ------

\usepackage{fancybox}


\newcommand{\mybox}[1]{
\setlength{\fboxsep}{7pt}
\begin{center}
\shadowbox{#1}
\end{center}}

\newcommand{\myboxinline}[1]{
\setlength{\fboxsep}{5pt}
\raisebox{-10pt}{
\shadowbox{#1}
}
}

%--------------- Commande beamer---------------
\newcommand{\beameronly}[1]{#1} % permet de mettre des pause dans beamer pas dans poly


\setbeamertemplate{navigation symbols}{}
\setbeamertemplate{footline}  % tiré du fichier beamerouterinfolines.sty
{
  \leavevmode%
  \hbox{%
  \begin{beamercolorbox}[wd=.333333\paperwidth,ht=2.25ex,dp=1ex,center]{author in head/foot}%
    % \usebeamerfont{author in head/foot}\insertshortauthor%~~(\insertshortinstitute)
    \usebeamerfont{section in head/foot}{\bf\insertshorttitle}
  \end{beamercolorbox}%
  \begin{beamercolorbox}[wd=.333333\paperwidth,ht=2.25ex,dp=1ex,center]{title in head/foot}%
    \usebeamerfont{section in head/foot}{\bf\insertsectionhead}
  \end{beamercolorbox}%
  \begin{beamercolorbox}[wd=.333333\paperwidth,ht=2.25ex,dp=1ex,right]{date in head/foot}%
    % \usebeamerfont{date in head/foot}\insertshortdate{}\hspace*{2em}
    \insertframenumber{} / \inserttotalframenumber\hspace*{2ex} 
  \end{beamercolorbox}}%
  \vskip0pt%
}


\definecolor{mygrey}{rgb}{0.5,0.5,0.5}
\setlength{\parindent}{0cm}
%\DeclareTextFontCommand{\helvetica}{\fontfamily{phv}\selectfont}

% background beamer
\definecolor{couleurhaut}{rgb}{0.85,0.9,1}  % creme
\definecolor{couleurmilieu}{rgb}{1,1,1}  % vert pale
\definecolor{couleurbas}{rgb}{0.85,0.9,1}  % blanc
\setbeamertemplate{background canvas}[vertical shading]%
[top=couleurhaut,middle=couleurmilieu,midpoint=0.4,bottom=couleurbas] 
%[top=fondtitre!05,bottom=fondtitre!60]



\makeatletter
\setbeamertemplate{theorem begin}
{%
  \begin{\inserttheoremblockenv}
  {%
    \inserttheoremheadfont
    \inserttheoremname
    \inserttheoremnumber
    \ifx\inserttheoremaddition\@empty\else\ (\inserttheoremaddition)\fi%
    \inserttheorempunctuation
  }%
}
\setbeamertemplate{theorem end}{\end{\inserttheoremblockenv}}

\newenvironment{theoreme}[1][]{%
   \setbeamercolor{block title}{fg=structure,bg=structure!40}
   \setbeamercolor{block body}{fg=black,bg=structure!10}
   \begin{block}{{\bf Th\'eor\`eme }#1}
}{%
   \end{block}%
}


\newenvironment{proposition}[1][]{%
   \setbeamercolor{block title}{fg=structure,bg=structure!40}
   \setbeamercolor{block body}{fg=black,bg=structure!10}
   \begin{block}{{\bf Proposition }#1}
}{%
   \end{block}%
}

\newenvironment{corollaire}[1][]{%
   \setbeamercolor{block title}{fg=structure,bg=structure!40}
   \setbeamercolor{block body}{fg=black,bg=structure!10}
   \begin{block}{{\bf Corollaire }#1}
}{%
   \end{block}%
}

\newenvironment{mydefinition}[1][]{%
   \setbeamercolor{block title}{fg=structure,bg=structure!40}
   \setbeamercolor{block body}{fg=black,bg=structure!10}
   \begin{block}{{\bf Définition} #1}
}{%
   \end{block}%
}

\newenvironment{lemme}[0]{%
   \setbeamercolor{block title}{fg=structure,bg=structure!40}
   \setbeamercolor{block body}{fg=black,bg=structure!10}
   \begin{block}{\bf Lemme}
}{%
   \end{block}%
}

\newenvironment{remarque}[1][]{%
   \setbeamercolor{block title}{fg=black,bg=structure!20}
   \setbeamercolor{block body}{fg=black,bg=structure!5}
   \begin{block}{Remarque #1}
}{%
   \end{block}%
}


\newenvironment{exemple}[1][]{%
   \setbeamercolor{block title}{fg=black,bg=structure!20}
   \setbeamercolor{block body}{fg=black,bg=structure!5}
   \begin{block}{{\bf Exemple }#1}
}{%
   \end{block}%
}


\newenvironment{miniexercice}[0]{%
   \setbeamercolor{block title}{fg=structure,bg=structure!20}
   \setbeamercolor{block body}{fg=black,bg=structure!5}
   \begin{block}{Mini-exercices}
}{%
   \end{block}%
}


\newenvironment{tp}[0]{%
   \setbeamercolor{block title}{fg=structure,bg=structure!40}
   \setbeamercolor{block body}{fg=black,bg=structure!10}
   \begin{block}{\bf Travaux pratiques}
}{%
   \end{block}%
}
\newenvironment{exercicecours}[1][]{%
   \setbeamercolor{block title}{fg=structure,bg=structure!40}
   \setbeamercolor{block body}{fg=black,bg=structure!10}
   \begin{block}{{\bf Exercice }#1}
}{%
   \end{block}%
}
\newenvironment{algo}[1][]{%
   \setbeamercolor{block title}{fg=structure,bg=structure!40}
   \setbeamercolor{block body}{fg=black,bg=structure!10}
   \begin{block}{{\bf Algorithme}\hfill{\color{gray}\texttt{#1}}}
}{%
   \end{block}%
}


\setbeamertemplate{proof begin}{
   \setbeamercolor{block title}{fg=black,bg=structure!20}
   \setbeamercolor{block body}{fg=black,bg=structure!5}
   \begin{block}{{\footnotesize Démonstration}}
   \footnotesize
   \smallskip}
\setbeamertemplate{proof end}{%
   \end{block}}
\setbeamertemplate{qed symbol}{\openbox}


\makeatother
\usecolortheme[RGB={204,0,0}]{structure}
   
%%%%%%%%%%%%%%%%%%%%%%%%%%%%%%%%%%%%%%%%%%%%%%%%%%%%%%%%%%%%%
%%%%%%%%%%%%%%%%%%%%%%%%%%%%%%%%%%%%%%%%%%%%%%%%%%%%%%%%%%%%%


\begin{document}


\title{{\bf Déterminants}}
\subtitle{Calculs de déterminants}

\begin{frame}
  
  \debutmontitre

  \pause

{\footnotesize
\hfill
\setbeamercovered{transparent=50}
\begin{minipage}{0.6\textwidth}
  \begin{itemize}
    \item<3-> Cofacteur
    \item<4-> Développement suivant une ligne
    \item<5-> Exemple
    \item<6-> Inverse d'une matrice
  \end{itemize}
\end{minipage}
}

\end{frame}

\setcounter{framenumber}{0}


%%%%%%%%%%%%%%%%%%%%%%%%%%%%%%%%%%%%%%%%%%%%%%%%%%%%%%%%%%%%%%%%
\section{Cofacteur}

\begin{frame}
\begin{mydefinition}

Soit $A = \big( a_{ij}\big) \in M_n(\Kk)$ une matrice carrée
 
\begin{itemize}
  \item\pause $A_{ij}$ est la matrice extraite obtenue en effaçant la ligne~$i$ et la colonne $j$ de $A$
  \item\pause Le nombre $\det A_{ij}$ est un \defi{mineur d'ordre $n-1$} de la matrice $A$
  \item\pause Le nombre $C_{ij} = (-1)^{i+j}\det A_{ij}$ est le \defi{cofacteur} de $A$ 
  relatif au coefficient $a_{ij}$
\end{itemize}
\end{mydefinition}

\end{frame}


\begin{frame}
\begin{itemize}
  \item $A_{ij} =$ matrice obtenue en effaçant la ligne~$i$ et la colonne $j$ de $A$
  \item<4-> $C_{ij} = (-1)^{i+j}\det A_{ij} $ cofacteur de $A$ relatif au coefficient $a_{ij}$
\end{itemize}

 \begin{overprint}
\onslide<2> 
\myfigure{2}{
\tikzinput{fig_determinants07} 
}	
  
\onslide<3,4> 
\vspace{.2cm}
\[
A_{ij} = \begin{pmatrix}
a_{1,1} & \dots & a_{1,j-1} & a_{1,j+1} &\dots & a_{1,n}\\
\vdots &&\vdots &\vdots &&\vdots \\
a_{i-1,1} & \dots & a_{i-1,j-1} & a_{i-1,j+1} &\dots & a_{i-1,n}\\
a_{i+1,1} & \dots & a_{i+1,j-1} & a_{i+1,j+1} &\dots & a_{i+1,n}\\
\vdots &&\vdots &&&\vdots \\
a_{n,1} & \dots & a_{n,j-1} & a_{n,j+1}& \dots & a_{n,n}  
\end{pmatrix}
\]
\onslide<5-> 
\vspace{.2cm}
\[
C_{ij} = (-1)^{i+j}\begin{vmatrix}
a_{1,1} & \dots & a_{1,j-1} & a_{1,j+1} &\dots & a_{1,n}\\
\vdots &&\vdots &\vdots &&\vdots \\
a_{i-1,1} & \dots & a_{i-1,j-1} & a_{i-1,j+1} &\dots & a_{i-1,n}\\
a_{i+1,1} & \dots & a_{i+1,j-1} & a_{i+1,j+1} &\dots & a_{i+1,n}\\
\vdots &&\vdots &&&\vdots \\
a_{n,1} & \dots & a_{n,j-1} & a_{n,j+1}& \dots & a_{n,n}  
\end{vmatrix}
\]
 \end{overprint}
 
 \vspace{-2.5cm}
\begin{itemize}
  \item<6-> $C_{ij} = +\det A_{ij} $ ou $C_{ij} = -\det A_{ij}$ ?
\end{itemize} 
\onslide<7> 
$$  A = 
\begin{pmatrix}
+ & - & + & - &\dots\\
- & + & - & + &\dots \\
+ & - & + & - &\dots\\
\vdots & \vdots & \vdots & \vdots &          
\end{pmatrix}
$$


\end{frame}

%--------------------------------------------------------------

\begin{frame}

\begin{exemple}
Soit $ A = 
\begin{pmatrix}
1 & 2 & 3\\
4 & 2 & 1\\
0 & 1 & 1  
\end{pmatrix}$.
Calculons $A_{11}, C_{11}, A_{32}, C_{32}$

\pause 

\myfigure{1}{
\tikzinput{fig_determinants07b-pres} 
}

\pause
\onslide<6->{

\myfigure{1}{
\tikzinput{fig_determinants07c-pres} 
}
}
\end{exemple} 

\end{frame}


%%%%%%%%%%%%%%%%%%%%%%%%%%%%%%%%%%%%%%%%%%%%%%%%%%%%%%%%%%%%%%%%
\section{Développement suivant une ligne ou une colonne}

\begin{frame}

\evidence{Développement suivant une ligne ou une colonne}

\begin{theoreme}
\pause
Formule de développement par rapport à la ligne $i$
$$\det A 
\pause = \sum_{j=1}^n (-1)^{i+j} a_{ij} \det A_{ij}
\pause = \sum_{j=1}^n a_{ij} C_{ij}$$

\pause
Formule de développement par rapport à la colonne $j$
$$\det A 
= \sum_{i=1}^n (-1)^{i+j} a_{ij} \det A_{ij}
\pause = \sum_{i=1}^n a_{ij}C_{ij}$$
\end{theoreme}

\end{frame}

\begin{frame}

\begin{exemple}
Retrouvons la règle de Sarrus en développement par rapport à la première ligne

\pause
$$\begin{array}{rcl}
\left|\begin{matrix}
a_{11}&a_{12}&a_{13}\\
a_{21}&a_{22}&a_{23}\\
a_{31}&a_{32}&a_{33}
\end{matrix}\right|& \pause = &a_{11}C_{11}+a_{12}C_{12}+a_{13}C_{13}\\ \pause
&=&a_{11}\left|\begin{matrix}
a_{22}&a_{23}\\
a_{32}&a_{33}
\end{matrix}\right| \pause - a_{12}\left|\begin{matrix}
a_{21}&a_{23}\\
a_{31}&a_{33}
\end{matrix}\right|+a_{13}\left|\begin{matrix}
a_{21}&a_{22}\\
a_{31}&a_{32}
\end{matrix}\right| \vspace{.3cm} \\ \pause
&=&a_{11}(a_{22}a_{33}-a_{32}a_{23}) \pause -a_{12}(a_{21}a_{33}-a_{31}a_{23})\\ 
&&{}+a_{13}(a_{21}a_{32}-a_{31}a_{22}) \vspace{.3cm} \\ \pause
&=&a_{11}a_{22}a_{33}-a_{11}a_{32}a_{23}+a_{12}a_{31}a_{23}-a_{12}a_{21}a_{33}\\ 
&&{}+a_{13}a_{21}a_{32}-a_{13}a_{31}a_{22}
\end{array}$$
\end{exemple}

\end{frame}



%%%%%%%%%%%%%%%%%%%%%%%%%%%%%%%%%%%%%%%%%%%%%%%%%%%%%%%%%%%%%%%%
\section{Exemple}

\begin{frame}

\begin{exemple}
$$A = \begin{pmatrix}
4 & 0 & 3 & 1\\
4 & 2 & 1 & 0\\
0 & 3 & 1 & -1\\
1 & 0 & 2 & 3
\end{pmatrix}
$$

\vspace{-10pt}
\pause

$$\begin{array}{rcl}
\det A 
  & = \pause & 0 C_{12} + 2 C_{22} + 3 C_{32}+0 C_{42} \quad \onslide<2->\color{gray}\text{dévelop. par rapport à C2} \\[7pt]
\pause   & = & +2 \begin{vmatrix}4&3&1\\0&1&-1\\1&2&3 \end{vmatrix}
        -3 \begin{vmatrix}4&3&1\\4&1&0\\1&2&3\\ \end{vmatrix} \quad   \pause
   \onslide<5->\begin{array}{c}
   \color{gray}\text{on développe} \\ \color{gray}\text{les déterminants $3\times 3$}
   \end{array}
    \\[7pt]
\pause\pause
  & = & 2 \left(+4\begin{vmatrix}1&-1\\2&3\end{vmatrix}-0\begin{vmatrix}3&1\\2&3\end{vmatrix}
  +1\begin{vmatrix}3&1\\1&-1\end{vmatrix}\right) \quad  \onslide<6->\color{gray}\text{par rapport à C1}\\[14pt]
\pause\pause
  &   & -3\left(-4\begin{vmatrix}3&1\\2&3\end{vmatrix}+1\begin{vmatrix}4&1\\1&3\end{vmatrix}
  -0\begin{vmatrix}4&3\\1&2\end{vmatrix}\right)\quad \onslide<8->\color{gray}\text{par rapport à L2}\\[14pt]
\pause & = & 2 \big(4\times 5 +1\times(-4)\big) 
 		-3\big(-4\times7 +1\times11  \big) \ \pause = \ 83\\
\end{array}$$
\end{exemple} 


\end{frame}


\begin{frame}

\begin{remarque}

\begin{itemize}
  \item Par développement par rapport à une ligne on se ramène
  \begin{itemize}
  \item à $n$ déterminants $(n-1)\times(n-1)$
  \item\pause et par récurrence à  $n!$ sous-déterminants...
\end{itemize}
  \item\pause Il faut que $A$ ait beaucoup de zéros
  \item\pause  On commence par faire apparaître des zéros par des opérations élémentaires sur les lignes et les colonnes
\end{itemize}
\end{remarque}

\end{frame}



%%%%%%%%%%%%%%%%%%%%%%%%%%%%%%%%%%%%%%%%%%%%%%%%%%%%%%%%%%%%%%%%
\section{Inverse d'une matrice}

\begin{frame}

Soit $A \in M_n(\Kk)$

La \defi{comatrice} $C$ est la matrice des cofacteurs

\pause
$$C = (C_{ij}) = \left(
\begin{array}{cccc}
C_{11} & C_{12} & \cdots & C_{1n}\\
C_{21} & C_{22} & \cdots & C_{2n}\\
\vdots & \vdots & & \vdots\\
C_{n1} & C_{n2} & \cdots & C_{nn}
\end{array}\right)
$$

\pause
\begin{theoreme}
Soient $A$ une matrice inversible et $C$ sa comatrice. \pause
On a alors
\mybox{$\displaystyle A^{-1} = \frac{1}{\det A} \, C^T$}
\end{theoreme}

\end{frame}


\begin{frame}

\begin{exemple}
Soit $\displaystyle A = \left(
\begin{array}{ccc}
1 & 1 & 0\\
0 & 1 & 1\\
1 & 0 & 1
\end{array}\right)$

\begin{itemize}
  \item\pause $\det A = 2$ \pause $\implies A $ est inversible 
  \item\pause La comatrice $C$ 
s'obtient en calculant $9$~déterminants
$2\times 2$ 

(sans oublier les signes $+/-$)
\pause
$$C = \begin{pmatrix}
1 & 1 & -1\\
-1 & 1 & 1\\
1 & -1 & 1        
      \end{pmatrix}$$
  \item\pause  Donc 
  $$
A^{-1} = \frac{1}{\det A} \cdot C^T \pause = \frac{1}{2}
\begin{pmatrix}
1 & -1 & 1\\
1 & 1 & -1\\
-1 & 1 & 1  
\end{pmatrix}
$$
\end{itemize}

\end{exemple} 
\end{frame}




%%%%%%%%%%%%%%%%%%%%%%%%%%%%%%%%%%%%%%%%%%%%%%%%%%%%%%%%%%%%%%%%
\section{Mini-exercices}

\begin{frame}
\begin{miniexercice}

\begin{enumerate}
  \item Soient $A=\begin{pmatrix}2&0&-2\\0&1&-1\\2&0&0\end{pmatrix}$
  et $B = \begin{pmatrix}t&0&t\\0&t&0\\-t&0&t\end{pmatrix}$.
  
  
  Calculer les matrices extraites, les mineurs d'ordre $2$ et les cofacteurs de chacune
  des matrices $A$ et $B$. En déduire $\det A$ et $\det B$, puis 
  l'inverse de $A$ et de $B$ lorsque c'est possible.
  
  \item Par développement suivant une ligne (ou une colonne) bien choisie, calculer les déterminants :
  $$\begin{vmatrix}1&0&1&2\\0&0&0&1\\1&1&1&0\\0&0&0&-1\end{vmatrix}
  \qquad\qquad
  \begin{vmatrix}t&0&1&0\\1&t&0&0\\0&1&t&0\\0&0&0&t\end{vmatrix}$$

  \item En utilisant la formule de développement par rapport à une ligne,
  recalculer le déterminant d'une matrice triangulaire.
\end{enumerate}
\end{miniexercice}
\end{frame}

\end{document}
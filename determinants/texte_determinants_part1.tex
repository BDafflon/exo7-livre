
%%%%%%%%%%%%%%%%%% PREAMBULE %%%%%%%%%%%%%%%%%%


\documentclass[12pt]{article}

\usepackage{amsfonts,amsmath,amssymb,amsthm}
\usepackage[utf8]{inputenc}
\usepackage[T1]{fontenc}
\usepackage[francais]{babel}


% packages
\usepackage{amsfonts,amsmath,amssymb,amsthm}
\usepackage[utf8]{inputenc}
\usepackage[T1]{fontenc}
%\usepackage{lmodern}

\usepackage[francais]{babel}
\usepackage{fancybox}
\usepackage{graphicx}

\usepackage{float}

%\usepackage[usenames, x11names]{xcolor}
\usepackage{tikz}
\usepackage{datetime}

\usepackage{mathptmx}
%\usepackage{fouriernc}
%\usepackage{newcent}
\usepackage[mathcal,mathbf]{euler}

%\usepackage{palatino}
%\usepackage{newcent}


% Commande spéciale prompteur

%\usepackage{mathptmx}
%\usepackage[mathcal,mathbf]{euler}
%\usepackage{mathpple,multido}

\usepackage[a4paper]{geometry}
\geometry{top=2cm, bottom=2cm, left=1cm, right=1cm, marginparsep=1cm}

\newcommand{\change}{{\color{red}\rule{\textwidth}{1mm}\\}}

\newcounter{mydiapo}

\newcommand{\diapo}{\newpage
\hfill {\normalsize  Diapo \themydiapo \quad \texttt{[\jobname]}} \\
\stepcounter{mydiapo}}


%%%%%%% COULEURS %%%%%%%%%%

% Pour blanc sur noir :
%\pagecolor[rgb]{0.5,0.5,0.5}
% \pagecolor[rgb]{0,0,0}
% \color[rgb]{1,1,1}



%\DeclareFixedFont{\myfont}{U}{cmss}{bx}{n}{18pt}
\newcommand{\debuttexte}{
%%%%%%%%%%%%% FONTES %%%%%%%%%%%%%
\renewcommand{\baselinestretch}{1.5}
\usefont{U}{cmss}{bx}{n}
\bfseries

% Taille normale : commenter le reste !
%Taille Arnaud
%\fontsize{19}{19}\selectfont

% Taille Barbara
%\fontsize{21}{22}\selectfont

%Taille François
%\fontsize{25}{30}\selectfont

%Taille Pascal
%\fontsize{25}{30}\selectfont

%Taille Laura
%\fontsize{30}{35}\selectfont


%\myfont
%\usefont{U}{cmss}{bx}{n}

%\Huge
%\addtolength{\parskip}{\baselineskip}
}


% \usepackage{hyperref}
% \hypersetup{colorlinks=true, linkcolor=blue, urlcolor=blue,
% pdftitle={Exo7 - Exercices de mathématiques}, pdfauthor={Exo7}}


%section
% \usepackage{sectsty}
% \allsectionsfont{\bf}
%\sectionfont{\color{Tomato3}\upshape\selectfont}
%\subsectionfont{\color{Tomato4}\upshape\selectfont}

%----- Ensembles : entiers, reels, complexes -----
\newcommand{\Nn}{\mathbb{N}} \newcommand{\N}{\mathbb{N}}
\newcommand{\Zz}{\mathbb{Z}} \newcommand{\Z}{\mathbb{Z}}
\newcommand{\Qq}{\mathbb{Q}} \newcommand{\Q}{\mathbb{Q}}
\newcommand{\Rr}{\mathbb{R}} \newcommand{\R}{\mathbb{R}}
\newcommand{\Cc}{\mathbb{C}} 
\newcommand{\Kk}{\mathbb{K}} \newcommand{\K}{\mathbb{K}}

%----- Modifications de symboles -----
\renewcommand{\epsilon}{\varepsilon}
\renewcommand{\Re}{\mathop{\text{Re}}\nolimits}
\renewcommand{\Im}{\mathop{\text{Im}}\nolimits}
%\newcommand{\llbracket}{\left[\kern-0.15em\left[}
%\newcommand{\rrbracket}{\right]\kern-0.15em\right]}

\renewcommand{\ge}{\geqslant}
\renewcommand{\geq}{\geqslant}
\renewcommand{\le}{\leqslant}
\renewcommand{\leq}{\leqslant}

%----- Fonctions usuelles -----
\newcommand{\ch}{\mathop{\mathrm{ch}}\nolimits}
\newcommand{\sh}{\mathop{\mathrm{sh}}\nolimits}
\renewcommand{\tanh}{\mathop{\mathrm{th}}\nolimits}
\newcommand{\cotan}{\mathop{\mathrm{cotan}}\nolimits}
\newcommand{\Arcsin}{\mathop{\mathrm{Arcsin}}\nolimits}
\newcommand{\Arccos}{\mathop{\mathrm{Arccos}}\nolimits}
\newcommand{\Arctan}{\mathop{\mathrm{Arctan}}\nolimits}
\newcommand{\Argsh}{\mathop{\mathrm{Argsh}}\nolimits}
\newcommand{\Argch}{\mathop{\mathrm{Argch}}\nolimits}
\newcommand{\Argth}{\mathop{\mathrm{Argth}}\nolimits}
\newcommand{\pgcd}{\mathop{\mathrm{pgcd}}\nolimits} 

\newcommand{\Card}{\mathop{\text{Card}}\nolimits}
\newcommand{\Ker}{\mathop{\text{Ker}}\nolimits}
\newcommand{\id}{\mathop{\text{id}}\nolimits}
\newcommand{\ii}{\mathrm{i}}
\newcommand{\dd}{\mathrm{d}}
\newcommand{\Vect}{\mathop{\text{Vect}}\nolimits}
\newcommand{\Mat}{\mathop{\mathrm{Mat}}\nolimits}
\newcommand{\rg}{\mathop{\text{rg}}\nolimits}
\newcommand{\tr}{\mathop{\text{tr}}\nolimits}
\newcommand{\ppcm}{\mathop{\text{ppcm}}\nolimits}

%----- Structure des exercices ------

\newtheoremstyle{styleexo}% name
{2ex}% Space above
{3ex}% Space below
{}% Body font
{}% Indent amount 1
{\bfseries} % Theorem head font
{}% Punctuation after theorem head
{\newline}% Space after theorem head 2
{}% Theorem head spec (can be left empty, meaning ‘normal’)

%\theoremstyle{styleexo}
\newtheorem{exo}{Exercice}
\newtheorem{ind}{Indications}
\newtheorem{cor}{Correction}


\newcommand{\exercice}[1]{} \newcommand{\finexercice}{}
%\newcommand{\exercice}[1]{{\tiny\texttt{#1}}\vspace{-2ex}} % pour afficher le numero absolu, l'auteur...
\newcommand{\enonce}{\begin{exo}} \newcommand{\finenonce}{\end{exo}}
\newcommand{\indication}{\begin{ind}} \newcommand{\finindication}{\end{ind}}
\newcommand{\correction}{\begin{cor}} \newcommand{\fincorrection}{\end{cor}}

\newcommand{\noindication}{\stepcounter{ind}}
\newcommand{\nocorrection}{\stepcounter{cor}}

\newcommand{\fiche}[1]{} \newcommand{\finfiche}{}
\newcommand{\titre}[1]{\centerline{\large \bf #1}}
\newcommand{\addcommand}[1]{}
\newcommand{\video}[1]{}

% Marge
\newcommand{\mymargin}[1]{\marginpar{{\small #1}}}



%----- Presentation ------
\setlength{\parindent}{0cm}

%\newcommand{\ExoSept}{\href{http://exo7.emath.fr}{\textbf{\textsf{Exo7}}}}

\definecolor{myred}{rgb}{0.93,0.26,0}
\definecolor{myorange}{rgb}{0.97,0.58,0}
\definecolor{myyellow}{rgb}{1,0.86,0}

\newcommand{\LogoExoSept}[1]{  % input : echelle
{\usefont{U}{cmss}{bx}{n}
\begin{tikzpicture}[scale=0.1*#1,transform shape]
  \fill[color=myorange] (0,0)--(4,0)--(4,-4)--(0,-4)--cycle;
  \fill[color=myred] (0,0)--(0,3)--(-3,3)--(-3,0)--cycle;
  \fill[color=myyellow] (4,0)--(7,4)--(3,7)--(0,3)--cycle;
  \node[scale=5] at (3.5,3.5) {Exo7};
\end{tikzpicture}}
}



\theoremstyle{definition}
%\newtheorem{proposition}{Proposition}
%\newtheorem{exemple}{Exemple}
%\newtheorem{theoreme}{Théorème}
\newtheorem{lemme}{Lemme}
\newtheorem{corollaire}{Corollaire}
%\newtheorem*{remarque*}{Remarque}
%\newtheorem*{miniexercice}{Mini-exercices}
%\newtheorem{definition}{Définition}




%definition d'un terme
\newcommand{\defi}[1]{{\color{myorange}\textbf{\emph{#1}}}}
\newcommand{\evidence}[1]{{\color{blue}\textbf{\emph{#1}}}}



 %----- Commandes divers ------

\newcommand{\codeinline}[1]{\texttt{#1}}

%%%%%%%%%%%%%%%%%%%%%%%%%%%%%%%%%%%%%%%%%%%%%%%%%%%%%%%%%%%%%
%%%%%%%%%%%%%%%%%%%%%%%%%%%%%%%%%%%%%%%%%%%%%%%%%%%%%%%%%%%%%


\newcommand{\deter}{déter\-mi\-nant\ }
\newcommand{\deters}{déter\-mi\-nants\ }

\begin{document}

\debuttexte


%%%%%%%%%%%%%%%%%%%%%%%%%%%%%%%%%%%%%%%%%%%%%%%%%%%%%%%%%%%
\diapo

Voici la  première leçon du chapitre consacré aux \deters.  Le \deter est un nombre
que l'on associe à une matrice, 
il permet de savoir si cette matrice est inversible ou pas, et de façon plus générale, 
joue un rôle important dans le calcul matriciel et la résolution de systèmes linéaires.

\change
Dans cette leçon, nous commençons l'étude des \deters 
en donnant l'expression du \deter d'une matrice en petites dimensions.

\change
Nous verrons que le \deter d'une matrice $2\times2$

\change
ou $3\times3$ a une expression explicite simple, et facile à calculer.

\change  

On verra que l'on peut aussi définir le \deter de $n$ vecteurs de $\Rr^n$. 
Nous donnerons une interprétation géo\-mé\-trique du \deter en dimension 2 
comme aire d'un paral\-lé\-logramme et en dimension 3 comme volume d'un paral\-lélé\-pipède.


%%%%%%%%%%%%%%%%%%%%%%%%%%%%%%%%%%%%%%%%%%%%%%%%%%%%%%%%%%%
\diapo
Dans tout ce qui suit, nous considérons des matrices à coefficients 
dans un corps $\Kk$, les principaux exemples étant $\Kk=\Rr$ ou $\Kk=\Cc$.

\change
En dimension $2$, le \deter est très simple à calculer:
$$\det \begin{pmatrix}a&b\\c&d\end{pmatrix} = ad-bc.$$

\change
C'est donc le produit des éléments sur la diagonale principale (en bleu) $a \times d$, 
moins le produit des éléments sur l'autre diagonale (en orange) $b \times c$.


%%%%%%%%%%%%%%%%%%%%%%%%%%%%%%%%%%%%%%%%%%%%%%%%%%%%%%%%%%%
\diapo
Considérons à présent une matrice $A$ de taille $3\times3$ à coefficients dans le corps $\Kk$.

\change
Voici alors la formule pour le \deter.

\change
Il existe un moyen facile de retenir cette formule, c'est la \defi{règle de Sarrus}:
on commence par recopier les deux premières colonnes à droite de la matrice (colonnes grisées), 

\change
puis on additionne les produits de trois termes en les regroupant selon la direction 
de la diagonale descendante (en bleu)

\change
et on soustrait ensuite les produits de trois termes 
regroupés selon la direction de la diagonale montante (en orange).

\change
On se souvient alors facilement de cette formule.

\change
Attention! cette méthode ne s'applique pas pour les matrices de taille supérieure à $3$. 
Nous verrons d'autres métho\-des qui s'appliquent aux matrices carrées de toutes tailles 
et donc aussi aux matrices $3\times3$.

%%%%%%%%%%%%%%%%%%%%%%%%%%%%%%%%%%%%%%%%%%%%%%%%%%%%%%%%%%%
\diapo

Passons tout de suite à un exemple et calculons le \deter de la matrice $A$ suivante.

\change
Par la règle de Sarrus, le \deter de $A$ se calcule ainsi : on recopie les deux premières colonnes à droite de $A$ 
% geste
et on considère en bleu les diagonales de la nouvelle matrice, et en orange les anti-diagonales.

\change
Le \deter est obtenu en faisant le produit $2\times (-1) \times 1$

\change
plus %le produit 
$1\times 3 \times  3$

\change
plus %le produit 
$0\times 1 \times 2$

\change
puis de même pour les diagonales oranges, mais avec un signe moins: 
moins $3\times (-1) \times 0$, et ainsi de suite...

\change
Au final, on obtient que $\det A = -6.$


%%%%%%%%%%%%%%%%%%%%%%%%%%%%%%%%%%%%%%%%%%%%%%%%%%%%%%%%%%%
\diapo

Passons à présent à l'interprétation géométrique du \deter :  \\
on va voir qu'en dimension $2$, les \deters correspondent à des aires, \\
et en dimension $3$ à des volumes.

\change
Donnons nous deux vecteurs du plan $\Rr^2$, 
$v_1$ de coordonnées $\left(\begin{smallmatrix}a \\[5pt]c\end{smallmatrix}\right)$ et 
$v_2$ de coordonnées $\left(\begin{smallmatrix}b \\[5pt] d\end{smallmatrix}\right)$.


\change
Ces deux vecteurs $v_1,v_2$ déterminent un parallélogramme.

\change
On a alors le résultat suivant : l'aire du parallélogramme est 
donnée par la valeur absolue du \deter de la matrice formée 
des vecteurs $v_1,v_2$.

Autrement dit, l'aire  est la valeur absolue du \deter de la matrice
$2\times 2$, $a, b, c, d$. et vaut donc $|ad-bc|$.


%%%%%%%%%%%%%%%%%%%%%%%%%%%%%%%%%%%%%%%%%%%%%%%%%%%%%%%%%%%
\diapo
De manière similaire, trois vecteurs de l'espace $\Rr^3$:
$v_1, v_2, v_3$

\change
définissent un parallélépipède.

\change
\`A partir de ces trois vecteurs on définit, en juxtaposant les colonnes, 
une matrice.


%%%%%%%%%%%%%%%%%%%%%%%%%%%%%%%%%%%%%%%%%%%%%%%%%%%%%%%%%%%
\diapo

Comme dans $\Rr^2$, le volume du parallélépipède est donné par la valeur absolue du \deter
de cette matrice.

\change
On prendra comme unité d'aire dans $\Rr^2$ l'aire du carré unité dont 
les côtés sont les vecteurs de la base canonique 
$\left(\left(\begin{smallmatrix} 1 \\[5pt]  0 \end{smallmatrix} \right),
\left(\begin{smallmatrix} 0 \\[5pt]  1 \end{smallmatrix} \right)\right)$, 

\change
et comme unité de volume dans $\Rr^3$, 
le volume du cube unité.

%%%%%%%%%%%%%%%%%%%%%%%%%%%%%%%%%%%%%%%%%%%%%%%%%%%%%%%%%%%
\diapo

Démontrons les résultats précédents. On traite le cas de la dimension $2$. 
Le cas tridimensionnel se traite de façon analogue.

\change
Remarquons tout d'abord que le résultat est vrai si 
$v_1=\left(\begin{smallmatrix}a \\[5pt] 0\end{smallmatrix}\right)$ est un vecteur horizontal
 et $v_2=\left(\begin{smallmatrix}0\\[5pt] d\end{smallmatrix}\right)$ est un vecteur vertical.

\change
En effet, dans ce cas on a affaire à un rectangle de côtés $|a|$ et $|d|$, donc d'aire
$|a \times d|$, 

\change
tandis que le \deter de la matrice 
$\begin{pmatrix}
	a&0\\0&d
\end{pmatrix}$
vaut $ad$.

\change
Remarquons aussi que si les vecteurs $v_1$ et $v_2$ sont colinéaires alors
le parallélogramme est aplati, donc d'aire nulle.

\change
On calcule facilement
que lorsque deux vecteurs sont colinéaires, leur \deter est nul lui aussi.

\change
Dans la suite on supposera donc que les vecteurs ne sont pas colinéaires.

%%%%%%%%%%%%%%%%%%%%%%%%%%%%%%%%%%%%%%%%%%%%%%%%%%%%%%%%%%%
\diapo
Notons $v_1= \left(\begin{smallmatrix}a \\[5pt] c\end{smallmatrix}\right)$ et 
$v_2= \left(\begin{smallmatrix}b \\[5pt] d\end{smallmatrix}\right)$ deux vecteurs 
quelconques mais non colinéaires.

\change
Si $a\neq0$, alors $v'_2=v_2-\frac{b}{a}v_1$ est un vecteur vertical:
$v_2'=\left(\begin{smallmatrix}0 \\[5pt] d-\frac{b}{a}c\end{smallmatrix}\right)$.

\change
Considérons l'opération consistant à remplacer $v_2$ par $v_2'$.

\change
Tout d'abord cette opération ne change pas l'aire du parallélogramme : 
en effet, c'est comme si on avait coupé le triangle vert
et collé à la place le triangle bleu.

\change
Cette opération ne change pas non plus le \deter car on a toujours:
$$\det (v_1,v_2')= 
\det \begin{pmatrix}
a & 0 \\ b & d-\frac{b}{a}c
\end{pmatrix}
=ad-bc=\det(v_1,v_2)\, .$$


%%%%%%%%%%%%%%%%%%%%%%%%%%%%%%%%%%%%%%%%%%%%%%%%%%%%%%%%%%%
\diapo

De la m\^eme manière, on pose maintenant
$v'_1= \left(\begin{smallmatrix}a \\[5pt] 0\end{smallmatrix}\right)$: c'est un vecteur horizontal. 

\change
Encore une fois, remplacer $v_1$ par $v_1'$:

\change
ne change pas l'aire des parallélogrammes, car l'aire initiale en bleue est devenue l'aire en rose.

\change
Cela ne change pas non plus le \deter car 
$$\det(v_1',v_2')=
\det \begin{pmatrix}
a&0\\
0&d-\frac{b}{a}c
\end{pmatrix}
=ad-bc=\det(v_1,v_2)\, .$$

\change
On s'est donc ramené au premier cas d'un rectangle 
aux côtés parallèle aux axes, pour lequel le résultat a déjà été montré.
La démonstration est terminée !

%%%%%%%%%%%%%%%%%%%%%%%%%%%%%%%%%%%%%%%%%%%%%%%%%%%%%%%%%%%
\diapo

Entraînez-vous avec ces exercices pour vérifier que vous avez bien compris le cours ;)

\end{document}

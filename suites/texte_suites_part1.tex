
%%%%%%%%%%%%%%%%%% PREAMBULE %%%%%%%%%%%%%%%%%%


\documentclass[12pt]{article}

\usepackage{amsfonts,amsmath,amssymb,amsthm}
\usepackage[utf8]{inputenc}
\usepackage[T1]{fontenc}
\usepackage[francais]{babel}


% packages
\usepackage{amsfonts,amsmath,amssymb,amsthm}
\usepackage[utf8]{inputenc}
\usepackage[T1]{fontenc}
%\usepackage{lmodern}

\usepackage[francais]{babel}
\usepackage{fancybox}
\usepackage{graphicx}

\usepackage{float}

%\usepackage[usenames, x11names]{xcolor}
\usepackage{tikz}
\usepackage{datetime}

\usepackage{mathptmx}
%\usepackage{fouriernc}
%\usepackage{newcent}
\usepackage[mathcal,mathbf]{euler}

%\usepackage{palatino}
%\usepackage{newcent}


% Commande spéciale prompteur

%\usepackage{mathptmx}
%\usepackage[mathcal,mathbf]{euler}
%\usepackage{mathpple,multido}

\usepackage[a4paper]{geometry}
\geometry{top=2cm, bottom=2cm, left=1cm, right=1cm, marginparsep=1cm}

\newcommand{\change}{{\color{red}\rule{\textwidth}{1mm}\\}}

\newcounter{mydiapo}

\newcommand{\diapo}{\newpage
\hfill {\normalsize  Diapo \themydiapo \quad \texttt{[\jobname]}} \\
\stepcounter{mydiapo}}


%%%%%%% COULEURS %%%%%%%%%%

% Pour blanc sur noir :
%\pagecolor[rgb]{0.5,0.5,0.5}
% \pagecolor[rgb]{0,0,0}
% \color[rgb]{1,1,1}



%\DeclareFixedFont{\myfont}{U}{cmss}{bx}{n}{18pt}
\newcommand{\debuttexte}{
%%%%%%%%%%%%% FONTES %%%%%%%%%%%%%
\renewcommand{\baselinestretch}{1.5}
\usefont{U}{cmss}{bx}{n}
\bfseries

% Taille normale : commenter le reste !
%Taille Arnaud
%\fontsize{19}{19}\selectfont

% Taille Barbara
%\fontsize{21}{22}\selectfont

%Taille François
%\fontsize{25}{30}\selectfont

%Taille Pascal
%\fontsize{25}{30}\selectfont

%Taille Laura
%\fontsize{30}{35}\selectfont


%\myfont
%\usefont{U}{cmss}{bx}{n}

%\Huge
%\addtolength{\parskip}{\baselineskip}
}


% \usepackage{hyperref}
% \hypersetup{colorlinks=true, linkcolor=blue, urlcolor=blue,
% pdftitle={Exo7 - Exercices de mathématiques}, pdfauthor={Exo7}}


%section
% \usepackage{sectsty}
% \allsectionsfont{\bf}
%\sectionfont{\color{Tomato3}\upshape\selectfont}
%\subsectionfont{\color{Tomato4}\upshape\selectfont}

%----- Ensembles : entiers, reels, complexes -----
\newcommand{\Nn}{\mathbb{N}} \newcommand{\N}{\mathbb{N}}
\newcommand{\Zz}{\mathbb{Z}} \newcommand{\Z}{\mathbb{Z}}
\newcommand{\Qq}{\mathbb{Q}} \newcommand{\Q}{\mathbb{Q}}
\newcommand{\Rr}{\mathbb{R}} \newcommand{\R}{\mathbb{R}}
\newcommand{\Cc}{\mathbb{C}} 
\newcommand{\Kk}{\mathbb{K}} \newcommand{\K}{\mathbb{K}}

%----- Modifications de symboles -----
\renewcommand{\epsilon}{\varepsilon}
\renewcommand{\Re}{\mathop{\text{Re}}\nolimits}
\renewcommand{\Im}{\mathop{\text{Im}}\nolimits}
%\newcommand{\llbracket}{\left[\kern-0.15em\left[}
%\newcommand{\rrbracket}{\right]\kern-0.15em\right]}

\renewcommand{\ge}{\geqslant}
\renewcommand{\geq}{\geqslant}
\renewcommand{\le}{\leqslant}
\renewcommand{\leq}{\leqslant}

%----- Fonctions usuelles -----
\newcommand{\ch}{\mathop{\mathrm{ch}}\nolimits}
\newcommand{\sh}{\mathop{\mathrm{sh}}\nolimits}
\renewcommand{\tanh}{\mathop{\mathrm{th}}\nolimits}
\newcommand{\cotan}{\mathop{\mathrm{cotan}}\nolimits}
\newcommand{\Arcsin}{\mathop{\mathrm{Arcsin}}\nolimits}
\newcommand{\Arccos}{\mathop{\mathrm{Arccos}}\nolimits}
\newcommand{\Arctan}{\mathop{\mathrm{Arctan}}\nolimits}
\newcommand{\Argsh}{\mathop{\mathrm{Argsh}}\nolimits}
\newcommand{\Argch}{\mathop{\mathrm{Argch}}\nolimits}
\newcommand{\Argth}{\mathop{\mathrm{Argth}}\nolimits}
\newcommand{\pgcd}{\mathop{\mathrm{pgcd}}\nolimits} 

\newcommand{\Card}{\mathop{\text{Card}}\nolimits}
\newcommand{\Ker}{\mathop{\text{Ker}}\nolimits}
\newcommand{\id}{\mathop{\text{id}}\nolimits}
\newcommand{\ii}{\mathrm{i}}
\newcommand{\dd}{\mathrm{d}}
\newcommand{\Vect}{\mathop{\text{Vect}}\nolimits}
\newcommand{\Mat}{\mathop{\mathrm{Mat}}\nolimits}
\newcommand{\rg}{\mathop{\text{rg}}\nolimits}
\newcommand{\tr}{\mathop{\text{tr}}\nolimits}
\newcommand{\ppcm}{\mathop{\text{ppcm}}\nolimits}

%----- Structure des exercices ------

\newtheoremstyle{styleexo}% name
{2ex}% Space above
{3ex}% Space below
{}% Body font
{}% Indent amount 1
{\bfseries} % Theorem head font
{}% Punctuation after theorem head
{\newline}% Space after theorem head 2
{}% Theorem head spec (can be left empty, meaning ‘normal’)

%\theoremstyle{styleexo}
\newtheorem{exo}{Exercice}
\newtheorem{ind}{Indications}
\newtheorem{cor}{Correction}


\newcommand{\exercice}[1]{} \newcommand{\finexercice}{}
%\newcommand{\exercice}[1]{{\tiny\texttt{#1}}\vspace{-2ex}} % pour afficher le numero absolu, l'auteur...
\newcommand{\enonce}{\begin{exo}} \newcommand{\finenonce}{\end{exo}}
\newcommand{\indication}{\begin{ind}} \newcommand{\finindication}{\end{ind}}
\newcommand{\correction}{\begin{cor}} \newcommand{\fincorrection}{\end{cor}}

\newcommand{\noindication}{\stepcounter{ind}}
\newcommand{\nocorrection}{\stepcounter{cor}}

\newcommand{\fiche}[1]{} \newcommand{\finfiche}{}
\newcommand{\titre}[1]{\centerline{\large \bf #1}}
\newcommand{\addcommand}[1]{}
\newcommand{\video}[1]{}

% Marge
\newcommand{\mymargin}[1]{\marginpar{{\small #1}}}



%----- Presentation ------
\setlength{\parindent}{0cm}

%\newcommand{\ExoSept}{\href{http://exo7.emath.fr}{\textbf{\textsf{Exo7}}}}

\definecolor{myred}{rgb}{0.93,0.26,0}
\definecolor{myorange}{rgb}{0.97,0.58,0}
\definecolor{myyellow}{rgb}{1,0.86,0}

\newcommand{\LogoExoSept}[1]{  % input : echelle
{\usefont{U}{cmss}{bx}{n}
\begin{tikzpicture}[scale=0.1*#1,transform shape]
  \fill[color=myorange] (0,0)--(4,0)--(4,-4)--(0,-4)--cycle;
  \fill[color=myred] (0,0)--(0,3)--(-3,3)--(-3,0)--cycle;
  \fill[color=myyellow] (4,0)--(7,4)--(3,7)--(0,3)--cycle;
  \node[scale=5] at (3.5,3.5) {Exo7};
\end{tikzpicture}}
}



\theoremstyle{definition}
%\newtheorem{proposition}{Proposition}
%\newtheorem{exemple}{Exemple}
%\newtheorem{theoreme}{Théorème}
\newtheorem{lemme}{Lemme}
\newtheorem{corollaire}{Corollaire}
%\newtheorem*{remarque*}{Remarque}
%\newtheorem*{miniexercice}{Mini-exercices}
%\newtheorem{definition}{Définition}




%definition d'un terme
\newcommand{\defi}[1]{{\color{myorange}\textbf{\emph{#1}}}}
\newcommand{\evidence}[1]{{\color{blue}\textbf{\emph{#1}}}}



 %----- Commandes divers ------

\newcommand{\codeinline}[1]{\texttt{#1}}

%%%%%%%%%%%%%%%%%%%%%%%%%%%%%%%%%%%%%%%%%%%%%%%%%%%%%%%%%%%%%
%%%%%%%%%%%%%%%%%%%%%%%%%%%%%%%%%%%%%%%%%%%%%%%%%%%%%%%%%%%%%



\begin{document}

\debuttexte

%%%%%%%%%%%%%%%%%%%%%%%%%%%%%%%%%%%%%%%%%%%%%%%%%%%%%%%%%%
\diapo

\change

Le plan de cette première partie sur les suites est le suivant :

\change

Nous commençons par la définition de suite et donnons quelques exemples simples.

\change

Puis nous définirons ce qu'est une suite majorée et une suite minorée.

\change

Enfin nous terminons avec la notion de suite croissante et celle de suite décroissante.


%%%%%%%%%%%%%%%%%%%%%%%%%%%%%%%%%%%%%%%%%%%%%%%%%%%%%%%%%%
\diapo

Commençons avec un exemple de suite venant de la vie quotidienne.

On note grand $S$ une somme d'argent placée à un taux annuel de $10\%$.

\change

En notant $S_n$ la somme obtenue après un nombre $n$ d'années, on a

\change

$S_0=S \quad    S_1=S\times 1,1\quad \ldots \quad   S_n=S\times (1,1)^n \;\;.$

\change 

Cette formule permet par exemple de calculer la somme obtenue au bout de dix ans :

$S_{10}=S_n=S\times (1,1)^{10}\thickapprox S\times 2,59$


%%%%%%%%%%%%%%%%%%%%%%%%%%%%%%%%%%%%%%%%%%%%%%%%%%%%%%%%%%
\diapo

Passons aux premières définitions de ce chapitre.

\change

Une \emph{suite} est une application $u$ de l'ensemble des entiers naturels dans l'ensemble des réels.

\change

On note $u$ indice $n$ plutôt que $u$ parenthèses $n$ et on appelle ce nombre réel le terme général de la suite ou 
le $n$-ième terme de la suite.

\change

Passons aux exemples : on commence par la suite des racines carrées des entiers naturels :

$u_0=\sqrt{0}=0$,
$u_1=\sqrt{1}=1$,
$u_2=\sqrt{2}$,...

\change

La suite des puissances de $-1$ qui alterne $+1$, $-1$, $+1$, $-1$ 
suivant la parité de l'indice $n$ :

$(-1)^n = +1$ si $n$ est pair, alors que $(-1)^n = -1$ si $n$ est impair.

\change 

La suite de Fibonacci définie  par l'initialisation $F_0=1$, $F_1=1$ 

\change

et la relation de récurrence $F_{n+2}=F_{n+1}+F_n$ . 

\change

Voici les premiers termes : $1,1,2,3,5,8$... 
Chaque terme est la somme des deux précédents.


\change

Pour finir,  la suite des inverses des carrés des entiers naturels, 
dont voici les premiers termes :
$u_1=1$,
$u_2=1/4$,
$u_3=1/9$,...

Notez que cette suite n'est définie que pour les rangs $n \ge 1$.



%%%%%%%%%%%%%%%%%%%%%%%%%%%%%%%%%%%%%%%%%%%%%%%%%%%%%%%%%%
\diapo

[grand $M$, petit $m$]

La suite $(u_n)_{n\in \Nn}$ est dite \emph{majorée} s'il existe un réel 
grand $M$ plus grand que tous les termes de la suite, 
c'est-à-dire tel pour tout $n$, on a $u_n\leq M$.

\change

Sur cette figure il existe bien un réel grand $M$ qui majore *tous* les termes de la suite.

\change


De manière similaire, la suite $(u_n)_{n\in \Nn}$ est dite \emph{minorée} 
s'il existe un réel petit $m$ plus petit que tous les termes de la suite, 
c'est-à-dire tel pour tout $n$, on a $u_n\geq m$.

\change

Ici on trouvé un réel petit $m$, qui est inférieur à *tous* les termes.

\change


Enfin la suite $(u_n)_{n\in \Nn}$ est dite \emph{bornée} si  elle est à la fois majorée 
et minorée, 

c'est équivalent à ce qu'il existe un réel $M$ 
tel que pour tout $n$, on a $|u_n| \leq M$.


%%%%%%%%%%%%%%%%%%%%%%%%%%%%%%%%%%%%%%%%%%%%%%%%%%%%%%%%%%
\diapo

La suite $(u_n)_{n\in \Nn}$ est \emph{croissante} si pour tout  $n$, on a $u_{n+1} \ge u_n $ .

\change

Autrement dit chaque terme est plus grand que le précédent.

\change

De même $(u_n)_{n\in \Nn}$ est \emph{strictement} croissante si pour tout $n$, on a $\quad u_{n+1} > u_n$.

Cette suite est croissante, mais pas strictement croissante, car ici ces deux termes sont égaux.

\change

Les notions analogues relatives à la \emph{décroissance} (suite décroissante, strictement décroissante) s'obtiennent en inversant le sens des inégalités.

\change

Enfin, $(u_n)_{n\in \Nn}$ est une suite \emph{monotone} si elle est croissante ou bien décroissante.



%%%%%%%%%%%%%%%%%%%%%%%%%%%%%%%%%%%%%%%%%%%%%%%%%%%%%%%%%%
\diapo

La suite $(S_n)$ de l'introduction est strictement croissante car 
$S_{n+1}/S_n=1,1$ donc $S_{n+1} > S_n$.

\change


Attention une suite peut être ni croissante ni décroissante, voici un exemple : 

on considère la suite de terme général $u_n=(-1)^n/n$ pour $n\geq 1$ :

elle n'est ni croissante, ni décroissante.

\change 

Cette suite est majorée par $1/2$ (borne qui est en plus atteinte 
pour le terme $u_2$).

\change

La suite est aussi minorée par $-1$ (en plus cette borne est atteinte pour le terme $u_1$).

Cette suite est majorée et minorée, donc bornée.

\change

Autre exemple la suite des $\frac1n$ qui est strictement décroissante. 

\change

Elle est majorée par $1$ 

\change


elle est minorée par $0$ mais notez que cette valeur n'est atteinte pour aucun terme de la suite.

%%%%%%%%%%%%%%%%%%%%%%%%%%%%%%%%%%%%%%%%%%%%%%%%%%%%%%%%%%
\diapo

Pour vous aidez à assimiler toutes ces définitions, il est temps de 
vous exercer à l'aide de ces énoncés.



\end{document}

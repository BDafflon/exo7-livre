
%%%%%%%%%%%%%%%%%% PREAMBULE %%%%%%%%%%%%%%%%%%


\documentclass[12pt]{article}

\usepackage{amsfonts,amsmath,amssymb,amsthm}
\usepackage[utf8]{inputenc}
\usepackage[T1]{fontenc}
\usepackage[francais]{babel}


% packages
\usepackage{amsfonts,amsmath,amssymb,amsthm}
\usepackage[utf8]{inputenc}
\usepackage[T1]{fontenc}
%\usepackage{lmodern}

\usepackage[francais]{babel}
\usepackage{fancybox}
\usepackage{graphicx}

\usepackage{float}

%\usepackage[usenames, x11names]{xcolor}
\usepackage{tikz}
\usepackage{datetime}

\usepackage{mathptmx}
%\usepackage{fouriernc}
%\usepackage{newcent}
\usepackage[mathcal,mathbf]{euler}

%\usepackage{palatino}
%\usepackage{newcent}


% Commande spéciale prompteur

%\usepackage{mathptmx}
%\usepackage[mathcal,mathbf]{euler}
%\usepackage{mathpple,multido}

\usepackage[a4paper]{geometry}
\geometry{top=2cm, bottom=2cm, left=1cm, right=1cm, marginparsep=1cm}

\newcommand{\change}{{\color{red}\rule{\textwidth}{1mm}\\}}

\newcounter{mydiapo}

\newcommand{\diapo}{\newpage
\hfill {\normalsize  Diapo \themydiapo \quad \texttt{[\jobname]}} \\
\stepcounter{mydiapo}}


%%%%%%% COULEURS %%%%%%%%%%

% Pour blanc sur noir :
%\pagecolor[rgb]{0.5,0.5,0.5}
% \pagecolor[rgb]{0,0,0}
% \color[rgb]{1,1,1}



%\DeclareFixedFont{\myfont}{U}{cmss}{bx}{n}{18pt}
\newcommand{\debuttexte}{
%%%%%%%%%%%%% FONTES %%%%%%%%%%%%%
\renewcommand{\baselinestretch}{1.5}
\usefont{U}{cmss}{bx}{n}
\bfseries

% Taille normale : commenter le reste !
%Taille Arnaud
%\fontsize{19}{19}\selectfont

% Taille Barbara
%\fontsize{21}{22}\selectfont

%Taille François
%\fontsize{25}{30}\selectfont

%Taille Pascal
%\fontsize{25}{30}\selectfont

%Taille Laura
%\fontsize{30}{35}\selectfont


%\myfont
%\usefont{U}{cmss}{bx}{n}

%\Huge
%\addtolength{\parskip}{\baselineskip}
}


% \usepackage{hyperref}
% \hypersetup{colorlinks=true, linkcolor=blue, urlcolor=blue,
% pdftitle={Exo7 - Exercices de mathématiques}, pdfauthor={Exo7}}


%section
% \usepackage{sectsty}
% \allsectionsfont{\bf}
%\sectionfont{\color{Tomato3}\upshape\selectfont}
%\subsectionfont{\color{Tomato4}\upshape\selectfont}

%----- Ensembles : entiers, reels, complexes -----
\newcommand{\Nn}{\mathbb{N}} \newcommand{\N}{\mathbb{N}}
\newcommand{\Zz}{\mathbb{Z}} \newcommand{\Z}{\mathbb{Z}}
\newcommand{\Qq}{\mathbb{Q}} \newcommand{\Q}{\mathbb{Q}}
\newcommand{\Rr}{\mathbb{R}} \newcommand{\R}{\mathbb{R}}
\newcommand{\Cc}{\mathbb{C}} 
\newcommand{\Kk}{\mathbb{K}} \newcommand{\K}{\mathbb{K}}

%----- Modifications de symboles -----
\renewcommand{\epsilon}{\varepsilon}
\renewcommand{\Re}{\mathop{\text{Re}}\nolimits}
\renewcommand{\Im}{\mathop{\text{Im}}\nolimits}
%\newcommand{\llbracket}{\left[\kern-0.15em\left[}
%\newcommand{\rrbracket}{\right]\kern-0.15em\right]}

\renewcommand{\ge}{\geqslant}
\renewcommand{\geq}{\geqslant}
\renewcommand{\le}{\leqslant}
\renewcommand{\leq}{\leqslant}

%----- Fonctions usuelles -----
\newcommand{\ch}{\mathop{\mathrm{ch}}\nolimits}
\newcommand{\sh}{\mathop{\mathrm{sh}}\nolimits}
\renewcommand{\tanh}{\mathop{\mathrm{th}}\nolimits}
\newcommand{\cotan}{\mathop{\mathrm{cotan}}\nolimits}
\newcommand{\Arcsin}{\mathop{\mathrm{Arcsin}}\nolimits}
\newcommand{\Arccos}{\mathop{\mathrm{Arccos}}\nolimits}
\newcommand{\Arctan}{\mathop{\mathrm{Arctan}}\nolimits}
\newcommand{\Argsh}{\mathop{\mathrm{Argsh}}\nolimits}
\newcommand{\Argch}{\mathop{\mathrm{Argch}}\nolimits}
\newcommand{\Argth}{\mathop{\mathrm{Argth}}\nolimits}
\newcommand{\pgcd}{\mathop{\mathrm{pgcd}}\nolimits} 

\newcommand{\Card}{\mathop{\text{Card}}\nolimits}
\newcommand{\Ker}{\mathop{\text{Ker}}\nolimits}
\newcommand{\id}{\mathop{\text{id}}\nolimits}
\newcommand{\ii}{\mathrm{i}}
\newcommand{\dd}{\mathrm{d}}
\newcommand{\Vect}{\mathop{\text{Vect}}\nolimits}
\newcommand{\Mat}{\mathop{\mathrm{Mat}}\nolimits}
\newcommand{\rg}{\mathop{\text{rg}}\nolimits}
\newcommand{\tr}{\mathop{\text{tr}}\nolimits}
\newcommand{\ppcm}{\mathop{\text{ppcm}}\nolimits}

%----- Structure des exercices ------

\newtheoremstyle{styleexo}% name
{2ex}% Space above
{3ex}% Space below
{}% Body font
{}% Indent amount 1
{\bfseries} % Theorem head font
{}% Punctuation after theorem head
{\newline}% Space after theorem head 2
{}% Theorem head spec (can be left empty, meaning ‘normal’)

%\theoremstyle{styleexo}
\newtheorem{exo}{Exercice}
\newtheorem{ind}{Indications}
\newtheorem{cor}{Correction}


\newcommand{\exercice}[1]{} \newcommand{\finexercice}{}
%\newcommand{\exercice}[1]{{\tiny\texttt{#1}}\vspace{-2ex}} % pour afficher le numero absolu, l'auteur...
\newcommand{\enonce}{\begin{exo}} \newcommand{\finenonce}{\end{exo}}
\newcommand{\indication}{\begin{ind}} \newcommand{\finindication}{\end{ind}}
\newcommand{\correction}{\begin{cor}} \newcommand{\fincorrection}{\end{cor}}

\newcommand{\noindication}{\stepcounter{ind}}
\newcommand{\nocorrection}{\stepcounter{cor}}

\newcommand{\fiche}[1]{} \newcommand{\finfiche}{}
\newcommand{\titre}[1]{\centerline{\large \bf #1}}
\newcommand{\addcommand}[1]{}
\newcommand{\video}[1]{}

% Marge
\newcommand{\mymargin}[1]{\marginpar{{\small #1}}}



%----- Presentation ------
\setlength{\parindent}{0cm}

%\newcommand{\ExoSept}{\href{http://exo7.emath.fr}{\textbf{\textsf{Exo7}}}}

\definecolor{myred}{rgb}{0.93,0.26,0}
\definecolor{myorange}{rgb}{0.97,0.58,0}
\definecolor{myyellow}{rgb}{1,0.86,0}

\newcommand{\LogoExoSept}[1]{  % input : echelle
{\usefont{U}{cmss}{bx}{n}
\begin{tikzpicture}[scale=0.1*#1,transform shape]
  \fill[color=myorange] (0,0)--(4,0)--(4,-4)--(0,-4)--cycle;
  \fill[color=myred] (0,0)--(0,3)--(-3,3)--(-3,0)--cycle;
  \fill[color=myyellow] (4,0)--(7,4)--(3,7)--(0,3)--cycle;
  \node[scale=5] at (3.5,3.5) {Exo7};
\end{tikzpicture}}
}



\theoremstyle{definition}
%\newtheorem{proposition}{Proposition}
%\newtheorem{exemple}{Exemple}
%\newtheorem{theoreme}{Théorème}
\newtheorem{lemme}{Lemme}
\newtheorem{corollaire}{Corollaire}
%\newtheorem*{remarque*}{Remarque}
%\newtheorem*{miniexercice}{Mini-exercices}
%\newtheorem{definition}{Définition}




%definition d'un terme
\newcommand{\defi}[1]{{\color{myorange}\textbf{\emph{#1}}}}
\newcommand{\evidence}[1]{{\color{blue}\textbf{\emph{#1}}}}



 %----- Commandes divers ------

\newcommand{\codeinline}[1]{\texttt{#1}}

%%%%%%%%%%%%%%%%%%%%%%%%%%%%%%%%%%%%%%%%%%%%%%%%%%%%%%%%%%%%%
%%%%%%%%%%%%%%%%%%%%%%%%%%%%%%%%%%%%%%%%%%%%%%%%%%%%%%%%%%%%%



\begin{document}

\debuttexte


%%%%%%%%%%%%%%%%%%%%%%%%%%%%%%%%%%%%%%%%%%%%%%%%%%%%%%%%%%%
\diapo



\change

Nous allons pouvoir répondre aux problèmes trois problème grecs.

\change
Pour cela c'est le théorème de Wantzel, qui va être la clé de nos problèmes.


\change
On résout d'un seul coup la duplication du cube,

\change
la quadrature du cercle,

\change
et la trisection des angles !


Il aura fallu près de 2000 ans pour répondre à ces questions. 
Mais pensez que pour montrer qu'une construction
est possible, il suffit de l'exhiber (même si ce n'est pas toujours évident). 
Par contre pour montrer qu'une construction n'est pas possible, c'est complètement différent. 
Ce n'est pas parce que personne n'a réussi
une construction qu'elle n'est pas possible ! Ce sont 
les outils algébriques qui vont permettre de 
résoudre ces problèmes géométriques.



%%%%%%%%%%%%%%%%%%%%%%%%%%%%%%%%%%%%%%%%%%%%%%%%%%%%%%%%%%%
\diapo

Rappelons le théorème de Wantzel, ou plus précisément le corollaire du théorème de Wantzel.



Tout d'abord si un nombre réel $x$ est constructible alors nécessairement $x$ 
est un nombre algébrique.

\change
C'est-à-dire qu'il existe un polynôme $P$ à coefficients rationnels non nul tel que  $x$ soit une racine de $P$.

\change
Si $x$ est un nombre constructible on a en plus des informations sur le degré du polynôme $P$.

En effet : le degré algébrique de $x$ doit être de la forme $2^n$. 

\change
Cela signifie que le plus petit degré, parmi tous les degrés des polynômes
  $P$ vérifiant $P(x)=0$, est une puissance de $2$.
  
  
%%%%%%%%%%%%%%%%%%%%%%%%%%%%%%%%%%%%%%%%%%%%%%%%%%%%%%%%%%%
\diapo

Rappelons le premier problème : étant donné un cube, peut-on construire
un second cube dont le volume est le double du premier ?
Si le premier cube a ses côtés de longueur $a$, alors le second doit
avoir ses côtés de longueur $a\sqrt[3]{2}$.

Donc la question est équivalent à : 
Est-ce que que $\sqrt[3]{2}$ est un nombre constructible ?
 

%%%%%%%%%%%%%%%%%%%%%%%%%%%%%%%%%%%%%%%%%%%%%%%%%%%%%%%%%%%
\diapo

Résolution du premier problème : 
La duplication du cube ne peut pas s'effectuer à la règle et au compas.

\change
Cela découle du fait suivant :
$\sqrt[3]{2}$ n'est pas un nombre constructible.


\change
Voici la preuve : $\sqrt[3]{2}$ est bien sûr racine du polynôme $P(X)=X^3-2$, qui est un polynôme
à coefficient rationnels.

\change
Ce polynôme est unitaire et irréductible dans $\Qq[X]$,

\change
donc $\sqrt[3]{2}$ est un nombre algébrique, et en plus son degré algébrique est le degré de ce polynôme $P$ 
donc de degré $3$.

\change
Ainsi le degré algébrique de $\sqrt[3]{2}$ n'est pas une puissance de $2$.

\change
Et donc par le théorème de
Wantzel : $\sqrt[3]{2}$ n'est pas constructible.


%%%%%%%%%%%%%%%%%%%%%%%%%%%%%%%%%%%%%%%%%%%%%%%%%%%%%%%%%%%
\diapo

Passons au célèbre problème de la quadrature du cercle :
\'Etant donné un cercle, 
peut-on construire à la règle et au compas 
un carré de même aire ?

Cela revient à construire un segment de longueur $\sqrt{\pi} \times r$ à la règle et au compas, à
partir d'un segment de longueur $r$.

Dans notre langage la question devient : 
Est-ce que $\sqrt\pi$ est un nombre constructible ?



%%%%%%%%%%%%%%%%%%%%%%%%%%%%%%%%%%%%%%%%%%%%%%%%%%%%%%%%%%%
\diapo

La quadrature du cercle ne peut pas s'effectuer à la règle et au compas.


\change
C'est une reformulation du théorème suivant, 
du à Ferdinand von Lindemann (en 1882) :

Théorème : $\pi$ n'est pas un nombre algébrique 


Comme $\pi$ n'est pas un nombre algébrique 
alors $\pi$ n'est pas nombre constructible,

\change
ce qui implique que $\sqrt \pi$ n'est pas constructible non plus.

En effet si $\sqrt \pi$ était constructible alors $\pi = \sqrt \pi^2$ serait aussi constructible.


Nous ne ferons pas ici la démonstration que $\pi$ n'est pas un nombre algébrique,
mais c'est une démonstration qui n'est pas si difficile et abordable dès la première année.



%%%%%%%%%%%%%%%%%%%%%%%%%%%%%%%%%%%%%%%%%%%%%%%%%%%%%%%%%%%
\diapo

Dernier problème : 
Peut-on diviser un angle qui nous est donné en trois 
angles égaux à l'aide de la règle et du compas ?

La formulation algébrique est pour nous :
\'Etant donné un réel constructible de la forme $\cos \theta$, 
est-ce que le réel $\cos \frac\theta3$ est aussi constructible ?
  

%%%%%%%%%%%%%%%%%%%%%%%%%%%%%%%%%%%%%%%%%%%%%%%%%%%%%%%%%%%
\diapo

La trisection des angles ne peut pas s'effectuer à la règle et au compas.

\change
Plus précisément nous allons exhiber un angle que l'on ne peut pas couper en trois.

L'angle $\frac{\pi}{3}$ est constructible mais ne peut pas être coupé en trois car 
$\cos \frac \pi 9$ n'est pas un nombre constructible.

%%%%%%%%%%%%%%%%%%%%%%%%%%%%%%%%%%%%%%%%%%%%%%%%%%%%%%%%%%%
\diapo

Bien sûr l'angle $\frac\pi3$ est constructible car $\cos \frac\pi3 = \frac12$.
Donc on construit $1/2$ puis en traçant la verticale on a un angle de mesure $\pi/3$.

\change

$\cos \frac \pi 9$ est un nombre algébrique de degré 
algébrique $3$, 

Les calculs de ce degré algébrique font l'objet d'un exercice.

\change 
Donc le degré n'est pas une puissance de $2$ ainsi 
$\cos \frac \pi 9$ n'est pas constructible.


La trisection n'est donc pas possible en général, mais attention,
pour certains angles particuliers c'est possible : par exemple on peut facilement couper en $3$ les angles $\pi$ et $\frac \pi 2$ !

\end{document}

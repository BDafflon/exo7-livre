
%%%%%%%%%%%%%%%%%% PREAMBULE %%%%%%%%%%%%%%%%%%

\documentclass[aspectratio=169,utf8]{beamer}
%\documentclass[aspectratio=169,handout]{beamer}

\usetheme{Boadilla}
%\usecolortheme{seahorse}
%\usecolortheme[RGB={245,66,24}]{structure}
\useoutertheme{infolines}

% packages
\usepackage{amsfonts,amsmath,amssymb,amsthm}
\usepackage[utf8]{inputenc}
\usepackage[T1]{fontenc}
\usepackage{lmodern}

\usepackage[francais]{babel}
\usepackage{fancybox}
\usepackage{graphicx}

\usepackage{float}
\usepackage{xfrac}

%\usepackage[usenames, x11names]{xcolor}
\usepackage{pgfplots}
\usepackage{datetime}


% ----------------------------------------------------------------------
% Pour les images
\usepackage{tikz}
\usetikzlibrary{calc,shadows,arrows.meta,patterns,matrix}

\newcommand{\tikzinput}[1]{\input{figures/#1.tikz}}
% --- les figures avec échelle éventuel
\newcommand{\myfigure}[2]{% entrée : échelle, fichier(s) figure à inclure
\begin{center}\small%
\tikzstyle{every picture}=[scale=1.0*#1]% mise en échelle + 0% (automatiquement annulé à la fin du groupe)
#2%
\end{center}}



%-----  Package unités -----
\usepackage{siunitx}
\sisetup{locale = FR,detect-all,per-mode = symbol}

%\usepackage{mathptmx}
%\usepackage{fouriernc}
%\usepackage{newcent}
%\usepackage[mathcal,mathbf]{euler}

%\usepackage{palatino}
%\usepackage{newcent}
% \usepackage[mathcal,mathbf]{euler}



% \usepackage{hyperref}
% \hypersetup{colorlinks=true, linkcolor=blue, urlcolor=blue,
% pdftitle={Exo7 - Exercices de mathématiques}, pdfauthor={Exo7}}


%section
% \usepackage{sectsty}
% \allsectionsfont{\bf}
%\sectionfont{\color{Tomato3}\upshape\selectfont}
%\subsectionfont{\color{Tomato4}\upshape\selectfont}

%----- Ensembles : entiers, reels, complexes -----
\newcommand{\Nn}{\mathbb{N}} \newcommand{\N}{\mathbb{N}}
\newcommand{\Zz}{\mathbb{Z}} \newcommand{\Z}{\mathbb{Z}}
\newcommand{\Qq}{\mathbb{Q}} \newcommand{\Q}{\mathbb{Q}}
\newcommand{\Rr}{\mathbb{R}} \newcommand{\R}{\mathbb{R}}
\newcommand{\Cc}{\mathbb{C}} 
\newcommand{\Kk}{\mathbb{K}} \newcommand{\K}{\mathbb{K}}

%----- Modifications de symboles -----
\renewcommand{\epsilon}{\varepsilon}
\renewcommand{\Re}{\mathop{\text{Re}}\nolimits}
\renewcommand{\Im}{\mathop{\text{Im}}\nolimits}
%\newcommand{\llbracket}{\left[\kern-0.15em\left[}
%\newcommand{\rrbracket}{\right]\kern-0.15em\right]}

\renewcommand{\ge}{\geqslant}
\renewcommand{\geq}{\geqslant}
\renewcommand{\le}{\leqslant}
\renewcommand{\leq}{\leqslant}
\renewcommand{\epsilon}{\varepsilon}

%----- Fonctions usuelles -----
\newcommand{\ch}{\mathop{\text{ch}}\nolimits}
\newcommand{\sh}{\mathop{\text{sh}}\nolimits}
\renewcommand{\tanh}{\mathop{\text{th}}\nolimits}
\newcommand{\cotan}{\mathop{\text{cotan}}\nolimits}
\newcommand{\Arcsin}{\mathop{\text{arcsin}}\nolimits}
\newcommand{\Arccos}{\mathop{\text{arccos}}\nolimits}
\newcommand{\Arctan}{\mathop{\text{arctan}}\nolimits}
\newcommand{\Argsh}{\mathop{\text{argsh}}\nolimits}
\newcommand{\Argch}{\mathop{\text{argch}}\nolimits}
\newcommand{\Argth}{\mathop{\text{argth}}\nolimits}
\newcommand{\pgcd}{\mathop{\text{pgcd}}\nolimits} 


%----- Commandes divers ------
\newcommand{\ii}{\mathrm{i}}
\newcommand{\dd}{\text{d}}
\newcommand{\id}{\mathop{\text{id}}\nolimits}
\newcommand{\Ker}{\mathop{\text{Ker}}\nolimits}
\newcommand{\Card}{\mathop{\text{Card}}\nolimits}
\newcommand{\Vect}{\mathop{\text{Vect}}\nolimits}
\newcommand{\Mat}{\mathop{\text{Mat}}\nolimits}
\newcommand{\rg}{\mathop{\text{rg}}\nolimits}
\newcommand{\tr}{\mathop{\text{tr}}\nolimits}


%----- Structure des exercices ------

\newtheoremstyle{styleexo}% name
{2ex}% Space above
{3ex}% Space below
{}% Body font
{}% Indent amount 1
{\bfseries} % Theorem head font
{}% Punctuation after theorem head
{\newline}% Space after theorem head 2
{}% Theorem head spec (can be left empty, meaning ‘normal’)

%\theoremstyle{styleexo}
\newtheorem{exo}{Exercice}
\newtheorem{ind}{Indications}
\newtheorem{cor}{Correction}


\newcommand{\exercice}[1]{} \newcommand{\finexercice}{}
%\newcommand{\exercice}[1]{{\tiny\texttt{#1}}\vspace{-2ex}} % pour afficher le numero absolu, l'auteur...
\newcommand{\enonce}{\begin{exo}} \newcommand{\finenonce}{\end{exo}}
\newcommand{\indication}{\begin{ind}} \newcommand{\finindication}{\end{ind}}
\newcommand{\correction}{\begin{cor}} \newcommand{\fincorrection}{\end{cor}}

\newcommand{\noindication}{\stepcounter{ind}}
\newcommand{\nocorrection}{\stepcounter{cor}}

\newcommand{\fiche}[1]{} \newcommand{\finfiche}{}
\newcommand{\titre}[1]{\centerline{\large \bf #1}}
\newcommand{\addcommand}[1]{}
\newcommand{\video}[1]{}

% Marge
\newcommand{\mymargin}[1]{\marginpar{{\small #1}}}

\def\noqed{\renewcommand{\qedsymbol}{}}


%----- Presentation ------
\setlength{\parindent}{0cm}

%\newcommand{\ExoSept}{\href{http://exo7.emath.fr}{\textbf{\textsf{Exo7}}}}

\definecolor{myred}{rgb}{0.93,0.26,0}
\definecolor{myorange}{rgb}{0.97,0.58,0}
\definecolor{myyellow}{rgb}{1,0.86,0}

\newcommand{\LogoExoSept}[1]{  % input : echelle
{\usefont{U}{cmss}{bx}{n}
\begin{tikzpicture}[scale=0.1*#1,transform shape]
  \fill[color=myorange] (0,0)--(4,0)--(4,-4)--(0,-4)--cycle;
  \fill[color=myred] (0,0)--(0,3)--(-3,3)--(-3,0)--cycle;
  \fill[color=myyellow] (4,0)--(7,4)--(3,7)--(0,3)--cycle;
  \node[scale=5] at (3.5,3.5) {Exo7};
\end{tikzpicture}}
}


\newcommand{\debutmontitre}{
  \author{} \date{} 
  \thispagestyle{empty}
  \hspace*{-10ex}
  \begin{minipage}{\textwidth}
    \titlepage  
  \vspace*{-2.5cm}
  \begin{center}
    \LogoExoSept{2.5}
  \end{center}
  \end{minipage}

  \vspace*{-0cm}
  
  % Astuce pour que le background ne soit pas discrétisé lors de la conversion pdf -> png
\begin{tikzpicture}
        \fill[opacity=0,green!60!black] (0,0)--++(0,0)--++(0,0)--++(0,0)--cycle; 
\end{tikzpicture}

% toc S'affiche trop tot :
% \tableofcontents[hideallsubsections, pausesections]
}

\newcommand{\finmontitre}{
  \end{frame}
  \setcounter{framenumber}{0}
} % ne marche pas pour une raison obscure

%----- Commandes supplementaires ------

% \usepackage[landscape]{geometry}
% \geometry{top=1cm, bottom=3cm, left=2cm, right=10cm, marginparsep=1cm
% }
% \usepackage[a4paper]{geometry}
% \geometry{top=2cm, bottom=2cm, left=2cm, right=2cm, marginparsep=1cm
% }

%\usepackage{standalone}


% New command Arnaud -- november 2011
\setbeamersize{text margin left=24ex}
% si vous modifier cette valeur il faut aussi
% modifier le decalage du titre pour compenser
% (ex : ici =+10ex, titre =-5ex

\theoremstyle{definition}
%\newtheorem{proposition}{Proposition}
%\newtheorem{exemple}{Exemple}
%\newtheorem{theoreme}{Théorème}
%\newtheorem{lemme}{Lemme}
%\newtheorem{corollaire}{Corollaire}
%\newtheorem*{remarque*}{Remarque}
%\newtheorem*{miniexercice}{Mini-exercices}
%\newtheorem{definition}{Définition}

% Commande tikz
\usetikzlibrary{calc}
\usetikzlibrary{patterns,arrows}
\usetikzlibrary{matrix}
\usetikzlibrary{fadings} 

%definition d'un terme
\newcommand{\defi}[1]{{\color{myorange}\textbf{\emph{#1}}}}
\newcommand{\evidence}[1]{{\color{blue}\textbf{\emph{#1}}}}
\newcommand{\assertion}[1]{\emph{\og#1\fg}}  % pour chapitre logique
%\renewcommand{\contentsname}{Sommaire}
\renewcommand{\contentsname}{}
\setcounter{tocdepth}{2}



%------ Encadrement ------

\usepackage{fancybox}


\newcommand{\mybox}[1]{
\setlength{\fboxsep}{7pt}
\begin{center}
\shadowbox{#1}
\end{center}}

\newcommand{\myboxinline}[1]{
\setlength{\fboxsep}{5pt}
\raisebox{-10pt}{
\shadowbox{#1}
}
}

%--------------- Commande beamer---------------
\newcommand{\beameronly}[1]{#1} % permet de mettre des pause dans beamer pas dans poly


\setbeamertemplate{navigation symbols}{}
\setbeamertemplate{footline}  % tiré du fichier beamerouterinfolines.sty
{
  \leavevmode%
  \hbox{%
  \begin{beamercolorbox}[wd=.333333\paperwidth,ht=2.25ex,dp=1ex,center]{author in head/foot}%
    % \usebeamerfont{author in head/foot}\insertshortauthor%~~(\insertshortinstitute)
    \usebeamerfont{section in head/foot}{\bf\insertshorttitle}
  \end{beamercolorbox}%
  \begin{beamercolorbox}[wd=.333333\paperwidth,ht=2.25ex,dp=1ex,center]{title in head/foot}%
    \usebeamerfont{section in head/foot}{\bf\insertsectionhead}
  \end{beamercolorbox}%
  \begin{beamercolorbox}[wd=.333333\paperwidth,ht=2.25ex,dp=1ex,right]{date in head/foot}%
    % \usebeamerfont{date in head/foot}\insertshortdate{}\hspace*{2em}
    \insertframenumber{} / \inserttotalframenumber\hspace*{2ex} 
  \end{beamercolorbox}}%
  \vskip0pt%
}


\definecolor{mygrey}{rgb}{0.5,0.5,0.5}
\setlength{\parindent}{0cm}
%\DeclareTextFontCommand{\helvetica}{\fontfamily{phv}\selectfont}

% background beamer
\definecolor{couleurhaut}{rgb}{0.85,0.9,1}  % creme
\definecolor{couleurmilieu}{rgb}{1,1,1}  % vert pale
\definecolor{couleurbas}{rgb}{0.85,0.9,1}  % blanc
\setbeamertemplate{background canvas}[vertical shading]%
[top=couleurhaut,middle=couleurmilieu,midpoint=0.4,bottom=couleurbas] 
%[top=fondtitre!05,bottom=fondtitre!60]



\makeatletter
\setbeamertemplate{theorem begin}
{%
  \begin{\inserttheoremblockenv}
  {%
    \inserttheoremheadfont
    \inserttheoremname
    \inserttheoremnumber
    \ifx\inserttheoremaddition\@empty\else\ (\inserttheoremaddition)\fi%
    \inserttheorempunctuation
  }%
}
\setbeamertemplate{theorem end}{\end{\inserttheoremblockenv}}

\newenvironment{theoreme}[1][]{%
   \setbeamercolor{block title}{fg=structure,bg=structure!40}
   \setbeamercolor{block body}{fg=black,bg=structure!10}
   \begin{block}{{\bf Th\'eor\`eme }#1}
}{%
   \end{block}%
}


\newenvironment{proposition}[1][]{%
   \setbeamercolor{block title}{fg=structure,bg=structure!40}
   \setbeamercolor{block body}{fg=black,bg=structure!10}
   \begin{block}{{\bf Proposition }#1}
}{%
   \end{block}%
}

\newenvironment{corollaire}[1][]{%
   \setbeamercolor{block title}{fg=structure,bg=structure!40}
   \setbeamercolor{block body}{fg=black,bg=structure!10}
   \begin{block}{{\bf Corollaire }#1}
}{%
   \end{block}%
}

\newenvironment{mydefinition}[1][]{%
   \setbeamercolor{block title}{fg=structure,bg=structure!40}
   \setbeamercolor{block body}{fg=black,bg=structure!10}
   \begin{block}{{\bf Définition} #1}
}{%
   \end{block}%
}

\newenvironment{lemme}[0]{%
   \setbeamercolor{block title}{fg=structure,bg=structure!40}
   \setbeamercolor{block body}{fg=black,bg=structure!10}
   \begin{block}{\bf Lemme}
}{%
   \end{block}%
}

\newenvironment{remarque}[1][]{%
   \setbeamercolor{block title}{fg=black,bg=structure!20}
   \setbeamercolor{block body}{fg=black,bg=structure!5}
   \begin{block}{Remarque #1}
}{%
   \end{block}%
}


\newenvironment{exemple}[1][]{%
   \setbeamercolor{block title}{fg=black,bg=structure!20}
   \setbeamercolor{block body}{fg=black,bg=structure!5}
   \begin{block}{{\bf Exemple }#1}
}{%
   \end{block}%
}


\newenvironment{miniexercice}[0]{%
   \setbeamercolor{block title}{fg=structure,bg=structure!20}
   \setbeamercolor{block body}{fg=black,bg=structure!5}
   \begin{block}{Mini-exercices}
}{%
   \end{block}%
}


\newenvironment{tp}[0]{%
   \setbeamercolor{block title}{fg=structure,bg=structure!40}
   \setbeamercolor{block body}{fg=black,bg=structure!10}
   \begin{block}{\bf Travaux pratiques}
}{%
   \end{block}%
}
\newenvironment{exercicecours}[1][]{%
   \setbeamercolor{block title}{fg=structure,bg=structure!40}
   \setbeamercolor{block body}{fg=black,bg=structure!10}
   \begin{block}{{\bf Exercice }#1}
}{%
   \end{block}%
}
\newenvironment{algo}[1][]{%
   \setbeamercolor{block title}{fg=structure,bg=structure!40}
   \setbeamercolor{block body}{fg=black,bg=structure!10}
   \begin{block}{{\bf Algorithme}\hfill{\color{gray}\texttt{#1}}}
}{%
   \end{block}%
}


\setbeamertemplate{proof begin}{
   \setbeamercolor{block title}{fg=black,bg=structure!20}
   \setbeamercolor{block body}{fg=black,bg=structure!5}
   \begin{block}{{\footnotesize Démonstration}}
   \footnotesize
   \smallskip}
\setbeamertemplate{proof end}{%
   \end{block}}
\setbeamertemplate{qed symbol}{\openbox}


\makeatother
\usecolortheme[RGB={102,102,255}]{structure}

% Commande spécifique à ce chapitre
\newcommand{\construc}{\mathcal{C}}
\newcommand{\plan}{\mathcal{P}}
\newcommand{\cercle}{\mathcal{C}}
   
%%%%%%%%%%%%%%%%%%%%%%%%%%%%%%%%%%%%%%%%%%%%%%%%%%%%%%%%%%%%%
%%%%%%%%%%%%%%%%%%%%%%%%%%%%%%%%%%%%%%%%%%%%%%%%%%%%%%%%%%%%%


\begin{document}


\title{{\bf La règle et le compas}}
\subtitle{\'Eléments de théorie des corps}

\begin{frame}
  
  \debutmontitre

  \pause

{\footnotesize
\hfill
\setbeamercovered{transparent=50}
\begin{minipage}{0.6\textwidth}
  \begin{itemize}
    \item<3-> Les exemples à comprendre
    \item<4-> Corps
    \item<5-> Extension de corps
    \item<6-> Nombre algébrique
  \end{itemize}
\end{minipage}
}

\end{frame}

\setcounter{framenumber}{0}


%%%%%%%%%%%%%%%%%%%%%%%%%%%%%%%%%%%%%%%%%%%%%%%%%%%%%%%%%%%%%%%%
\section{Les exemples à comprendre}

\begin{frame}
\evidence{Premier exemple}

$$\Qq(\sqrt 2) = \left\{ a+b\sqrt 2 \mid a,b \in \Qq \right\}$$

%$\Qq(\sqrt 2)$ contient $0$, $1$, $\frac13$, $\Qq$, $\sqrt 2$, $\frac 12 - \frac 23 \sqrt 2$, mais pas $\pi$

\pause

Soient $a+b\sqrt 2, a'+b'\sqrt 2 \in \Qq(\sqrt 2)$ 
\pause
\begin{itemize}
  \item $(a+b\sqrt 2)+(a'+b'\sqrt 2) \in \Qq(\sqrt 2)$
  \pause
  \item $-(a+b\sqrt 2) \in \Qq(\sqrt 2)$
  \pause  
  \item $(a+b\sqrt 2)\times(a'+b'\sqrt 2) = aa'+2bb' + (ab'+a'b)\sqrt 2 \in \Qq(\sqrt 2)$
  \pause
  \item $\frac{1}{a+b\sqrt 2} = \frac{1}{a^2-2b^2}(a-b\sqrt 2)\in \Qq(\sqrt 2)$
\end{itemize}

\bigskip
\pause
 
\begin{itemize}
  \item $\Qq(\sqrt 2)$ est un \defi{corps}
  \pause
  \item $\Qq(\sqrt 2)$ est une \defi{extension quadratique} de $\Qq$
\end{itemize}

\end{frame}


\begin{frame}
\evidence{Deuxième série d'exemples}

\pause
\begin{itemize}
  \item Soit $K$ un corps et $\delta \in K$
  \pause
  \item $K(\sqrt \delta) = \left\{ a+b\sqrt \delta \mid a,b \in K \right\}$ est un corps
  \pause
  \item $K_0 = \Qq$, $\delta_0 = 2$
  \pause
  \item $K_1 = \Qq(\sqrt 2)$\pause, $\delta_1 = 3$ alors $\sqrt 3 \notin \Qq(\sqrt 2)$
  \pause
  \item $K_2 = K_1(\sqrt 3) = \Qq(\sqrt2)(\sqrt 3) 
    \pause = \left\{ a+b\sqrt 2 + c\sqrt 3 + d\sqrt 2 \sqrt 3 \mid a,b,c,d \in \Qq \right\}$
  \pause
  \item $K_3 = \Qq(\sqrt2)(\sqrt 3)(\sqrt {11})$
  \pause
  \quad\  $= \left\{ a_1+a_2\sqrt 2 + a_3\sqrt 3 + a_4\sqrt{11}+ a_5\sqrt 2 \sqrt 3 \right. $
  
  \qquad\qquad $\left.+ a_6\sqrt 2 \sqrt {11} + a_7\sqrt 3 \sqrt {11} + a_8\sqrt 2 \sqrt 3 \sqrt {11} \right\}$
  \pause
  \item $K_0 = \Qq$, $K_1 = \Qq(\sqrt 2)$, $\delta_1=1+\sqrt2$
  \pause
  \item $K_2=K_1(\sqrt{1+\sqrt2}) = \Qq(\sqrt2)(\sqrt{1+\sqrt2})$
  
  \pause
  \quad\  $= \left\{ a+b\sqrt 2 + c\sqrt{1+\sqrt2} + d\sqrt 2\sqrt{1+\sqrt2} \right\}$ 
\end{itemize}

\end{frame}

\begin{frame}

\evidence{Une propriété}

\pause

\begin{itemize}[<+->]
  \item Chaque élément de $\Qq(\sqrt 2)$ est la racine d'un polynôme de degré $2$
à coefficients dans $\Qq$

  \item Exemple $3+\sqrt 2$ est annulé par $P(X) = (X-3)^2 -2 = X^2-6X+7$
  
  \item Un réel annulé par un polynôme à coefficients rationnels est un \defi{nombre algébrique}
  
  \item Si $K$ est un corps et $\delta \in K$ alors tout élément de $K(\sqrt \delta)$
est annulé par un polynôme de degré $1$ ou $2$ à coefficients dans $K$

  \item Chaque élément de $\Qq(\sqrt2)(\sqrt 3)$ (ou de $\Qq(\sqrt2)(\sqrt{1+\sqrt2})$)
est racine d'un polynôme de $\Qq[X]$ de degré $1$, $2$ ou $4$

  \item Chaque élément de $\Qq(\sqrt2)(\sqrt 3)(\sqrt {11})$ est racine d'un polynôme 
de $\Qq[X]$ de degré $1$, $2$, $4$ ou $8$
\end{itemize}
\end{frame}



%%%%%%%%%%%%%%%%%%%%%%%%%%%%%%%%%%%%%%%%%%%%%%%%%%%%%%%%%%%%%%%%
\section{Corps}

\begin{frame}
\begin{mydefinition}
 Un \defi{corps} $(K,+,\times)$ est un ensemble $K$
 muni des deux opérations $+$ et $\times$ 
 qui vérifient :
 \pause
 \begin{enumerate}  
 \setcounter{enumi}{-1}
  \item $+$ et $\times$ sont des lois de composition interne, c'est à dire
  $x+y \in K$ et $x\times y \in K$ pour tout $x,y \in K$
  \pause
   \item $(K,+)$ est un groupe commutatif
   \pause :
   \begin{itemize}
      \item Il existe $0 \in K$ tel que $0+x = x$
      
      \item Pour tout $x\in K$ il existe $-x$ tel que $x+(-x)=0$
      
      \item $+$ est associative : $(x+y)+z = x + (y+z)$
      
      \item $x+y=y+x$ (pour tout $x,y \in K$)
    \end{itemize} 
  \pause
  \item $(K\setminus\{0\},\times)$ est un groupe commutatif 
  \pause :  
     \begin{itemize} 
      \item Il existe $1 \in K\setminus\{0\}$ tel que $1 \times x = x$
      
      \item Pour tout $x\in K\setminus\{0\}$, il existe $x^{-1}$ tel que $x\times x^{-1}=1$
      
      \item $\times$ est associative : $(x\times y)\times z = x \times (y \times z)$
      
      \item $x \times y = y \times x$
    \end{itemize} 
  
  \pause
  \item $\times$ est distributive par rapport à $+$ : $(x+y)\times z= x\times z + y\times z$

 \end{enumerate}

\end{mydefinition}

\end{frame}


\begin{frame}

\evidence{Exemples}

\pause

\begin{itemize}
  \item $\Qq$, $\Rr$, $\Cc$ sont des corps
  \pause
  \item Par contre $(\Zz,+,\times)$ n'est pas un corps
\pause
 \item $\Qq(\sqrt 2) = \big\{ a+b\sqrt 2 \mid a,b \in \Qq \big\}$ est un corps
 \pause
 \item  $\Qq(\ii) =  \big\{ a+\ii b \mid a,b \in \Qq  \big\}$ est un corps
\pause
 \item Par contre  $ \big\{ a+b \pi \mid a,b \in \Qq  \big\}$ n'est pas un corps
\end{itemize}

\pause
\medskip

\begin{proposition}
L'ensemble des nombre réels constructibles $(\construc_\Rr, +, \times)$ est un corps
\end{proposition}

\end{frame}





%%%%%%%%%%%%%%%%%%%%%%%%%%%%%%%%%%%%%%%%%%%%%%%%%%%%%%%%%%%%%%%%
\section{Extension de corps}

\begin{frame}

\begin{proposition}
Soient $K, L$ deux corps avec $K \subset L$. 
Alors $L$ est un espace vectoriel sur $K$
\end{proposition}

\pause

\begin{mydefinition}
\begin{itemize}
  \item $L$ est appelé une \defi{extension} de $K$
  \pause
  \item Si la dimension de cet espace vectoriel est finie, alors
on l'appelle le \defi{degré} de l'extension, et on notera :
\vspace*{-2ex}
$$[L:K] = \dim_K L$$
\vspace*{-5ex} 
\pause
  \item Si ce degré vaut $2$, c'est une \defi{extension quadratique}
\end{itemize}
\end{mydefinition}

\pause

\begin{proposition}
\label{prop:KLM}
Si $K,L,M$ sont trois corps avec $K \subset L \subset M$
et si les extensions ont un degré fini alors :
\vspace*{-1ex}
$$[M:K] = [M:L] \times [L:K]$$
\end{proposition}

\end{frame}


\begin{frame}
\begin{exemple}
\begin{itemize}[<+->]  
  \item $\Cc$ est une extension de degré $2$ de $\Rr$
  \begin{itemize}[<+->]
    \item tout élément de $\Cc$ s'écrit $a+\ii b$
    \item les vecteurs $1$ et $\ii$ forment une base de $\Cc$, vu comme un espace vectoriel sur $\Rr$
  \end{itemize}
  
\medskip  
  \item $\Qq(\sqrt 2)$ est une extension quadratique de $\Qq$
  \begin{itemize}[<+->]
    \item $\Qq(\sqrt 2) = \big\{ a+b\sqrt 2 \mid a,b \in \Qq \big\}$
    \item $\Qq(\sqrt 2)$ est un espace vectoriel
    \item $(1,\sqrt 2)$ est une base
    \item en effet : $\sqrt{2} \notin \Qq$ équivaut à $1$ et $\sqrt{2}$ linéairement indépendants
  \end{itemize}

\medskip  
  \item $\Qq(\sqrt 2, \sqrt 3) = \Qq(\sqrt 2)(\sqrt 3) = \big\{ a+ b \sqrt 3 \mid a,b \in \Qq(\sqrt 2) \big\}$
  \begin{itemize}[<+->]
    \item $\Qq \subset \Qq(\sqrt 2) \subset \Qq(\sqrt 2, \sqrt 3)$
    \item calculer le degré des extensions
    \item expliciter une base sur $\Qq$ de $\Qq(\sqrt 2, \sqrt 3)$
  \end{itemize}
\end{itemize}
\end{exemple}

\end{frame}


\begin{frame}
\begin{itemize}
  \item Pour $x\in\Rr$, le \defi{corps engendré} par $x$, noté $\Qq(x)$, est le plus petit corps 
contenant $\Qq$ et $x$
  \pause
  \item Cohérent pour $\Qq(\sqrt{\delta})$ qui est bien le plus petit 
corps contenant $\Qq$ et $\sqrt\delta$
\end{itemize}

\medskip
\pause

\begin{exemple}
  $x = \sqrt[3]{2} = 2^{\frac13}$ 
  \vspace*{-2ex}\pause
  $$\Qq(\sqrt[3]{2}) = \left\{ a+b\sqrt[3]{2}+c\sqrt[3]{2}^2 \mid a,b,c \in \Qq \right\}$$
  \vspace*{-4ex}\pause
  \begin{itemize}
    \item $\Qq(\sqrt[3]{2})$ contient $x,x^2,x^3,\ldots$ mais aussi $\frac1x, \frac 1{x^2},\ldots$
    \pause
    \item Mais $x^3 = 2 \in \Qq$ \pause et $\frac1x = \frac{x^2}{2}$
    \pause
    \item $1,x,x^2$ engendrent tous les éléments de $\Qq(x)$
    \pause
    \item $[\Qq(\sqrt[3]{2}):\Qq] = 3$
  \end{itemize}
\end{exemple}

\end{frame}



%%%%%%%%%%%%%%%%%%%%%%%%%%%%%%%%%%%%%%%%%%%%%%%%%%%%%%%%%%%%%%%%
\section{Nombre algébrique}

\begin{frame}
L'ensemble des \defi{nombres algébriques} est
$$\overline{\Qq} = \big\{ x \in \Rr \mid \text{ il existe } P \in \Qq[X] \text{ non nul tel que } P(x)=0\big\}$$

\pause

\begin{proposition}
$\overline{\Qq}$ est un corps
\end{proposition}

\pause

\begin{proof}
\vspace*{-1ex}
 \begin{enumerate}
  \setcounter{enumi}{-1}
  \item $+$ et $\times$ sont des lois de composition interne 
  
\pause

  \item $(\overline{\Qq},+)$ est un groupe commutatif

\pause

  Si $x\in \overline{\Qq}$, soit $P(X)$ un polynôme qui annule $x$, alors $P(-X)$ annule $-x$,
  ainsi $-x \in \overline{\Qq}$
   

\pause

  \item $(\overline{\Qq}\setminus\{0\},\times)$ est un groupe commutatif

\pause
  Si $x\in \overline{\Qq}\setminus\{0\}$, soit $P(X)$ un polynôme de degré $n$ annulant $x$, alors 
      $X^nP(\frac{1}{X})$ est un polynôme annulant $\frac 1 x$, ainsi $x^{-1} \in \overline{\Qq}\setminus\{0\}$
  
\pause
    
  \item $\times$ est distributive par rapport à $+$ \qedhere
 \end{enumerate}
\end{proof}

\end{frame}


\begin{frame}

Si $x \in \overline{\Qq}$ est un nombre algébrique alors le plus petit degré, 
parmi tous les degrés des polynômes $P \in \Qq[X]$ tels que $P(x)=0$,
est le \defi{degré algébrique} de $x$

\pause

\begin{itemize}
  \item le degré algébrique de $\sqrt 2$ est $2$
  \pause
  \item le degré algébrique de $\sqrt \delta$ est $1$ ou $2$ ($\delta \in \Qq$)
  \pause
  \item le degré algébrique de $\sqrt[3]{2}$ est $3$
\end{itemize}
\vspace*{-1ex}
\pause
\begin{proposition}
\begin{enumerate}
  \item Soit $L$ une extension finie du corps $\Qq$. Si $x\in L$, alors $x$ est un nombre algébrique
\pause  
  \item Soit $x$ un nombre algébrique alors $\Qq(x)$ est une extension finie de $\Qq$
\pause  
  \item Si $x$ est un nombre algébrique alors le degré de l'extension $[\Qq(x):\Qq]$ et le degré algébrique de $x$ coïncident
\end{enumerate}
\end{proposition}

\pause

\begin{corollaire}
Si $x$ et $y$ sont des nombres réels algébriques alors $x+y$ et $x\times y$ aussi
\end{corollaire}
\end{frame}



\end{document}


%%%%%%%%%%%%%%%%%% PREAMBULE %%%%%%%%%%%%%%%%%%


\documentclass[12pt]{article}

\usepackage{amsfonts,amsmath,amssymb,amsthm}
\usepackage[utf8]{inputenc}
\usepackage[T1]{fontenc}
\usepackage[francais]{babel}


% packages
\usepackage{amsfonts,amsmath,amssymb,amsthm}
\usepackage[utf8]{inputenc}
\usepackage[T1]{fontenc}
%\usepackage{lmodern}

\usepackage[francais]{babel}
\usepackage{fancybox}
\usepackage{graphicx}

\usepackage{float}

%\usepackage[usenames, x11names]{xcolor}
\usepackage{tikz}
\usepackage{datetime}

\usepackage{mathptmx}
%\usepackage{fouriernc}
%\usepackage{newcent}
\usepackage[mathcal,mathbf]{euler}

%\usepackage{palatino}
%\usepackage{newcent}


% Commande spéciale prompteur

%\usepackage{mathptmx}
%\usepackage[mathcal,mathbf]{euler}
%\usepackage{mathpple,multido}

\usepackage[a4paper]{geometry}
\geometry{top=2cm, bottom=2cm, left=1cm, right=1cm, marginparsep=1cm}

\newcommand{\change}{{\color{red}\rule{\textwidth}{1mm}\\}}

\newcounter{mydiapo}

\newcommand{\diapo}{\newpage
\hfill {\normalsize  Diapo \themydiapo \quad \texttt{[\jobname]}} \\
\stepcounter{mydiapo}}


%%%%%%% COULEURS %%%%%%%%%%

% Pour blanc sur noir :
%\pagecolor[rgb]{0.5,0.5,0.5}
% \pagecolor[rgb]{0,0,0}
% \color[rgb]{1,1,1}



%\DeclareFixedFont{\myfont}{U}{cmss}{bx}{n}{18pt}
\newcommand{\debuttexte}{
%%%%%%%%%%%%% FONTES %%%%%%%%%%%%%
\renewcommand{\baselinestretch}{1.5}
\usefont{U}{cmss}{bx}{n}
\bfseries

% Taille normale : commenter le reste !
%Taille Arnaud
%\fontsize{19}{19}\selectfont

% Taille Barbara
%\fontsize{21}{22}\selectfont

%Taille François
%\fontsize{25}{30}\selectfont

%Taille Pascal
%\fontsize{25}{30}\selectfont

%Taille Laura
%\fontsize{30}{35}\selectfont


%\myfont
%\usefont{U}{cmss}{bx}{n}

%\Huge
%\addtolength{\parskip}{\baselineskip}
}


% \usepackage{hyperref}
% \hypersetup{colorlinks=true, linkcolor=blue, urlcolor=blue,
% pdftitle={Exo7 - Exercices de mathématiques}, pdfauthor={Exo7}}


%section
% \usepackage{sectsty}
% \allsectionsfont{\bf}
%\sectionfont{\color{Tomato3}\upshape\selectfont}
%\subsectionfont{\color{Tomato4}\upshape\selectfont}

%----- Ensembles : entiers, reels, complexes -----
\newcommand{\Nn}{\mathbb{N}} \newcommand{\N}{\mathbb{N}}
\newcommand{\Zz}{\mathbb{Z}} \newcommand{\Z}{\mathbb{Z}}
\newcommand{\Qq}{\mathbb{Q}} \newcommand{\Q}{\mathbb{Q}}
\newcommand{\Rr}{\mathbb{R}} \newcommand{\R}{\mathbb{R}}
\newcommand{\Cc}{\mathbb{C}} 
\newcommand{\Kk}{\mathbb{K}} \newcommand{\K}{\mathbb{K}}

%----- Modifications de symboles -----
\renewcommand{\epsilon}{\varepsilon}
\renewcommand{\Re}{\mathop{\text{Re}}\nolimits}
\renewcommand{\Im}{\mathop{\text{Im}}\nolimits}
%\newcommand{\llbracket}{\left[\kern-0.15em\left[}
%\newcommand{\rrbracket}{\right]\kern-0.15em\right]}

\renewcommand{\ge}{\geqslant}
\renewcommand{\geq}{\geqslant}
\renewcommand{\le}{\leqslant}
\renewcommand{\leq}{\leqslant}

%----- Fonctions usuelles -----
\newcommand{\ch}{\mathop{\mathrm{ch}}\nolimits}
\newcommand{\sh}{\mathop{\mathrm{sh}}\nolimits}
\renewcommand{\tanh}{\mathop{\mathrm{th}}\nolimits}
\newcommand{\cotan}{\mathop{\mathrm{cotan}}\nolimits}
\newcommand{\Arcsin}{\mathop{\mathrm{Arcsin}}\nolimits}
\newcommand{\Arccos}{\mathop{\mathrm{Arccos}}\nolimits}
\newcommand{\Arctan}{\mathop{\mathrm{Arctan}}\nolimits}
\newcommand{\Argsh}{\mathop{\mathrm{Argsh}}\nolimits}
\newcommand{\Argch}{\mathop{\mathrm{Argch}}\nolimits}
\newcommand{\Argth}{\mathop{\mathrm{Argth}}\nolimits}
\newcommand{\pgcd}{\mathop{\mathrm{pgcd}}\nolimits} 

\newcommand{\Card}{\mathop{\text{Card}}\nolimits}
\newcommand{\Ker}{\mathop{\text{Ker}}\nolimits}
\newcommand{\id}{\mathop{\text{id}}\nolimits}
\newcommand{\ii}{\mathrm{i}}
\newcommand{\dd}{\mathrm{d}}
\newcommand{\Vect}{\mathop{\text{Vect}}\nolimits}
\newcommand{\Mat}{\mathop{\mathrm{Mat}}\nolimits}
\newcommand{\rg}{\mathop{\text{rg}}\nolimits}
\newcommand{\tr}{\mathop{\text{tr}}\nolimits}
\newcommand{\ppcm}{\mathop{\text{ppcm}}\nolimits}

%----- Structure des exercices ------

\newtheoremstyle{styleexo}% name
{2ex}% Space above
{3ex}% Space below
{}% Body font
{}% Indent amount 1
{\bfseries} % Theorem head font
{}% Punctuation after theorem head
{\newline}% Space after theorem head 2
{}% Theorem head spec (can be left empty, meaning ‘normal’)

%\theoremstyle{styleexo}
\newtheorem{exo}{Exercice}
\newtheorem{ind}{Indications}
\newtheorem{cor}{Correction}


\newcommand{\exercice}[1]{} \newcommand{\finexercice}{}
%\newcommand{\exercice}[1]{{\tiny\texttt{#1}}\vspace{-2ex}} % pour afficher le numero absolu, l'auteur...
\newcommand{\enonce}{\begin{exo}} \newcommand{\finenonce}{\end{exo}}
\newcommand{\indication}{\begin{ind}} \newcommand{\finindication}{\end{ind}}
\newcommand{\correction}{\begin{cor}} \newcommand{\fincorrection}{\end{cor}}

\newcommand{\noindication}{\stepcounter{ind}}
\newcommand{\nocorrection}{\stepcounter{cor}}

\newcommand{\fiche}[1]{} \newcommand{\finfiche}{}
\newcommand{\titre}[1]{\centerline{\large \bf #1}}
\newcommand{\addcommand}[1]{}
\newcommand{\video}[1]{}

% Marge
\newcommand{\mymargin}[1]{\marginpar{{\small #1}}}



%----- Presentation ------
\setlength{\parindent}{0cm}

%\newcommand{\ExoSept}{\href{http://exo7.emath.fr}{\textbf{\textsf{Exo7}}}}

\definecolor{myred}{rgb}{0.93,0.26,0}
\definecolor{myorange}{rgb}{0.97,0.58,0}
\definecolor{myyellow}{rgb}{1,0.86,0}

\newcommand{\LogoExoSept}[1]{  % input : echelle
{\usefont{U}{cmss}{bx}{n}
\begin{tikzpicture}[scale=0.1*#1,transform shape]
  \fill[color=myorange] (0,0)--(4,0)--(4,-4)--(0,-4)--cycle;
  \fill[color=myred] (0,0)--(0,3)--(-3,3)--(-3,0)--cycle;
  \fill[color=myyellow] (4,0)--(7,4)--(3,7)--(0,3)--cycle;
  \node[scale=5] at (3.5,3.5) {Exo7};
\end{tikzpicture}}
}



\theoremstyle{definition}
%\newtheorem{proposition}{Proposition}
%\newtheorem{exemple}{Exemple}
%\newtheorem{theoreme}{Théorème}
\newtheorem{lemme}{Lemme}
\newtheorem{corollaire}{Corollaire}
%\newtheorem*{remarque*}{Remarque}
%\newtheorem*{miniexercice}{Mini-exercices}
%\newtheorem{definition}{Définition}




%definition d'un terme
\newcommand{\defi}[1]{{\color{myorange}\textbf{\emph{#1}}}}
\newcommand{\evidence}[1]{{\color{blue}\textbf{\emph{#1}}}}



 %----- Commandes divers ------

\newcommand{\codeinline}[1]{\texttt{#1}}

%%%%%%%%%%%%%%%%%%%%%%%%%%%%%%%%%%%%%%%%%%%%%%%%%%%%%%%%%%%%%
%%%%%%%%%%%%%%%%%%%%%%%%%%%%%%%%%%%%%%%%%%%%%%%%%%%%%%%%%%%%%

\newcommand{\construc}{\mathcal{C}}

\begin{document}

\debuttexte


%%%%%%%%%%%%%%%%%%%%%%%%%%%%%%%%%%%%%%%%%%%%%%%%%%%%%%%%%%%
\diapo


La théorie des corps n'est pas évidente et mériterait un chapitre
entier. 

\change
Nous résumons ici seulement les grandes lignes utiles à nos fins.

\change
Il est important de bien comprendre d'abord les exemples, que l'on va rencontrer.

\change
Nous définirons ensuite ce qu'est un corps,

\change
puis une extension de corps,

\change
avant de terminer par la notion de nombre algébrique.


%%%%%%%%%%%%%%%%%%%%%%%%%%%%%%%%%%%%%%%%%%%%%%%%%%%%%%%%%%%
\diapo

Voici l'exemple le plus important pour nous,

considérons l'ensemble :
$\Qq(\sqrt 2)$ qui est par définition 
$\left\{ a+b\sqrt 2 \mid a,b \in \Qq \right\}$

C'est un sous-ensemble de $\Rr$, qui contient par exemple $0$, $1$, $\frac13$ et tous les nombres rationnels
mais aussi $\sqrt 2$, et par exemples $\frac 12 - \frac 23 \sqrt 2$. 

\change

Dégageons quelques propriétés :
Fixons $a+b\sqrt 2$ et $a'+b'\sqrt 2$ deux éléments de $\Qq(\sqrt 2)$.

\change
Alors leur somme est $(a+a') + (b+b')\sqrt 2$ ; c'est encore un élément de $\Qq(\sqrt 2)$.

\change
De même l'opposé $-(a+b\sqrt 2)$ est aussi un élément de $\Qq(\sqrt 2)$.

\change
Le produit   $(a+b\sqrt 2)\times(a'+b'\sqrt 2)$ s'écrit 
$aa'+2bb' + (ab'+a'b)\sqrt 2$ et c'est toujours un élément de $\Qq(\sqrt 2)$.

\change
Enfin l'inverse d'un élément non nul $a+b\sqrt 2$ est $\frac{1}{a^2-2b^2}(a-b\sqrt 2)$,

c'est encore un élément de $\Qq(\sqrt 2)$.

\change
Ces propriétés font de $\Qq(\sqrt 2)$ ce que l'on appelle un \defi{corps}.

\change
Comme ce corps contient $\Qq$ on parle d'une \defi{extension} de $\Qq$.
De plus, il est étendu avec un élément du type $\sqrt \delta$, on parle alors d'une \defi{extension
quadratique}.


%%%%%%%%%%%%%%%%%%%%%%%%%%%%%%%%%%%%%%%%%%%%%%%%%%%%%%%%%%%
\diapo

On peut généraliser l'exemple précédent : 

\change
Si $K$ est lui-même un corps et $\delta$ est un élément de $K$ 

\change
alors on définit 
$K(\sqrt \delta) = \left\{ a+b\sqrt \delta \mid a,b \in K \right\}$

On vérifie comme ci-dessus que la somme et le produit de deux éléments restent
dans $K(\sqrt \delta)$, ainsi que l'opposé et l'inverse.
ce qui fait de $K$ un corps. 


\change
Cela permet de construire de nouveaux corps :
partant de $K_0 = \Qq$, on choisit un élément, disons $\delta_0 = 2$ 

\change
on obtient le corps plus gros $K_1 = \Qq(\sqrt 2)$.

Car bien sur $\sqrt 2$ n'est pas un nombre rationnel.

\change
Si on prend $\delta_1 = 3$ alors $\sqrt 3 \notin \Qq(\sqrt 2)$ 

\change 
et on définit $K_2 = K_1(\sqrt 3)$ qui est un nouveau corps (qui contient $K_1$).

\change
Le corps $K_2$ est :
$$K_2 = K_1(\sqrt 3) = \Qq(\sqrt2)(\sqrt 3) 
= \left\{ a+b\sqrt 2 + c\sqrt 3 + d\sqrt 2 \sqrt 3 \mid a,b,c,d \in \Qq \right\}.$$

\change
On pourrait continuer avec $\delta_2 = 11$ 

\change
et exprimer chaque élément de $\Qq(\sqrt2)(\sqrt 3)(\sqrt {11})$ comme une somme
de $8$ éléments ainsi avec les $a_i\in \Qq$.

\change
Revenons en arrière.
En partant de $K_1 = \Qq(\sqrt 2)$ on aurait pu considérer $\delta_1=1+\sqrt2$ 
(au lieu de $\delta = \sqrt 3$).

\change
Alors $K_2$ serait $K_2=K_1(\sqrt{1+\sqrt2}) = \Qq(\sqrt2)(\sqrt{1+\sqrt2})$. 


\change
Chaque élément de $K_2$ peut s'écrire comme une somme de $4$ éléments 
$a+b\sqrt 2 + c\sqrt{1+\sqrt2} + d\sqrt 2\sqrt{1+\sqrt2}$.


%%%%%%%%%%%%%%%%%%%%%%%%%%%%%%%%%%%%%%%%%%%%%%%%%%%%%%%%%%%
\diapo

Présentons sur des exemples les propriétés à bien comprendre pour la suite.

\change
Tout d'abord il faut remarquer que chaque élément de $\Qq(\sqrt 2)$ est la racine d'un polynôme de degré $2$
à coefficients dans $\Qq$. 

\change
Prenons par exemple $3+\sqrt 2$ il est est annulé par 
$P(X) = (X-3)^2 -2 = X^2-6X+7$.

En effet on commence par retirer $3$ donc $3+\sqrt 2-3 = \sqrt 2$.
Puis on élève au carré, ce qui donne $2$ puis on soustrait $2$ ce qui donne bien $0$.

\change
Les nombres qui sont annulés par un polynôme non nul à coefficients rationnels s'appellent 
les 
\defi{nombres algébriques}.

\change
Plus généralement, si $K$ est un corps et $\delta \in K$ alors tout élément de $K(\sqrt \delta)$
est annulé par un polynôme de degré $1$ ou $2$ à coefficients dans $K$.

\change
On en déduit que pour chaque élément de $\Qq(\sqrt2)(\sqrt 3)$ (ou de $\Qq(\sqrt2)(\sqrt{1+\sqrt2})$)
est racine d'un polynôme de $\Qq[X]$ de degré $1$, $2$ ou $4$.

\change
Et chaque élément de $\Qq(\sqrt2)(\sqrt 3)(\sqrt {11})$ est racine d'un polynôme 
de $\Qq[X]$ de degré $1$, $2$, $4$ ou $8$. Etc.


%%%%%%%%%%%%%%%%%%%%%%%%%%%%%%%%%%%%%%%%%%%%%%%%%%%%%%%%%%%
\diapo

Nous allons maintenant reprendre ces exemples d'une manière plus théorique.


Un corps $K$ est un ensemble sur lequel sont définies deux opérations : une addition et une multiplication.

\change
Tout d'abord ces deux lois $+$ et $\times$ sont des lois de composition interne, ce qui veut simplement dire que 
pour tout $x,y \in K$ $x+y \in K$ et $x\times y \in K$.

\change
Ensuite $(K,+)$ est un groupe commutatif,

\change
C'est-à-dire il existe un élément neutre pour l'addition, c'est $0$.

Chaque élément admet un opposé.

L'addition est associative.

et commutative.

\change
Autre condition $(K\setminus\{0\},\times)$ est un groupe commutatif, 

\change
Il existe un élément neutre pour la multiplication, c'est $1$.

Tout élément non nul admet un inverse.

La multiplication est associative.

et commutative.

\change
Enfin la multiplication est distributive par rapport à l'addition


%%%%%%%%%%%%%%%%%%%%%%%%%%%%%%%%%%%%%%%%%%%%%%%%%%%%%%%%%%%
\diapo

Voici des exemples classiques :

\change
$\Qq$, $\Rr$, $\Cc$ sont des corps.
 L'addition et la multiplication sont les opérations usuelles.
 
\change
Par contre $(\Zz,+,\times)$ n'est pas un corps. Je vous laisse réfléchir pourquoi ?

\change
Voici des exemples qui vont être importants pour la suite :
$\Qq(\sqrt 2) = \big\{ a+b\sqrt 2 \mid a,b \in \Qq \big\}$ est un corps
comme on l'a vu précédemment.

\change
Pour $\ii$ le nombre complexe vérifiant $\ii^2 = -1$ on pose
$\Qq(\ii) =  \big\{ a+\ii b \mid a,b \in \Qq  \big\}$ est un corps
avec l'addition et la multiplication habituelles des nombres complexes.

\change
Par contre  
$\big\{ a+b \pi \mid a,b \in \Qq  \big\}$ n'est pas un corps (où $\pi = 3,14\ldots$). 

Réfléchissez pourquoi ?

\change
Les constructions que l'on vu dans les parties précédentes avec les nombres constructibles
se reformulent avec la notion de corps en la proposition :

L'ensemble des nombre réels constructible $(\construc_\Rr, +, \times)$ est un corps inclus dans $\Rr$.



%%%%%%%%%%%%%%%%%%%%%%%%%%%%%%%%%%%%%%%%%%%%%%%%%%%%%%%%%%%
\diapo

Nous cherchons des propositions qui lient deux corps, lorsque l'un est inclus dans l'autre.
Les résultats de ce paragraphe seront admis.

Premier résultat :

Si $K, L$ sont deux corps avec $K \subset L$, alors $L$ est un espace-vectoriel sur $K$.


\change
Dans ces conditions on dira que 
$L$ est une \defi{extension} de $K$.

\change
Et si la dimension de cet espace-vectoriel est finie alors
on appelle cette dimension le \defi{degré} de l'extension, et on notera :
$$[L:K] = \dim_K L.$$

\change
Dans le cas d'un degré $2$ nous parlerons d'une \defi{extension quadratique}.

\change
Deuxième résultat : 

Si $K,L,M$ sont trois corps avec $K \subset L \subset M$
et si les extensions ont un degré fini alors :
$$[M:K] = [M:L] \times [L:K].$$


%%%%%%%%%%%%%%%%%%%%%%%%%%%%%%%%%%%%%%%%%%%%%%%%%%%%%%%%%%%
\diapo

Voyons quelques exemples pour assimiler toutes ces notions.

Tout d'abord 
$\Cc$ est une extension de degré $2$ de $\Rr$ 

\change
car tout élément de $\Cc$ s'écrit $a+\ii b$. 
  
\change
Donc les vecteurs $1$ et $\ii$ forment une base de $\Cc$, vu comme un espace vectoriel sur $\Rr$.

\change
Revenons sur l'exemple du début : $\Qq(\sqrt 2)$. 

C'est une extension de $\Qq$.

\change
En effet tout élément de $\Qq(\sqrt 2)$ s'écrit sous la forme $a+b\sqrt 2$

\change
Ce qui fait bien de $\Qq(\sqrt 2)$ un espace vectoriel sur $\Qq$.

\change
De plus $(1,\sqrt 2)$ en est une base, c'est donc un espace vectoriel  de dimension $2$ sur $\Qq$, autrement dit
une extension quadratique.

\change  
Attention : ici $1$ est un vecteur 
  et $\sqrt 2$ est un autre vecteur. Le fait que $\sqrt{2} \notin \Qq$ se 
  traduit en : ces deux vecteurs sont linéairement indépendants sur $\Qq$. 
  C'est un peu déroutant au début !
 
\change
Terminons par un petit exercice avec $\Qq(\sqrt 2, \sqrt 3) = \Qq(\sqrt 2)(\sqrt 3) =
\big\{ a+ b \sqrt 3 \mid a,b \in \Qq(\sqrt 2) \big\}$.

\change
On a bien sûr les inclusions : 
$\Qq \subset \Qq(\sqrt 2) \subset \Qq(\sqrt 2, \sqrt 3)$.

\change
A vous de calculer le degré de chacune des extensions. 

\change
Puis d'expliciter une base sur $\Qq$ de 
$\Qq(\sqrt 2, \sqrt 3)$.


%%%%%%%%%%%%%%%%%%%%%%%%%%%%%%%%%%%%%%%%%%%%%%%%%%%%%%%%%%%
\diapo

Pour $x\in\Rr$, on note $\Qq(x)$ le plus petit corps 
contenant $\Qq$ et $x$, c'est le \defi{corps engendré} par $x$.

\change
C'est cohérent avec la notation pour les 
extensions quadratiques $\Qq(\sqrt{\delta})$ qui est bien le plus petit 
corps contenant $\sqrt\delta$.

\change
Autre exemple, si $x = \sqrt[3]{2} = 2^{\frac13}$ 

\change
alors il n'est pas dur de calculer que 
$$\Qq(\sqrt[3]{2}) = \left\{ a+b\sqrt[3]{2}+c\sqrt[3]{2}^2 \mid a,b,c \in \Qq \right\}.$$

\change
En effet comme $\Qq(\sqrt[3]{2})$ est un corps, il contient contient $x$, $x^2 = x \times x$, $x^3,\ldots$ 
mais aussi $\frac1x, \frac 1{x^2},\ldots$.

\change
Cela fait beaucoup de monde mais en fait comme $x^3 = 2 \in \Qq$ 
puis $x^4 = 2x$, $x^5 = 2x^2$.

\change
Aussi comme $x^3 = 2$ alors $\frac1x = \frac{x^2}{2}$ et ainsi de suite 
tous les $x^k$ ,et $1/x^k$
 s'expriment en fonction de $1, x, x^2$.

\change
Ce qui fait que tous les éléments de $\Qq(x)$
peuvent s'écrire simplement $a + bx + cx^2$ avec $a,b,c \in \Qq$.

\change
Conclusion : $1,x1x^2$ est une base de $\Qq(\sqrt[3]{2})$, donc c'est une extension de $\Qq$ et son degré est $3$.

%%%%%%%%%%%%%%%%%%%%%%%%%%%%%%%%%%%%%%%%%%%%%%%%%%%%%%%%%%%
\diapo

[petit $x$]

L'ensemble des \defi{nombres algébriques} est par définition
$$\overline{\Qq} = \big\{ x \in \Rr \mid \text{ il existe } P \in \Qq[X] \text{ non nul tel que } P(x)=0\big\}.$$

Par exemple $\sqrt{2}$ est un nombre algébrique car $X^2-2$ annule $\sqrt 2$.
De même pour $\sqrt[3]{2}$ qui est annulé par $X^3-2$.

\change
Proposition :L'ensemble $\overline{\Qq}$ des nombres algébriques est un corps.


\change
L'addition et la multiplication définie sur $\overline{\Qq}$ sont celles du corps $\Rr$.

Ainsi beaucoup de propriétés découlent du fait que 
l'ensemble des réels est un corps (on parle de sous-corps).

La première chose à vérifier c'est que $+$ et $\times$ sont des lois de compositions internes,
c'est-à-dire que si $x$ et $y$ sont des nombres réels algébriques alors $x+y$ et $x\times y$ le sont aussi.

De façon surprenante c'est la partie la plus difficile, on y revient dans la diapo suivante.

\change
$(\overline{\Qq},+)$ est un groupe commutatif :

Toutes les propriétés sont à peu près évidentes :
 Par exemple   $0 \in \overline{\Qq}$ car annulé par le polynôme $X$.
 
\change 
Autre propriétés : 
Si $x\in \overline{\Qq}$ et est annulé par un polynôme $P(X)$
alors $-x$ est annulé pra $P(-X)$ donc $-x \in \overline{\Qq}$.

Les autres axiomes découlent de propriétés de $\Rr$.


\change 
$(\overline{\Qq}\setminus\{0\},\times)$ est un groupe commutatif, 

encore une fois le seul point un peu délicat est le suivant :

\change     
Si $x$ est un nombre algébrique non nul, il existe un polynôme $P(X)$ de degré $n$ annulant $x$, 
 alors $X^nP(\frac{1}{X})$ est un polynôme annulant $\frac 1 x$ donc 
 $x^{-1}$ est un nombre algébrique.
 
 \change
 Pour finir la multiplication est distributive par rapport à l'addition 
 : cela découle une nouvelle fois de la distributivité sur $\Rr$.
 
 
%%%%%%%%%%%%%%%%%%%%%%%%%%%%%%%%%%%%%%%%%%%%%%%%%%%%%%%%%%%
\diapo

[petit $x$]

Si $x \in \overline{\Qq}$ est un nombre algébrique définissons le degré algébrique de $x$.

On considère tous les polynômes non nuls à coefficients rationnels  tels que $P(x)=0$.

Le plus petit degré parmi tous les degrés de ces polynômes est le degré algébrique de $x$.


\change
Par exemple le degré algébrique de $\sqrt 2$ est $2$ :
en effet un polynôme annulant ce nombre est $X^2-2$ qui est de degré $2$
et on ne trouvera pas un polynome de degré $1$ à coefficients rationnels annulant $\sqrt 2$.

\change
Plus généralement $\sqrt \delta$ avec $\delta \in \Qq$
est de degré algébrique égal à $1$ ou $2$ 
(de degré algébrique $1$ si $\sqrt{\delta} \in \Qq$, de degré $2$ sinon).

\change
Par contre $\sqrt[3]{2}$ est de degré $3$, car il est annulé par $P(X) = X^3-2$ 
mais pas par des polynômes de degré plus petit.

\change
On termine avec cette proposition qui généralise ce que l'on constate sur ces exemples :

Tout d'abord soit $L$ une extension finie du corps $\Qq$. Si $x\in L$, alors $x$ est un nombre algébrique.

\change
Réciproquement si $x$ un nombre algébrique alors $\Qq(x)$ est une extension finie de $\Qq$.

\change
Enfin un point important est que si $x$ est un nombre algébrique alors le degré de l'extension 
$[\Qq(x):\Qq]$ et le degré algébrique de $x$ coïncident.

\change
Avec cette proposition on prouve facilement que si
$x$ et $y$ sont des nombres réels algébriques alors $x+y$ et $x\times y$ aussi.

Donc l'addition et la multiplication sont des lois de composition internes 
sur l'ensemble des nombres algébriques.


Pour les preuves de tous ces résultats je vous renvoie au chapitre du cours écrit.

\end{document}

%%%%%%%%%%%%%%%%%% PREAMBULE %%%%%%%%%%%%%%%%%%

\documentclass[11pt,a4paper]{article}

\usepackage{amsfonts,amsmath,amssymb,amsthm}
\usepackage[utf8]{inputenc}
\usepackage[T1]{fontenc}
\usepackage[francais]{babel}
\usepackage{fancybox}
\usepackage{graphicx}

%----- Ensembles : entiers, reels, complexes -----
\newcommand{\Nn}{\mathbb{N}} \newcommand{\N}{\mathbb{N}}
\newcommand{\Zz}{\mathbb{Z}} \newcommand{\Z}{\mathbb{Z}}
\newcommand{\Qq}{\mathbb{Q}} \newcommand{\Q}{\mathbb{Q}}
\newcommand{\Rr}{\mathbb{R}} \newcommand{\R}{\mathbb{R}}
\newcommand{\Cc}{\mathbb{C}} \newcommand{\C}{\mathbb{C}}
\newcommand{\Kk}{\mathbb{K}} \newcommand{\K}{\mathbb{K}}

%----- Modifications de symboles -----
\renewcommand{\epsilon}{\varepsilon}
\renewcommand{\Re}{\mathop{\mathrm{Re}}\nolimits}
\renewcommand{\Im}{\mathop{\mathrm{Im}}\nolimits}
\newcommand{\llbracket}{\left[\kern-0.15em\left[}
\newcommand{\rrbracket}{\right]\kern-0.15em\right]}

%----- Fonctions usuelles -----
\newcommand{\ch}{\mathop{\mathrm{ch}}\nolimits}
\newcommand{\sh}{\mathop{\mathrm{sh}}\nolimits}
\renewcommand{\tanh}{\mathop{\mathrm{th}}\nolimits}
\newcommand{\cotan}{\mathop{\mathrm{cotan}}\nolimits}
\newcommand{\Arcsin}{\mathop{\mathrm{Arcsin}}\nolimits}
\newcommand{\Arccos}{\mathop{\mathrm{Arccos}}\nolimits}
\newcommand{\Arctan}{\mathop{\mathrm{Arctan}}\nolimits}
\newcommand{\Argsh}{\mathop{\mathrm{Argsh}}\nolimits}
\newcommand{\Argch}{\mathop{\mathrm{Argch}}\nolimits}
\newcommand{\Argth}{\mathop{\mathrm{Argth}}\nolimits}
\newcommand{\pgcd}{\mathop{\mathrm{pgcd}}\nolimits} 

%----- Structure des exercices ------
\theoremstyle{definition}
\newtheorem{exo}{Exercice}
\newtheorem{ind}{Indications}
\newtheorem{cor}{Correction}

\newcommand{\exercice}[1]{} \newcommand{\finexercice}{}
%\newcommand{\exercice}[1]{{\tiny\texttt{#1}}\vspace{-2ex}} % pour afficher le numero absolu, l'auteur...
\newcommand{\enonce}{\begin{exo}} \newcommand{\finenonce}{\end{exo}}
\newcommand{\indication}{\begin{ind}} \newcommand{\finindication}{\end{ind}}
\newcommand{\correction}{\begin{cor}} \newcommand{\fincorrection}{\end{cor}}

\newcommand{\noindication}{\stepcounter{ind}}
\newcommand{\nocorrection}{\stepcounter{cor}}

\newcommand{\fiche}[1]{} \newcommand{\finfiche}{}
\newcommand{\titre}[1]{\centerline{\large \bf #1}}
\newcommand{\addcommand}[1]{}
\newcommand{\video}[1]{}

%----- Presentation ------
\setlength{\parindent}{0cm}

\newcommand{\ExoSept}{\textbf{\textsf{Exo7}}}
\newcommand{\LogoExoSept}{\setlength{\unitlength}{0.6em}
\begin{picture}(0,0)  \thicklines     \put(0,4){\line(0,1){3}}   \put(0,7){\line(1,0){3}}
  \put(3,7){\line(0,-1){7}}  \put(0,4){\line(1,0){7}}   \put(3,0){\line(1,0){4}}
  \put(7,0){\line(0,1){4}}   \put(3,7){\line(4,-3){4}}  \put(7,4){\line(3,4){3}}  
  \put(10,8){\line(-4,3){4}} \put(3,7){\line(3,4){3}}   \put(4.6,6.8){\mbox{\ExoSept}}
\end{picture}}

%----- Commandes supplementaires ------

\newcommand{\ii}{\mathrm{i}}

\usepackage[usenames, x11names]{xcolor}

%----- Commandes tikz -----
\usepackage{tikz}
\usepackage{pgfplots}
\usetikzlibrary{calc}
\usetikzlibrary{shadows}
\usetikzlibrary{arrows}
\usetikzlibrary{patterns}

\begin{document}


%%%%%%%%%%%%%%%%%% ENTETE %%%%%%%%%%%%%%%%%%

\LogoExoSept

%\kern-2em
\hfill\textsf{Ann\'ee 2014}

\vspace*{0.5ex}
\hrule\vspace*{1.5ex} 
\hfil\textsf{\textbf{\Large Exercices de math\'ematiques}}
\vspace*{1ex} \hrule 
\vspace*{5ex} 


%%%%%%%%%%%%%%%%%% EXERCICES %%%%%%%%%%%%%%%%%%
\fiche{f00xxx, bodin, 2014/04/24} 

\titre{Exercices -- La règle et le compas}


% % Décommenter pour chapitre seul -- Commenter pour livre
% %%%%%%%%%%%%%%%%%%% PREAMBULE %%%%%%%%%%%%%%%%%%

\documentclass[11pt]{book}

%----- Principaux packages -----
\usepackage{amsfonts,amsmath,amssymb}
\usepackage[utf8]{inputenc}
\usepackage[T1]{fontenc}
\usepackage[francais]{babel}
\usepackage{fancybox}
\usepackage{graphicx}
\usepackage{float}
\usepackage[usenames, x11names]{xcolor}
\usepackage{fouriernc}
\usepackage{ifthen}
\usepackage{titletoc} % Required for manipulating the tables of contents

%----- Ensembles : entiers, reels, complexes -----
\newcommand{\Nn}{\mathbb{N}} \newcommand{\N}{\mathbb{N}}
\newcommand{\Zz}{\mathbb{Z}} \newcommand{\Z}{\mathbb{Z}}
\newcommand{\Qq}{\mathbb{Q}} \newcommand{\Q}{\mathbb{Q}}
\newcommand{\Rr}{\mathbb{R}} \newcommand{\R}{\mathbb{R}}
\newcommand{\Cc}{\mathbb{C}} \newcommand{\C}{\mathbb{C}}
\newcommand{\Kk}{\mathbb{K}} \newcommand{\K}{\mathbb{K}}

%----- Modifications de symboles -----
\renewcommand{\epsilon}{\varepsilon}
\renewcommand{\Re}{\mathop{\text{Re}}\nolimits}
\renewcommand{\Im}{\mathop{\text{Im}}\nolimits}
\renewcommand{\ge}{\geqslant} \renewcommand{\geq}{\geqslant}
\renewcommand{\le}{\leqslant} \renewcommand{\leq}{\leqslant}


%----- Fonctions usuelles -----
\newcommand{\ch}{\mathop{\mathrm{ch}}\nolimits}
\newcommand{\sh}{\mathop{\mathrm{sh}}\nolimits}
\renewcommand{\tanh}{\mathop{\mathrm{th}}\nolimits}
\newcommand{\cotan}{\mathop{\mathrm{cotan}}\nolimits}
\newcommand{\Arcsin}{\mathop{\mathrm{arcsin}}\nolimits}
\newcommand{\Arccos}{\mathop{\mathrm{arccos}}\nolimits}
\newcommand{\Arctan}{\mathop{\mathrm{arctan}}\nolimits}
\newcommand{\Argsh}{\mathop{\mathrm{argsh}}\nolimits}
\newcommand{\Argch}{\mathop{\mathrm{argch}}\nolimits}
\newcommand{\Argth}{\mathop{\mathrm{argth}}\nolimits}
\newcommand{\pgcd}{\mathop{\mathrm{pgcd}}\nolimits} 

 %----- Commandes divers ------
\newcommand{\ii}{\mathrm{i}}
\newcommand{\dd}{\mathrm{d}}
\newcommand{\Ker}{\mathop{\text{Ker}}\nolimits}
\newcommand{\id}{\mathop{\text{id}}\nolimits}
\newcommand{\Card}{\mathop{\text{Card}}\nolimits}
\newcommand{\Vect}{\mathop{\text{Vect}}\nolimits}
\newcommand{\Mat}{\mathop{\mathrm{Mat}}\nolimits}
\newcommand{\rg}{\mathop{\text{rg}}\nolimits}
\newcommand{\tr}{\mathop{\text{tr}}\nolimits}
\newcommand{\ppcm}{\mathop{\text{ppcm}}\nolimits}


%----- Definition d'un terme -----
\newcommand{\defi}[1]{{\color{myorange}\textbf{\emph{#1}}}}
\newcommand{\evidence}[1]{{\color{blue}\textbf{\emph{#1}}}}
\newcommand{\assertion}[1]{{\og\emph{#1}\fg}} % pour chapitre logique


%-----  Package liens hypertexts ----- 
\usepackage{hyperref}
\hypersetup{colorlinks=true, linkcolor=blue, urlcolor=blue,
pdftitle={Exo7 - Cours de mathématiques}, pdfauthor={Exo7}}

%----- Commandes tikz -----
\usepackage{tikz}
\usepackage{pgfplots}
%\pgfplotsset{compat=newest}
%\pgfplotsset{compat=2.10}
\usetikzlibrary{calc}
\usetikzlibrary{shadows}
\usetikzlibrary{arrows}
\usetikzlibrary{patterns}
\usetikzlibrary{matrix}

%-----  Multi colonnes- ---- 
\usepackage{multicol}
\setlength{\columnseprule}{0.2mm}

%-----  Package d'importation ----- 
\usepackage{import}

%-----  Package unités -----
\usepackage{siunitx}
\sisetup{locale = FR,detect-all,per-mode = symbol}
%\sisetup{round-mode = places}

%----- Format de la page ------
\usepackage[a4paper]{geometry}
\geometry{top=2.5cm, bottom=2cm, left=2.5cm, right=2.5cm, marginparsep=1cm}
\setlength{\parindent}{0cm}


% Espace interligne (+ 10%)
\renewcommand{\baselinestretch}{1.1}

% Microtype
%\frenchbsetup{AutoSpacePunctuation=false}
\usepackage[babel=true, activate={true,nocompatibility},final,tracking=true,spacing=true,factor=1100,stretch=10,shrink=10]{microtype}
% activate={true,nocompatibility} - activate protrusion and expansion
% final - enable microtype; use "draft" to disable
% tracking=true, kerning=true, spacing=true - activate these techniques
% factor=1100 - add 10% to the protrusion amount (default is 1000)
% stretch=10, shrink=10 - reduce stretchability/shrinkability (default is 20/20)
% Overfull box

\emergencystretch=1.5em

% Police des titres
\newcommand{\policeheader}{\color{Ivory4}\fontencoding{T1}\fontfamily{fvs}\fontseries{b}\fontshape{n}\selectfont}
\newcommand{\policechapter}{\color{Red3}\fontencoding{T1}\fontfamily{fvs}\fontseries{bx}\fontshape{n}\selectfont}
\newcommand{\policesection}{\color{Tomato3}\fontencoding{T1}\fontfamily{fvs}\fontseries{bx}\fontshape{n}\selectfont}
\newcommand{\policesubsection}{\color{Tomato4}\fontencoding{T1}\fontfamily{fvs}\fontseries{b}\fontshape{n}\selectfont}


% Haut (et pied) de page
\usepackage{fancyhdr}

\fancypagestyle{plain}{%
\fancyhf{} % clear all header and footer fields
\fancyfoot[C]{\fcolorbox{white}{Ivory4}{\policeheader\color{white}\thepage}} % except the center
\renewcommand{\headrulewidth}{0pt}
\renewcommand{\footrulewidth}{0pt}}

\pagestyle{fancy}
\renewcommand{\chaptermark}[1]{\markboth{#1}{}}
%\lhead{\policeheader\leftmark}
%\chead{}
%\rhead{\fcolorbox{white}{Ivory4}{\policeheader\color{white}\thepage}}
\fancyhead[LO,RE]{\policeheader\leftmark}
\fancyhead[RO,LE]{\fcolorbox{white}{Ivory4}{\policeheader\color{white}\thepage}}
\lfoot{}
\cfoot{}
\rfoot{}

\renewcommand{\headrulewidth}{1pt}
\renewcommand{\headrule}{{\color{Ivory4}%
\hrule width\headwidth height\headrulewidth \vskip-\headrulewidth}}


%----- New Style sections ----- 
\usepackage{sectsty}
\sectionfont{\policesection}
\subsectionfont{\policesubsection}
\subsubsectionfont{\policesubsection}

\makeatletter
\renewcommand{\thesection}{\@arabic\c@section}
% \renewcommand{\thechapter}{}
% \renewcommand{\chaptername}{}
\makeatother

% Numérotation dans la marge + point après numéro
\makeatletter \def\@seccntformat#1{\llap{\csname the#1\endcsname.\ }} \makeatother


%----- New Chapter Title ----- 
\usepackage[explicit]{titlesec}
\newcommand*\chapterlabel{}
\titleformat{\chapter}
  {\gdef\chapterlabel{}
   \Large\policechapter}
  {\gdef\chapterlabel{\thechapter\quad }}{0pt}
  {\begin{tikzpicture}[remember picture,overlay]
    \node[yshift=-5cm] at (current page.north west)
      {\begin{tikzpicture}[remember picture, overlay]
        \draw[fill=Ivory2,Ivory2] (0,0) rectangle
          (\paperwidth,3cm);
        \node[anchor=east,xshift=.9\paperwidth,rectangle,
              rounded corners=18pt,inner sep=11pt,
              fill=Firebrick1]
              {\color{white}\chapterlabel#1};
       \end{tikzpicture}
      };
      \node[yshift=-3.5cm,xshift=2cm] at (current page.north west)      
      {{\normalsize\LogoExoSept{2.5}}
       };
   \end{tikzpicture}
  }
\titlespacing*{\chapter}{0pt}{150pt}{-80pt}


%Link to video Youtube

% variable myvideo : 0 no video, otherwise youtube reference
\newcommand{\video}[1]{\def\myvideo{#1}}
\newcommand{\insertvideo}[2]{\video{#1}%
{\small\texttt{\href{http://www.youtube.com/watch?v=\myvideo}{Vidéo $\blacksquare$ #2}}}}

% Liens vers les fiches d'exercices
\newcommand{\mafiche}[1]{\def\mymafiche{#1}}
\newcommand{\insertfiche}[2]{\mafiche{#1}%
{\small\texttt{\href{http://exo7.emath.fr/ficpdf/\mymafiche}{Exercices $\blacklozenge$ #2}}}}



%----- Sommaire et mini-sommaires ------ from latextemplate
% Package Titletoc


\contentsmargin{0cm} % Removes the default margin
% Chapter text styling
\titlecontents{chapter}[1.25cm] % Indentation
{\addvspace{10pt}\large%
\color{Red3}\fontencoding{T1}\fontfamily{fvs}\fontseries{m}\fontshape{n}\selectfont
} % Spacing and font options for chapters
{\contentslabel[\Large\thecontentslabel]{1.25cm}} % Chapter number
{}  
{\;\titlerule*[.5pc]{.}\;\thecontentspage} % Page number

% Section text styling
\titlecontents{section}[1.25cm] % Indentation
{\addvspace{2pt}%
\color{Tomato3}\fontencoding{T1}\fontfamily{fvs}\fontseries{m}\fontshape{n}\selectfont
} % Spacing and font options for sections
{\contentslabel[\thecontentslabel]{1.25cm}} % Section number
{}
{\;\titlerule*[.5pc]{.}\;\thecontentspage} % Page number
[]

% Subsection text styling
% \titlecontents{subsection}[1.25cm] % Indentation
% {\addvspace{1pt}\small%
% \color{Tomato4}\fontencoding{T1}\fontfamily{fvs}\fontseries{m}\fontshape{n}\selectfont
% } % Spacing and font options for subsections
% {\contentslabel[\thecontentslabel]{1.25cm}} % Subsection number
% {}
% {\sffamily\;\titlerule*[.5pc]{.}\;\thecontentspage} % Page number
% [] 

%----- Mini-sommaires ------ from latextemplate

% Section text styling
\titlecontents{lsection}[0.7em]{\color{Tomato3}\small\sffamily}{\contentslabel[\thecontentslabel]{1em}}{}{}

% Subsection text styling
\titlecontents{lsubsection}[1em]{\normalfont\footnotesize\sffamily}{\contentslabel[\thecontentslabel]{1.25cm}}{}{}


%----- Logo Exo7 ------
\definecolor{myred}{rgb}{0.93,0.26,0}
\definecolor{myorange}{rgb}{0.97,0.58,0}
\definecolor{myyellow}{rgb}{1,0.86,0}

\newcommand{\LogoExoSept}[1]{  % input : echelle
{\usefont{U}{cmss}{bx}{n}
\begin{tikzpicture}[scale=0.1*#1,transform shape]
  \fill[color=myorange] (0,0)--(4,0)--(4,-4)--(0,-4)--cycle;
  \fill[color=myred] (0,0)--(0,3)--(-3,3)--(-3,0)--cycle;
  \fill[color=myyellow] (4,0)--(7,4)--(3,7)--(0,3)--cycle;
  \node[scale=5] at (3.5,3.5) {Exo7};
\end{tikzpicture}}
}

%------ Titre livre -------------
\newcommand{\montitre}[1]{
\thispagestyle{empty}
~ \vfil
\begin{center}
{\Huge \policechapter #1}  \\
\vspace{2cm}
\LogoExoSept{5}
\end{center}
\addtocontents{toc}{\setcounter{tocdepth}{1}}
{\footnotesize
\bigskip
\tableofcontents
}
\finsommaire
\pagestyle{fancy}}

\newcommand{\finsommaire}{
\vfill\par\href{http://www.unisciel.fr/}{\includegraphics[scale=1]{logo_unisciel.png}}
\hfill\hspace*{9ex}\begin{minipage}{0.5\textwidth}\vspace*{-5.5ex}\center Cours et exercices 
de maths \\ \texttt{\href{http://exo7.emath.fr}{exo7.emath.fr}}\end{minipage}
\hfill\href{http://www.univ-lille1.fr/}{\includegraphics[scale=0.15]{logo_lille1_new.png}}
% \vspace*{-3.5ex}
\centerline{Licence Creative Commons \ -- BY-NC-SA -- \ 3.0 FR}
\newpage}

%%%%% SUITE A COMMENTER SI CHAPITRE SEUL %%%%%

%------ Chapitre dans livre -------------

% \newcommand{\chapitre}[1]{          % pour chapitre dans livre
% \chapter{#1}%\vspace*{-21.5ex}
% \thispagestyle{empty}
% \startcontents
% \printcontents{l}{1}{\addtocontents{ptc}{\setcounter{tocdepth}{1}}}
% 
% %\LogoExoSept{1.6} \vspace*{5ex}
% %{\small\minitoc}
% \vspace*{2ex} 
% }
% 
% \newcommand{\finchapitre}{}  % pour chapitre dans livre

%----------------------------------

%%%%% SUITE A COMMENTER SI LIVRE %%%%%

%------ Chapitre seul -------------

\newcommand{\chapitre}[1]{          % pour chapitre seul
\begin{document}
\chapter*{#1}
%\tableofcontents
}

\newcommand{\finchapitre}{\end{document}} % pour chapitre seul


%%%%% FIN  %%%%%
%----------------------------------

%----- Personnalisation tiret itemize -----
\renewcommand{\FrenchLabelItem}{{\bf--}}  % Evite confusion avec signe -

%----- Personnalisation pour les theoremes,... -----

%\usepackage[babel=true,kerning=true]{microtype} % to avoid conflict tikz/babel
\usepackage[framemethod=tikz]{mdframed}

%---- Theorem style ----
\mdfdefinestyle{theoremestyle}{%
linecolor=Tomato3,
middlelinewidth=2pt,%
roundcorner=5pt,
frametitlerule=false,%
frametitlefont=\bfseries,
frametitlealignment=\raggedright,
theoremseparator={.},
apptotikzsetting={\tikzset{mdfframetitlebackground/.append style={%
shade,left color=DarkOliveGreen4!30, right color=DarkOliveGreen4!10}}},
apptotikzsetting={\tikzset{mdfbackground/.append style={%
shade,left color=DarkOliveGreen3!30, right color=DarkOliveGreen3!10}}},
frametitlerulecolor=green!60,
frametitlerulewidth=1pt,
innertopmargin=0.6\topskip,
splittopskip=2pt, splitbottomskip=2pt,
skipabove=5pt, skipbelow=5pt,
needspace=10\baselineskip,
}
\mdtheorem[style=theoremestyle]{theoreme}{Théorème}
\mdtheorem[style=theoremestyle]{proposition}{Proposition}
\mdtheorem[style=theoremestyle]{propriete}{Propriété}
\mdtheorem[style=theoremestyle]{lemme}{Lemme}
\mdtheorem[style=theoremestyle]{corollaire}{Corollaire}



     
%---- Definition style ----
\mdfdefinestyle{definitionstyle}{%
linecolor=Tomato3,
middlelinewidth=2pt,%
rightline=false,
leftline=true,
topline=false,
bottomline=false,
%roundcorner=5pt,
frametitleaboveskip=3pt,
frametitlebelowskip=3pt,
frametitlerule=false,%
frametitlefont=\bfseries,
frametitlealignment=\raggedright,
theoremseparator={.},
apptotikzsetting={\tikzset{mdfframetitlebackground/.append style={%
shade,left color=DarkOliveGreen4!20, right color=DarkOliveGreen4!5}}},
apptotikzsetting={\tikzset{mdfbackground/.append style={%
shade,left color=DarkOliveGreen3!20, right color=DarkOliveGreen3!5}}},
frametitlerulecolor=green!60,
frametitlerulewidth=1pt,
innertopmargin=0.6\topskip,
splittopskip=2pt, splitbottomskip=2pt,
skipabove=5pt, skipbelow=5pt,
needspace=7\baselineskip,
}
\mdtheorem[style=definitionstyle]{definition}{Définition}

     
%---- Definition style ----
\mdfdefinestyle{exemplestyle}{%
linecolor=Tomato3,
middlelinewidth=2pt,%
rightline=false,
leftline=true,
topline=false,
bottomline=false,
%roundcorner=5pt,
frametitleaboveskip=3pt,
frametitlebelowskip=3pt,
frametitlerule=false,%
frametitlefont=\bfseries,
frametitlealignment=\raggedright,
theoremseparator={.},
apptotikzsetting={\tikzset{mdfframetitlebackground/.append style={%
shade,left color=DarkOliveGreen4!20, right color=white}}},
apptotikzsetting={\tikzset{mdfbackground/.append style={%
shade,left color=white, right color=white}}},
frametitlerulecolor=green!60,
frametitlerulewidth=1pt,
innertopmargin=0.6\topskip,
splittopskip=2pt, splitbottomskip=2pt,
skipabove=5pt, skipbelow=5pt,
needspace=7\baselineskip,
}

\mdtheorem[style=exemplestyle]{remarque}{Remarque}
\mdtheorem[style=exemplestyle]{exemple}{Exemple}
\mdtheorem[style=exemplestyle]{tp}{Travaux pratiques}
\mdtheorem[style=exemplestyle]{exercicecours}{Exercice}

%---- Definition style ----
\mdfdefinestyle{proofstyle}{%
middlelinewidth=2pt,%
leftmargin=12pt,
linecolor=Ivory4,
rightline=false,
leftline=true,
topline=false,
bottomline=false,
font=\small,
%roundcorner=5pt,
frametitleaboveskip=3pt,
frametitlebelowskip=3pt,
frametitlerule=false,%
frametitlefont=\bfseries,
frametitlealignment=\raggedright,
theoremseparator={}, %theoremseparator={.},
apptotikzsetting={\tikzset{mdfframetitlebackground/.append style={%
shade,left color=DarkOliveGreen4!20, right color=white}}},
apptotikzsetting={\tikzset{mdfbackground/.append style={%
shade,left color=DarkOliveGreen3!10, right color=white}}},
frametitlerulecolor=green!60,
frametitlerulewidth=1pt,
innertopmargin=0.6\topskip,
splittopskip=2pt, splitbottomskip=2pt,
skipabove=5pt, skipbelow=5pt,
needspace=7\baselineskip,
}
\mdtheorem[style=proofstyle]{proof}{Démonstration}
\renewcommand\theproof{} % No numeratotion to proof


%---- Mini-exercices style ----
\mdfdefinestyle{miniexercicesstyle}{%
linecolor=Ivory4,
middlelinewidth=2pt,%
roundcorner=0pt,
frametitlerule=false,%
frametitlefont=\bfseries,
frametitlealignment=\raggedright,
theoremseparator={.},
apptotikzsetting={\tikzset{mdfframetitlebackground/.append style={%
shade,left color=DarkOliveGreen4!30, right color=DarkOliveGreen4!10}}},
apptotikzsetting={\tikzset{mdfbackground/.append style={%
shade,left color=DarkOliveGreen3!30, right color=DarkOliveGreen3!5}}},
frametitlerulecolor=green!60,
frametitlerulewidth=0pt,
innertopmargin=0.6\topskip,
splittopskip=2pt, splitbottomskip=2pt,
skipabove=25pt, skipbelow=5pt,
needspace=7\baselineskip,
}
\mdtheorem[style=miniexercicesstyle]{miniexercices}{Mini-exercices}
\renewcommand\theminiexercices{} % No numeratotion to mini-exercices

%---- Algo style ----
\mdfdefinestyle{algostyle}{%
linecolor=DarkOliveGreen3,
middlelinewidth=2pt,%
roundcorner=0pt,
rightline=false,
leftline=true,
topline=false,
bottomline=false,
frametitlerule=false,%
frametitlefont=\bfseries,
frametitlealignment=\raggedright,
theoremseparator={},
apptotikzsetting={\tikzset{mdfframetitlebackground/.append style={%
shade,left color=DarkOliveGreen4!30, right color=white}}},
apptotikzsetting={\tikzset{mdfbackground/.append style={%
shade,left color=white, right color=white}}},
frametitlerulecolor=green!60,
frametitlerulewidth=1pt,
innertopmargin=0.6\topskip,
splittopskip=2pt, splitbottomskip=2pt,
skipabove=2pt, skipbelow=5pt,
needspace=7\baselineskip,
}
\mdtheorem[style=algostyle]{algo}{Code}
\renewcommand\thealgo{} % No numeratotion to algo


%---- Auteurs style ----
\mdfdefinestyle{auteurstyle}{%
linecolor=black!50,
middlelinewidth=0pt,%
roundcorner=5pt,
frametitlerule=false,%
frametitlefont=\bfseries,
fontcolor=red,
frametitlealignment=\raggedright,
theoremseparator={.},
apptotikzsetting={\tikzset{mdfframetitlebackground/.append style={%
shade,left color=black!15, right color=black!1}}},
apptotikzsetting={\tikzset{mdfbackground/.append style={%
shade,left color=black!1, right color=black!5}}},
frametitlerulecolor=green!60,
frametitlerulewidth=1pt,
innertopmargin=0.6\topskip,
splittopskip=2pt, splitbottomskip=2pt,
skipabove=5pt, skipbelow=0pt,
needspace=7\baselineskip,
}
\mdtheorem[style=auteurstyle]{auteur}{\color{black!80} Auteurs}
\newcommand{\auteurs}[1]{
\vfill
\begin{auteur*}
\color{black!70}
#1  
\end{auteur*}}



%----- Commandes anti-beamer -----
\newcommand{\pause}{}  % permet de mettre des \pause dans beamer pas dans poly
\newcommand{\beameronly}[1]{}

%------ Figures ------
\def\myscale{1} % par défaut 
\newcommand{\myfigure}[2]{  % entrée : echelle, fichier figure
\def\myscale{#1}\begin{center}\footnotesize{#2}\end{center}}


%------ Encadrement des formules ------
 \usepackage{fancybox}
% %\setlength{\fboxsep}{7pt}
\newcommand{\mybox}[1]{\begin{center}\shadowbox{#1}\end{center}}
\newcommand{\myboxinline}[1]{\raisebox{-2ex}{\shadowbox{#1}}}

% \tikzstyle{myboxstyle} = [draw=black, ultra thick, fill=white,
%     rectangle, drop shadow={
%                         top color=gray,
%                         bottom color=white,
%                         fill=gray,
%                         opacity=0.4,
%                         shadow xshift=3pt,
%                         shadow yshift=-3pt
%                         }, inner sep=10pt, inner ysep=10pt]
% \newmdenv[tikzsetting={fill=green!20},
%           roundcorner=10pt,shadow=true]{myshadowbox}
% \newcommand{\mybox}[1]{\begin{center}
% \begin{tikzpicture}\node[myboxstyle] (box) {#1};\end{tikzpicture}\end{center}}
% \newcommand{\myboxinline}[1]{\begin{tikzpicture}\node[myboxstyle] (box) {#1};\end{tikzpicture}}

% %----- Structure des exercices ------
% \newtheoremstyle{styleexo}% name
% {2ex}% Space above
% {3ex}% Space below
% {}% Body font
% {}% Indent amount 1
% {\bfseries} % Theorem head font
% {}% Punctuation after theorem head
% {\newline}% Space after theorem head 2
% {}% Theorem head spec (can be left empty, meaning ‘normal’)
% 
% \theoremstyle{styleexo}
% \newtheorem{exo}{Exercice}
% \newtheorem{ind}{Indications}
% \newtheorem{cor}{Correction}
% 
% \newcommand{\exercice}[1]{} \newcommand{\finexercice}{}
% %\newcommand{\exercice}[1]{{\tiny\texttt{#1}}\vspace{-2ex}} % pour afficher le numero absolu, l'auteur...
% \newcommand{\enonce}{\begin{exo}} \newcommand{\finenonce}{\end{exo}}
% \newcommand{\indication}{\begin{ind}} \newcommand{\finindication}{\end{ind}}
% \newcommand{\correction}{\begin{cor}} \newcommand{\fincorrection}{\end{cor}}
% \newcommand{\noindication}{\stepcounter{ind}}
% \newcommand{\nocorrection}{\stepcounter{cor}}
% \newcommand{\fiche}[1]{} \newcommand{\finfiche}{}
% \newcommand{\titre}[1]{\centerline{\large \bf #1}}
% \newcommand{\addcommand}[1]{}
% 
% 
%  
 
%------ Algorithmes ------

\newcommand{\Python}{\texttt{Python}}
\newcommand{\Sage}{\texttt{Sage}}

% Pour afficher du code
\usepackage{listingsutf8}

\lstset{
  language=Python,
  upquote=true,
  columns=flexible,
  keepspaces=true,
  basicstyle=\ttfamily,
  commentstyle=\color{gray},
  showspaces=false,
  showstringspaces=false
}

% \makeatletter
% \def\lst@outputspace{\lst@bkgcolor\empty\color{white}}
% \makeatother

\makeatletter
\def\lst@outputspace{{\ifx\lst@bkgcolor\empty\color{white}\else
\lst@bkgcolor\fi\lst@visiblespace}}
\lst@keepspacestrue
\lst@keepspacestrue 
\makeatother

% Code inline
\newcommand{\codeinline}[1]{\lstinline!#1!}

% Long code
\newcommand{\insertcode}[2]{
\begin{algo}[{\sl #2}]
\lstinputlisting[inputencoding=utf8/latin1]{../#1}  
\end{algo}
}

% \newcommand{\insertcodebis}[2]{
% \begin{algo}[#2]
% \lstinputlisting[firstline=10,inputencoding=utf8/latin1]{../#1}  
% \end{algo}
% }
% \input{/home/arnaud/Maths/Geometrie/preamb-geo.tex}
% 
% % Commandes specifiques a ce chapitre
% % \newcommand{\construc}{\mathcal{C}}
% % \newcommand{\plan}{\mathcal{P}}
% % \newcommand{\cercle}{\mathcal{C}}
% 
% %%%%%%%%%%%%%%%%%% ENTETE %%%%%%%%%%%%%%%%%%
% 
% %\chapitre{Exercices -- La règle et le compas}
% 
% \begin{document}
% \montitre{Exercices -- La règle et le compas}

% Autres idées d'exercices :  
  % Réciproque du théorème de Wantzel est fausse
 
% Références pour cette fiches :
% André Pillons, Vassallo-Royer, Carrega


%%%%%%%%%%%%%%%%%%%%%%%%%%%%%%%%%%%%%%%%%%%%%%%%%%%%%%%%%%%%%%%%
\section{Constructions et les trois problèmes grecs}


\exercice{}
\enonce[Constructions élémentaires]
\'Etant donné deux points $A,B$, dessiner à la règle et au compas :
\begin{enumerate}
    \item un triangle équilatéral de base $[AB]$,
    \item un carré de base $[AB]$,
    \item un rectangle dont l'un des côtés est $[AB]$ et l'autre est de longueur double,
    \item un losange dont l'une des diagonales est $[AB]$ 
    et l'autre est de longueur $\frac14 AB$,    
    \item un hexagone régulier dont l'un des côtés est $[AB]$.    
\end{enumerate} 
\finenonce
\finexercice


\exercice{}
\enonce[Centres de cercles]
\begin{itemize}
  \item \'Etant donné un cercle dont on a perdu le centre, retrouver 
  le centre à la règle et compas.
  
  \item \'Etant donné un triangle, tracer son cercle circonscrit.
  
  \item \'Etant donné un triangle, tracer son cercle inscrit.  
\end{itemize}
\finenonce
\finexercice



%%%%%%%%%%%%%%%%%%%%%%%%%%%%%%%%%%%%%%%%%%%%%%%%%%%%%%%%%%%%%%%%%%%%%%%%%%%%%%%%%%%%%%%%%%%%%%

\exercice{}
\enonce[Construction au compas seul]
Construire au compas seulement :
\begin{enumerate}
 \item Le symétrique de $P$ par rapport à une droite $(AB)$. (Seuls les points $P, A, B$ sont tracés, 
 pas la droite.)

 \item Le symétrique d'un point $P$ par rapport à un point $O$.

 \item(*) Le milieu de deux points $A$, $B$. (La droite $(AB)$ n'est pas tracée !)
\end{enumerate}
\finenonce
\finexercice



\exercice{}
\enonce[Pentagone régulier]
% \exercice{77, bodin, 1998/09/01}
% \video{wHlb0IsMB7Q}

Soit $(A_{0},A_{1},A_{2},A_{3},A_{4})$ un pentagone r\'egulier. On note $O$ son centre et on
choisit un rep\`ere orthonorm\'e $(O,\overrightarrow{u},\overrightarrow{v})$ avec
$\overrightarrow{u}=\overrightarrow{OA_{0}}$, qui nous permet d'identifier le plan avec
l'ensemble des nombres complexes $\Cc$.


\begin{enumerate}
\item
Donner les affixes $\omega_{0},\ldots,\omega_{4}$ des points $A_{0},\ldots,A_{4}$. Montrer
que  $\omega_{k}={\omega_{1}}^k$ pour $ k\in\{0,1,2,3,4\}$. Montrer que
$1+\omega_{1}+\omega_{1}^2+\omega_{1}^3+\omega_{1}^4=0$.

\item
En d\'eduire que $\cos(\frac{2\pi}{5})$ est l'une des solutions de l'\'equation $4z^2+2z-1=0$.
En d\'eduire la valeur de $\cos(\frac{2\pi}{5})$.

\item
On consid\`ere le point $B$ d'affixe $-1$. Calculer la longueur $BA_{2}$ en fonction de
$\cos\frac{2\pi}{5}$ puis de $\sqrt{5}$.

\item
On consid\`ere le point $I$ d'affixe $\frac{\ii}{2}$, le cercle $\mathcal{C}$ de centre $I$ de
rayon $\frac{1}{2}$ et enfin le point $J$ d'intersection de $\mathcal{C}$ avec le segment
$[BI]$. Calculer la longueur $BI$ puis  la longueur $BJ$.

\item
\textbf{Application:} Dessiner un pentagone r\'egulier \`a la r\`egle et au compas. Expliquer.
\end{enumerate}


   \begin{tikzpicture}[scale=2]
      \coordinate (a0) at (0:1);
      \coordinate (a1) at (72:1);      
      \coordinate (a2) at (72*2:1);
      \coordinate (a3) at (72*3:1);
      \coordinate (a4) at (72*4:1);     
      \coordinate (o) at (0,0);
      
      \draw (0,0) circle (1);
      \draw[thick] (a0)--(a1)--(a2)--(a3)--(a4)--cycle;
      
      \node at (o) [below right] {$O$};
      \node at (a0) [above right] {$A_0$};      
      \node at (a1) [above] {$A_1$};
      \node at (a2) [left] {$A_2$};
      \node at (a3) [left] {$A_3$};
      \node at (a4) [below] {$A_4$};
      \node at (a0) [below right] {$1$};      
      \node at (0,1) [above left] {$\ii$};

      \draw[->] (-1.2,0)--(1.2,0);
      \draw[->] (0,-1.2)--(0,1.2);         
   \end{tikzpicture}


\finenonce 

\noindication

\correction
\begin{enumerate}
\item
Comme $(A_{0},\ldots,A_{4})$ est un pentagone r\'egulier, on a
$OA_{0}=OA_{1}=OA_{2}=OA_{3}=OA_{4}=1$ et $
  (\overrightarrow{OA_{0}},\overrightarrow{OA_{1}})=\frac{2\pi}{5}[2\pi],
  (\overrightarrow{OA_{0}},\overrightarrow{OA_{2}})=\frac{4\pi}{5}[2\pi],
  (\overrightarrow{OA_{0}},\overrightarrow{OA_{3}})=-\frac{4\pi}{5}[2\pi],
  (\overrightarrow{OA_{0}},\overrightarrow{OA_{4}})=-\frac{2\pi}{5}[2\pi],
 $.
On en d\'eduit:
 $
  \omega_{0}=1,
  \omega_{1}=e^{\frac{2\ii\pi}{5}},
  \omega_{2}=e^{\frac{4\ii\pi}{5}},
  \omega_{3}=e^{-\frac{4\ii\pi}{5}}=e^{\frac{6\ii\pi}{5}},
  \omega_{4}=e^{-\frac{2\ii\pi}{5}}=e^{\frac{8\ii\pi}{5}},
 $.
On a bien $\omega_{i}=\omega_{1}^i$. Enfin, comme
$\omega_{1}\neq0$, $1+\omega_{1}+\ldots+\omega_{1}^4=
\frac{1-\omega_{1}^5}{1-\omega_{1}}=\frac{1-1}{1-\omega_{1}}=0$.

\item $\mathop{\mathrm{Re}}\nolimits(1+\omega_{1}+\ldots+\omega_{1}^4)=
1+2\cos(\frac{2\pi}{5})+2\cos(\frac{4\pi}{5})$. Comme
$\cos(\frac{4\pi}{5})=2\cos^2(\frac{2\pi}{5})-1$ on en d\'eduit:
$4\cos^2(\frac{2\pi}{5})+2\cos(\frac{2\pi}{5})-1=0$.
$\cos(\frac{2\pi}{5})$ est donc bien une solution de l'\'equation
$4z^2+2z-1=0$. Etudions cette \'equation: $\Delta=20=2^2.5$. Les
solutions sont donc $\frac{-1-\sqrt{5}}{4}$ et
$\frac{-1+\sqrt{5}}{4}$. Comme $\cos(\frac{2\pi}{5})>0$, on en
d\'eduit que $\cos(\frac{2\pi}{5})=\frac{\sqrt{5}-1}{4}$.

\item
 $
  BA_{2}^2=|\omega_{2}+1|^2
          =|\cos(\frac{4\pi}{5})+i\sin(\frac{4\pi}{5})+1|^2
          =1+2\cos(\frac{4\pi}{5})+\cos^2(\frac{4\pi}{5})+\sin^2(\frac{4\pi}{5})
          =4\cos^2(\frac{2\pi}{5})
  $. Donc $BA_{2}=\frac{\sqrt{5}-1}{2}$.

\item
$BI=|\ii/2+1|=\frac{\sqrt{5}}{2}$. $BJ=BI-1/2=\frac{\sqrt{5}-1}{2}$.

\item
Pour tracer un pentagone r\'egulier, on commence par tracer un
cercle $C_{1}$ et deux diam\`etres orthogonaux, qui jouent le r\^ole
du cercle passant par les sommets et des axes de coordonn\'ees. On
trace ensuite le milieu d'un des rayons: on obtient le point I de
la question 4. On trace le  cercle de centre $I$ passant par le
centre de $C_{1}$: c'est le cercle $\mathcal{C}$. On trace le
segment $[BI]$ pour obtenir son point $J$ d'intersection avec
$\mathcal{C}$. On trace enfin le cercle de centre $B$ passant par
$J$: il coupe $C_{1}$ en $A_{2}$ et $A_{3}$, deux sommets du
pentagone. Il suffit pour obtenir tous les sommets de reporter la
distance $A_{2}A_{3}$ sur $C_{1}$, une fois depuis $A_{2}$, une
fois depuis $A_{3}$. (en fait le cercle de centre $B$ et passant
par $J'$, le point de $\mathcal{C}$ diam\'etralement oppos\'e \`a
$J$, coupe $C_{1}$ en $A_{1}$ et $A_{4}$, mais nous ne l'avons pas
justifi\'e par le calcul : c'est un exercice !)
\end{enumerate}
% $$
% \includegraphics[65mm,50mm]{penta.eps}
% $$
\fincorrection
\finexercice

%%%%%%%%%%%%%%%%%%%%%%%%%%%%%%%%%%%%%%%%%%%%%%%%%%%%%%%%%%%%%%%%
\section{Les nombres constructibles à la règle et au compas}


\exercice{}
\enonce[Construction de l'axe des ordonnées]
\begin{enumerate}
  \item Montrer qu'il possible de construire un point de l'axe des ordonnées 
  (autre que l'origine) dans l'ensemble des points constructibles $\mathcal{C}_3$.
  
  \item En déduire que $\mathcal{C}_4$ contient le point $(0,1)$.
\end{enumerate}
\finenonce
\finexercice


\exercice{}
\enonce[Constructions approchant $\pi$]
\ 
\begin{enumerate}
 \item Construire les approximations suivantes de $\pi$ : 
$\frac{22}{7}=3,1428\ldots$, $\sqrt{2}+\sqrt{3}=3,1462\ldots$
 
 \item Construire $\frac{\sqrt{2}}{\sqrt{3}}$, $5^{\frac{1}{4}}$, $\sqrt{3-\sqrt{3}}$. 
     
\end{enumerate}
\finenonce

\finexercice


\exercice{}
\enonce[Approximation de Kochanski]
Une valeur approchée de $\pi$ avec quatre décimales exactes est donnée par
$$\phi = \sqrt{\frac{40}{3} - 2 \sqrt{3}} = 3,141533\ldots$$
Soient les points suivants $O(0,0)$, $I(1,0)$, $Q(2,0)$. 
On construit les points $P_1, P_2,\ldots$ ainsi:
\begin{itemize}
 \item $P_1$ est l'intersection des cercles centrés en $O$ et $I$ de rayon $1$ ayant une ordonnée positive.
 \item $P_2$ est l'intersection des cercles centrés en $O$ et $P_1$ de rayon $1$ (l'autre intersection est $I$).
 \item $P_3$ est l'intersection de  la droite $P_2I$ avec l'axe des ordonnées.
 \item $P_4$ est l'image de $P_3$ par une translation de vecteur $(0,-3)$.
\end{itemize}
Calculer les coordonnées de chacun des $P_i$.
Montrer que la longueur $P_4Q$ vaut $\phi$.
\finenonce

\finexercice

%%%%%%%%%%%%%%%%%%%%%%%%%%%%%%%%%%%%%%%%%%%%%%%%%%%%%%%%%%%%%%%%
\section{\'Eléments de théorie des corps}


\exercice{}
\enonce[Extensions quadratiques]
\begin{enumerate}
  \item Montrer que $\sqrt{3} \notin \Qq(\sqrt2)$.
  \item Montrer que $\sqrt{1+\sqrt2} \notin \Qq(\sqrt2)$.
  \item Montrer que $\sqrt 5 \notin \Qq(\sqrt2)(\sqrt 3)$.
  \item Trouver un polynôme $P \in \Qq[X]$ tel que que $P(\sqrt{1+\sqrt2}) = 0$.   
  \item Trouver un polynôme $P \in \Qq[X]$ tel que que $P(\sqrt2 + \sqrt 3) = 0$. 
  % C'est x4 − 10x2 + 1 = (x − √2 − √3)(x + √2 − √3)(x − √2 + √3)(x + √2 + √3)
  \item Montrer que $\Qq(\sqrt[3]{2})$ 
\end{enumerate}
\finenonce
\finexercice


\exercice{}
\enonce[Degré algébrique]

\begin{enumerate}
  \item Soit $x \in \Rr$ montrer que $\Qq(x)$, le plus petit corps contenant $x$, vérifie :
  $$\Qq(x) =\left\{ \frac{P(x)}{Q(x)} \mid P,Q \in \Qq[X] \text{ et } Q(x) \neq 0 \right\}.$$
  
  \item Montrer que $K = \left\{ a+b\sqrt[3]{2}+c\sqrt[3]{2}^2 \mid a,b,c \in \Qq \right\}$ est un corps.
  
  \item Montrer que  $K=\Qq(\sqrt[3]{2})$ ; c'est-à-dire que $K$ est le 
  plus petit corps contenant $\Qq$ et $\sqrt[3]{2}$.
  
  \item Vérifier sur cet exemple que le degré algébrique de $\sqrt[3]{2}$ égale 
  le degré de l'extension $[\Qq(\sqrt[3]{2}):\Qq]$.
  
  \item Expliciter une extension de $\Qq$ ayant le degré $4$.
\end{enumerate}
\finenonce

\correction
\begin{enumerate}
  \item 
  \begin{enumerate}
    \item   
    \item    
  \end{enumerate} 
  
  \item Montrons que $K = \left\{ a+b\sqrt[3]{2}+c\sqrt[3]{2}^2 \mid a,b,c \in \Qq \right\}$ est un corps
  L'addition et la multiplication définie sur $K$ sont celles du corps $(\Rr,+,\times)$.
  Beaucoup de propriétés découlent du fait que l'ensemble des réels est un corps.
  
  [pb : mq $+$ et $\times$ lci]
  
 \begin{enumerate}
  \item $(K,+)$ est un groupe commutatif, car :
    \begin{itemize}
      
      \item $0 \in K$ (prendre $a=b=c=0$) et $0+x = x$ (pour tout $x\in K$).
      
      \item Si $x\in K$ alors $-x \in K$.
      
      \item $+$ est associative : cela découle de l'associativité sur $\Rr$.
      
      \item $x+y=y+x$ : idem.
    \end{itemize} 
  
  \item $(K\setminus\{0\},\times)$ est un groupe commutatif, en effet :
     \begin{itemize} 
      \item $1 \in K\setminus\{0\}$ et $1 \times x = x$ (pour tout $x\in K\setminus\{0\}$).
      
      \item Si $x\in K\setminus\{0\}$ alors $x^{-1} \in K\setminus\{0\}$ :
      [à faire]
      \item $\times$ est associative :  cela découle de l'associativité sur $\Rr\setminus\{0\}$.
      
      \item $x\times y=y\times x$ : .     

    \end{itemize} 
    
  \item $\times$ est distributive par rapport à $+$ : cela découle de la distributivité sur $\Rr$.
 \end{enumerate}
  
  \begin{enumerate}
    \item   
    \item    
  \end{enumerate} 
  
  \item Par définition $\Qq(\sqrt[3]{2})$ est le plus petit corps contenant $\Qq$ et $\sqrt[3]{2}$.
  Mais il est clair que $K = \left\{ a+b\sqrt[3]{2}+c\sqrt[3]{2}^2 \mid a,b,c \in \Qq \right\}$ contient $\sqrt[3]{2}$ (prendre $a=0$, $b=1$, $c=0$) et que $\Qq \subset K$.
  Donc comme $\Qq(\sqrt[3]{2}) \subset K$.
  
  Nous allons maintenant montrer la partie "le plus petit". Soit donc $K'$ un autre corps contenant
  $\Qq$ et $\sqrt[3]{2}$. On veut montrer $K \subset K'$.
  $K'$ doit contenir tout élément $a\in \Qq$ et contient $\sqrt[3]{2}$ donc il contient
  tout élément de la forme $a+b \sqrt[3]{2}$ (avec $a,b \in \Qq$).
  Mais comme $K'$ est un corps contenant $\sqrt[3]{2}$ il contient aussi $\sqrt[3]{2}^2$ et donc tout élément de la forme
  $a+b\sqrt[3]{2}+c\sqrt[3]{2}^2$ avec ($a,b,c \in \Qq$). Ainsi $K \subset K'$.
  
  Conclusion : $K$ est bien le plus petit corps contenant  $\Qq$ et $\sqrt[3]{2}$, c'est-à-dire
  $K = \Qq(\sqrt[3]{2})$.
  
  
  \item
  \begin{enumerate}
    \item Le degré algébrique d'un réel $x$ est le plus petit degré 
    d'un polynôme $P\in\Qq[X]$ tel que $P(x)=0$.
    Pour $x = \sqrt[3]{2}$, un polynôme qui annule $x$ est $P(X) = X^3-2$.
    Donc le degré algébrique de $x$ est $\le 3$. 
    Mais le nombre algébrique de degré $1$ sont les rationnels, alors que le nombre réels
    algébrique de degré $2$ sont de la forme $a+b \sqrt{\delta}$ (avec $a,b,\delta\in\Qq$).
    Donc $x = \sqrt[3]{2}$ n'est ni de degré $1$, ni de degré $2$. Ainsi le degré algébrique de
    $x = \sqrt[3]{2}$ est $3$.
    
    \item On a vu que $K=\Qq(\sqrt[3]{2}) = \left\{ a+b\sqrt[3]{2}+c\sqrt[3]{2}^2 \mid a,b,c \in \Qq \right\}$
    c'est donc un $\Qq$-espace vectoriel dont une base est $(1,\sqrt[3]{2},\sqrt[3]{2}^2)$.
    C'est donc un espace vectoriel sur $\Qq$ de dimension $3$.
    Ainsi le degré de l'extension $[\Qq(\sqrt[3]{2}):\Qq] = \dim_\Qq K = 3$.
  
  \end{enumerate}  
  C'est un résultat plus général (voir le cours) : le degré algébrique d'un réel $x$
  égale le degré de l'extension $[\Qq(x):\Qq]$.
  
  \item 
  \begin{enumerate}
    \item 
  Premier exemple avec $x = \sqrt{\sqrt{2}} = $, alors
  $$\Qq\big(2^{\frac14}\big) = \big\{ a + b 2^{\frac14} + c 2^{\frac24} + d  2^{\frac34} \mid a,b,c,d \in \Qq \big\}.$$
  C'est une extension de $\Qq$ degré $4$.
    \item
  Second exemple avec $x = \sqrt{2}\ii$, alors
  $$\Qq\big(\sqrt{2}\ii\big) = \big\{ a + b \sqrt{2} + c \ii  + d \sqrt{2}\ii  \mid a,b,c,d \in \Qq   \big\}.$$
  C'est aussi une extension de $\Qq$ de degré $4$.
  \end{enumerate}
\end{enumerate}


\fincorrection
\finexercice


\exercice{}
\enonce[Nombres transcendants]
\ 
\begin{enumerate}
 \item Montrer que l'ensemble des nombres réels algébriques est un ensemble dénombrable.
 \item En déduire l'existence de nombres réels qui ne soient pas algébriques.
\end{enumerate}
\finenonce

\indication
\begin{enumerate}
  \item Un nombre algébrique est par définition
  une racine d'un polynôme $P$ de $\Qq[X]$. 
  
  \item $\Rr$ n'est pas dénombrable.
\end{enumerate}


\finindication

\correction
\begin{enumerate}
  \item Un nombre $x \in \Rr$ est un \emph{nombre algébrique}
  s'il existe un polynôme $P \in \Qq[X]$, tel que $P(x)=0$.
  Pour un degré $n$ fixé, il y a un nombre dénombrable de polynômes
  de degré $n$ à coefficients rationnels. En effet, un tel polynôme s'écrit :
  $$P(X) = a_nX^n + a_{n-1}X^{n-1}+\cdots +a_1X+a_0$$
  Et comme la liste des coefficients forment un sous-ensemble de $\Qq^{n+1}$, 
  c'est un ensemble  dénombrable.
  Chaque polynôme de degré $n$, admet au plus $n$ racines. Donc 
  l'ensemble des racines des polynômes de degré $n$ est un ensemble dénombrable.
  
  Enfin on décompose l'ensemble $\overline{\Qq}$ des nombres algébriques comme
  l'union (sur l'indice $n$) des racines des polynômes de degré $n$ :
  $$\overline{\Qq} = \bigcup_{n=0}^{+\infty} \Big\{ x \in \Rr \mid 
  \exists Q \in \Qq[X] \text{ avec } \deg Q = n \text{ et } Q(x)=0 \Big\}$$
  Ainsi $\overline{\Qq}$ est l'union dénombrable d'ensembles dénombrables, ce qui implique
  que $\overline{\Qq}$ est un ensemble dénombrable.
  
  \item $\overline{\Qq}$ est un sous-ensemble dénombrable de $\Rr$, qui lui n'est pas dénombrable.
  Ainsi $\Rr \setminus \overline{\Qq}$ est un ensemble non dénombrable, 
  et en particulier non vide !
\end{enumerate}

Remarque : un nombre qui n'est pas algébrique s'appelle un \emph{nombre transcendant}.
Notre démonstration prouve l'existence de nombres transcendants, mais ne nous permet 
pas d'en expliciter un seul. Par exemples $\pi$, $e$ sont des nombres transcendants.

\fincorrection

\finexercice


%%%%%%%%%%%%%%%%%%%%%%%%%%%%%%%%%%%%%%%%%%%%%%%%%%%%%%%%%%%%%%%%
\section{Corps et nombres constructibles}

\exercice{}
\enonce[Corps stables par racine carrée]
Un corps $K \subset \Rr$ est \emph{stable par racine carrée} s'il vérifie la propriété suivante:
$$\forall x \in K \qquad   x  \ge 0 \Rightarrow \sqrt x \in K.$$

Montrer que l'ensemble $\mathcal{C}_\Rr$ des nombres constructibles est 
le plus petit sous-corps de $\Rr$ stable par racine carrée.
\finenonce

\indication
Utiliser le théorème de Wantzel.
\finindication

\correction
Il y a une étape facile : le sous-corps $\mathcal{C}_\Rr$ est stable par racine carrée.
En effet on a vu que si $x \in \mathcal{C}_\Rr$ est un nombre constructible, alors
$\sqrt x$ est aussi constructible, donc $\sqrt x \in \mathcal{C}_\Rr$.

\bigskip

Soit maintenant $K$ un corps stable par racine carrée, il s'agit de montrer que
$\mathcal{C}_\Rr \subset K$.
Par le théorème de Wantzel, si $x \in \mathcal{C}_\Rr$ est un nombre constructible alors
il existe une suite d'extensions quadratiques :
$$\Qq = K_0 \subset K_1 \subset \cdots \subset K_r$$
telles que $x \in K_r$.

\bigskip

Montrons par récurrence que $K_i \subset K$.
\begin{itemize}
  \item Comme $K_0 = \Qq$ alors $K_0 \subset K$.
  
  \item Supposons $K_{i-1} \subset K$. Le fait que $K_i$ est une extension quadratique de 
  $K_{i-1}$, signifie qu'il existe $\delta \in K_{i-1}$ tel que  $K_i = K_{i-1}(\sqrt{\delta})$.
  Pour $x \in K_i$ alors il existe $a,b \in K_{i-1}$ tels que $x = a + b \sqrt{\delta}$.
  Par hypothèse de récurrence $a,b \in K_{i-1}$, donc $a,b \in K$.
  Mais on sait aussi que $\delta \in K_{i-1}$, donc $\delta \in K$. Par hypothèse
  $K$ est stable par racine carrée donc $\sqrt\delta \in K$.
  Ainsi $a,b, \sqrt\delta \in K$, donc $x  = a + b \sqrt{\delta} \in K$.
  Bilan : $K_i \subset K$.
  
  \item Par le principe de récurrence, on a prouvé $K_r \subset K$.
\end{itemize}

\bigskip


C'est presque terminé. Pour $x \in \mathcal{C}_\Rr$, le théorème de Wantzel, nous a
dit $x \in K_r$, et par ce que l'on vient de prouver $x \in K$.

\bigskip

Conclusion : $\mathcal{C}_\Rr$ est corps stable par racine carrée et tout autre corps
stable par racine carrée contient $\mathcal{C}_\Rr$.
Ainsi $\mathcal{C}_\Rr$ est bien  
le plus petit sous-corps de $\Rr$ stable par racine carrée.
\fincorrection

\finexercice



%%%%%%%%%%%%%%%%%%%%%%%%%%%%%%%%%%%%%%%%%%%%%%%%%%%%%%%%%%%%%%%%
\section{Applications aux problèmes grecs}



\exercice{}
\enonce[Trissection des angles]
Le but est de montrer que tous les angles ne sont pas \emph{trissectables} (divisibles en trois) à la règle et au compas.
Nous allons le prouver pour l'angle $\frac \pi 3$ : plus précisément le point de coordonnées 
$(\cos \frac \pi 3,\sin \frac \pi 3)$ est constructible mais le point $(\cos \frac \pi 9,\sin \frac \pi 9)$ ne l'est pas.
\begin{enumerate}
 \item Exprimer $\cos 3\theta$ en fonction de $\cos \theta$.
 \item Soit $P(X) = X^3-\frac34X-\frac 18$. Montrer que $P(\cos \frac\pi 9) = 0$.
 Montrer que $P(X)$ est irréductible dans $\Qq[X]$.
 \item Conclure.
\end{enumerate}
\finenonce

\indication
\begin{enumerate}
  \item \`A l'aide des nombres complexes, calculer $(e^{\ii\theta})^3$ de deux façons.
  \item Tout d'abord montrer que s'il était réductible alors il aurait racine dans $\Qq$.
Si $\frac ab$ est cette racine avec $\pgcd(a,b)=1$ alors à partir de $P(\frac ab)=0$ obtenir une équation d'entiers.
  \item Utiliser le théorème de Wantzel.
\end{enumerate}
\finindication

\finexercice



%%%%%%%%%%%%%%%%%%%%%%%%%%%%%%%%%%%%%%%%%%%%%%%%%%%%%%%%%%%%%%%%
\section{Constructions assistées}

\exercice{}
\enonce[Spirale d'Archimède]
Soit $(\mathcal{S})$ la \emph{spirale d'Archimède} paramétrée par 
$$M_t = (t \cos(2\pi t),t \sin(2\pi t)), \quad t \ge 0.$$
\begin{enumerate}
 \item Tracer $\mathcal{S}$.
 
 \item \emph{Trissection des angles}. \'Etant tracée la spirale d'Archimède, construire avec la règle et le compas la  trissection d'un angle donné. 
 \emph{Indications.} Supposer l'angle $\theta < \frac \pi 2$. \'Ecrire l'angle sous la forme $\theta = 2\pi t$ ; placer $M_t$ et tracer un cercle centré à l'origine du bon rayon.
 
 \item \emph{Quadrature du cercle.}
   \begin{enumerate}
     \item Calculer une équation de la tangente à $(\mathcal{S})$ en un point $M_t$.
     \item La tangente en $M_1$ (pour $t=1$) coupe l'axe des ordonnées en $N$. Calculer la longueur $ON$.
     \item En déduire qu'avec le tracé de la spirale d'Archimède et le tracé de la tangente en $M_1$ on peut résoudre la quadrature du cercle à la règle et au compas.
   \end{enumerate}
\end{enumerate}
\finenonce

\finexercice


\exercice{}
\enonce[La règle tournante]
On souhaite trissecter les angles à l'aide d'un compas et d'une règle graduée.
Soit $\mathcal{C}$ le cercle de rayon $r>0$ centré en $O$.
Soient $A, B \in \mathcal{C}$ de telle sorte que 
l'angle en $O$ d'un triangle $OAB$ soit aigu. Notons $\theta$ cet angle.
\emph{Sur la règle} marquer deux points $O'$ et $C$ tel que $O'C=r$.
Faire pivoter et glisser la règle autour du point $B$ afin que 
le point $C$ appartienne à $\mathcal{C}$ et le point $O'$ appartiennent 
à la droite $(OA)$ (de sorte que $O$ soit dans le segment $[O'A]$).
Montrer que l'angle en $O'$ du triangle $AO'B$ vaut $\frac{\theta}{3}$.


   \begin{tikzpicture}[scale=2]
      \def\maincircle{(0,0) circle (1)};
      \coordinate (o) at (0,0);
      \coordinate (a) at (1,0);      
      \coordinate (b) at (60:1);
      \coordinate (c) at (160:1);      
      \coordinate (oo) at (-1.8793,0);
      
      \node at (o) [below] {$O$};
      \node at (a) [right] {$A$};      
      \node at (b) [above right] {$B$};
      \node at (c) [above left] {$C$};
      \node at (oo) [below] {$O'$};
      
      \draw (o)--(b);
      \draw (o)--(c);
      \draw (oo)--(b);
      \draw (oo)--(a);
  
      \draw \maincircle;

      \draw (0.2,0) arc(0:60:0.2);
      \draw (-1.6793,0) arc (0:20:0.2);
      \node at (0.3,0.1) {$\theta$};
      \node at (-1.7,0.2) {$\theta/3$};      
   \end{tikzpicture}

\finenonce
\finexercice



\exercice{}
\enonce[Cissoïde de Dioclès]
Soient les points $O(0,0)$ et $I(1,0)$.
Soit $\mathcal{C}$ le cercle de diamètre $[OI]$
et $\mathcal{L}$ la droite d'équation $(x=1)$.
Pour un point $M$ de $\mathcal{C}$, soit $M'$ l'intersection de $(OM)$ avec $\mathcal{L}$.
Soit enfin $M''$ le point tel que $\overrightarrow{OM''}= \overrightarrow{MM'}$.
L'ensemble des points $M''$ lorsque $M$ parcourt $\mathcal{C}$ est \emph{la cissoïde
de Dioclès}, notée $\mathcal{D}$.

Le but de l'exercice est de montrer que la duplication du cube est possible à l'aide de la règle,
du compas et du tracé de la cissoïde.

\begin{enumerate}
 \item La droite $(OM)$ ayant pour équation $(y=tx)$, exprimer les coordonnées de $M$, $M'$ puis $M''$ en fonction de $t$.
% M(1/(1+t^2), t*x)   M'(1,t)  M'' (1-1/1+t^2, t*x)
\item En déduire une équation paramétrique de $\mathcal{D}$ :
 $$x(t) = \frac{t^2}{1+t^2}, \quad y(t) = \frac{t^3}{1+t^2}.$$

\item Montrer qu'une équation cartésienne de $\mathcal{D}$ est :
 $$x(x^2+y^2)-y^2 = 0.$$
 
\item \'Etudier et tracer $\mathcal{D}$.
 
\item Soit $P(0,2)$. La droite $(PI)$ coupe $\mathcal{D}$ en un point noté $Q$.
La droite $(OQ)$ coupe $\mathcal{L}$ en un point noté $R$. Calculer une équation de $(PI)$ ainsi que 
les coordonnées de $Q$ et $R$.
% R(1, \sqrt[3]{2})

\item Conclure.
\end{enumerate}

\shorthandoff{:}
\begin{center}
\begin{tikzpicture}[scale=3]
  \draw[domain=-2:2,samples=100, thick] plot ({(\x*\x/(1+\x*\x))},{(\x*\x*\x/(1+\x*\x))});
  \draw (0.5,0) circle (0.5);
 % \draw (0,-1.5)--(0,1.5);
  \draw (1,-1.7)--(1,1.7); 
  
  \draw (0,0) node[below left] {$O$} -- (1.5,1.1);
  \node at (1.2,0.7) [left] {$M'$};
  \node at (0.6,0.25) [left] {$M''$}; 
  \node at (0.8,0.6) [left] {$M$};
  \node at (1,1.5) [right] {$\mathcal{L}$};
  \node at (1,0) [right] {$I$};  
  \node at (0.1,0.5) {$\mathcal{C}$};
\end{tikzpicture}
\hspace*{3cm}
\begin{tikzpicture}[scale=2.5]
  \draw[domain=-2:2.8,samples=100, thick] plot ({(\x*\x/(1+\x*\x))},{(\x*\x*\x/(1+\x*\x))});
  \draw (0.5,0) circle (0.5);
  \draw (0,-1.5)--(0,2.5);
  \draw (1,-1.5)--(1,2.5);   
  \draw (0,2)--(1,0);
    \draw (0,0) node[left] {$O(0,0)$} -- (1.5,1.5*1.2599);
  \node at (1.1,1.2599) [right] {$R$};
  \node at (0.6,0.8) [left] {$Q$};
  \node at (1,2) [right] {$\mathcal{L}$};
  \node at (0,2) [left] {$P(0,2)$};
  \node at (1,0) [right] {$I(1,0)$};
\end{tikzpicture}
\end{center}
\shorthandon{:}
\finenonce

\finexercice


\exercice{}
\enonce[Lunules d'Hippocrate de Chios]
Montrer que l'aire des quatre lunules égale l'aire du carré. 

\begin{center}
   \begin{tikzpicture}[scale=0.5]
      \def\maincircle{(0,0) circle (1.4142*2)} 
      \begin{scope}
        \begin{scope}[even odd rule]% first circle without the second
            \clip \maincircle (-5,-5) rectangle (5,5);
            \fill[orange] (0,2) circle (2);
            \fill[orange] (0,-2) circle (2);
            \fill[orange] (2,0) circle (2);
            \fill[orange] (-2,0) circle (2);
        \end{scope}
      \end{scope}

      \draw (-2,2) rectangle (2,-2);
      \draw \maincircle;
      \draw (2,2) arc(0:180:2);
      \draw (-2,2) arc(90:270:2);
      \draw (-2,-2) arc(180:360:2);
      \draw (2,2) arc(90:-90:2);     
   \end{tikzpicture}
\end{center}
\finenonce

\indication
C'est un simple calcul d'aires : calculer l'aire totale formée 
par la figure de deux façons différentes.
\finindication

\correction

Notons $\mathcal{A}$ l'aire totale formée par la figure.
On décompose la figure totale de deux façon différente.

\begin{center}
 \begin{tikzpicture}[scale=.6]
      \useasboundingbox (0,-5) rectangle (10,5);
      \def\maincircle{(0,0) circle (1.4142*2)}  
      \fill[blue!50] (-2,2) rectangle (2,-2);
      \fill[blue] (2,2) arc(0:180:2);     
      \fill[blue] (-2,2) arc(90:270:2);
      \fill[blue] (-2,-2) arc(180:360:2);     
       \fill[blue] (2,2) arc(90:-90:2);   
      
      \draw (-2,2) rectangle (2,-2);
      \draw \maincircle;
      \draw (2,2) arc(0:180:2);
      \draw (-2,2) arc(90:270:2);
      \draw (-2,-2) arc(180:360:2);
      \draw (2,2) arc(90:-90:2);     
 \end{tikzpicture} 
 \begin{tikzpicture}[scale=.6]
      \useasboundingbox (0,-5) rectangle (0,5);
      \def\maincircle{(0,0) circle (1.4142*2)}
      
      \begin{scope}
        \begin{scope}[even odd rule]% first circle without the second
            \clip \maincircle (-5,-5) rectangle (5,5);
            \fill[green!60!black] (0,2) circle (2);
            \fill[green!60!black] (0,-2) circle (2);
            \fill[green!60!black] (2,0) circle (2);
            \fill[green!60!black] (-2,0) circle (2);
        \end{scope}
      \end{scope}
      
      \fill[green!60] \maincircle;
      \draw (-2,2) rectangle (2,-2);
      \draw \maincircle;
      \draw (2,2) arc(0:180:2);
      \draw (-2,2) arc(90:270:2);
      \draw (-2,-2) arc(180:360:2);
      \draw (2,2) arc(90:-90:2);     
   \end{tikzpicture}
\end{center}

Première façon, sur la figure de gauche.
L'aire totale $\mathcal{A}$ se décompose en l'aire $\mathcal{A}_\text{carré}$ du carré
(zone bleu clair, dont on veut calculer l'aire)
et l'aire $\mathcal{A}_\text{demi-disques}$ formée par $4$ demi-disques (zone bleu foncé) :
$$\mathcal{A} = \mathcal{A}_\text{carré} + \mathcal{A}_\text{demi-disques}$$
Si on note $a$ la longueur d'un des côtés du carré alors 
$$\mathcal{A}_\text{demi-disques} = 4 \times \frac12 \times \pi \left(\frac{a}{2}\right)^2 = \frac{\pi a^2}{2}$$


\begin{center}
   \begin{tikzpicture}[scale=0.5]
      \def\maincircle{(0,0) circle (1.4142*2)} 
      \begin{scope}
        \begin{scope}[even odd rule]% first circle without the second
            \clip \maincircle (-5,-5) rectangle (5,5);
            \fill[orange] (0,2) circle (2);
            \fill[orange] (0,-2) circle (2);
            \fill[orange] (2,0) circle (2);
            \fill[orange] (-2,0) circle (2);
        \end{scope}
      \end{scope}

      \draw (-2,2) rectangle (2,-2);
      \draw \maincircle;
      \draw (2,2) arc(0:180:2);
      \draw (-2,2) arc(90:270:2);
      \draw (-2,-2) arc(180:360:2);
      \draw (2,2) arc(90:-90:2);
      
      \draw[very thick, blue] (2,-2)--(2,2);
      \node at (2,0) [left] {$a$};
      \draw[very thick, blue] (0,0)--(45:2.82);
      \node at (1.1,1) [left] {$r$};      
   \end{tikzpicture}
\end{center}

Deuxième façon, sur la figure de droite.
L'aire totale $\mathcal{A}$ se décompose cette fois en l'aire $\mathcal{A}_\text{disque}$ 
du disque (zone vert clair)
et l'aire $\mathcal{A}_\text{lunules}$ formée par $4$ lunules (zone vert foncé, dont 
on veut aussi calculer l'aire) :
$$\mathcal{A} = \mathcal{A}_\text{disque} + \mathcal{A}_\text{lunules}$$

Le rayon du cercle est $r = \frac{\sqrt{2}}{2}a$.
Ainsi 
$$\mathcal{A}_\text{disque} = \pi r^2 = \pi \left(\frac{\sqrt{2}}{2}a\right)^2 = \frac{\pi a^2}{2}$$

\begin{center}
   \begin{tikzpicture}[scale=0.5]
      \def\maincircle{(0,0) circle (1.4142*2)} 
      \begin{scope}
        \begin{scope}[even odd rule]% first circle without the second
            \clip \maincircle (-5,-5) rectangle (5,5);
            \fill[gray] (0,2) circle (2);
            \fill[gray] (0,-2) circle (2);
            \fill[gray] (2,0) circle (2);
            \fill[gray] (-2,0) circle (2);
        \end{scope}
      \end{scope}

      \draw (-2,2) rectangle (2,-2);
      \draw \maincircle;
      \draw (2,2) arc(0:180:2);
      \draw (-2,2) arc(90:270:2);
      \draw (-2,-2) arc(180:360:2);
      \draw (2,2) arc(90:-90:2);
      
      \draw[very thick, blue] (2,-2)--(2,2);
      \node at (2,0) [left] {$a$};
      \draw[very thick, blue] (0,0)--(45:2.82);
      \node at (1.1,1) [left] {$r$};      
   \end{tikzpicture}
\end{center}

Conséquence $\mathcal{A}_\text{demi-disques} = \mathcal{A}_\text{disque}$
donc
$$\mathcal{A}_\text{lunules} 
= \mathcal{A} - \mathcal{A}_\text{disque}
= \mathcal{A} - \mathcal{A}_\text{demi-disques}
= \mathcal{A}_\text{carré}$$
L'aire des lunules égale l'aire du carrée ! C'est une variante résoluble de la quadrature du cercle.

\fincorrection

\finexercice

\end{document}

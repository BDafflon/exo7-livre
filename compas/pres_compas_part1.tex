
%%%%%%%%%%%%%%%%%% PREAMBULE %%%%%%%%%%%%%%%%%%

\documentclass[aspectratio=169,utf8]{beamer}
%\documentclass[aspectratio=169,handout]{beamer}

\usetheme{Boadilla}
%\usecolortheme{seahorse}
%\usecolortheme[RGB={245,66,24}]{structure}
\useoutertheme{infolines}

% packages
\usepackage{amsfonts,amsmath,amssymb,amsthm}
\usepackage[utf8]{inputenc}
\usepackage[T1]{fontenc}
\usepackage{lmodern}

\usepackage[francais]{babel}
\usepackage{fancybox}
\usepackage{graphicx}

\usepackage{float}
\usepackage{xfrac}

%\usepackage[usenames, x11names]{xcolor}
\usepackage{pgfplots}
\usepackage{datetime}


% ----------------------------------------------------------------------
% Pour les images
\usepackage{tikz}
\usetikzlibrary{calc,shadows,arrows.meta,patterns,matrix}

\newcommand{\tikzinput}[1]{\input{figures/#1.tikz}}
% --- les figures avec échelle éventuel
\newcommand{\myfigure}[2]{% entrée : échelle, fichier(s) figure à inclure
\begin{center}\small%
\tikzstyle{every picture}=[scale=1.0*#1]% mise en échelle + 0% (automatiquement annulé à la fin du groupe)
#2%
\end{center}}



%-----  Package unités -----
\usepackage{siunitx}
\sisetup{locale = FR,detect-all,per-mode = symbol}

%\usepackage{mathptmx}
%\usepackage{fouriernc}
%\usepackage{newcent}
%\usepackage[mathcal,mathbf]{euler}

%\usepackage{palatino}
%\usepackage{newcent}
% \usepackage[mathcal,mathbf]{euler}



% \usepackage{hyperref}
% \hypersetup{colorlinks=true, linkcolor=blue, urlcolor=blue,
% pdftitle={Exo7 - Exercices de mathématiques}, pdfauthor={Exo7}}


%section
% \usepackage{sectsty}
% \allsectionsfont{\bf}
%\sectionfont{\color{Tomato3}\upshape\selectfont}
%\subsectionfont{\color{Tomato4}\upshape\selectfont}

%----- Ensembles : entiers, reels, complexes -----
\newcommand{\Nn}{\mathbb{N}} \newcommand{\N}{\mathbb{N}}
\newcommand{\Zz}{\mathbb{Z}} \newcommand{\Z}{\mathbb{Z}}
\newcommand{\Qq}{\mathbb{Q}} \newcommand{\Q}{\mathbb{Q}}
\newcommand{\Rr}{\mathbb{R}} \newcommand{\R}{\mathbb{R}}
\newcommand{\Cc}{\mathbb{C}} 
\newcommand{\Kk}{\mathbb{K}} \newcommand{\K}{\mathbb{K}}

%----- Modifications de symboles -----
\renewcommand{\epsilon}{\varepsilon}
\renewcommand{\Re}{\mathop{\text{Re}}\nolimits}
\renewcommand{\Im}{\mathop{\text{Im}}\nolimits}
%\newcommand{\llbracket}{\left[\kern-0.15em\left[}
%\newcommand{\rrbracket}{\right]\kern-0.15em\right]}

\renewcommand{\ge}{\geqslant}
\renewcommand{\geq}{\geqslant}
\renewcommand{\le}{\leqslant}
\renewcommand{\leq}{\leqslant}
\renewcommand{\epsilon}{\varepsilon}

%----- Fonctions usuelles -----
\newcommand{\ch}{\mathop{\text{ch}}\nolimits}
\newcommand{\sh}{\mathop{\text{sh}}\nolimits}
\renewcommand{\tanh}{\mathop{\text{th}}\nolimits}
\newcommand{\cotan}{\mathop{\text{cotan}}\nolimits}
\newcommand{\Arcsin}{\mathop{\text{arcsin}}\nolimits}
\newcommand{\Arccos}{\mathop{\text{arccos}}\nolimits}
\newcommand{\Arctan}{\mathop{\text{arctan}}\nolimits}
\newcommand{\Argsh}{\mathop{\text{argsh}}\nolimits}
\newcommand{\Argch}{\mathop{\text{argch}}\nolimits}
\newcommand{\Argth}{\mathop{\text{argth}}\nolimits}
\newcommand{\pgcd}{\mathop{\text{pgcd}}\nolimits} 


%----- Commandes divers ------
\newcommand{\ii}{\mathrm{i}}
\newcommand{\dd}{\text{d}}
\newcommand{\id}{\mathop{\text{id}}\nolimits}
\newcommand{\Ker}{\mathop{\text{Ker}}\nolimits}
\newcommand{\Card}{\mathop{\text{Card}}\nolimits}
\newcommand{\Vect}{\mathop{\text{Vect}}\nolimits}
\newcommand{\Mat}{\mathop{\text{Mat}}\nolimits}
\newcommand{\rg}{\mathop{\text{rg}}\nolimits}
\newcommand{\tr}{\mathop{\text{tr}}\nolimits}


%----- Structure des exercices ------

\newtheoremstyle{styleexo}% name
{2ex}% Space above
{3ex}% Space below
{}% Body font
{}% Indent amount 1
{\bfseries} % Theorem head font
{}% Punctuation after theorem head
{\newline}% Space after theorem head 2
{}% Theorem head spec (can be left empty, meaning ‘normal’)

%\theoremstyle{styleexo}
\newtheorem{exo}{Exercice}
\newtheorem{ind}{Indications}
\newtheorem{cor}{Correction}


\newcommand{\exercice}[1]{} \newcommand{\finexercice}{}
%\newcommand{\exercice}[1]{{\tiny\texttt{#1}}\vspace{-2ex}} % pour afficher le numero absolu, l'auteur...
\newcommand{\enonce}{\begin{exo}} \newcommand{\finenonce}{\end{exo}}
\newcommand{\indication}{\begin{ind}} \newcommand{\finindication}{\end{ind}}
\newcommand{\correction}{\begin{cor}} \newcommand{\fincorrection}{\end{cor}}

\newcommand{\noindication}{\stepcounter{ind}}
\newcommand{\nocorrection}{\stepcounter{cor}}

\newcommand{\fiche}[1]{} \newcommand{\finfiche}{}
\newcommand{\titre}[1]{\centerline{\large \bf #1}}
\newcommand{\addcommand}[1]{}
\newcommand{\video}[1]{}

% Marge
\newcommand{\mymargin}[1]{\marginpar{{\small #1}}}

\def\noqed{\renewcommand{\qedsymbol}{}}


%----- Presentation ------
\setlength{\parindent}{0cm}

%\newcommand{\ExoSept}{\href{http://exo7.emath.fr}{\textbf{\textsf{Exo7}}}}

\definecolor{myred}{rgb}{0.93,0.26,0}
\definecolor{myorange}{rgb}{0.97,0.58,0}
\definecolor{myyellow}{rgb}{1,0.86,0}

\newcommand{\LogoExoSept}[1]{  % input : echelle
{\usefont{U}{cmss}{bx}{n}
\begin{tikzpicture}[scale=0.1*#1,transform shape]
  \fill[color=myorange] (0,0)--(4,0)--(4,-4)--(0,-4)--cycle;
  \fill[color=myred] (0,0)--(0,3)--(-3,3)--(-3,0)--cycle;
  \fill[color=myyellow] (4,0)--(7,4)--(3,7)--(0,3)--cycle;
  \node[scale=5] at (3.5,3.5) {Exo7};
\end{tikzpicture}}
}


\newcommand{\debutmontitre}{
  \author{} \date{} 
  \thispagestyle{empty}
  \hspace*{-10ex}
  \begin{minipage}{\textwidth}
    \titlepage  
  \vspace*{-2.5cm}
  \begin{center}
    \LogoExoSept{2.5}
  \end{center}
  \end{minipage}

  \vspace*{-0cm}
  
  % Astuce pour que le background ne soit pas discrétisé lors de la conversion pdf -> png
\begin{tikzpicture}
        \fill[opacity=0,green!60!black] (0,0)--++(0,0)--++(0,0)--++(0,0)--cycle; 
\end{tikzpicture}

% toc S'affiche trop tot :
% \tableofcontents[hideallsubsections, pausesections]
}

\newcommand{\finmontitre}{
  \end{frame}
  \setcounter{framenumber}{0}
} % ne marche pas pour une raison obscure

%----- Commandes supplementaires ------

% \usepackage[landscape]{geometry}
% \geometry{top=1cm, bottom=3cm, left=2cm, right=10cm, marginparsep=1cm
% }
% \usepackage[a4paper]{geometry}
% \geometry{top=2cm, bottom=2cm, left=2cm, right=2cm, marginparsep=1cm
% }

%\usepackage{standalone}


% New command Arnaud -- november 2011
\setbeamersize{text margin left=24ex}
% si vous modifier cette valeur il faut aussi
% modifier le decalage du titre pour compenser
% (ex : ici =+10ex, titre =-5ex

\theoremstyle{definition}
%\newtheorem{proposition}{Proposition}
%\newtheorem{exemple}{Exemple}
%\newtheorem{theoreme}{Théorème}
%\newtheorem{lemme}{Lemme}
%\newtheorem{corollaire}{Corollaire}
%\newtheorem*{remarque*}{Remarque}
%\newtheorem*{miniexercice}{Mini-exercices}
%\newtheorem{definition}{Définition}

% Commande tikz
\usetikzlibrary{calc}
\usetikzlibrary{patterns,arrows}
\usetikzlibrary{matrix}
\usetikzlibrary{fadings} 

%definition d'un terme
\newcommand{\defi}[1]{{\color{myorange}\textbf{\emph{#1}}}}
\newcommand{\evidence}[1]{{\color{blue}\textbf{\emph{#1}}}}
\newcommand{\assertion}[1]{\emph{\og#1\fg}}  % pour chapitre logique
%\renewcommand{\contentsname}{Sommaire}
\renewcommand{\contentsname}{}
\setcounter{tocdepth}{2}



%------ Encadrement ------

\usepackage{fancybox}


\newcommand{\mybox}[1]{
\setlength{\fboxsep}{7pt}
\begin{center}
\shadowbox{#1}
\end{center}}

\newcommand{\myboxinline}[1]{
\setlength{\fboxsep}{5pt}
\raisebox{-10pt}{
\shadowbox{#1}
}
}

%--------------- Commande beamer---------------
\newcommand{\beameronly}[1]{#1} % permet de mettre des pause dans beamer pas dans poly


\setbeamertemplate{navigation symbols}{}
\setbeamertemplate{footline}  % tiré du fichier beamerouterinfolines.sty
{
  \leavevmode%
  \hbox{%
  \begin{beamercolorbox}[wd=.333333\paperwidth,ht=2.25ex,dp=1ex,center]{author in head/foot}%
    % \usebeamerfont{author in head/foot}\insertshortauthor%~~(\insertshortinstitute)
    \usebeamerfont{section in head/foot}{\bf\insertshorttitle}
  \end{beamercolorbox}%
  \begin{beamercolorbox}[wd=.333333\paperwidth,ht=2.25ex,dp=1ex,center]{title in head/foot}%
    \usebeamerfont{section in head/foot}{\bf\insertsectionhead}
  \end{beamercolorbox}%
  \begin{beamercolorbox}[wd=.333333\paperwidth,ht=2.25ex,dp=1ex,right]{date in head/foot}%
    % \usebeamerfont{date in head/foot}\insertshortdate{}\hspace*{2em}
    \insertframenumber{} / \inserttotalframenumber\hspace*{2ex} 
  \end{beamercolorbox}}%
  \vskip0pt%
}


\definecolor{mygrey}{rgb}{0.5,0.5,0.5}
\setlength{\parindent}{0cm}
%\DeclareTextFontCommand{\helvetica}{\fontfamily{phv}\selectfont}

% background beamer
\definecolor{couleurhaut}{rgb}{0.85,0.9,1}  % creme
\definecolor{couleurmilieu}{rgb}{1,1,1}  % vert pale
\definecolor{couleurbas}{rgb}{0.85,0.9,1}  % blanc
\setbeamertemplate{background canvas}[vertical shading]%
[top=couleurhaut,middle=couleurmilieu,midpoint=0.4,bottom=couleurbas] 
%[top=fondtitre!05,bottom=fondtitre!60]



\makeatletter
\setbeamertemplate{theorem begin}
{%
  \begin{\inserttheoremblockenv}
  {%
    \inserttheoremheadfont
    \inserttheoremname
    \inserttheoremnumber
    \ifx\inserttheoremaddition\@empty\else\ (\inserttheoremaddition)\fi%
    \inserttheorempunctuation
  }%
}
\setbeamertemplate{theorem end}{\end{\inserttheoremblockenv}}

\newenvironment{theoreme}[1][]{%
   \setbeamercolor{block title}{fg=structure,bg=structure!40}
   \setbeamercolor{block body}{fg=black,bg=structure!10}
   \begin{block}{{\bf Th\'eor\`eme }#1}
}{%
   \end{block}%
}


\newenvironment{proposition}[1][]{%
   \setbeamercolor{block title}{fg=structure,bg=structure!40}
   \setbeamercolor{block body}{fg=black,bg=structure!10}
   \begin{block}{{\bf Proposition }#1}
}{%
   \end{block}%
}

\newenvironment{corollaire}[1][]{%
   \setbeamercolor{block title}{fg=structure,bg=structure!40}
   \setbeamercolor{block body}{fg=black,bg=structure!10}
   \begin{block}{{\bf Corollaire }#1}
}{%
   \end{block}%
}

\newenvironment{mydefinition}[1][]{%
   \setbeamercolor{block title}{fg=structure,bg=structure!40}
   \setbeamercolor{block body}{fg=black,bg=structure!10}
   \begin{block}{{\bf Définition} #1}
}{%
   \end{block}%
}

\newenvironment{lemme}[0]{%
   \setbeamercolor{block title}{fg=structure,bg=structure!40}
   \setbeamercolor{block body}{fg=black,bg=structure!10}
   \begin{block}{\bf Lemme}
}{%
   \end{block}%
}

\newenvironment{remarque}[1][]{%
   \setbeamercolor{block title}{fg=black,bg=structure!20}
   \setbeamercolor{block body}{fg=black,bg=structure!5}
   \begin{block}{Remarque #1}
}{%
   \end{block}%
}


\newenvironment{exemple}[1][]{%
   \setbeamercolor{block title}{fg=black,bg=structure!20}
   \setbeamercolor{block body}{fg=black,bg=structure!5}
   \begin{block}{{\bf Exemple }#1}
}{%
   \end{block}%
}


\newenvironment{miniexercice}[0]{%
   \setbeamercolor{block title}{fg=structure,bg=structure!20}
   \setbeamercolor{block body}{fg=black,bg=structure!5}
   \begin{block}{Mini-exercices}
}{%
   \end{block}%
}


\newenvironment{tp}[0]{%
   \setbeamercolor{block title}{fg=structure,bg=structure!40}
   \setbeamercolor{block body}{fg=black,bg=structure!10}
   \begin{block}{\bf Travaux pratiques}
}{%
   \end{block}%
}
\newenvironment{exercicecours}[1][]{%
   \setbeamercolor{block title}{fg=structure,bg=structure!40}
   \setbeamercolor{block body}{fg=black,bg=structure!10}
   \begin{block}{{\bf Exercice }#1}
}{%
   \end{block}%
}
\newenvironment{algo}[1][]{%
   \setbeamercolor{block title}{fg=structure,bg=structure!40}
   \setbeamercolor{block body}{fg=black,bg=structure!10}
   \begin{block}{{\bf Algorithme}\hfill{\color{gray}\texttt{#1}}}
}{%
   \end{block}%
}


\setbeamertemplate{proof begin}{
   \setbeamercolor{block title}{fg=black,bg=structure!20}
   \setbeamercolor{block body}{fg=black,bg=structure!5}
   \begin{block}{{\footnotesize Démonstration}}
   \footnotesize
   \smallskip}
\setbeamertemplate{proof end}{%
   \end{block}}
\setbeamertemplate{qed symbol}{\openbox}


\makeatother
\usecolortheme[RGB={102,102,255}]{structure}

% Commande spécifique à ce chapitre
\newcommand{\construc}{\mathcal{C}}
\newcommand{\plan}{\mathcal{P}}
\newcommand{\cercle}{\mathcal{C}}

% Spiral of Theodorus  tex.stackexchange.com  "Manuel"
%\tikzinput{fig_compas18} 
%%%%%%%%%%%%%%%%%%%%%%%%%%%%%%%%%%%%%%%%%%%%%%%%%%%%%%%%%%%%%
%%%%%%%%%%%%%%%%%%%%%%%%%%%%%%%%%%%%%%%%%%%%%%%%%%%%%%%%%%%%%


\begin{document}


\title{{\bf La règle et le compas}}
\subtitle{Constructions et les trois problèmes grecs}

\begin{frame}
  
  \debutmontitre

  \pause

{\footnotesize
\hfill
\setbeamercovered{transparent=50}
\begin{minipage}{0.6\textwidth}
  \begin{itemize}
    \item<3-> Premières constructions géométriques
%    \item<4-> Règles du jeu
%    \item<5-> Conserver l'écartement du compas
%    \item<5-> Thalès et Pythagore
    \item<4-> La trisection des angles
    \item<5-> La duplication du cube
    \item<6-> La quadrature du cercle
  \end{itemize}
\end{minipage}
}

\end{frame}

\setcounter{framenumber}{0}


%%%%%%%%%%%%%%%%%%%%%%%%%%%%%%%%%%%%%%%%%%%%%%%%%%%%%%%%%%%%%%%%
\section{Premières constructions géométriques}

\begin{frame}

\begin{itemize} 
  \item Avec un compas et une règle (non graduée)
  
  \uncover<2->{
  \item On se donne deux points $A$ et $B$
  }
  \uncover<3->{
  \item Le \evidence{symétrique} de $B$ par rapport à $A$
  }
  \uncover<7->{
  \item La \evidence{médiatrice} de $[AB]$
  }
  \uncover<12->{
  \item Le \evidence{milieu} de $[AB]$
  }
\end{itemize} 
\vspace*{-5ex}
\myfigure{0.9}{
 \uncover<2->{\tikzinput{fig_compas01-pres}\qquad}
 \uncover<7->{\tikzinput{fig_compas02-pres}}
}
\end{frame}


\begin{frame}
\begin{itemize}
  \item $A, B, C$ trois points donnés
  \uncover<2->{
  \item La \evidence{parallèle} à $(AB)$ passant par $C$
  }
  \begin{itemize} 
    \uncover<3->{\item construire le milieu $I$ de $[AC]$}
    \uncover<4->{\item construire $D$ le symétrique de $B$ par rapport à $I$}
    \uncover<5->{\item $ABCD$ est un \evidence{parallélogramme}}
  \end{itemize}
  

  
  \uncover<6->{\item La \evidence{perpendiculaire} à $(AB)$ passant $C$}
  \begin{itemize}
    \uncover<7->{\item construire la médiatrice de $[AB]$}
    \uncover<8->{\item la parallèle à cette médiatrice passant par $C$}
  \end{itemize}
\end{itemize} 
\vspace*{-3ex} 
\myfigure{1}{
             \tikzinput{fig_compas03-pres}
\uncover<6->{\tikzinput{fig_compas04-pres}}
}

\end{frame}


%%%%%%%%%%%%%%%%%%%%%%%%%%%%%%%%%%%%%%%%%%%%%%%%%%%%%%%%%%%%%%%%
\section{Règles du jeu}

\begin{frame}

\evidence{Règles du jeu}

\`A partir de points déjà construits
\begin{itemize}
  \uncover<2->{\item Vous pouvez tracer une droite entre deux points déjà construits}
  \uncover<4->{\item Vous pouvez tracer un cercle dont le centre est un point construit et qui passe
  par un autre point construit}
  \uncover<6->{\item Vous pouvez utiliser les points obtenus comme intersections de deux droites tracées,
  ou bien intersections d'une droite et d'un cercle tracé, ou bien intersections de deux cercles tracés}
\end{itemize}
\vspace*{-1ex}
\only<1-6>{
\myfigure{0.8}{
\tikzinput{fig_compas05-pres}\qquad
\tikzinput{fig_compas06-pres}
}
}
\only<7->{
\myfigure{0.55}{
\uncover<7->{\tikzinput{fig_compas07-pres}\vspace*{-2em}}
\uncover<11->{\tikzinput{fig_compas08-pres}}
\uncover<15->{\tikzinput{fig_compas09-pres}}
\vspace*{30ex}
}
}
\end{frame}


%%%%%%%%%%%%%%%%%%%%%%%%%%%%%%%%%%%%%%%%%%%%%%%%%%%%%%%%%%%%%%%%
\section{Conserver l'écartement du compas}

\begin{frame}
\evidence{Conserver l'écartement du compas}. 


\begin{itemize}
  \uncover<2->{\item Pour des points $A,B,A'$}
  
  \uncover<3->{\item Mesurer le segment $[AB]$ avec le compas}
  
  \uncover<4->{\item Soulever le compas en gardant l'écartement}
  
  \uncover<5->{\item Tracer le cercle centré en $A'$ et d'écartement $AB$}
\end{itemize}

 
 \myfigure{0.8}{
\tikzinput{fig_compas10-pres}
}

\end{frame}


\begin{frame}

\evidence{Construction facile d'un parallélogramme}
  
 \myfigure{0.8}{
\tikzinput{fig_compas11-pres}
}
\end{frame}



%%%%%%%%%%%%%%%%%%%%%%%%%%%%%%%%%%%%%%%%%%%%%%%%%%%%%%%%%%%%%%%%
\section{Thalès et Pythagore}

\begin{frame}


\evidence{Diviser un segment en $n$ morceaux}

\vspace*{-6ex}
\myfigure{1.2}{
\tikzinput{fig_compas16-pres}
}
\end{frame}


\begin{frame}[fragile]


% Depuis fig_compas18
% Spiral of Theodorus  tex.stackexchange.com  "Manuel"

% Problème à corriger !! L'importation figure 18 ne fonctionne plus

\newcommand*{\sqrtspiral}[2][scale=2]{
    \begin{tikzpicture}[#1]
        \def\sqrtlast{#2};
        \coordinate (A) at (0,0);
        \coordinate (B) at (1cm,0);
        \draw[thick, black] (A) edge node[auto, swap] {1} (B);
        \foreach \n in {1,...,\sqrtlast}{
            \pgfmathtruncatemacro{\currentsqrt}{\n+1};
            \coordinate (C) at ($(B)!1cm!-90:(A)$);           
%             \pgfdeclareradialshading{glow}{\pgfpoint{0cm}{0cm}}{
%                        color(0mm)=(white);
%                       color(3mm)=(white);
%                       color(7mm)=(black);
%                       color(10mm)=(black);
%                       }

%                       \begin{tikzfadingfrompicture}[name=glow fading]
%                               \shade [shading=glow] (0,0) circle (1);
%                       \end{tikzfadingfrompicture}
     %        \draw[thick, black] (A) edge node[fill=white, circle,inner sep=6pt,path fading=glow fading]{$\sqrt{\currentsqrt}$} (C);
            \draw[thick, black] (A) edge node[fill=white, circle,inner sep=6pt]{$\sqrt{\currentsqrt}$} (C);
            \draw[thick, black] (C) edge node[auto] {1} (B);
            \coordinate (w) at ($(B)!4pt!-90:(A)$);
            \coordinate (z) at ($(B)!4pt!0:(A)$);
            \coordinate (t) at ($(w)!4pt!-90:(B)$);
            \draw (w) -- (t) -- (z);
            \coordinate (B) at (C);
       };
    \end{tikzpicture}
}




\myfigure{1}{\hspace*{-5ex}
  \sqrtspiral{1}\pause\quad%\hspace*{-5ex}
  \sqrtspiral{2}\quad\pause
  \sqrtspiral{3}
%  \sqrtspiral[scale=1.5]{9}
}
\end{frame}



%%%%%%%%%%%%%%%%%%%%%%%%%%%%%%%%%%%%%%%%%%%%%%%%%%%%%%%%%%%%%%%%
\section{La trisection des angles}

\begin{frame}

\evidence{Diviser un angle en deux}

% \begin{itemize}
%   \item Un angle $\theta$
%   \item Un point $A$ et de deux demi-droites issues de $A$
% \end{itemize}

\bigskip

\myfigure{1}{
\tikzinput{fig_compas20}\quad
\pause
\tikzinput{fig_compas21-pres}
}

% \begin{itemize}  
%   \item Pour diviser l'angle en deux, tracer la bissectrice
%   \begin{itemize}
%     \item on fixe un écartement de compas
%     \item on trace un cercle centré en $A$, il recoupe les demi-droites en $B$ et $C$
%     \item on trace deux cercles centrés en $B$ puis $C$ de même rayon
%     \item on note $D$ un point de l'intersection de ces deux cercles
%     \item $(AD)$ est la bissectrice de l'angle
%   \end{itemize}
% \end{itemize}

\end{frame}


\begin{frame}
\begin{center}
\shadowbox{
\begin{minipage}{0.7\textwidth}
\center
\smallskip \defi{Problème de la trisection d'un angle} \\ \smallskip
Peut-on diviser un angle donné en trois 
angles égaux à l'aide de la règle et du compas ?
\end{minipage}
}
\end{center}

\myfigure{1}{
\tikzinput{fig_compas22}
}
\end{frame}



%%%%%%%%%%%%%%%%%%%%%%%%%%%%%%%%%%%%%%%%%%%%%%%%%%%%%%%%%%%%%%%%
\section{La duplication du cube}

\begin{frame}
\begin{itemize}
  \item \'Etant donné un carré, construire un carré dont l'aire est le double
  
  \uncover<2->{\item Premier carré de côté $a$}
  
  \uncover<3->{\item Second carré de côté $a\sqrt 2$}
\end{itemize}

\myfigure{1.2}{
\tikzinput{fig_compas23-pres}
}

\end{frame}


\begin{frame}

\begin{itemize}
  \item \'Etant donné un cube, peut-on construire
un second cube dont le volume est le double du premier ?

  \uncover<2->{\item Premier cube de côté $a$}

  \uncover<3->{\item Second cube doit être de côté $a\sqrt[3]{2}$}
\end{itemize}


\myfigure{1}
{
\tikzinput{fig_compas24}\qquad\qquad
\uncover<3->{\tikzinput{fig_compas25}}
}

\uncover<4->{
\begin{center}
\shadowbox{\begin{minipage}{0.7\textwidth}
\center
\smallskip \defi{Problème de la duplication du cube} \\ \smallskip
\'Etant donné un segment de longueur $1$, peut-on 
construire à la règle et au compas 
un segment de longueur $\sqrt[3]{2}$ ?
\end{minipage}}
\end{center}
}
\end{frame}



%%%%%%%%%%%%%%%%%%%%%%%%%%%%%%%%%%%%%%%%%%%%%%%%%%%%%%%%%%%%%%%%
\section{La quadrature du cercle}

\begin{frame}
\begin{center}
\shadowbox{\begin{minipage}{0.7\textwidth}
\center
\smallskip \defi{Problème de la quadrature du cercle}  \\ \smallskip
\'Etant donné un cercle, 
peut-on construire à la règle et au compas 
un carré de même aire ?
\end{minipage}}
\end{center}


\myfigure{1.2}{
\tikzinput{fig_compas26}
}
\pause
Cela revient à construire un segment de longueur $\sqrt{\pi}$ à la règle et au compas, à
partir d'un segment de longueur $1$
\end{frame}


\end{document}

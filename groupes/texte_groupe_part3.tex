
%%%%%%%%%%%%%%%%%% PREAMBULE %%%%%%%%%%%%%%%%%%


\documentclass[12pt]{article}

\usepackage{amsfonts,amsmath,amssymb,amsthm}
\usepackage[utf8]{inputenc}
\usepackage[T1]{fontenc}
\usepackage[francais]{babel}


% packages
\usepackage{amsfonts,amsmath,amssymb,amsthm}
\usepackage[utf8]{inputenc}
\usepackage[T1]{fontenc}
%\usepackage{lmodern}

\usepackage[francais]{babel}
\usepackage{fancybox}
\usepackage{graphicx}

\usepackage{float}

%\usepackage[usenames, x11names]{xcolor}
\usepackage{tikz}
\usepackage{datetime}

\usepackage{mathptmx}
%\usepackage{fouriernc}
%\usepackage{newcent}
\usepackage[mathcal,mathbf]{euler}

%\usepackage{palatino}
%\usepackage{newcent}


% Commande spéciale prompteur

%\usepackage{mathptmx}
%\usepackage[mathcal,mathbf]{euler}
%\usepackage{mathpple,multido}

\usepackage[a4paper]{geometry}
\geometry{top=2cm, bottom=2cm, left=1cm, right=1cm, marginparsep=1cm}

\newcommand{\change}{{\color{red}\rule{\textwidth}{1mm}\\}}

\newcounter{mydiapo}

\newcommand{\diapo}{\newpage
\hfill {\normalsize  Diapo \themydiapo \quad \texttt{[\jobname]}} \\
\stepcounter{mydiapo}}


%%%%%%% COULEURS %%%%%%%%%%

% Pour blanc sur noir :
%\pagecolor[rgb]{0.5,0.5,0.5}
% \pagecolor[rgb]{0,0,0}
% \color[rgb]{1,1,1}



%\DeclareFixedFont{\myfont}{U}{cmss}{bx}{n}{18pt}
\newcommand{\debuttexte}{
%%%%%%%%%%%%% FONTES %%%%%%%%%%%%%
\renewcommand{\baselinestretch}{1.5}
\usefont{U}{cmss}{bx}{n}
\bfseries

% Taille normale : commenter le reste !
%Taille Arnaud
%\fontsize{19}{19}\selectfont

% Taille Barbara
%\fontsize{21}{22}\selectfont

%Taille François
%\fontsize{25}{30}\selectfont

%Taille Pascal
%\fontsize{25}{30}\selectfont

%Taille Laura
%\fontsize{30}{35}\selectfont


%\myfont
%\usefont{U}{cmss}{bx}{n}

%\Huge
%\addtolength{\parskip}{\baselineskip}
}


% \usepackage{hyperref}
% \hypersetup{colorlinks=true, linkcolor=blue, urlcolor=blue,
% pdftitle={Exo7 - Exercices de mathématiques}, pdfauthor={Exo7}}


%section
% \usepackage{sectsty}
% \allsectionsfont{\bf}
%\sectionfont{\color{Tomato3}\upshape\selectfont}
%\subsectionfont{\color{Tomato4}\upshape\selectfont}

%----- Ensembles : entiers, reels, complexes -----
\newcommand{\Nn}{\mathbb{N}} \newcommand{\N}{\mathbb{N}}
\newcommand{\Zz}{\mathbb{Z}} \newcommand{\Z}{\mathbb{Z}}
\newcommand{\Qq}{\mathbb{Q}} \newcommand{\Q}{\mathbb{Q}}
\newcommand{\Rr}{\mathbb{R}} \newcommand{\R}{\mathbb{R}}
\newcommand{\Cc}{\mathbb{C}} 
\newcommand{\Kk}{\mathbb{K}} \newcommand{\K}{\mathbb{K}}

%----- Modifications de symboles -----
\renewcommand{\epsilon}{\varepsilon}
\renewcommand{\Re}{\mathop{\text{Re}}\nolimits}
\renewcommand{\Im}{\mathop{\text{Im}}\nolimits}
%\newcommand{\llbracket}{\left[\kern-0.15em\left[}
%\newcommand{\rrbracket}{\right]\kern-0.15em\right]}

\renewcommand{\ge}{\geqslant}
\renewcommand{\geq}{\geqslant}
\renewcommand{\le}{\leqslant}
\renewcommand{\leq}{\leqslant}

%----- Fonctions usuelles -----
\newcommand{\ch}{\mathop{\mathrm{ch}}\nolimits}
\newcommand{\sh}{\mathop{\mathrm{sh}}\nolimits}
\renewcommand{\tanh}{\mathop{\mathrm{th}}\nolimits}
\newcommand{\cotan}{\mathop{\mathrm{cotan}}\nolimits}
\newcommand{\Arcsin}{\mathop{\mathrm{Arcsin}}\nolimits}
\newcommand{\Arccos}{\mathop{\mathrm{Arccos}}\nolimits}
\newcommand{\Arctan}{\mathop{\mathrm{Arctan}}\nolimits}
\newcommand{\Argsh}{\mathop{\mathrm{Argsh}}\nolimits}
\newcommand{\Argch}{\mathop{\mathrm{Argch}}\nolimits}
\newcommand{\Argth}{\mathop{\mathrm{Argth}}\nolimits}
\newcommand{\pgcd}{\mathop{\mathrm{pgcd}}\nolimits} 

\newcommand{\Card}{\mathop{\text{Card}}\nolimits}
\newcommand{\Ker}{\mathop{\text{Ker}}\nolimits}
\newcommand{\id}{\mathop{\text{id}}\nolimits}
\newcommand{\ii}{\mathrm{i}}
\newcommand{\dd}{\mathrm{d}}
\newcommand{\Vect}{\mathop{\text{Vect}}\nolimits}
\newcommand{\Mat}{\mathop{\mathrm{Mat}}\nolimits}
\newcommand{\rg}{\mathop{\text{rg}}\nolimits}
\newcommand{\tr}{\mathop{\text{tr}}\nolimits}
\newcommand{\ppcm}{\mathop{\text{ppcm}}\nolimits}

%----- Structure des exercices ------

\newtheoremstyle{styleexo}% name
{2ex}% Space above
{3ex}% Space below
{}% Body font
{}% Indent amount 1
{\bfseries} % Theorem head font
{}% Punctuation after theorem head
{\newline}% Space after theorem head 2
{}% Theorem head spec (can be left empty, meaning ‘normal’)

%\theoremstyle{styleexo}
\newtheorem{exo}{Exercice}
\newtheorem{ind}{Indications}
\newtheorem{cor}{Correction}


\newcommand{\exercice}[1]{} \newcommand{\finexercice}{}
%\newcommand{\exercice}[1]{{\tiny\texttt{#1}}\vspace{-2ex}} % pour afficher le numero absolu, l'auteur...
\newcommand{\enonce}{\begin{exo}} \newcommand{\finenonce}{\end{exo}}
\newcommand{\indication}{\begin{ind}} \newcommand{\finindication}{\end{ind}}
\newcommand{\correction}{\begin{cor}} \newcommand{\fincorrection}{\end{cor}}

\newcommand{\noindication}{\stepcounter{ind}}
\newcommand{\nocorrection}{\stepcounter{cor}}

\newcommand{\fiche}[1]{} \newcommand{\finfiche}{}
\newcommand{\titre}[1]{\centerline{\large \bf #1}}
\newcommand{\addcommand}[1]{}
\newcommand{\video}[1]{}

% Marge
\newcommand{\mymargin}[1]{\marginpar{{\small #1}}}



%----- Presentation ------
\setlength{\parindent}{0cm}

%\newcommand{\ExoSept}{\href{http://exo7.emath.fr}{\textbf{\textsf{Exo7}}}}

\definecolor{myred}{rgb}{0.93,0.26,0}
\definecolor{myorange}{rgb}{0.97,0.58,0}
\definecolor{myyellow}{rgb}{1,0.86,0}

\newcommand{\LogoExoSept}[1]{  % input : echelle
{\usefont{U}{cmss}{bx}{n}
\begin{tikzpicture}[scale=0.1*#1,transform shape]
  \fill[color=myorange] (0,0)--(4,0)--(4,-4)--(0,-4)--cycle;
  \fill[color=myred] (0,0)--(0,3)--(-3,3)--(-3,0)--cycle;
  \fill[color=myyellow] (4,0)--(7,4)--(3,7)--(0,3)--cycle;
  \node[scale=5] at (3.5,3.5) {Exo7};
\end{tikzpicture}}
}



\theoremstyle{definition}
%\newtheorem{proposition}{Proposition}
%\newtheorem{exemple}{Exemple}
%\newtheorem{theoreme}{Théorème}
\newtheorem{lemme}{Lemme}
\newtheorem{corollaire}{Corollaire}
%\newtheorem*{remarque*}{Remarque}
%\newtheorem*{miniexercice}{Mini-exercices}
%\newtheorem{definition}{Définition}




%definition d'un terme
\newcommand{\defi}[1]{{\color{myorange}\textbf{\emph{#1}}}}
\newcommand{\evidence}[1]{{\color{blue}\textbf{\emph{#1}}}}



 %----- Commandes divers ------

\newcommand{\codeinline}[1]{\texttt{#1}}

%%%%%%%%%%%%%%%%%%%%%%%%%%%%%%%%%%%%%%%%%%%%%%%%%%%%%%%%%%%%%
%%%%%%%%%%%%%%%%%%%%%%%%%%%%%%%%%%%%%%%%%%%%%%%%%%%%%%%%%%%%%



\begin{document}

\debuttexte



%%%%%%%%%%%%%%%%%%%%%%%%%%%%%%%%%%%%%%%%%%%%%%%%%%%%%%%%%%%
\diapo


\change

Maintenant que  nous savons ce que sont des groupes, nous allons étudier les fonctions
entre les groupes : ce sont les morphismes de groupes.

Le programme est le suivant :

\change

la définition d'un morphisme,

\change

quelques propriétés,

\change

l'étude du noyau et de l'image,

\change

des exemples.


%%%%%%%%%%%%%%%%%%%%%%%%%%%%%%%%%%%%%%%%%%%%%%%%%%%%%%%%%%%
\diapo

Considérons un groupe $G$ avec un loi $\star$

et un groupe $G'$ avec un loi $\diamond$ [[losange]]

et soit $f$ de $G $ vers $G'$ une application.

\change

Par définition $f$ est un morphisme entre le groupe $G$ et le groupe $G'$ si

$\text{pour tout } x,x' \in G  \qquad f(x \star x') = f(x) \diamond f(x')$

\change

Voici un exemple que vous connaissez déjà :

$G$ est le groupe $(\Rr,+)$

$G'$ est le groupe $(\Rr_+^*,\times)$

et $f$ est juste l'application que a $x$ associe exponentielle $x$.

\change

Alors $f$ est bien un morphisme de groupes car

$f(x+x') =\exp(x+x')= \exp(x) \times \exp(x') = f(x) \times f(x')$


%%%%%%%%%%%%%%%%%%%%%%%%%%%%%%%%%%%%%%%%%%%%%%%%%%%%%%%%%%%
\diapo

Voici deux propriétés faciles à démontrer en utilisant la définition :

tout d'abord l'élément neutre s'envoie par $f$ sur l'élément neutre :

$f(e_G) = e_{G'}$

\change

Deuxième propriété $f(x^{-1}) = \big(f(x)\big)^{-1}$

l'image de l'inverse est l'inverse de l'image !

Il faut bien être conscient que $x^{-1}$ est l'inverse de
$x$ dans le groupe $G$, 

alors que $\big(f(x)\big)^{-1}$
est l'inverse de $f(x)$ dans le groupe $G'$.


\change

Voyons ce que cela donne sur notre morphisme définie par $\exp(x)$.

\change

Nous avons $f(0)=1$,

$0$ est l'élément neutre du groupe $(\Rr,+)$
est envoyé sur $1$, l'élément neutre du groupe $(\Rr_+^*,\times)$.

\change

Calculons l'image de $f(-x)$

c'est $f(-x)=\exp(-x)= \frac{1}{\exp(x)}=\frac{1}{f(x)}$

Donc l'image de l'inverse de $x$ qui est ici $-x$

est bien l'inverse de $f(x)$ qui est ici $\frac{1}{f(x)}$

%%%%%%%%%%%%%%%%%%%%%%%%%%%%%%%%%%%%%%%%%%%%%%%%%%%%%%%%%%%
\diapo

Voici une proposition que je vous encourage à démontrer tout seul :

Si l'on a deux morphismes groupes 

$f : G \longrightarrow G'$ et $g : G'  \longrightarrow G''$

Alors $g \circ f $ est un morphisme de groupes de
 $G$ vers $G''$.

\change

Voici une autre proposition concernant la bijection réciproque
d'un morphisme :

Si $f : G \longrightarrow G'$ est un morphisme de groupe et est bijectif 

alors sa bijection réciproque  $f^{-1}$ est aussi un morphisme du groupe
$G'$ vers le groupe $G$.

\change

Prouvons cette deuxième proposition

Notons $\star$ le loi du groupe $G$ et $\diamond$ celle de $G'$.

\change

$f^{-1}$ va de $G'$ vers $G$

\change

Partons de deux éléments $y,y'$ de $G'$

Comme $f$ est bijective, elle en particulier surjective donc

il existe $x \in G$ tel que $f(x)=y$ 

et aussi $x' \in G$ tel que $f(x')=y'$


Il nous reste à calculer 
$f^{-1}(y\diamond y') $




\change

C'est parti !

$f^{-1}(y\diamond y') =  f^{-1}\big(f(x)\diamond f(x') \big)$

\change

égal $f^{-1}\big(f(x \star x') \big) $

car $f$ est morphisme de groupe 

\change

qui vaut $ x\star x'$

car $f^{-1}$ est la bijection réciproque de $f$

\change

Mais comme $f(x)=y$ alors $x=f^{-1}(y)$
et de même $x'=f^{-1}(y')$

Ce qui nous donne 
$f^{-1}(y)\star f^{-1}(y')$

C'est exactement la formule qui nous dit
que $f^{-1}$ est un morphisme de $G'$ vers $G$.


%%%%%%%%%%%%%%%%%%%%%%%%%%%%%%%%%%%%%%%%%%%%%%%%%%%%%%%%%%%
\diapo


Un morphisme bijectif s'appelle un isomorphisme


et on dit alors que les groupes $G, G'$ sont isomorphes

\change


Reprenons notre exemple avec le morphisme défini par l'exponentielle

\change


En plus d'être un morphisme c'est aussi une application bijective

\change

Dont nous connaissons la bijection réciproque, c'est le logarithme.

\change

Une des propriétés de la page précédente nous affirme que $f^{-1}$ est aussi un morphisme.


C'est-à-dire ici que $f^{-1}(x\times x')=f^{-1}(x) + f^{-1}(x')$

En remplaçant $f^{-1}$ par le logarithme, nous retrouvons la formule bien connue :

$\ln(x \times x') = \ln(x) + \ln(x')$


\change

Ainsi $f$ est un isomorphisme

et donc les groupes $\Rr$ muni de l'addition  et $\Rr_+^*$ muni de la multiplication sont
isomorphes.


%%%%%%%%%%%%%%%%%%%%%%%%%%%%%%%%%%%%%%%%%%%%%%%%%%%%%%%%%%%
\diapo

Soit $f$ un morphisme entre deux groupes $G$ et $G'$.

Le noyau de $f$ est par définition l'ensemble des éléments de $G$
tel $f(x)=e_{G'}$ l'élément neutre de $G'$.

C'est un ensemble important pour l'étude des morphismes.

\change

Quelques remarques :

$\Ker f$ est une partie de l'ensemble de départ $G$.

\change

En terme d'image réciproque $\Ker f$ est l'image réciproque du singleton $\{ e_{G'}\}$.

\change

En français le noyau est l'ensemble des éléments de $G$ qui s'envoient 
par $f$ sur l'élément neutre de $G'$


%%%%%%%%%%%%%%%%%%%%%%%%%%%%%%%%%%%%%%%%%%%%%%%%%%%%%%%%%%%
\diapo


[plusieurs prises]


Voici deux propriétés importantes concernant le noyau.


Premièrement le noyau $\Ker f$ est plus qu'une partie de $G$, 
c'est un sous-groupe de $G$.

\change

Deuxièmement le morphisme de groupes
$f$ est injectif si et seulement si $\Ker f$
ne contient que le singleton $\{e\}$, l'élément neutre de l'ensemble de départ.

La premier point permet de construire facilement des sous-groupes,

le second permet de tester si une application est injective ou pas.

\change

Voici la démonstration que vous pouvez passer en première lecture.

Pour le premier point il s'agit de vérifier que $\Ker f$
est bien un sous-groupe de $G$.


Nous avons vu que $f(e_G)=e_{G'}$ 

c'est donc que $e_G$ appartient au noyau.

\change

Si l'on prend deux éléments $x$ et $x'$ du noyau

 $f(x\star x') = f(x) \diamond f(x')= e_{G'} \diamond e_{G'} = e_{G'}$

et donc $x \star x'$ est dans le noyau.

\change

Enfin si $x$ appartient au noyau alors

$f(x^{-1}) = f(x)^{-1}=e_{G'}^{-1} = e_{G'}$

et donc $x^{-1}$ est aussi dans le noyau.

Cela fait bien de $\Ker f$ un sous-groupe de $G$.

\change

Pour le deuxième point nous allons montrer l'équivalence par double implications.

Tout d'abord supposons $f$ injective.

Supposons que $x$ soit un élément du noyau, 
cela signifie  $f(x)=e_{G'}$ ce qui revient à l'égalité $f(x)=f(e_G)$

Mais comme $f$ est injective, alors cela implique $x=e_G$.

Le seul élément pouvant appartenir au noyau est donc $x=e$.

Le noyau est donc le singleton $\{e_G\}$.

\change

Réciproquement, nous supposons que le noyau est réduit au singleton $\{e_G\}$.

Prenons deux élément $x,x'$ tels que $f(x)=f(x')$,

en composant cette égalité par $\big( f(x') \big)^{-1}$ de chaque coté

on obtient $f(x)\diamond \big( f(x') \big)^{-1} = e_{G'}$ 

et comme $f$ est un morphisme

alors [[skip une formule]]

 $f(x \star x'^{-1}) = e_{G'}$


Mais ceci veut exactement dire que $x \star x'^{-1}$ appartient à $\Ker f$

Comme $\Ker f$ ne contient que $e_G$

alors $x \star x'^{-1} = e_G$ 

et donc $x=x'$. 

Conclusion si $f(x)=f(x')$ alors $x=x'$, ainsi $f$ est injective.

%%%%%%%%%%%%%%%%%%%%%%%%%%%%%%%%%%%%%%%%%%%%%%%%%%%%%%%%%%%
\diapo


Continuons avec notre morphisme $f: G \longrightarrow G'$

L'image de $f$ est l'ensemble des $f(x)$ pour $x$ parcourant $G$.

\change

c'est une partie de l'ensemble d'arrivée $G'$

en terme d'image direct on écrirait $\Im f = f(G)$

en terme d'antécédents  $\Im f$ est l'ensemble 
des éléments de $G'$ qui ont (au moins) un antécédent par~$f$

\change

Voici deux propriétés :

$\Im f$ est un sous-groupe de $G'$

ce qui fournit une nouvelle façon de construire des sous-groupes.

Enfin pour rappel $f$ surjective $\iff$ $\Im f = G'$, 
ce qui est vrai pour toutes les applications et en particulier pour notre morphisme.



%%%%%%%%%%%%%%%%%%%%%%%%%%%%%%%%%%%%%%%%%%%%%%%%%%%%%%%%%%%
\diapo

Soit $\Zz$ muni de l'addition.

L'application qui à $k$ associe $3k$...

\change

est bien un morphisme de groupe.

\change

  $f(k+k')$ c'est $3(k+k')$ donc $3k + 3k'$ qui est bien $f(k)+f(k')$

\change

Le noyau de $f$ c'est l'ensemble des $k$ tel que $f(k)=0$.

\change

Mais  $f(k)=0$ équivaut à $3k=0$ équivaut à $k=0$

donc le seul élément du noyau est $k=0$.

\change

Ainsi par la proposition sur les noyaux $f$ est injective

\change

L'image de $f$ est l'ensemble des $f(k)$ pour $k$ parcourant $\Zz$.

C'est donc l'ensemble des $3k$, c'est-à-dire l'ensemble des multiples de $3$.

\change

Comme l'image d'un morphisme est un sous-groupe,
nous retrouvons que $3\Zz$ est un sous-groupe de $\Zz$.


%%%%%%%%%%%%%%%%%%%%%%%%%%%%%%%%%%%%%%%%%%%%%%%%%%%%%%%%%%%
\diapo

Etudions maintenant l'application qui a $t$ associe $e^{it}$.

L'ensemble de départ est le groupe $\Rr$ muni de l'addition.
L'ensemble d'arrivée est le groupe formé des nombres complexes de module $1$.
muni de la multiplication.


\change

$f$ est bien un morphisme car $f(t+t')=e^{\ii (t+t')}= e^{\ii t} \times e^{\ii t'} = f(t)\times f(t')$

\change

Déterminons le noyau $\Ker f$ : c'est l'ensemble des $t$ tels que 
$f(t)$ vaut $1$, l'élément neutre de $\mathbb{U}$.

$f(t) = 1$ ssi $e^{it}=1$ ssi $t=0$ modulo $2\pi$.

Le noyau est donc l'ensemble des $2k\pi$, c'est-à-dire ce sont les multiples de $2\pi$.

\change

Comme le noyau contient d'autres éléments que $0$, $f$ n'est pas injective.

\change

Enfin tout nombre complexe de module $1$ s'écrit sous la forme $f(t)=e^{\ii t}$
donc $f$ est surjective et l'image de $f$ est $\mathbb{U}$ tout entier.


%%%%%%%%%%%%%%%%%%%%%%%%%%%%%%%%%%%%%%%%%%%%%%%%%%%%%%%%%%%
\diapo

Les notations suivantes ne doivent pas être confondues.

\change

 $x^{-1}$ est l'inverse de $x$ dans un groupe $(G,\star)$



Si le groupe est $(\Rr^*,\times)$ alors la notation est cohérente avec ce que vous utilisiez car alors $x^{-1}=\frac 1 x$

\change

Pour une application bijective $f^{-1}$ désigne sa bijection réciproque

c'est donc une application qui n'a rien à voir avec $1/f$.

\change

Enfin pour une application quelconque, bijective ou pas,


l'ensemble $f^{-1}(B)$ est  l'image réciproque de $B$

c'est l'ensemble des $x$ de l'ensemble de départ dont l'image par $f$ est dans $B$

En particulier le noyau est l'image réciproque  du singleton $e_{G'}$

%%%%%%%%%%%%%%%%%%%%%%%%%%%%%%%%%%%%%%%%%%%%%%%%%%%%%%%%%%%
\diapo

Utilisez vos connaissance pour répondre aux questions suivantes.




\end{document}
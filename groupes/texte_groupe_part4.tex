
%%%%%%%%%%%%%%%%%% PREAMBULE %%%%%%%%%%%%%%%%%%


\documentclass[12pt]{article}

\usepackage{amsfonts,amsmath,amssymb,amsthm}
\usepackage[utf8]{inputenc}
\usepackage[T1]{fontenc}
\usepackage[francais]{babel}


% packages
\usepackage{amsfonts,amsmath,amssymb,amsthm}
\usepackage[utf8]{inputenc}
\usepackage[T1]{fontenc}
%\usepackage{lmodern}

\usepackage[francais]{babel}
\usepackage{fancybox}
\usepackage{graphicx}

\usepackage{float}

%\usepackage[usenames, x11names]{xcolor}
\usepackage{tikz}
\usepackage{datetime}

\usepackage{mathptmx}
%\usepackage{fouriernc}
%\usepackage{newcent}
\usepackage[mathcal,mathbf]{euler}

%\usepackage{palatino}
%\usepackage{newcent}


% Commande spéciale prompteur

%\usepackage{mathptmx}
%\usepackage[mathcal,mathbf]{euler}
%\usepackage{mathpple,multido}

\usepackage[a4paper]{geometry}
\geometry{top=2cm, bottom=2cm, left=1cm, right=1cm, marginparsep=1cm}

\newcommand{\change}{{\color{red}\rule{\textwidth}{1mm}\\}}

\newcounter{mydiapo}

\newcommand{\diapo}{\newpage
\hfill {\normalsize  Diapo \themydiapo \quad \texttt{[\jobname]}} \\
\stepcounter{mydiapo}}


%%%%%%% COULEURS %%%%%%%%%%

% Pour blanc sur noir :
%\pagecolor[rgb]{0.5,0.5,0.5}
% \pagecolor[rgb]{0,0,0}
% \color[rgb]{1,1,1}



%\DeclareFixedFont{\myfont}{U}{cmss}{bx}{n}{18pt}
\newcommand{\debuttexte}{
%%%%%%%%%%%%% FONTES %%%%%%%%%%%%%
\renewcommand{\baselinestretch}{1.5}
\usefont{U}{cmss}{bx}{n}
\bfseries

% Taille normale : commenter le reste !
%Taille Arnaud
%\fontsize{19}{19}\selectfont

% Taille Barbara
%\fontsize{21}{22}\selectfont

%Taille François
%\fontsize{25}{30}\selectfont

%Taille Pascal
%\fontsize{25}{30}\selectfont

%Taille Laura
%\fontsize{30}{35}\selectfont


%\myfont
%\usefont{U}{cmss}{bx}{n}

%\Huge
%\addtolength{\parskip}{\baselineskip}
}


% \usepackage{hyperref}
% \hypersetup{colorlinks=true, linkcolor=blue, urlcolor=blue,
% pdftitle={Exo7 - Exercices de mathématiques}, pdfauthor={Exo7}}


%section
% \usepackage{sectsty}
% \allsectionsfont{\bf}
%\sectionfont{\color{Tomato3}\upshape\selectfont}
%\subsectionfont{\color{Tomato4}\upshape\selectfont}

%----- Ensembles : entiers, reels, complexes -----
\newcommand{\Nn}{\mathbb{N}} \newcommand{\N}{\mathbb{N}}
\newcommand{\Zz}{\mathbb{Z}} \newcommand{\Z}{\mathbb{Z}}
\newcommand{\Qq}{\mathbb{Q}} \newcommand{\Q}{\mathbb{Q}}
\newcommand{\Rr}{\mathbb{R}} \newcommand{\R}{\mathbb{R}}
\newcommand{\Cc}{\mathbb{C}} 
\newcommand{\Kk}{\mathbb{K}} \newcommand{\K}{\mathbb{K}}

%----- Modifications de symboles -----
\renewcommand{\epsilon}{\varepsilon}
\renewcommand{\Re}{\mathop{\text{Re}}\nolimits}
\renewcommand{\Im}{\mathop{\text{Im}}\nolimits}
%\newcommand{\llbracket}{\left[\kern-0.15em\left[}
%\newcommand{\rrbracket}{\right]\kern-0.15em\right]}

\renewcommand{\ge}{\geqslant}
\renewcommand{\geq}{\geqslant}
\renewcommand{\le}{\leqslant}
\renewcommand{\leq}{\leqslant}

%----- Fonctions usuelles -----
\newcommand{\ch}{\mathop{\mathrm{ch}}\nolimits}
\newcommand{\sh}{\mathop{\mathrm{sh}}\nolimits}
\renewcommand{\tanh}{\mathop{\mathrm{th}}\nolimits}
\newcommand{\cotan}{\mathop{\mathrm{cotan}}\nolimits}
\newcommand{\Arcsin}{\mathop{\mathrm{Arcsin}}\nolimits}
\newcommand{\Arccos}{\mathop{\mathrm{Arccos}}\nolimits}
\newcommand{\Arctan}{\mathop{\mathrm{Arctan}}\nolimits}
\newcommand{\Argsh}{\mathop{\mathrm{Argsh}}\nolimits}
\newcommand{\Argch}{\mathop{\mathrm{Argch}}\nolimits}
\newcommand{\Argth}{\mathop{\mathrm{Argth}}\nolimits}
\newcommand{\pgcd}{\mathop{\mathrm{pgcd}}\nolimits} 

\newcommand{\Card}{\mathop{\text{Card}}\nolimits}
\newcommand{\Ker}{\mathop{\text{Ker}}\nolimits}
\newcommand{\id}{\mathop{\text{id}}\nolimits}
\newcommand{\ii}{\mathrm{i}}
\newcommand{\dd}{\mathrm{d}}
\newcommand{\Vect}{\mathop{\text{Vect}}\nolimits}
\newcommand{\Mat}{\mathop{\mathrm{Mat}}\nolimits}
\newcommand{\rg}{\mathop{\text{rg}}\nolimits}
\newcommand{\tr}{\mathop{\text{tr}}\nolimits}
\newcommand{\ppcm}{\mathop{\text{ppcm}}\nolimits}

%----- Structure des exercices ------

\newtheoremstyle{styleexo}% name
{2ex}% Space above
{3ex}% Space below
{}% Body font
{}% Indent amount 1
{\bfseries} % Theorem head font
{}% Punctuation after theorem head
{\newline}% Space after theorem head 2
{}% Theorem head spec (can be left empty, meaning ‘normal’)

%\theoremstyle{styleexo}
\newtheorem{exo}{Exercice}
\newtheorem{ind}{Indications}
\newtheorem{cor}{Correction}


\newcommand{\exercice}[1]{} \newcommand{\finexercice}{}
%\newcommand{\exercice}[1]{{\tiny\texttt{#1}}\vspace{-2ex}} % pour afficher le numero absolu, l'auteur...
\newcommand{\enonce}{\begin{exo}} \newcommand{\finenonce}{\end{exo}}
\newcommand{\indication}{\begin{ind}} \newcommand{\finindication}{\end{ind}}
\newcommand{\correction}{\begin{cor}} \newcommand{\fincorrection}{\end{cor}}

\newcommand{\noindication}{\stepcounter{ind}}
\newcommand{\nocorrection}{\stepcounter{cor}}

\newcommand{\fiche}[1]{} \newcommand{\finfiche}{}
\newcommand{\titre}[1]{\centerline{\large \bf #1}}
\newcommand{\addcommand}[1]{}
\newcommand{\video}[1]{}

% Marge
\newcommand{\mymargin}[1]{\marginpar{{\small #1}}}



%----- Presentation ------
\setlength{\parindent}{0cm}

%\newcommand{\ExoSept}{\href{http://exo7.emath.fr}{\textbf{\textsf{Exo7}}}}

\definecolor{myred}{rgb}{0.93,0.26,0}
\definecolor{myorange}{rgb}{0.97,0.58,0}
\definecolor{myyellow}{rgb}{1,0.86,0}

\newcommand{\LogoExoSept}[1]{  % input : echelle
{\usefont{U}{cmss}{bx}{n}
\begin{tikzpicture}[scale=0.1*#1,transform shape]
  \fill[color=myorange] (0,0)--(4,0)--(4,-4)--(0,-4)--cycle;
  \fill[color=myred] (0,0)--(0,3)--(-3,3)--(-3,0)--cycle;
  \fill[color=myyellow] (4,0)--(7,4)--(3,7)--(0,3)--cycle;
  \node[scale=5] at (3.5,3.5) {Exo7};
\end{tikzpicture}}
}



\theoremstyle{definition}
%\newtheorem{proposition}{Proposition}
%\newtheorem{exemple}{Exemple}
%\newtheorem{theoreme}{Théorème}
\newtheorem{lemme}{Lemme}
\newtheorem{corollaire}{Corollaire}
%\newtheorem*{remarque*}{Remarque}
%\newtheorem*{miniexercice}{Mini-exercices}
%\newtheorem{definition}{Définition}




%definition d'un terme
\newcommand{\defi}[1]{{\color{myorange}\textbf{\emph{#1}}}}
\newcommand{\evidence}[1]{{\color{blue}\textbf{\emph{#1}}}}



 %----- Commandes divers ------

\newcommand{\codeinline}[1]{\texttt{#1}}

%%%%%%%%%%%%%%%%%%%%%%%%%%%%%%%%%%%%%%%%%%%%%%%%%%%%%%%%%%%%%
%%%%%%%%%%%%%%%%%%%%%%%%%%%%%%%%%%%%%%%%%%%%%%%%%%%%%%%%%%%%%



\begin{document}

\debuttexte

%%%%%%%%%%%%%%%%%%%%%%%%%%%%%%%%%%%%%%%%%%%%%%%%%%%%%%%%%%%
\diapo

\change

Nous allons étudier un groupe particulièrement important le groupe $\Zz/n\Zz$.

\change

Nous verrons comment définir sur l'ensemble  $\Zz/n\Zz$ un loi d'addition.

\change

Ensuite nous trouverons des propriétés qui rendent ce groupe unique.

%%%%%%%%%%%%%%%%%%%%%%%%%%%%%%%%%%%%%%%%%%%%%%%%%%%%%%%%%%%
\diapo

Pour un entier $n$ fixé, $\Zz/n\Zz$ est l'ensemble
des classes $0 bar$, $1 bar$,  jusqu'à $n-1$ bar.

\change

$\bar p$ désigne la classe d'équivalence de l'entier $p$ modulo $n$


On a la relation fondamentale la classe de $p$ égale la classe de $q$ ssi $p \equiv q$ modulo $n$.

\change

autrement dit $\bar p = \bar q$ ssi  $p = q + kn$

\change

Nous allons considérer les classes $\bar p$ comme des nombres.

En particulier voici comment additionner deux classes :

par définition : classe de $p$ + classe de $q$ est
la classe de $p+q$.

Autrement dit $\overline p + \overline q = \overline{p+q}$


%%%%%%%%%%%%%%%%%%%%%%%%%%%%%%%%%%%%%%%%%%%%%%%%%%%%%%%%%%%
\diapo

Calculons quelques exemples pour $n=60$.

$\overline{60}=\overline{0}$, car $60$ est congru à $0$ modulo $60$

$\overline{61}=\overline{1}$,

$\overline{62}=\overline{2}$

\change

$\overline{135} = \overline{15}$ car $135 = 15 + 2\times 60$.

\change

$\Zz/60\Zz$ est donc constitué de $60$ éléments 
$\big\{\overline{0},\overline{1},\overline{2},\ldots,\overline{59}\big\}$

\change

Maintenant calculons quelques sommes.

$\overline{31} + \overline{46}  = \overline{31 + 46} =\overline{77} = \overline{17}$

\change

$\overline{15} + \overline{50}  = \overline{15 + 50} =\overline{65} = \overline{5}$


\change

  $\overline{135}+\overline{50} = \overline{185} = \overline{5}$

C'est normal de retrouver $\overline 5$ car $\overline{135}=\overline{15}$

c'est donc la même addition qu'au dessus.

\change

On aurait aussi pu écrire 

$\overline{135}+\overline{50} = \overline{15} - \overline{10} = \overline{5}$

et on retrouve encore le même résultat.

\change

Ce qui fait que l'on retrouve le même résultat, c'est que l'addition est bien définie,
le résultat est indépendant du choix des représentants.

\change

$\begin{array}{rcl}
 \overline{p'}= \overline p \ \text{ et } \ \overline{q'} = \overline q 
    & \Rightarrow & p' \equiv p \pmod n \text{ et } q' \equiv q \pmod n \\
    & \Rightarrow & p'+q' \equiv p+q \pmod n \\
    & \Rightarrow & \overline{p'+q'} = \overline{p+q} \\
    & \Rightarrow & \overline{p'}+\overline{q'} = \overline p + \overline q \\
\end{array}
$


%%%%%%%%%%%%%%%%%%%%%%%%%%%%%%%%%%%%%%%%%%%%%%%%%%%%%%%%%%%
\diapo

Avec cette addition $\Zz/n\Zz$ est un groupe commutatif.

\change

L'élément neutre $\bar 0$,

\change

L'opposé d'un élément $\bar p$, que l'on notera $-\overline p$, est $\overline{-p}$,
ou ce que revient au même $\overline{n-p}$.

\change

Enfin l'associativité et la commutativité découlent
de l'associativité et la commutativité sur $\Zz$

%%%%%%%%%%%%%%%%%%%%%%%%%%%%%%%%%%%%%%%%%%%%%%%%%%%%%%%%%%%
\diapo

Nous dirons  qu'un groupe $G$ est cyclique
si l'on peut trouver un élément $a$ du groupe, tel que tout élément 
 $x$ de $G$ s'écrivent de la forme $x=a^k$.

\change

Une autre formulation est de dire que $G$ est cyclique si et seulement si
il est engendré par un seul élément (en l'occurrence $a$).

\change

Notre groupe $\Zz/n\Zz$ est un groupe cyclique, car l'élément $\bar 1$ engendre 
tout le groupe :
en effet tout élément $\bar k$ peut s'écrire 
$\bar k = \overline 1 + \overline 1 + \cdots \overline 1$ ($k$ fois)
ici la notation est additive mais c'est bien $\bar 1$ composé $k$ fois.


%%%%%%%%%%%%%%%%%%%%%%%%%%%%%%%%%%%%%%%%%%%%%%%%%%%%%%%%%%%
\diapo

Nous avons vu que $\Zz/n\Zz$ est un groupe cyclique et bien sûr son cardinal
est $n$. 

Un théorème intéressant est que la réciproque est vraie.

Théorème : Si $(G,\star)$ est un groupe cyclique de cardinal $n$, alors  
$(G,\star)$ est isomorphe à $(\Zz/n\Zz,+)$

\change

Avec d'autres mots :

Il existe, à isomorphisme près, qu'un seul groupe 
cyclique à $n$ éléments, c'est $\Zz/n\Zz$


Nous allons nous concentrer sur la démonstration que vous pouvez éluder
en première instance.


%%%%%%%%%%%%%%%%%%%%%%%%%%%%%%%%%%%%%%%%%%%%%%%%%%%%%%%%%%%
\diapo

Première partie de la démonstration : nous explicitons les éléments de $G$.

Nous savons que $G$ est cyclique, il est donc engendré par un élément $a$,
c'est-à-dire que $G$ égal l'ensemble de $a^k$ pour $k$ parcourant $\Zz$.

\change

Mais nous savons que $G$ n'a que $n$ élément.

Nous allons montrer que $G$ peut s'écrit plus simplement
 $G = \big\{e, a, a^2, \ldots, a^{n-1}\big\}$

et que $a^n = e$


\change

Tout d'abord
 $\big\{e, a, a^2, \ldots, a^{n-1}\big\}$ est inclus dans $G$

\change

Mais nous devons montrer que tous ses éléments sont différents 

par l'absurde si $a^p=a^q$ (avec $q < p \le n-1$) alors 
en composant par $a^{-q}$ on obtient $a^{p-q}=e$ 


$G$ n'aurait plus que $p-q$ éléments, car 
$a^{p-q+1}$ vaut alors $a$, $a^{p-q+2}$ vaut $a^2$, etc.



$G$ n'a donc plus $n$ élément ce qui est contradictoire.

Donc $a^p \neq a^q$


\change  

Nous avons $n$ éléments différents appartenant à $G$ de cardinal $n$.

Ainsi $G$ est bien égal à l'ensemble des $a^p$

pur $p$ variant de $0$ à $n-1$.


\change

Calculons maintenant $a^n$ où $n$ est toujours le cardinal du groupe,

$a^n$ est un élément de $G$ 

donc c'est l'un des $a^p$ pour un $p$ compris entre $0$ et $n-1$

Si $p>0$ alors nous aurions $a^{n-p}=e$ et par le même type d'arguments 
$G$ n'aurait pas $n$ éléments.

\change

Ainsi $p=0$ donc $a^n=a^0=e$



%%%%%%%%%%%%%%%%%%%%%%%%%%%%%%%%%%%%%%%%%%%%%%%%%%%%%%%%%%%
\diapo

Dans cette seconde partie 
nous construisons l'isomorphisme entre $G$ et $\Zz/n\Zz$.

\change

Nous définissons une application $f$ de $\Zz/n\Zz$ dans $G$
qui a $\bar k$  associe  $a^k$.

Nous devons montrer que $f$ est un morphisme bijectif.

\change

Mais avant cela, comme à chaque fois qu'un formule est définie
à partir d'un représentant (et pas de la classe) nous commençons
par prouver que le résultat est indépendant du choix du représentant.

En effet

$\begin{array}{rcl}
 \overline{k}=\overline{k'} 
    & \Rightarrow & k \equiv k' \pmod n \\
    & \Rightarrow & k = k' + \ell n \\
    & \Rightarrow & f(\overline{k}) = a^k = a^{k'+\ell n}  \\
    & \Rightarrow & f(\overline{k}) = a^{k'} \star a^{\ell n}=
a^{k'} \star (a^n)^\ell \\
& = a^{k'} \star e^\ell = a^{k'} = f(\overline{k'}) \\
\end{array}
$


\change

Montrons maintenant que $f$ est un morphisme :

Pour $\bar k$ et $\bar k'$ quelconques,

$f(\overline{k}+\overline{k'})
= f(\overline{k+k'}) = a^{k+k'}=a^{k} \star a^{k'}= f(\overline k) \star f(\overline{k'})$ 

\change

Enfin $f$ est surjective car comme $G$ est cyclique tout élément 
de $G$ est de la forme $a^k$.

\change

Nous concluons par un argument à propos des ensembles finis.

Une application surjective entre deux ensembles de même cardinal
 est automatiquement  bijective. 

Ceci s'applique à $f$ qui est une surjection de $\Zz/n\Zz$ vers $G$ 
tous deux de cardinal $n$. 

$f$ est bien une bijection.

\change

Conclusion $f$ est un morphisme bijectif, c'est un isomorphisme.

\change

%%%%%%%%%%%%%%%%%%%%%%%%%%%%%%%%%%%%%%%%%%%%%%%%%%%%%%%%%%%
\diapo

Pour terminer voici les exercices.

\end{document}

%%%%%%%%%%%%%%%%%% PREAMBULE %%%%%%%%%%%%%%%%%%

\documentclass[aspectratio=169,utf8]{beamer}
%\documentclass[aspectratio=169,handout]{beamer}

\usetheme{Boadilla}
%\usecolortheme{seahorse}
%\usecolortheme[RGB={245,66,24}]{structure}
\useoutertheme{infolines}

% packages
\usepackage{amsfonts,amsmath,amssymb,amsthm}
\usepackage[utf8]{inputenc}
\usepackage[T1]{fontenc}
\usepackage{lmodern}

\usepackage[francais]{babel}
\usepackage{fancybox}
\usepackage{graphicx}

\usepackage{float}
\usepackage{xfrac}

%\usepackage[usenames, x11names]{xcolor}
\usepackage{pgfplots}
\usepackage{datetime}


% ----------------------------------------------------------------------
% Pour les images
\usepackage{tikz}
\usetikzlibrary{calc,shadows,arrows.meta,patterns,matrix}

\newcommand{\tikzinput}[1]{\input{figures/#1.tikz}}
% --- les figures avec échelle éventuel
\newcommand{\myfigure}[2]{% entrée : échelle, fichier(s) figure à inclure
\begin{center}\small%
\tikzstyle{every picture}=[scale=1.0*#1]% mise en échelle + 0% (automatiquement annulé à la fin du groupe)
#2%
\end{center}}



%-----  Package unités -----
\usepackage{siunitx}
\sisetup{locale = FR,detect-all,per-mode = symbol}

%\usepackage{mathptmx}
%\usepackage{fouriernc}
%\usepackage{newcent}
%\usepackage[mathcal,mathbf]{euler}

%\usepackage{palatino}
%\usepackage{newcent}
% \usepackage[mathcal,mathbf]{euler}



% \usepackage{hyperref}
% \hypersetup{colorlinks=true, linkcolor=blue, urlcolor=blue,
% pdftitle={Exo7 - Exercices de mathématiques}, pdfauthor={Exo7}}


%section
% \usepackage{sectsty}
% \allsectionsfont{\bf}
%\sectionfont{\color{Tomato3}\upshape\selectfont}
%\subsectionfont{\color{Tomato4}\upshape\selectfont}

%----- Ensembles : entiers, reels, complexes -----
\newcommand{\Nn}{\mathbb{N}} \newcommand{\N}{\mathbb{N}}
\newcommand{\Zz}{\mathbb{Z}} \newcommand{\Z}{\mathbb{Z}}
\newcommand{\Qq}{\mathbb{Q}} \newcommand{\Q}{\mathbb{Q}}
\newcommand{\Rr}{\mathbb{R}} \newcommand{\R}{\mathbb{R}}
\newcommand{\Cc}{\mathbb{C}} 
\newcommand{\Kk}{\mathbb{K}} \newcommand{\K}{\mathbb{K}}

%----- Modifications de symboles -----
\renewcommand{\epsilon}{\varepsilon}
\renewcommand{\Re}{\mathop{\text{Re}}\nolimits}
\renewcommand{\Im}{\mathop{\text{Im}}\nolimits}
%\newcommand{\llbracket}{\left[\kern-0.15em\left[}
%\newcommand{\rrbracket}{\right]\kern-0.15em\right]}

\renewcommand{\ge}{\geqslant}
\renewcommand{\geq}{\geqslant}
\renewcommand{\le}{\leqslant}
\renewcommand{\leq}{\leqslant}
\renewcommand{\epsilon}{\varepsilon}

%----- Fonctions usuelles -----
\newcommand{\ch}{\mathop{\text{ch}}\nolimits}
\newcommand{\sh}{\mathop{\text{sh}}\nolimits}
\renewcommand{\tanh}{\mathop{\text{th}}\nolimits}
\newcommand{\cotan}{\mathop{\text{cotan}}\nolimits}
\newcommand{\Arcsin}{\mathop{\text{arcsin}}\nolimits}
\newcommand{\Arccos}{\mathop{\text{arccos}}\nolimits}
\newcommand{\Arctan}{\mathop{\text{arctan}}\nolimits}
\newcommand{\Argsh}{\mathop{\text{argsh}}\nolimits}
\newcommand{\Argch}{\mathop{\text{argch}}\nolimits}
\newcommand{\Argth}{\mathop{\text{argth}}\nolimits}
\newcommand{\pgcd}{\mathop{\text{pgcd}}\nolimits} 


%----- Commandes divers ------
\newcommand{\ii}{\mathrm{i}}
\newcommand{\dd}{\text{d}}
\newcommand{\id}{\mathop{\text{id}}\nolimits}
\newcommand{\Ker}{\mathop{\text{Ker}}\nolimits}
\newcommand{\Card}{\mathop{\text{Card}}\nolimits}
\newcommand{\Vect}{\mathop{\text{Vect}}\nolimits}
\newcommand{\Mat}{\mathop{\text{Mat}}\nolimits}
\newcommand{\rg}{\mathop{\text{rg}}\nolimits}
\newcommand{\tr}{\mathop{\text{tr}}\nolimits}


%----- Structure des exercices ------

\newtheoremstyle{styleexo}% name
{2ex}% Space above
{3ex}% Space below
{}% Body font
{}% Indent amount 1
{\bfseries} % Theorem head font
{}% Punctuation after theorem head
{\newline}% Space after theorem head 2
{}% Theorem head spec (can be left empty, meaning ‘normal’)

%\theoremstyle{styleexo}
\newtheorem{exo}{Exercice}
\newtheorem{ind}{Indications}
\newtheorem{cor}{Correction}


\newcommand{\exercice}[1]{} \newcommand{\finexercice}{}
%\newcommand{\exercice}[1]{{\tiny\texttt{#1}}\vspace{-2ex}} % pour afficher le numero absolu, l'auteur...
\newcommand{\enonce}{\begin{exo}} \newcommand{\finenonce}{\end{exo}}
\newcommand{\indication}{\begin{ind}} \newcommand{\finindication}{\end{ind}}
\newcommand{\correction}{\begin{cor}} \newcommand{\fincorrection}{\end{cor}}

\newcommand{\noindication}{\stepcounter{ind}}
\newcommand{\nocorrection}{\stepcounter{cor}}

\newcommand{\fiche}[1]{} \newcommand{\finfiche}{}
\newcommand{\titre}[1]{\centerline{\large \bf #1}}
\newcommand{\addcommand}[1]{}
\newcommand{\video}[1]{}

% Marge
\newcommand{\mymargin}[1]{\marginpar{{\small #1}}}

\def\noqed{\renewcommand{\qedsymbol}{}}


%----- Presentation ------
\setlength{\parindent}{0cm}

%\newcommand{\ExoSept}{\href{http://exo7.emath.fr}{\textbf{\textsf{Exo7}}}}

\definecolor{myred}{rgb}{0.93,0.26,0}
\definecolor{myorange}{rgb}{0.97,0.58,0}
\definecolor{myyellow}{rgb}{1,0.86,0}

\newcommand{\LogoExoSept}[1]{  % input : echelle
{\usefont{U}{cmss}{bx}{n}
\begin{tikzpicture}[scale=0.1*#1,transform shape]
  \fill[color=myorange] (0,0)--(4,0)--(4,-4)--(0,-4)--cycle;
  \fill[color=myred] (0,0)--(0,3)--(-3,3)--(-3,0)--cycle;
  \fill[color=myyellow] (4,0)--(7,4)--(3,7)--(0,3)--cycle;
  \node[scale=5] at (3.5,3.5) {Exo7};
\end{tikzpicture}}
}


\newcommand{\debutmontitre}{
  \author{} \date{} 
  \thispagestyle{empty}
  \hspace*{-10ex}
  \begin{minipage}{\textwidth}
    \titlepage  
  \vspace*{-2.5cm}
  \begin{center}
    \LogoExoSept{2.5}
  \end{center}
  \end{minipage}

  \vspace*{-0cm}
  
  % Astuce pour que le background ne soit pas discrétisé lors de la conversion pdf -> png
\begin{tikzpicture}
        \fill[opacity=0,green!60!black] (0,0)--++(0,0)--++(0,0)--++(0,0)--cycle; 
\end{tikzpicture}

% toc S'affiche trop tot :
% \tableofcontents[hideallsubsections, pausesections]
}

\newcommand{\finmontitre}{
  \end{frame}
  \setcounter{framenumber}{0}
} % ne marche pas pour une raison obscure

%----- Commandes supplementaires ------

% \usepackage[landscape]{geometry}
% \geometry{top=1cm, bottom=3cm, left=2cm, right=10cm, marginparsep=1cm
% }
% \usepackage[a4paper]{geometry}
% \geometry{top=2cm, bottom=2cm, left=2cm, right=2cm, marginparsep=1cm
% }

%\usepackage{standalone}


% New command Arnaud -- november 2011
\setbeamersize{text margin left=24ex}
% si vous modifier cette valeur il faut aussi
% modifier le decalage du titre pour compenser
% (ex : ici =+10ex, titre =-5ex

\theoremstyle{definition}
%\newtheorem{proposition}{Proposition}
%\newtheorem{exemple}{Exemple}
%\newtheorem{theoreme}{Théorème}
%\newtheorem{lemme}{Lemme}
%\newtheorem{corollaire}{Corollaire}
%\newtheorem*{remarque*}{Remarque}
%\newtheorem*{miniexercice}{Mini-exercices}
%\newtheorem{definition}{Définition}

% Commande tikz
\usetikzlibrary{calc}
\usetikzlibrary{patterns,arrows}
\usetikzlibrary{matrix}
\usetikzlibrary{fadings} 

%definition d'un terme
\newcommand{\defi}[1]{{\color{myorange}\textbf{\emph{#1}}}}
\newcommand{\evidence}[1]{{\color{blue}\textbf{\emph{#1}}}}
\newcommand{\assertion}[1]{\emph{\og#1\fg}}  % pour chapitre logique
%\renewcommand{\contentsname}{Sommaire}
\renewcommand{\contentsname}{}
\setcounter{tocdepth}{2}



%------ Encadrement ------

\usepackage{fancybox}


\newcommand{\mybox}[1]{
\setlength{\fboxsep}{7pt}
\begin{center}
\shadowbox{#1}
\end{center}}

\newcommand{\myboxinline}[1]{
\setlength{\fboxsep}{5pt}
\raisebox{-10pt}{
\shadowbox{#1}
}
}

%--------------- Commande beamer---------------
\newcommand{\beameronly}[1]{#1} % permet de mettre des pause dans beamer pas dans poly


\setbeamertemplate{navigation symbols}{}
\setbeamertemplate{footline}  % tiré du fichier beamerouterinfolines.sty
{
  \leavevmode%
  \hbox{%
  \begin{beamercolorbox}[wd=.333333\paperwidth,ht=2.25ex,dp=1ex,center]{author in head/foot}%
    % \usebeamerfont{author in head/foot}\insertshortauthor%~~(\insertshortinstitute)
    \usebeamerfont{section in head/foot}{\bf\insertshorttitle}
  \end{beamercolorbox}%
  \begin{beamercolorbox}[wd=.333333\paperwidth,ht=2.25ex,dp=1ex,center]{title in head/foot}%
    \usebeamerfont{section in head/foot}{\bf\insertsectionhead}
  \end{beamercolorbox}%
  \begin{beamercolorbox}[wd=.333333\paperwidth,ht=2.25ex,dp=1ex,right]{date in head/foot}%
    % \usebeamerfont{date in head/foot}\insertshortdate{}\hspace*{2em}
    \insertframenumber{} / \inserttotalframenumber\hspace*{2ex} 
  \end{beamercolorbox}}%
  \vskip0pt%
}


\definecolor{mygrey}{rgb}{0.5,0.5,0.5}
\setlength{\parindent}{0cm}
%\DeclareTextFontCommand{\helvetica}{\fontfamily{phv}\selectfont}

% background beamer
\definecolor{couleurhaut}{rgb}{0.85,0.9,1}  % creme
\definecolor{couleurmilieu}{rgb}{1,1,1}  % vert pale
\definecolor{couleurbas}{rgb}{0.85,0.9,1}  % blanc
\setbeamertemplate{background canvas}[vertical shading]%
[top=couleurhaut,middle=couleurmilieu,midpoint=0.4,bottom=couleurbas] 
%[top=fondtitre!05,bottom=fondtitre!60]



\makeatletter
\setbeamertemplate{theorem begin}
{%
  \begin{\inserttheoremblockenv}
  {%
    \inserttheoremheadfont
    \inserttheoremname
    \inserttheoremnumber
    \ifx\inserttheoremaddition\@empty\else\ (\inserttheoremaddition)\fi%
    \inserttheorempunctuation
  }%
}
\setbeamertemplate{theorem end}{\end{\inserttheoremblockenv}}

\newenvironment{theoreme}[1][]{%
   \setbeamercolor{block title}{fg=structure,bg=structure!40}
   \setbeamercolor{block body}{fg=black,bg=structure!10}
   \begin{block}{{\bf Th\'eor\`eme }#1}
}{%
   \end{block}%
}


\newenvironment{proposition}[1][]{%
   \setbeamercolor{block title}{fg=structure,bg=structure!40}
   \setbeamercolor{block body}{fg=black,bg=structure!10}
   \begin{block}{{\bf Proposition }#1}
}{%
   \end{block}%
}

\newenvironment{corollaire}[1][]{%
   \setbeamercolor{block title}{fg=structure,bg=structure!40}
   \setbeamercolor{block body}{fg=black,bg=structure!10}
   \begin{block}{{\bf Corollaire }#1}
}{%
   \end{block}%
}

\newenvironment{mydefinition}[1][]{%
   \setbeamercolor{block title}{fg=structure,bg=structure!40}
   \setbeamercolor{block body}{fg=black,bg=structure!10}
   \begin{block}{{\bf Définition} #1}
}{%
   \end{block}%
}

\newenvironment{lemme}[0]{%
   \setbeamercolor{block title}{fg=structure,bg=structure!40}
   \setbeamercolor{block body}{fg=black,bg=structure!10}
   \begin{block}{\bf Lemme}
}{%
   \end{block}%
}

\newenvironment{remarque}[1][]{%
   \setbeamercolor{block title}{fg=black,bg=structure!20}
   \setbeamercolor{block body}{fg=black,bg=structure!5}
   \begin{block}{Remarque #1}
}{%
   \end{block}%
}


\newenvironment{exemple}[1][]{%
   \setbeamercolor{block title}{fg=black,bg=structure!20}
   \setbeamercolor{block body}{fg=black,bg=structure!5}
   \begin{block}{{\bf Exemple }#1}
}{%
   \end{block}%
}


\newenvironment{miniexercice}[0]{%
   \setbeamercolor{block title}{fg=structure,bg=structure!20}
   \setbeamercolor{block body}{fg=black,bg=structure!5}
   \begin{block}{Mini-exercices}
}{%
   \end{block}%
}


\newenvironment{tp}[0]{%
   \setbeamercolor{block title}{fg=structure,bg=structure!40}
   \setbeamercolor{block body}{fg=black,bg=structure!10}
   \begin{block}{\bf Travaux pratiques}
}{%
   \end{block}%
}
\newenvironment{exercicecours}[1][]{%
   \setbeamercolor{block title}{fg=structure,bg=structure!40}
   \setbeamercolor{block body}{fg=black,bg=structure!10}
   \begin{block}{{\bf Exercice }#1}
}{%
   \end{block}%
}
\newenvironment{algo}[1][]{%
   \setbeamercolor{block title}{fg=structure,bg=structure!40}
   \setbeamercolor{block body}{fg=black,bg=structure!10}
   \begin{block}{{\bf Algorithme}\hfill{\color{gray}\texttt{#1}}}
}{%
   \end{block}%
}


\setbeamertemplate{proof begin}{
   \setbeamercolor{block title}{fg=black,bg=structure!20}
   \setbeamercolor{block body}{fg=black,bg=structure!5}
   \begin{block}{{\footnotesize Démonstration}}
   \footnotesize
   \smallskip}
\setbeamertemplate{proof end}{%
   \end{block}}
\setbeamertemplate{qed symbol}{\openbox}


\makeatother
\usecolortheme[RGB={0,153,0}]{structure}

% Commande spécifique à ce chapitre
\newcommand{\GL}{\mathcal{G}\ell}
\newcounter{saveenumi}




%%%%%%%%%%%%%%%%%%%%%%%%%%%%%%%%%%%%%%%%%%%%%%%%%%%%%%%%%%%%%
%%%%%%%%%%%%%%%%%%%%%%%%%%%%%%%%%%%%%%%%%%%%%%%%%%%%%%%%%%%%%



\begin{document}



\title{{\bf Groupes}}
\subtitle{Morphismes de groupes}

\begin{frame}
  
  \debutmontitre

  \pause

{\footnotesize
\hfill
\setbeamercovered{transparent=50}
\begin{minipage}{0.6\textwidth}
  \begin{itemize}
    \item<3-> Définition
    \item<4-> Propriétés
    \item<5-> Noyau et image
    \item<6-> Exemples
  \end{itemize}
\end{minipage}
}

\end{frame}

\setcounter{framenumber}{0}


%%%%%%%%%%%%%%%%%%%%%%%%%%%%%%%%%%%%%%%%%%%%%%%%%%%%%%%%%%%%%%%%


%---------------------------------------------------------------
\section{Définition}

\begin{frame}
Soient $(G,\star)$ et $(G',\diamond)$ deux groupes
\begin{mydefinition}
$f : G \longrightarrow G'$ 
\pause
est un \defi{morphisme de groupes} si 
\mybox{$\text{pour tout } x,x' \in G  \qquad f(x \star x') = f(x) \diamond f(x')$}
\end{mydefinition}

\pause

\begin{exemple}
$G$ le groupe $(\Rr,+)$ \quad $G'$ le groupe $(\Rr_+^*,\times)$

$$
\begin{array}{lccc}
f : & \Rr & \longrightarrow & \Rr^*_+ \\
    & x  & \longmapsto & \exp(x) \\
\end{array}
$$

\pause

$f$ est un morphisme de groupes car
$$f(x+x') =\exp(x+x')= \exp(x) \times \exp(x') = f(x) \times f(x')$$
\end{exemple}

\end{frame}

%---------------------------------------------------------------
\section{Propriétés}

\begin{frame}


\begin{proposition}
Soit $f : G \longrightarrow G'$ un morphisme de groupes alors 
\begin{itemize}
  \item $f(e_G) = e_{G'}$
\pause
  \item pour tout $x \in G$,\ \  $f(x^{-1}) = \big(f(x)\big)^{-1}$
\end{itemize} 
\end{proposition}

\pause
\bigskip

\begin{exemple}
$f : \Rr \longrightarrow \Rr_+^*$ défini par $f(x)= \exp(x)$

\pause
\medskip

\centerline{$f(0)=1$ \pause\qquad $f(-x)= \exp(-x)= \frac{1}{\exp(x)}=\frac{1}{f(x)}$}
\end{exemple}
 
\end{frame}



\begin{frame}

\begin{proposition}
Soient deux morphismes de groupes $f : G \longrightarrow G'$ et $g : G'  \longrightarrow G''$

Alors $g \circ f : G \longrightarrow G''$ est un morphisme de groupes
\end{proposition}

\pause

\begin{proposition}
Si $f : G \longrightarrow G'$ est un morphisme bijectif 

Alors $f^{-1} : G' \longrightarrow G$ est aussi un morphisme de groupes
\end{proposition}

\pause

\begin{proof}
$$(G,\tikz[remember picture]\coordinate(invtop);\star) 
\overset{f}{\relbar\joinrel\relbar\joinrel\relbar\joinrel\relbar\joinrel\longrightarrow} 
(G',\tikz[remember picture]\coordinate(invbot);\diamond)$$

\pause


\begin{tikzpicture}[x=1mm,y=1mm, remember picture, overlay]
   \coordinate (mytop) at ($(invtop)+(-1,-2)$);
   \coordinate (mybot) at ($(invbot)+(-1,-2)$);
   \draw[->, myred, very thick] (mybot) to[bend left,thick]node[above, midway]{$f^{-1}$} (mytop);
\end{tikzpicture}

\pause

Soient $y,y'\in G'$

Comme $f$ est bijective, il existe $x,x'\in G$ tels que $f(x)=y$ et $f(x')=y'$
 $$f^{-1}(y\diamond y') = 
\pause
 f^{-1}\big(f(x)\diamond f(x') \big) 
\pause
= f^{-1}\big(f(x \star x') \big) 
\pause
= x\star x' 
\pause
= f^{-1}(y)\star f^{-1}(y')$$
\end{proof}  
\end{frame}


\begin{frame}
\begin{mydefinition}
Un morphisme bijectif est un \defi{isomorphisme}

$G, G'$ sont \defi{isomorphes}
s'il existe un morphisme bijectif $f : G \longrightarrow G'$
\end{mydefinition}
\pause
\begin{exemple}
$$
\begin{array}{lccc}
f : & \Rr & \longrightarrow & \Rr^*_+ \\
    & x  & \longmapsto & \exp(x) \\
\end{array}
$$
\pause
\begin{itemize}
  \item $f$ est une application bijective
\pause
  \item $f^{-1} :  \Rr_+^* \longrightarrow \Rr$ est définie par $f^{-1}(x) = \ln(x)$
\pause
  \item $f^{-1}$ est aussi un morphisme

$f^{-1}(x\times x')=f^{-1}(x) + f^{-1}(x')$

$\ln(x \times x') = \ln(x) + \ln(x')$
\pause
  \item $f$ est un isomorphisme
  \item les groupes $(\Rr,+)$ et $(\Rr_+^*,\times)$ sont
isomorphes
\end{itemize}
\end{exemple}
  
\end{frame}



%---------------------------------------------------------------
\section{Noyau et image}


\begin{frame}
Soit $f: G \longrightarrow G'$ un morphisme de groupes
\medskip

\begin{mydefinition}
Le \defi{noyau} de $f$ est 
\myboxinline{$\Ker f = \big\{ x \in G \mid f(x) = e_{G'} \big\}$}
\end{mydefinition}

\pause
\bigskip

\begin{itemize}
  \item $\Ker f \subset G$
\pause
  \item $\Ker f = f^{-1}\big(\{ e_{G'} \}\big)$
\pause
  \item $\Ker f$ sont les éléments de $G$ qui sont envoyés 
par $f$ sur l'élément neutre de $G'$
\end{itemize}  


\end{frame}

\begin{frame}

\begin{proposition}
\begin{enumerate}
  \item $\Ker f$ est un sous-groupe de $G$
\pause
  \item $f$ injectif $\iff$ $\Ker f = \{ e_G \}$
\end{enumerate}
\end{proposition}

\pause

\begin{proof}
\vspace*{-1ex}
\centerline{$f : (G,\star) 
{\relbar\joinrel\relbar\joinrel\relbar\joinrel\relbar\joinrel\longrightarrow} 
(G',\diamond)$}
\begin{enumerate}
  \item 

   \begin{itemize}   {\footnotesize
     \item $f(e_G)=e_{G'}$ donc $e_G \in \Ker f$
\pause
     \item Soient $x,x' \in \Ker f$, 

 $f(x\star x') = f(x) \diamond f(x')= e_{G'} \diamond e_{G'} = e_{G'}$ 
et donc $x \star x'\in \Ker f$
\pause
     \item Soit $x \in \Ker f$ \qquad
$f(x^{-1}) = f(x)^{-1}=e_{G'}^{-1} = e_{G'}$ et donc $x^{-1} \in \Ker f$
    }
    \end{itemize}        

\pause
   \item
   \begin{itemize}   {\footnotesize
     \item[$\Rightarrow$] Supposons $f$ est injective
 
Soit $x\in \Ker f$, alors $f(x)=e_{G'}$ donc $f(x)=f(e_G)$

Comme $f$ est injective alors $x=e_G$. Donc $\Ker f = \{ e_G \}$

\pause

     \item[$\Leftarrow$] Supposons $\Ker f = \{ e_G \}$

Soient $x,x' \in G$ tels que $f(x)=f(x')$,
donc $f(x)\diamond \big( f(x') \big)^{-1} = e_{G'}$ d'où $f(x)\diamond f(x'^{-1}) = e_{G'}$ et
donc $f(x \star x'^{-1}) = e_{G'}$

Ceci implique que $x \star x'^{-1} \in \Ker f$.
Comme $\Ker f = \{ e_G \}$ alors $x \star x'^{-1} = e_G$ et donc $x=x'$. Ainsi $f$ est injective
   }
   \end{itemize}
\end{enumerate}
\vspace*{-2ex}
\end{proof}

\end{frame}


\begin{frame}

Soit $f: G \longrightarrow G'$ un morphisme de groupes

\medskip

\begin{mydefinition}
L'\defi{image} de $f$ est 
\myboxinline{$\Im f = \big\{ f(x)  \mid x \in G \big\}$}
\end{mydefinition}

\pause


\begin{itemize}
  \item $\Im f \subset G'$
  \item $\Im f = f(G)$
  \item $\Im f$ sont les éléments de $G'$ qui ont (au moins) un antécédent par~$f$
\end{itemize}  

\pause 


\begin{proposition}
\begin{enumerate}
  \item $\Im f$ est un sous-groupe de $G'$
  \item $f$ surjectif $\iff$ $\Im f = G'$
\end{enumerate}
\end{proposition}
  
\end{frame}





%---------------------------------------------------------------
\section{Exemples}


\begin{frame}

\begin{exemple}

$$
\begin{array}{lccc}
f : & \Zz & \longrightarrow & \Zz  \\
    & k & \longmapsto & 3k  \\
\end{array}$$

\pause

\begin{itemize}
  \item $f$ est un morphisme de $(\Zz,+)$ dans $(\Zz,+)$

\pause

  $f(k+k')= 3(k+k')= 3k + 3k'=f(k)+f(k')$

\pause

   \item Calcul du noyau $\Ker f = \{ k \in \Zz \mid f(k)=0\}$

\pause

Si $f(k)=0$ alors $3k=0$ donc $k=0$

\pause

Ainsi $\Ker f = \{ 0 \}$  donc $f$ est injective

\pause

  \item $\Im f = \{ f(k) \mid k \in \Zz \} = \{ 3k \mid k \in\Zz\} = 3\Zz$

\pause

Nous retrouvons que $3\Zz$ est un sous-groupe de $(\Zz,+)$
\end{itemize}

\end{exemple}
\end{frame}

\begin{frame}
\begin{exemple}
$$
\begin{array}{lccc}
f : & \Rr & \longrightarrow & \mathbb{U}  \\
    & t   & \longmapsto  & e^{\ii t} \\
\end{array}
$$
\pause
\begin{itemize}
  \item $f$ est un morphisme

 $f(t+t')=e^{\ii (t+t')}= e^{\ii t} \times e^{\ii t'} = f(t)\times f(t')$

\pause

  \item Calcul du noyau $\Ker f = \{ t \in \Rr \mid f(t)=1 \}$

 $f(t)=1 \iff  e^{\ii t}= 1 \iff t= 0 \pmod{2\pi}$

D'où $\Ker f = \{ 2k\pi \mid k \in \Zz\} = 2\pi \Zz$

\pause

 Ainsi $f$ n'est pas injective

\pause
  \item $\Im f = \mathbb{U}$ car tout nombre complexe de module $1$ 
s'écrit sous la forme $f(t)=e^{\ii t}$
\end{itemize}
\end{exemple}

% \begin{exemple}
% $$
% \begin{array}{lccc}
% f : & G\ell_2 & \longrightarrow & \Rr^*  \\
%     & M & \longmapsto & \det M \\
% \end{array}
% $$
% \pause
% \begin{itemize}
%   \item $f$ est un morphisme de groupes
% 
% $\det(M\times M')=\det M \times \det M'$
% 
% \pause
% 
%   \item Le morphisme $f$ n'est pas injectif 
% 
% Par exemple $\det\left(\begin{smallmatrix} 1 & 0 \\ 0 & t \\ \end{smallmatrix}\right)=
% \det\left(\begin{smallmatrix} t & 0 \\ 0 & 1 \\ \end{smallmatrix}\right)$
% 
%   \item Le morphisme $f$ est surjectif 
% si $t \in \Rr^*$ alors $\det\left(\begin{smallmatrix} 1 & 0 \\ 0 & t \\ \end{smallmatrix}\right)
% = t$
% 
% \end{itemize}
% \end{exemple}
\end{frame}

\begin{frame}
Attention : ne pas confondre les notations
$$x^{-1}\qquad  f^{-1} \qquad f^{-1}\big(\{ e_{G'} \}\big)$$
\pause
\begin{itemize}
  \item $x^{-1}$ est l'inverse de $x$ dans un groupe $(G,\star)$

Si le groupe est $(\Rr^*,\times)$ alors $x^{-1}=\frac 1 x$
\pause

  \item Pour une application bijective $f^{-1}$ est la bijection réciproque
\pause

  \item Pour $f : E \longrightarrow F$ quelconque, l'image réciproque d'une partie $B\subset F$ est 
$f^{-1}(B) = \big\{ x \in E \mid f(x) = B \big\}$


Le noyau d'un morphisme $\Ker f = f^{-1}\big(\{ e_{G'} \}\big)$ est défini même si $f$ n'est pas bijective
\end{itemize}  
\end{frame}



%---------------------------------------------------------------
\section*{Mini-exercices}

\begin{frame}

\begin{miniexercice}
\begin{enumerate}
  \item Soit $f : (\Zz,+) \longrightarrow (\Qq^*,\times)$ défini par $f(n)=2^n$. Montrer que 
$f$ est un morphisme de groupes. Déterminer le noyau de $f$. $f$ est-elle injective ? surjective ?

  \item Mêmes questions pour $f : (\Rr,+) \longrightarrow (\mathcal{R},\circ)$, qui à un réel
$\theta$ associe la rotation d'angle $\theta$ de centre l'origine. 

  \item Soit $(G,\star)$ un groupe et $f : G \longrightarrow G$ l'application définie par
$f(x)=x^2$. (Rappel : $x^2=x\star x$.) Montrer que si $(G,\star)$ est commutatif alors $f$
est un morphisme. Montrer ensuite la réciproque.
  \item Montrer qu'il n'existe pas de morphisme $f : (\Zz,+) \to (\Zz,+)$ tel que $f(2)=3$.
  \item Montrer que $f, g : (\Rr^*,\times) \to (\Rr^*,\times)$ définis par $f(x)=x^2$,
$g(x)=x^3$ sont des morphismes de groupes. Calculer leur image et leur noyau.
\end{enumerate}


\end{miniexercice}
\end{frame}


\end{document}
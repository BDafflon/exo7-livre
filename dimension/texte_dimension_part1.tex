
%%%%%%%%%%%%%%%%%% PREAMBULE %%%%%%%%%%%%%%%%%%


\documentclass[12pt]{article}

\usepackage{amsfonts,amsmath,amssymb,amsthm}
\usepackage[utf8]{inputenc}
\usepackage[T1]{fontenc}
\usepackage[francais]{babel}


% packages
\usepackage{amsfonts,amsmath,amssymb,amsthm}
\usepackage[utf8]{inputenc}
\usepackage[T1]{fontenc}
%\usepackage{lmodern}

\usepackage[francais]{babel}
\usepackage{fancybox}
\usepackage{graphicx}

\usepackage{float}

%\usepackage[usenames, x11names]{xcolor}
\usepackage{tikz}
\usepackage{datetime}

\usepackage{mathptmx}
%\usepackage{fouriernc}
%\usepackage{newcent}
\usepackage[mathcal,mathbf]{euler}

%\usepackage{palatino}
%\usepackage{newcent}


% Commande spéciale prompteur

%\usepackage{mathptmx}
%\usepackage[mathcal,mathbf]{euler}
%\usepackage{mathpple,multido}

\usepackage[a4paper]{geometry}
\geometry{top=2cm, bottom=2cm, left=1cm, right=1cm, marginparsep=1cm}

\newcommand{\change}{{\color{red}\rule{\textwidth}{1mm}\\}}

\newcounter{mydiapo}

\newcommand{\diapo}{\newpage
\hfill {\normalsize  Diapo \themydiapo \quad \texttt{[\jobname]}} \\
\stepcounter{mydiapo}}


%%%%%%% COULEURS %%%%%%%%%%

% Pour blanc sur noir :
%\pagecolor[rgb]{0.5,0.5,0.5}
% \pagecolor[rgb]{0,0,0}
% \color[rgb]{1,1,1}



%\DeclareFixedFont{\myfont}{U}{cmss}{bx}{n}{18pt}
\newcommand{\debuttexte}{
%%%%%%%%%%%%% FONTES %%%%%%%%%%%%%
\renewcommand{\baselinestretch}{1.5}
\usefont{U}{cmss}{bx}{n}
\bfseries

% Taille normale : commenter le reste !
%Taille Arnaud
%\fontsize{19}{19}\selectfont

% Taille Barbara
%\fontsize{21}{22}\selectfont

%Taille François
%\fontsize{25}{30}\selectfont

%Taille Pascal
%\fontsize{25}{30}\selectfont

%Taille Laura
%\fontsize{30}{35}\selectfont


%\myfont
%\usefont{U}{cmss}{bx}{n}

%\Huge
%\addtolength{\parskip}{\baselineskip}
}


% \usepackage{hyperref}
% \hypersetup{colorlinks=true, linkcolor=blue, urlcolor=blue,
% pdftitle={Exo7 - Exercices de mathématiques}, pdfauthor={Exo7}}


%section
% \usepackage{sectsty}
% \allsectionsfont{\bf}
%\sectionfont{\color{Tomato3}\upshape\selectfont}
%\subsectionfont{\color{Tomato4}\upshape\selectfont}

%----- Ensembles : entiers, reels, complexes -----
\newcommand{\Nn}{\mathbb{N}} \newcommand{\N}{\mathbb{N}}
\newcommand{\Zz}{\mathbb{Z}} \newcommand{\Z}{\mathbb{Z}}
\newcommand{\Qq}{\mathbb{Q}} \newcommand{\Q}{\mathbb{Q}}
\newcommand{\Rr}{\mathbb{R}} \newcommand{\R}{\mathbb{R}}
\newcommand{\Cc}{\mathbb{C}} 
\newcommand{\Kk}{\mathbb{K}} \newcommand{\K}{\mathbb{K}}

%----- Modifications de symboles -----
\renewcommand{\epsilon}{\varepsilon}
\renewcommand{\Re}{\mathop{\text{Re}}\nolimits}
\renewcommand{\Im}{\mathop{\text{Im}}\nolimits}
%\newcommand{\llbracket}{\left[\kern-0.15em\left[}
%\newcommand{\rrbracket}{\right]\kern-0.15em\right]}

\renewcommand{\ge}{\geqslant}
\renewcommand{\geq}{\geqslant}
\renewcommand{\le}{\leqslant}
\renewcommand{\leq}{\leqslant}

%----- Fonctions usuelles -----
\newcommand{\ch}{\mathop{\mathrm{ch}}\nolimits}
\newcommand{\sh}{\mathop{\mathrm{sh}}\nolimits}
\renewcommand{\tanh}{\mathop{\mathrm{th}}\nolimits}
\newcommand{\cotan}{\mathop{\mathrm{cotan}}\nolimits}
\newcommand{\Arcsin}{\mathop{\mathrm{Arcsin}}\nolimits}
\newcommand{\Arccos}{\mathop{\mathrm{Arccos}}\nolimits}
\newcommand{\Arctan}{\mathop{\mathrm{Arctan}}\nolimits}
\newcommand{\Argsh}{\mathop{\mathrm{Argsh}}\nolimits}
\newcommand{\Argch}{\mathop{\mathrm{Argch}}\nolimits}
\newcommand{\Argth}{\mathop{\mathrm{Argth}}\nolimits}
\newcommand{\pgcd}{\mathop{\mathrm{pgcd}}\nolimits} 

\newcommand{\Card}{\mathop{\text{Card}}\nolimits}
\newcommand{\Ker}{\mathop{\text{Ker}}\nolimits}
\newcommand{\id}{\mathop{\text{id}}\nolimits}
\newcommand{\ii}{\mathrm{i}}
\newcommand{\dd}{\mathrm{d}}
\newcommand{\Vect}{\mathop{\text{Vect}}\nolimits}
\newcommand{\Mat}{\mathop{\mathrm{Mat}}\nolimits}
\newcommand{\rg}{\mathop{\text{rg}}\nolimits}
\newcommand{\tr}{\mathop{\text{tr}}\nolimits}
\newcommand{\ppcm}{\mathop{\text{ppcm}}\nolimits}

%----- Structure des exercices ------

\newtheoremstyle{styleexo}% name
{2ex}% Space above
{3ex}% Space below
{}% Body font
{}% Indent amount 1
{\bfseries} % Theorem head font
{}% Punctuation after theorem head
{\newline}% Space after theorem head 2
{}% Theorem head spec (can be left empty, meaning ‘normal’)

%\theoremstyle{styleexo}
\newtheorem{exo}{Exercice}
\newtheorem{ind}{Indications}
\newtheorem{cor}{Correction}


\newcommand{\exercice}[1]{} \newcommand{\finexercice}{}
%\newcommand{\exercice}[1]{{\tiny\texttt{#1}}\vspace{-2ex}} % pour afficher le numero absolu, l'auteur...
\newcommand{\enonce}{\begin{exo}} \newcommand{\finenonce}{\end{exo}}
\newcommand{\indication}{\begin{ind}} \newcommand{\finindication}{\end{ind}}
\newcommand{\correction}{\begin{cor}} \newcommand{\fincorrection}{\end{cor}}

\newcommand{\noindication}{\stepcounter{ind}}
\newcommand{\nocorrection}{\stepcounter{cor}}

\newcommand{\fiche}[1]{} \newcommand{\finfiche}{}
\newcommand{\titre}[1]{\centerline{\large \bf #1}}
\newcommand{\addcommand}[1]{}
\newcommand{\video}[1]{}

% Marge
\newcommand{\mymargin}[1]{\marginpar{{\small #1}}}



%----- Presentation ------
\setlength{\parindent}{0cm}

%\newcommand{\ExoSept}{\href{http://exo7.emath.fr}{\textbf{\textsf{Exo7}}}}

\definecolor{myred}{rgb}{0.93,0.26,0}
\definecolor{myorange}{rgb}{0.97,0.58,0}
\definecolor{myyellow}{rgb}{1,0.86,0}

\newcommand{\LogoExoSept}[1]{  % input : echelle
{\usefont{U}{cmss}{bx}{n}
\begin{tikzpicture}[scale=0.1*#1,transform shape]
  \fill[color=myorange] (0,0)--(4,0)--(4,-4)--(0,-4)--cycle;
  \fill[color=myred] (0,0)--(0,3)--(-3,3)--(-3,0)--cycle;
  \fill[color=myyellow] (4,0)--(7,4)--(3,7)--(0,3)--cycle;
  \node[scale=5] at (3.5,3.5) {Exo7};
\end{tikzpicture}}
}



\theoremstyle{definition}
%\newtheorem{proposition}{Proposition}
%\newtheorem{exemple}{Exemple}
%\newtheorem{theoreme}{Théorème}
\newtheorem{lemme}{Lemme}
\newtheorem{corollaire}{Corollaire}
%\newtheorem*{remarque*}{Remarque}
%\newtheorem*{miniexercice}{Mini-exercices}
%\newtheorem{definition}{Définition}




%definition d'un terme
\newcommand{\defi}[1]{{\color{myorange}\textbf{\emph{#1}}}}
\newcommand{\evidence}[1]{{\color{blue}\textbf{\emph{#1}}}}



 %----- Commandes divers ------

\newcommand{\codeinline}[1]{\texttt{#1}}

%%%%%%%%%%%%%%%%%%%%%%%%%%%%%%%%%%%%%%%%%%%%%%%%%%%%%%%%%%%%%
%%%%%%%%%%%%%%%%%%%%%%%%%%%%%%%%%%%%%%%%%%%%%%%%%%%%%%%%%%%%%


\begin{document}

\debuttexte


%%%%%%%%%%%%%%%%%%%%%%%%%%%%%%%%%%%%%%%%%%%%%%%%%%%%%%%%%%%
\diapo

\change
Cette leçon est la \defi{première} d'une série consacrées aux espaces vectoriels de \defi{dimension} finie. 
Pour définir la notion de dimension, nous avons besoin \defi{tout d'abord} de définir  
ce qu'est une famille \defi{libre} de vecteurs. C'est le but de cette vidéo. 

\change
Dans cette séquence, nous allons dans un premier temps \defi{rappeler} la définition de \defi{combinaison linéaire},

\change
puis nous verrons la \defi{définitions} de famille \defi{libre} de vecteurs,

\change
nous illustrerons cette définition par des \defi{exemples,} 

\change
puis nous décrirons \defi{dans quel cas} une famille de vecteurs \defi{n'est pas} libre, c'est-à-dire est \defi{liée},

\change 
et enfin nous en donnerons une \defi{interprétation géométrique}.


%%%%%%%%%%%%%%%%%%%%%%%%%%%%%%%%%%%%%%%%%%%%%%%%%%%%%%%%%%%
\diapo
Commen\c{c}ons par rappeler la définition de \defi{combinaison linéaire} de vecteurs.


\change
Considérons  $v_1, v_2, \ldots, v_p$, \defi{$p$}  vecteurs d'un $\Kk$-espace vectoriel $E$ 

\change
(nous supposerons que \defi{$p \geq 1$}, c'est-à-dire que nous avons au moins un vecteur). 

\change
Considérons également
  $\lambda_1, \lambda_2, \ldots,  \lambda_p$ des éléments du \defi{corps de base} $\Kk$.
  
\change
 Alors le vecteur $u$ défini comme la somme
 $\lambda_1 v_1+\lambda_2v_2+ \cdots + \lambda_pv_p$
 est appelé \defi{combinaison linéaire} des vecteurs $v_1, v_2, \ldots, v_p$
 
 \change
 et les scalaires $\lambda_1, \lambda_2, \ldots , \lambda_p$ qui apparaissent dans la somme
 sont appelés \defi{coefficients} de la combinaison linéaire. 


%%%%%%%%%%%%%%%%%%%%%%%%%%%%%%%%%%%%%%%%%%%%%%%%%%%%%%%%%%%
\diapo
Passons à la définition de famille libre de vecteurs.

\change
Une famille $\{ v_1, v_2,\ldots, v_p \}$ de $E$ est appelée une \defi{famille libre} ou 
aussi \defi{linéairement indépendante} 

\change
si toute combinaison linéaire \defi{nulle}
$$\lambda_1 v_1+\lambda_2 v_2+\cdots+\lambda_p v_p=0$$

\change
est telle que tous ses coefficients sont nuls, c'est-à-dire 
$\lambda_1=0$, $\lambda_2=0$, $\cdots$, $\lambda_p=0$. 

\change
Si la famille $\{ v_1, v_2,\ldots, v_p \}$ de $E$ \defi{n'est pas libre}, on dit que la famille est \defi{liée} ou \defi{linéairement dépendante}.

\change
Il existe alors une combinaison linéaire \defi{nulle}
\defi{avec au moins un coefficient non nul}.\\

Une telle combinaison linéaire s'appelle alors une \defi{relation de dépendance linéaire} entre les vecteurs.

%%%%%%%%%%%%%%%%%%%%%%%%%%%%%%%%%%%%%%%%%%%%%%%%%%%%%%%%%%%
\diapo
Voyons quelques exemples.
Dans l'espace vectoriel $\Rr^3$, considérons la famille des 3 vecteurs suivante

\change
Est-ce une \defi{famille libre} ou \defi{une famille liée}?

\change
Pour répondre à cette question, on cherche les scalaires $(\lambda_1, \lambda_2, \lambda_{3})$ tels que 

\change
$\lambda_1\times$ premier vecteur 

\change
+ $\lambda_2\times $ deuxi\`eme vecteur 

\change
+ $\lambda_3\times$ troisi\`eme vecteur
soit égal au vecteur nul.

\change
En regardant coordonnée par coordonnée, nous aboutissons au syst\`eme suivant

\change
premi\`ere ligne

\change
deuxi\`eme ligne

\change
troisi\`eme ligne


%%%%%%%%%%%%%%%%%%%%%%%%%%%%%%%%%%%%%%%%%%%%%%%%%%%%%%%%%%%
\diapo
En appliquant \defi{la méthode de réduction de Gauss} au syst\`eme linéaire précédent, 
nous pouvons montrer que le syst\`eme est équivalent au syst\`eme suivant, 
qui a une \defi{infinité de solutions}.

\change
Par exemple en affectant la valeur 1 à $\lambda_3$ nous en déduisons que $ \lambda_1=2$ et $ \lambda_2=-1$, 

\change
ce qui correspond à la relation linéaire suivante :\\
 2 x le premier vecteur - 1 x le deuxi\`eme vecteur + 1 x le troisi\`eme vecteur = le vecteur nul.


\change
Nous avons trouvé une \defi{relation de dépendance linéaire} entre les 3 vecteurs,
c'est-à-dire une combinaison linéaire \defi{nulle} des trois vecteurs 
avec au moins un des coefficients \defi{non nul} ;
nous pouvons en conclure que le famille est \defi{liée}.


%%%%%%%%%%%%%%%%%%%%%%%%%%%%%%%%%%%%%%%%%%%%%%%%%%%%%%%%%%%
\diapo
Considérons un nouvel exemple.

\change
La famille $v_1$, $v_2$, $v_3$ est-elle libre ou liée?

\change
 Posons
$\lambda_1 v_1+ \lambda_2 v_2 + \lambda_3 v_3 = 0$.

\change
Cette équation transcrite coordonnée par coordonnée donne lieu au syst\`eme suivant

\change
La résolution de ce syst\`eme conduit à une seule possibilité : c'est celle où tous les coefficients sont nuls.

\change 
On en conclut que la famille $\{v_1, v_2, v_3\}$ est une famille libre.

% 
% %%%%%%%%%%%%%%%%%%%%%%%%%%%%%%%%%%%%%%%%%%%%%%%%%%%%%%%%%%%
% \diapo
% Considérons maintenant cette famille de 3 vecteurs de l'espace vectoriel $\Rr^4$.
% 
% \change
% En regardant attentivement les coordonnées des 3 vecteurs, on s'aperçoit qu'ils sont reliés par la relation de dépendance linéaire suivante.
% 
% \change
% On en conclut que $\{v_1, v_2, v_3\}$ forme une famille liée.



%%%%%%%%%%%%%%%%%%%%%%%%%%%%%%%%%%%%%%%%%%%%%%%%%%%%%%%%%%%
\diapo
Changeons à présent de décor, et considérons le $\Rr$-espace vectoriel des polynômes à coefficients réels noté $\Rr[X]$. Soient $P_1$, $P_2$ et $P_3$ les trois polynômes suivants.

\change
Nous pouvons montrer que la relation de dépendance linéaire suivante existe entre ces 3 polynômes.

\change
La famille $\{P_1, P_2, P_3\}$ est donc liée 

%%%%%%%%%%%%%%%%%%%%%%%%%%%%%%%%%%%%%%%%%%%%%%%%%%%%%%%%%%%
\diapo
La proposition suivante décrit \defi{dans quel cas} une famille de deux vecteurs  est \defi{liée}.


La famille $\{ v_1, v_2\}$ est liée si
et seulement si $v_1$ est un multiple de $v_2$ ou 
$v_2$ est un multiple de $v_1$


La démonstration de cette proposition se fait en deux temps.

\change
Dans un premier temps considérons le cas où la famille $\{ v_1, v_2\}$ est liée. Dans ce cas il existe une combinaison linéaire nulle avec au moins un coefficient non nul

\change
si c'est le premier coefficient $\lambda_1$ qui est non nul, alors on peut diviser la relation $\lambda_1 v_1+\lambda_2 v_2= 0$ par $\lambda_1$, ce qui donne 
$ v_1=-\frac{\lambda_2}{\lambda_1} v_2$ autrement dit $ v_1$ est un multiple de $v_2$

\change
si c'est le deuxi\`eme coefficient $\lambda_2$ qui est non nul, alors en divisant par $\lambda_2$ on obtient $ v_2=-\frac{\lambda_1}{\lambda_2} v_1$ et $v_2$ est un multiple de $ v_1$. 


\change
Réciproquement si $ v_1$ est un multiple de $ v_2$, alors il
existe un  $\mu$ tel que $ v_1=\mu  v_2$, 

\change
autrement dit $1 v_1+(-\mu)
v_2= 0$
ce qui est une relation de dépendance linéaire entre $v_1$ et $ v_2$ 
puisque le premier coefficient vaut $1$ et est donc non nul. 

\change
la famille $\{ v_1, v_2\}$ est alors liée

\change
On arrive à la m\^eme conclusion si c'est $v_2$ qui est un multiple de $ v_1$.

%%%%%%%%%%%%%%%%%%%%%%%%%%%%%%%%%%%%%%%%%%%%%%%%%%%%%%%%%%%
\diapo
La proposition précédente se généralise à une famille d'un nombre quelconque de vecteurs pour donner le théor\`eme suivant :

\change
Une famille $\mathcal{F}=\{v_1, v_2,\ldots, v_p\}$  d'au moins deux vecteurs
est une famille \defi{liée} si et seulement si 
au moins un des vecteurs de $\mathcal{F}$ est combinaison linéaire 
des \defi{autres} vecteurs de $\mathcal{F}$

%%%%%%%%%%%%%%%%%%%%%%%%%%%%%%%%%%%%%%%%%%%%%%%%%%%%%%%%%%%
\diapo
Dans $\Rr^2$ ou $\Rr^3$  deux vecteurs sont linéairement \defi{dépendants} si et seulement s'ils sont colinéaires.
Ils sont donc sur une même droite vectorielle.


\change
Dans $\Rr^3$, trois vecteurs sont linéairement dépendants si et seulement si ils sont coplanaires.

\change
Ils sont donc dans un même plan vectoriel.

\change
Terminons en nous plaçons dans l'espace vectoriel $\Rr^n$.
Que peut-on dire lorsque l'on a fixé $p$ vecteurs. 
Et bien lorsque $p > n$, c'est-à-dire lorsque la famille 
 contient strictement plus de $n$ éléments, alors la famille ne peut pas être libre.
 
 C'est le sens de cette proposition :
  soit $\mathcal{F}$ une famille de vecteurs de $\Rr^n$. Si $\mathcal{F}$ est une famille
   de $p$ vecteurs dans $\Rr^n$ et que $p > n$, alors 
  $\mathcal{F}$ est une famille liée.
 

%%%%%%%%%%%%%%%%%%%%%%%%%%%%%%%%%%%%%%%%%%%%%%%%%%%%%%%%%%%
\diapo

Entrainez-vous à faire ces exercices pour vérifier que vous avez bien compris le cours.

\end{document}

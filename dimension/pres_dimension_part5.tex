
%%%%%%%%%%%%%%%%%% PREAMBULE %%%%%%%%%%%%%%%%%%

\documentclass[aspectratio=169,utf8]{beamer}
%\documentclass[aspectratio=169,handout]{beamer}

\usetheme{Boadilla}
%\usecolortheme{seahorse}
%\usecolortheme[RGB={245,66,24}]{structure}
\useoutertheme{infolines}

% packages
\usepackage{amsfonts,amsmath,amssymb,amsthm}
\usepackage[utf8]{inputenc}
\usepackage[T1]{fontenc}
\usepackage{lmodern}

\usepackage[francais]{babel}
\usepackage{fancybox}
\usepackage{graphicx}

\usepackage{float}
\usepackage{xfrac}

%\usepackage[usenames, x11names]{xcolor}
\usepackage{pgfplots}
\usepackage{datetime}


% ----------------------------------------------------------------------
% Pour les images
\usepackage{tikz}
\usetikzlibrary{calc,shadows,arrows.meta,patterns,matrix}

\newcommand{\tikzinput}[1]{\input{figures/#1.tikz}}
% --- les figures avec échelle éventuel
\newcommand{\myfigure}[2]{% entrée : échelle, fichier(s) figure à inclure
\begin{center}\small%
\tikzstyle{every picture}=[scale=1.0*#1]% mise en échelle + 0% (automatiquement annulé à la fin du groupe)
#2%
\end{center}}



%-----  Package unités -----
\usepackage{siunitx}
\sisetup{locale = FR,detect-all,per-mode = symbol}

%\usepackage{mathptmx}
%\usepackage{fouriernc}
%\usepackage{newcent}
%\usepackage[mathcal,mathbf]{euler}

%\usepackage{palatino}
%\usepackage{newcent}
% \usepackage[mathcal,mathbf]{euler}



% \usepackage{hyperref}
% \hypersetup{colorlinks=true, linkcolor=blue, urlcolor=blue,
% pdftitle={Exo7 - Exercices de mathématiques}, pdfauthor={Exo7}}


%section
% \usepackage{sectsty}
% \allsectionsfont{\bf}
%\sectionfont{\color{Tomato3}\upshape\selectfont}
%\subsectionfont{\color{Tomato4}\upshape\selectfont}

%----- Ensembles : entiers, reels, complexes -----
\newcommand{\Nn}{\mathbb{N}} \newcommand{\N}{\mathbb{N}}
\newcommand{\Zz}{\mathbb{Z}} \newcommand{\Z}{\mathbb{Z}}
\newcommand{\Qq}{\mathbb{Q}} \newcommand{\Q}{\mathbb{Q}}
\newcommand{\Rr}{\mathbb{R}} \newcommand{\R}{\mathbb{R}}
\newcommand{\Cc}{\mathbb{C}} 
\newcommand{\Kk}{\mathbb{K}} \newcommand{\K}{\mathbb{K}}

%----- Modifications de symboles -----
\renewcommand{\epsilon}{\varepsilon}
\renewcommand{\Re}{\mathop{\text{Re}}\nolimits}
\renewcommand{\Im}{\mathop{\text{Im}}\nolimits}
%\newcommand{\llbracket}{\left[\kern-0.15em\left[}
%\newcommand{\rrbracket}{\right]\kern-0.15em\right]}

\renewcommand{\ge}{\geqslant}
\renewcommand{\geq}{\geqslant}
\renewcommand{\le}{\leqslant}
\renewcommand{\leq}{\leqslant}
\renewcommand{\epsilon}{\varepsilon}

%----- Fonctions usuelles -----
\newcommand{\ch}{\mathop{\text{ch}}\nolimits}
\newcommand{\sh}{\mathop{\text{sh}}\nolimits}
\renewcommand{\tanh}{\mathop{\text{th}}\nolimits}
\newcommand{\cotan}{\mathop{\text{cotan}}\nolimits}
\newcommand{\Arcsin}{\mathop{\text{arcsin}}\nolimits}
\newcommand{\Arccos}{\mathop{\text{arccos}}\nolimits}
\newcommand{\Arctan}{\mathop{\text{arctan}}\nolimits}
\newcommand{\Argsh}{\mathop{\text{argsh}}\nolimits}
\newcommand{\Argch}{\mathop{\text{argch}}\nolimits}
\newcommand{\Argth}{\mathop{\text{argth}}\nolimits}
\newcommand{\pgcd}{\mathop{\text{pgcd}}\nolimits} 


%----- Commandes divers ------
\newcommand{\ii}{\mathrm{i}}
\newcommand{\dd}{\text{d}}
\newcommand{\id}{\mathop{\text{id}}\nolimits}
\newcommand{\Ker}{\mathop{\text{Ker}}\nolimits}
\newcommand{\Card}{\mathop{\text{Card}}\nolimits}
\newcommand{\Vect}{\mathop{\text{Vect}}\nolimits}
\newcommand{\Mat}{\mathop{\text{Mat}}\nolimits}
\newcommand{\rg}{\mathop{\text{rg}}\nolimits}
\newcommand{\tr}{\mathop{\text{tr}}\nolimits}


%----- Structure des exercices ------

\newtheoremstyle{styleexo}% name
{2ex}% Space above
{3ex}% Space below
{}% Body font
{}% Indent amount 1
{\bfseries} % Theorem head font
{}% Punctuation after theorem head
{\newline}% Space after theorem head 2
{}% Theorem head spec (can be left empty, meaning ‘normal’)

%\theoremstyle{styleexo}
\newtheorem{exo}{Exercice}
\newtheorem{ind}{Indications}
\newtheorem{cor}{Correction}


\newcommand{\exercice}[1]{} \newcommand{\finexercice}{}
%\newcommand{\exercice}[1]{{\tiny\texttt{#1}}\vspace{-2ex}} % pour afficher le numero absolu, l'auteur...
\newcommand{\enonce}{\begin{exo}} \newcommand{\finenonce}{\end{exo}}
\newcommand{\indication}{\begin{ind}} \newcommand{\finindication}{\end{ind}}
\newcommand{\correction}{\begin{cor}} \newcommand{\fincorrection}{\end{cor}}

\newcommand{\noindication}{\stepcounter{ind}}
\newcommand{\nocorrection}{\stepcounter{cor}}

\newcommand{\fiche}[1]{} \newcommand{\finfiche}{}
\newcommand{\titre}[1]{\centerline{\large \bf #1}}
\newcommand{\addcommand}[1]{}
\newcommand{\video}[1]{}

% Marge
\newcommand{\mymargin}[1]{\marginpar{{\small #1}}}

\def\noqed{\renewcommand{\qedsymbol}{}}


%----- Presentation ------
\setlength{\parindent}{0cm}

%\newcommand{\ExoSept}{\href{http://exo7.emath.fr}{\textbf{\textsf{Exo7}}}}

\definecolor{myred}{rgb}{0.93,0.26,0}
\definecolor{myorange}{rgb}{0.97,0.58,0}
\definecolor{myyellow}{rgb}{1,0.86,0}

\newcommand{\LogoExoSept}[1]{  % input : echelle
{\usefont{U}{cmss}{bx}{n}
\begin{tikzpicture}[scale=0.1*#1,transform shape]
  \fill[color=myorange] (0,0)--(4,0)--(4,-4)--(0,-4)--cycle;
  \fill[color=myred] (0,0)--(0,3)--(-3,3)--(-3,0)--cycle;
  \fill[color=myyellow] (4,0)--(7,4)--(3,7)--(0,3)--cycle;
  \node[scale=5] at (3.5,3.5) {Exo7};
\end{tikzpicture}}
}


\newcommand{\debutmontitre}{
  \author{} \date{} 
  \thispagestyle{empty}
  \hspace*{-10ex}
  \begin{minipage}{\textwidth}
    \titlepage  
  \vspace*{-2.5cm}
  \begin{center}
    \LogoExoSept{2.5}
  \end{center}
  \end{minipage}

  \vspace*{-0cm}
  
  % Astuce pour que le background ne soit pas discrétisé lors de la conversion pdf -> png
\begin{tikzpicture}
        \fill[opacity=0,green!60!black] (0,0)--++(0,0)--++(0,0)--++(0,0)--cycle; 
\end{tikzpicture}

% toc S'affiche trop tot :
% \tableofcontents[hideallsubsections, pausesections]
}

\newcommand{\finmontitre}{
  \end{frame}
  \setcounter{framenumber}{0}
} % ne marche pas pour une raison obscure

%----- Commandes supplementaires ------

% \usepackage[landscape]{geometry}
% \geometry{top=1cm, bottom=3cm, left=2cm, right=10cm, marginparsep=1cm
% }
% \usepackage[a4paper]{geometry}
% \geometry{top=2cm, bottom=2cm, left=2cm, right=2cm, marginparsep=1cm
% }

%\usepackage{standalone}


% New command Arnaud -- november 2011
\setbeamersize{text margin left=24ex}
% si vous modifier cette valeur il faut aussi
% modifier le decalage du titre pour compenser
% (ex : ici =+10ex, titre =-5ex

\theoremstyle{definition}
%\newtheorem{proposition}{Proposition}
%\newtheorem{exemple}{Exemple}
%\newtheorem{theoreme}{Théorème}
%\newtheorem{lemme}{Lemme}
%\newtheorem{corollaire}{Corollaire}
%\newtheorem*{remarque*}{Remarque}
%\newtheorem*{miniexercice}{Mini-exercices}
%\newtheorem{definition}{Définition}

% Commande tikz
\usetikzlibrary{calc}
\usetikzlibrary{patterns,arrows}
\usetikzlibrary{matrix}
\usetikzlibrary{fadings} 

%definition d'un terme
\newcommand{\defi}[1]{{\color{myorange}\textbf{\emph{#1}}}}
\newcommand{\evidence}[1]{{\color{blue}\textbf{\emph{#1}}}}
\newcommand{\assertion}[1]{\emph{\og#1\fg}}  % pour chapitre logique
%\renewcommand{\contentsname}{Sommaire}
\renewcommand{\contentsname}{}
\setcounter{tocdepth}{2}



%------ Encadrement ------

\usepackage{fancybox}


\newcommand{\mybox}[1]{
\setlength{\fboxsep}{7pt}
\begin{center}
\shadowbox{#1}
\end{center}}

\newcommand{\myboxinline}[1]{
\setlength{\fboxsep}{5pt}
\raisebox{-10pt}{
\shadowbox{#1}
}
}

%--------------- Commande beamer---------------
\newcommand{\beameronly}[1]{#1} % permet de mettre des pause dans beamer pas dans poly


\setbeamertemplate{navigation symbols}{}
\setbeamertemplate{footline}  % tiré du fichier beamerouterinfolines.sty
{
  \leavevmode%
  \hbox{%
  \begin{beamercolorbox}[wd=.333333\paperwidth,ht=2.25ex,dp=1ex,center]{author in head/foot}%
    % \usebeamerfont{author in head/foot}\insertshortauthor%~~(\insertshortinstitute)
    \usebeamerfont{section in head/foot}{\bf\insertshorttitle}
  \end{beamercolorbox}%
  \begin{beamercolorbox}[wd=.333333\paperwidth,ht=2.25ex,dp=1ex,center]{title in head/foot}%
    \usebeamerfont{section in head/foot}{\bf\insertsectionhead}
  \end{beamercolorbox}%
  \begin{beamercolorbox}[wd=.333333\paperwidth,ht=2.25ex,dp=1ex,right]{date in head/foot}%
    % \usebeamerfont{date in head/foot}\insertshortdate{}\hspace*{2em}
    \insertframenumber{} / \inserttotalframenumber\hspace*{2ex} 
  \end{beamercolorbox}}%
  \vskip0pt%
}


\definecolor{mygrey}{rgb}{0.5,0.5,0.5}
\setlength{\parindent}{0cm}
%\DeclareTextFontCommand{\helvetica}{\fontfamily{phv}\selectfont}

% background beamer
\definecolor{couleurhaut}{rgb}{0.85,0.9,1}  % creme
\definecolor{couleurmilieu}{rgb}{1,1,1}  % vert pale
\definecolor{couleurbas}{rgb}{0.85,0.9,1}  % blanc
\setbeamertemplate{background canvas}[vertical shading]%
[top=couleurhaut,middle=couleurmilieu,midpoint=0.4,bottom=couleurbas] 
%[top=fondtitre!05,bottom=fondtitre!60]



\makeatletter
\setbeamertemplate{theorem begin}
{%
  \begin{\inserttheoremblockenv}
  {%
    \inserttheoremheadfont
    \inserttheoremname
    \inserttheoremnumber
    \ifx\inserttheoremaddition\@empty\else\ (\inserttheoremaddition)\fi%
    \inserttheorempunctuation
  }%
}
\setbeamertemplate{theorem end}{\end{\inserttheoremblockenv}}

\newenvironment{theoreme}[1][]{%
   \setbeamercolor{block title}{fg=structure,bg=structure!40}
   \setbeamercolor{block body}{fg=black,bg=structure!10}
   \begin{block}{{\bf Th\'eor\`eme }#1}
}{%
   \end{block}%
}


\newenvironment{proposition}[1][]{%
   \setbeamercolor{block title}{fg=structure,bg=structure!40}
   \setbeamercolor{block body}{fg=black,bg=structure!10}
   \begin{block}{{\bf Proposition }#1}
}{%
   \end{block}%
}

\newenvironment{corollaire}[1][]{%
   \setbeamercolor{block title}{fg=structure,bg=structure!40}
   \setbeamercolor{block body}{fg=black,bg=structure!10}
   \begin{block}{{\bf Corollaire }#1}
}{%
   \end{block}%
}

\newenvironment{mydefinition}[1][]{%
   \setbeamercolor{block title}{fg=structure,bg=structure!40}
   \setbeamercolor{block body}{fg=black,bg=structure!10}
   \begin{block}{{\bf Définition} #1}
}{%
   \end{block}%
}

\newenvironment{lemme}[0]{%
   \setbeamercolor{block title}{fg=structure,bg=structure!40}
   \setbeamercolor{block body}{fg=black,bg=structure!10}
   \begin{block}{\bf Lemme}
}{%
   \end{block}%
}

\newenvironment{remarque}[1][]{%
   \setbeamercolor{block title}{fg=black,bg=structure!20}
   \setbeamercolor{block body}{fg=black,bg=structure!5}
   \begin{block}{Remarque #1}
}{%
   \end{block}%
}


\newenvironment{exemple}[1][]{%
   \setbeamercolor{block title}{fg=black,bg=structure!20}
   \setbeamercolor{block body}{fg=black,bg=structure!5}
   \begin{block}{{\bf Exemple }#1}
}{%
   \end{block}%
}


\newenvironment{miniexercice}[0]{%
   \setbeamercolor{block title}{fg=structure,bg=structure!20}
   \setbeamercolor{block body}{fg=black,bg=structure!5}
   \begin{block}{Mini-exercices}
}{%
   \end{block}%
}


\newenvironment{tp}[0]{%
   \setbeamercolor{block title}{fg=structure,bg=structure!40}
   \setbeamercolor{block body}{fg=black,bg=structure!10}
   \begin{block}{\bf Travaux pratiques}
}{%
   \end{block}%
}
\newenvironment{exercicecours}[1][]{%
   \setbeamercolor{block title}{fg=structure,bg=structure!40}
   \setbeamercolor{block body}{fg=black,bg=structure!10}
   \begin{block}{{\bf Exercice }#1}
}{%
   \end{block}%
}
\newenvironment{algo}[1][]{%
   \setbeamercolor{block title}{fg=structure,bg=structure!40}
   \setbeamercolor{block body}{fg=black,bg=structure!10}
   \begin{block}{{\bf Algorithme}\hfill{\color{gray}\texttt{#1}}}
}{%
   \end{block}%
}


\setbeamertemplate{proof begin}{
   \setbeamercolor{block title}{fg=black,bg=structure!20}
   \setbeamercolor{block body}{fg=black,bg=structure!5}
   \begin{block}{{\footnotesize Démonstration}}
   \footnotesize
   \smallskip}
\setbeamertemplate{proof end}{%
   \end{block}}
\setbeamertemplate{qed symbol}{\openbox}


\makeatother
\usecolortheme[RGB={150,93,42}]{structure}
   
%%%%%%%%%%%%%%%%%%%%%%%%%%%%%%%%%%%%%%%%%%%%%%%%%%%%%%%%%%%%%
%%%%%%%%%%%%%%%%%%%%%%%%%%%%%%%%%%%%%%%%%%%%%%%%%%%%%%%%%%%%%


\begin{document}


\title{{\bf Dimension finie}}
\subtitle{Dimension des sous-espaces vectoriels}

\begin{frame}
  
  \debutmontitre

  \pause

{\footnotesize
\hfill
\setbeamercovered{transparent=50}
\begin{minipage}{0.6\textwidth}
  \begin{itemize}
    \item<3-> Dimension d'un sous-espace
    \item<4-> Exemples
    \item<5-> Théorème des quatre dimensions
  \end{itemize}
\end{minipage}
}

\end{frame}

\setcounter{framenumber}{0}


%%%%%%%%%%%%%%%%%%%%%%%%%%%%%%%%%%%%%%%%%%%%%%%%%%%%%%%%%%%%%%%%
\section{Dimension d'un sous-espace}

\begin{frame}
Soit $E$ un $\Kk$-espace vectoriel de dimension finie
\pause
\begin{theoreme}
\begin{enumerate}
  \item Tout sous-espace vectoriel $F$ de $E$ est de dimension finie 
  \pause
  \item $\dim F \le \dim E$ 
  \pause
  \item $\dim F = \dim E \iff F=E$
\end{enumerate}
\end{theoreme}


\end{frame}




%%%%%%%%%%%%%%%%%%%%%%%%%%%%%%%%%%%%%%%%%%%%%%%%%%%%%%%%%%%%%%%%
\section{Exemples}

\begin{frame}
\begin{exemple}
\begin{itemize}
\item 
Soit $E$ un $\Kk$-espace vectoriel, $\dim E = 2$ 
\pause
\item
Quels sont les sous-espaces vectoriels $F$ de $E$?
\pause
  \item Si $\dim F = 0$, $F = \{0\}$ 
  \pause
  \item Si $\dim F = 1$, $F = \Kk u = \Vect \{ u \}$, droite vectorielle  engendr\'ee par $u\neq0$, $u\in E$
  \pause
  \item Si $\dim F = 2$, $F = E$
\end{itemize}
\end{exemple}

\pause
\bigskip

Dans un $\Kk$-espace vectoriel $E$ de dimension $n$ 
\pause
\begin{itemize}
  \item un sous-espace vectoriel de dimension $1$ est 
une \defi{droite vectorielle}
\pause
  \item un sous-espace vectoriel de dimension $2$ est appelé \defi{plan vectoriel}
\pause
  \item un sous-espace vectoriel de dimension $n-1$ est 
appelé \defi{hyperplan}
\end{itemize}


\end{frame}


\begin{frame}
\begin{corollaire}
Soient $F$ et $G$ sous-espaces vectoriels d'un $\Kk$-espace vectoriel $E$ \\
avec $F$ de dimension finie et $G \subset F$
Alors :
$$F=G \iff \dim F = \dim G$$
\end{corollaire}

\pause
\begin{exemple}
\begin{itemize}
\item
$F=\Big\{\left(\begin{smallmatrix}x\\y\\z\end{smallmatrix}\right) \in \Rr^{3}\mid 2x-3y+z=0\Big\}$
\pause
\item
$
G = \Vect (u,v)\quad
u=\left(\begin{smallmatrix}1\\1\\1\end{smallmatrix}\right) \quad
v=\left(\begin{smallmatrix}2\\1\\-1\end{smallmatrix}\right)$ 

\pause

\item 
Est-ce que $F=G$ ?

\pause
\begin{itemize}
\item $\dim G = 2$ car $u$ et $v$ ne sont pas colin\'eaires

\pause
\item
$u\in F$, $v\in F$ donc $G\subset F$
  
  \pause
  \item $\dim F < \dim \Rr^3 = 3$ car  $\left(\begin{smallmatrix}1\\0\\0\end{smallmatrix}\right)\notin F$
  
  \pause
  \item
 $\dim F \ge \dim G = 2$ car $G\subset F$
 

 
 \pause
 \item  
  $G \subset F$, $\dim G = \dim F = 2$ $\Rightarrow G=F$
  \end{itemize}
\end{itemize}
\end{exemple}

\end{frame}




%%%%%%%%%%%%%%%%%%%%%%%%%%%%%%%%%%%%%%%%%%%%%%%%%%%%%%%%%%%%%%%%
\section{Théorème des quatre dimensions}

\begin{frame}
Soit $E$ un espace vectoriel de dimension finie \\
Soient $F,G$ des sous-espaces vectoriels de $E$
\begin{theoreme}[des quatre dimensions]
\mybox{$\dim(F+G) = \dim F + \dim G - \dim (F\cap G)$} 
\end{theoreme}

\medskip
\pause

\begin{corollaire}
Si $E = F \oplus G$, alors $\dim E = \dim F + \dim G$
\end{corollaire}

\medskip
\pause

\begin{corollaire}
Tout sous-espace vectoriel $F$ d'un espace vectoriel $E$
de dimension finie admet un supplémentaire
\end{corollaire}

\end{frame}


\begin{frame}
\begin{exemple}
\begin{itemize}
\item
$\dim E = 6 \quad \dim F = 3 \quad \dim G = 4$

\pause
\item
Que peut-on dire de $F\cap G$ ? de $F+G$ ? Peut-on avoir $F\oplus G=E$ ?
  \pause

  \item $F \cap G$  inclus dans $F$  $\implies \dim (F \cap G) \le \dim F=3$
 
  
  \pause
  \item $G \subset (F+G) \subset E$ 
  $\implies 4=\dim G \le \dim(F+G) \le \dim E=6$
  
    \pause
  \item 
  $\dim(F\cap G) =  \dim F + \dim G - \dim(F+G) =  7-\dim(F+G)$

  \pause
  \item Conclusion : $\dim (F+G) = 4$, $5$ ou $6$ ; $\dim F \cap G = 3$, $2$ ou $1$
  
  \pause
  \item $\dim F \cap G \geq 1\implies F \cap G \neq \{0\}$ et  $F$ et $G$ ne sont jamais en somme directe
  dans $E$

\end{itemize}
\end{exemple}
\end{frame}



\begin{frame}
\begin{proof}[Preuve du théorème \ref{th:4dim}]
\pause
\begin{itemize}
  \item Pour $A$ et $B$ ensembles finis
$\Card (A\cup B) = \Card A + \Card B - \Card (A\cap B)$
  
  \pause 
  \item $\{u_1,\ldots,u_p\}$ base de $F \cap G$
  
  \pause
  \item
$ \{u_1,\ldots,u_p,v_{p+1},\ldots,v_q\}$ base de $F$

\pause
\item
$ \{u_1,\ldots,u_p,w_{p+1},\ldots,w_r\}$ base de $G$

  \pause
  \item 
  $\{ u_1,\ldots,u_p,v_{p+1},\ldots,v_q,w_{p+1},\ldots,w_r\}$ famille g\'en\'eratrice de  $F+G$
  
  
  \pause
  \item $\{ u_1,\ldots,u_p,v_{p+1},\ldots,v_q,w_{p+1},\ldots,w_r\}$  est-elle libre ?
  
  \pause
 $$
  \sum_{i=1}^p \alpha_i u_i + \sum_{j=p+1}^q \beta_j v_j + \sum_{k=p+1}^r \gamma_k w_k = 0
 $$
 
 \pause
 \item
  $u=\sum_{i=1}^p \alpha_i u_i \qquad v=\sum_{j=p+1}^q \beta_j v_j \qquad w=\sum_{k=p+1}^r \gamma_k w_k$
  
  \pause
  \item $u+v+w = 0\Rightarrow  u+v = -w \in F\cap G$
  
  
  
  \pause
  \item    
  $u\in F\cap G$ et $u+v  \in F\cap G\Rightarrow v = (u+v) - u\in F\cap G$
$\Rightarrow \beta_j=0$ pour tout $j$
\pause
\item  
 Alors 
  $\sum_{i=1}^p \alpha_i u_i  + \sum_{k=p+1}^r \gamma_k w_k = 0\Rightarrow \alpha_i = 0 $ et  $\gamma_k = 0$
 pour tout $i,k$
 
 \pause
 \item
  Ainsi $ \{ u_1,\ldots,u_p,v_{p+1},\ldots,v_q,w_{p+1},\ldots,w_r\}$ est une base de $F+G$
  \qedhere
\end{itemize}
\end{proof}

\end{frame}

%%%%%%%%%%%%%%%%%%%%%%%%%%%%%%%%%%%%%%%%%%%%%%%%%%%%%%%%%%%%%%%%
\section{Mini-exercices}

\begin{frame}
\begin{miniexercice}
\vspace*{-1ex}
\begin{enumerate}
  \item Soient 
  $F = \Vect \left( \left(\begin{smallmatrix}1\\2\\3\end{smallmatrix}\right),
  \left(\begin{smallmatrix}3\\-1\\2 \end{smallmatrix}\right)\right)$
  et $G = \Vect \left( \left(\begin{smallmatrix}-7\\7\\0\end{smallmatrix}\right),
  \left(\begin{smallmatrix}6\\5\\11 \end{smallmatrix}\right)\right)$.
  Montrer que $F=G$.
  
  \item Dans $\Rr^3$, on considère 
  $F = \Vect \left( \left(\begin{smallmatrix}1\\t\\-1\end{smallmatrix}\right),
  \left(\begin{smallmatrix}t\\1\\1 \end{smallmatrix}\right)\right)$, 
  $G=\Vect \left(\begin{smallmatrix}1\\1\\1 \end{smallmatrix}\right)$. Calculer les dimensions
  de $F,G,F\cap G,F+G$ en fonction de $t\in \Rr$.
  
  \item Dans un espace vectoriel de dimension $7$, on considère des sous-espaces $F$ et $G$ vérifiant
  $\dim F=3$ et $\dim G \le 2$. Que peut-on dire pour $\dim (F\cap G)$ ? Et pour $\dim (F+G)$ ?

  \item Dans un espace vectoriel $E$ de dimension finie, montrer l'équivalence entre :
  (i) $F\oplus G = E$~; 
  (ii) $F+G=E$ et $\dim F + \dim G = \dim E$ ;
  (iii) $F \cap G = \{0_E\}$ et $\dim F + \dim G = \dim E$.
  
  \item Soit $H$ un hyperplan dans un espace vectoriel de dimension finie $E$. Soit $v\in E\setminus H$.
  Montrer que $H$ et $\Vect(v)$ sont des sous-espaces supplémentaires dans $E$.

\end{enumerate}
\vspace*{-2ex}
\end{miniexercice}
\end{frame}

\end{document}
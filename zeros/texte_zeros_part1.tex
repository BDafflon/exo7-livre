
%%%%%%%%%%%%%%%%%% PREAMBULE %%%%%%%%%%%%%%%%%%


\documentclass[12pt]{article}

\usepackage{amsfonts,amsmath,amssymb,amsthm}
\usepackage[utf8]{inputenc}
\usepackage[T1]{fontenc}
\usepackage[francais]{babel}


% packages
\usepackage{amsfonts,amsmath,amssymb,amsthm}
\usepackage[utf8]{inputenc}
\usepackage[T1]{fontenc}
%\usepackage{lmodern}

\usepackage[francais]{babel}
\usepackage{fancybox}
\usepackage{graphicx}

\usepackage{float}

%\usepackage[usenames, x11names]{xcolor}
\usepackage{tikz}
\usepackage{datetime}

\usepackage{mathptmx}
%\usepackage{fouriernc}
%\usepackage{newcent}
\usepackage[mathcal,mathbf]{euler}

%\usepackage{palatino}
%\usepackage{newcent}


% Commande spéciale prompteur

%\usepackage{mathptmx}
%\usepackage[mathcal,mathbf]{euler}
%\usepackage{mathpple,multido}

\usepackage[a4paper]{geometry}
\geometry{top=2cm, bottom=2cm, left=1cm, right=1cm, marginparsep=1cm}

\newcommand{\change}{{\color{red}\rule{\textwidth}{1mm}\\}}

\newcounter{mydiapo}

\newcommand{\diapo}{\newpage
\hfill {\normalsize  Diapo \themydiapo \quad \texttt{[\jobname]}} \\
\stepcounter{mydiapo}}


%%%%%%% COULEURS %%%%%%%%%%

% Pour blanc sur noir :
%\pagecolor[rgb]{0.5,0.5,0.5}
% \pagecolor[rgb]{0,0,0}
% \color[rgb]{1,1,1}



%\DeclareFixedFont{\myfont}{U}{cmss}{bx}{n}{18pt}
\newcommand{\debuttexte}{
%%%%%%%%%%%%% FONTES %%%%%%%%%%%%%
\renewcommand{\baselinestretch}{1.5}
\usefont{U}{cmss}{bx}{n}
\bfseries

% Taille normale : commenter le reste !
%Taille Arnaud
%\fontsize{19}{19}\selectfont

% Taille Barbara
%\fontsize{21}{22}\selectfont

%Taille François
%\fontsize{25}{30}\selectfont

%Taille Pascal
%\fontsize{25}{30}\selectfont

%Taille Laura
%\fontsize{30}{35}\selectfont


%\myfont
%\usefont{U}{cmss}{bx}{n}

%\Huge
%\addtolength{\parskip}{\baselineskip}
}


% \usepackage{hyperref}
% \hypersetup{colorlinks=true, linkcolor=blue, urlcolor=blue,
% pdftitle={Exo7 - Exercices de mathématiques}, pdfauthor={Exo7}}


%section
% \usepackage{sectsty}
% \allsectionsfont{\bf}
%\sectionfont{\color{Tomato3}\upshape\selectfont}
%\subsectionfont{\color{Tomato4}\upshape\selectfont}

%----- Ensembles : entiers, reels, complexes -----
\newcommand{\Nn}{\mathbb{N}} \newcommand{\N}{\mathbb{N}}
\newcommand{\Zz}{\mathbb{Z}} \newcommand{\Z}{\mathbb{Z}}
\newcommand{\Qq}{\mathbb{Q}} \newcommand{\Q}{\mathbb{Q}}
\newcommand{\Rr}{\mathbb{R}} \newcommand{\R}{\mathbb{R}}
\newcommand{\Cc}{\mathbb{C}} 
\newcommand{\Kk}{\mathbb{K}} \newcommand{\K}{\mathbb{K}}

%----- Modifications de symboles -----
\renewcommand{\epsilon}{\varepsilon}
\renewcommand{\Re}{\mathop{\text{Re}}\nolimits}
\renewcommand{\Im}{\mathop{\text{Im}}\nolimits}
%\newcommand{\llbracket}{\left[\kern-0.15em\left[}
%\newcommand{\rrbracket}{\right]\kern-0.15em\right]}

\renewcommand{\ge}{\geqslant}
\renewcommand{\geq}{\geqslant}
\renewcommand{\le}{\leqslant}
\renewcommand{\leq}{\leqslant}

%----- Fonctions usuelles -----
\newcommand{\ch}{\mathop{\mathrm{ch}}\nolimits}
\newcommand{\sh}{\mathop{\mathrm{sh}}\nolimits}
\renewcommand{\tanh}{\mathop{\mathrm{th}}\nolimits}
\newcommand{\cotan}{\mathop{\mathrm{cotan}}\nolimits}
\newcommand{\Arcsin}{\mathop{\mathrm{Arcsin}}\nolimits}
\newcommand{\Arccos}{\mathop{\mathrm{Arccos}}\nolimits}
\newcommand{\Arctan}{\mathop{\mathrm{Arctan}}\nolimits}
\newcommand{\Argsh}{\mathop{\mathrm{Argsh}}\nolimits}
\newcommand{\Argch}{\mathop{\mathrm{Argch}}\nolimits}
\newcommand{\Argth}{\mathop{\mathrm{Argth}}\nolimits}
\newcommand{\pgcd}{\mathop{\mathrm{pgcd}}\nolimits} 

\newcommand{\Card}{\mathop{\text{Card}}\nolimits}
\newcommand{\Ker}{\mathop{\text{Ker}}\nolimits}
\newcommand{\id}{\mathop{\text{id}}\nolimits}
\newcommand{\ii}{\mathrm{i}}
\newcommand{\dd}{\mathrm{d}}
\newcommand{\Vect}{\mathop{\text{Vect}}\nolimits}
\newcommand{\Mat}{\mathop{\mathrm{Mat}}\nolimits}
\newcommand{\rg}{\mathop{\text{rg}}\nolimits}
\newcommand{\tr}{\mathop{\text{tr}}\nolimits}
\newcommand{\ppcm}{\mathop{\text{ppcm}}\nolimits}

%----- Structure des exercices ------

\newtheoremstyle{styleexo}% name
{2ex}% Space above
{3ex}% Space below
{}% Body font
{}% Indent amount 1
{\bfseries} % Theorem head font
{}% Punctuation after theorem head
{\newline}% Space after theorem head 2
{}% Theorem head spec (can be left empty, meaning ‘normal’)

%\theoremstyle{styleexo}
\newtheorem{exo}{Exercice}
\newtheorem{ind}{Indications}
\newtheorem{cor}{Correction}


\newcommand{\exercice}[1]{} \newcommand{\finexercice}{}
%\newcommand{\exercice}[1]{{\tiny\texttt{#1}}\vspace{-2ex}} % pour afficher le numero absolu, l'auteur...
\newcommand{\enonce}{\begin{exo}} \newcommand{\finenonce}{\end{exo}}
\newcommand{\indication}{\begin{ind}} \newcommand{\finindication}{\end{ind}}
\newcommand{\correction}{\begin{cor}} \newcommand{\fincorrection}{\end{cor}}

\newcommand{\noindication}{\stepcounter{ind}}
\newcommand{\nocorrection}{\stepcounter{cor}}

\newcommand{\fiche}[1]{} \newcommand{\finfiche}{}
\newcommand{\titre}[1]{\centerline{\large \bf #1}}
\newcommand{\addcommand}[1]{}
\newcommand{\video}[1]{}

% Marge
\newcommand{\mymargin}[1]{\marginpar{{\small #1}}}



%----- Presentation ------
\setlength{\parindent}{0cm}

%\newcommand{\ExoSept}{\href{http://exo7.emath.fr}{\textbf{\textsf{Exo7}}}}

\definecolor{myred}{rgb}{0.93,0.26,0}
\definecolor{myorange}{rgb}{0.97,0.58,0}
\definecolor{myyellow}{rgb}{1,0.86,0}

\newcommand{\LogoExoSept}[1]{  % input : echelle
{\usefont{U}{cmss}{bx}{n}
\begin{tikzpicture}[scale=0.1*#1,transform shape]
  \fill[color=myorange] (0,0)--(4,0)--(4,-4)--(0,-4)--cycle;
  \fill[color=myred] (0,0)--(0,3)--(-3,3)--(-3,0)--cycle;
  \fill[color=myyellow] (4,0)--(7,4)--(3,7)--(0,3)--cycle;
  \node[scale=5] at (3.5,3.5) {Exo7};
\end{tikzpicture}}
}



\theoremstyle{definition}
%\newtheorem{proposition}{Proposition}
%\newtheorem{exemple}{Exemple}
%\newtheorem{theoreme}{Théorème}
\newtheorem{lemme}{Lemme}
\newtheorem{corollaire}{Corollaire}
%\newtheorem*{remarque*}{Remarque}
%\newtheorem*{miniexercice}{Mini-exercices}
%\newtheorem{definition}{Définition}




%definition d'un terme
\newcommand{\defi}[1]{{\color{myorange}\textbf{\emph{#1}}}}
\newcommand{\evidence}[1]{{\color{blue}\textbf{\emph{#1}}}}



 %----- Commandes divers ------

\newcommand{\codeinline}[1]{\texttt{#1}}

%%%%%%%%%%%%%%%%%%%%%%%%%%%%%%%%%%%%%%%%%%%%%%%%%%%%%%%%%%%%%
%%%%%%%%%%%%%%%%%%%%%%%%%%%%%%%%%%%%%%%%%%%%%%%%%%%%%%%%%%%%%



\begin{document}

\debuttexte


%%%%%%%%%%%%%%%%%%%%%%%%%%%%%%%%%%%%%%%%%%%%%%%%%%%%%%%%%%%
\diapo


Dans ces leçons nous allons appliquer toutes les notions sur les suites et les fonctions,
à la recherche des zéros des fonctions. Plus précisément, nous allons voir trois méthodes afin
de trouver des approximations des solutions d'une équation du type $(f(x)=0)$ :
La dichotomie, la méthode de la sécante, et la méthode de Newton.


\change

On commence par la méthode de la dichotomie,

\change

on en explique le principe,

\change

on l'applique au calcul d'une approximation de $\sqrt{10}$

\change

et de $(1,10)^{1/12}$

\change

On donne ensuite une majoration de l'erreur

\change

Et on termine par la mise en oeuvre pratique sous la forme d'un algorithme.


%%%%%%%%%%%%%%%%%%%%%%%%%%%%%%%%%%%%%%%%%%%%%%%%%%%%%%%%%%
\diapo

Le principe de dichotomie repose sur la version suivante du théorème des valeurs intermédiaires :

Soit $f:[a,b]\to\Rr$ une fonction continue sur un segment. 
Si $f(a)\cdot f(b) \le 0$, alors il existe $\ell\in[a,b]$ tel que $f(\ell)=0$.



La condition $f(a)\cdot f(b)\le0$ signifie que $f(a)$ et $f(b)$ 
sont de signes opposés (ou que l'un des deux est nul). 


\change

Par exemple ici $f(a)$ négatif et $f(b)$ positif.
La fonction s'annule ici en $\ell$.


\change

L'autre situation est $f(a)$ positif et $f(b)$ négatif.

La fonction admet un zéro ici en $\ell$.

L'hypothèse de continuité est bien sûr essentielle !


%%%%%%%%%%%%%%%%%%%%%%%%%%%%%%%%%%%%%%%%%%%%%%%%%%%%%%%%%%%
\diapo

Pour rendre effectif le théorème des valeurs intermédiaires, et trouver une solution (approchée) de l'équation $(f(x)=0)$, 
il s'agit maintenant de l'appliquer sur un intervalle suffisamment petit. 
On va voir que cela permet d'obtenir un $\ell$, solution de l'équation $(f(x)=0)$ comme la limite d'une suite.

On part d'une fonction $f : [a,b] \to \Rr$ continue, avec $a < b$, et $f(a)\cdot f(b)\le0$. 

\change

Voici la première étape de la construction : on évalue la fonction au point milieu $\frac{a+b}{2}$ 
et on regarde le signe.

\change

Si $f(a)\cdot f(\frac{a+b}{2})\le0$, alors par le théorème des valeurs intermédiaires
entre $a$ et $\frac{a+b}{2}]$ il existe 
  $c$ dans cet intervalle tel que $f(c)=0$.
  
\change

Voici une telle situation.

\change

Si par contre $f(a)\cdot f(\frac{a+b}{2})>0$, alors nécessairement c'est 
  $f(\frac{a+b}{2})\cdot f(b)$ qui est négatif, et on applique le théorème des valeurs intermédiaires
  entre $\frac{a+b}{2}$ et $b$, et alors il existe $c$ dans cet intervalle qui est un zéro de $f$.
  
\change 

Voici un exemple pour ce deuxième cas.

\change


Nous avons obtenu dans les deux cas un intervalle d'une longueur divisé par $2$ dans 
lequel l'équation $(f(x)=0)$ a une solution. On va itérer ce processus
pour diviser de nouveau l'intervalle en deux.



%%%%%%%%%%%%%%%%%%%%%%%%%%%%%%%%%%%%%%%%%%%%%%%%%%%%%%%%%%
\diapo

Voici le processus complet :


On commence par poser $a_0=a$, $b_0=b$. Comme $f(a_0)$ et $f(b_0)$ sont de signe contraire,
il existe au moins une solution, que l'on appelle $x_0$ de l'équation $(f(x)=0)$ dans l'intervalle $[a_0,b_0]$.

\change

On divise notre intervalle $[a_0,b_0]$ en deux, on évalue la fonction à la moitié  

si $f(a_0)$ et $f(\frac{a_0+b_0}{2})$ sont de signe contraire alors on pose
$a_1=a_0$ et $b_1=\frac{a_0+b_0}{2}$, 

 sinon on pose $a_1=\frac{a_0+b_0}{2}$ et $b_1=b$.
 
 Dans les deux cas, il existe une solution $x_1$ de l'équation $(f(x)=0)$ 
 dans l'intervalle $[a_1,b_1]$. 
 
 \change
 
 On itère ainsi le processus en divisant l'intervalle précédent en deux à chaque fois.
 
 
 \change
 
 Voici l'étape au rang $n$ : supposons construit un intervalle $[a_n,b_n]$, 
 %de longueur $\frac{b-a}{2^n}$ et 
 contenant une solution $x_n$ de l'équation $(f(x)=0)$. Alors :

 On évalue $f$ à la moitié de l'intervalle $[a_n,b_n]$,
 
si $f(a_n)$ et $f(\frac{a_n+b_n}{2})$ sont de signe contraire alors
on   pose $a_{n+1}=a_n$ et $b_{n+1}=\frac{a_n+b_n}{2}$, 

sinon on pose $a_{n+1}=\frac{a_n+b_n}{2}$ et $b_{n+1}=b_n$.  

Dans les deux cas, il existe une solution $x_{n+1}$ de l'équation $(f(x)=0)$ dans l'intervalle $[a_{n+1},b_{n+1}]$. 



%%%%%%%%%%%%%%%%%%%%%%%%%%%%%%%%%%%%%%%%%%%%%%%%%%%%%%%%%%%
\diapo

On a construit une suite $(a_n)$ et une suite $(b_n)$ qui correspondent 
aux bornes des intervalles et on a l'existence d'une suite $(x_n)$ qui sont des zéros de $f$.

Et on a la double inégalité $a_n \le x_n \le b_n.$


\change


Comme $(a_n)$ est par construction une suite croissante, $(b_n)$ une suite décroissante, 
et $(b_n-a_n) \to 0$ lorsque $n\to +\infty$, les suites $(a_n)$ et $(b_n)$ sont des suites 
adjacentes et donc elles admettent une même limite. 

D'après le théorème des suites adjacentes, $(a_n)$ et $(b_n)$ convergent et ont une même limite $\ell$

\change

en plus par le théorème des gendarmes $(x_n)$ tend aussi vers $\ell$.

\change

Par continuité $f(x_n)$ tend vers $f(\ell)$
mais comme pour tout $n$, $f(x_n)=0$ alors $f(\ell)$ vaut aussi $0$.

\change

Donc les suites $(a_n)$ et $(b_n)$ tendent toutes les deux vers $\ell$, 
qui est un zéro de $f$.

\change

Dans la pratique on arrête le processus dès que $b_n-a_n=\frac{b-a}{2^n}$ 
est inférieur à la précision souhaitée.

%%%%%%%%%%%%%%%%%%%%%%%%%%%%%%%%%%%%%%%%%%%%%%%%%%%%%%%%%%
\diapo

Nous allons calculer une approximation de $\sqrt{10}$.


\change

Soit la fonction $f$ définie par $f(x)=x^2 - 10$, c'est une fonction continue sur $\Rr$
qui s'annule en $\pm\sqrt{10}$. Alors $\sqrt{10}$ est l'unique solution positive de l'équation $(f(x)=0)$.


\change

Nous pouvons restreindre la fonction $f$ à l'intervalle $[3,4]$ : en effet $3^2=9\le 10$ donc 
$3 \le \sqrt{10}$ et $4^2 = 16 \ge 10$ donc $4 \ge \sqrt{10}$. 
En d'autre termes $f(3) \le 0$ et $f(4) \ge 0$, donc l'équation $(f(x)=0)$ 
admet une solution dans l'intervalle $[3,4]$ 
d'après le théorème des valeurs intermédiaires, 
et par unicité c'est $\sqrt{10}$, donc $\sqrt{10} \in [3,4]$.

Notez que l'on ne choisit pas pour $f$ la fonction $x\mapsto x-\sqrt{10}$ 
car on ne connaît pas la valeur de $\sqrt{10}$. C'est ce que l'on cherche à calculer !




%%%%%%%%%%%%%%%%%%%%%%%%%%%%%%%%%%%%%%%%%%%%%%%%%%%%%%%%%%%
\diapo


Expliquons d'abord le processus sur le graphique.

On a vu que $f(3)$ est négatif, $f(4)$ est positif.

On évalue la fonction en le milieu de l'intervalle $[3;4]$ donc en $3,5$.

Comme $f(3,5)$ est positif (et que $f(3)$ est toujours négatif) alors
$f$ s'annule maintenant entre $3$ et $3,5$.


On continue en évaluant $f$ en $3,25$ où $f$ est encore positif. Donc
$f$ s'annule entre $3$ et $3,25$.

Par contre au rang suivant $f$ est négative en $3,125$ ce qui fait que 
$f$ s'annule entre $3,125$ et $3,25$.




%%%%%%%%%%%%%%%%%%%%%%%%%%%%%%%%%%%%%%%%%%%%%%%%%%%%%%%%%%

\diapo

Reprenons numériquement ce que l'on vient de voir graphiquement.

On pose $a_0=3$ et $b_0=4$,


On calcule $f(\frac{a_0+b_0}{2})$ qui vaut $f(3,5)=3,5^2-10 = 2,25 \ge0$. 

Donc $f$ s'annule dans l'intervalle $[3 ; 3,5]$ 



Pour l'itération suivante on part de $a_1 = 3$ et $b_1 = 3,5$.

\change


Au début de l'étape suivante on sait donc que $f(a_1) \le 0$ et $f(b_1) \ge0$. 

On calcule $f(\frac{a_1+b_1}{2})=f(3,25)=0,5625 \ge 0$, on pose $a_2=3$ et $b_2=3,25$.
  
\change

On calcule $f(\frac{a_2+b_2}{2})=f(3,125)=-0.23\ldots \le 0$.
  Comme $f(b_2) \ge 0$ alors cette fois $f$ s'annule sur le second intervalle 
  $[\frac{a_2+b_2}{2},b_2]$ 
  

\`A ce stade, on a prouvé : $3,125 \le \sqrt{10} \le 3,25$.


%%%%%%%%%%%%%%%%%%%%%%%%%%%%%%%%%%%%%%%%%%%%%%%%%%%%%%%%%%%
\diapo

  
Voici la suite des étapes :

Donc en $8$ étapes on obtient l'encadrement :
$$3,160  \le \sqrt{10} \le 3,165$$

En particulier, on vient d'obtenir deux décimales exactes après la virgule :
$\sqrt{10}=3,16\ldots$


%%%%%%%%%%%%%%%%%%%%%%%%%%%%%%%%%%%%%%%%%%%%%%%%%%%%%%%%%%%
\diapo

Évidemment tout ce que l'on vient de faire pour $\sqrt{10}$,

nous pouvons le faire pour $(1,10)^{1/12}$.

Cette fois la fonction qui convient est 
 $f(x) = x^{12} - 1,10$ sont seul zéro positif est bien $(1,10)^{1/12}$
 et comme $f(1)=-0,10 \le 0$ et $f(1,1)=2,038\ldots \ge 0$
 
 alors on se retreint à l'intervalle $[1; 1,1]$. 
 
Voici les premières étapes de la dichotomie.
et au bout de $8$ étapes on trouve que $(1,10)^{1/12}$ est compris entre 
$ 1,00781$ et $1,00821$

%%%%%%%%%%%%%%%%%%%%%%%%%%%%%%%%%%%%%%%%%%%%%%%%%%%%%%%%%%%
\diapo

La méthode de dichotomie a l'énorme avantage de fournir un encadrement d'une solution $\ell$ de l'équation $(f(x)=0)$. 
Il est donc facile d'avoir une majoration de l'erreur. En effet, à chaque étape, 
la taille l'intervalle contenant $\ell$ est divisée par $2$. 
Au départ, on sait que $\ell \in [a,b]$ (de longueur $b-a$) ;
puis $\ell  \in [a_1,b_1]$ (de longueur $\frac{b-a}{2}$) ; 
puis $\ell \in [a_2,b_2]$ (de longueur $\frac{b-a}{4}$) ; ... ;
$[a_n,b_n]$ est de longueur $\frac{b-a}{2^n}$.

\change

[grand $N$, petit $n$]

Si, par exemple, on souhaite obtenir une approximation de $\ell$ à $10^{-N}$ près, 
 il suffit de choisir $n$ tel que  $\frac{b-a}{2^n} \le 10^{-N}$.

Nous allons utiliser le logarithme décimal pour résoudre cette inéquation :
\begin{align*}
\frac{b-a}{2^n} \le 10^{-N} 
 & \iff (b-a)10^N \le 2^n  \\
 & \iff \log(b-a) + \log(10^N)  \le \log(2^n) \\
 & \iff \log(b-a) + N \le n \log 2 \\
 & \iff n \ge \frac{N +  \log(b-a)}{\log 2} \\
\end{align*}
  
\change

Sachant $\log 2 = 0,301\ldots$, si par exemple $b-a \le 1$, 
voici le nombre d'itérations suffisantes pour avoir une 
précision de $10^{-N}$ 
(ce qui correspond, à peu près, à $N$ chiffres exacts après la virgule).
\begin{center}
\begin{tabular}{ll}
  $10^{-10}$ ($\sim 10$ décimales) &  $34$ itérations \\
  $10^{-100}$ ($\sim 100$ décimales) &  $333$ itérations \\ 
  $10^{-1000}$ ($\sim 1000$ décimales) &  $3322$ itérations \\ 
\end{tabular}  
\end{center}

Il faut entre $3$ et $4$ itérations supplémentaires pour obtenir une nouvelle décimale.
 


%%%%%%%%%%%%%%%%%%%%%%%%%%%%%%%%%%%%%%%%%%%%%%%%%%%%%%%%%%%
\diapo

Voici comment implémenter la dichotomie dans le langage Python. Tout d'abord 
on définit une fonction $f$, ici on prend l'exemple de $f(x)=x^2-10$, pour calculer $\sqrt{10}$.


Puis la dichotomie proprement dite : en entrée de la fonction "dicho", on a 
pour variables $a$ et $b$ les bornes de l'intervalle initial et $n$ le nombre d'étapes voulues.

Cet algorithme est constitué d'une boucle qui correspond aux $n$ étapes.

Pour chaque étape on commence par calculer le milieu $c$ du segment $[a,b]$,

Puis on teste les valeurs. 

Si $f(a)$ et $f(c)$ sont de signe contraire alors on pose $b=c$, c'est-à-dire
que maintenant on se restreint au premier demi-intervalle.

Sinon on pose $a=c$, c'est-à-dire on se restreint au second intervalle.

L'intervalle $[a,b]$ change à chaque étape, il correspond en fait 
à l'intervalle $[a_i,b_i]$ du principe de la dichotomie.


Une fois que les $n$ étapes sont effectuées, on renvoie les dernières valeurs de $a$ et $b$,
c'est à dire ce que l'on avait noté $a_n$ et $b_n$ [$a$ indice $n$, et $b$ indice $n$]
et on sait que $f$ s'annule entre ces valeurs. 



%%%%%%%%%%%%%%%%%%%%%%%%%%%%%%%%%%%%%%%%%%%%%%%%%%%%%%%%%%%
\diapo

Voici quelques exercices pratiques pour vous entraîner.



\end{document}

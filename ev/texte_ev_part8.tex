
%%%%%%%%%%%%%%%%%% PREAMBULE %%%%%%%%%%%%%%%%%%


\documentclass[12pt]{article}

\usepackage{amsfonts,amsmath,amssymb,amsthm}
\usepackage[utf8]{inputenc}
\usepackage[T1]{fontenc}
\usepackage[francais]{babel}


% packages
\usepackage{amsfonts,amsmath,amssymb,amsthm}
\usepackage[utf8]{inputenc}
\usepackage[T1]{fontenc}
%\usepackage{lmodern}

\usepackage[francais]{babel}
\usepackage{fancybox}
\usepackage{graphicx}

\usepackage{float}

%\usepackage[usenames, x11names]{xcolor}
\usepackage{tikz}
\usepackage{datetime}

\usepackage{mathptmx}
%\usepackage{fouriernc}
%\usepackage{newcent}
\usepackage[mathcal,mathbf]{euler}

%\usepackage{palatino}
%\usepackage{newcent}


% Commande spéciale prompteur

%\usepackage{mathptmx}
%\usepackage[mathcal,mathbf]{euler}
%\usepackage{mathpple,multido}

\usepackage[a4paper]{geometry}
\geometry{top=2cm, bottom=2cm, left=1cm, right=1cm, marginparsep=1cm}

\newcommand{\change}{{\color{red}\rule{\textwidth}{1mm}\\}}

\newcounter{mydiapo}

\newcommand{\diapo}{\newpage
\hfill {\normalsize  Diapo \themydiapo \quad \texttt{[\jobname]}} \\
\stepcounter{mydiapo}}


%%%%%%% COULEURS %%%%%%%%%%

% Pour blanc sur noir :
%\pagecolor[rgb]{0.5,0.5,0.5}
% \pagecolor[rgb]{0,0,0}
% \color[rgb]{1,1,1}



%\DeclareFixedFont{\myfont}{U}{cmss}{bx}{n}{18pt}
\newcommand{\debuttexte}{
%%%%%%%%%%%%% FONTES %%%%%%%%%%%%%
\renewcommand{\baselinestretch}{1.5}
\usefont{U}{cmss}{bx}{n}
\bfseries

% Taille normale : commenter le reste !
%Taille Arnaud
%\fontsize{19}{19}\selectfont

% Taille Barbara
%\fontsize{21}{22}\selectfont

%Taille François
%\fontsize{25}{30}\selectfont

%Taille Pascal
%\fontsize{25}{30}\selectfont

%Taille Laura
%\fontsize{30}{35}\selectfont


%\myfont
%\usefont{U}{cmss}{bx}{n}

%\Huge
%\addtolength{\parskip}{\baselineskip}
}


% \usepackage{hyperref}
% \hypersetup{colorlinks=true, linkcolor=blue, urlcolor=blue,
% pdftitle={Exo7 - Exercices de mathématiques}, pdfauthor={Exo7}}


%section
% \usepackage{sectsty}
% \allsectionsfont{\bf}
%\sectionfont{\color{Tomato3}\upshape\selectfont}
%\subsectionfont{\color{Tomato4}\upshape\selectfont}

%----- Ensembles : entiers, reels, complexes -----
\newcommand{\Nn}{\mathbb{N}} \newcommand{\N}{\mathbb{N}}
\newcommand{\Zz}{\mathbb{Z}} \newcommand{\Z}{\mathbb{Z}}
\newcommand{\Qq}{\mathbb{Q}} \newcommand{\Q}{\mathbb{Q}}
\newcommand{\Rr}{\mathbb{R}} \newcommand{\R}{\mathbb{R}}
\newcommand{\Cc}{\mathbb{C}} 
\newcommand{\Kk}{\mathbb{K}} \newcommand{\K}{\mathbb{K}}

%----- Modifications de symboles -----
\renewcommand{\epsilon}{\varepsilon}
\renewcommand{\Re}{\mathop{\text{Re}}\nolimits}
\renewcommand{\Im}{\mathop{\text{Im}}\nolimits}
%\newcommand{\llbracket}{\left[\kern-0.15em\left[}
%\newcommand{\rrbracket}{\right]\kern-0.15em\right]}

\renewcommand{\ge}{\geqslant}
\renewcommand{\geq}{\geqslant}
\renewcommand{\le}{\leqslant}
\renewcommand{\leq}{\leqslant}

%----- Fonctions usuelles -----
\newcommand{\ch}{\mathop{\mathrm{ch}}\nolimits}
\newcommand{\sh}{\mathop{\mathrm{sh}}\nolimits}
\renewcommand{\tanh}{\mathop{\mathrm{th}}\nolimits}
\newcommand{\cotan}{\mathop{\mathrm{cotan}}\nolimits}
\newcommand{\Arcsin}{\mathop{\mathrm{Arcsin}}\nolimits}
\newcommand{\Arccos}{\mathop{\mathrm{Arccos}}\nolimits}
\newcommand{\Arctan}{\mathop{\mathrm{Arctan}}\nolimits}
\newcommand{\Argsh}{\mathop{\mathrm{Argsh}}\nolimits}
\newcommand{\Argch}{\mathop{\mathrm{Argch}}\nolimits}
\newcommand{\Argth}{\mathop{\mathrm{Argth}}\nolimits}
\newcommand{\pgcd}{\mathop{\mathrm{pgcd}}\nolimits} 

\newcommand{\Card}{\mathop{\text{Card}}\nolimits}
\newcommand{\Ker}{\mathop{\text{Ker}}\nolimits}
\newcommand{\id}{\mathop{\text{id}}\nolimits}
\newcommand{\ii}{\mathrm{i}}
\newcommand{\dd}{\mathrm{d}}
\newcommand{\Vect}{\mathop{\text{Vect}}\nolimits}
\newcommand{\Mat}{\mathop{\mathrm{Mat}}\nolimits}
\newcommand{\rg}{\mathop{\text{rg}}\nolimits}
\newcommand{\tr}{\mathop{\text{tr}}\nolimits}
\newcommand{\ppcm}{\mathop{\text{ppcm}}\nolimits}

%----- Structure des exercices ------

\newtheoremstyle{styleexo}% name
{2ex}% Space above
{3ex}% Space below
{}% Body font
{}% Indent amount 1
{\bfseries} % Theorem head font
{}% Punctuation after theorem head
{\newline}% Space after theorem head 2
{}% Theorem head spec (can be left empty, meaning ‘normal’)

%\theoremstyle{styleexo}
\newtheorem{exo}{Exercice}
\newtheorem{ind}{Indications}
\newtheorem{cor}{Correction}


\newcommand{\exercice}[1]{} \newcommand{\finexercice}{}
%\newcommand{\exercice}[1]{{\tiny\texttt{#1}}\vspace{-2ex}} % pour afficher le numero absolu, l'auteur...
\newcommand{\enonce}{\begin{exo}} \newcommand{\finenonce}{\end{exo}}
\newcommand{\indication}{\begin{ind}} \newcommand{\finindication}{\end{ind}}
\newcommand{\correction}{\begin{cor}} \newcommand{\fincorrection}{\end{cor}}

\newcommand{\noindication}{\stepcounter{ind}}
\newcommand{\nocorrection}{\stepcounter{cor}}

\newcommand{\fiche}[1]{} \newcommand{\finfiche}{}
\newcommand{\titre}[1]{\centerline{\large \bf #1}}
\newcommand{\addcommand}[1]{}
\newcommand{\video}[1]{}

% Marge
\newcommand{\mymargin}[1]{\marginpar{{\small #1}}}



%----- Presentation ------
\setlength{\parindent}{0cm}

%\newcommand{\ExoSept}{\href{http://exo7.emath.fr}{\textbf{\textsf{Exo7}}}}

\definecolor{myred}{rgb}{0.93,0.26,0}
\definecolor{myorange}{rgb}{0.97,0.58,0}
\definecolor{myyellow}{rgb}{1,0.86,0}

\newcommand{\LogoExoSept}[1]{  % input : echelle
{\usefont{U}{cmss}{bx}{n}
\begin{tikzpicture}[scale=0.1*#1,transform shape]
  \fill[color=myorange] (0,0)--(4,0)--(4,-4)--(0,-4)--cycle;
  \fill[color=myred] (0,0)--(0,3)--(-3,3)--(-3,0)--cycle;
  \fill[color=myyellow] (4,0)--(7,4)--(3,7)--(0,3)--cycle;
  \node[scale=5] at (3.5,3.5) {Exo7};
\end{tikzpicture}}
}



\theoremstyle{definition}
%\newtheorem{proposition}{Proposition}
%\newtheorem{exemple}{Exemple}
%\newtheorem{theoreme}{Théorème}
\newtheorem{lemme}{Lemme}
\newtheorem{corollaire}{Corollaire}
%\newtheorem*{remarque*}{Remarque}
%\newtheorem*{miniexercice}{Mini-exercices}
%\newtheorem{definition}{Définition}




%definition d'un terme
\newcommand{\defi}[1]{{\color{myorange}\textbf{\emph{#1}}}}
\newcommand{\evidence}[1]{{\color{blue}\textbf{\emph{#1}}}}



 %----- Commandes divers ------

\newcommand{\codeinline}[1]{\texttt{#1}}

%%%%%%%%%%%%%%%%%%%%%%%%%%%%%%%%%%%%%%%%%%%%%%%%%%%%%%%%%%%%%
%%%%%%%%%%%%%%%%%%%%%%%%%%%%%%%%%%%%%%%%%%%%%%%%%%%%%%%%%%%%%



\begin{document}

\debuttexte


%%%%%%%%%%%%%%%%%%%%%%%%%%%%%%%%%%%%%%%%%%%%%%%%%%%%%%%%%%%
\diapo

\change

Voici la partie fondamentale sur les applications linéaires.

\change

Nous allons définir l'image d'une application linéaire

\change

ainsi que le noyau.

\change

On parlera de l'espace vectoriel des applications linéaires entre deux espaces vectoriels.

\change

Et on terminera avec la composition et l'inverse d'applications linéaires.



%%%%%%%%%%%%%%%%%%%%%%%%%%%%%%%%%%%%%%%%%%%%%%%%%%%%%%%%%%
\diapo
     
Commençons par des rappels.
Soient $E$ et $F$ deux ensembles et $f$ une application de $E$ dans $F$. 
Soit $A$ un sous-ensemble de $E$. 
L'ensemble des images par $f$ des éléments de $A$, est appelé \defi{image directe} 
de $A$ par $f$, et est noté $f(A)$. C'est un sous-ensemble de $F$. On a par définition :
$$f(A)=\big\{ f(x)  \mid x\in A \big\}.$$

\change

Dans toute la suite, $E$ et $F$ désigneront des $\Kk$-espaces vectoriels
et $f : E \to F$ sera une application linéaire.

$f(E)$ s'appelle l'\defi{image} de l'application linéaire $f$ et est noté $\Im f$. 

\change

Voyons quelle structure a l'image d'un sous-espace vectoriel :


 Si $E'$ est un sous-espace vectoriel de $E$, alors $f(E')$ est un sous-espace vectoriel de $F$.
 
 Autrement dit l'image d'un sous-espace vectoriel par une application linéaire est un sous-espace vectoriel.

\change
 
Et le cas qui nous intéresse le plus, c'est l'image : on a aussi que $\Im f$ est un sous-espace vectoriel de $F$.

C'est un conséquence du premier point car $\Im f = f(E)$.

\change

Enfin par définition de l'image directe $f(E)$ : 
$f$ est surjective si et seulement si $\Im f =F$.

Donc pour savoir si une application, en particulier une application linéaire, est surjective, on étudie son image.



%%%%%%%%%%%%%%%%%%%%%%%%%%%%%%%%%%%%%%%%%%%%%%%%%%%%%%%%%%%
\diapo

Attardons nous sur la preuve de la proposition précédente :
l'image d'un sous-espace vectoriel par une application linéaire est un sous-espace vectoriel.

\change


Tout d'abord, $0_{E} \in E'$  

mais une application linéaire envoie le vecteur nul sur le vecteur nul ainsi
$0_{F} = f(0_{E})$ et donc $0_{F}$ appartient à l'image directe $f(E')$.

\change


Ensuite on montre que pour toute paire $y_1,y_2$ d'éléments de l'image directe $f(E')$
et pour tous scalaires $\lambda,\mu$, la combinaison linéaire $\lambda y_1 + \mu y_2$ appartient aussi à $f(E')$.

\change

Comme $y_1 \in f(E')$ alors il existe $x_1\in E' , f(x_1)=y_1$.

\change

De même comme $y_2 \in f(E')$, il existe $x_2\in E' , f(x_2)=y_2$.

\change

Or $\lambda x_1 + \mu x_2$ est un élément de $E'$, car $E'$ est un 
sous-espace vectoriel de $E$,

\change

Enfin comme $f$ est linéaire, on a 
$$f(\lambda x_1+ \mu x_2)=\lambda f(x_1)+ \mu f(x_2)=\lambda y_1 + \mu y_2.$$

\change

donc $\lambda y_1 + \mu y_2$ est bien un élément de $f(E')$.

%%%%%%%%%%%%%%%%%%%%%%%%%%%%%%%%%%%%%%%%%%%%%%%%%%%%%%%%%%
\diapo

Passons à une notion plus délicate, celle du noyau d'une application linéaire

Soient $E$ et $F$ deux $\Kk$-espaces vectoriels et $f$ une application linéaire de $E$ dans $F$.

Le \defi{noyau} de $f$ est l'ensemble des 
éléments de $E$ dont l'image est $0_{F}$ :

Le noyau est noté  $\Ker(f)$.

Par définition $\Ker (f)$ est donc $\big\{x \in E \mid f(x)=0_{F}\big\}$

\change


 
 Autrement dit, le noyau est l'image réciproque du vecteur nul de l'espace d'arrivée :
 $\Ker(f) = f^{-1} \{0_F\}$.
 
\change

Comme l'image, le noyau est un sous-espace vectoriel.

Proposition : Si $f$ une application linéaire entre deux espaces vectoriels $E$ et $F$.
Alors le noyau de $f$ est un sous-espace vectoriel de $E$.

\change

Prouvons le !

Tout d'abord $\Ker(f)$ est non vide car $f(0_E)=0_F$ donc $0_{E} \in \Ker (f)$. 

\change

Soient maintenant $x_1,x_2$ deux vecteurs du noyau et soient $\lambda,\mu$ deux scalaires.

Il s'agit de montrer que $\lambda x_1+\mu x_2$ est un élément de $\Ker (f)$. 

\change

Calculons 
$f(\lambda x_1+\mu x_2)$

\change

Comme $f$ est linaire alors
$f(\lambda x_1+\mu x_2)=\lambda f(x_1)+\mu f(x_2)$

\change

et comme $x_1$ et $x_2$ sont dans le noyau alors $f(x_1)=0$ et $f(x_2)=0$.

\change

Donc $f(\lambda x_1+\mu x_2) =0_{F}.$

Ce qui est exactement dire que $\lambda x_1+\mu x_2$ est un élément du noyau.


%%%%%%%%%%%%%%%%%%%%%%%%%%%%%%%%%%%%%%%%%%%%%%%%%%%%%%%%%%
\diapo

Reprenons l'exemple de l'application linéaire
$f : \Rr^3 \to \Rr^2$ définie par $f(x,y,z) = (-2x,y+3z)$.


\change
Nous allons calculer le noyau et l'image de $f$.


Commençons par le noyau $\Ker(f)$.

\change

Pour cela on part d'un élément de l'ensemble de départ $(x,y,z) \in \Ker(f)$

\change

cela équivaut $f(x,y,z)=(0,0)$

\change

par définition de $f$ cela donne $(-2x,y+3z)=(0,0)$

\change 

ce qui équivaut au système : $\left\{\begin{array}{rcl}
             -2x  & = & 0 \\
             y+3z & = & 0 \\
          \end{array}\right.$
          
\change

On peut par exemple paramétrer les solutions par la variable $z$ :
$(x,y,z) = (0,-3z,z)$, $z$ parcourant $\Rr$.


\change

Conclusion : le noyau est l'ensemble $\big\{ (0,-3z,z) \mid  z \in \Rr \big\}$.

\change

Autrement dit, $\Ker(f) = \Vect\big\{ (0,-3,1) \big\}$ : c'est une droite vectorielle. 

\change

Calculons maintenant l'image de $f$.

\change

Pour cela on fixe un élément $(x',y')$ de l'espace d'arrivée et on regarde 
s'il peut s'écrire comme l'image d'un élément de l'ensemble de départ.

\change

$(x',y')= f(x,y,z)$

ssi 

\change 

$(-2x,y+3z)=(x',y')$

\change

ce qui équivaut au système 

$\left\{\begin{array}{rcl}
             -2x  & = & x' \\
             y+3z & = & y' \\
          \end{array}\right.$
où les inconnues sont $x,y,z$.
          
\change


On peut prendre par exemple $x = -\frac{x'}{2}$, $y'=y$, $z=0$.

\change

Conclusion : pour n'importe quel $(x',y') \in \Rr^2$, on a
$f(-\frac{x'}{2},y',0) = (x',y')$. 

\change

Donc $\Im(f)= \Rr^2$, ce qui signifie que $f$ est surjective.


%%%%%%%%%%%%%%%%%%%%%%%%%%%%%%%%%%%%%%%%%%%%%%%%%%%%%%%%%%%
\diapo

Voyons d'autres exemples.


Soit $A$ une matrice à $n$ lignes et $p$ colonnes.

\change

et soit $f$ l'application linéaire définie par $f(X) = AX$.

\change


Alors $\Ker(f) = \big\{ X \in \Rr^p \mid AX=0 \big\}$ : c'est donc l'ensemble des $X \in \Rr^p$
solutions du système linéaire homogène $AX=0$. 

\change

On verra plus tard que l'image de $f$ se calcule comme le sous-espace vectoriel engendré 
par les colonnes de la matrice $A$.

\change


Le noyau fournit une nouvelle façon d'obtenir des sous-espaces vectoriels.

soit $f : \Rr^3 \to \Rr$ l'application définie par $f(x,y,z)=ax+by+cz$.

C'est clairement une application linéaire.

\change

Par définition $\Ker f = \big\{ (x,y,z) \in \Rr^3 \mid ax+by+cz=0 \big\}$

\change

Le noyau est donc un plan $\mathcal{P}$ passant par l'origine, d'équation $(ax+by+cz=0)$.

\change

Cela rédemontre un résultat déjà vu : un plan passant par l'origine est un sous-espace vectoriel de $\Rr^3$.



%%%%%%%%%%%%%%%%%%%%%%%%%%%%%%%%%%%%%%%%%%%%%%%%%%%%%%%%%%
\diapo

Nous allons voir que le noyau permet de caractériser les applications linéaires injectives


Soient $E$ et $F$ deux $\Kk$-espaces vectoriels et $f$ une application linéaire de $E$ dans $F$. Alors :
$f$ injective \quad $\iff \quad \Ker(f) = \big\{0_E\big\}$


Autrement dit, $f$ est injective si et seulement si son noyau ne contient que le vecteur nul.

\change
En particulier, pour montrer que $f$ est injective, il suffit de vérifier que : 

si $f(x)=0_F$ alors $x=0_E$.

\change

C'est vraiment une propriété fondamentale. Prenons un peu de temps pour la prouver.

\change

Pour l'implication directe, supposons que $f$ soit injective et montrons que $\Ker(f)=\{0_E\}$. 

\change

Soit donc $x$ un élément du noyau, cela signifie que $f(x)=0_{F}$. 

\change

Or, comme $f$ est linéaire, on a aussi $f(0_{E})=0_{F}$. 

\change

On vient d'obtenir l'égalité $f(x)=f(0_{E})$,

mais comme $f$ est injective, on en déduit $x=0_{E}$ 

Donc $\Ker (f)$ ne contient que le vecteur nul.

\change

Réciproquement, supposons maintenant que $\Ker (f)=\{0_E\}$. 
Il s'agit de montrer que $f$ est injective.

\change

Soient $x$ et $y$ deux éléments de $E$ tels que 
$f(x)=f(y)$. 

On a donc $f(x)-f(y)=0_{F}$.

\change

Mais $f$ est une application linéaire, alors $f(x-y)=0_{F}$, 

\change

c'est-à-dire $x-y$ est un élément du noyau. 

\change

Mais le noyau est réduit au vecteur nul donc $x-y=0_{E}$, 

ce qui prouve $x=y$ et que $f$ est injective.


%%%%%%%%%%%%%%%%%%%%%%%%%%%%%%%%%%%%%%%%%%%%%%%%%%%%%%%%%%
\diapo

Considérons, pour $n\ge1$, l'application 
\begin{eqnarray*}
f : \Rr_n[X] & \longrightarrow & \Rr_{n+1}[X]\\
P(X)  & \longmapsto & X \cdot P(X).
\end{eqnarray*}

\change

C'est bien une application linéaire et l'on va calculer son noyau et son image.


\change

\'Etudions d'abord le noyau de $f$: 

\change

soit $P(X) = a_n X^n + \cdots +a_1 X + a_0 \in \Rr_n[X]$ 
tel que $X \cdot P(X) = 0$. 

\change

Alors $a_n X^{n+1} + \cdots+ a_1 X^2 + a_0 X = 0.$

\change

Ainsi, $a_i = 0$ pour tout $i \in \{0,\ldots, n\}$ et donc $P$ est le polynôme nul.

\change

Le noyau de $f$ est donc réduit au polynôme nul. 

\change

Cela implique donc que $f$ est injective.

\change

Passons à l'image :

\change

L'espace $\Im(f)$ est l'ensemble des polynômes de $\Rr_{n+1}[X]$ 
dont le terme constant est nul : 

\change

$\Im(f) = \Vect \big\{X, X^2,\dots, X^{n+1}\big\}$ (il n'y a pas le polynôme cst égale à $1$ [dans Vect])

\change

Conclusion : $f$ est injective, mais n'est pas surjective.



%%%%%%%%%%%%%%%%%%%%%%%%%%%%%%%%%%%%%%%%%%%%%%%%%%%%%%%%%%%
\diapo

L'ensemble des applications linéaires entre deux $\Kk$-espaces vectoriels $E$ et $F$,  est lui-même un $\Kk$-espace vectoriel.

On note $\mathcal{L}(E,F)$ cet ensemble d'applications linéaires.

\change

La loi d'addition sur les fonctions et la loi de multiplication par un scalaire sont définis
naturellement :

$f, g$ étant deux application linéaire, 
alors pour tout vecteur $u$ de $E$, $(f+g)(u)=f(u)+g(u)$

et si $\lambda$ est un scalaire alors 
$(\lambda \cdot f)(u)=\lambda f(u)$.


%%%%%%%%%%%%%%%%%%%%%%%%%%%%%%%%%%%%%%%%%%%%%%%%%%%%%%%%%%
\diapo

Les applications linéaires peuvent se composer 

Si l'on a trois $\Kk$-espaces vectoriels, $E, F, G$ 
et que  $f$ est une application linéaire de $E$ dans $F$
et que $g$ est une application linéaire de $F$ dans $G$,

alors Proposition : ``$g \circ f$ est une application linéaire de $E$ dans $G$.''


\change

Le cas des endomorphismes est intéressant : rappelons qu'un endomorphisme de $E$ est simplement 
une application linéaire de $E$ dans $E$ (l'espace de départ et l'espace d'arrivée sont les mêmes).

Cette proposition implique que le composé de deux endomorphismes de $E$ 
reste un endomorphisme de $E$. 

En termes savants la loi de composition $\circ$ est une loi de 
composition interne sur $\mathcal{L}(E)$ !

\change

La composition des applications linéaires se comporte bien :

$$g \circ (f_1+f_2)=g \circ f_1+ g \circ f_2 
\qquad 
(g_1+g_2) \circ f =g_1 \circ f + g_2 \circ f
\qquad 
(\lambda g) \circ f =g \circ (\lambda f) =\lambda (g \circ f)
$$



%%%%%%%%%%%%%%%%%%%%%%%%%%%%%%%%%%%%%%%%%%%%%%%%%%%%%%%%%%%
\diapo

Il y a un vocabulaire spécifique aux applications linéaires *bijectives*.

Tout d'abord une application linéaire \evidence{bijective} de $E$ sur $F$ est appelée 
  \defi{isomorphisme} d'espaces vectoriels. 
 
 \change
  
Je vous encourage à prouver par vous même que si $f$ est une application linéaire bijective
alors $f^{-1}$ est aussi une application linéaire.

Ce qui fait que $f^{-1}$ est un isomorphisme de $F$ sur $E$.  

\change
 
 S'il existe un isomorphisme alors on dit que les deux espaces vectoriels $E$ et $F$ 
  sont \defi{isomorphes}.

  \change
  
Un endomorphisme bijectif de $E$ (c'est-à-dire une application linéaire bijective de $E$ dans lui même)
  s'appelle un \defi{automorphisme} de $E$.
  
  \change
  
   L'ensemble des automorphismes de $E$ est noté $GL(E)$.


%%%%%%%%%%%%%%%%%%%%%%%%%%%%%%%%%%%%%%%%%%%%%%%%%%%%%%%%%%
\diapo

Soit $f : \Rr^2 \to \Rr^2$ l'application définie par $f(x,y)=(2x+3y,x+y)$.

\change

Il est facile de prouver que $f$ est linéaire. 

\change

Pour prouver que $f$ est bijective, on pourrait calculer son noyau et son image.

\change

Mais ici nous allons calculer directement son inverse : 

\change

on cherche à résoudre $f(x,y)=(x',y')$. Cela correspond au 
système linéaire à deux équations et deux inconnues


$2x+3y=x', x+y=y'$

\change

On trouve $(x,y) = (-x'+3y',x'-2y')$.

\change

On pose donc $f^{-1}(x',y')= (-x'+3y',x'-2y')$.

\change

On vérifie aisément que $f^{-1}$ est l'inverse de $f$, 

\change

et on note que $f^{-1}$ est bien une application linéaire.


\change


Plus généralement, soit  $f : \Rr^n \to \Rr^n$ l'application linéaire
définie par $f(X)=AX$ (où $A$ est une matrice carrée de taille $n$).

\change


Si la matrice $A$ est inversible, alors $f^{-1}$ est une application linéaire bijective 
et réciproquement.

\change

En plus $f^{-1}$ est définie par $f^{-1}(X)= A^{-1} X$.

\change

Dans l'exemple précédent, 
$$X = \begin{pmatrix} x \\ y \end{pmatrix} \qquad 
A = \begin{pmatrix} 2 & 3 \\ 1 & 1 \end{pmatrix} \qquad 
A^{-1} = \begin{pmatrix} -1 & 3 \\ 1 & -2 \end{pmatrix}.$$


%%%%%%%%%%%%%%%%%%%%%%%%%%%%%%%%%%%%%%%%%%%%%%%%%%%%%%%%%%%
\diapo

On termine par une série d'exercices sur les applications linéaires.

\end{document}

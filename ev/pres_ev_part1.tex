
%%%%%%%%%%%%%%%%%% PREAMBULE %%%%%%%%%%%%%%%%%%

\documentclass[aspectratio=169,utf8]{beamer}
%\documentclass[aspectratio=169,handout]{beamer}

\usetheme{Boadilla}
%\usecolortheme{seahorse}
%\usecolortheme[RGB={245,66,24}]{structure}
\useoutertheme{infolines}

% packages
\usepackage{amsfonts,amsmath,amssymb,amsthm}
\usepackage[utf8]{inputenc}
\usepackage[T1]{fontenc}
\usepackage{lmodern}

\usepackage[francais]{babel}
\usepackage{fancybox}
\usepackage{graphicx}

\usepackage{float}
\usepackage{xfrac}

%\usepackage[usenames, x11names]{xcolor}
\usepackage{pgfplots}
\usepackage{datetime}


% ----------------------------------------------------------------------
% Pour les images
\usepackage{tikz}
\usetikzlibrary{calc,shadows,arrows.meta,patterns,matrix}

\newcommand{\tikzinput}[1]{\input{figures/#1.tikz}}
% --- les figures avec échelle éventuel
\newcommand{\myfigure}[2]{% entrée : échelle, fichier(s) figure à inclure
\begin{center}\small%
\tikzstyle{every picture}=[scale=1.0*#1]% mise en échelle + 0% (automatiquement annulé à la fin du groupe)
#2%
\end{center}}



%-----  Package unités -----
\usepackage{siunitx}
\sisetup{locale = FR,detect-all,per-mode = symbol}

%\usepackage{mathptmx}
%\usepackage{fouriernc}
%\usepackage{newcent}
%\usepackage[mathcal,mathbf]{euler}

%\usepackage{palatino}
%\usepackage{newcent}
% \usepackage[mathcal,mathbf]{euler}



% \usepackage{hyperref}
% \hypersetup{colorlinks=true, linkcolor=blue, urlcolor=blue,
% pdftitle={Exo7 - Exercices de mathématiques}, pdfauthor={Exo7}}


%section
% \usepackage{sectsty}
% \allsectionsfont{\bf}
%\sectionfont{\color{Tomato3}\upshape\selectfont}
%\subsectionfont{\color{Tomato4}\upshape\selectfont}

%----- Ensembles : entiers, reels, complexes -----
\newcommand{\Nn}{\mathbb{N}} \newcommand{\N}{\mathbb{N}}
\newcommand{\Zz}{\mathbb{Z}} \newcommand{\Z}{\mathbb{Z}}
\newcommand{\Qq}{\mathbb{Q}} \newcommand{\Q}{\mathbb{Q}}
\newcommand{\Rr}{\mathbb{R}} \newcommand{\R}{\mathbb{R}}
\newcommand{\Cc}{\mathbb{C}} 
\newcommand{\Kk}{\mathbb{K}} \newcommand{\K}{\mathbb{K}}

%----- Modifications de symboles -----
\renewcommand{\epsilon}{\varepsilon}
\renewcommand{\Re}{\mathop{\text{Re}}\nolimits}
\renewcommand{\Im}{\mathop{\text{Im}}\nolimits}
%\newcommand{\llbracket}{\left[\kern-0.15em\left[}
%\newcommand{\rrbracket}{\right]\kern-0.15em\right]}

\renewcommand{\ge}{\geqslant}
\renewcommand{\geq}{\geqslant}
\renewcommand{\le}{\leqslant}
\renewcommand{\leq}{\leqslant}
\renewcommand{\epsilon}{\varepsilon}

%----- Fonctions usuelles -----
\newcommand{\ch}{\mathop{\text{ch}}\nolimits}
\newcommand{\sh}{\mathop{\text{sh}}\nolimits}
\renewcommand{\tanh}{\mathop{\text{th}}\nolimits}
\newcommand{\cotan}{\mathop{\text{cotan}}\nolimits}
\newcommand{\Arcsin}{\mathop{\text{arcsin}}\nolimits}
\newcommand{\Arccos}{\mathop{\text{arccos}}\nolimits}
\newcommand{\Arctan}{\mathop{\text{arctan}}\nolimits}
\newcommand{\Argsh}{\mathop{\text{argsh}}\nolimits}
\newcommand{\Argch}{\mathop{\text{argch}}\nolimits}
\newcommand{\Argth}{\mathop{\text{argth}}\nolimits}
\newcommand{\pgcd}{\mathop{\text{pgcd}}\nolimits} 


%----- Commandes divers ------
\newcommand{\ii}{\mathrm{i}}
\newcommand{\dd}{\text{d}}
\newcommand{\id}{\mathop{\text{id}}\nolimits}
\newcommand{\Ker}{\mathop{\text{Ker}}\nolimits}
\newcommand{\Card}{\mathop{\text{Card}}\nolimits}
\newcommand{\Vect}{\mathop{\text{Vect}}\nolimits}
\newcommand{\Mat}{\mathop{\text{Mat}}\nolimits}
\newcommand{\rg}{\mathop{\text{rg}}\nolimits}
\newcommand{\tr}{\mathop{\text{tr}}\nolimits}


%----- Structure des exercices ------

\newtheoremstyle{styleexo}% name
{2ex}% Space above
{3ex}% Space below
{}% Body font
{}% Indent amount 1
{\bfseries} % Theorem head font
{}% Punctuation after theorem head
{\newline}% Space after theorem head 2
{}% Theorem head spec (can be left empty, meaning ‘normal’)

%\theoremstyle{styleexo}
\newtheorem{exo}{Exercice}
\newtheorem{ind}{Indications}
\newtheorem{cor}{Correction}


\newcommand{\exercice}[1]{} \newcommand{\finexercice}{}
%\newcommand{\exercice}[1]{{\tiny\texttt{#1}}\vspace{-2ex}} % pour afficher le numero absolu, l'auteur...
\newcommand{\enonce}{\begin{exo}} \newcommand{\finenonce}{\end{exo}}
\newcommand{\indication}{\begin{ind}} \newcommand{\finindication}{\end{ind}}
\newcommand{\correction}{\begin{cor}} \newcommand{\fincorrection}{\end{cor}}

\newcommand{\noindication}{\stepcounter{ind}}
\newcommand{\nocorrection}{\stepcounter{cor}}

\newcommand{\fiche}[1]{} \newcommand{\finfiche}{}
\newcommand{\titre}[1]{\centerline{\large \bf #1}}
\newcommand{\addcommand}[1]{}
\newcommand{\video}[1]{}

% Marge
\newcommand{\mymargin}[1]{\marginpar{{\small #1}}}

\def\noqed{\renewcommand{\qedsymbol}{}}


%----- Presentation ------
\setlength{\parindent}{0cm}

%\newcommand{\ExoSept}{\href{http://exo7.emath.fr}{\textbf{\textsf{Exo7}}}}

\definecolor{myred}{rgb}{0.93,0.26,0}
\definecolor{myorange}{rgb}{0.97,0.58,0}
\definecolor{myyellow}{rgb}{1,0.86,0}

\newcommand{\LogoExoSept}[1]{  % input : echelle
{\usefont{U}{cmss}{bx}{n}
\begin{tikzpicture}[scale=0.1*#1,transform shape]
  \fill[color=myorange] (0,0)--(4,0)--(4,-4)--(0,-4)--cycle;
  \fill[color=myred] (0,0)--(0,3)--(-3,3)--(-3,0)--cycle;
  \fill[color=myyellow] (4,0)--(7,4)--(3,7)--(0,3)--cycle;
  \node[scale=5] at (3.5,3.5) {Exo7};
\end{tikzpicture}}
}


\newcommand{\debutmontitre}{
  \author{} \date{} 
  \thispagestyle{empty}
  \hspace*{-10ex}
  \begin{minipage}{\textwidth}
    \titlepage  
  \vspace*{-2.5cm}
  \begin{center}
    \LogoExoSept{2.5}
  \end{center}
  \end{minipage}

  \vspace*{-0cm}
  
  % Astuce pour que le background ne soit pas discrétisé lors de la conversion pdf -> png
\begin{tikzpicture}
        \fill[opacity=0,green!60!black] (0,0)--++(0,0)--++(0,0)--++(0,0)--cycle; 
\end{tikzpicture}

% toc S'affiche trop tot :
% \tableofcontents[hideallsubsections, pausesections]
}

\newcommand{\finmontitre}{
  \end{frame}
  \setcounter{framenumber}{0}
} % ne marche pas pour une raison obscure

%----- Commandes supplementaires ------

% \usepackage[landscape]{geometry}
% \geometry{top=1cm, bottom=3cm, left=2cm, right=10cm, marginparsep=1cm
% }
% \usepackage[a4paper]{geometry}
% \geometry{top=2cm, bottom=2cm, left=2cm, right=2cm, marginparsep=1cm
% }

%\usepackage{standalone}


% New command Arnaud -- november 2011
\setbeamersize{text margin left=24ex}
% si vous modifier cette valeur il faut aussi
% modifier le decalage du titre pour compenser
% (ex : ici =+10ex, titre =-5ex

\theoremstyle{definition}
%\newtheorem{proposition}{Proposition}
%\newtheorem{exemple}{Exemple}
%\newtheorem{theoreme}{Théorème}
%\newtheorem{lemme}{Lemme}
%\newtheorem{corollaire}{Corollaire}
%\newtheorem*{remarque*}{Remarque}
%\newtheorem*{miniexercice}{Mini-exercices}
%\newtheorem{definition}{Définition}

% Commande tikz
\usetikzlibrary{calc}
\usetikzlibrary{patterns,arrows}
\usetikzlibrary{matrix}
\usetikzlibrary{fadings} 

%definition d'un terme
\newcommand{\defi}[1]{{\color{myorange}\textbf{\emph{#1}}}}
\newcommand{\evidence}[1]{{\color{blue}\textbf{\emph{#1}}}}
\newcommand{\assertion}[1]{\emph{\og#1\fg}}  % pour chapitre logique
%\renewcommand{\contentsname}{Sommaire}
\renewcommand{\contentsname}{}
\setcounter{tocdepth}{2}



%------ Encadrement ------

\usepackage{fancybox}


\newcommand{\mybox}[1]{
\setlength{\fboxsep}{7pt}
\begin{center}
\shadowbox{#1}
\end{center}}

\newcommand{\myboxinline}[1]{
\setlength{\fboxsep}{5pt}
\raisebox{-10pt}{
\shadowbox{#1}
}
}

%--------------- Commande beamer---------------
\newcommand{\beameronly}[1]{#1} % permet de mettre des pause dans beamer pas dans poly


\setbeamertemplate{navigation symbols}{}
\setbeamertemplate{footline}  % tiré du fichier beamerouterinfolines.sty
{
  \leavevmode%
  \hbox{%
  \begin{beamercolorbox}[wd=.333333\paperwidth,ht=2.25ex,dp=1ex,center]{author in head/foot}%
    % \usebeamerfont{author in head/foot}\insertshortauthor%~~(\insertshortinstitute)
    \usebeamerfont{section in head/foot}{\bf\insertshorttitle}
  \end{beamercolorbox}%
  \begin{beamercolorbox}[wd=.333333\paperwidth,ht=2.25ex,dp=1ex,center]{title in head/foot}%
    \usebeamerfont{section in head/foot}{\bf\insertsectionhead}
  \end{beamercolorbox}%
  \begin{beamercolorbox}[wd=.333333\paperwidth,ht=2.25ex,dp=1ex,right]{date in head/foot}%
    % \usebeamerfont{date in head/foot}\insertshortdate{}\hspace*{2em}
    \insertframenumber{} / \inserttotalframenumber\hspace*{2ex} 
  \end{beamercolorbox}}%
  \vskip0pt%
}


\definecolor{mygrey}{rgb}{0.5,0.5,0.5}
\setlength{\parindent}{0cm}
%\DeclareTextFontCommand{\helvetica}{\fontfamily{phv}\selectfont}

% background beamer
\definecolor{couleurhaut}{rgb}{0.85,0.9,1}  % creme
\definecolor{couleurmilieu}{rgb}{1,1,1}  % vert pale
\definecolor{couleurbas}{rgb}{0.85,0.9,1}  % blanc
\setbeamertemplate{background canvas}[vertical shading]%
[top=couleurhaut,middle=couleurmilieu,midpoint=0.4,bottom=couleurbas] 
%[top=fondtitre!05,bottom=fondtitre!60]



\makeatletter
\setbeamertemplate{theorem begin}
{%
  \begin{\inserttheoremblockenv}
  {%
    \inserttheoremheadfont
    \inserttheoremname
    \inserttheoremnumber
    \ifx\inserttheoremaddition\@empty\else\ (\inserttheoremaddition)\fi%
    \inserttheorempunctuation
  }%
}
\setbeamertemplate{theorem end}{\end{\inserttheoremblockenv}}

\newenvironment{theoreme}[1][]{%
   \setbeamercolor{block title}{fg=structure,bg=structure!40}
   \setbeamercolor{block body}{fg=black,bg=structure!10}
   \begin{block}{{\bf Th\'eor\`eme }#1}
}{%
   \end{block}%
}


\newenvironment{proposition}[1][]{%
   \setbeamercolor{block title}{fg=structure,bg=structure!40}
   \setbeamercolor{block body}{fg=black,bg=structure!10}
   \begin{block}{{\bf Proposition }#1}
}{%
   \end{block}%
}

\newenvironment{corollaire}[1][]{%
   \setbeamercolor{block title}{fg=structure,bg=structure!40}
   \setbeamercolor{block body}{fg=black,bg=structure!10}
   \begin{block}{{\bf Corollaire }#1}
}{%
   \end{block}%
}

\newenvironment{mydefinition}[1][]{%
   \setbeamercolor{block title}{fg=structure,bg=structure!40}
   \setbeamercolor{block body}{fg=black,bg=structure!10}
   \begin{block}{{\bf Définition} #1}
}{%
   \end{block}%
}

\newenvironment{lemme}[0]{%
   \setbeamercolor{block title}{fg=structure,bg=structure!40}
   \setbeamercolor{block body}{fg=black,bg=structure!10}
   \begin{block}{\bf Lemme}
}{%
   \end{block}%
}

\newenvironment{remarque}[1][]{%
   \setbeamercolor{block title}{fg=black,bg=structure!20}
   \setbeamercolor{block body}{fg=black,bg=structure!5}
   \begin{block}{Remarque #1}
}{%
   \end{block}%
}


\newenvironment{exemple}[1][]{%
   \setbeamercolor{block title}{fg=black,bg=structure!20}
   \setbeamercolor{block body}{fg=black,bg=structure!5}
   \begin{block}{{\bf Exemple }#1}
}{%
   \end{block}%
}


\newenvironment{miniexercice}[0]{%
   \setbeamercolor{block title}{fg=structure,bg=structure!20}
   \setbeamercolor{block body}{fg=black,bg=structure!5}
   \begin{block}{Mini-exercices}
}{%
   \end{block}%
}


\newenvironment{tp}[0]{%
   \setbeamercolor{block title}{fg=structure,bg=structure!40}
   \setbeamercolor{block body}{fg=black,bg=structure!10}
   \begin{block}{\bf Travaux pratiques}
}{%
   \end{block}%
}
\newenvironment{exercicecours}[1][]{%
   \setbeamercolor{block title}{fg=structure,bg=structure!40}
   \setbeamercolor{block body}{fg=black,bg=structure!10}
   \begin{block}{{\bf Exercice }#1}
}{%
   \end{block}%
}
\newenvironment{algo}[1][]{%
   \setbeamercolor{block title}{fg=structure,bg=structure!40}
   \setbeamercolor{block body}{fg=black,bg=structure!10}
   \begin{block}{{\bf Algorithme}\hfill{\color{gray}\texttt{#1}}}
}{%
   \end{block}%
}


\setbeamertemplate{proof begin}{
   \setbeamercolor{block title}{fg=black,bg=structure!20}
   \setbeamercolor{block body}{fg=black,bg=structure!5}
   \begin{block}{{\footnotesize Démonstration}}
   \footnotesize
   \smallskip}
\setbeamertemplate{proof end}{%
   \end{block}}
\setbeamertemplate{qed symbol}{\openbox}


\makeatother
\usecolortheme[RGB={205,0,0}]{structure}

 
%%%%%%%%%%%%%%%%%%%%%%%%%%%%%%%%%%%%%%%%%%%%%%%%%%%%%%%%%%%%%
%%%%%%%%%%%%%%%%%%%%%%%%%%%%%%%%%%%%%%%%%%%%%%%%%%%%%%%%%%%%%


\begin{document}


\title{{\bf Espaces vectoriels}}
\subtitle{Espace vectoriel (début)}

\begin{frame}
  
  \debutmontitre

  \pause

{\footnotesize
\hfill
\setbeamercovered{transparent=50}
\begin{minipage}{0.6\textwidth}
  \begin{itemize}
    \item<3-> Définition d'un espace vectoriel
    \item<4-> Premiers exemples
    \item<5-> Terminologie et notations
  \end{itemize}
\end{minipage}
}

\end{frame}

\setcounter{framenumber}{0}


%%%%%%%%%%%%%%%%%%%%%%%%%%%%%%%%%%%%%%%%%%%%%%%%%%%%%%%%%%%%%%%%
\section{Motivation}

\begin{frame}

\begin{itemize}[<+->]
  \item La notion d'espace vectoriel est une structure fondamentale des mathématiques modernes
  
  \item Propriétés communes que partagent des ensembles pourtant très différents
  
  \item On peut additionner deux vecteurs du plan, et aussi multiplier un vecteur par un réel
  
  \item On peut aussi additionner deux fonctions, ou multiplier une fonction par un réel
  
  \item Même chose avec les polynômes, les matrices,...
  
  \item But : obtenir des théorèmes généraux
\end{itemize}



\end{frame}



%%%%%%%%%%%%%%%%%%%%%%%%%%%%%%%%%%%%%%%%%%%%%%%%%%%%%%%%%%%%%%%%
\section{Définition d'un espace vectoriel}

\begin{frame}

\begin{center}
\begin{minipage}{0.75\textwidth}
\textit{Un espace vectoriel est un ensemble formé de vecteurs, de sorte que l'on puisse
additionner (et soustraire) deux vecteurs $u,v$ pour en former un troisième $u+v$ (ou $u-v$)
et aussi afin que l'on puisse multiplier chaque vecteur $u$ d'un facteur $\lambda$ pour obtenir 
un vecteur $\lambda \cdot u$}
\end{minipage}
\end{center}



\pause

Un \defi{$\Kk$-espace vectoriel} est un ensemble non vide $E$ muni :
\begin{itemize}
  \item d'une loi de composition interne, c'est-à-dire 
  d'une application de $E \times E$ dans $E$ :
$$\begin{array}{rcl}
E \times E & \to & E\\
(u, v) & \mapsto & u+v
\end{array}$$


\pause

  \item d'une loi de composition externe, 
  c'est-à-dire d'une application de $\Kk \times E$ dans $E$ : 
$$\begin{array}{rcl}
\Kk \times E & \to & E\\
(\lambda, u ) & \mapsto & \lambda \cdot u 
\end{array}$$

\end{itemize}

\pause

$\Kk$ est un corps (souvent $\Kk=\Rr$)
\end{frame}


\begin{frame}
 \setbeamercovered{transparent=60}
 \begin{enumerate}[<+->] \setlength{\itemsep}{6pt}
 \item $u + v = v + u$ \quad (pour tous $u,v \in E$)
 \item $u + (v+w) = (u+v) +w$ \quad (pour tous $u,v,w \in E$)
 \item Il existe un \defi{élément neutre} $0_E \in E$ tel que $u + 0_E = u$ \ ($\forall u \in E$)
 \item Tout $u \in E$ admet un \defi{symétrique} $u'$ tel que $u + u' = 0_E$
 
 Cet élément $u'$ est noté $-u$
 \item $1 \cdot u = u$ \quad (pour tout $u \in E$)
 \item $\lambda \cdot (\mu \cdot u) = (\lambda\mu )\cdot u$ \quad (pour tous $\lambda, \mu \in \Kk$, $u \in E$)
 \item $\lambda \cdot (u+v) = \lambda \cdot u + \lambda \cdot v$ \quad (pour tous $\lambda \in \Kk$, $u,v \in E$)
 \item $(\lambda + \mu ) \cdot u = \lambda \cdot u + \mu \cdot u$ \quad (pour tous $\lambda,\mu \in \Kk$, $u \in E$)
 \end{enumerate} 

\end{frame}


%%%%%%%%%%%%%%%%%%%%%%%%%%%%%%%%%%%%%%%%%%%%%%%%%%%%%%%%%%%%%%%%
\section{Premiers exemples}

\begin{frame}
\centerline{\evidence{Le $\Rr$-espace vectoriel $\Rr^2$}}

\begin{itemize}[<+->] 
  \item Posons $\Kk=\Rr$ et $E=\Rr^2$
  
  \item Un élément $u\in E$ est donc un
  couple $(x,y)$ avec $x$ élément de $\Rr$ et $y$ élément de $\Rr$
  
  \item $\Rr^2=\big\{(x,y)\mid x \in \Rr, y \in \Rr\big\}$
 

  \item \emph{Définition de la loi interne}
    
    Si $(x,y)$ et $(x',y')$ sont deux éléments de $\Rr^2$, alors :
  $$(x,y)+(x',y')=(x+x',y+y')$$
    
  \item \emph{Définition  de la loi externe}
    
    Si $\lambda$ est un réel et $(x,y)$ est un élément de $\Rr^2$, alors :
  $$\lambda \cdot (x,y)=(\lambda x, \lambda y)$$

  \item L'élément neutre de la loi interne est le vecteur nul $(0,0)$
  
  \item Le symétrique de $(x,y)$ est $(-x,-y)$, que l'on note aussi $-(x,y)$
  
\end{itemize} 
\end{frame}


\begin{frame}
\myfigure{1}{
\tikzinput{fig_ev01} 
}
 
\end{frame}


\begin{frame}
\centerline{\evidence{Le $\Rr$-espace vectoriel $\Rr^n$}}

\begin{itemize}[<+->]
  \item $\Kk=\Rr$, $E=\Rr^n$ ($n\ge1$) 
  
  \item $u\in E$ est donc un $n$-uplet $(x_1,x_2, \ldots , x_n)$

  \item \emph{Définition de la loi interne} 
  
  Si $(x_1, \dots , x_n)$ et $(x'_1, \dots , x'_n)$ sont deux éléments de $\Rr^n$, alors :
  $$(x_1, \dots , x_n)+(x'_1, \dots , x'_n) =  (x_1+x'_1, \dots , x_n+x'_n)$$    
  
  
  \item \emph{Définition  de la loi externe}
  
      Si $\lambda$ est un réel et $(x_1, \dots , x_n)$ est un élément de $\Rr^n$, alors :
  $$\lambda \cdot (x_1, \dots , x_n)=(\lambda x_1,\dots ,  \lambda x_n)$$
    
  \item L'élément neutre de la loi interne est le vecteur nul $(0,0, \dots, 0)$
  
  \item Le symétrique de    $(x_1, \dots , x_n)$ est $(-x_1, \dots , -x_n)$, que l'on note 
$-(x_1, \dots , x_n)$
\end{itemize}


\end{frame}


\begin{frame}
\centerline{\evidence{Tout plan passant par l'origine dans $\Rr^3$ est un espace vectoriel}}
\pause
\hfill\hfill\begin{minipage}{0.3\textwidth}
\myfigure{0.7}{
\tikzinput{fig_ev02} 
}  
\end{minipage}
\vspace*{-5ex} 
\begin{itemize}
  \item Soient $\Kk=\Rr$ et $E$ un plan passant par l'origine
 
\pause  
  \item \'Equation $ax + by + cz = 0$
 
\pause
  \item Un élément $u$ de $E$ est un triplet 
  $\left(\begin{smallmatrix}x\\ y\\ z\end{smallmatrix}\right)$ tel que
$ax + by + cz = 0$
\pause  
  \item 
  \begin{itemize}
     \item Soient $u=\left(\begin{smallmatrix}x\\ y\\ z\end{smallmatrix}\right)$ et 
$v=\left(\begin{smallmatrix}x'\\ y'\\ z'\end{smallmatrix}\right)$ deux éléments de $E$
\pause
     \item $a x + b y + c z  =  0$  \ \  et \ \  $a x' + b y' + c z' = 0$  
\pause     
     \item Donc $a (x + x') + b(y + y') + c (z + z') = 0$
\pause     
     \item Ainsi $u+v=\left(\begin{smallmatrix}x + x'\\ y + y'\\ z + z'\end{smallmatrix}\right)$ 
appartient à $E$
  \end{itemize}
\pause  
  \item L'élément neutre est $\left(\begin{smallmatrix}0\\ 0\\ 0\end{smallmatrix}\right) \in E$
  
\pause  
  \item Le symétrique est $-\left(\begin{smallmatrix}x\\ y\\ z\end{smallmatrix}\right) \in E$
\end{itemize}



\end{frame}

%%%%%%%%%%%%%%%%%%%%%%%%%%%%%%%%%%%%%%%%%%%%%%%%%%%%%%%%%%%%%%%%
\section{Terminologie et notations}

\begin{frame}
 
\begin{itemize}[<+->]\setlength{\itemsep}{6pt}
  \item Les éléments de $E$ sont des \defi{vecteurs}

  \item Les éléments de $\Kk$ seront des \defi{scalaires}
  
  \item L'\,\defi{élément neutre} $0_E$ s'appelle aussi le \defi{vecteur nul}
   
  \item Le \defi{symétrique} $-u$ d'un vecteur $u \in E$ s'appelle aussi l'\defi{opposé}
  
  \item La loi de composition interne sur $E$, notée $+$, est l'\defi{addition}

  \item La loi de composition externe sur $E$ est la \defi{multiplication par un scalaire}
\end{itemize}


\end{frame}


\begin{frame}\setlength{\itemsep}{6pt}
\defi{Somme de $n$ vecteurs}

\pause

\begin{itemize}
  \item Somme de $2$ vecteurs :  $v_1+v_2$

\pause

  \item Somme de $n$ vecteurs $ v_1,v_2, \ldots, v_n$
$$v_1+v_2+\cdots+v_n=(v_1+v_2+\cdots+v_{n-1})+v_n$$  

\pause

  \item $v_1+v_2+\cdots+v_n={\displaystyle \sum_{i=1}^nv_{i}}$

\end{itemize}

\end{frame}


%%%%%%%%%%%%%%%%%%%%%%%%%%%%%%%%%%%%%%%%%%%%%%%%%%%%%%%%%%%%%%%%
\section{Mini-exercices}

\begin{frame}

\begin{miniexercice}
\begin{enumerate}
  \item Vérifier les $8$ axiomes qui font de $\Rr^3$ un $\Rr$-espace vectoriel.

  \item Idem pour une droite $\mathcal{D}$ de $\Rr^3$ passant par l'origine définie par 
  $\left\{\begin{array}{rcl}ax+by+cz & = & 0 \\a'x+b'y+c'z & = & 0   \end{array}\right.$.
  
  \item Justifier que les ensembles suivants \emph{ne sont pas} des espaces vectoriels :
  $\big\{ (x,y) \in \Rr^2 \mid xy=0\big\}$ ;
  $\big\{ (x,y) \in \Rr^2 \mid x=1\big\}$ ;
  $\big\{ (x,y) \in \Rr^2 \mid x\ge0 \text{ et } y\ge0 \big\}$ ;  
  $\big\{ (x,y) \in \Rr^2 \mid -1 \le x \le 1 \text{ et } -1 \le y \le 1 \big\}$.
  
  \item Montrer par récurrence que si les $v_i$ sont des éléments d'un $\Kk$-espace vectoriel
  $E$, alors pour tous $\lambda_i\in \Kk$ : $\lambda_1 v_1+\lambda_2v_2+\cdots + \lambda_n v_n \in E$.
\end{enumerate} 
\end{miniexercice}

\end{frame}

\end{document}